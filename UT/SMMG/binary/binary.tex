\documentclass[11pt,leqno]{article}
\headheight=13.6pt

% packages
\usepackage[top=1in, bottom=1in, left=0.5in, right=0.5in]{geometry}
\usepackage{amsfonts, amsmath, amssymb, amsthm}
\usepackage{fancyhdr}
\usepackage{enumitem}

% header/footer formatting
\pagestyle{fancy}
\fancyhead{}
\fancyfoot{}
\fancyhead[L]{Some problems involving binary and n-ary expansion}
\fancyhead[R]{\thepage}
\renewcommand{\headrulewidth}{1pt}

% paragraph indentation/spacing
\setlength{\parindent}{0cm}
\setlength{\parskip}{10pt}
\renewcommand{\baselinestretch}{1.25}

% smileys frownies
\usepackage{wasysym}
\newcommand{\smallhappy}{\smiley}
\newcommand{\happy}{\raisebox{-.14em}{\resizebox{1.2em}{!}{\smiley}}}
\newcommand{\smallsad}{\frownie}
\newcommand{\sad}{\raisebox{-.14em}{\resizebox{1.2em}{!}{\frownie}}}
\DeclareMathOperator{\mathhappy}{\!\happy\!}
\DeclareMathOperator{\smallmathhappy}{\!\smallhappy\!}
\DeclareMathOperator{\mathsad}{\!\sad\!}
\DeclareMathOperator{\smallmathsad}{\!\smallsad\!}

% set page count index to begin from 1
\setcounter{page}{1}

\begin{document}
In the decimal system, we use the digits $0$ through $9$ to describe numbers. For example, the number 
\[\text{one thousand, seven hundred twenty-eight}\]
is denoted in decimal by
\[1728.\]
But what is meant by $1728$ is the sum $8\times 10^0 + 2\times 10^1 + 7\times 10^2 + 1\times 10^3$. That is, given an arbitrary string of digits
\[a_ma_{m-1}\dots a_1a_0,\]
the decimal number it represents is the sum
\[a_0\times 10^0 + a_1\times 10^1+\cdots + a_{m-1}\times 10^{m-1} + a_m\times 10^m.\] Fractions are similar. Digits to the right of the decimal point, called the radix point in general, represent adding negative powers of $10$. For example, 
\[2.683\]
represents the sum 
\[3\times 10^{-3} + 8\times 10^{-2} + 6\times 10^{-1} + 2\times 10^0.\]

Binary, and other bases, are similar to decimal. In binary we only use the digits $0$ and $1$ to represent numbers via sums of powers of $2$. For example, 
\[10110.101\]
in binary represents the sum
\[1\times 2^{-3} + 0\times 2^{-2} + 1\times 2^{-1} + 0\times 2^0 + 1\times 2^1 + 1\times 2^2 + 0\times 2^4 + 1\times 2^5.\]
To avoid confusion, we should indicate what system we are using whenever we write a string of digits. The following convention is not bad:
\begin{align*}
    {\texttt{<digits>}}_{10} &= \text{in decimal system (or base $10$)}\\
    {\texttt{<digits>}}_{2} &= \text{in binary system (or base $2$)}
\end{align*}

In general, expressing a number in base $n$ for an integer $n$ means to express that number using the digits $0,\dots,n-1$ as a sum of powers of $n$ via 
\[a_s\dots a_0.a_{-1}\dots a_{-r} = a_{-r}\times n^{-r} + \cdots a_{-1}\times n^{-1} + a_0\times n^0 + \cdots a_s\times n^s.\]
\newpage
\begin{enumerate}
    \item In the decimal system, some numbers have repeating blocks of digits after the decimal point, denoted with a line above the block that repeats. For example, $0.142857142857\cdots_{10} = 0.\overline{142857}_{10}$ can be shown to be the fraction $1/7$, and $0.999\cdots_{10} = 0.\overline{9}_{10} = 1$.

    What is $0.\overline{1}_2$, $0.\overline{101}_2$, and $0.\overline{12}_3$? \vspace*{\fill} 
    \item Write down a few digits of $\sqrt{2}$ in base $10$ and in base $2$.\vspace*{\fill}
    \newpage \item Let $B$ be the set of all binary integers that can be written using exactly $5$ zeros and $8$ ones where leading zeros are allowed. If all possible subtractions are performed in which one element of $B$ is subtracted from another, find the number of times the answer $1$ is obtained. \vspace*{\fill} % https://artofproblemsolving.com/wiki/index.php/2012_AIME_I_Problems/Problem_5 
    \item In the binary expansion of $\dfrac{2^{2007}-1}{2^{225}-1}$, how many of the first $10,000$ digits to the right of the radix point are $0$'s?\vspace*{\fill} % https://artofproblemsolving.com/wiki/index.php/2007_iTest_Problems/Problem_56 
    \newpage \item  Let $P(x)$ be a polynomial with non-negative integer coefficients. If $P(1) = 7$, and $P(8) = 9352$, determine the coefficients of $P(x)$. \vspace*{\fill} % https://www.reddit.com/r/math/comments/yx0i7r/determine_a_polynomial_from_just_two_inputs/ 
    \item Alice chooses a set $A$ of positive integers. Then Bob lists all finite nonempty sets $B$ of positive integers with the property that the maximum element of $B$ belongs to $A$. Bob's list has 2024 sets. Find the sum of the elements of A. \vspace*{\fill} % https://artofproblemsolving.com/wiki/index.php/2024_AIME_II_Problems/Problem_6 
\end{enumerate}
\end{document}