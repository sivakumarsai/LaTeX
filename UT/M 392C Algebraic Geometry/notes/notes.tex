\documentclass[11pt,leqno]{article}
\headheight=13.6pt

% packages
\usepackage[alphabetic]{amsrefs}
\usepackage{physics}
% margin spacing
\usepackage[top=1in, bottom=1in, left=0.5in, right=0.5in]{geometry}
\usepackage{hanging}
\usepackage{amsfonts, amsmath, amssymb, amsthm}
\usepackage{extarrows}
\usepackage[none]{hyphenat}
\usepackage{fancyhdr}
\usepackage{graphicx}
\graphicspath{{./images/}}
\usepackage{float}
\usepackage{siunitx}
\usepackage{esint}
\usepackage{color}
\usepackage{enumitem}
\usepackage{mathrsfs}
\usepackage{hyperref}
\usepackage[noabbrev, capitalise]{cleveref}
\crefformat{equation}{equation~#2#1#3}
\crefformat{lemma}{\textrm{Lemma}~#2#1#3}

% theorems
\theoremstyle{plain}
\newtheorem{lem}{Lemma}
\newtheorem{lemma}[lem]{Lemma}
\newtheorem{thm}[lem]{Theorem}
\newtheorem{theorem}[lem]{Theorem}
\newtheorem{prop}[lem]{Proposition}
\newtheorem{proposition}[lem]{Proposition}
\newtheorem{cor}[lem]{Corollary}
\newtheorem{corollary}[lem]{Corollary}
\newtheorem{conj}[lem]{Conjecture}
\newtheorem{fact}[lem]{Fact}
\newtheorem{form}[lem]{Formula}

\theoremstyle{definition}
\newtheorem{defn}[lem]{Definition}
\newtheorem{definition/}[lem]{Definition}
\newenvironment{definition}
  {\renewcommand{\qedsymbol}{\textdagger}%
   \pushQED{\qed}\begin{definition/}}
  {\popQED\end{definition/}}
\newtheorem{example}[lem]{Example}
\newtheorem{remark}[lem]{Remark}
\newtheorem{exercise}[lem]{Exercise}
\newtheorem{notation}[lem]{Notation}

\numberwithin{equation}{section}
\numberwithin{lem}{section}

% header/footer formatting
\pagestyle{fancy}
\fancyhead{}
\fancyfoot{}
\fancyhead[L]{M 392C}
\fancyhead[C]{}
\fancyhead[R]{Algebraic Geometry}
\fancyfoot[R]{\thepage}
\renewcommand{\headrulewidth}{1pt}

% paragraph indentation/spacing
\setlength{\parindent}{0cm}
\setlength{\parskip}{10pt}
\renewcommand{\baselinestretch}{1.25}

% extra commands defined here
\newcommand{\br}[1]{\left(#1\right)}
\newcommand{\sbr}[1]{\left[#1\right]}
\newcommand{\cbr}[1]{\left\{#1\right\}}
\newcommand{\eq}[1]{\overset{(#1)}{=}}

% bracket notation for inner product
\usepackage{mathtools}

\DeclarePairedDelimiterX{\abr}[1]{\langle}{\rangle}{#1}

% set page count index to begin from 1
\setcounter{page}{1}

\begin{document}
M 392C Algebraic Geometry, Fall 2024: Dr. Chiara Damiolini's notes, augmented and \TeX ed by Sai Sivakumar. Text references: \textit{Algebraic Geometry I: Schemes} by G\"ortz and Wedhorn, and \textit{Algebraic Geometry} by Hartshorne.
\subsection*{08/26}
Some examples of objects in algebraic geometry:

Consider the quadric surface in $\mathbb R^3$ given by the solution set of the equation $xy = z$.
\textcolor{red}{insert picture}
Sometimes families of geometric objects (e.g., hyperbolae) are not compact, but degenerate into objects of different natures.
We would like to be able to treat degenerate curves as well using algebraic geometry.

Consider the curve $C$ in $\mathbb R^2$ given by the solution set of the equation $y^2 = x^3$.
\textcolor{red}{insert picture}
We parameterize $C$ by $t\mapsto (t^2,t^3)$ for $t$ a real parameter (a bijection), but we would like to say that $C$ and the real line $\mathbb R$ are not ``isomorphic'' curves due to the cusp $C$ has that $\mathbb R$ does not have (though we say that these curves are \textit{birational} to each other). The parameterization $t\mapsto (t^2, t^3)$ induces the following map of rings
\begin{equation}
  \frac{\mathbb C[x,y]}{(y^2-x^3)}\to \mathbb C[t]\quad \text{given by}\quad  x\mapsto t^2, y\mapsto t^3.
\end{equation}
We would like to use commutative algebra to study curves like the ones above.

Some important spaces are not given by solution sets to systems of polynomial equations, but are instead defined implicitly or even by universal properties.
Consider the real projective line $\mathbb P_{\mathbb R}^1 = \cbr{\text{linear subspaces of }\mathbb R^2} = (\mathbb R^2\setminus\cbr{0})/\mathbb R^\times$, in which points are identified under scaling: $(x,y)\equiv (\lambda x,\lambda y)$ for $\lambda\in \mathbb R^\times$, and denote the equivalence class of $(x,y)$ by $[x:y]$. \textcolor{red}{insert picture}
Observe that $\mathbb P_{\mathbb R}^1$ is covered by two copies of $\mathbb R$; that is, we can find two open subsets of $\mathbb P_{\mathbb R}^1$ each homeomorphic to $\mathbb R$ whose union is $\mathbb P_{\mathbb R}^1$: The subsets $\cbr{[a:1]\mid a\in\mathbb R}$ and $\cbr{[1:b]\mid b\in \mathbb R}$ form a cover of $\mathbb P_{\mathbb R}^1$, noting that for $c\in\mathbb R^\times$, $[c:1] = [1:c^{-1}]$.
We seek to generalize this example, but run into difficulties. 
In considering the quotient $\mathbb R^2/\mathbb R^\times$, one complication is that the origin $(0,0)$ is fixed, so finding a parameterization is not easy.

The choice of coefficients is necessary data when posing questions. Consider the polynomial equation $x^2+y^2 = -1$. If $x,y$ take on real values, there are no solutions, whereas complex solutions exist.
In particular, if $x,y\in \mathbb C$, then we may factor $x^2 + y^2$ to obtain $(x+iy)(x-iy) = -1$. The change of coordinates given by $s = x+iy$ and $t = -(x-iy)$ yields the polynomial equation $st = 1$, whose solution set is $(\mathbb C^\ast)^2$.
Another example: Fermat's last theorem says that the equation $x^n + y^n = 1$ for $n\geq 3$ has no solutions over $\mathbb Q$ unless $x = 0$ or $y = 0$, but this is false over $\mathbb R$.

\newpage\subsection*{08/28}
Throughout this course, rings are commutative and unital; in turn ideals are two-sided.
Recall that if $\mathfrak p$ is a prime ideal of a ring $R$, then $R/\mathfrak p$ is an integral domain (in what follows, we suppress ``integral'' in ``integral domain'').
If $\mathfrak m$ is a maximal ideal in $R$, then $R/\mathfrak m$ is a field.

We explore the relationship between maximal ideals and solution sets to polynomial equations.
For $K$ a field, let $R = K[x_1,\dots,x_n]$. Recall that $R$ is a Noetherian ring, that is, every ideal $I$ of $R$ is finitely generated, so $I = \abr{f_1,\dots,f_m}$.
That $R$ is Noetherian follows from Hilbert's basis theorem, which states that if $A$ is Noetherian, then $A[x]$ is also Noetherian.

Let $I$ be an ideal of $R$ and consider the map
\begin{equation}
  V(I)\coloneqq\cbr{\underline a = (a_1,\dots,a_n)\in K^n \mid f_i(\underline a) = 0}\xlongrightarrow{\varphi}\cbr{\text{max. ideals of }R/I}\cong\cbr{\text{max. ideals of $R$ containing $I$}}
\end{equation}
that sends $\underline a$ to $\mathfrak m_{\underline a} \coloneqq \abr{(x_1-a_1),\dots,(x_n-a_n)}$. Note that $I$ is contained in $\mathfrak m_{\underline a}$ if and only if $f(\underline a) = 0$ for all $f\in I$, because $\mathfrak m_{\underline a} = \ker (R\to R/\mathfrak m_{\underline a}\cong K)$, where $R\to R/\mathfrak m_{\underline a}$ is the quotient map.


\newpage\subsection*{08/30}

\newpage\subsection*{09/09}

\newpage\subsection*{09/11}

\newpage\subsection*{09/13}
Consider presheaves valued in Ab or Ring. We can rephrase the axioms for a sheaf $F$ in the following way: 
\begin{enumerate}[label=(\arabic*)]
    \item (Separatedness)
    \item (Gluing)
\end{enumerate}

\end{document}