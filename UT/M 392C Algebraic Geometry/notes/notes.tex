\documentclass[11pt,leqno]{article}
\headheight=13.6pt

% packages
\usepackage[alphabetic]{amsrefs}
\usepackage{physics}
% margin spacing
\usepackage[top=1in, bottom=1in, left=0.5in, right=0.5in]{geometry}
\usepackage{hanging}
\usepackage{amsfonts, amsmath, amssymb, amsthm}
\usepackage{extarrows}
\usepackage[none]{hyphenat}
\usepackage{fancyhdr}
\usepackage{graphicx}
\graphicspath{{./images/}}
\usepackage{float}
\usepackage{imakeidx}
    % \AtBeginDocument{\renewcommand\indexname{Index}}
    \makeindex
\indexsetup{level=\section*,toclevel=section,noclearpage,firstpagestyle=fancy}
\usepackage{color}
\usepackage{enumitem}
\usepackage{mathrsfs}
\usepackage{hyperref}
\usepackage[noabbrev, capitalise]{cleveref}
\crefformat{equation}{equation~#2#1#3}
\crefformat{lemma}{\textrm{Lemma}~#2#1#3}
\usepackage{quiver}

% theorems
\theoremstyle{plain}
\newtheorem{lem}{Lemma}
\newtheorem{lemma}[lem]{Lemma}
\newtheorem{thm}[lem]{Theorem}
\newtheorem{theorem}[lem]{Theorem}
\newtheorem{prop}[lem]{Proposition}
\newtheorem{proposition}[lem]{Proposition}
\newtheorem{cor}[lem]{Corollary}
\newtheorem{corollary}[lem]{Corollary}
\newtheorem{conj}[lem]{Conjecture}
\newtheorem{fact}[lem]{Fact}
\newtheorem{form}[lem]{Formula}

\theoremstyle{definition}
\newtheorem{defn}[lem]{Definition}
\newtheorem{definition/}[lem]{Definition}
\newenvironment{definition}
  {\renewcommand{\qedsymbol}{\textdagger}%
   \pushQED{\qed}\begin{definition/}}
  {\popQED\end{definition/}}
\newtheorem{example}[lem]{Example}
\newtheorem{remark}[lem]{Remark}
\newtheorem{exercise}[lem]{Exercise}
\newtheorem{notation}[lem]{Notation}

\numberwithin{equation}{section}
\numberwithin{lem}{section}

% header/footer formatting
\pagestyle{fancy}
\fancyhead{}
\fancyfoot{}
\fancyhead[L]{M 392C}
\fancyhead[C]{}
\fancyhead[R]{Algebraic Geometry}
\fancyfoot[R]{\thepage}
\renewcommand{\headrulewidth}{1pt}

% paragraph indentation/spacing
\setlength{\parindent}{0cm}
\setlength{\parskip}{10pt}
\renewcommand{\baselinestretch}{1.25}

% extra commands defined here
\newcommand{\br}[1]{\left(#1\right)}
\newcommand{\sbr}[1]{\left[#1\right]}
\newcommand{\cbr}[1]{\left\{#1\right\}}
\newcommand{\eq}[1]{\overset{(#1)}{=}}
\newcommand{\idx}[1]{#1\index{#1}} % adds to index
\newcommand{\bidx}[1]{\textbf{#1\index{#1}}} % adds to index

% bracket notation for inner product
\usepackage{mathtools}

\DeclarePairedDelimiterX{\abr}[1]{\langle}{\rangle}{#1}

\DeclareMathOperator{\im}{im}
\DeclareMathOperator{\Hom}{Hom}

% categories
\newcommand{\catname}[1]{{\normalfont\mathbf{#1}}}
\newcommand{\Set}{\catname{Set}}
\newcommand{\Kalg}{K\text{-}\catname{Alg}}

% set page count index to begin from 1
\setcounter{page}{1}

\begin{document}
M 392C Algebraic Geometry, Fall 2024: Dr. Chiara Damiolini's notes, augmented and \TeX ed by Sai Sivakumar. Text references: \textit{Algebraic Geometry I: Schemes} by G\"ortz and Wedhorn, and \textit{Algebraic Geometry} by Hartshorne.

\textcolor{red}{ask jemma for her notes at some point for inspiring art}
\subsection*{08/26}
Some examples of objects in algebraic geometry:

Consider the quadric surface in $\mathbb R^3$ given by the solution set of the equation $xy = z$.
\textcolor{red}{insert picture}
Sometimes families of geometric objects (e.g., hyperbolae) are not compact, but degenerate into objects of different natures.
We would like to be able to treat degenerate curves as well using algebraic geometry.

Consider the curve $C$ in $\mathbb R^2$ given by the solution set of the equation $y^2 = x^3$.
\textcolor{red}{insert picture}
We parameterize $C$ by $t\mapsto (t^2,t^3)$ for $t$ a real parameter (a bijection), but we would like to say that $C$ and the real line $\mathbb R$ are not ``isomorphic'' curves due to the cusp $C$ has that $\mathbb R$ does not have (though we say that these curves are \textit{birational} to each other). The parameterization $t\mapsto (t^2, t^3)$ induces the following map of rings
\begin{equation}
  \frac{\mathbb C[x,y]}{(y^2-x^3)}\to \mathbb C[t]\quad \text{given by}\quad  x\mapsto t^2, y\mapsto t^3.
\end{equation}
We would like to use commutative algebra to study curves like the ones above.

Some important spaces are not given by solution sets to systems of polynomial equations, but are instead defined implicitly or even by universal properties.
Consider the real projective line $\mathbb P_{\mathbb R}^1 = \cbr{\text{linear subspaces of }\mathbb R^2} = (\mathbb R^2\setminus\cbr{0})/\mathbb R^\times$, in which points are identified under scaling: $(x,y)\equiv (\lambda x,\lambda y)$ for $\lambda\in \mathbb R^\times$, and denote the equivalence class of $(x,y)$ by $[x:y]$. \textcolor{red}{insert picture}
Observe that $\mathbb P_{\mathbb R}^1$ is covered by two copies of $\mathbb R$; that is, we can find two open subsets of $\mathbb P_{\mathbb R}^1$ each homeomorphic to $\mathbb R$ whose union is $\mathbb P_{\mathbb R}^1$: The subsets $\cbr{[a:1]\mid a\in\mathbb R}$ and $\cbr{[1:b]\mid b\in \mathbb R}$ form a cover of $\mathbb P_{\mathbb R}^1$, noting that for $c\in\mathbb R^\times$, $[c:1] = [1:c^{-1}]$.
We seek to generalize this example, but run into difficulties. 
In considering the quotient $\mathbb R^2/\mathbb R^\times$, one complication is that the origin $(0,0)$ is fixed, so finding a parameterization is not easy.

The choice of coefficients is necessary data when posing questions. Consider the polynomial equation $x^2+y^2 = -1$. If $x,y$ take on real values, there are no solutions, whereas complex solutions exist.
In particular, if $x,y\in \mathbb C$, then we may factor $x^2 + y^2$ to obtain $(x+iy)(x-iy) = -1$. The change of coordinates given by $s = x+iy$ and $t = -(x-iy)$ yields the polynomial equation $st = 1$, whose solution set is $(\mathbb C^\ast)^2$.
Another example: Fermat's last theorem says that the equation $x^n + y^n = 1$ for $n\geq 3$ has no solutions over $\mathbb Q$ unless $x = 0$ or $y = 0$, but this is false over $\mathbb R$.

\newpage\subsection*{08/28}
Throughout this course, rings are commutative and unital; in turn ideals are two-sided.
Recall that if $\mathfrak p$ is a prime ideal of a ring $R$, then $R/\mathfrak p$ is an integral domain (in what follows, we suppress ``integral'' in ``integral domain'').
If $\mathfrak m$ is a maximal ideal in $R$, then $R/\mathfrak m$ is a field.

We explore the relationship between maximal ideals and solution sets to polynomial equations.
For $K$ a field, let $R = K[x_1,\dots,x_n]$. Recall that $R$ is a Noetherian ring, that is, every ideal $I$ of $R$ is finitely generated, so $I = \abr{f_1,\dots,f_m}$.
That $R$ is Noetherian follows from Hilbert's basis theorem, which states that if $A$ is Noetherian, then $A[x]$ is also Noetherian.

Let $I = \abr{f_1,\dots,f_m}$ be an ideal of $R$ and consider the map
\begin{multline}
  V(I)\coloneqq\cbr{\underline a = (a_1,\dots,a_n)\in K^n \mid f_i(\underline a) = 0, 1\leq i\leq n}\xlongrightarrow{\varphi}\cbr{\text{max. ideals of }R/I}\\(\cong\cbr{\text{max. ideals of $R$ containing $I$}})
\end{multline}
that sends $\underline a$ to $\mathfrak m_{\underline a} \coloneqq \abr{(x_1-a_1),\dots,(x_n-a_n)}$. Note that $I$ is contained in $\mathfrak m_{\underline a}$ if and only if $f(\underline a) = 0$ for all $f\in I$, because $\mathfrak m_{\underline a} = \ker (R\to R/\mathfrak m_{\underline a}\cong K)$, where $R\to R/\mathfrak m_{\underline a}$ is the quotient map.
The map $\varphi$ is injective. and if $K$ is not algebraically closed, the map $\varphi$ may not be surjective. For example, take $K = \mathbb R$, $n =1$, and $I = 0$; the ideal $\abr{x^2+1}$ has no preimage under $\varphi$. Another possibility if $K$ is not algebraically closed is that $V(I)$ may be empty. For example, take $K = \mathbb R$, $n=1$, and $I= \abr{x^2+1}$; in this case $V(I)$ is empty but $R/I$ is a field.

A version of Hilbert's Nullstellensatz implies that $\varphi$ is a bijection if $K$ is algebraically closed.
\begin{theorem}[Hilbert's \idx{Nullstellensatz} \cite{gw}*{Theorem 1.7}]
  Let $K$ be a \textup{(}not necessarily algebraically closed\textup{)} field and let $A$ be a finitely generated $K$-algebra. Then for every prime ideal $\mathfrak p\subset A$ we have
  \begin{equation}
    \mathfrak p = \bigcap_{\substack{\mathfrak m \supseteq \mathfrak p \\ \textup{max. ideal}}} \mathfrak m;
  \end{equation}
  in other words, $A$ is Jacobson.
  Furthermore, if $\mathfrak m \subset A$ is a maximal ideal, the field extension $K\subseteq A/\mathfrak m$ is finite.
\end{theorem}
Let $K$ be algebraically closed, and let $\mathfrak m$ be a maximal ideal of $R$ containing $I$. Then with $R/\mathfrak m\cong K$, let $a_i$ be the image of $x_i$ in $K$. Then $\underline a = (a_1,\dots,a_n)$ is a suitable preimage of $\mathfrak m$ under $\varphi$.

This thought experiment reveals some limitations of using maximal ideals to parameterize solutions to systems of polynomial equations, since the map $\varphi$ above is not surjective in general. It may be helpful to consider a slightly more general perspective.

We continue, using the language of category theory. A summary of basic definitions and results in category theory is found in \cite{gw}*{Appendix A}.

In what follows, we do not assume that $K$ is algebraically closed. We can ``upgrade'' the map $\varphi$ above into a natural transformation. The bijection of sets 
\begin{equation}
  V(I)\cong \Hom_{\Kalg}(R/I,K)\quad \text{given by}\quad  \underline a\mapsto f\colon x_i\mapsto a_i
\end{equation}
(which holds since any homomorphism $R/I\to K$ of $K$-algebras is uniquely determined by its action on each $x_i$) implies that we may ``upgrade'' $V(-)$ to the functor $\Hom_{\Kalg}(-,K)\colon \Kalg\to\Set$. (Here we can think of $V(-)$ as a map that takes in $R/I$ and returns $V(I)$)

We want to refashion taking the set of maximal ideals of $R/I$ into a functor $\Kalg\to\Set$ given by the assignment
\begin{equation}
  \begin{tikzcd}[ampersand replacement=\&]
    A \& {\cbr{\text{max. ideals of }A}} \&\& A \& {\cbr{\text{max. ideals of }A}} \\
    \&\&\& B \& {\cbr{\text{max. ideals of }B}}
    \arrow[maps to, from=1-1, to=1-2]
    \arrow[""{name=0, anchor=center, inner sep=0}, "f", from=1-4, to=2-4]
    \arrow[""{name=1, anchor=center, inner sep=0}, "{f^{-1}}"', from=2-5, to=1-5]
    \arrow[shorten <=12pt, shorten >=12pt, maps to, from=0, to=1]
  \end{tikzcd},
\end{equation}
but this assignment is not functorial unless we restrict . It suffices to show that preimages of maximal ideals under $K$-algebra homomorphisms are maximal ideals. Let $f\colon A\to B$ be a $K$-algebra homomorphism and let $\mathfrak m$


\newpage\subsection*{08/30}

\newpage\subsection*{09/09}

\newpage\subsection*{09/11}

\newpage\subsection*{09/13}
Consider presheaves valued in Ab or Ring. We can rephrase the axioms for a sheaf $F$ in the following way: 
\begin{enumerate}[label=(\arabic*)]
    \item (Separatedness)
    \item (Gluing)
\end{enumerate}

\newpage
\printindex\newpage
\begin{bibdiv}
\begin{biblist}

\bib{gw}{book}{
    title = {Algebraic Geometry I: Schemes},
    author = {G\"ortz, Ulrich},
    author = {Wedhorn, Torsten},
    isbn = {978-3-658-30732-5},
    series = {Springer Studium Mathematik - Master},
    year = {2020},
    publisher = {Springer Spektrum Wiesbaden}
}

\bib{h}{book}{
    title = {Algebraic Geometry},
    author = {Hartshorne, Robin},
    isbn = {978-0-387-90244-9},
    series = {Graduate Texts in Mathematics},
    year = {1977},
    publisher = {Springer New York, NY}
}

\end{biblist}
\end{bibdiv}
\end{document}