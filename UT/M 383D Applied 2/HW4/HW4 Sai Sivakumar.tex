\documentclass[11pt,leqno]{article}
\headheight=13.6pt

% packages
\usepackage[alphabetic]{amsrefs}
\usepackage{physics}
% margin spacing
\usepackage[top=1in, bottom=1in, left=0.5in, right=0.5in]{geometry}
\usepackage{hanging}
\usepackage{amsfonts, amsmath, amssymb, amsthm}
\usepackage{systeme}
\usepackage[none]{hyphenat}
\usepackage{fancyhdr}
\usepackage{graphicx}
\graphicspath{{./images/}}
\usepackage{float}
\usepackage{siunitx}
\usepackage{esint}
\usepackage{color}
\usepackage{enumitem}
\usepackage{mathrsfs}
\usepackage{hyperref}
\usepackage[noabbrev, capitalise]{cleveref}
\crefformat{equation}{equation~#2#1#3}
\crefformat{lemma}{\textrm{Lemma}~#2#1#3}

% theorems
\theoremstyle{plain}
\newtheorem{lem}{Lemma}
\newtheorem{lemma}[lem]{Lemma}
\newtheorem{thm}[lem]{Theorem}
\newtheorem{theorem}[lem]{Theorem}
\newtheorem{prop}[lem]{Proposition}
\newtheorem{proposition}[lem]{Proposition}
\newtheorem{cor}[lem]{Corollary}
\newtheorem{corollary}[lem]{Corollary}
\newtheorem{conj}[lem]{Conjecture}
\newtheorem{fact}[lem]{Fact}
\newtheorem{form}[lem]{Formula}

\theoremstyle{definition}
\newtheorem{defn}[lem]{Definition}
\newtheorem{definition/}[lem]{Definition}
\newenvironment{definition}
  {\renewcommand{\qedsymbol}{\textdagger}%
   \pushQED{\qed}\begin{definition/}}
  {\popQED\end{definition/}}
\newtheorem{example}[lem]{Example}
\newtheorem{remark}[lem]{Remark}
\newtheorem{exercise}[lem]{Exercise}
\newtheorem{notation}[lem]{Notation}

\numberwithin{equation}{section}
\numberwithin{lem}{section}

% header/footer formatting
\pagestyle{fancy}
\fancyhead{}
\fancyfoot{}
\fancyhead[L]{M 383D}
\fancyhead[C]{HW4}
\fancyhead[R]{Sai Sivakumar}
\fancyfoot[R]{\thepage}
\renewcommand{\headrulewidth}{1pt}

% paragraph indentation/spacing
\setlength{\parindent}{0cm}
\setlength{\parskip}{10pt}
\renewcommand{\baselinestretch}{1.25}

% extra commands defined here
\newcommand{\br}[1]{\left(#1\right)}
\newcommand{\sbr}[1]{\left[#1\right]}
\newcommand{\cbr}[1]{\left\{#1\right\}}
\newcommand{\eq}[1]{\overset{#1}{=}}
\let\O\relax
\newcommand{\O}{\mathrm O}

% bracket notation for inner product
\usepackage{mathtools}

\DeclarePairedDelimiterX{\abr}[1]{\langle}{\rangle}{#1}

\DeclareMathOperator{\Span}{span}
\DeclareMathOperator{\im}{im}
\DeclareMathOperator{\dist}{dist}
\DeclareMathOperator{\diam}{diam}
\DeclareMathOperator{\supp}{supp}
\DeclareMathOperator{\EV}{ev}
\DeclareMathOperator{\co}{co}
\newcommand{\res}[1]{\operatorname*{res}_{#1}}
\DeclareMathOperator{\id}{id}
\let\PV\relax
\DeclareMathOperator{\PV}{PV}
\DeclareMathOperator{\sgn}{sgn}

% smileys frownies
\usepackage{wasysym}
\newcommand{\smallhappy}{\smiley}
\newcommand{\happy}{\raisebox{-.14em}{\resizebox{1.2em}{!}{\smiley}}}
\newcommand{\smallsad}{\frownie}
\newcommand{\sad}{\raisebox{-.14em}{\resizebox{1.2em}{!}{\frownie}}}
\DeclareMathOperator{\mathhappy}{\!\happy\!}
\DeclareMathOperator{\smallmathhappy}{\!\smallhappy\!}
\DeclareMathOperator{\mathsad}{\!\sad\!}
\DeclareMathOperator{\smallmathsad}{\!\smallsad\!}

\let\norm\undefined % <-- "Undefine" \norm
\DeclarePairedDelimiter\norm{\lVert}{\rVert}

% lol
\newcommand{\lqq}[1]{\overset{#1}{\leq}}

% set page count index to begin from 1
\setcounter{page}{1}

\begin{document}
\subsection*{7.8 Exercises (from \href{https://users.oden.utexas.edu/~arbogast/appMath08c.pdf}{online notes})}
\begin{enumerate}
    \item[2.] Let $\cbr{f_j}\subseteq \mathcal D(0,1)\cap H^1_0(0,1)$ converge to $f\in H^1_0(0,1)$. Then $f_j(0) = 0$ since each $f_j$ is compactly supported and $\norm{f_j}_2^2 = \int_0^1|\int_0^x f_j^\prime|^2\leq \int_0^1(\int_0^x|f_j^\prime|)^2\leq \int_0^1\norm{f_j^\prime}_2^2 = \norm{f_j^\prime}_2^2$. By taking limits we have $\norm{f}_2\leq \norm{f^\prime}_2$ (the maps $\norm{\cdot}_2$ and $\norm{D(\cdot)}_2$ are continuous on $H^1_0(0,1)$).
    
    Let $\cbr{f_j}\subseteq C^\infty(0,1)\cap H^1(0,1)$ converge to $f\in H^1(0,1)$ with $\int_0^1f_j = 0$ and $\int_0^1f = 0$ This can be done by choosing the $f_j$ in $C^\infty(0,1)\cap H^1(0,1)$ by density, passing to an increasing subsequence that converges almost everywhere to $f$, and shifting them by appropriate constants $c_j = \int_0^1 f_j$. Furthermore, take the $f_j$ to be uniformly bounded above by $\abs{f}$, since $f$ is also in $L_1(0,1)$. Then $\norm{D(f_j-c_j-f)}_2 = \norm{f_j^\prime-f^\prime}$ goes to zero and $\norm{f_j-c_j-f}_2\leq \norm{f_j-f}_2 + \abs{c_j}$, but $c_j$ tends to $0$ by the dominated convergence theorem. Since $\norm{f_j-f}_2$ goes to zero, $f_j-c_j$ converges to $f$. So without loss of generality, we may choose $f_j$ so that $\int_0^1 f_j = 0$.
    
    Then $f_j = \int_0^1f_j(x) - f_j(y)\dd y = \int_0^1\int_y^x f_j^\prime$ so that $\abs{f_j} \leq \int_0^1\int_y^x \abs{f^\prime}\leq \int_0^1\norm{\chi_{[y,x]}}_2\norm{f_j^\prime}_2\leq \norm{f_j^\prime}_2$. Then $\norm{f_j}_2 \leq \norm{f_j^\prime}_2$; by taking limits we obtain $\norm{f}_2\leq \norm{f^\prime}_2$.
    \item[4.] It suffices to prove the result on the dense subspace $C^\infty(0,1)\cap H^1(0,1)$ of $H^1(0,1)$ since uniformly continuous maps on dense subsets of metric spaces may be extended uniquely to the whole space. Let $\cbr{f_j}\subseteq C^\infty(0,1)\cap H^1(0,1)$ converge to zero. The $f_j$ are already continuous, and we have $\abs{f_j}\leq \int_{x_0}^x|f_j^\prime|\leq \norm{\chi_{[x_0,x]}}_2\norm{f_j^\prime}_2 \leq \norm{f_j^\prime}_2$ so that $\norm{f_j}_\infty\leq \norm{f_j^\prime}_2$. Since $\norm{f_j^\prime}_2$ tends to zero, the $f_j$ are bounded and $f_j$ converges to zero in the supremum norm on $C_B(0,1)$. Therefore the identity map is a continuous embedding of $C^\infty(0,1)\cap H^1(0,1)$ in $C_B(0,1)$, which implies $H^1(0,1)$ embeds continuously in $C_B(0,1)$.
    \item[5.] Let $R(x,i)$ be the maximum of $1$ and the maximal radius of a ball centered at $x\in \Omega$ contained in $U_i$. Form an open cover of $\overline \Omega$ by taking the union of balls $B_{R(x,i)}(x)\subseteq U_i$ for each $x\in \Omega$, and by compactness only finitely many $x_i$ are required to produce a cover of $\Omega$. Let $R_i$ be the radius of the ball containing $x_i$ generated above.
    
    With $f(t) = \exp(-1/t^2)\chi_{(0,\infty)}(t)$, $g_i(t) = f(x-R_i/2)f(R_i-x)$, and $h_i(t) = (\int_{R_i/2}^xg(t)\dd t)/(\int_{R_i/2}^{R_i}g(t)\dd t)$, let 
    \[\psi_i(x) = \begin{cases}
      1 & \text{for }x\in B_{R_i/2}(x_i)\\
      1-h_i(\abs{x-x_i}) & \text{for }x\in B_{R_i}(x_i)\setminus B_{R_i/2}(x_i)\\
      0 & \text{for } x\in \mathbb R^d\setminus B_{R_i}(x_i).
    \end{cases}\]

    Let $\phi_i = \psi_i/(\sum_j\psi_j)$; it follows that each $\phi_i$ is smooth, compactly supported on $B_{R_i}(x_i)\subseteq U_{i_k}$ for some $i_k$, and $\sum_i\phi_i(x) = (\sum_i\psi_i(x))/(\sum_j\psi_j(x)) = 1$ as required.
    \item[6.] Let $S$ be the set of points of $\Omega$ with rational coordinates, so that $S$ is countable. The countable collection of balls $\big\{B_{R(x)}(x)\mid R(x)\in \mathbb Q,~x\in S,~\overline{B_{R(x)}(x)}\subseteq U_\alpha\text{ for some }\alpha\in \mathcal I\big\}$ constitutes a cover of $\Omega$ and we order the balls by $B_{R_i}(x_i) \coloneqq B_{R(x_i)}(x_i)$ for $x_i\in S$. We repeat the construction in the previous problem: Let $f(t) = \exp(-1/t^2)\chi_{(0,\infty)}(t)$, $g_i(t) = f(x-R_i/2)f(R_i-x)$, and $h_i(t) = (\int_{R_i/2}^xg(t)\dd t)/(\int_{R_i/2}^{R_i}g(t)\dd t)$, let 
    \[\phi_i(x) = \begin{cases}
      1 & \text{for }x\in B_{R_i/2}(x_i)\\
      1-h_i(\abs{x-x_i}) & \text{for }x\in B_{R_i}(x_i)\setminus B_{R_i/2}(x_i)\\
      0 & \text{for } x\in \mathbb R^d\setminus B_{R_i}(x_i).
    \end{cases}\]
    Let $\psi_i = \phi_i\cdot\prod_{j=1}^{i-1}(1-\phi_j)$, where $\prod_{j=1}^{0}(1-\phi_j) = 1$.
    
    For $K$ compactly supported in $\Omega$, finitely many balls $B_{R_{i_1}}(x_{i_1}),\dots,B_{R_{i_n}}(x_{i_n})$ constitute a cover of $K$. Choose $j> \max{i_1,\dots,i_n}$ so that $\psi_j$ vanishes on $K$, on account of $K\subseteq \bigcup_kB_{R_{i_k}}(x_k)$ and the product $\prod_k(1-\phi_{i_k})$ appearing in $\psi_j$. Hence all but finitely many $\psi_j$ vanish on $K$. Each $\psi_j$ is nonnegative by construction. Since $\cbr{x}$ is compact for a point $x\in \Omega$ the function $N(x)$, given by the smallest index for which $\phi_{N(x)+1}(x) = 0$, is defined. Then $\phi_{N(x)}(x) = 1$ since $1-\phi_{N(x)}(x) = 0$ in the expansion for $\phi_{N(x)+1}(x)$. Therefore $\sum_{j=1}^\infty\psi_j(x) = \sum_{j=1}^{N(x)}\psi_j(x) = \phi_1(x) + \phi_2(x)(1-\phi_1(x)) + \cdots + \phi_{N(x)-1}(x)(1-\phi_{N(x)-2}(x))\cdots(1-\phi_1(x)) + (1-\phi_{N(x)-1}(x))(1-\phi_{N(x)-2}(x))\cdots(1-\phi_1(x)) = \cdots = 1$. The support of $\psi_j$ is the closure of $B_{R_j}(x_j)\setminus (\bigcup_{i=1}^{j-1}B_{R_i/2}(x_i))$, which is contained in some $U_{\alpha_j}$ by construction.
\end{enumerate}
\end{document}