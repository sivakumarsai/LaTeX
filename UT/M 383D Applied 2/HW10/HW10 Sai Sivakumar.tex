\documentclass[11pt,leqno]{article}
\headheight=13.6pt

% packages
\usepackage[alphabetic]{amsrefs}
\usepackage{physics}
% margin spacing
\usepackage[top=1in, bottom=1in, left=0.5in, right=0.5in]{geometry}
\usepackage{hanging}
\usepackage{amsfonts, amsmath, amssymb, amsthm}
\usepackage{systeme}
\usepackage[none]{hyphenat}
\usepackage{fancyhdr}
\usepackage{graphicx}
\graphicspath{{./images/}}
\usepackage{float}
\usepackage{siunitx}
\usepackage{esint}
\usepackage{color}
\usepackage{enumitem}
\usepackage{mathrsfs}
\usepackage{hyperref}
\usepackage[noabbrev, capitalise]{cleveref}
\crefformat{equation}{equation~#2#1#3}
\crefformat{lemma}{\textrm{Lemma}~#2#1#3}

% theorems
\theoremstyle{plain}
\newtheorem{lem}{Lemma}
\newtheorem{lemma}[lem]{Lemma}
\newtheorem{thm}[lem]{Theorem}
\newtheorem{theorem}[lem]{Theorem}
\newtheorem{prop}[lem]{Proposition}
\newtheorem{proposition}[lem]{Proposition}
\newtheorem{cor}[lem]{Corollary}
\newtheorem{corollary}[lem]{Corollary}
\newtheorem{conj}[lem]{Conjecture}
\newtheorem{fact}[lem]{Fact}
\newtheorem{form}[lem]{Formula}

\theoremstyle{definition}
\newtheorem{defn}[lem]{Definition}
\newtheorem{definition/}[lem]{Definition}
\newenvironment{definition}
  {\renewcommand{\qedsymbol}{\textdagger}%
   \pushQED{\qed}\begin{definition/}}
  {\popQED\end{definition/}}
\newtheorem{example}[lem]{Example}
\newtheorem{remark}[lem]{Remark}
\newtheorem{exercise}[lem]{Exercise}
\newtheorem{notation}[lem]{Notation}

\numberwithin{equation}{section}
\numberwithin{lem}{section}

% header/footer formatting
\pagestyle{fancy}
\fancyhead{}
\fancyfoot{}
\fancyhead[L]{M 383D}
\fancyhead[C]{HW10}
\fancyhead[R]{Sai Sivakumar}
\fancyfoot[R]{\thepage}
\renewcommand{\headrulewidth}{1pt}

% paragraph indentation/spacing
\setlength{\parindent}{0cm}
\setlength{\parskip}{10pt}
\renewcommand{\baselinestretch}{1.25}

% extra commands defined here
\newcommand{\br}[1]{\left(#1\right)}
\newcommand{\sbr}[1]{\left[#1\right]}
\newcommand{\cbr}[1]{\left\{#1\right\}}
\newcommand{\eq}[1]{\overset{#1}{=}}

% bracket notation for inner product
\usepackage{mathtools}

\DeclarePairedDelimiterX{\abr}[1]{\langle}{\rangle}{#1}

\DeclareMathOperator{\Span}{span}
\DeclareMathOperator{\im}{im}
\DeclareMathOperator{\dist}{dist}
\DeclareMathOperator{\diam}{diam}
\DeclareMathOperator{\supp}{supp}
\newcommand{\res}[1]{\operatorname*{res}_{#1}}
\DeclareMathOperator{\id}{id}
\let\PV\relax
\DeclareMathOperator{\PV}{PV}
\DeclareMathOperator{\sgn}{sgn}

% smileys frownies
\usepackage{wasysym}
\newcommand{\smallhappy}{\smiley}
\newcommand{\happy}{\raisebox{-.14em}{\resizebox{1.2em}{!}{\smiley}}}
\newcommand{\smallsad}{\frownie}
\newcommand{\sad}{\raisebox{-.14em}{\resizebox{1.2em}{!}{\frownie}}}
\DeclareMathOperator{\mathhappy}{\!\happy\!}
\DeclareMathOperator{\smallmathhappy}{\!\smallhappy\!}
\DeclareMathOperator{\mathsad}{\!\sad\!}
\DeclareMathOperator{\smallmathsad}{\!\smallsad\!}

\let\norm\undefined % <-- "Undefine" \norm
\DeclarePairedDelimiter\norm{\lVert}{\rVert}

% set page count index to begin from 1
\setcounter{page}{1}

\begin{document}
\subsection*{8.8 Exercises (from \href{https://users.oden.utexas.edu/~arbogast/appMath08c.pdf}{online notes})}
\begin{enumerate}
    \item[14.] 
    \item[17.] 
\end{enumerate}
\subsection*{9.9 Exercises (from \href{https://users.oden.utexas.edu/~arbogast/appMath08c.pdf}{online notes})}
\begin{enumerate}
    \item[1.] 
    \begin{enumerate}
        \item Bilinearity of $P$ yields $P(y_1+h_1, y_2 + h_2) - P(y_1,y_2) = P(y_1,h_2) + P(h_1,y_2) + P(h_1,h_2)$. Since $P(h_1,h_2)$ is continuous, $P(h_1,h_2) = o((h_1,h_2))$ so that $DP(y_1,y_2)(\hat y_1,\hat y_2) = P(y_1,\hat y_2) + P(\hat y_1,y_2)$.
        \item The chain rule yields $D(Pf)(x)h = DP(f(x))Df(x)h = P(Df_1(x)h, f_2(x)) + P(f_1(x),Df_2(x)h)$, where $f = (f_1,f_2)$ and $Df(x) = (Df_1(x),Df_2(x))$ as functions on $Y_1\times Y_2$. That $[Df(x)]_1$ agrees with $Df_1(x)$ is due to linearity.
    \end{enumerate}
    \item[2.] We have $f(x+h) - f(x) = (\abr{x,A_1h} + \abr{h,A_1x})A_2x + \abr{x,A_1x}A_2h + o(h)$ by continuity of $A_1$, $A_2$, and the inner product on $X$. So for any $x\in X$, $Df(x)h = (\abr{x,A_1h} + \abr{h,A_1x})A_2x + \abr{x,A_1x}A_2h$, which we should expect from a na\"ive product rule, Exercise 1, and that linear maps are their own derivatives.
    \item[3.] Differentiability of $f$ implies 
    \[F(u+h) - F(u) = \int_0^1K(\,\cdot\, ,y)[f(u(y) + h(y)) - f(u(y))]\dd y = \int_0^1K(\,\cdot\, ,y)[f^\prime(u(y))h(y) + R(u(y), h(y))]\dd y,\]
    where $R(u,h)$ given by $R(u(x), h(y)) = f(u(x) + h(y)) - f(u(y)) - f^\prime(u(x))h(y)$ is continuous on $[0,1]^2$ and $o(h(y))$ for each $x,y\in [0,1]$. Then $\norm{\int_0^1K(\,\cdot\, ,y)R(u(y), h(y))\dd y}_\infty\leq \norm{K}_\infty\norm{R(u,h)}_\infty$, which implies $\int_0^1K(\,\cdot\, ,y)R(u(y), h(y))\dd y = o(h)$ since continuous functions on compact sets attain their maximum value (in particular, the estimate we need is $\norm{R(u,h)}_\infty/\norm{h}_\infty\leq \abs{R(u(x^\ast),h(y^\ast_h))}/\abs{h(y^\ast_h)}\to 0$ as $\norm{h}_\infty\to 0$). Hence $DF(u)h = \int_0^1K(\,\cdot\, ,y)f^\prime(u(y))h(y)\dd y$ for any $h\in C([0,1])$.

    Let $\cbr{u_j}$ converge to $u$ in $C([0,1])$. The uniform continuity of $f^\prime$ on a sufficiently large closed interval implies
    \[\norm{DF(u)-DF(u_j)} = \sup_{\substack{h\in C([0,1])\\ \norm{h}_\infty = 1}}\norm*{\int_0^1K(\,\cdot\, ,y)[f^\prime(u(y))- f^\prime(u_j(y))]h(y)\dd y}_\infty\] may be made arbitrarily small by taking $j$ arbitrarily large. Therefore the map $u\mapsto DF(u)$ is continuous.
    \item[5.] Apply the Fourier transform on both sides of $(1-D^2)u - \epsilon u^2 = f$ to obtain $(1+\xi^2)\mathcal F u - \epsilon \mathcal F(u^2) = \mathcal F f$. Therefore $u = \mathcal F^{-1}[(1+\xi^2)^{-1}\mathcal F(f+\epsilon (u^2))] = \exp\relax(-\abs{\,\cdot\,}^2)/2\ast (f+\epsilon u^2)$, so the partial differential equation is (at least formally) equivalent to the previous equation. The space $C^\infty_B(\mathbb R)$ is complete, so define $\Phi\colon C^\infty_B(\mathbb R)\to C^\infty_B(\mathbb R)$ by $\Phi(u) = \exp\relax(-\abs{\,\cdot\,}^2)/2\ast (f+\epsilon u^2)$ and observe that $\rho_n(\Phi(u)-\Phi(v)) = \epsilon\norm{\exp\relax(-\abs{\,\cdot\,}^2)/2\ast D^n(u^2-v^2)}_\infty \leq \epsilon\norm{\exp\relax(-\abs{\,\cdot\,}^2)/2}_1\norm{D^n(u^2-v^2)}_\infty$ for all $n\geq 0$ by Young's inequality. Thus $d(\Phi(u),\Phi(v))\leq \epsilon\norm{\exp\relax(-\abs{\,\cdot\,}^2)/2}_1d(u,v)$, which implies $\Phi$ is a contraction for $\epsilon$ taken sufficiently small. Thus there is a unique fixed point $u$ of $\Phi$ that is a solution to the equation $u = \exp\relax(-\abs{\,\cdot\,}^2)/2\ast (f+\epsilon u^2)$ and hence a formal solution to the partial differential equation $(1-D^2)u - \epsilon u^2 = f$.
    \item[6.] Let $\Phi\colon C([a,b])\to C([a,b])$ be given by $\Phi(f) = \varphi + \lambda\int_a^b K(\,\cdot\,,y)f(y)\dd y$. That $\Phi$ is a contraction is due to the estimate $d(\Phi(f)-\Phi(g)) = \norm{\lambda\int_a^bK(\,\cdot\,,y)(f(y)-g(y))\dd y}_\infty\leq \lambda(b-a)\norm{K}_\infty\norm{f-g}_\infty$, where $\lambda$ is chosen sufficiently small. Therefore there is a unique fixed point $f$ of $\Phi$ that is a solution to the Fredholm integral equation.
\end{enumerate}
\end{document}