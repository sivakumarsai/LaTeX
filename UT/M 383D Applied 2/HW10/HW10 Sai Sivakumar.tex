\documentclass[11pt,leqno]{article}
\headheight=13.6pt

% packages
\usepackage[alphabetic]{amsrefs}
\usepackage{physics}
% margin spacing
\usepackage[top=1in, bottom=1in, left=0.5in, right=0.5in]{geometry}
\usepackage{hanging}
\usepackage{amsfonts, amsmath, amssymb, amsthm}
\usepackage{systeme}
\usepackage[none]{hyphenat}
\usepackage{fancyhdr}
\usepackage{graphicx}
\graphicspath{{./images/}}
\usepackage{float}
\usepackage{siunitx}
\usepackage{esint}
\usepackage{color}
\usepackage{enumitem}
\usepackage{mathrsfs}
\usepackage{hyperref}
\usepackage[noabbrev, capitalise]{cleveref}
\crefformat{equation}{equation~#2#1#3}
\crefformat{lemma}{\textrm{Lemma}~#2#1#3}

% theorems
\theoremstyle{plain}
\newtheorem{lem}{Lemma}
\newtheorem{lemma}[lem]{Lemma}
\newtheorem{thm}[lem]{Theorem}
\newtheorem{theorem}[lem]{Theorem}
\newtheorem{prop}[lem]{Proposition}
\newtheorem{proposition}[lem]{Proposition}
\newtheorem{cor}[lem]{Corollary}
\newtheorem{corollary}[lem]{Corollary}
\newtheorem{conj}[lem]{Conjecture}
\newtheorem{fact}[lem]{Fact}
\newtheorem{form}[lem]{Formula}

\theoremstyle{definition}
\newtheorem{defn}[lem]{Definition}
\newtheorem{definition/}[lem]{Definition}
\newenvironment{definition}
  {\renewcommand{\qedsymbol}{\textdagger}%
   \pushQED{\qed}\begin{definition/}}
  {\popQED\end{definition/}}
\newtheorem{example}[lem]{Example}
\newtheorem{remark}[lem]{Remark}
\newtheorem{exercise}[lem]{Exercise}
\newtheorem{notation}[lem]{Notation}

\numberwithin{equation}{section}
\numberwithin{lem}{section}

% header/footer formatting
\pagestyle{fancy}
\fancyhead{}
\fancyfoot{}
\fancyhead[L]{M 383D}
\fancyhead[C]{HW10}
\fancyhead[R]{Sai Sivakumar}
\fancyfoot[R]{\thepage}
\renewcommand{\headrulewidth}{1pt}

% paragraph indentation/spacing
\setlength{\parindent}{0cm}
\setlength{\parskip}{10pt}
\renewcommand{\baselinestretch}{1.25}

% extra commands defined here
\newcommand{\br}[1]{\left(#1\right)}
\newcommand{\sbr}[1]{\left[#1\right]}
\newcommand{\cbr}[1]{\left\{#1\right\}}
\newcommand{\eq}[1]{\overset{#1}{=}}

% bracket notation for inner product
\usepackage{mathtools}

\DeclarePairedDelimiterX{\abr}[1]{\langle}{\rangle}{#1}

\DeclareMathOperator{\Span}{span}
\DeclareMathOperator{\im}{im}
\DeclareMathOperator{\dist}{dist}
\DeclareMathOperator{\diam}{diam}
\DeclareMathOperator{\supp}{supp}
\newcommand{\res}[1]{\operatorname*{res}_{#1}}
\DeclareMathOperator{\id}{id}
\let\PV\relax
\DeclareMathOperator{\PV}{PV}
\DeclareMathOperator{\sgn}{sgn}

% smileys frownies
\usepackage{wasysym}
\newcommand{\smallhappy}{\smiley}
\newcommand{\happy}{\raisebox{-.14em}{\resizebox{1.2em}{!}{\smiley}}}
\newcommand{\smallsad}{\frownie}
\newcommand{\sad}{\raisebox{-.14em}{\resizebox{1.2em}{!}{\frownie}}}
\DeclareMathOperator{\mathhappy}{\!\happy\!}
\DeclareMathOperator{\smallmathhappy}{\!\smallhappy\!}
\DeclareMathOperator{\mathsad}{\!\sad\!}
\DeclareMathOperator{\smallmathsad}{\!\smallsad\!}

\let\norm\undefined % <-- "Undefine" \norm
\DeclarePairedDelimiter\norm{\lVert}{\rVert}

% set page count index to begin from 1
\setcounter{page}{1}

\begin{document}
\subsection*{8.8 Exercises (from \href{https://users.oden.utexas.edu/~arbogast/appMath08c.pdf}{online notes})}
\begin{enumerate}
    \item[14.] \begin{enumerate}
        \item Assume also that $\Omega$ is bounded. The spaces $V_n$ are finite-dimensional and hence closed in $H^1(\Omega)$. The symmetric bilinear form $B\colon H^1(\Omega)\times H^1(\Omega)\to \mathbb R$ given by $B(u,v) = \abr{\nabla u,\nabla v}_{L^2(\Omega)^d} + \abr{u,v}_{L^2(\Omega)}$ is continuous since $\abs{B(u,v)}\leq \norm{\nabla u}_{L^2(\Omega)^d}\norm{\nabla v}_{L^2(\Omega)^d} + \norm{u}_{L^2(\Omega)}\norm{v}_{L^2(\Omega)}\leq M\norm{u}_{H^1(\Omega)}\norm{v}_{H^1(\Omega)}$ and coercive since $B(u,u) = \norm{u}_{H^1(\Omega)}^2$; but we should also see that $B$ is just the $H^1(\Omega)$ inner product. Integrating against $f$ defines $F = \abr{f,\cdot}_{L^2(\Omega)}\in H^1(\Omega)^\ast$. Thus for each $n$, there exists a unique solution $u_n$ solving the $n$-th variational problem given in the problem statement. Furthermore, we have $\norm{f}_{L^2(\Omega)}\norm{u_n}_{L^2{\Omega}}\geq \abr{f,u_n}_{L^2(\Omega)} = B(u_n,u_n) = \norm{u_n}_{H^1(\Omega)}^2$ so that $\norm{u_n}_{L^2(\Omega)}\leq \norm{f}_{L^2}$ for each $n$.
        \item Since $\norm{u_n}_{L^2(\Omega)}\leq \norm{f}_{L^2}$ for each $n$, by the Banach-Alaoglu theorem there exists a $u\in H^1(\Omega)$ that $\cbr{u_n}$ converges weakly to. The variational problem that $u$ satisfies is essentially the same problem the $u_n$ satisfy, but with $V_n$ replaced by $H^1(\Omega)$, since we may take limits in the inner products defining $B(u_n,v_n)$ (first in $u_n$ by weak convergence above and then in $v_n$ since polynomials are dense in $H^1(\Omega)$) and $F(v_n)$: We ask for solutions in $H^1(\Omega)$ to $B(u,v) = F(v)$ for any $v\in H^1(\Omega)$.
        \item We have for any $v_n\in V_n\subset H^1(\Omega)$ that $B(u-u_n,v_n) = 0$ since both $B(u,v_n) = F(v_n)$ and $B(u_n,v_n) = F(v_n)$. Then by taking $v_n = u_n - v_n$, we have that $\norm{u-u_n}_{H^1(\Omega)}^2 = B(u-u_n,u-u_n) = B(u-u_n,u-v_n) \leq \norm{u-u_n}_{H^1(\Omega)}\norm{u-v_n}_{H^1(\Omega)}$. Then by taking the infimum over $v_n\in V_n$ we have $\norm{u-u_n}_{H^1(\Omega)}\leq \inf_{v_n\in V_n}\norm{u-v_n}_{H^1(\Omega)}$ for each $n$, in which the quantity $\inf_{v_n\in V_n}\norm{u-v_n}_{H^1(\Omega)}$ may be made monotonically small in $n$ since the orthogonal complements of the $V_n$ form a decreasing filtration; in other words, because the union of the $V_n$ (which form an increasing filtration) are dense in $H^1(\Omega)$.
        \item The trace functions $\gamma_0$ and $\gamma_1$ on $H^1(\Omega)$ are continuous, so $\gamma_0 u = \lim_{n\to\infty}\gamma_0 u_n$ and similarly $\gamma_1 u = \lim_{n\to\infty}\gamma_1 u_n$. So the restrictions of $u$ and the normal derivative of $u$ to the boundary of $\Omega$ may be obtained by restricting the functions $u_n$ and their normal derivatives to the boundary and taking limits.
        
        Not sure what else we can do? Somehow there must be an implicit Neumann condition (if the variational problem is obtained from a partial differential equation) in each of the problems posed for each $n$, so that the normal derivatives of $u_n$ will have to be zero on the boundary, and thus also for the normal derivative of $u$ as well.
    \end{enumerate}
    \item[17.] \begin{enumerate}
        \item Since $1\cdot 2>1$, by the Sobolev embedding theorem $H^1_0(0,1)$ embeds continuously into $C_B(0,1)$, on which we may genuinely evaluate functions at any of the $x_j$ and obtain finite values. It follows that $\mathcal I_h$ is well defined.
    \end{enumerate}
\end{enumerate}
\subsection*{9.9 Exercises (from \href{https://users.oden.utexas.edu/~arbogast/appMath08c.pdf}{online notes})}
\begin{enumerate}
    \item[1.] 
    \begin{enumerate}
        \item Bilinearity of $P$ yields $P(y_1+h_1, y_2 + h_2) - P(y_1,y_2) = P(y_1,h_2) + P(h_1,y_2) + P(h_1,h_2)$. Since $P(h_1,h_2)$ is continuous, $P(h_1,h_2) = o((h_1,h_2))$ so that $DP(y_1,y_2)(\hat y_1,\hat y_2) = P(y_1,\hat y_2) + P(\hat y_1,y_2)$.
        \item The chain rule yields $D(Pf)(x)h = DP(f(x))Df(x)h = P(Df_1(x)h, f_2(x)) + P(f_1(x),Df_2(x)h)$, where $f = (f_1,f_2)$ and $Df(x) = (Df_1(x),Df_2(x))$ as functions on $Y_1\times Y_2$. That $[Df(x)]_1$ agrees with $Df_1(x)$ is due to linearity.
    \end{enumerate}
    \item[2.] We have $f(x+h) - f(x) = (\abr{x,A_1h} + \abr{h,A_1x})A_2x + \abr{x,A_1x}A_2h + o(h)$ by continuity of $A_1$, $A_2$, and the inner product on $X$. So for any $x\in X$, $Df(x)h = (\abr{x,A_1h} + \abr{h,A_1x})A_2x + \abr{x,A_1x}A_2h$, which we should expect from a na\"ive product rule, Exercise 1, and that linear maps are their own derivatives.
    \item[3.] The map $F$ is the composition of the linear map $\tilde K\colon C([0,1])\to C([0,1])$ given by $u\mapsto \int_0^1K(\,\cdot\,,y)u(y)\dd y$ and $G\colon C([0,1])\to C([0,1])$ given by $u\mapsto f\circ u$. Since $\tilde K$ is linear, it is differentiable. It remains to show that $G$ is differentiable. We have $G(u+h) - G(u) = f(u+h) - f(u) = f^\prime(u)\cdot h + R(u,h)$ as elements of $C([0,1])$, where $R(x,t)= o(t)$. Since $f^\prime$ is uniformly continuous on compact sets, we have that $g(\,\cdot\,,t) = [f(u(\,\cdot\,) + t) - f(u(\,\cdot\,))]/t$ converges uniformly (on a large enough compact set depending on $u$ or otherwise containing the image of $u$) to $f^\prime(u(\,\cdot\,))$. Therefore for $\varepsilon>0$ we may choose $h$ with small enough $L^\infty([0,1])$-norm so that $\abs{R(u(y),h(y))}\leq \varepsilon\abs{h(y)}$ for all $y\in [0,1]$, which will imply that $\norm{R(u,h)}_\infty/\norm{h}_\infty$ may be made arbitrarily small; that is $R(u,h) = o(\norm{h}_\infty)$. Thus $G$ is differentiable with $DG(u) = f^\prime(u)\cdot h$. The chain rule implies that $DF(u)h = \int_0^1K(\,\cdot\,,y)f^\prime(u(y))h(y)\dd y$.
    % Differentiability of $f$ implies 
    % \[F(u+h) - F(u) = \int_0^1K(\,\cdot\, ,y)[f(u(y) + h(y)) - f(u(y))]\dd y = \int_0^1K(\,\cdot\, ,y)[f^\prime(u(y))h(y) + R(u(y), h(y))]\dd y,\]
    % where $R(u,h)$ given by $R(u(x), h(y)) = f(u(x) + h(y)) - f(u(y)) - f^\prime(u(x))h(y)$ is continuous on $[0,1]^2$ and $o(h(y))$ for each $x,y\in [0,1]$. Then $\norm{\int_0^1K(\,\cdot\, ,y)R(u(y), h(y))\dd y}_\infty\leq \norm{K}_\infty\norm{R(u,h)}_\infty$, which implies $\int_0^1K(\,\cdot\, ,y)R(u(y), h(y))\dd y = o(h)$ since continuous functions on compact sets attain their maximum value (in particular, the estimate we need is $\norm{R(u,h)}_\infty/\norm{h}_\infty\leq \abs{R(u(x^\ast),h(y^\ast_h))}/\abs{h(y^\ast_h)}\to 0$ as $\norm{h}_\infty\to 0$). Hence $DF(u)h = \int_0^1K(\,\cdot\, ,y)f^\prime(u(y))h(y)\dd y$ for any $h\in C([0,1])$.

    Let $\cbr{u_j}$ converge to $u$ in $C([0,1])$. The uniform continuity of $f^\prime$ on a sufficiently large closed interval implies
    \[\norm{DF(u)-DF(u_j)} = \sup_{\substack{h\in C([0,1])\\ \norm{h}_\infty = 1}}\norm*{\int_0^1K(\,\cdot\, ,y)[f^\prime(u(y))- f^\prime(u_j(y))]h(y)\dd y}_\infty\] may be made arbitrarily small by taking $j$ arbitrarily large. Therefore the map $u\mapsto DF(u)$ is continuous.
    \item[5.] (for next week)
    % Apply the Fourier transform on both sides of $(1-D^2)u - \epsilon u^2 = f$ to obtain $(1+\xi^2)\mathcal F u - \epsilon \mathcal F(u^2) = \mathcal F f$. Therefore $u = \mathcal F^{-1}[(1+\xi^2)^{-1}\mathcal F(f+\epsilon (u^2))] = \exp\relax(-\abs{\,\cdot\,}^2)/2\ast (f+\epsilon u^2)$, so the partial differential equation is (at least formally) equivalent to the previous equation. The space $C^\infty_B(\mathbb R)$ is complete, so define $\Phi\colon C^\infty_B(\mathbb R)\to C^\infty_B(\mathbb R)$ by $\Phi(u) = \exp\relax(-\abs{\,\cdot\,}^2)/2\ast (f+\epsilon u^2)$ and observe that $\rho_n(\Phi(u)-\Phi(v)) = \epsilon\norm{\exp\relax(-\abs{\,\cdot\,}^2)/2\ast D^n(u^2-v^2)}_\infty \leq \epsilon\norm{\exp\relax(-\abs{\,\cdot\,}^2)/2}_1\norm{D^n(u^2-v^2)}_\infty$ for all $n\geq 0$ by Young's inequality. Thus $d(\Phi(u),\Phi(v))\leq \epsilon\norm{\exp\relax(-\abs{\,\cdot\,}^2)/2}_1d(u,v)$, which implies $\Phi$ is a contraction for $\epsilon$ taken sufficiently small. Thus there is a unique fixed point $u$ of $\Phi$ that is a solution to the equation $u = \exp\relax(-\abs{\,\cdot\,}^2)/2\ast (f+\epsilon u^2)$ and hence a formal solution to the partial differential equation $(1-D^2)u - \epsilon u^2 = f$.
    \item[6.] (for next week)
    % Let $\Phi\colon C([a,b])\to C([a,b])$ be given by $\Phi(f) = \varphi + \lambda\int_a^b K(\,\cdot\,,y)f(y)\dd y$. That $\Phi$ is a contraction is due to the estimate $d(\Phi(f)-\Phi(g)) = \norm{\lambda\int_a^bK(\,\cdot\,,y)(f(y)-g(y))\dd y}_\infty\leq \lambda(b-a)\norm{K}_\infty\norm{f-g}_\infty$, where $\lambda$ is chosen sufficiently small. Therefore there is a unique fixed point $f$ of $\Phi$ that is a solution to the Fredholm integral equation.
\end{enumerate}
I had not finished typing up problem 17 above at this time; there may be a late second submission that has the remainder of the problem. Apologies for the late submission if there is one!
\end{document}