\documentclass[11pt,leqno]{article}
\headheight=13.6pt

% packages
\usepackage[alphabetic]{amsrefs}
\usepackage{physics}
% margin spacing
\usepackage[top=1in, bottom=1in, left=0.5in, right=0.5in]{geometry}
\usepackage{hanging}
\usepackage{amsfonts, amsmath, amssymb, amsthm}
\usepackage{systeme}
\usepackage[none]{hyphenat}
\usepackage{fancyhdr}
\usepackage{graphicx}
\graphicspath{{./images/}}
\usepackage{float}
\usepackage{siunitx}
\usepackage{esint}
\usepackage{color}
\usepackage{enumitem}
\usepackage{mathrsfs}
\usepackage{hyperref}
\usepackage[noabbrev, capitalise]{cleveref}
\crefformat{equation}{equation~#2#1#3}
\crefformat{lemma}{\textrm{Lemma}~#2#1#3}

% theorems
\theoremstyle{plain}
\newtheorem{lem}{Lemma}
\newtheorem{lemma}[lem]{Lemma}
\newtheorem{thm}[lem]{Theorem}
\newtheorem{theorem}[lem]{Theorem}
\newtheorem{prop}[lem]{Proposition}
\newtheorem{proposition}[lem]{Proposition}
\newtheorem{cor}[lem]{Corollary}
\newtheorem{corollary}[lem]{Corollary}
\newtheorem{conj}[lem]{Conjecture}
\newtheorem{fact}[lem]{Fact}
\newtheorem{form}[lem]{Formula}

\theoremstyle{definition}
\newtheorem{defn}[lem]{Definition}
\newtheorem{definition/}[lem]{Definition}
\newenvironment{definition}
  {\renewcommand{\qedsymbol}{\textdagger}%
   \pushQED{\qed}\begin{definition/}}
  {\popQED\end{definition/}}
\newtheorem{example}[lem]{Example}
\newtheorem{remark}[lem]{Remark}
\newtheorem{exercise}[lem]{Exercise}
\newtheorem{notation}[lem]{Notation}

\numberwithin{equation}{section}
\numberwithin{lem}{section}

% header/footer formatting
\pagestyle{fancy}
\fancyhead{}
\fancyfoot{}
\fancyhead[L]{M 383D}
\fancyhead[C]{HW2}
\fancyhead[R]{Sai Sivakumar}
\fancyfoot[R]{\thepage}
\renewcommand{\headrulewidth}{1pt}

% paragraph indentation/spacing
\setlength{\parindent}{0cm}
\setlength{\parskip}{10pt}
\renewcommand{\baselinestretch}{1.25}

% extra commands defined here
\newcommand{\br}[1]{\left(#1\right)}
\newcommand{\sbr}[1]{\left[#1\right]}
\newcommand{\cbr}[1]{\left\{#1\right\}}
\newcommand{\eq}[1]{\overset{#1}{=}}
\let\O\relax
\newcommand{\O}{\mathrm O}

% bracket notation for inner product
\usepackage{mathtools}

\DeclarePairedDelimiterX{\abr}[1]{\langle}{\rangle}{#1}

\DeclareMathOperator{\Span}{span}
\DeclareMathOperator{\im}{im}
\DeclareMathOperator{\dist}{dist}
\DeclareMathOperator{\diam}{diam}
\DeclareMathOperator{\supp}{supp}
\DeclareMathOperator{\EV}{ev}
\DeclareMathOperator{\co}{co}
\newcommand{\res}[1]{\operatorname*{res}_{#1}}
\DeclareMathOperator{\id}{id}
\let\PV\relax
\DeclareMathOperator{\PV}{PV}

% smileys frownies
\usepackage{wasysym}
\newcommand{\smallhappy}{\smiley}
\newcommand{\happy}{\raisebox{-.14em}{\resizebox{1.2em}{!}{\smiley}}}
\newcommand{\smallsad}{\frownie}
\newcommand{\sad}{\raisebox{-.14em}{\resizebox{1.2em}{!}{\frownie}}}
\DeclareMathOperator{\mathhappy}{\!\happy\!}
\DeclareMathOperator{\smallmathhappy}{\!\smallhappy\!}
\DeclareMathOperator{\mathsad}{\!\sad\!}
\DeclareMathOperator{\smallmathsad}{\!\smallsad\!}

% lol
\newcommand{\gq}[1]{\overset{#1}{\geq}}

% set page count index to begin from 1
\setcounter{page}{1}

\begin{document}
\subsection*{6.6 Exercises (from \href{https://users.oden.utexas.edu/~arbogast/appMath08c.pdf}{online notes})}
I worked on this problem set.
\begin{enumerate}
    \item[6.] The canonical example is $f(x) = \prod_{i=1}^d\sin(x_i)/x_i$. The function $f$ is square integrable but not integrable, and since $\mathcal F\chi_{[-1,1]^d}(\xi) = \prod_{i=1}^d(2/\pi)^{d/2}\sin(\xi_i)/\xi_i$, by the Schwartz theory we have $\mathcal Ff = (2/\pi)^{-d/2}\chi_{[-1,1]^d}(\xi)$, which is integrable. \textcolor{red}{when can this happen?}
    \item[7.] Let $f\in L_p(\mathbb R^d)$ for $1 < p < 2$ (there is nothing to show when $p = 1,2$  ). \begin{enumerate}
      \item Let $f_1 = f\chi_{\{\abs{f}>\norm{f}_p\}}$ and $f_2 = f\chi_{\{\abs{f}\leq \norm{f}_p\}}$. If $a\geq b$ and $p\geq q$, then $a^q/b^q\leq a^p/b^p$ so that $a^q\leq a^pb^{q-p}$ and $b^p\leq b^{q}a^{p-q}$. Hence 
      \[\norm{f_1}_1 = \int \abs{f_1}\leq \norm{f}_p^{1-p}\int \abs{f}^p = \norm{f}_p\quad \text{and}\quad \norm{f_2}_2^2 = \int \abs{f_2}^2\leq \norm{f}_p^{2-p}\int\abs{f}^p = \norm{f}_p^2.\]
      \item If $f = f_1 + f_2 = f_1^\prime + f_2^\prime$, then $f_1 - f_1^\prime = f_2^\prime - f_2$. By Lemma 6.26, $\hat{f_1} - \hat{f_1^\prime} = \widehat{f_1 - f_1^\prime} = \widehat{f_2^\prime - f_2} = \hat{f_2^\prime} - \hat{f_2}$, so that $\hat{f} = \hat{f_1} + \hat{f_2} = \hat{f_1^\prime} + \hat{f_2^\prime}$ as desired.
    \end{enumerate}
    \item[8.] Observe that $Tf = g\ast f$ with $g(x)= \exp\relax(-\abs{x}^2/2)$. We have 
    \[\abr{Tf,f} = \abr{\mathcal F(g\ast f),\mathcal Ff} = \abr{\mathcal Fg\cdot \mathcal Ff,\mathcal Ff} = \abr{g\cdot \mathcal Ff,\mathcal Ff} = \abr{g^{1/2}\cdot\mathcal Ff,g^{1/2}\cdot\mathcal Ff} = \norm{g^{1/2}\cdot\mathcal Ff}_2^2\geq 0.\]
    Let $Tf = g\ast f = 0$. Then $\mathcal F(g\ast f) = g\cdot\mathcal Ff = \mathcal F0 = 0$, so $\mathcal Ff = 0$. By injectivity of the Fourier transform, $f = 0$, so $T$ is injective. Suppose that $T$ is surjective so that $T$ has a right inverse $T_R^{-1}$. Then $g = \mathcal Fg = \mathcal FTT_R^{-1}g = g\cdot\mathcal FT_R^{-1}g$, so that $\mathcal FT_R^{-1}g = 1$. There is no operator $T_R^{-1}\colon L_2(\mathbb R^d)\to L_2(\mathbb R^d)$ that can be chosen to make this equation true since $1\not\in L_2(\mathbb R^d)$.
    \item[14.] Let $\psi\in \mathcal S(\mathbb R^d)$. Then $\abr{\varphi_\epsilon,\psi} = \int\psi(x)\epsilon^{-d}\varphi(x/\epsilon)\dd x = \int\psi(\epsilon x)\varphi(x)\dd x$, so by the dominated convergence theorem, $\lim_{\epsilon\to 0^+}\abr{\varphi_\epsilon, \psi} = (2\pi)^{-d/2}\psi(0)$, so $\varphi_\epsilon\to (2\pi)^{-d/2}\delta_0$ in the sense of tempered distributions.
    
    We have $\abr{\mathcal F\varphi_\epsilon,\psi} = \abr{\varphi_\epsilon,\mathcal F\psi} = \int\mathcal F\psi(x)\epsilon^{-d}\varphi(x/\epsilon)\dd x = \int\mathcal F\psi(\epsilon x)\varphi(x)\dd x$, so again by the dominated convergence theorem $\lim_{\epsilon\to 0^+}\abr{\mathcal F\varphi_\epsilon,\psi}= (2\pi)^{-d/2}\mathcal F\psi(0) = \int\psi(x)(2\pi)^{-d/2}\dd x$. Therefore $\mathcal F\varphi_\epsilon\to (2\pi)^{-d/2}$ in the sense of tempered distributions. 
    \item[15.] 
\end{enumerate}
\end{document}