\documentclass[11pt,leqno]{article}
\headheight=13.6pt

% packages
\usepackage[alphabetic]{amsrefs}
\usepackage{physics}
% margin spacing
\usepackage[top=1in, bottom=1in, left=0.5in, right=0.5in]{geometry}
\usepackage{hanging}
\usepackage{amsfonts, amsmath, amssymb, amsthm}
\usepackage{systeme}
\usepackage[none]{hyphenat}
\usepackage{fancyhdr}
\usepackage{graphicx}
\graphicspath{{./images/}}
\usepackage{float}
\usepackage{siunitx}
\usepackage{esint}
\usepackage{color}
\usepackage{enumitem}
\usepackage{mathrsfs}
\usepackage{hyperref}
\usepackage[noabbrev, capitalise]{cleveref}
\crefformat{equation}{equation~#2#1#3}
\crefformat{lemma}{\textrm{Lemma}~#2#1#3}

% theorems
\theoremstyle{plain}
\newtheorem{lem}{Lemma}
\newtheorem{lemma}[lem]{Lemma}
\newtheorem{thm}[lem]{Theorem}
\newtheorem{theorem}[lem]{Theorem}
\newtheorem{prop}[lem]{Proposition}
\newtheorem{proposition}[lem]{Proposition}
\newtheorem{cor}[lem]{Corollary}
\newtheorem{corollary}[lem]{Corollary}
\newtheorem{conj}[lem]{Conjecture}
\newtheorem{fact}[lem]{Fact}
\newtheorem{form}[lem]{Formula}

\theoremstyle{definition}
\newtheorem{defn}[lem]{Definition}
\newtheorem{definition/}[lem]{Definition}
\newenvironment{definition}
  {\renewcommand{\qedsymbol}{\textdagger}%
   \pushQED{\qed}\begin{definition/}}
  {\popQED\end{definition/}}
\newtheorem{example}[lem]{Example}
\newtheorem{remark}[lem]{Remark}
\newtheorem{exercise}[lem]{Exercise}
\newtheorem{notation}[lem]{Notation}

\numberwithin{equation}{section}
\numberwithin{lem}{section}

% header/footer formatting
\pagestyle{fancy}
\fancyhead{}
\fancyfoot{}
\fancyhead[L]{M 383D}
\fancyhead[C]{HW8}
\fancyhead[R]{Sai Sivakumar}
\fancyfoot[R]{\thepage}
\renewcommand{\headrulewidth}{1pt}

% paragraph indentation/spacing
\setlength{\parindent}{0cm}
\setlength{\parskip}{10pt}
\renewcommand{\baselinestretch}{1.25}

% extra commands defined here
\newcommand{\br}[1]{\left(#1\right)}
\newcommand{\sbr}[1]{\left[#1\right]}
\newcommand{\cbr}[1]{\left\{#1\right\}}
\newcommand{\eq}[1]{\overset{#1}{=}}

% bracket notation for inner product
\usepackage{mathtools}

\DeclarePairedDelimiterX{\abr}[1]{\langle}{\rangle}{#1}

\DeclareMathOperator{\Span}{span}
\DeclareMathOperator{\im}{im}
\DeclareMathOperator{\dist}{dist}
\DeclareMathOperator{\diam}{diam}
\DeclareMathOperator{\supp}{supp}
\newcommand{\res}[1]{\operatorname*{res}_{#1}}
\DeclareMathOperator{\id}{id}
\let\PV\relax
\DeclareMathOperator{\PV}{PV}
\DeclareMathOperator{\sgn}{sgn}

% smileys frownies
\usepackage{wasysym}
\newcommand{\smallhappy}{\smiley}
\newcommand{\happy}{\raisebox{-.14em}{\resizebox{1.2em}{!}{\smiley}}}
\newcommand{\smallsad}{\frownie}
\newcommand{\sad}{\raisebox{-.14em}{\resizebox{1.2em}{!}{\frownie}}}
\DeclareMathOperator{\mathhappy}{\!\happy\!}
\DeclareMathOperator{\smallmathhappy}{\!\smallhappy\!}
\DeclareMathOperator{\mathsad}{\!\sad\!}
\DeclareMathOperator{\smallmathsad}{\!\smallsad\!}

\let\norm\undefined % <-- "Undefine" \norm
\DeclarePairedDelimiter\norm{\lVert}{\rVert}

% set page count index to begin from 1
\setcounter{page}{1}

\begin{document}
\subsection*{8.8 Exercises (from \href{https://users.oden.utexas.edu/~arbogast/appMath08c.pdf}{online notes})}
\begin{enumerate}
    \item[1.] Let $A\in \mathbb R^{d\times d}$ be a matrix.
    
    Let $A$ be positive definite and let $\lambda$ be an eigenvalue of $A$. There exists a nonzero $\xi\in\mathbb R^d$ such that $A\xi = \lambda\xi$. Then $0 < \abr{\xi,A\xi} = \abr{\xi,\lambda\xi} = {\lambda}\norm{\xi}^2$ so that $\lambda > 0$ as needed. 
    
    If $A$ is symmetric and has positive eigenvalues, there exists an orthonormal basis $\cbr{f_i}$ of $\mathbb R^d$ such that $Af_i = \lambda_if_i$ where $\lambda_i>0$. For $\xi = \sum_i c_if_i\neq 0$, we have $\abr{\xi,A\xi} = \sum_i\lambda_ic_i^2\norm{f_i}^2 > 0$ as needed.
    \item[3.] The Sobolev embedding theorem simplies that if $d\geq 4$, then $u\in L^q(\Omega)$ for all finite $q\in [1,2d/(d-4)]$ and if $d<4$, then $u\in L^q(\Omega)$ for $q\geq 1$. H\"older's inequality yields $\norm{cu^2}_2^2\leq \norm{c}_{2p}^2\norm{u}_r^4$ for $p,r/4\geq 1$ and $p = r/(r-4)$. The quantity $\norm{u}_r^4$ is finite for $u\in L^q(\Omega)$ in the following cases (we take the intersection of the admissible values for $q$ and $r$ and then set $q = r$ in the new range): If $d>4$, the Sobolev embedding theorem gave $1\leq q \leq 2d/(d-4)$ and since $r\geq 4$, we have $4 \leq r\leq 2d/(d-4)$. This adds the additional restriction that $d\leq 8$, which is strange. In this case $p = r/(r-4)$ (which decreases in $r$) and $p\geq 1$, so $d/(8-d)\leq p \leq \infty$ for $4<d\leq 8$. In the case that $d = 4$, $1\leq q< \infty$, so we may choose $4\leq r <\infty$ as well. Then $1 < p <\infty$. If $d<4$, then $1\leq q\leq \infty$, so that $4\leq r\leq \infty$ and $1\leq p\leq \infty$. 
    
    It follows that for $1\leq d\leq 3$, it suffices to have $c\in L^2(\Omega)$. For $d = 4$, it suffices to have $c\in L^{2+\varepsilon}(\Omega)$ for arbitrarily small $\varepsilon>0$. For $4<d\leq 8$, it suffices to have $c\in L^{d/(8-d)}(\Omega)$ (for $d=8$, we obtain $c\in L^\infty(\Omega)$). For $d>8$, the above method does not guarantee a way to ensure that $cu^2$ is square-integrable.
    \item[4.] \begin{enumerate}
      \item It is clear that $H$ is a subspace of $H^1(\Omega)$ so it remains to show closedness. Let $\cbr{u_j}\subset H$ converge to $u\in H^1(\Omega)$. Then $\norm{u|_V}_{L^2(V)} = \norm{u_j|_V-u|_V}_{L^2(V)}\leq \norm{u_j-u}_{L^2{\Omega}}\leq \norm{u_j-u}_{H^1{\Omega}}$; by taking limits we find that $\norm{u|_V}_{L^2(V)} = 0$ so that $u\in H$. It follows that $H$ is closed in $H^1(\Omega)$, so $H$ is a Hilbert space.
      \item Suppose that there is a Cauchy sequence $\cbr{u_j}\subset H$ such that $\norm{u_j}_H = 1$ and $\norm{u_j}_{L^2(\Omega)}> j\norm{D^\alpha u_j}_{L^2(\Omega)}$ for any $\abs{\alpha} = 1$. In particular observe that $\norm{D^\alpha u_j}_{L^2(\Omega)}\leq \norm{u_j}_{L^2(\Omega)}/j\leq \norm{u_j}_H/j = 1/j$ for $\abs{\alpha} = 1$. Compactly embed $H^1(\Omega)$ into $L^2(\Omega)$ to obtain a convergent subsequence $u_{j_k}$ in $L^2(\Omega)$, converging to some $u\in L^2(\Omega)$; so in $L^2(\Omega)$, the sequence $\cbr{u_j}$ converges to $u$. The distributional derivative of $u$ is $0$ since for any test function $\varphi$ and $\abs{\alpha} = 1$, $|\abr{D^\alpha u,\varphi}| = \big|-\int_\Omega u D^\alpha \varphi\big| = \lim_{j\to\infty}-\int_\Omega u_jD^\alpha \varphi \leq \lim_{j \to\infty}\int_\Omega \abs{D^\alpha u_j}\abs{\varphi} \leq \lim_{j\to\infty}\norm{D^\alpha u_j}_{L^2(\Omega)}\norm{\varphi}_{L^2(\Omega)} = 0$. Since the $L^2(\Omega)$-norms of the derivatives of $u_j$ tend to zero, the $L^2(\Omega)$ norms of $u_j$ tend to $1$ so that $u$ will have nonzero $L^2(\Omega)$ norm. Moreover, since $\cbr{u_j}$ is Cauchy in $H$, we must have that $u$ also belongs to $H$ so that $u|_V$ is zero. Since $u$ is nonzero, it must have nonzero derivative since it is zero on $V$ and nonzero somewhere outside $V$. This is a contradiction.
    \end{enumerate}
\end{enumerate}
\end{document}