\documentclass[11pt,leqno]{article}
\headheight=13.6pt

% packages
\usepackage[alphabetic]{amsrefs}
\usepackage{physics}
% margin spacing
\usepackage[top=1in, bottom=1in, left=0.5in, right=0.5in]{geometry}
\usepackage{hanging}
\usepackage{amsfonts, amsmath, amssymb, amsthm}
\usepackage{systeme}
\usepackage[none]{hyphenat}
\usepackage{fancyhdr}
\usepackage{graphicx}
\graphicspath{{./images/}}
\usepackage{float}
\usepackage{siunitx}
\usepackage{esint}
\usepackage{color}
\usepackage{enumitem}
\usepackage{mathrsfs}
\usepackage{hyperref}
\usepackage[noabbrev, capitalise]{cleveref}
\crefformat{equation}{equation~#2#1#3}
\crefformat{lemma}{\textrm{Lemma}~#2#1#3}

% theorems
\theoremstyle{plain}
\newtheorem{lem}{Lemma}
\newtheorem{lemma}[lem]{Lemma}
\newtheorem{thm}[lem]{Theorem}
\newtheorem{theorem}[lem]{Theorem}
\newtheorem{prop}[lem]{Proposition}
\newtheorem{proposition}[lem]{Proposition}
\newtheorem{cor}[lem]{Corollary}
\newtheorem{corollary}[lem]{Corollary}
\newtheorem{conj}[lem]{Conjecture}
\newtheorem{fact}[lem]{Fact}
\newtheorem{form}[lem]{Formula}

\theoremstyle{definition}
\newtheorem{defn}[lem]{Definition}
\newtheorem{definition/}[lem]{Definition}
\newenvironment{definition}
  {\renewcommand{\qedsymbol}{\textdagger}%
   \pushQED{\qed}\begin{definition/}}
  {\popQED\end{definition/}}
\newtheorem{example}[lem]{Example}
\newtheorem{remark}[lem]{Remark}
\newtheorem{exercise}[lem]{Exercise}
\newtheorem{notation}[lem]{Notation}

\numberwithin{equation}{section}
\numberwithin{lem}{section}

% header/footer formatting
\pagestyle{fancy}
\fancyhead{}
\fancyfoot{}
\fancyhead[L]{M 383D}
\fancyhead[C]{HW1}
\fancyhead[R]{Sai Sivakumar}
\fancyfoot[R]{\thepage}
\renewcommand{\headrulewidth}{1pt}

% paragraph indentation/spacing
\setlength{\parindent}{0cm}
\setlength{\parskip}{10pt}
\renewcommand{\baselinestretch}{1.25}

% extra commands defined here
\newcommand{\br}[1]{\left(#1\right)}
\newcommand{\sbr}[1]{\left[#1\right]}
\newcommand{\cbr}[1]{\left\{#1\right\}}
\newcommand{\eq}[1]{\overset{#1}{=}}
\let\O\relax
\newcommand{\O}{\mathrm O}

% bracket notation for inner product
\usepackage{mathtools}

\DeclarePairedDelimiterX{\abr}[1]{\langle}{\rangle}{#1}

\DeclareMathOperator{\Span}{span}
\DeclareMathOperator{\im}{im}
\DeclareMathOperator{\dist}{dist}
\DeclareMathOperator{\diam}{diam}
\DeclareMathOperator{\supp}{supp}
\DeclareMathOperator{\EV}{ev}
\DeclareMathOperator{\co}{co}
\newcommand{\res}[1]{\operatorname*{res}_{#1}}
\DeclareMathOperator{\id}{id}
\let\PV\relax
\DeclareMathOperator{\PV}{PV}

% smileys frownies
\usepackage{wasysym}
\newcommand{\smallhappy}{\smiley}
\newcommand{\happy}{\raisebox{-.14em}{\resizebox{1.2em}{!}{\smiley}}}
\newcommand{\smallsad}{\frownie}
\newcommand{\sad}{\raisebox{-.14em}{\resizebox{1.2em}{!}{\frownie}}}
\DeclareMathOperator{\mathhappy}{\!\happy\!}
\DeclareMathOperator{\smallmathhappy}{\!\smallhappy\!}
\DeclareMathOperator{\mathsad}{\!\sad\!}
\DeclareMathOperator{\smallmathsad}{\!\smallsad\!}

% lol
\newcommand{\gq}[1]{\overset{#1}{\geq}}

% set page count index to begin from 1
\setcounter{page}{1}

\begin{document}
\subsection*{6.6 Exercises (from \href{https://users.oden.utexas.edu/~arbogast/appMath08c.pdf}{online notes})}
I worked with or spoke to Michael Han, Sara Ansari, Jake Wellington, Iris Jiang, and John Teague.
\begin{enumerate}
    \item[1.] Let $f(x) = \exp(-\abs{x})$ for $x\in\mathbb R$. Then \begin{align*}
      \mathcal Ff(\xi) &= (2\pi)^{-1/2}\int_{\mathbb R} \exp(-\abs{x} - ix\xi)\dd x\\
      &= (2\pi)^{-1/2}\int_{x>0} \exp((-i\xi-1)x)\dd x + (2\pi)^{-1/2}\int_{x<0} \exp((-i\xi+1)x)\dd x\\
      &= (2\pi)^{-1/2}(-i/(-i + \xi)) + (2\pi)^{-1/2}(i/(i + \xi))\\
      &= 2(2\pi)^{-1/2}/(\xi^2+1).
    \end{align*}
    \item[2.] Let $f(x) = \exp(-ax^2)$ for $x\in\mathbb R$ and $a>0$. Then \begin{align*}
      \mathcal Ff(\xi) &= (2\pi)^{-1/2}\int_{\mathbb R} \exp(-ax^2-ix\xi)\dd x\\
      &= (2\pi)^{-1/2}\exp(-\xi^2/4a)\int_{\mathbb R} \exp(- a (x +i\xi/2a)^2)\dd x\\
      &\eq{\smallmathhappy} (2\pi)^{-1/2}\exp(-\xi^2/4a)\int_{\mathbb R} \exp(- ax^2)\dd x\\
      &= (2a)^{-1/2}\exp(-\xi^2/4a),
    \end{align*}
    where equality $\mathhappy$ holds due to Cauchy's theorem: The integral of $\exp(- az^2)$ along the vertical line segments $[\pm R,\pm R + i\xi/2a]$ may be made arbitrarily small as $R$ grows unboundedly.
    \item[4.] That $f(x) = g(\abs{x})$ for some $g$ is equivalent to $f$ being $\O(n)$-invariant; that is, $f(Ax) = f(x)$ for any $A\in \O(n)$. For any $A\in \O(n)$, \begin{align*}
      \mathcal Ff(A\xi) &= (2\pi)^{d/2}\int_{\mathbb R^d} f(x)\exp(-i\abr{x,A\xi})\dd x \\
      &= (2\pi)^{d/2}\int_{\mathbb R^d} f(AA^{-1}x)\exp(-i\abr{A^{-1}x,\xi})\dd x \\
      &\eq{\smallmathsad} (2\pi)^{d/2}\int_{\mathbb R^d} f(Ax)\exp(-i\abr{x,\xi})\dd x\\
      &= (2\pi)^{d/2}\int_{\mathbb R^d} f(x)\exp(-i\abr{x,\xi})\dd x = \mathcal Ff(\xi), 
    \end{align*}
    where equality $\mathsad$ holds by the change of variables $u = A^{-1}x$ ($\det A = \pm 1$). Hence $\mathcal Ff(\xi) = h(\abs{\xi})$ for some $h$. 

    The functions $h$ and $g$ are related by the following calculation: \begin{align*}
      h(t) &= \mathcal Ff(te_1)\\
      &= (2\pi)^{-d/2}\int_{\mathbb R^d}g(\abs{x})\exp(-i\abr{x,te_1})\dd x\\
      &= (2\pi)^{-d/2}\int_{\mathbb R^d}g(\abs{x})\exp(-itx_1)\dd x \\
      &= (2\pi)^{-d/2}\int_0^\infty r^{d-1}g(r)\int_0^{2\pi}\int_0^\pi\!\!\!\!\cdots\!\int_0^\pi\exp(-itr\cos(\theta_1))\sin^{d-2}(\theta_1)\sin^{d-3}(\theta_2)\cdots\sin(\theta_{d-2})\dd\theta_1\cdots\dd\theta_{d-1}\dd r\\
      &= (2\pi)^{-d/2}\sigma_{n-2}\int_0^\infty r^{d-1}g(r)\int_0^\pi\exp(-itr\cos(\theta_1))\sin^{d-2}(\theta_1)\dd\theta_1\dd r,
      % &= (2\pi)^{d/2}(2\pi)\pi^{d-3}\int_0^\infty r^{d-1}g(r)\int_0^\pi\exp(-itr\cos(\theta_1))\dd\theta_1\dd r\\
      % &\eq{\eighthnote} (2/\pi)^{1-d/2}\int_0^\infty r^{d-1}g(r)J_0(tr)\dd r,
    \end{align*}
    where $\sigma_{n-1}$ is the surface area of the $(n-1)$-sphere in $\mathbb R^n$. (If $d$ is odd, we should be able to evaluate the innermost integral...)
    % where $J_n(z) = i^{-n}\pi^{-1}\int_0^\pi\exp(iz\cos(\theta))\dd\theta$ is the Bessel function of the first kind (\href{https://dlmf.nist.gov/10.9#E2}{dlmf.nist.gov}), which yields equality \resizebox{0.8em}{!}{\eighthnote}.
    \item[11.] The Fourier transform $\mathcal F\colon L_1(\mathbb R^d)\to C_v(\mathbb R^d)$ is injective. Let $\mathcal Ff = 0$ and let $\cbr{f_n}\subseteq \mathcal S$ be a sequence of Schwartz class functions converging to $f$ (we may take $f_n = g_n\ast \chi_{B_n(0)}f\in C_c^\infty(\mathbb R^d)$ where $g_n$ are suitably chosen Gaussians such that $\norm{f_n}_1 = \norm{\chi_{B_n(0)}f}_1$, or something similar). Then by continuity of the Fourier transform, the sequence $\cbr{\mathcal Ff_n}\subseteq \mathcal S$ converges to zero. But $\cbr{\mathcal F\mathcal Ff_n} = \cbr{Rf_n}$ must also converge to zero, so $\cbr{f_n}$ converges to zero. 
    
    Now assume that $\mathcal F$ is surjective. Then by the open mapping theorem, $\mathcal F^{-1}\colon C_v(\mathbb R^d)\colon L_1(\mathbb R^d)$ is continuous. But for $R_n(0) = [-n,n]^d$, $\norm{\mathcal F^{-1}(\chi_{R_1(0)}\ast \chi_{R_n(0)})}_1 = \norm{(2/\pi)^{1/2}\sin(x)\sin(nx)/x^2}_1^d$. We have \begin{align*}
      2/\pi\int_{\mathbb R}\abs{\sin(x)\sin(nx)/x^2}\dd x &= 2n/\pi\int_0^n\abs{\sin(x/n)\sin(x)/x^2}\dd x\\
      &\gq{\quarternote} 10/6\pi \int_0^n \abs{\sin(x)/x}\dd x,
    \end{align*}
    where $\quarternote$ holds due to the estimate $\sin(x)\geq x-x^3/6\geq x-x/6 = 5x/6$ on $[0,1]$. But $\int_0^n \abs{\sin(x)/x}\dd x$ grows unboundedly in $n$ whereas $\chi_{R_1(0)}\ast \chi_{R_n(0)}$ is uniformly bounded above for all $n$, which is a contradiction. Therefore $\mathcal F\colon L_1(\mathbb R^d)\to C_v(\mathbb R^d)$ is not surjective. 

    However, $\im \mathcal F$ is dense in $C_v(\mathbb R^d)$ since $C_c^\infty(\mathbb R^d)$ is dense in $C_v(\mathbb R^d)$: $C_c^\infty(\mathbb R^d)$ is dense in $C_c(\mathbb R^d)$, which is dense in $C_v(\mathbb R^d)$ (truncate $f\in C_v(\mathbb R^d)$ to balls of radius $n$ and form continuous extensions from the function on these balls to $0$, via linear functions for example, then convolute with suitably chosen Gaussians to smoothen these functions out on a slightly larger compact set). Since $C_c^\infty\subseteq \mathcal S$, it follows that $C_c^\infty\subseteq \im \mathcal F|_{\mathcal S \subseteq L_1(\mathbb R^n)}$ and hence $\im \mathcal F$ is dense in $C_v(\mathbb R^d)$.
    \item[15.] On the next homework set.
\end{enumerate}
\end{document}