\documentclass[11pt,leqno]{article}
\headheight=13.6pt

% packages
\usepackage[alphabetic]{amsrefs}
\usepackage{physics}
% margin spacing
\usepackage[top=1in, bottom=1in, left=0.5in, right=0.5in]{geometry}
\usepackage{hanging}
\usepackage{amsfonts, amsmath, amssymb, amsthm}
\usepackage{systeme}
\usepackage[none]{hyphenat}
\usepackage{fancyhdr}
\usepackage{graphicx}
\graphicspath{{./images/}}
\usepackage{float}
\usepackage{siunitx}
\usepackage{esint}
\usepackage{color}
\usepackage{enumitem}
\usepackage{mathrsfs}
\usepackage{hyperref}
\usepackage[noabbrev, capitalise]{cleveref}
\crefformat{equation}{equation~#2#1#3}
\crefformat{lemma}{\textrm{Lemma}~#2#1#3}

% theorems
\theoremstyle{plain}
\newtheorem{lem}{Lemma}
\newtheorem{lemma}[lem]{Lemma}
\newtheorem{thm}[lem]{Theorem}
\newtheorem{theorem}[lem]{Theorem}
\newtheorem{prop}[lem]{Proposition}
\newtheorem{proposition}[lem]{Proposition}
\newtheorem{cor}[lem]{Corollary}
\newtheorem{corollary}[lem]{Corollary}
\newtheorem{conj}[lem]{Conjecture}
\newtheorem{fact}[lem]{Fact}
\newtheorem{form}[lem]{Formula}

\theoremstyle{definition}
\newtheorem{defn}[lem]{Definition}
\newtheorem{definition/}[lem]{Definition}
\newenvironment{definition}
  {\renewcommand{\qedsymbol}{\textdagger}%
   \pushQED{\qed}\begin{definition/}}
  {\popQED\end{definition/}}
\newtheorem{example}[lem]{Example}
\newtheorem{remark}[lem]{Remark}
\newtheorem{exercise}[lem]{Exercise}
\newtheorem{notation}[lem]{Notation}

\numberwithin{equation}{section}
\numberwithin{lem}{section}

% header/footer formatting
\pagestyle{fancy}
\fancyhead{}
\fancyfoot{}
\fancyhead[L]{M 383D}
\fancyhead[C]{HW5}
\fancyhead[R]{Sai Sivakumar}
\fancyfoot[R]{\thepage}
\renewcommand{\headrulewidth}{1pt}

% paragraph indentation/spacing
\setlength{\parindent}{0cm}
\setlength{\parskip}{10pt}
\renewcommand{\baselinestretch}{1.25}

% extra commands defined here
\newcommand{\br}[1]{\left(#1\right)}
\newcommand{\sbr}[1]{\left[#1\right]}
\newcommand{\cbr}[1]{\left\{#1\right\}}
\newcommand{\eq}[1]{\overset{#1}{=}}
\let\O\relax
\newcommand{\O}{\mathrm O}

% bracket notation for inner product
\usepackage{mathtools}

\DeclarePairedDelimiterX{\abr}[1]{\langle}{\rangle}{#1}

\DeclareMathOperator{\Span}{span}
\DeclareMathOperator{\im}{im}
\DeclareMathOperator{\dist}{dist}
\DeclareMathOperator{\diam}{diam}
\DeclareMathOperator{\supp}{supp}
\DeclareMathOperator{\EV}{ev}
\DeclareMathOperator{\co}{co}
\newcommand{\res}[1]{\operatorname*{res}_{#1}}
\DeclareMathOperator{\id}{id}
\let\PV\relax
\DeclareMathOperator{\PV}{PV}
\DeclareMathOperator{\sgn}{sgn}

% smileys frownies
\usepackage{wasysym}
\newcommand{\smallhappy}{\smiley}
\newcommand{\happy}{\raisebox{-.14em}{\resizebox{1.2em}{!}{\smiley}}}
\newcommand{\smallsad}{\frownie}
\newcommand{\sad}{\raisebox{-.14em}{\resizebox{1.2em}{!}{\frownie}}}
\DeclareMathOperator{\mathhappy}{\!\happy\!}
\DeclareMathOperator{\smallmathhappy}{\!\smallhappy\!}
\DeclareMathOperator{\mathsad}{\!\sad\!}
\DeclareMathOperator{\smallmathsad}{\!\smallsad\!}

\let\norm\undefined % <-- "Undefine" \norm
\DeclarePairedDelimiter\norm{\lVert}{\rVert}

% lol
\newcommand{\lqq}[1]{\overset{#1}{\leq}}

% set page count index to begin from 1
\setcounter{page}{1}

\begin{document}
\subsection*{7.8 Exercises (from \href{https://users.oden.utexas.edu/~arbogast/appMath08c.pdf}{online notes})}
\begin{enumerate}
    \item[1.] Let $f\in H^k(\mathbb R^d)$. Then 
    \begin{multline*}
        \norm{f}_{H^1}^2 = \sum_{j=0}^m\sum_{\abs{\alpha} = j}\norm{D^\alpha f}_2^2 = \sum_{j=0}^m\sum_{\abs{\alpha} = j}\norm{\mathcal FD^\alpha f}_2^2 = \sum_{j=0}^m\sum_{\abs{\alpha} = j}\norm{(i\xi)^\alpha \mathcal F f}_2^2 \\ = \int_{\mathbb R^d}\Big[\sum_{j=0}^m\sum_{\abs{\alpha} = j}(\xi^\alpha)^2\Big]\abs{\mathcal Ff}^2 = \int_{\mathbb R^d}\Big[\sum_{j=0}^m\abs{\xi}^{2j}\Big]\abs{\mathcal Ff}^2
    \end{multline*} and specializing to the case $m = 1$ yields $\norm{f}_{H^1}^2 = \int_{\mathbb R^d}[1+\abs{\xi}^{2}]\abs{\mathcal Ff}^2$.
    \item[3.] We define $\delta_0$ on $H^1(\mathbb R)$ by the unique extension of $\delta_0$ on the dense subspace $C^\infty(\mathbb R)\cap H^1(\mathbb R)$ to $H^1(\mathbb R)$. For $f\in C^\infty(\mathbb R)\cap H^1(\mathbb R)$ we have the estimate $\abs{f(0)-f(y)} = \big|\int_y^0Df\big|\leq \int_y^0\abs{Df}\leq \abs{y-0}^{1/2}\norm{Df}_2$ (so by density, $H^1(\mathbb R)$ is H\"older continuous with exponent $1/2$). Then $\abs{f(0)} - \abs{f(y)}\leq \abs{y}^{1/2}\norm{Df}_2$ so that $\abs{f(0)} = \int_0^1\abs{f(0)}\leq \int_0^1(\abs{y}^{1/2}\norm{Df}_2 + \abs{f(y)})\leq \norm{Df}_2 + \norm{f}_2$. So if $\cbr{f_n}\subseteq C^\infty(\mathbb R)\cap H^1(\mathbb R)$ converges to zero, then $\cbr{\delta_0f_n}\subseteq$ converges to zero since $\abs{\delta_0f_n} = \abs{f_n(0)}\leq \norm{Df_n}_2 + \norm{f_n}_2$ can be made arbitrarily small.
    
    Consider the sequence of functions 
    \item[7.] 
    \item[8.] 
\end{enumerate}
\end{document}