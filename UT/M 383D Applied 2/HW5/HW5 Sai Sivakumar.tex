\documentclass[11pt,leqno]{article}
\headheight=13.6pt

% packages
\usepackage[alphabetic]{amsrefs}
\usepackage{physics}
% margin spacing
\usepackage[top=1in, bottom=1in, left=0.5in, right=0.5in]{geometry}
\usepackage{hanging}
\usepackage{amsfonts, amsmath, amssymb, amsthm}
\usepackage{systeme}
\usepackage[none]{hyphenat}
\usepackage{fancyhdr}
\usepackage{graphicx}
\graphicspath{{./images/}}
\usepackage{float}
\usepackage{siunitx}
\usepackage{esint}
\usepackage{color}
\usepackage{enumitem}
\usepackage{mathrsfs}
\usepackage{hyperref}
\usepackage[noabbrev, capitalise]{cleveref}
\crefformat{equation}{equation~#2#1#3}
\crefformat{lemma}{\textrm{Lemma}~#2#1#3}

% theorems
\theoremstyle{plain}
\newtheorem{lem}{Lemma}
\newtheorem{lemma}[lem]{Lemma}
\newtheorem{thm}[lem]{Theorem}
\newtheorem{theorem}[lem]{Theorem}
\newtheorem{prop}[lem]{Proposition}
\newtheorem{proposition}[lem]{Proposition}
\newtheorem{cor}[lem]{Corollary}
\newtheorem{corollary}[lem]{Corollary}
\newtheorem{conj}[lem]{Conjecture}
\newtheorem{fact}[lem]{Fact}
\newtheorem{form}[lem]{Formula}

\theoremstyle{definition}
\newtheorem{defn}[lem]{Definition}
\newtheorem{definition/}[lem]{Definition}
\newenvironment{definition}
  {\renewcommand{\qedsymbol}{\textdagger}%
   \pushQED{\qed}\begin{definition/}}
  {\popQED\end{definition/}}
\newtheorem{example}[lem]{Example}
\newtheorem{remark}[lem]{Remark}
\newtheorem{exercise}[lem]{Exercise}
\newtheorem{notation}[lem]{Notation}

\numberwithin{equation}{section}
\numberwithin{lem}{section}

% header/footer formatting
\pagestyle{fancy}
\fancyhead{}
\fancyfoot{}
\fancyhead[L]{M 383D}
\fancyhead[C]{HW5}
\fancyhead[R]{Sai Sivakumar}
\fancyfoot[R]{\thepage}
\renewcommand{\headrulewidth}{1pt}

% paragraph indentation/spacing
\setlength{\parindent}{0cm}
\setlength{\parskip}{10pt}
\renewcommand{\baselinestretch}{1.25}

% extra commands defined here
\newcommand{\br}[1]{\left(#1\right)}
\newcommand{\sbr}[1]{\left[#1\right]}
\newcommand{\cbr}[1]{\left\{#1\right\}}
\newcommand{\eq}[1]{\overset{#1}{=}}
\let\O\relax
\newcommand{\O}{\mathrm O}

% bracket notation for inner product
\usepackage{mathtools}

\DeclarePairedDelimiterX{\abr}[1]{\langle}{\rangle}{#1}

\DeclareMathOperator{\Span}{span}
\DeclareMathOperator{\im}{im}
\DeclareMathOperator{\dist}{dist}
\DeclareMathOperator{\diam}{diam}
\DeclareMathOperator{\supp}{supp}
\DeclareMathOperator{\EV}{ev}
\DeclareMathOperator{\co}{co}
\newcommand{\res}[1]{\operatorname*{res}_{#1}}
\DeclareMathOperator{\id}{id}
\let\PV\relax
\DeclareMathOperator{\PV}{PV}
\DeclareMathOperator{\sgn}{sgn}

% smileys frownies
\usepackage{wasysym}
\newcommand{\smallhappy}{\smiley}
\newcommand{\happy}{\raisebox{-.14em}{\resizebox{1.2em}{!}{\smiley}}}
\newcommand{\smallsad}{\frownie}
\newcommand{\sad}{\raisebox{-.14em}{\resizebox{1.2em}{!}{\frownie}}}
\DeclareMathOperator{\mathhappy}{\!\happy\!}
\DeclareMathOperator{\smallmathhappy}{\!\smallhappy\!}
\DeclareMathOperator{\mathsad}{\!\sad\!}
\DeclareMathOperator{\smallmathsad}{\!\smallsad\!}

\let\norm\undefined % <-- "Undefine" \norm
\DeclarePairedDelimiter\norm{\lVert}{\rVert}

% lol
\newcommand{\lqq}[1]{\overset{#1}{\leq}}

% set page count index to begin from 1
\setcounter{page}{1}

\begin{document}
\subsection*{7.8 Exercises (from \href{https://users.oden.utexas.edu/~arbogast/appMath08c.pdf}{online notes})}
\begin{enumerate}
    \item[1.] Let $f\in H^k(\mathbb R^d)$. Then 
    \begin{multline*}
        \norm{f}_{H^1}^2 = \sum_{j=0}^m\sum_{\abs{\alpha} = j}\norm{D^\alpha f}_2^2 = \sum_{j=0}^m\sum_{\abs{\alpha} = j}\norm{\mathcal FD^\alpha f}_2^2 = \sum_{j=0}^m\sum_{\abs{\alpha} = j}\norm{(i\xi)^\alpha \mathcal F f}_2^2 \\ = \int_{\mathbb R^d}\Big[\sum_{j=0}^m\sum_{\abs{\alpha} = j}(\xi^\alpha)^2\Big]\abs{\mathcal Ff}^2 = \int_{\mathbb R^d}\Big[\sum_{j=0}^m\abs{\xi}^{2j}\Big]\abs{\mathcal Ff}^2
    \end{multline*} and specializing to the case $m = 1$ yields $\norm{f}_{H^1}^2 = \int_{\mathbb R^d}[1+\abs{\xi}^{2}]\abs{\mathcal Ff}^2$.
    \item[3.] We define $\delta_0$ on $H^1(\mathbb R)$ by the unique extension of $\delta_0$ on the dense subspace $C^\infty(\mathbb R)\cap H^1(\mathbb R)$ to $H^1(\mathbb R)$. For $f\in C^\infty(\mathbb R)\cap H^1(\mathbb R)$ we have the estimate $\abs{f(x)-f(y)} = \big|\int_y^xDf\big|\leq \int_y^x\abs{Df}\leq \abs{y-x}^{1/2}\norm{Df}_2$ (so by density, $H^1(\mathbb R)$ is H\"older continuous with exponent $1/2$). Then $\abs{f(0)} - \abs{f(y)}\leq \abs{y}^{1/2}\norm{Df}_2$ so that $\abs{f(0)} = \int_0^1\abs{f(0)}\leq \int_0^1(\abs{y}^{1/2}\norm{Df}_2 + \abs{f(y)})\leq \norm{Df}_2 + \norm{f}_2$. So if $\cbr{f_n}\subseteq C^\infty(\mathbb R)\cap H^1(\mathbb R)$ converges to zero, then $\cbr{\delta_0f_n}$ converges to zero since $\abs{\delta_0f_n} = \abs{f_n(0)}\leq \norm{Df_n}_2 + \norm{f_n}_2$ can be made arbitrarily small.
    
    % Suppose that $\delta_0$ is defined on the dense subspace $C^\infty_0(\mathbb R^d)\cap H^1(\mathbb R^d)$ by $f\mapsto f(0)$, and that $\delta_0$ is continuous so that it uniquely extends to a functional on $H^1(\mathbb R^d)$. Then there is a sequence of 
    Suppose that $\delta_0$ is defined on $H^1(\mathbb R^d)$ for $d\geq 2$.
    
    Consider the function $f$ in problem 8 part (c) specialized to the case $p = d = 2$ and $\Omega = \mathbb R^2$. Since it is unbounded around the origin, we may consider a sequence of translates $f_n = f(\cdot - c_n)$ of $f$ where $c_n$ tends to zero, so that $\cbr{\delta_0f_n}$ could not converge. Thus $\delta_0$ does not belong to $(H^1(\mathbb R^2))^\ast$.

    For $d\geq 3$, consider the function defined by $f(x) = \chi_{\abs{x}\leq 1}(x)(1-\abs{x}^2)$ and the sequence of functions defined by $f_n(x) = f(nx)$. Then $\norm{f_n}_2^2 = \int_{\abs{x}\leq 1/n}(1-n^2\abs{x}^2)^2$ is proportional to $1/n^d$ by polar coordinates, and $\norm{D_if_n}_2^2 = \int_{\abs{x}\leq 1/n}4n^4x_i^2$ is proportional to $1/n^{d-2}$ by polar coordinates. Therefore $\norm{f_n}_{H^1}$ tends to zero as $n$ grows unboundedly, but $\delta_0(f_n) = 1$ for all $n$.
    \item[7.] \begin{enumerate}
      \item We have \begin{align*}
        \abr{u,\tau_y\phi} - \abr{u,\phi} &= \abr{u,\tau_y\phi - \phi} \\
        &= \abr{u(x),\textstyle\int_0^1\abr{(\nabla\phi)(x-ty),-y}\dd t}\\
        &= \abr{u(x),\abr{\textstyle\int_0^1(\nabla\phi)(x-ty)\dd t,-y}}\\
        &= \abr{\abr{-y,u(x)},\nabla\textstyle\int_0^1\phi(x-ty)\dd t}\\
        &= \abr{\abr{y,\nabla u},\textstyle\int_0^1\tau_{tx}\phi\dd t}\\
        &= \textstyle\int_0^1\abr{\abr{y,\nabla u},\tau_{ty}\phi}\dd t
      \end{align*} as needed.
      \item In part (a), choose $u = \Lambda_f$ (integration against $f$) for $f\in W^{1,1}_{\mathrm{loc}}$ to obtain the equality of distributions $\tau_{-y}f-f = \int_0^1\tau_{-ty}\abr{y,\nabla f}\dd t$ (by Fubini); that is, $f(x+y)-f(x) = \int_0^1 \abr{y,\nabla f(x+ty)}\dd t$. By the Lebesgue lemma, this equality must hold for almost every $x$ as $L^1_{\mathrm{loc}}(\mathbb R^d)$ functions.
      \item Observe that $W^{1,\infty}_{\mathrm{loc}}$ is contained in $W^{1,1}_{\mathrm{loc}}$. For $R>0$ and any $x,x+y\in B_R(0)$ (i.e. any $x\in B_R(0)$ and $y\in B_{R-\abs{x}}(0)$), by part (b) we have $\abs{f(x+y) - f(x)} = \big|\int_0^1\abr{y,\nabla f(x+ty)}\dd t\big|\leq \int_0^1\abs{y}\abs{\nabla f(x+ty)}\dd t\leq \abs{y}\norm{\nabla f}_{L^{\infty}(B_R(0))}$. Hence $f$ is locally Lipschitz.
    \end{enumerate}
    \item[8.] \begin{enumerate}
      \item We require $r$ to be negative to have a chance at a counterexample. For $f(x) = \abs{x}^{r}$ to define a function in $W^{1,p}(\Omega)$, the singularity at zero for $f$ and $Df$ must be integrable; that is, we require $-pr<d$ and $-(pr-p)<d$ so that $r>-d/p$ and $r>-d/p+1$ via polar coordinates. For $f$ to not belong to $L^q(\Omega)$ we require $-qr\geq d$; that is, $r\leq -d/q$. Since $q > dp/(d-p)$, $q = dp/(d-p) + \varepsilon$ for some fixed $\varepsilon>0$. Then $r\leq -d/[dp/(d-p) + \varepsilon]= -d(d-p)/(dp+\varepsilon(d-p))$. It remains to show that there exists $r\in (-d/p+1, -d/[dp/(d-p) + \varepsilon]]$. Indeed, $-d/[dp/(d-p) + \varepsilon] + d/p-1 = \varepsilon(d-p)^2/(p(dp+\varepsilon(d-p)))>0$ since $1\leq p< d$. So choose $r = -d/p+1 + \varepsilon(d-p)^2/(2p(dp+\varepsilon(d-p))) = -(d-p)/p + \varepsilon(d-p)^2/(2p(dp+\varepsilon(d-p))) = 1/2-d/2p-d/2q$.
      \item By translation, let $0$ belong to $\Omega$ without loss of generality. The function defined by $\abs{x}^r$ for $r = 1/2-d/2p-d/2q$ in part (a) belongs to $W^{1,p}(\Omega)$ but is unbounded so there is no way to embed $W^{1,p}(\Omega)$ into $C^0_B(\Omega)$. This implies that there is no hope for Dirac masses to belong to the Sobolev spaces $W^{-1,p}(\Omega)$ for $1\leq p < d$. 
      \item Without loss of generality translate so that $\Omega$ contains the origin. Let $\eta$ be a bump function which is the identity on $B_R(0)\subset \Omega$, supported on $B_{R+\varepsilon}(0)$, for some suitable $R,\varepsilon>0$. Then the function defined by $f(x) = \eta(x)\log(\log(4R/\abs{x}))$ belongs to $W^{1,p}(\Omega)$: It suffices to estimate integrals in $B_R(0)$. We have $\norm{f}_p^p$ is proportional to $\int_0^R \log(\log(4R/r))^d r^{d-1}\dd r = 4R\int_{\log(4)}^\infty \log(u)^d\exp(-du)\dd u < \infty$ since exponential decay dominates logarithmic growth, and $\norm{D_if}$ is proportional to $\int_0^R1/(r\log(4R/r)^d)\dd r = \int_{\log(4)}^\infty1/u^d\dd u <\infty$ since $d>1$. Therefore there is no embedding of $W^{1,p}(\Omega)$ in $L^\infty(\Omega)$.
      \item The function defined by $u(x) = \abs{x}$ on $(-1,1)\subseteq \mathbb R$ is bounded and its derivative $\sgn(x)$ is also bounded on $(-1,1)$. Suppose that there is a sequence $\cbr{u_n}\subseteq C^\infty\cap W^{1,\infty}$ converging to $u$. Then $u_n$ is a Cauchy sequence, and in particular we may view $u_n$ as a Cauchy sequence in $C^1_B(-1,1)$, which converges by completeness. But the function $u$ has a discontinuous first derivative at $0$, so $u_n$ could not converge in $C^1_B(-1,1)$. In general, translate $\Omega\subseteq \mathbb R^d$ so that it contains the origin and consider the function defined by $u(x) = \abs{x}$ supported in an open ball around zero. Assume that there is a sequence $\cbr{u_n}$ from $C^{\infty}(\Omega)\cap W^{1,\infty}(\Omega)$ converging to $u$. Then by restricting to to the subspace $\Span\cbr{e_1}\subseteq \mathbb R^d$ so that $\cbr{u_n}$ converges in the the $W^{1,\infty}(\mathbb R)$ norm in a small neighborhood around the origin to $u$, we arrive at the same contradiction as above.
    \end{enumerate}
\end{enumerate}
\end{document}