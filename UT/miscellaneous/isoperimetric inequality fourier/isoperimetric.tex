\documentclass[11pt,leqno]{article}
\headheight=13.6pt

% packages
\usepackage[alphabetic]{amsrefs}
% margin spacing
\usepackage[top=1in, bottom=1in, left=0.5in, right=0.5in]{geometry}
\usepackage{hanging}
\usepackage{amsfonts, amsmath, amssymb, amsthm}
\usepackage{physics}
\usepackage[none]{hyphenat}
\usepackage{fancyhdr}
\usepackage{graphicx}
\graphicspath{{./images/}}
\usepackage{color}
\newcommand{\sai}[1]{\textcolor{red}{#1}}
\usepackage{enumitem}
\usepackage{mathrsfs}
\usepackage{hyperref}
\usepackage[noabbrev, capitalise]{cleveref}
\crefformat{equation}{equation~#2#1#3}
\crefformat{lemma}{\textrm{Lemma}~#2#1#3}
\usepackage{quiver}

% theorems
\theoremstyle{plain}
\newtheorem{lem}{Lemma}
\newtheorem{lemma}[lem]{Lemma}
\newtheorem{thm}[lem]{Theorem}
\newtheorem{theorem}[lem]{Theorem}
\newtheorem{prop}[lem]{Proposition}
\newtheorem{proposition}[lem]{Proposition}
\newtheorem{cor}[lem]{Corollary}
\newtheorem{corollary}[lem]{Corollary}

\theoremstyle{definition}
\newtheorem{defn}[lem]{Definition}
\newtheorem{definition/}[lem]{Definition}
\newenvironment{definition}
  {\renewcommand{\qedsymbol}{\textdagger}%
   \pushQED{\qed}\begin{definition/}}
  {\popQED\end{definition/}}

\numberwithin{equation}{section}
\numberwithin{lem}{section}

% header/footer formatting
\pagestyle{fancy}
\fancyhead{}
\fancyfoot{}
\fancyhead[L]{An isoperimetric inequality}
\fancyhead[C]{}
\fancyhead[R]{\thepage}
\fancyfoot[R]{}
\renewcommand{\headrulewidth}{1pt}

% paragraph indentation/spacing
\setlength{\parindent}{0cm}
\setlength{\parskip}{10pt}
\renewcommand{\baselinestretch}{1.25}

% new commands
\newcommand{\eq}[1]{\overset{(#1)}{=}}
\DeclareMathOperator{\im}{im}
\DeclareMathOperator{\Tot}{Tot}

% smileys frownies
\usepackage{wasysym}
\newcommand{\smallhappy}{\raisebox{-.14em}{\smiley}}
\newcommand{\happy}{\raisebox{-.24em}{\resizebox{1.2em}{!}{\smiley}}}
\newcommand{\smallsad}{\raisebox{-.14em}{\frownie}}
\newcommand{\sad}{\raisebox{-.24em}{\resizebox{1.2em}{!}{\frownie}}}
\DeclareMathOperator{\mathhappy}{\!\happy\!}
\DeclareMathOperator{\smallmathhappy}{\!\smallhappy\!}
\DeclareMathOperator{\mathsad}{\!\sad\!}
\DeclareMathOperator{\smallmathsad}{\!\smallsad\!}

\begin{document}
These notes closely follow \textit{Fourier Analysis: An Introduction} by Stein and Shakarchi.

Let $\Gamma$ be a closed and simple curve in the plane; that is, the endpoints of the curve coincide and the curve does not intersect itself except at its endpoints (it would look like a loop in the plane). If the arclength $\ell$ of $\Gamma$ is fixed, varying $\Gamma$ will vary the area $A$ of the region in the plane enclosed by $\Gamma$.

A natural question is to ask how big or how small $A$ can be. It seems reasonable to think that by ``squishing'' $\Gamma$ flat, the area $A$ can be made arbitrarily small, if not zero. This is true, but what may be more interesting is how big $A$ can get. Some experimenting (with a piece of string made into a loop, for example) suggests that the maximal area may be obtained when $\Gamma$ is a circle, which is the most ``round'' $\Gamma$ could be in some sense.

We will not address some important questions about what it means for a closed curve to enclose a region in the plane, how to make sense of the area of the region enclosed, the ``roundness'' of a curve, or what can be said if the curve $\Gamma$ is not very well behaved (imagine a super jagged, not-smooth curve that still has a finite arclength, somehow). We will ignore most of these difficulties for the purposes of illustrating the outcome we should expect. 

We recall some definitions and make some assumptions before proving the following theorem: 
\begin{theorem}
    The area $A$ of a simple closed curve $\Gamma$ in $\mathbb R^2$ of fixed arclength $\ell$ satisfies the \textbf{isoperimetric inequality} 
    \[A\leq \frac{\ell^2}{4\pi},\]
    in which equality occurs if and only if $\Gamma$ is a circle.
\end{theorem}
A curve $\Gamma\subset \mathbb R^2$ is parameterized by a map $\gamma\colon [a,b]\to\mathbb R^2$ if $\im\gamma = \Gamma$. Throughout, we will assume our curves $\Gamma$ are smooth enough; in particular, we will only consider smooth parameterizations $\gamma$ of $\Gamma$ that are of class $C^1$ with $\gamma^\prime$ never zero. This smoothness condition ensure that $\Gamma$ has a well-defined tangent vector at each point that varies continuously, which leads to the notion that $\Gamma$ is oriented, in the sense that as $t$ varies, the curve $\Gamma$ is traced out either ``clockwise'' or ``counterclockwise''.

That $\Gamma$ is simple and closed translates to the condition that $\gamma(t_1)\neq \gamma(t_2)$ unless $t_1 = a$ and $t_2 = b$, so that $\gamma(a) = \gamma(b)$. Thus $\gamma$ extends to a periodic function on $\mathbb R$ of period $b-a$, which we think of as a function on the circle $S^1$. We define the arclength $\ell$ of the curve $\Gamma$ using a smooth parameterization $\gamma(t) = (x(t),y(t))$ by 
\[\ell = \int_a^b |\gamma^\prime(t)| \,dt = \int_a^b \sqrt{x^\prime(t)^2 + y^\prime(t)^2}\,dt.\] The area of the region $\Gamma$ encloses is given by the formula 
\[A = \frac{1}{2}\abs{\int_\Gamma(x\,dy - y \,dx)} = \frac{1}{2}\abs{\int_a^b x(t)y^\prime(t) + y(t)x^\prime(t)\,dt}\]
arising from Green's theorem from multivariable calculus. That $\Gamma$ is simple, closed, and its length $\ell$ and the area $A$ it encloses do not depend on the choice of smooth parameterization $\gamma$.

We will henceforth use a parameterization of $\Gamma$ by arclength. A smooth parameterization $\gamma$ of $\Gamma$ for which $\abs{\gamma^\prime(s)} = 1$ is a parameterization by arclength; that is, a parameterization that travels at constant speed. It is called a parameterization by arclength because in this case, the length of $\gamma$ becomes $b-a$, so by translating the domain we may define $\gamma$ on $[0,\ell]$. We change to the variable $s$ to indicate that $s$ has units of length, since $s$ ranges from $0$ to $\ell$. Any smooth parameterization $\gamma$ admits a parameterization of $\Gamma$ by arclength.

We recall some definitions and results of Fourier series as well. Let $f\colon [0,2\pi]\to\mathbb C$ be an integrable function. Then the $n$-th Fourier coefficient of $f$ is 
\[\hat f(n) = \frac{1}{2\pi}\int_0^{2\pi}f(x)\exp(-inx)\,dx.\]
\end{document}