\documentclass[11pt,leqno]{article}
\headheight=13.6pt

% packages
\usepackage[alphabetic]{amsrefs}
\usepackage{physics}
% margin spacing
\usepackage[top=1in, bottom=1in, left=0.5in, right=0.5in]{geometry}
\usepackage{hanging}
\usepackage{amsfonts, amsmath, amssymb, amsthm}
\usepackage{extarrows}
\usepackage[none]{hyphenat}
\usepackage{fancyhdr}
\usepackage{graphicx}
\graphicspath{{./images/}}
\usepackage{float}
\usepackage{color}
\newcommand{\sai}[1]{\textcolor{red}{#1}}
\usepackage{enumitem}
% \usepackage{mathrsfs}
\usepackage{hyperref}
\usepackage[noabbrev, capitalise]{cleveref}
\crefformat{equation}{equation~#2#1#3}
\crefformat{lemma}{\textrm{Lemma}~#2#1#3}
\usepackage{quiver}

% theorems
\theoremstyle{plain}
\newtheorem{lem}{Lemma}
\newtheorem{lemma}[lem]{Lemma}
\newtheorem{thm}[lem]{Theorem}
\newtheorem{theorem}[lem]{Theorem}
\newtheorem{prop}[lem]{Proposition}
\newtheorem{proposition}[lem]{Proposition}
\newtheorem{cor}[lem]{Corollary}
\newtheorem{corollary}[lem]{Corollary}
\newtheorem{conj}[lem]{Conjecture}
\newtheorem{fact}[lem]{Fact}
\newtheorem{form}[lem]{Formula}

\theoremstyle{definition}
\newtheorem{defn}[lem]{Definition}
\newtheorem{definition/}[lem]{Definition}
\newenvironment{definition}
  {\renewcommand{\qedsymbol}{\textdagger}%
   \pushQED{\qed}\begin{definition/}}
  {\popQED\end{definition/}}
\newtheorem{example}[lem]{Example}
\newtheorem{remark}[lem]{Remark}
\newtheorem{exercise}[lem]{Exercise}
\newtheorem{notation}[lem]{Notation}

\numberwithin{equation}{section}
\numberwithin{lem}{section}

% header/footer formatting
\pagestyle{fancy}
\fancyhead{}
\fancyfoot{}
\fancyhead[L]{M 392C}
\fancyhead[C]{HW1}
\fancyhead[R]{Riemann Surfaces}
\fancyfoot[R]{\thepage}
\renewcommand{\headrulewidth}{1pt}

% paragraph indentation/spacing
\setlength{\parindent}{0cm}
\setlength{\parskip}{10pt}
\renewcommand{\baselinestretch}{1.25}

% mathscript S
\usepackage[mathscr]{euscript}

% math operators
\DeclareMathOperator{\Span}{span}

% bracket notation for inner product
\usepackage{mathtools}

\DeclarePairedDelimiterX{\abr}[1]{\langle}{\rangle}{#1}

% smileys frownies
\usepackage{wasysym}
\newcommand{\smallhappy}{\raisebox{-.14em}{\smiley}}
\newcommand{\happy}{\raisebox{-.24em}{\resizebox{1.2em}{!}{\smiley}}}
\newcommand{\smallsad}{\raisebox{-.14em}{\frownie}}
\newcommand{\sad}{\raisebox{-.24em}{\resizebox{1.2em}{!}{\frownie}}}
\DeclareMathOperator{\mathhappy}{\!\happy\!}
\DeclareMathOperator{\smallmathhappy}{\!\smallhappy\!}
\DeclareMathOperator{\mathsad}{\!\sad\!}
\DeclareMathOperator{\smallmathsad}{\!\smallsad\!}

% set page count index to begin from 1
\setcounter{page}{1}

\begin{document}
Sudharshan K.V. (1.1, 1.2) and Sai Sivakumar (1.3, 1.4, 1.5)

\section{The Schwarzian derivative}
\begin{definition}
    The \textit{Schwarzian derivative} of a holomorphic function $f\colon \mathbb C\to \mathbb C$ is defined by
    \[\mathscr Sf = \frac{f^{\prime\prime\prime}}{f^\prime}-\frac{3}{2}\biggl(\frac{f^{\prime\prime}}{f^\prime}\biggr)^2 = \biggl(\frac{f^{\prime\prime}}{f^\prime}\biggr)^\prime - \frac{1}{2}\biggl(\frac{f^{\prime\prime}}{f^\prime}\biggr)^2.\qedhere\]
\end{definition}
The second description of the Schwarzian derivative is as a Riccati equation (see \href{https://ftfsite.ru/wp-content/files/tfkp_endlish_2.2.pdf}{Ablowitz, Fokas (2003)}, p.383 and the Wolfram MathWorld \href{https://mathworld.wolfram.com/RiccatiDifferentialEquation.html}{article} on the Riccati equation).
\subsection{M\"obius maps}
The Schwarzian derivative measures a function's deviation from being a M\"obius map.
\subsection{Jets}
\subsection{Sturm-Liouville theory}
Given two linearly independent holomorphic functions $v,w$ on an open set $U\subseteq \mathbb C$ with $w\neq 0$ on $U$, we can define the Schwarzian derivative of $f = v/w$ directly. The Wronskian of $v,w$
\[W = W(v,w) = \det\begin{pmatrix}
    v & w \\ v^\prime & w^\prime
\end{pmatrix} = vw^\prime - wv^\prime\]
is not identically zero since $v,w$ are linearly independent and because $w$ is nonvanishing on $U$ (see \href{https://www.ams.org/journals/tran/1901-002-02/S0002-9947-1901-1500560-5/S0002-9947-1901-1500560-5.pdf}{B\^ocher (1901)}, by the contrapositive of Theorem I. on p.140). Calculations (using WolframAlpha since doing it by hand would be painful) give
\begin{align*}
    \frac{f^{\prime\prime\prime}}{f^\prime} &= -\frac{6 v{} w^{\prime}{}^3}{w{}^2 (w{} v^{\prime}{} - v{} w^{\prime}{})} + \frac{6 v^{\prime}{} w^{\prime}{}^2}{w{} (w{} v^{\prime}{} - v{} w^{\prime}{})} - \frac{v{} w^{\prime\prime\prime}{}}{w{} v^{\prime}{} - v{} w^{\prime}{}} \\
    &\phantom{=}+ \frac{6 v{} w^{\prime}{} w^{\prime\prime}{}}{w{} (w{} v^{\prime}{} - v{} w^{\prime}{})} - \frac{3 v^{\prime}{} w^{\prime\prime}{}}{w{} v^{\prime}{} - v{} w^{\prime}{}} + \frac{v^{\prime\prime\prime}{} w{}}{w{} v^{\prime}{} - v{} w^{\prime}{}} \\
    &\phantom{=}- \frac{3 v^{\prime\prime}{} w^{\prime}{}}{w{} v^{\prime}{} - v{} w^{\prime}{}} \\
    \text{and}\quad -\frac{3}{2}\biggl(\frac{f^{\prime\prime}}{f^\prime}\biggr)^2 &= -\frac{6 v{}^2 w^{\prime}{}^4}{w{}^2 (w{} v^{\prime}{} - v{} w^{\prime}{})^2} + \frac{12 v{} v^{\prime}{} w^{\prime}{}^3}{w{} (w{} v^{\prime}{} - v{} w^{\prime}{})^2} - \frac{6 v^{\prime}{}^2 w^{\prime}{}^2}{(w{} v^{\prime}{} - v{} w^{\prime}{})^2} \\
    &\phantom{=}+ \frac{6 v{}^2 w^{\prime}{}^2 w^{\prime\prime}{}}{w{} (w{} v^{\prime}{} - v{} w^{\prime}{})^2} - \frac{6 v{} v^{\prime}{} w^{\prime}{} w^{\prime\prime}{}}{(w{} v^{\prime}{} - v{} w^{\prime}{})^2} - \frac{3 v{}^2 w^{\prime\prime}{}^2}{2 (w{} v^{\prime}{} - v{} w^{\prime}{})^2}\\
    &\phantom{=} - \frac{6 v{} v^{\prime\prime}{} w^{\prime}{}^2}{(w{} v^{\prime}{} - v{} w^{\prime}{})^2} + \frac{6 w{} v^{\prime}{} v^{\prime\prime}{} w^{\prime}{}}{(w{} v^{\prime}{} - v{} w^{\prime}{})^2} - \frac{3 w{}^2 v^{\prime\prime}{}^2}{2 (w{} v^{\prime}{} - v{} w^{\prime}{})^2} + \frac{3 v{} w{} v^{\prime\prime}{} w^{\prime\prime}{}}{(w{} v^{\prime}{} - v{} w^{\prime}{})^2}.
\end{align*}
The denominators in every expression above are nonvanishing on $U$ except at possibly a discrete set of points, so the Schwarzian derivative $\mathscr S(v/w)$ is defined on $U$ (or most of it).

Further, if $v,w$ satisfy the Sturm-Liouville equation
\[g^{\prime\prime} + ug = 0;\]
that is, the solution space of this Sturm-Liouville equation is $\Span_{\mathbb C}\{v,w\}$, then we can further argue that $u = \mathscr S(v/w)/2$.

\subsection{Quadratic differentials}
\subsection{Cocycles and central extensions}

\newpage\section*{References}

Certain cases in which the vanishing of the Wronskian is a sufficient condition for linear dependence by Maxime B\^ocher, in Trans.~Amer.~Math.~Soc.~2 (1901), 139-149. (\href{https://www.ams.org/journals/tran/1901-002-02/S0002-9947-1901-1500560-5/S0002-9947-1901-1500560-5.pdf}{link to PDF})

Complex Variables: Introduction and Applications by Mark J.~Ablowitz and Athanassios S.~Fokas, Cambridge University Press (2003), 383. (\href{https://ftfsite.ru/wp-content/files/tfkp_endlish_2.2.pdf}{link to PDF})

Wolfram MathWorld article on the Riccati equation. (\href{https://mathworld.wolfram.com/RiccatiDifferentialEquation.html}{link to webpage})

\end{document}