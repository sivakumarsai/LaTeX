\documentclass[11pt,leqno]{article}
\headheight=13.6pt

% packages
\usepackage[alphabetic]{amsrefs}
\usepackage{physics}
% margin spacing
\usepackage[top=1in, bottom=1in, left=0.5in, right=0.5in]{geometry}
\usepackage{amsfonts, amsmath, amssymb, amsthm}
\usepackage{extarrows}
\usepackage[none]{hyphenat}
\usepackage{fancyhdr}
\usepackage{graphicx}
\graphicspath{{./images/}}
\usepackage{float}
\usepackage{color}
\newcommand{\sai}[1]{\textcolor{red}{#1}}
\usepackage{enumitem}
% \usepackage{mathrsfs}
\usepackage{hyperref}
\usepackage[noabbrev, capitalise]{cleveref}
\crefformat{equation}{equation~#2#1#3}
\crefformat{lemma}{\textrm{Lemma}~#2#1#3}
\usepackage{quiver}

% theorems
\theoremstyle{plain}
\newtheorem{lem}{Lemma}
\newtheorem{lemma}[lem]{Lemma}
\newtheorem{thm}[lem]{Theorem}
\newtheorem{theorem}[lem]{Theorem}
\newtheorem{prop}[lem]{Proposition}
\newtheorem{proposition}[lem]{Proposition}
\newtheorem{cor}[lem]{Corollary}
\newtheorem{corollary}[lem]{Corollary}
\newtheorem{conj}[lem]{Conjecture}
\newtheorem{fact}[lem]{Fact}
\newtheorem{form}[lem]{Formula}

\theoremstyle{definition}
\newtheorem{defn}[lem]{Definition}
\newtheorem{definition/}[lem]{Definition}
\newenvironment{definition}
  {\renewcommand{\qedsymbol}{\textdagger}%
   \pushQED{\qed}\begin{definition/}}
  {\popQED\end{definition/}}
\newtheorem{example}[lem]{Example}
\newtheorem{remark}[lem]{Remark}
\newtheorem{exercise}[lem]{Exercise}
\newtheorem{notation}[lem]{Notation}

\numberwithin{equation}{section}
\numberwithin{lem}{section}

% header/footer formatting
\pagestyle{fancy}
\fancyhead{}
\fancyfoot{}
\fancyhead[L]{M 392C}
\fancyhead[C]{HW1}
\fancyhead[R]{Riemann Surfaces}
\fancyfoot[R]{\thepage}
\renewcommand{\headrulewidth}{1pt}

% paragraph indentation/spacing
\setlength{\parindent}{0cm}
\setlength{\parskip}{10pt}
\renewcommand{\baselinestretch}{1.25}

% mathscript S
\usepackage[mathscr]{euscript}

% math operators
\DeclareMathOperator{\Span}{span}
\DeclareMathOperator{\Sym}{Sym}
\renewcommand{\ev}{\mathrm{ev}}
\DeclareMathOperator{\SL}{SL}
\DeclareMathOperator{\GL}{GL}
\newcommand{\bbC}{\mathbb C}
\DeclareMathOperator{\Diff}{Diff}
\newcommand{\Vir}{\mathrm{Vir}}
\DeclareMathOperator{\ad}{ad}
\DeclareMathOperator{\Vect}{Vect}

% bracket notation for inner product
\usepackage{mathtools}

\DeclarePairedDelimiterX{\abr}[1]{\langle}{\rangle}{#1}

% smileys frownies
\usepackage{wasysym}
\newcommand{\smallhappy}{\raisebox{-.14em}{\smiley}}
\newcommand{\happy}{\raisebox{-.24em}{\resizebox{1.2em}{!}{\smiley}}}
\newcommand{\smallsad}{\raisebox{-.14em}{\frownie}}
\newcommand{\sad}{\raisebox{-.24em}{\resizebox{1.2em}{!}{\frownie}}}
\DeclareMathOperator{\mathhappy}{\!\happy\!}
\DeclareMathOperator{\smallmathhappy}{\!\smallhappy\!}
\DeclareMathOperator{\mathsad}{\!\sad\!}
\DeclareMathOperator{\smallmathsad}{\!\smallsad\!}

% set page count index to begin from 1
\setcounter{page}{1}

% array column and row separation
\arraycolsep = 2pt
\renewcommand{\arraystretch}{.8}

\begin{document}
Sudharshan K.V. (1.1), Sai Sivakumar (1.2, 1.3)
% , and Ryan Wandsnider ( )

\section{The Schwarzian derivative}
Let $f: \mathbb C \to \mathbb C$ be a holomorphic function. 
\begin{definition}
    The \textit{Schwarzian derivative} of $f$ is defined by
    \[\mathscr Sf = \frac{f^{\prime\prime\prime}}{f^\prime}-\frac{3}{2}\biggl(\frac{f^{\prime\prime}}{f^\prime}\biggr)^2 = \biggl(\frac{f^{\prime\prime}}{f^\prime}\biggr)^\prime - \frac{1}{2}\biggl(\frac{f^{\prime\prime}}{f^\prime}\biggr)^2.\qedhere\]
\end{definition}
The second description of the Schwarzian derivative is as a Riccati equation (see \href{https://ftfsite.ru/wp-content/files/tfkp_endlish_2.2.pdf}{Ablowitz, Fokas (2003)}, p.383). The Schwarzian is an obstruction to a function being a M\"obius transform. More precisely, for a local diffeomorphism $f$, the Schwarzian $\mathscr Sf$ vanishes if and only if $f$ is locally a M\"obius transform.

\subsection{Jets}
The $k$-jet of $f$ at $z$ is the $k$-tuple $$J^3(f) = (f(z), \dots , f^{(k)}(z)).$$ The jet of $f$ at $x$ records the Taylor expansion coefficients of $f$ at $x$. A M\"obius transform is determined by its $3$-jet at a point - one may see this by a direct computation. Suppose we want to construct a function, the \textit{Schwarzian}, with the property that it evaluates to $0$ precisely at the M\"obius transforms.

We expect the Schwarzian to be a function of the $3$-jet of $f$, and we would like for it to be well-behaved under change of variables. How $\mathscr S$ changes under pullback is given by the cocycle condition (see \cref{s: cocycle}). Recall that for a local diffeomorphism $f$, the derivative $f'$ is non-vanishing. Scaling $f$ by a constant $C$ should preserve the Schwarzian - since scaling is a M\"obius transform. This suggests that the Schwarzian must be a homogeneous expression in $f',f'', f'''$. Since $f'$ is non-zero, we can consider the Schwarzian as a function of $f''/f', f'''/f'$. Changing the variable $z$ by $\lambda z$ should not affect the Schwarzian either, but this transformation scales $f^{(k)}$ by $\lambda^k$. Thus, a suitable candidate for the Schwarzian would be a linear combination of $f'''/f'$ and $(f''/f')^2$. Of course we may normalize the first term to have coefficient $1$. We will explain why the second term has coefficient $-3/2$.

For instance, we can just evaluate the Schwarzian for the function $1/z$. We have $f'''/f' = 6/z^2$ and $(f''/f')^2 = 4/z^2$. This already shows that the second coefficient must be $-3/2$. Furthermore, we have: 

\begin{theorem}
    For a M\"obius map $M$ corresponding to $\big(\!\begin{smallmatrix}
        a & b \\ c & d
    \end{smallmatrix}\!\big)$ in $\GL_2 (\mathbb C)$, the Schwarzian $\mathscr SM$ vanishes.
\end{theorem}
\begin{proof}
    Let $u = \log M'(z)$. See that $\mathscr SM = u''-\frac12 (u')^2$. For a M\"obius function $M$, the log of $M'$ is easy to compute. Indeed, $\log M'(z) = \log \gamma - 2 \log (cz+d)$, where $\gamma = ad-bc$ is non-zero. So one has $$u'(z) = \frac{-2c}{cz+d}, u''(z) = \frac{2c^2}{(cz+d)^2}.$$ It is now clear that $\mathscr SM = u'' - \frac 12 (u')^2 = 0$.
\end{proof}
Furthermore, the converse of the above theorem holds: 
\begin{theorem}
    For a holomorphic map $f$, if $\mathscr Sf = 0$ then $f$ is locally a M\"obius function.
\end{theorem}
\begin{proof}
    Assume $f$ is not of the form $f(z) = az+b$; that is, $f''\neq 0$ (such functions are locally M\"obius functions and do satisfy $\mathscr Sf = 0$). Then if $\mathscr Sf = 0$, then
    \[\frac{f^{\prime\prime\prime}}{f^\prime}-\frac{3}{2}\biggl(\frac{f^{\prime\prime}}{f^\prime}\biggr)^2 = \biggl(\frac{f^{\prime\prime}}{f^\prime}\biggr)^\prime - \frac{1}{2}\biggl(\frac{f^{\prime\prime}}{f^\prime}\biggr)^2 = 0\]
    Substituting $p = f''/f'$ yields the Riccati equation $p'-p^2/2 = 0$, which by the substitution $p = -2y'/y$ we can reduce to the linear differential equation $y'' = 0$. Solutions are of the form $y = az+b$ for constants $a,b$; assume $a\neq 0$ (if $a = 0$, then $p = 0$, which leads to $f = ax+b$, which was disallowed). By undoing one substitution we obtain $p = -2a/(az+b)$. It remains to solve the differential equation $f''/f' = p = -2a/(az+b)$. This can be done by substituting $u = f'$, which reduces the problem to a first order linear differential equation whose solution is $u = c/(az+b)^2$ for some constant $c$; to obtain nontrivial solutions, let $c\neq 0$. Then $f = -c/(a(az+b)) + d = $ for some constant $d$; by the theory of ordinary differential equations this solution exists at least locally. That $f$ is a M\"obius map requires $a^2c\neq 0$, which we have arranged for.
\end{proof}
% \newpage\subsection{Derivation of the Schwarzian Derivative}

% At first glance, the motivation for the definition of the Schwarzian Derivative is difficult to discern. Another method for deriving such a formula that satisfies the desired property of characterizing M\"obius transformations involves the cross-ratio.

% Recall that the cross-ratio for $4$ collinear points $A,B,C,D$ in $\bbC$ is given by
% $$
% [A,B;C,D] = \frac{AC\cdot BD}{BC \cdot AD}.
% $$
% The group of automorphisms of $\bbC$ cross-ratio is 


\subsection{Schwarzian derivatives of quotients and Sturm-Liouville theory}
Given two linearly independent analytic (or sufficiently smooth enough) functions $v,w$ on an open set $U\subseteq \mathbb R$ with $w\neq 0$ on $U$, we can define the Schwarzian derivative of $f = v/w$ directly. The Wronskian of $v,w$
\[W = W(v,w) = \det\begin{pmatrix}
    v & w \\ v^\prime & w^\prime
\end{pmatrix} = vw^\prime - wv^\prime\]
is not identically zero since $v,w$ are linearly independent and because $w$ is nonvanishing on $U$ (see \href{https://www.ams.org/journals/tran/1901-002-02/S0002-9947-1901-1500560-5/S0002-9947-1901-1500560-5.pdf}{B\^ocher (1901)}, by the contrapositive of Theorem I. on p.140). Calculations (using WolframAlpha since doing it by hand would be painful) give
\begin{align*}
    \frac{f^{\prime\prime\prime}}{f^\prime} &= -\frac{6 v{} w^{\prime}{}^3}{w{}^2 (w{} v^{\prime}{} - v{} w^{\prime}{})} + \frac{6 v^{\prime}{} w^{\prime}{}^2}{w{} (w{} v^{\prime}{} - v{} w^{\prime}{})} - \frac{v{} w^{\prime\prime\prime}{}}{w{} v^{\prime}{} - v{} w^{\prime}{}} \\
    &\phantom{=}+ \frac{6 v{} w^{\prime}{} w^{\prime\prime}{}}{w{} (w{} v^{\prime}{} - v{} w^{\prime}{})} - \frac{3 v^{\prime}{} w^{\prime\prime}{}}{w{} v^{\prime}{} - v{} w^{\prime}{}} + \frac{v^{\prime\prime\prime}{} w{}}{w{} v^{\prime}{} - v{} w^{\prime}{}} \\
    &\phantom{=}- \frac{3 v^{\prime\prime}{} w^{\prime}{}}{w{} v^{\prime}{} - v{} w^{\prime}{}} \\\intertext{and}
    -\frac{3}{2}\biggl(\frac{f^{\prime\prime}}{f^\prime}\biggr)^2 &= -\frac{6 v{}^2 w^{\prime}{}^4}{w{}^2 (w{} v^{\prime}{} - v{} w^{\prime}{})^2} + \frac{12 v{} v^{\prime}{} w^{\prime}{}^3}{w{} (w{} v^{\prime}{} - v{} w^{\prime}{})^2} - \frac{6 v^{\prime}{}^2 w^{\prime}{}^2}{(w{} v^{\prime}{} - v{} w^{\prime}{})^2} \\
    &\phantom{=}+ \frac{6 v{}^2 w^{\prime}{}^2 w^{\prime\prime}{}}{w{} (w{} v^{\prime}{} - v{} w^{\prime}{})^2} - \frac{6 v{} v^{\prime}{} w^{\prime}{} w^{\prime\prime}{}}{(w{} v^{\prime}{} - v{} w^{\prime}{})^2} - \frac{3 v{}^2 w^{\prime\prime}{}^2}{2 (w{} v^{\prime}{} - v{} w^{\prime}{})^2}\\
    &\phantom{=} - \frac{6 v{} v^{\prime\prime}{} w^{\prime}{}^2}{(w{} v^{\prime}{} - v{} w^{\prime}{})^2} + \frac{6 w{} v^{\prime}{} v^{\prime\prime}{} w^{\prime}{}}{(w{} v^{\prime}{} - v{} w^{\prime}{})^2} - \frac{3 w{}^2 v^{\prime\prime}{}^2}{2 (w{} v^{\prime}{} - v{} w^{\prime}{})^2} + \frac{3 v{} w{} v^{\prime\prime}{} w^{\prime\prime}{}}{(w{} v^{\prime}{} - v{} w^{\prime}{})^2}.
\end{align*}
The denominators in every expression above are nonvanishing on $U$ except at possibly a discrete set of points, so the Schwarzian derivative $\mathscr S(v/w)$ is defined on most of $U$, if not all of it.

Further, if $v,w$ satisfy the Sturm-Liouville equation
\begin{equation}\label{eq: sl}
    g^{\prime\prime} + ug = 0;
\end{equation}
that is, the solution space of this Sturm-Liouville equation is $\Span_{\mathbb R}\{v,w\}$, then we can further argue that $u = \mathscr S(v/w)/2$. We can use a clever ODE-style argument to obtain this. Let $f = v/w$ so that $v = wf$. Repeated differentiation of $v$ gives 
\begin{align*}
    v^\prime &= w^\prime f + wf^\prime\\
    v^{\prime\prime} &= w^{\prime\prime}f + 2w^\prime f^\prime + wf^{\prime\prime}.
\end{align*}
Since $v$ and $w$ are solutions to \cref{eq: sl}, we have 
\[0 = v^{\prime\prime} + uv = (w^{\prime\prime} + uw)f + 2w^\prime f^\prime + wf^{\prime\prime} = 2w^\prime f^\prime + wf^{\prime\prime}\]
Thus $w^\prime/w = -f^{\prime\prime}/2f^\prime$. Differentiating $w^\prime/w$ yields
\[\biggl(\frac{w^\prime}{w}\biggr)^\prime = \frac{ww^{\prime\prime}-w^\prime{}^2}{w^2} = \frac{w^{\prime\prime}}{w}-\biggl(\frac{w^\prime}{w}\biggr)^2.\]
Again since $w$ satisfies \cref{eq: sl}, we finally obtain 
\[-u = \frac{w^{\prime\prime}}{w} = \biggl(\frac{w^\prime}{w}\biggr)^\prime+\biggl(\frac{w^\prime}{w}\biggr)^2 = \biggl(-\frac{f^{\prime\prime}}{2f^\prime}\biggr)^\prime+\biggl(-\frac{f^{\prime\prime}}{2f^\prime}\biggr)^2 = -\frac{1}{2}\biggl[\biggl(\frac{f^{\prime\prime}}{f^\prime}\biggr)^\prime-\frac{1}{2}\biggl(\frac{f^{\prime\prime}}{f^\prime}\biggr)^2\biggr]=-\frac{1}{2}\mathscr S\biggl(\frac{v}{w}\biggr);\] that is, $u = \mathscr S(v/w)/2$ as desired. Note that the argument is symmetric in $v$ and $w$.

Sturm-Liouville equations of the form $g^{\prime\prime} + ug = 0$ on $\mathbb R$ are in one-to-one correspondence with equivalence classes of non-degenerate curves in $\mathbb{RP}^1$. To a Sturm-Liouville equation $g^{\prime\prime} + ug = 0$, we know its solution space $V$ is a two-dimensional real vector space. To each $t\in \mathbb R$, the evaluate at $t$ map $V\xrightarrow{\ev_t} \mathbb R$ has one-dimensional kernel $V_t$ given by all the solutions vanishing at $t$. By identifying $V$ with $\mathbb R^2$ by any choice of basis, the family of subspaces $V_t$ of $V$ depending on $t$ gives rise to a smoothly changing line passing through the origin; in other words, we obtain a curve in $\mathbb{RP}^1$ up to projective equivalence.

Conversely, any non-degenerate curve $\gamma$ in $\mathbb{RP}^1$ can be uniquely lifted to a curve $\Gamma$ in $\mathbb R^2$ for which the usual inner product of the vectors $\Gamma$, $\Gamma^\prime$ always has length $1$; that is, for all $t$ we have $\abs{\abr{\Gamma(t),\Gamma^\prime(t)}} = 1$. By differentiating this we obtain that $\Gamma^{\prime\prime}(t)$ is proportional to $\Gamma(t)$ for all $t$; that is, $\Gamma$ satisfies the Sturm-Liouville equation $g^{\prime\prime} + ug = 0$ where $u$ is $\Gamma^{\prime\prime}/\Gamma$. By replacing $\gamma$ with a projectively equivalent curve, its lift $\Gamma$ is replaced by $A\Gamma$ for $A\in \SL_2(\mathbb R)$, so the Sturm-Liouville equation remains the same (these matrices would preserve the area form on $\mathbb R^2$). The procedure above recovers $u$ in terms of the components of $\gamma$.

\subsection{Quadratic differentials, cocycles and central extensions}\label{s: cocycle}
Let $f$ be a diffeomorphism of $\mathbb{RP}^1$ with itself (and think of $\mathbb{RP}^1$ as the affine line with coordinate $x$ and the point at infinity). The Schwarzian derivative of $f$ measures how much the cross-ratio of four infinitesimally close points changes under $f$.

Let $x\in\mathbb{RP}^1$ and $v\in T_x\mathbb{RP}^1$. Extend $v$ to a vector field in a neighborhood of $x$, and denote by $\phi_t$ the one-parameter group of flows along this vector field. It turns out that for $\varepsilon> 0$ small, the four points $x, \phi_{\varepsilon}(x), \phi_{2\varepsilon}(x), \phi_{3\varepsilon}(x)$ satisfy 
\[[f(x),f(\phi_{\varepsilon}(x)), f(\phi_{2\varepsilon}(x)), \phi_{3\varepsilon}(x)] = [x,\phi_{\varepsilon}(x), \phi_{2\varepsilon}(x), \phi_{3\varepsilon}(x)] - 2\varepsilon^2\mathscr Sf(x) + O(\varepsilon^3)\]
where indeed the first-order term in $\varepsilon$ is zero. The diffeomorphism $f$ only changes the cross-ratio of the four points $x, \phi_{\varepsilon}(x), \phi_{2\varepsilon}(x), \phi_{3\varepsilon}(x)$ in the second-order in $\varepsilon$ and higher.

It also turns out that the coefficient $2\mathscr Sf(x)$ of $\varepsilon^2$ in the expansion above does not depend on the choice of extension of $v$ to a vector field in a neighborhood of $x$ (e.g., it would not matter which chart around $x$ we chose to build this extension on). It will, however, depend on $f$ and $x$, and it is homogeneous of degree $2$ in $v$ (manifested in the expansion above as scaling $2\mathscr Sf(x)$ by $\varepsilon^2$). In this setting the Schwarzian derivative defines a \textit{quadratic differential} 
\[\mathscr Sf \coloneqq \mathscr Sf(\dd x)^2\in \Sym^2(T^\ast\mathbb{RP}^1),\]
(so at a point $x$, we have $\mathscr Sf(x)(\dd x)^2\in \Sym^2(T^\ast_x\mathbb{RP}^1)$).

For any quadratic differential $\phi = a(\dd x)^2$, changing coordinates by $h$ is given by the formula $h^\ast\phi = (a\circ h)\cdot h^\prime{}^2(\dd x)^2$. For diffeomorphisms $f,g$ of $\mathbb{RP}^1$ with itself, we have the equality of quadratic differentials
\begin{equation}\label{eq: chain}
    \mathscr S(f\circ g) = g^\ast\mathscr Sf + \mathscr Sg,
\end{equation}
which follows from the ordinary chain rule for the Schwarzian, given by
\[\mathscr S(f\circ g)(z) = (\mathscr Sf)(g(z))\cdot g^{\prime}(z)^2 + \mathscr Sg(z).\]
Indeed, 
\begin{align*}
    (f\circ g)^{\prime} &= g^\prime\cdot (f^\prime\circ g)\\
    (f\circ g)^{\prime\prime} &= g^\prime{}^2\cdot(f^{\prime\prime}\circ g) + g^{\prime\prime}\cdot (f^\prime\circ g)\\
    (f\circ g)^{\prime\prime\prime} &= g^\prime{}^3\cdot(f^{\prime\prime\prime}\circ g) + 3g^\prime g^{\prime\prime}\cdot (f^{\prime\prime}\circ g) + g^{\prime\prime\prime}\cdot(f^\prime\circ g)
\end{align*}
and 
\begin{align*}
    \mathscr S(f\circ g) &= \frac{g^\prime{}^3\cdot(f^{\prime\prime\prime}\circ g) + 3g^\prime g^{\prime\prime}\cdot (f^{\prime\prime}\circ g) + g^{\prime\prime\prime}\cdot(f^\prime\circ g)}{g^\prime\cdot (f^\prime\circ g)} - \frac{3}{2}\biggl(\frac{g^\prime{}^2\cdot(f^{\prime\prime}\circ g) + g^{\prime\prime}\cdot (f^\prime\circ g)}{g^\prime\cdot (f^\prime\circ g)}\biggr)^2\\
    &=g^\prime{}^2\frac{f^{\prime\prime\prime}\circ g}{f^\prime\circ g} + 3g^{\prime\prime}\frac{f^{\prime\prime}\circ g}{f^\prime\circ g} + \frac{g^{\prime\prime\prime}}{g^\prime} - \frac{3}{2}\biggl(g^\prime\frac{f^{\prime\prime}\circ g}{f^\prime \circ g} + \frac{g^{\prime\prime}}{g^\prime}\biggr)^2\\
    &= g^\prime{}^2\biggl[\frac{f^{\prime\prime\prime}\circ g}{f^\prime \circ g}-\frac{3}{2}\biggl(\frac{f^{\prime\prime}\circ g}{f^\prime\circ g}\biggr)^2\biggr] + \biggl[\frac{g^{\prime\prime\prime}}{g^\prime} - \frac{3}{2}\biggl(\frac{g^\prime\prime}{g^\prime}\biggr)^2\biggr] = g^{\prime}{}^2\cdot [\mathscr Sf\circ g] + \mathscr Sg.
\end{align*}
Let $G$ be a group and $(V,\rho)$ a representation of $G$ (here $\rho$ is the homomorphism $G\to \GL(V)$). In this setting, a map $c\colon G\to V$ is called a $1$-cocycle on $G$ with coefficients in $V$ if $c(gh) = \rho(g)c(h) + c(g)$ for all $g,h\in G$. A $1$-cocycle is called a coboundary if $c(g) = \rho(g)v - v$ for some $v\in V$. The cocycles form an Abelian group with the coboundaries an Abelian subgroup; we denote by $H^1(G,V)$ the quotient group of $1$-cocycles by coboundaries.

The group $\Diff(\mathbb{RP}^1)$of diffeomorphisms of $\mathbb{RP}^1$ naturally acts on the space $\mathcal F_2(\mathbb{RP}^1)$ of quadratic differentials on $\mathbb{RP}^1$ by the assignment $f\mapsto T^2_{f^{-1}}$ where $T^2_{f^{-1}}\phi = f^\prime{}^2\cdot f^\ast\phi$ for $f$ a diffeomorphism of $\mathbb{RP}^1$ and $\phi$ a quadratic differential. From \cref{eq: chain} we deduce that the Schwarzian derivative $\mathscr S\colon \Diff(\mathbb{RP}^1)\to \mathcal F_2(\mathbb{RP}^1)$ is a $1$-cocycle. It is not a coboundary; it turns out that coboundaries in this setting will depend only on the $1$-jet of a diffeomorphism, which is not the case of the Schwarzian derivative. A further result says that $H^1(\Diff(\mathbb{RP}^1), \mathcal F_2(\mathbb{RP}^1)) \cong \mathbb R$, and so the Schwarzian derivative is a representative of any $1$-cocycle.

A central extension of a Lie algebra $\mathfrak g$ is a Lie algebra structure on the space $\mathfrak g\oplus \mathbb R$ given by the commutator
\[[(X,a), (Y,b)] = ([X,Y], c(X,Y))\]
where $X,Y\in \mathfrak g$, $a,b\in\mathbb R$, and $c\colon \mathfrak g \to \mathbb R$ is a $2$-cocycle (in this setting $c$ is just a skew-symmetric bilinear map). The Virasoro algebra $\Vir$ is obtained by a central extension of the space $\Vect(S^1)$ of vector fields on $S^1$ (equivalently $\mathbb{RP}^1$) by the Gelfand-Fuchs cocycle
\[c\biggl(g(x)\dv{x}, h(x)\dv{x}\biggr) = \int_{S^1}g^\prime(x)h^{\prime\prime}(x)\dd x = \int_{S^1}g'''(x)h(x)\dd x.\] The Schwarzian derivative tells us everything we need to know about the Gelfand-Fuchs cocycle; for example, that the above formula defines a $2$-cocycle follows from the chain rule of the Schwarzian derivative.

\newpage\section*{References}

Certain cases in which the vanishing of the Wronskian is a sufficient condition for linear dependence by Maxime B\^ocher, in Trans.~Amer.~Math.~Soc.~2 (1901), 139-149. (\href{https://www.ams.org/journals/tran/1901-002-02/S0002-9947-1901-1500560-5/S0002-9947-1901-1500560-5.pdf}{link to PDF})

Complex Variables: Introduction and Applications by Mark J.~Ablowitz and Athanassios S.~Fokas, Cambridge University Press (2003), 383. (\href{https://ftfsite.ru/wp-content/files/tfkp_endlish_2.2.pdf}{link to PDF})

Is there an underlying explanation for the magical powers of the Schwarzian derivative? asked by Paul Siegel (2010). (\href{https://mathoverflow.net/questions/38105/is-there-an-underlying-explanation-for-the-magical-powers-of-the-schwarzian-deri}{link to MO})

Projective Differential Geometry Old and New by Valentin Ovsienko and Sergei Tabachnikov, Cambridge University Press (2004), sections 1.3, 1.5. (\href{https://drive.google.com/file/d/1pT7cxcGNZxCzmrkcY65SB-9dfruKkrSY/view?usp=share_link}{link to PDF})

The Schwarzian derivative in one-dimensional dynamics by Ben Cooper, UChicago REU (2020), section 3. (\href{https://math.uchicago.edu/~may/REU2020/REUPapers/Cooper.pdf}{link to PDF})

What is... the Schwarzian Derivative? by Valentin Ovsienko and Sergei Tabachnikov (2009). \href{https://www.ams.org/notices/200901/tx090100034p.pdf}{link to PDF})

Wolfram MathWorld article on the Riccati equation. (\href{https://mathworld.wolfram.com/RiccatiDifferentialEquation.html}{link to webpage})

Zippers and Univalent Functions by William Thurston (1986). (\href{https://people.math.harvard.edu/~ctm/home/text/others/thurston/zippers/zippers.pdf}{link to PDF})

\end{document}