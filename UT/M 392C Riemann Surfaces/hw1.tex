\documentclass[11pt,leqno]{article}
\headheight=13.6pt

% packages
\usepackage[alphabetic]{amsrefs}
\usepackage{physics}
% margin spacing
\usepackage[top=1in, bottom=1in, left=0.5in, right=0.5in]{geometry}
\usepackage{hanging}
\usepackage{amsfonts, amsmath, amssymb, amsthm}
\usepackage{extarrows}
\usepackage[none]{hyphenat}
\usepackage{fancyhdr}
\usepackage{graphicx}
\graphicspath{{./images/}}
\usepackage{float}
\usepackage{color}
\newcommand{\sai}[1]{\textcolor{red}{#1}}
\usepackage{enumitem}
% \usepackage{mathrsfs}
\usepackage{hyperref}
\usepackage[noabbrev, capitalise]{cleveref}
\crefformat{equation}{equation~#2#1#3}
\crefformat{lemma}{\textrm{Lemma}~#2#1#3}
\usepackage{quiver}

% theorems
\theoremstyle{plain}
\newtheorem{lem}{Lemma}
\newtheorem{lemma}[lem]{Lemma}
\newtheorem{thm}[lem]{Theorem}
\newtheorem{theorem}[lem]{Theorem}
\newtheorem{prop}[lem]{Proposition}
\newtheorem{proposition}[lem]{Proposition}
\newtheorem{cor}[lem]{Corollary}
\newtheorem{corollary}[lem]{Corollary}
\newtheorem{conj}[lem]{Conjecture}
\newtheorem{fact}[lem]{Fact}
\newtheorem{form}[lem]{Formula}

\theoremstyle{definition}
\newtheorem{defn}[lem]{Definition}
\newtheorem{definition/}[lem]{Definition}
\newenvironment{definition}
  {\renewcommand{\qedsymbol}{\textdagger}%
   \pushQED{\qed}\begin{definition/}}
  {\popQED\end{definition/}}
\newtheorem{example}[lem]{Example}
\newtheorem{remark}[lem]{Remark}
\newtheorem{exercise}[lem]{Exercise}
\newtheorem{notation}[lem]{Notation}

\numberwithin{equation}{section}
\numberwithin{lem}{section}

% header/footer formatting
\pagestyle{fancy}
\fancyhead{}
\fancyfoot{}
\fancyhead[L]{M 392C}
\fancyhead[C]{HW1}
\fancyhead[R]{Riemann Surfaces}
\fancyfoot[R]{\thepage}
\renewcommand{\headrulewidth}{1pt}

% paragraph indentation/spacing
\setlength{\parindent}{0cm}
\setlength{\parskip}{10pt}
\renewcommand{\baselinestretch}{1.25}

% mathscript S
\usepackage[mathscr]{euscript}

% math operators
\DeclareMathOperator{\Span}{span}

% bracket notation for inner product
\usepackage{mathtools}

\DeclarePairedDelimiterX{\abr}[1]{\langle}{\rangle}{#1}

% smileys frownies
\usepackage{wasysym}
\newcommand{\smallhappy}{\raisebox{-.14em}{\smiley}}
\newcommand{\happy}{\raisebox{-.24em}{\resizebox{1.2em}{!}{\smiley}}}
\newcommand{\smallsad}{\raisebox{-.14em}{\frownie}}
\newcommand{\sad}{\raisebox{-.24em}{\resizebox{1.2em}{!}{\frownie}}}
\DeclareMathOperator{\mathhappy}{\!\happy\!}
\DeclareMathOperator{\smallmathhappy}{\!\smallhappy\!}
\DeclareMathOperator{\mathsad}{\!\sad\!}
\DeclareMathOperator{\smallmathsad}{\!\smallsad\!}

% set page count index to begin from 1
\setcounter{page}{1}

\begin{document}
Sudharshan K.V. (1.1, 1.2) and Sai Sivakumar (1.3, 1.4, 1.5)

\section{The Schwarzian derivative}
\begin{definition}
    The \textit{Schwarzian derivative} of a holomorphic function $f\colon \mathbb C\to \mathbb C$ is
    \[\mathscr Sf = \frac{f^{\prime\prime\prime}}{f^\prime}-\frac{3}{2}\biggl(\frac{f^{\prime\prime}}{f^\prime}\biggr)^2\qedhere\]
\end{definition}
\subsection{M\"obius maps}
The Schwarzian derivative measures a function's deviation from being a M\"obius map.
\subsection{Jets}
\subsection{Sturm-Liouville theory}
Given two linearly independent holomorphic functions $v,w$ on an open set of $\mathbb C$, then $v,w$ satisfy the Sturm-Liouville equation
\[g^{\prime\prime} + \frac{1}{2}\mathscr S\biggl(\frac{v}{w}\biggr)g = 0;\]
that is, the solution space of this Sturm-Liouville equation is $\Span_{\mathbb C}\{v,w\}$.

Let $f = v/w$. By calculating derivatives of $v = wf$, we obtain 
\begin{align*}
    v^\prime &= w^\prime f + wf^\prime\\
    v^{\prime\prime} &= w^{\prime\prime}f + 2w^\prime f^\prime + wf^{\prime\prime}\\
    v^{\prime\prime\prime} &= w^{\prime\prime\prime}f + 3w^{\prime\prime}f^\prime + 3w^\prime f^{\prime\prime} + wf^{\prime\prime\prime}.
\end{align*}
Then 
\begin{align*}
    \mathscr Sf &= \frac{f^{\prime\prime\prime}}{f^\prime}-\frac{3}{2}\biggl(\frac{f^{\prime\prime}}{f^\prime}\biggr)^2 = \frac{wf^{\prime\prime\prime}}{wf^\prime}-\frac{3}{2}\biggl(\frac{wf^{\prime\prime}}{wf^\prime}\biggr)^2\\
    &= \frac{v^{\prime\prime\prime}-w^{\prime\prime\prime}f-3w^{\prime\prime}f-3w^\prime f^{\prime\prime}}{v^\prime-w^\prime f} - \frac{3({v^{\prime\prime}}^2 + {w^{\prime\prime}}^2f^2 + 4{w^{\prime}}^2{f^{\prime}}^2 -2v^{\prime\prime}w^{\prime\prime}f -4v^{\prime\prime}w^\prime f^\prime +4w^\prime w^{\prime\prime}ff^\prime)}{2({v^\prime}^2 + {w^\prime}^2 - 2v^\prime w^\prime f)}\\
    &= 
\end{align*}
\subsection{Quadratic differentials}
\subsection{Cocycles and central extensions}

\newpage\section*{References}

\end{document}