\documentclass[11pt,leqno]{article}
\headheight=13.6pt

% packages
\usepackage[alphabetic]{amsrefs}
\usepackage{physics}
% margin spacing
\usepackage[top=1in, bottom=1in, left=0.5in, right=0.5in]{geometry}
\usepackage{amsfonts, amsmath, amssymb, amsthm}
\usepackage{extarrows}
\usepackage[none]{hyphenat}
\usepackage{fancyhdr}
\usepackage{graphicx}
\graphicspath{{./images/}}
\usepackage{float}
\usepackage{color}
\newcommand{\sai}[1]{\textcolor{red}{#1}}
\usepackage{enumitem}
% \usepackage{mathrsfs}
\usepackage{hyperref}
\usepackage[noabbrev, capitalise]{cleveref}
\crefformat{equation}{equation~#2#1#3}
\crefformat{lemma}{\textrm{Lemma}~#2#1#3}
\usepackage{quiver}

% theorems
\theoremstyle{plain}
\newtheorem{lem}{Lemma}
\newtheorem{lemma}[lem]{Lemma}
\newtheorem{thm}[lem]{Theorem}
\newtheorem{theorem}[lem]{Theorem}
\newtheorem{prop}[lem]{Proposition}
\newtheorem{proposition}[lem]{Proposition}
\newtheorem{cor}[lem]{Corollary}
\newtheorem{corollary}[lem]{Corollary}
\newtheorem{conj}[lem]{Conjecture}
\newtheorem{fact}[lem]{Fact}
\newtheorem{form}[lem]{Formula}

\theoremstyle{definition}
\newtheorem{defn}[lem]{Definition}
\newtheorem{definition/}[lem]{Definition}
\newenvironment{definition}
  {\renewcommand{\qedsymbol}{\textdagger}%
   \pushQED{\qed}\begin{definition/}}
  {\popQED\end{definition/}}
\newtheorem{example}[lem]{Example}
\newtheorem{remark}[lem]{Remark}
\newtheorem{exercise}[lem]{Exercise}
\newtheorem{notation}[lem]{Notation}

\numberwithin{equation}{section}
\numberwithin{lem}{section}

% header/footer formatting
\pagestyle{fancy}
\fancyhead{}
\fancyfoot{}
\fancyhead[L]{M 392C}
\fancyhead[C]{HW2}
\fancyhead[R]{Riemann Surfaces}
\fancyfoot[R]{\thepage}
\renewcommand{\headrulewidth}{1pt}

% paragraph indentation/spacing
\setlength{\parindent}{0cm}
\setlength{\parskip}{10pt}
\renewcommand{\baselinestretch}{1.25}

% mathscript S
\usepackage[mathscr]{euscript}

% math operators
\DeclareMathOperator{\Span}{span}
\DeclareMathOperator{\Sym}{Sym}
\renewcommand{\ev}{\mathrm{ev}}
\DeclareMathOperator{\SL}{SL}
\newcommand{\smod}[1]{\;(\bmod\; #1)}
\DeclareMathOperator{\PSL}{PSL}
\DeclareMathOperator{\GL}{GL}
\newcommand{\bbC}{\mathbb C}
\DeclareMathOperator{\Diff}{Diff}
\newcommand{\Vir}{\mathrm{Vir}}
\DeclareMathOperator{\ad}{ad}
\DeclareMathOperator{\Vect}{Vect}
\DeclareMathOperator{\Div}{Div}
\renewcommand{\div}{\text{div}}
% \DeclareMathOperator{\div}{div}

% bracket notation for inner product
\usepackage{mathtools}

\DeclarePairedDelimiterX{\abr}[1]{\langle}{\rangle}{#1}

% smileys frownies
\usepackage{wasysym}
\newcommand{\smallhappy}{\raisebox{-.14em}{\smiley}}
\newcommand{\happy}{\raisebox{-.24em}{\resizebox{1.2em}{!}{\smiley}}}
\newcommand{\smallsad}{\raisebox{-.14em}{\frownie}}
\newcommand{\sad}{\raisebox{-.24em}{\resizebox{1.2em}{!}{\frownie}}}
\DeclareMathOperator{\mathhappy}{\!\happy\!}
\DeclareMathOperator{\smallmathhappy}{\!\smallhappy\!}
\DeclareMathOperator{\mathsad}{\!\sad\!}
\DeclareMathOperator{\smallmathsad}{\!\smallsad\!}

% set page count index to begin from 1
\setcounter{page}{1}

% array column and row separation
\arraycolsep = 2pt
\renewcommand{\arraystretch}{.6}

\begin{document}
Sai Sivakumar

\section{Modular curves and modular forms}
We fix some conventions and notation via the following definitions.
\begin{definition}
    The modular group $\Gamma$ is $\SL_2(\mathbb Z)$. (It is often the case that the modular group is taken to be $\PSL_2(\mathbb Z) = \SL_2(\mathbb Z)/\{\pm I\}$, but the difference will not matter here.)
\end{definition}
It is well known that the matrices $T = \bigl(\!\begin{smallmatrix}
    1 & 1 \\ 0 & 1
\end{smallmatrix}\!\bigr)$ and $S = \bigl(\!\begin{smallmatrix}
    0 & -1 \\ 1 & 0
\end{smallmatrix}\!\bigr)$ generate $\Gamma$ as a group, and that $\Gamma$ acts on the Riemann sphere $\widehat{\mathbb C}$ by fractional linear transformations. In one coordinate patch, $\widehat{\mathbb C}$ is $\mathbb C\cup \{\infty\}$; the action of $\Gamma$ in these coordinates manifests as 
\[\begin{pmatrix}
    a & b \\ c & d
\end{pmatrix}(z) = \frac{az+b}{cz+d}.\]
For $c$ nonzero, $-d/c$ is sent to $\infty$ and $\infty$ is sent to $a/c$; if $c = 0$ then $\infty$ is sent to $\infty$. We will also look at the action of particular subgroups of $\Gamma$ by fractional linear transformations.
\begin{definition}
    For any positive integer $N$, the principal congruence subgroup of level $N$ is 
    \[\Gamma(N) = \{\gamma\in \SL_2(\mathbb Z)\mid \gamma\equiv I \smod N\}\]
    where matrix congruence is to be taken entrywise.
\end{definition} Since $\Gamma(N)$ is the kernel of the natural map $\Gamma = \SL_2(\mathbb Z)\to\SL_2(\mathbb Z/N\mathbb Z)$, $\Gamma(N)$ is normal in $\Gamma$.

The modular group and its subgroups (of which we will only consider $\Gamma$ and $\Gamma(N)$ in this homework) act on the upper half plane $\mathbb H = \{z\in\mathbb C\mid \Im(z)>0\}$ by fractional linear transformations. Denote by $Y$ the quotient space $\mathbb H/\Gamma$; that is, $Y$ is obtained by identifying the orbits of points in $\mathbb H$ under the action of $\Gamma$. Similarly, denote by $Y(N)$ the quotient space $\mathbb H/\Gamma(N)$. Since the action of $\Gamma$ on $\mathbb H$ is properly discontinuous and the maps $\mathbb H\to Y$, $\mathbb H\to Y(N)$ are quotient maps of topological spaces, it is possible to give holomorphic coordinates on $Y$ and $Y(N)$ in a way that would make them Riemann surfaces. These charts are the usual ones for most points of $Y$ or $Y(N)$, but there are finitely many points for which the charts are slightly annoying.

For most points in $\mathbb H$, their stabilizer under the action of $\Gamma$ or $\Gamma(N)$ is $\{\pm I\}$, but there are finitely many points for which the stabilizer is nontrivial; specifically finite cyclic. These points are called elliptic points. Passing to $Y$ or $Y(N)$, a chart can be chosen around an elliptic curve for which the action of the generator of the stabilizer group $G$ for that point in coordinates is multiplication by $\exp(2\pi i/h)$ for some positive integer $h$. The integer $h$ is the ``period'' of the corresponding elliptic point $p$, given by $\abs{\{\pm I\}G/\{\pm I\}}$.

We can complete $Y$ and $Y(N)$ to compact Riemann surfaces by adding in cusps, or ``points at infinity''. Concretely, cusps are the orbits of points in $\mathbb Q\cup\{\infty\}$. Representatives for these orbits need to be added to $\mathbb H$ in a way so that taking the quotient by $\Gamma$, $\Gamma(N)$ yields the compactifications of $Y$, $Y(N)$. The charts around these cusps are not so annoying but come from disks either around $\infty$ or are disks tangent to the real axis. 
\begin{definition}
    Denote by $X$, $X(N)$ the compactifications of the Riemann surfaces $Y$, $Y(N)$ respectively.
\end{definition}
Modular forms can be defined from different points of view, but the most basic definition is the following.
\begin{definition}
    For $\gamma = \bigl(\!\begin{smallmatrix}
    a & b \\ c & d
\end{smallmatrix}\!\bigr)\in \SL_2(\mathbb Z)$, the factor of automorphy is $j(\gamma,z) = cz+d$ for $z\in\mathbb C$.
\end{definition}
\begin{definition}
    For $G = \Gamma$ or $G = \Gamma(N)$, and an integer $k$, a modular form of weight $k$ with respect to $G$ is a holomorphic function $f\colon \mathbb H\to \mathbb C$ for which 
    \[f(z) = j(\gamma,z)^{-k}f(\gamma(z))\]
    for any $\gamma\in G$ and $z\in\mathbb H$. Furthermore $f$ should satisfy the property that for any $\alpha\in\SL_2(\mathbb Z)$, $j(\alpha,z)^{-k}f(\alpha(z))$ is holomorphic at $\infty$. This last condition is just saying that $f$ is holomorphic at the cusps for $G$.
\end{definition}
We will consider modular forms of even weight, that is, of weight $2k$.
\subsection{Modular forms as meromorphic differentials}
% Explain precisely how a modular form of weight 2k for the group Γ(N) ⊂ Γ = PSL_2(Z) can be interpreted as a meromorphic k-fold differential on the curve X(N). You will need to take care over the elliptic points and cusps.
A weakly modular function (of weight $2k$) is a meromorphic function $f\colon \mathbb H\to \mathbb C$ that transforms with respect to a congruence subgroup like $G = \Gamma$ or $G=\Gamma(N)$ in the same way as the definition of a modular form; that is, for any $\gamma\in G$ and $z\in\mathbb H$ we should have $f(z) = j(\gamma,z)^{-2k}f(\gamma(z))$ (i.e., we do not insist on any holomorphy of any kind). Let $G = \Gamma(N)$ from now on. A weakly modular function (of weight $2k$) for $\Gamma(N)$ can be interpreted as a meromorphic differential on $X(N)$ in the following way: 

We can take a weakly modular function $f$ and map it to the differential form $f(\dd z)^{k}$ on $\mathbb H$. This differential form is genuinely $\Gamma(N)$-invariant, since for any $\gamma = \bigl(\!\begin{smallmatrix}
    a & b \\ c & d
\end{smallmatrix}\!\bigr)\in\Gamma(N)$ we have 
\[\dd(\gamma(z)) = \dd\Bigl(\bigl(\!\begin{smallmatrix}
    a & b \\ c & d
\end{smallmatrix}\!\bigr)(z)\Bigr)= \dv{z}\frac{az+b}{cz+d}\dd z  = \frac{1}{(cz+d)^2}\dd z = j(\gamma,z)^{-2}\dd z\]
(since $\det\gamma = 1$). Thus $f(\gamma(z))(\dd(\gamma(z)))^k = j(\gamma,z)^{-2k}f(z)j(\gamma,z)^{2k}(\dd z)^k = f(z)(\dd z)^k$. What remains is to see that we can find meromorphic differentials on $X(N)$ that make sense and which pull back to differential forms of the kind $f(\dd z)^k$ for $f$ weakly modular with respect to $\Gamma(N)$.

\begin{definition}
    The meromorphic differentials of degree $n$ on an open set $U\subseteq\mathbb C$ are 
    \[\Omega^n(U) = \{f(z)(\dd z)^n\mid f\text{ meromorphic on $U$}\}.\qedhere\]
\end{definition}
If $\varphi\colon U\to V$ is a holomorphic map of open sets in $\mathbb C$, then the pullback map $\varphi^\ast\colon \colon\Omega^n(V)\to \Omega^n(U)$ is given by $f(\dd z)^n\mapsto (f\circ\varphi)(\varphi^\prime)^n(\dd w)^n$ (where $z$ is the coordinate for $V$ and $w$ is the coordinate for $U$).
%  A nice fact is that if $\pi\colon U\to V$ is a holomorphic surjection of open sets in $\mathbb C$, then the pullback $\pi^\ast\colon\Omega^n(V)\to \Omega^n(U)$ is an injection. 

Since $X(N)$ is a (compact) Riemann surface, (finitely many) charts may be used to cover it. Each chart in $X(N)$ is homeomorphic to an open set of $\mathbb C$, so we obtain the following definition: 
\begin{definition}
    Let $U_i$ be charts covering $X(N)$ which are homeomorphic via $\varphi_i$ to open sets $V_i$ in $\mathbb C$. Then a meromorphic differential on $X(N)$ is given by the data of meromorphic differentials $f_i(\dd z)^n$ on each $V_i$ which glue together. That is, for any two charts $V_i$ and $V_j$ if we restrict $f_i(\dd z)^n$ to $\varphi_i(U_i\cap U_j)$ and $f_j(\dd z)^n$ to $\varphi_j(U_i\cap U_j)$, then pulling back the restricted $f_i(\dd z)^n$ along the transition map between $V_j$ and $V_i$ yields equality with the restricted $f_j(\dd z)^n$.
\end{definition}
To go from a meromorphic differential on $X(N)$ to a meromorphic differential on $\mathbb H$, we need to pull back along the quotient map $\pi\colon \mathbb H\to X(N)$. To give this map explicitly we start with a known set of charts for $X(N)$ and pull back on each one of those charts. On $X(N)$, these charts are mostly the ones given by taking disks in $\mathbb H$ and quotienting by the group action of $\Gamma(N)$, but for the elliptic points and the cusps some modifications are made. We will not go through the calculations for these pullback maps but the point is that because the collection of forms on the charts covering $X(N)$ glue together, they will still glue together when pulled back to forms on $\mathbb H$.

The last thing is to show that the form on $\mathbb H$ is $\Gamma(N)$-invariant. But this is almost tautological because such a form $f(z)(\dd z)^n$ is the pullpack of a form on $X(N)$. That is, for any $\gamma\in \Gamma(N)$, we can pull back $f(z)(\dd z)^n$ along the action of $\gamma$ on $\mathbb H$ to obtain equalities $f(z)(\dd z)^n = \gamma^\ast(f(z)(\dd z)^n) = f(\gamma(z))(\gamma^\prime(z))^n(\dd z)^n = j(\gamma, z)^{-2n}f(\gamma(z))(\dd z)^n$. This shows that the function $f$ is weakly modular with weight $2n$. To see that $f$ is meromorphic at cusps, it suffices to see that the meromorphic differential it comes from expanded locally around the cusps of $X(N)$ is given by a function which is meromorphic at zero, so $f$ should also be meromorphic at its cusps.

Now we should go in the other direction. Given a weakly modular meromorphic function $f$ of weight $2k$, we should try to obtain a meromorphic differential on $X(N)$ of degree $k$. We just need to find differentials on any chart of $X(N)$ which glue together and pull back to $f(\dd z)^k$ along the quotient map $\mathbb H\to X(N)$. To do this we can start by finding local coordinate expressions for $f(\dd z)^k$.

The charts $U_j$ on $\mathbb H$ are given local coordinates by composing two maps. The first map $\delta_j$ is

\subsection{Riemann-Roch and dimension formulas}
% Use Riemann–Roch to compute a dimension formula for the space M2k(Γ(N)) of modular forms, and for its subspace M^0_2k(Γ(N)) of cusp forms (modular forms vanishing at the cusps).
Due to the elliptic points coming from $\Gamma(N)$, meromorphic modular functions of weight $2k$ may have orders that are fractional and not just integers. So to start we will consider the $\mathbb Q$-vector space of divisors $\Div_{\mathbb Q}(X(N))$ on $X(N)$ and carefully reduce calculations to ones involving integer coefficients so that we can use the Riemann-Roch theorem.

Let $f$ be any nonzero meromorphic modular function of weight $2k$. Since the quotient of two meromorphic modular functions of the same weight is weight zero; that is, just a meromorphic function, it follows that any meromorphic modular function of weight $2k$ may be obtained by multiplying $f$ by a suitable meromorphic function on $X(N)$ (in notation an element of $\mathbb C(X(N))$; which of course pulls back to a meromorphic modular function on $\mathbb H$). The space of modular forms $M_{2k}(\Gamma(N))$ with respect to $\Gamma(N)$ may be described as the elements $f_0f$ with $f_0\in \mathbb C(X(N))$ such that either $f_0f = 0$ or $\div(f_0f) \geq 0$. Since $f$ is nonzero; the space of modular forms $M_{2k}(\Gamma(N))$ is isomorphic to the vector space $\{f_0\in \mathbb C(X(N))\mid \div(f_0) + \div(f)\geq 0\}$. Since $f_0$ is meromorphic on $X(N)$, $\div(f_0)$ is integral, so the condition $\div(f_0) + \div(f)\geq 0$ is equivalent to the condition $\div(f_0) + \lfloor\div(f)\rfloor\geq 0$ (where the floor function on divisors means to apply it to the coefficients). It follows that $M_{2k}(\Gamma(N))$ is isomorphic to $L(\lfloor\div(f)\rfloor)$, so their dimensions are the same. Let $\omega$ be the meromorphic differential on $X(N)$ of degree $k$ which pulls back to $f(\dd z)^{k}$. Identifying $\omega$ with $f(\dd z)^{k}$, it is the case that (here the divisor of a meromorphic differential is calculated by expanding locally and summing the divisors of the local expansions) $\div(\omega) = \div(f) + (2k/2)(-\sum_{\text{period 2 elliptic points}}(1/2)a_i -\sum_{\text{period 3 elliptic points}}(2/3)b_i - \sum_{\text{cusps from $\Gamma(N)$}}c_i)$. The left hand side has integer coefficients, so by taking the floor and then the degree on both sides we obtain the equation
\[\deg(\lfloor\div(f)\rfloor) = \deg(\div(\omega)) + \lfloor(2k/4)\rfloor \varepsilon_2 + \lfloor(2k/3)\rfloor\varepsilon_3 + (2k/2)\varepsilon_\infty,\] where $\varepsilon_2,\varepsilon_3,\varepsilon_\infty$ are the number of period $2$, $3$ elliptic points and cusps respectively. Note further that since $\div(\omega)$ is a $2k$ power of a canonical divisor, its degree is $2k(g-1)$ with $g$ the genus of $X(N)$. Then with $k\geq 1$, simple estimates give $\deg(\lfloor\deg(f)\rfloor)>2g-2$.

The Riemann-Roch theorem simplifies to the following expression since the degree of $\deg(\lfloor\deg(f)\rfloor)$ is greater than $2g-2$: 
\[l(\lfloor\deg(f)\rfloor) = (2k-1)(g-1) + \lfloor(k/2)\rfloor\varepsilon_2 + \lfloor(2k/3)\rfloor\varepsilon_3 + k\varepsilon_\infty;\] this expression is hence equal to the dimension of $M_{2k}(\Gamma(N))$.

To obtain the dimension formula for the space of cusp forms $M^0_{2k}(\Gamma(N))$, observe that since cusp forms vanish at cusps, if $f$ is a weakly meromorphic modular function on $\mathbb H$ then $M^0_{2k}(\Gamma(N))\cong \{f_0\in \mathbb C(X(N))\mid \div(f_0) + \lfloor\div(f) - \sum_{\text{cusps}}c_i\rfloor\geq 0\}$; i.e. $M^0_{2k}(\Gamma(N))\cong L(\lfloor\div(f) - \sum_{\text{cusps}}c_i\rfloor)$. Repeat similar estimates as above (and here we need $k\geq 2$ instead of $k\geq 1$ to use the version of Riemann-Roch as above) and use the Riemann-Roch theorem to find that for $k\geq 2$, $\dim(M^0_{2k}(\Gamma(N))) = l(\lfloor\div(f)\rfloor) - \varepsilon_\infty = \dim(M_{2k}(\Gamma(N)))-\varepsilon_\infty$. For $k = 1$, the divisor $\lfloor\div(f) - \sum_{\text{cusps}}c_i\rfloor$ is a canonical divisor, so its linear space (which is isomorphic to $M^0_2(\Gamma(N))$) has dimension $g$.

Observe that modular forms of weight zero correspond to holomorphic functions $X(N)\to\widehat{\mathbb C}$; since $X(N)$ and $\widehat{\mathbb C}$ are compact, such maps are either surjective or constant. Since these maps have no poles they could not be surjective, so they are constants; that is, $M_0(\Gamma(N))\cong \mathbb C$ is the space of constant functions. This implies that $M^0_0(\Gamma(N)) = 0$.

\subsection{Eisenstein series and cusp forms}
% In the case of the full modular group Γ, show that the graded algebra M = ⊕ k∈Z M_2k(Γ) is the polynomial algebra C[G4, G6] in the Eisenstein series G4 and G3 (which have degrees 2k = 4, 6 respectively). Show that the cusp forms constitute the ideal ∆ · C[G4, G6] generated by by the modular discriminant ∆ ∈ M^0_12(Γ).
Observe that $\Gamma(1) = \Gamma$, that $X(1)\cong\mathbb{CP}^1$ is genus $0$ and has one elliptic point with period $3$, one elliptic point with period $2$ and one cusp, so the dimension formulas from the previous section become
\[\dim(M_0(\Gamma)) = 1,\quad \dim(M_2(\Gamma)) = 0, \quad\dim(M_4(\Gamma)) = 1, \quad\dim(M_6(\Gamma)) = 1\]
\[\dim(M^0_{12}(\Gamma)) = 1,\quad \text{all other spaces have dimension $0$}\]

\newpage\section*{References}

The course notes.

A First Course in Modular Forms by Fred Diamond and Jerry Shurman, in Graduate Texts in Mathematics. (\href{https://link.springer.com/book/10.1007/978-0-387-27226-9}{link to book})

A Course in Arithmetic by Jean-Pierre Serre, in Graduate Texts in Mathematics. (\href{https://link.springer.com/book/10.1007/978-1-4684-9884-4}{link to book})

\end{document}