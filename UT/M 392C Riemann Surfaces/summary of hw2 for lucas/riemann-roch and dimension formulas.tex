\documentclass[11pt,leqno]{article}
\headheight=13.6pt

% packages
\usepackage[alphabetic]{amsrefs}
\usepackage{physics}
% margin spacing
\usepackage[top=1in, bottom=1in, left=0.5in, right=0.5in]{geometry}
\usepackage{amsfonts, amsmath, amssymb, amsthm}
\usepackage{extarrows}
\usepackage[none]{hyphenat}
\usepackage{fancyhdr}
\usepackage{graphicx}
\graphicspath{{./images/}}
\usepackage{float}
\usepackage{color}
\newcommand{\sai}[1]{\textcolor{red}{#1}}
\usepackage{enumitem}
% \usepackage{mathrsfs}
\usepackage{hyperref}
\usepackage[noabbrev, capitalise]{cleveref}
\crefformat{equation}{equation~#2#1#3}
\crefformat{lemma}{\textrm{Lemma}~#2#1#3}
\usepackage{quiver}

% theorems
\theoremstyle{plain}
\newtheorem{lem}{Lemma}
\newtheorem{lemma}[lem]{Lemma}
\newtheorem{thm}[lem]{Theorem}
\newtheorem{theorem}[lem]{Theorem}
\newtheorem{prop}[lem]{Proposition}
\newtheorem{proposition}[lem]{Proposition}
\newtheorem{cor}[lem]{Corollary}
\newtheorem{corollary}[lem]{Corollary}
\newtheorem{conj}[lem]{Conjecture}
\newtheorem{fact}[lem]{Fact}
\newtheorem{form}[lem]{Formula}

\theoremstyle{definition}
\newtheorem{defn}[lem]{Definition}
\newtheorem{definition/}[lem]{Definition}
\newenvironment{definition}
  {\renewcommand{\qedsymbol}{\textdagger}%
   \pushQED{\qed}\begin{definition/}}
  {\popQED\end{definition/}}
\newtheorem{example}[lem]{Example}
\newtheorem{remark}[lem]{Remark}
\newtheorem{exercise}[lem]{Exercise}
\newtheorem{notation}[lem]{Notation}

\numberwithin{equation}{section}
\numberwithin{lem}{section}

% header/footer formatting
\pagestyle{fancy}
\fancyhead{}
\fancyfoot{}
\fancyhead[L]{Riemann-Roch and dimension formulas}
\fancyhead[C]{}
\fancyhead[R]{Sai Sivakumar}
\fancyfoot[R]{\thepage}
\renewcommand{\headrulewidth}{1pt}

% paragraph indentation/spacing
\setlength{\parindent}{0cm}
\setlength{\parskip}{10pt}
\renewcommand{\baselinestretch}{1.25}

% mathscript S
\usepackage[mathscr]{euscript}

% math operators
\DeclareMathOperator{\Span}{span}
\DeclareMathOperator{\Sym}{Sym}
\renewcommand{\ev}{\mathrm{ev}}
\DeclareMathOperator{\SL}{SL}
\newcommand{\smod}[1]{\;(\bmod\; #1)}
\DeclareMathOperator{\PSL}{PSL}
\DeclareMathOperator{\GL}{GL}
\newcommand{\bbC}{\mathbb C}
\DeclareMathOperator{\Diff}{Diff}
\newcommand{\Vir}{\mathrm{Vir}}
\DeclareMathOperator{\ad}{ad}
\DeclareMathOperator{\Vect}{Vect}
\DeclareMathOperator{\Div}{Div}
\renewcommand{\div}{\text{div}}
\DeclareMathOperator{\Stab}{Stab}

% bracket notation for inner product
\usepackage{mathtools}

\DeclarePairedDelimiterX{\abr}[1]{\langle}{\rangle}{#1}

% smileys frownies
\usepackage{wasysym}
\newcommand{\smallhappy}{\raisebox{-.14em}{\smiley}}
\newcommand{\happy}{\raisebox{-.24em}{\resizebox{1.2em}{!}{\smiley}}}
\newcommand{\smallsad}{\raisebox{-.14em}{\frownie}}
\newcommand{\sad}{\raisebox{-.24em}{\resizebox{1.2em}{!}{\frownie}}}
\DeclareMathOperator{\mathhappy}{\!\happy\!}
\DeclareMathOperator{\smallmathhappy}{\!\smallhappy\!}
\DeclareMathOperator{\mathsad}{\!\sad\!}
\DeclareMathOperator{\smallmathsad}{\!\smallsad\!}

% set page count index to begin from 1
\setcounter{page}{1}

% array column and row separation
\arraycolsep = 2pt
\renewcommand{\arraystretch}{.6}

\begin{document}

\section{Conventions, notation, and definitions}

\begin{definition}
    For $\gamma = \bigl(\!\begin{smallmatrix}
    a & b \\ c & d
\end{smallmatrix}\!\bigr)\in \SL_2(\mathbb Z)$, the factor of automorphy is $j(\gamma,z) = cz+d$ for $z\in\mathbb C$.
\end{definition}

So a modular form of weight $k$ with respect to a congruence subgroup $\Gamma$ is a holomorphic function $f\colon \mathbb H\to \mathbb C$ for which 
    \[f(\gamma(z))=j(\gamma,z)^kf(z)\]
    for any $\gamma\in G$ and $z\in\mathbb H$. Furthermore, for any $\alpha\in\SL_2(\mathbb Z)$, $j(\alpha,z)^{-k}f(\alpha(z))$ is holomorphic at $\infty$. This last condition is just saying that $f$ is holomorphic at the cusps for $\Gamma$.

A weakly modular function is a meromorphic function $f\colon \mathbb H\to \mathbb C$ that transforms with respect to a congruence subgroup in the same way as in the definition of a modular form. That is, for any $\gamma\in \Gamma$ and $z\in\mathbb H$ we should have $f(z) = j(\gamma,z)^{-2k}f(\gamma(z))$. Note that we do not insist that $f$ be holomorphic on $\mathbb H$ or at cusps of $\Gamma$.

Let $z$ be a coordinate of $\mathbb H\subseteq \mathbb C$. The meromorphic differentials on $\mathbb H$ are meromorphic sections of the cotangent bundle $\Omega$ of $\mathbb H$, and a meromorphic section of $\Omega$ is a section $\omega = f(z)\dd z\in \Omega(U)$ for some open set $U\subseteq \mathbb H$ where $f(z)$ is meromorphic on $\mathbb H$ and $\mathbb H\setminus U$ contains the poles of $f(z)$. In other words, the meromorphic differentials on $\mathbb H$ are sections of the form $f(z)\dd z$ where $f(z)$ is meromorphic on $\mathbb H$.

The meromorphic differentials of degree $n$ on $\mathbb H$ are meromorphic sections of the $n$-th tensor (or symmetric) power of the cotangent bundle of $\mathbb H$, $\Omega^{\otimes n}$. Since $\Omega$ is a trivial line bundle, its tensor powers are also trivial, so sections of $\Omega^{\otimes n}$ are

\end{document}