\documentclass[11pt,leqno]{article}
\headheight=13.6pt

% packages
\usepackage[alphabetic]{amsrefs}
\usepackage{physics}
% margin spacing
\usepackage[top=1in, bottom=1in, left=0.5in, right=0.5in]{geometry}
\usepackage{amsfonts, amsmath, amssymb, amsthm}
\usepackage{extarrows}
\usepackage[none]{hyphenat}
\usepackage{fancyhdr}
\usepackage{graphicx}
\graphicspath{{./images/}}
\usepackage{float}
\usepackage{color}
\newcommand{\sai}[1]{\textcolor{red}{#1}}
\usepackage{enumitem}
% \usepackage{mathrsfs}
\usepackage{hyperref}
\usepackage[noabbrev, capitalise]{cleveref}
\crefformat{equation}{equation~#2#1#3}
\crefformat{lemma}{\textrm{Lemma}~#2#1#3}
\usepackage{quiver}

% theorems
\theoremstyle{plain}
\newtheorem{lem}{Lemma}
\newtheorem{lemma}[lem]{Lemma}
\newtheorem{thm}[lem]{Theorem}
\newtheorem{theorem}[lem]{Theorem}
\newtheorem{prop}[lem]{Proposition}
\newtheorem{proposition}[lem]{Proposition}
\newtheorem{cor}[lem]{Corollary}
\newtheorem{corollary}[lem]{Corollary}
\newtheorem{conj}[lem]{Conjecture}
\newtheorem{fact}[lem]{Fact}
\newtheorem{form}[lem]{Formula}

\theoremstyle{definition}
\newtheorem{defn}[lem]{Definition}
\newtheorem{definition/}[lem]{Definition}
\newenvironment{definition}
  {\renewcommand{\qedsymbol}{\textdagger}%
   \pushQED{\qed}\begin{definition/}}
  {\popQED\end{definition/}}
\newtheorem{example}[lem]{Example}
\newtheorem{remark}[lem]{Remark}
\newtheorem{exercise}[lem]{Exercise}
\newtheorem{notation}[lem]{Notation}

\numberwithin{equation}{section}
\numberwithin{lem}{section}

% header/footer formatting
\pagestyle{fancy}
\fancyhead{}
\fancyfoot{}
\fancyhead[L]{Riemann-Roch and dimension formulas for spaces of modular forms}
\fancyhead[C]{}
\fancyhead[R]{Sai Sivakumar}
\fancyfoot[R]{\thepage}
\renewcommand{\headrulewidth}{1pt}

% paragraph indentation/spacing
\setlength{\parindent}{0cm}
\setlength{\parskip}{10pt}
\renewcommand{\baselinestretch}{1.25}

% mathscript S
\usepackage[mathscr]{euscript}

% math operators
\DeclareMathOperator{\Span}{span}
\DeclareMathOperator{\Sym}{Sym}
\renewcommand{\ev}{\mathrm{ev}}
\DeclareMathOperator{\SL}{SL}
\newcommand{\smod}[1]{\;(\bmod\; #1)}
\DeclareMathOperator{\PSL}{PSL}
\DeclareMathOperator{\GL}{GL}
\newcommand{\bbC}{\mathbb C}
\DeclareMathOperator{\Diff}{Diff}
\newcommand{\Vir}{\mathrm{Vir}}
\DeclareMathOperator{\ad}{ad}
\DeclareMathOperator{\Vect}{Vect}
\DeclareMathOperator{\Div}{Div}
\renewcommand{\div}{\text{div}}
\DeclareMathOperator{\Stab}{Stab}

% bracket notation for inner product
\usepackage{mathtools}

\DeclarePairedDelimiterX{\abr}[1]{\langle}{\rangle}{#1}

% smileys frownies
\usepackage{wasysym}
\newcommand{\smallhappy}{\raisebox{-.14em}{\smiley}}
\newcommand{\happy}{\raisebox{-.24em}{\resizebox{1.2em}{!}{\smiley}}}
\newcommand{\smallsad}{\raisebox{-.14em}{\frownie}}
\newcommand{\sad}{\raisebox{-.24em}{\resizebox{1.2em}{!}{\frownie}}}
\DeclareMathOperator{\mathhappy}{\!\happy\!}
\DeclareMathOperator{\smallmathhappy}{\!\smallhappy\!}
\DeclareMathOperator{\mathsad}{\!\sad\!}
\DeclareMathOperator{\smallmathsad}{\!\smallsad\!}

% set page count index to begin from 1
\setcounter{page}{1}

% array column and row separation
\arraycolsep = 2pt
\renewcommand{\arraystretch}{.6}

\begin{document}

\section{Conventions, notation, and definitions}

For $\gamma = \bigl(\!\begin{smallmatrix}
    a & b \\ c & d
\end{smallmatrix}\!\bigr)\in \SL_2(\mathbb Z)$, the factor of automorphy is $j(\gamma,z) = cz+d$ for $z\in\mathbb C$.

A modular form of weight $k$ with respect to a congruence subgroup $\Gamma$ is a holomorphic function $f\colon \mathbb H\to \mathbb C$ for which 
\[f(\gamma(z))=j(\gamma,z)^kf(z)\]
for any $\gamma\in G$ and $z\in\mathbb H$. Furthermore, we require that $j(\alpha,z)^{-k}f(\alpha(z))$ is holomorphic at $\infty$ for any $\alpha\in\SL_2(\mathbb Z)$ (meaning the Fourier coefficients of this function are concentrated in indices greater than or equal to zero). This last condition captures the condition that $f$ be holomorphic at the cusps for $\Gamma$.

A weakly modular function (of weight $k$) is a meromorphic function $f\colon \mathbb H\to \mathbb C$ which is meromorphic at the cusps of $\Gamma$ and transforms with respect to a congruence subgroup in the same way as in the definition of a modular form. That is, for any $\gamma\in \Gamma$ and $z\in\mathbb H$ we should have $f(\gamma(z))=j(\gamma,z)^kf(z)$.

Let $z$ be a coordinate of $U\subseteq \mathbb C$. The meromorphic differentials on $U$ are meromorphic (global) sections of the cotangent bundle $\Omega$ of $U$, and a meromorphic section of $\Omega$ is a (holomorphic) section $\omega = f(z)\dd z\in \Omega(V)$ for some open set $V\subseteq U$ where $f(z)$ is meromorphic on $U$ and $U\setminus V$ contains the poles of $f(z)$. In other words, the meromorphic differentials on $U$ are sections of the form $f(z)\dd z$ where $f(z)$ is meromorphic on $U$.

The meromorphic differentials of degree $n$ on $U$ are meromorphic (global) sections of the $n$-th tensor (or symmetric) power of the cotangent bundle of $U$, $\Omega^{\otimes n}$. Since $\Omega$ is a trivial line bundle, its tensor powers are also trivial, so sections of $\Omega^{\otimes n}$ are of the form $f(z)(\dd z)^{\otimes n}$. Since the cotangent bundle is rank one, its tensor powers agree with its symmetric powers, so we suppress the notation $\otimes$ from now on. It follows that meromorphic differentials of degree $n$ on $U$ are of the form $f(z)(\dd z)^n$ for $f(z)$ meromorphic on $U$. Denote the space of meromorphic differentials on $U$ by $\Omega^n(U)$.

If $\varphi\colon U\to V$ is a holomorphic map of open sets in $\mathbb C$, then the pullback map $\varphi^\ast\colon \Omega^n(V)\to \Omega^n(U)$ is given by the change of variables $w = \varphi(z)$:
\[f(w)(\dd w)^n\mapsto f(\varphi(z))(\varphi^\prime(z))^n(\dd z)^n,\]
where $w$ is the coordinate for $V$ and $z$ is the coordinate for $U$.

\newpage
\section{Meromorphic differentials and modular forms}

A weakly modular function $f(z)$ on $\mathbb H$ can be sent to the meromorphic differential $f(z)(\dd z)^n$ on $\mathbb H$. This differential form is $\Gamma$-invariant: for any $\gamma = \bigl(\!\begin{smallmatrix}
    a & b \\ c & d
\end{smallmatrix}\!\bigr)\in\Gamma$ we have 
\[\dd(\gamma(z)) = \dd\Bigl(\bigl(\!\begin{smallmatrix}
    a & b \\ c & d
\end{smallmatrix}\!\bigr)(z)\Bigr)= \dv{z}\biggl(\frac{az+b}{cz+d}\biggr)\dd z  = \frac{1}{(cz+d)^2}\dd z = j(\gamma,z)^{-2}\dd z\]
(since $\det\gamma = 1$). Thus $f(\gamma(z))(\dd(\gamma(z)))^k = j(\gamma,z)^{-2k}f(z)j(\gamma,z)^{2k}(\dd z)^k = f(z)(\dd z)^k$. In other words, this gives a map 
\[\{\text{weakly modular functions }f(z)\text{ of weight $2k$}\}\xrightarrow{(-)(\dd z)^k} \{\Gamma\text{-invariant meromorphic differentials }f(z)(\dd z)^k\text{ on }\mathbb H\}.\]
By reversing the earlier argument, see that the above map is actually an isomorphism of vector spaces.

We will show that meromorphic differentials on $X(\Gamma)$ pull back to meromorphic differentials $f(z)(\dd z)^k$ for $f$ weakly modular with respect to $\Gamma$.

Since $X(\Gamma)$ is a (compact) Riemann surface, (finitely many) charts may be used to cover it. Each chart in $X$ is homeomorphic to an open set of $\mathbb C$, so we obtain the following definition: 


Let $\{(V_i, V_i\xrightarrow{\varphi_i}U_i)\}$ with $V_i\subseteq X(\Gamma)$ and $U_i\subseteq \mathbb C$ open be a (finite) collection of charts for $X(\Gamma)$.
    
A meromorphic differential on $X(\Gamma)$ is given by the data of meromorphic differentials $f_i(z)(\dd z)^n$ on each $U_i$ which glue together in the right way. 
    
That is, for any two charts $U_i$ and $U_j$, if we restrict $f_i(z)(\dd z)^n$ to $\varphi_i(V_i\cap V_j)$ and $f_j(z)(\dd z)^n$ to $\varphi_j(V_i\cap V_j)$, then the pullback of $f_i(z)(\dd z)^n|_{\varphi_i(V_i\cap V_j)}$ along the transition map $\varphi_j(V_i\cap V_j)\xrightarrow{\varphi_i\varphi_j^{-1}}\varphi_i(V_i\cap V_j)$ is equal to $f_j(\dd z)^n|_{\varphi_j(V_i\cap V_j)}$.

We will henceforth specify meromorphic differentials on $X(\Gamma)$ using the notation $\{f_i(z)(\dd z)^n\}$.

A meromorphic differential on $X(\Gamma)$ may be pulled back along the quotient map $\pi\colon \mathbb H\subseteq \mathbb H^\ast \to X$ to obtain a meromorphic differential on $\mathbb H$. To understand this pullback explicitly, start with a known set of charts $V_i$ for $X(\Gamma)$ and pull back each $f_i(z)(\dd z)^n$ on each one of those charts. We will not record the calculations for the pullback maps using the usual charts for $X(\Gamma)$, but the point is that because a meromorphic differential on $X(\Gamma)$ is specified by a collection of meromorphic differentials on open sets which glue together, these meromorphic differentials will still glue together after being pulled back to meromorphic differentials on $\mathbb H$.

Since $\pi$ is the quotient given by identifying orbits of the action of $\Gamma$ on $\mathbb H$ to points, a meromorphic differential $f(z)(\dd z)^n$ on $\mathbb H$ obtained by pulling back a meromorphic differential on $X(\Gamma)$ along the quotient map $\pi$ is $\Gamma$-invariant. So for any $\gamma\in \Gamma$, we can pull back $f(z)(\dd z)^n$ along the action of $\gamma$ on $\mathbb H$ to obtain equalities 
\[f(z)(\dd z)^n = \gamma^\ast(f(z)(\dd z)^n) = f(\gamma(z))(\gamma^\prime(z))^n(\dd z)^n = j(\gamma, z)^{-2n}f(\gamma(z))(\dd z)^n.\]
This shows that the function $f(z)$ is weakly modular with weight $2n$. To see that $f(z)$ is meromorphic at cusps, it suffices to see that the meromorphic differential it comes from expanded locally around the cusps of $X(\Gamma)$ is given by a function which is meromorphic at zero, so $f(z)$ should also be meromorphic at its cusps.

The above discussion establishes the following map
\begin{multline*}
    \{\text{meromorphic differentials on $X(\Gamma)$}\}\xlongrightarrow{\pi^\ast}\{\Gamma\text{-invariant meromorphic differentials }f(z)(\dd z)^k\text{ on }\mathbb H\}\\\cong \{\text{weakly modular functions }f(z)\text{ of weight $2k$}\}.
\end{multline*}
That $\pi^\ast$ is an isomorphism is not difficult but is tedious to show (this amounts to Theorem 2.3.1 in \cref{miyake}, which proves that the composite of the maps above is an isomorphism). The calculation one must do is to show that a weakly meromorphic function $f(z)$ of weight $2k$ gives rise to a meromorphic differential on $X(\Gamma)$ of degree $k$ in a way which is (left and right) inverse to the composite map above. In summary the three vector spaces below are isomorphic:
\begin{enumerate}
    \item the space of weakly modular functions $f(z)$ of weight $2k$
    \item the space of $\Gamma$-invariant meromorphic differentials $f(z)(\dd z)^k$ on $\mathbb H$
    \item the space of meromorphic differentials on $X(\Gamma)$ of degree $k$.
\end{enumerate}

Replacing meromorphic with holomorphic above gives the correspondence of modular forms of weight $2k$ with $\Gamma$-invariant holomorphic differentials $f(z)(\dd z)^k$ on $\mathbb H$ and with holomorphic differentials on $X(\Gamma)$ of degree $k$. We will give a fourth characterization using divisors in the next section.

\newpage
\section{Riemann-Roch and dimension formulas}
Due to the elliptic points coming from $\Gamma$, meromorphic modular functions of weight $2k$ may have orders that are fractional and not just integers. So to start we will consider the $\mathbb Q$-vector space of divisors $\Div_{\mathbb Q}(X(\Gamma))=\Div(X(\Gamma))\otimes_{\mathbb Z} \mathbb Q$ on $X(\Gamma)$ and carefully reduce calculations to ones involving integer coefficients so that we can use the Riemann-Roch theorem. Henceforth we also suppress the argument $(z)$ of functions.

Let $f$ be any fixed nonzero weakly modular function of weight $2k$. Since the quotient of two meromorphic modular functions of the same weight is weight zero; that is, just a meromorphic function, it follows that any weakly modular function of weight $2k$ may be obtained by multiplying $f$ by a suitable meromorphic function on $X(\Gamma)$ (in notation an element of $\mathbb C(X(\Gamma))$; which of course pulls back to a weakly modular function on $\mathbb H$).

The space of modular forms $M_{2k}(\Gamma)$ with respect to $\Gamma$ may be described as the elements $f_0f$ with $f_0\in \mathbb C(X(\Gamma))$ such that either $f_0f = 0$ or $\div(f_0f) \geq 0$. Since $f$ is nonzero; the space of modular forms $M_{2k}(\Gamma)$ is isomorphic to the vector space $\{f_0\in \mathbb C(X(\Gamma))\mid f_0 = 0 \text{ or } \div(f_0) + \div(f)\geq 0\}$. Since $f_0$ is meromorphic on $X(\Gamma)$, $\div(f_0)$ is integral, so the condition $\div(f_0) + \div(f)\geq 0$ is equivalent to the condition $\div(f_0) + \lfloor\div(f)\rfloor\geq 0$ (where the floor function on divisors means to apply it to the coefficients). It follows that $M_{2k}(\Gamma)$ is isomorphic to $L(\lfloor\div(f)\rfloor) = \{f_0\in \mathbb C(X(\Gamma))\mid f_0 = 0 \text{ or }\div(f_0) + \lfloor\div(f)\rfloor\geq 0\}$, so their dimensions are the same; that is, $\dim (M_{2k}(\Gamma)) = l(\lfloor\div(f)\rfloor)$.

-- After here is bad --

Let $\omega$ be the meromorphic differential on $X(\Gamma)$ of degree $k$ which pulls back to $f(\dd z)^{k}$. Identifying $\omega$ with $f(\dd z)^{k}$, it is the case that (here the divisor of a meromorphic differential is calculated by expanding locally and summing the divisors of the local expansions) $\div(\omega) = \div(f) + (2k/2)(-\sum_{\text{period 2 elliptic points}}(1/2)a_i -\sum_{\text{period 3 elliptic points}}(2/3)b_i - \sum_{\text{cusps from $\Gamma$}}c_i)$. The left hand side has integer coefficients, so by taking the floor and then the degree on both sides we obtain the equation
\[\deg(\lfloor\div(f)\rfloor) = \deg(\div(\omega)) + \lfloor(2k/4)\rfloor \varepsilon_2 + \lfloor(2k/3)\rfloor\varepsilon_3 + (2k/2)\varepsilon_\infty,\] where $\varepsilon_2,\varepsilon_3,\varepsilon_\infty$ are the number of period $2$, $3$ elliptic points and cusps respectively. Note further that since $\div(\omega)$ is a $k$ power of a canonical divisor, its degree is $2k(g-1)$ with $g$ the genus of $X(\Gamma)$. Then with $k\geq 1$, simple estimates give $\deg(\lfloor\div(f)\rfloor)>2g-2$.

The Riemann-Roch theorem simplifies to the following expression since the degree of $\deg(\lfloor\div(f)\rfloor)$ is greater than $2g-2$: 
\[l(\lfloor\div(f)\rfloor) = (2k-1)(g-1) + \lfloor(k/2)\rfloor\varepsilon_2 + \lfloor(2k/3)\rfloor\varepsilon_3 + k\varepsilon_\infty;\] this expression is hence equal to the dimension of $M_{2k}(\Gamma)$.

To obtain the dimension formula for the space of cusp forms $M^0_{2k}(\Gamma)$, observe that since cusp forms vanish at cusps, if $f$ is a weakly meromorphic modular function on $\mathbb H$ then $M^0_{2k}(\Gamma)\cong \{f_0\in \mathbb C(X(\Gamma))\mid \div(f_0) + \lfloor\div(f) - \sum_{\text{cusps}}c_i\rfloor\geq 0\}$; i.e. $M^0_{2k}(\Gamma)\cong L(\lfloor\div(f) - \sum_{\text{cusps}}c_i\rfloor)$. Repeat similar estimates as above (and here we need $k\geq 2$ instead of $k\geq 1$ to use the version of Riemann-Roch as above) and use the Riemann-Roch theorem to find that for $k\geq 2$, $\dim(M^0_{2k}(\Gamma)) = l(\lfloor\div(f)\rfloor) - \varepsilon_\infty = \dim(M_{2k}(\Gamma))-\varepsilon_\infty$. For $k = 1$, the divisor $\lfloor\div(f) - \sum_{\text{cusps}}c_i\rfloor$ is a canonical divisor, so its linear space (which is isomorphic to $M^0_2(\Gamma)$) has dimension $g$.

Observe that modular forms of weight zero correspond to holomorphic functions $X(N)\to\widehat{\mathbb C}$; since $X(N)$ and $\widehat{\mathbb C}$ are compact, such maps are either surjective or constant. Since these maps have no poles they could not be surjective, so they are constants; that is, $M_0(\Gamma)\cong \mathbb C$ is the space of constant functions. This implies that $M^0_0(\Gamma) = 0$. For negative $k$, if there were modular forms $f$ of weight $2k$, then $f^6\Delta$ is a weight zero cusp form; that is, the zero function. So $f$ had to be zero. Hence there are no modular forms of negative even weight.
\end{document}