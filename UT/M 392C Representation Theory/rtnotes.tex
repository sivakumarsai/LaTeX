\documentclass[11pt,leqno]{article}
\headheight=13.6pt

% packages
\usepackage[alphabetic]{amsrefs}
\usepackage{physics}
% margin spacing
\usepackage[top=1in, bottom=1in, left=0.5in, right=0.5in]{geometry}
\usepackage{amsfonts, amsmath, amssymb, amsthm}
\usepackage{extarrows}
\usepackage[none]{hyphenat}
\usepackage{fancyhdr}
\usepackage{graphicx}
\graphicspath{{./images/}}
\usepackage{float}
\usepackage{color}
\newcommand{\sai}[1]{\textcolor{red}{#1}}
\usepackage{enumitem}
\usepackage{mathrsfs}
\usepackage{hyperref}
\usepackage[noabbrev, capitalise]{cleveref}
\crefformat{equation}{equation~#2#1#3}
\crefformat{lemma}{\textrm{Lemma}~#2#1#3}
\usepackage{quiver}
\usepackage{tikzit}
\input{rt.tikzstyles}
\usepackage{titlesec}
\titleformat{\section}
  {\bfseries}{Lecture \thesection}{1em}{}
\titleformat{\subsection}
  {\bfseries}{}{1em}{}

% theorems
\theoremstyle{plain}
\newtheorem{lem}{Lemma}
\newtheorem{lemma}[lem]{Lemma}
\newtheorem{thm}[lem]{Theorem}
\newtheorem{theorem}[lem]{Theorem}
\newtheorem{prop}[lem]{Proposition}
\newtheorem{proposition}[lem]{Proposition}
\newtheorem{cor}[lem]{Corollary}
\newtheorem{corollary}[lem]{Corollary}
\newtheorem{conj}[lem]{Conjecture}
\newtheorem{fact}[lem]{Fact}
\newtheorem{form}[lem]{Formula}

\theoremstyle{definition}
\newtheorem{defn}[lem]{Definition}
\newtheorem{definition/}[lem]{Definition}
\newenvironment{definition}
  {\renewcommand{\qedsymbol}{\textdagger}%
   \pushQED{\qed}\begin{definition/}}
  {\popQED\end{definition/}}
\newtheorem{example}[lem]{Example}
\newtheorem{remark}[lem]{Remark}
\newtheorem{exercise}[lem]{Exercise}
\newtheorem{notation}[lem]{Notation}

\numberwithin{equation}{section}
\numberwithin{lem}{section}

% header/footer formatting
\pagestyle{fancy}
\fancyhead{}
\fancyfoot{}
\fancyhead[L]{M 392C}
\fancyhead[C]{}
\fancyhead[R]{Representation Theory}
\fancyfoot[R]{\thepage}
\renewcommand{\headrulewidth}{1pt}

% paragraph indentation/spacing
\setlength{\parindent}{0cm}
\setlength{\parskip}{10pt}
\renewcommand{\baselinestretch}{1.25}

% extra commands defined here
\newcommand{\br}[1]{\left(#1\right)}
\newcommand{\sbr}[1]{\left[#1\right]}
\newcommand{\cbr}[1]{\left\{#1\right\}}
\newcommand{\eq}[1]{\overset{(#1)}{=}}

% bracket notation for inner product
\usepackage{mathtools}

\DeclarePairedDelimiterX{\abr}[1]{\langle}{\rangle}{#1}

\DeclareMathOperator{\im}{im}
\DeclareMathOperator{\GL}{GL}
\DeclareMathOperator{\Hom}{Hom}
\DeclareMathOperator{\id}{id}
\DeclareMathOperator{\Aut}{Aut}
\DeclareMathOperator{\End}{End}
\DeclareMathOperator{\SO}{SO}
\DeclareMathOperator{\SL}{SL}
\DeclareMathOperator{\PSL}{PSL}
\DeclareMathOperator{\Fun}{Fun}
\DeclareMathOperator{\Spec}{Spec}
\DeclareMathOperator{\U}{U}

% categories
\newcommand{\catname}[1]{{\normalfont\mathbf{#1}}}
\renewcommand{\op}{\mathrm{op}} % where is \op defined?
\newcommand{\Rmod}{R\text{-}\catname{mod}}
\newcommand{\Vect}{\catname{Vect}}
\newcommand{\Rep}{\catname{Rep}}
\newcommand{\alg}{\text{-}\catname{alg}}
\newcommand{\Group}{\catname{Group}}
\newcommand{\Ring}{\catname{Ring}}
\newcommand{\av}{\mathrm{av}}

% double hat, widecheck macros
\makeatletter
\newcommand{\dwidehat}[1]{% 
\begingroup%
  \let\macc@kerna\z@%
  \let\macc@kernb\z@%
  \let\macc@nucleus\@empty%
  \hspace{0.165em}\widehat{\hspace{-0.165em}\raisebox{.3ex}{\vphantom{\ensuremath{#1}}}\smash{\widehat{#1}}}%
\endgroup%
}
\makeatother

\makeatletter
\newcommand{\dhat}[1]{% 
\begingroup%
  \let\macc@kerna\z@%
  \let\macc@kernb\z@%
  \let\macc@nucleus\@empty%
  \hat{\raisebox{.3ex}{\vphantom{\ensuremath{#1}}}\smash{\hat{#1}}}%
\endgroup%
}
\makeatother

% smileys frownies
\usepackage{wasysym}
\newcommand{\smallhappy}{\raisebox{-.14em}{\smiley}}
\newcommand{\happy}{\raisebox{-.24em}{\resizebox{1.2em}{!}{\smiley}}}
\newcommand{\smallsad}{\raisebox{-.14em}{\frownie}}
\newcommand{\sad}{\raisebox{-.24em}{\resizebox{1.2em}{!}{\frownie}}}
\DeclareMathOperator{\mathhappy}{\!\happy\!}
\DeclareMathOperator{\smallmathhappy}{\!\smallhappy\!}
\DeclareMathOperator{\mathsad}{\!\sad\!}
\DeclareMathOperator{\smallmathsad}{\!\smallsad\!}

% set page count index to begin from 1
\setcounter{page}{1}

\begin{document}
M 392C Representation Theory, Fall 2025: Dr.~David Ben-Zvi's lectures, \TeX ed by Sai Sivakumar. The main goal of these notes was to clarify and even expand on several of the points David made during the lectures. As a warning, I made minimal effort to reorder the content of the course, and many digressions in these notes are much longer (hopefully clearer) than they actually were in class.
\tableofcontents

\newpage\section{08/26}
\subsection{Definitions}
Let $G$ be a group and $k$ be a field. A representation of $G$ is a vector space $V\in \Vect_k$ with a linear action of $G$ on $V$. (For convenience, let's assume representations are finite-dimensional vector spaces unless otherwise stated.)

This is to say there is a homomorphism $G\xrightarrow{\rho}\Aut(V)$ so that we may define the action of $G$ on $V$ to be $g\cdot v \coloneqq \rho(g)v$ or alternatively a product $G\times V\to V$ given by $(g,v)\mapsto \rho(g)v$ that is linear in $V$ and associative in $G$, among other properties. (We often suppress the $\,\cdot\,$ in the action/product.)

Note $\Aut(V) \eqqcolon \GL(V) \cong \GL_{\dim V}(k)$, so we could even think of the elements of $G$ as matrices if we fix a basis of $V$.

The (finite dimensional) representations $(V,\rho)$ of $G$ over a field $k$ form the objects of a category $\Rep_k(G)$ whose morphisms are linear transformations which commute with the action of $G$; i.e., $\Hom_G(V,W)$ consists of the linear transformations $T\colon V\to W$ for which $g(Tv) = T(gv)$ for all $g\in G$ and $v\in V$. (These are also called intertwining operators, $G$-linear maps, or $G$-equivariant maps, etc. Also note that $\Hom_G(V,W)$ is merely a vector space, but see \href{https://math.stackexchange.com/questions/2334514/whats-the-internal-hom-of-linear-representations-of-categories}{this MSE post}.) 

The field $k$ does matter, but for the majority of what follows $k$ will be $\mathbb C$. Sometimes it may be possible to replace $\mathbb C$ with any algebraically closed field, but the characteristic of the field is rather important (see \href{https://en.wikipedia.org/wiki/Modular_representation_theory}{modular representation theory} for example). 

\subsection{Variants of representations}
We can add adjectives in various places to the definitions above to get different kinds of representations. 

For example, if $V$ has an inner product we can ask about linear actions of $G$ on $V$ that preserve that inner product. If $V$ is a Hilbert space we can ask about linear actions of $G$ on $V$ for which $\rho(g)$ is a unitary operator; these are the unitary representations of $G$.

If $G$ has a topology and $V$ is a topological vector space, we say $(V,\rho)$ is a continuous representation if the product map $G\times V\to V$ defined before is continuous. If $G$ has a smooth structure (e.g. is a Lie group) or is an algebraic group, we can ask about smooth or algebraic representations, respectively.

\subsection{Aspects of representations we study}
\begin{enumerate}
    \item Irreducible representations (henceforth called ``irreps''): An irrep of $G$ are the representations $V$ which have no invariant (or stable some might say) subspaces under the group action; that is, if $W$ is a $G$-invariant subspace of $V$ then $W$ is either $0$ or $V$. These are like the ``atoms'' in representation theory.
    
    For a given group $G$ one goal of representation theory is to classify up to isomorphism its irreps. One approach is to try to attach numerical invariants to representations and see if they help to classify irreps, and generalizations of this idea lead to character theory. Sometimes there are ``isotopes'' that are not isomorphic but nevertheless cannot be distinguished by certain invariants.

    Another thing we do is to take a subgroup $H$ of $G$ and study how irreps of $G$ decompose when the group action is restricted to $H$. On the other hand, we can also look at how to build bigger representations or ``molecules'' out of the atomic irreps via extension. Of course, it is easy to take direct sums of irreps but depending on the context it may be possible to obtain indecomposable reps which are extensions of irreps but do not decompose into a direct sum of irreps (so to summarize, irreducible implies indecomposable but not the other way around in general).

    Maschke's theorem implies that indecomposable representations of finite groups over fields with characteristic not dividing the order of the group are irreducible. Alternatively, the theorem implies that in this setting, all short exact sequences of representations split.
    
    For example, the indecomposable representations over $\mathbb C$ coincide with the irreps when $G$ is finite. On the other hand, the shearing representation 
    \begin{align*}
      \mathbb Z &\to \GL_2(\mathbb R)\\
      1 &\mapsto \big(\!\begin{smallmatrix}
        1 & 1 \\ 0 & 1
      \end{smallmatrix}\!\big)
    \end{align*}
    preserves the horizontal axis, so it is not an irrep, but is not decomposable since $\big(\!\begin{smallmatrix}
        1 & 1 \\ 0 & 1
    \end{smallmatrix}\!\big)$ is not diagonalizable. The representations $(k^n,\rho)$ of $\mathbb Z$ are given by specifying an invertible $n\times n$ matrix, and any two such representations are isomorphic if the matrices specifying them are conjugate. If $k$ is algebraically closed, the indecomposable representations in this setting correspond to Jordan blocks. Of course, every representation is the direct sum of indecomposables, and in this setting the Jordan normal form of an $n\times n$ matrix would describe the decomposition of $k^n$ into indecomposables.

    \item Harmonic analysis is the study of naturally appearing ``large'' representations; we would like to perform some kind of ``spectroscopy'' to determine what representations appear within them.
    
    If a group $G$ acts on some object $X$ (which could be a set, manifold, or otherwise a ``geometric'' object), we say $X$ has some symmetries which we would like to ``linearize''. We achieve this by considering the $k$-valued functions on $X$ for some field $k$; if $X$ has additional structure we can restrict to functions on $X$ which interact with that structure (e.g. measurable/continuous/etc maps)

    From a right action of $G$ on $X$, function spaces on $X$ inherit a natural linear left action of $G$; one is given by $gf(x) = f(xg)$. These representations are typically very decomposable or reducible.

    \item Different groups give rise to different phenomena. Below are some examples of groups we will see again.
    \begin{table}[H]
\centering
\begin{tabular}{lllll}
\multicolumn{1}{l|}{}        & \multicolumn{1}{l|}{compact} & noncompact  &  &  \\ \cline{1-3}
\multicolumn{1}{l|}{Abelian} & \multicolumn{1}{l|}{$S^1$}   & $\mathbb R$ &  &  \\ \cline{1-3}
\multicolumn{1}{l|}{non-Abelian} & \multicolumn{1}{l|}{$\SO(3)$} & $\SL_2(\mathbb R)$, $\SL_2(\mathbb C)$ &  &  \\
                             &                              &             &  & 
\end{tabular}
\end{table}
    
    Irreps of compact groups coincide with their indecomposables. Irreps of Abelian groups are all one-dimensional.
    
    The representation theory of $S^1$ leads to the theory of Fourier series, and in a similar way we can recover the Fourier transform from the representation theory of $\mathbb R$.

    The Lie group $\SO(3)$ can be thought of as the group of rotations of a $2$-sphere. One nice result is that the spherical harmonics are basis functions for the irreps of $\SO(3)$, which occur naturally as the atomic orbitals (see \href{https://en.wikipedia.org/wiki/Spherical_harmonics}{the Wikipedia article on spherical harmonics}).

    The representations of the groups $\SL_2(k)$ for various fields $k$ appear all over math.  We can study special functions like the Bessel or hypergeometric functions, or even modular forms. The hard Lefschetz theorem in algebraic geometry says that $\SL_2(\mathbb C)$ acts on $H^\ast(X,\mathbb C)$ for $X$ a nice enough smooth projective variety. In physics, the special linear group sort of appears in the Lorentz group $\SO(1,3)^+ = \PSL_2(\mathbb C)$.
\end{enumerate}

\subsection{Spectral theory}
A slogan for what is to come: ``commutativity implies geometry''.

Let $k = \mathbb C$ and $X$ a set. Then the complex-valued functions on $X$ form a commutative algebra. This is some example of a functor suggestively called $\mathcal O$ from the category of some kind of geometric objects to commutative algebras.

To expand on the previous idea, here is a motto originating from Gelfand and Grothendieck's work:
\begin{enumerate}
  \item Any commutative ring should be thought of as functions on some space.
  
  That is, there is some functor going from commutative rings to some category of geometric objects that realizes this idea. In particular we should be able to adjust the functor and its source/target to obtain an equivalence of categories.
  \[\text{commutative rings}\longleftrightarrow \text{geometry}\]
  \item Once we are in the situation where we have an equivalence of categories between commutative rings and geometric objects, we should further obtain a correspondence between the modules over a ring $R$ and sheaves (special families of vector spaces or Abelian groups) on the corresponding geometric object $X$ to $R$.
  \[R\text{-modules}\longleftrightarrow \text{sheaves on }X\]
\end{enumerate}
For example, if $X$ is a finite set, the corresponding ring $R$ is the finite-dimensional commutative algebra of complex-valued functions on $X$. This algebra is semisimple with $R = \bigoplus_i \mathbb Ce_i$ where $e_ie_j = \delta_{ij}$. We can think of the $e_i$ as delta/indicator functions on points of $X$.

An $R$-module $M$ has the decomposition $M = \oplus_i e_i M$, which corresponds to a sheaf on $X$ where at each point of $X$ we imagine the corresponding module $e_iM$ lying on it:
\begin{align*}
  M &~~~~ e_1M \phantom{\oplus} e_2M \phantom{\oplus} \cdots \phantom{\oplus} e_n M\\
  X&~~~~~\bullet~\phantom{\oplus}~~\bullet~~\phantom{\oplus} \cdots \phantom{\oplus}~~\bullet
\end{align*}

\subsection{A short word about the group algebra}
A representation $V$ of a group $G$ is given by a group homomorphism $G\xrightarrow{\rho}\Aut(V)$. Since $\Aut(V)$ is contained in $\End(V)$, a $k$-algebra, there is an object $k[G]$ called the group algebra for which the homomorphism $\rho$ factors through $k[G]$; that is, the diagram 
% https://q.uiver.app/#q=WzAsMyxbMCwwLCJHIl0sWzEsMCwiXFxFbmQoVikiXSxbMCwxLCJrW0ddIl0sWzAsMSwiXFxyaG8iXSxbMCwyLCIiLDIseyJzdHlsZSI6eyJ0YWlsIjp7Im5hbWUiOiJob29rIiwic2lkZSI6InRvcCJ9fX1dLFsyLDEsIlxcb3ZlcmxpbmUgXFxyaG8iLDIseyJzdHlsZSI6eyJib2R5Ijp7Im5hbWUiOiJkYXNoZWQifX19XV0=
\[\begin{tikzcd}
	G & {\End(V)} \\
	{k[G]}
	\arrow["\rho", from=1-1, to=1-2]
	\arrow[hook, from=1-1, to=2-1]
	\arrow["{\overline \rho}"', dashed, from=2-1, to=1-2]
\end{tikzcd}\]
commutes. One explicit description of $k[G]$ is the set of finite linear combinations of elements of $G$ with coefficients in $k$:
\[k[G] = \biggl\{\sum_{g\in G}c_gg \biggm\vert c_g\in k, \text{ all but finitely many $c_k$ are zero}\biggr\}.\]
The addition and multiplication in $k[G]$ are defined using the addition in $k$ and multiplication in $G$, respectively. We will return to the group algebra when investigating representations from a module-theoretic point of view.

\newpage\section{08/28}
\subsection{Spectral theory over $\mathbb C$}
Consider an operator $T$ on a finite dimensional complex vector space $V$. The spectrum of $T$, $\sigma(T)$, is a finite subset of $\mathbb C$ consisting of the eigenvalues of $T$. To each $\lambda \in \sigma(T)$ there is a corresponding eigenspace $V_\lambda$ of $V$, and $V = \bigoplus_{\lambda\in \sigma(T)}V_\lambda$. We can visualize placing each of the eigenspaces $V_\lambda$ above each point $\lambda$ in the spirit of sheaf theory:
\begin{figure}[h]
  \centering
  \begin{tikzpicture}
	\begin{pgfonlayer}{nodelayer}
		\node [style=none] (0) at (0, 0) {};
		\node [style=none] (1) at (8, 1) {};
		\node [style=none] (2) at (4, -3) {};
		\node [style=none] (3) at (12, -2) {};
		\node [style=none] (4) at (9.75, -1.5) {$\mathbb C$};
		\node [style=none] (5) at (2.5, -0.75) {$\lambda_1$};
		\node [style=none] (6) at (4, -2.25) {$\lambda_2$};
		\node [style=none] (7) at (5.5, -0.25) {$\lambda_3$};
		\node [style=none] (8) at (8, -2) {$\lambda_n$};
		\node [style=none] (9) at (7, -0.8) {$\ddots$};
		\node [style=none] (10) at (3, -0.5) {$\bullet$};
		\node [style=none] (11) at (4.5, -2) {$\bullet$};
		\node [style=none] (12) at (8.5, -1.75) {$\bullet$};
		\node [style=none] (13) at (6, 0) {$\bullet$};
		\node [style=none] (14) at (3, 1) {};
		\node [style=none] (16) at (6, 1.5) {};
		\node [style=none] (18) at (8.5, -0.25) {};
		\node [style=none] (20) at (4.5, -0.5) {};
		\node [style=none] (21) at (2.5, 1.5) {$V_{\lambda_1}$};
		\node [style=none] (22) at (4, 0) {$V_{\lambda_2}$};
		\node [style=none] (23) at (5.5, 2) {$V_{\lambda_3}$};
		\node [style=none] (24) at (8, 0.25) {$V_{\lambda_n}$};
		\node [style=none] (25) at (3, -2) {};
		\node [style=none] (26) at (4.5, -3.5) {};
		\node [style=none] (27) at (6, -1.5) {};
		\node [style=none] (28) at (8.5, -3.25) {};
	\end{pgfonlayer}
	\begin{pgfonlayer}{edgelayer}
		\draw (0.center) to (1.center);
		\draw (2.center) to (3.center);
		\draw (0.center) to (2.center);
		\draw (1.center) to (3.center);
		\draw [style=to] (10.center) to (14.center);
		\draw [style=to] (11.center) to (20.center);
		\draw [style=to] (13.center) to (16.center);
		\draw [style=to] (12.center) to (18.center);
		\draw [style={to:dash}] (10.center) to (25.center);
		\draw [style={to:dash}] (11.center) to (26.center);
		\draw [style={to:dash}] (13.center) to (27.center);
		\draw [style={to:dash}] (12.center) to (28.center);
	\end{pgfonlayer}
\end{tikzpicture}
\end{figure}

On $\mathbb C$, the coordinate function $x$ where $x(z) = z$ has the same action as $T$ on the eigenspaces $V_\lambda$:
\[Tv = \sum_{\lambda\in\sigma(T)}Tv_\lambda = \sum_{\lambda\in\sigma(T)}\lambda v_\lambda = \sum_{\lambda\in\sigma(T)}x(\lambda) v_\lambda = x\cdot \biggl(\sum_\lambda v_\lambda\biggr) = x\cdot v\] 
In this sense $T$ corresponds to $x\in \Fun(\mathbb C)|_{\sigma(T)}$ (complex-valued functions on $\mathbb C$, restricted to $\sigma(T)$.).

Now consider the commutative ring $R = k[T_1,\dots,T_n]/(f_1,\dots,f_m)$ and an $R$-module $M$. We would like to simultaneously diagonalize the action of the $T_i$ on $M$, or rather, diagonalize the action of $R$ on $M$, which amounts to finding a basis of $M$ where $R$ acts diagonally. Assume we can do this. 

We should attach a set $X = \Spec(R)$ to $R$ called the spectrum of $R$ for which we can decompose $M$ into a direct sum of modules $M_{x_i}$, each summand lying over their corresponding point $x_i\in X$:
\begin{align*}
  M &= M_{x_1} \oplus M_{x_2} \oplus \cdots \oplus M_{x_\ell}\\
  X&~~~~~x_1~\phantom{\oplus}~~x_2~~\phantom{\oplus} \cdots \phantom{\oplus}~~x_\ell
\end{align*}
Furthermore, we form an assignment $R\xrightarrow{\varphi} \Fun(X)$ for which we can recast the action of $R$ on $M$ through this assignment:
\[rm = \varphi(r)\sum_{x\in X}m_x = \sum_{x\in X}\varphi(r)|_{\{x\}}m_x\]
The idea here is to turn what was an algebraic notion of rings acting on modules to thinking about sheaves on a particular space which ``diagonalize'' the ring action.

\subsection{Some examples of the algebra-geometry dictionary}
An important philosophy we've been looking at the broad strokes of is this dictionary between algebra and geometry, specifically the two following ideas coming from Gelfand and Grothendieck:
\begin{enumerate}
  \item Commutative rings correspond to geometric objects via functors 
  % https://q.uiver.app/#q=WzAsMixbMCwwLCJcXHRleHR7Y29tbXV0YXRpdmUgcmluZ3N9Il0sWzIsMCwiXFx0ZXh0e2dlb21ldHJpYyBvYmplY3RzfSJdLFswLDEsIlxcU3BlYyIsMCx7ImN1cnZlIjotMX1dLFsxLDAsIlxcbWF0aGNhbCBPIiwwLHsiY3VydmUiOi0xfV1d
\[\begin{tikzcd}[ampersand replacement=\&]
	{\text{commutative rings}} \&\& {\text{geometric objects}}
	\arrow["\Spec", curve={height=-6pt}, from=1-1, to=1-3]
	\arrow["{\mathcal O}", curve={height=-6pt}, from=1-3, to=1-1]
\end{tikzcd}\]
In this vague setting we should think of $\Spec$ as in taking the spectrum of some collection of simultaneously diagonalized operators coming from the ring and $\mathcal O$ as returning the space of functions on these geometric objects. Again we should think of $R$ as being the space of functions on $\Spec R$ in this correspondence.
\item Modules over rings $R$ correspond to sheaves on the corresponding geometric object $\Spec R$ via
% https://q.uiver.app/#q=WzAsMixbMCwwLCJcXHRleHR7bW9kdWxlc30iXSxbMiwwLCJcXHRleHR7c2hlYXZlc30iXSxbMCwxLCJcXHRleHR7c3BlY3RyYWwgZGVjb21wb3NpdGlvbn0iLDAseyJjdXJ2ZSI6LTF9XSxbMSwwLCJcXHRleHR7Z2xvYmFsIHNlY3Rpb25zfSIsMCx7ImN1cnZlIjotMX1dXQ==
\[\begin{tikzcd}[ampersand replacement=\&]
	{\text{modules}} \&\& {\text{sheaves}}
	\arrow["{\text{spectral decomposition}}", curve={height=-6pt}, from=1-1, to=1-3]
	\arrow["{\text{global sections}}", curve={height=-6pt}, from=1-3, to=1-1]
\end{tikzcd}\]
In particular in the previous examples we have that taking global sections amounts to taking the direct sum of modules, but this can also appear as a direct integral of modules in continuous versions of the previous examples. Spectral decomposition as we have seen is to break up a module into submodules where the ring action is pointwise multiplication.
\end{enumerate}

We look at some examples of part 1. of the above philosophy.

Grothendieck's version of this idea is the heart of modern algebraic geometry. One correspondence is 
\[\text{commutative rings}\longleftrightarrow\text{affine schemes}\]
but an earlier version might have been
\[\text{finitely presented, reduced, etc. $\mathbb C$-algebras}\longleftrightarrow\text{complex affine varieties}\]
In both correspondences, the $\Spec$ functor is given by taking the set of prime ideals. In the top correspondence, $\mathcal O$ is taking the structure sheaf of a scheme, but this amounts to taking polynomial functions on a space in the bottom correspondence.

Gelfand's version of this idea is the correspondence 
\[\text{commutative $C^\ast$-algebras}\longleftrightarrow\text{Hausdorff locally compact topological spaces}\]
The functor $\Spec$ in this case is the eponymous Gelfand spectrum and $\mathcal O$ takes the continuous compactly supported functions on Hausdorff, locally compact spaces.

A special case of the above is the correspondence 
\[\text{commutative von Neumann algebras}\longleftrightarrow\text{measure spaces}\]
One direction is some kind of spectrum, but the other direction is taking $L^\infty$ functions on a measure space. As a side remark, there are only five von Neumann algebras up to equivalence; they are: finite sets, $\mathbb N$, $[0,1]$, $[0,1]$ union a finite set, and $[0,1]$ union a countably infinite set.

There is a version of part 2. for each of the above examples involving modules and sheaves, but we will not discuss them here, aside from mentioning that we can talk about algebraic, continuous, or measurable families of vector spaces (sheaves) in the various settings above.

\subsection{The group algebra (over $\mathbb C$)}
Let $G$ be any group and $(V,\rho)$ any complex representation of $G$ (the below discussion would work for other fields instead of $\mathbb C$). Recall that $\mathbb C[G]$ is the unique object for which the diagram 
% https://q.uiver.app/#q=WzAsNCxbMCwwLCJHIl0sWzAsMSwiXFxtYXRoYmIgQ1tHXSJdLFsxLDAsIlxcQXV0KFYpIl0sWzIsMCwiXFxFbmQoVikiXSxbMCwxLCIiLDAseyJzdHlsZSI6eyJ0YWlsIjp7Im5hbWUiOiJob29rIiwic2lkZSI6InRvcCJ9fX1dLFswLDIsIlxccmhvIl0sWzEsMywiXFxvdmVybGluZVxccmhvIiwyLHsic3R5bGUiOnsiYm9keSI6eyJuYW1lIjoiZGFzaGVkIn19fV0sWzIsMywiIiwxLHsic3R5bGUiOnsidGFpbCI6eyJuYW1lIjoiaG9vayIsInNpZGUiOiJ0b3AifX19XV0=
\[\begin{tikzcd}[ampersand replacement=\&]
	G \& {\Aut(V)} \& {\End(V)} \\
	{\mathbb C[G]}
	\arrow["\rho", from=1-1, to=1-2]
	\arrow[hook, from=1-1, to=2-1]
	\arrow[hook, from=1-2, to=1-3]
	\arrow["{\overline\rho}"', dashed, from=2-1, to=1-3]
\end{tikzcd}\]
commutes; moreover, we think of $\mathbb C[G]$ as a ``free'' $\mathbb C$-algebra on $G$. This is because the functor taking $G$ to $\mathbb C[G]$ is the left adjoint to the forgetful functor from $\mathbb C\alg$ to $\Group$ (given by taking the group of units).

An explicit description of $\mathbb C[G]$ is the $\mathbb C$-vector space generated by the elements of $G$ with multiplication induced by the products in $\mathbb C, G$. An alternative description of $\mathbb C[G]$ is the set of finitely supported functions from $G$ to $\mathbb C$, with multiplication given by convolution:
\[(f\ast h)(g) = \sum_{\substack{(x,y)\\xy=g}}f(x)h(y) = \sum_{x}f(x)h(x^{-1}g)\quad\text{(defined since $f,h$ are finitely supported)}\]
This product is no different from the product defined in the first description of $\mathbb C[G]$.

What is really happening is that we can take $f,h$ from above and form the box product $f\boxtimes h\colon G\times G \to \mathbb C$, which is just the map $(x,y)\mapsto f(x)h(y)$. We want to form a map from $G$ to $\mathbb C$, so consider $G\times G\xrightarrow{m}G$, the product in $G$. The pushforward of $f\boxtimes h$ via $m$ is $f\ast h$:
\[m_\ast(f\boxtimes h) \quad\text{also denoted}\quad \int_mf\boxtimes h = \sum_{{\substack{(x,y)\\xy=-}}}f(x)h(y) = f\ast h\]

An important observation to make is that complex representations of $G$ coincide with $\mathbb C[G]$-modules. 

\subsection{Representation theory of finite Abelian groups and the dual group}
If $G$ is Abelian, then $\mathbb C[G]$ is a commutative ring. So representation theory may be thought of as a special case of the study of modules over commutative rings. Using language from earlier, note that the corresponding geometric object to $\mathbb C[G]$ is $\Spec(\mathbb C[G])$, and representations of $G$ correspond to vector bundles or sheaves over $\Spec(\mathbb C[G])$.

Let $G$ be a finite Abelian group, and let $\widehat G\coloneqq \Spec(\mathbb C[G])$; we call this the dual of $G$. Then $\mathbb C[G]$ (functions on $G$) with multiplication given by convolution is isomorphic as an algebra to $\mathbb C[\widehat G]$ (functions on $\widehat{G}$, which we will see is a finite set) with pointwise multiplication. This isomorphism is usually given by some kind of finite Fourier transform. On the other hand, representations of $G$ correspond to vector bundles (sheaves) on $\widehat G$, and this is some kind of finite spectral theorem.

Let $G = \abr{x}$ be a cyclic group of order $n$. The dual object $\widehat{G}$ is given by the $n$-th roots of unity in $\mathbb C$: (the picture is for $n=6$)

\begin{figure}[h]
  \centering
  \begin{tikzpicture}
	\begin{pgfonlayer}{nodelayer}
		\node [style=none] (0) at (2, 0) {$\bullet$};
		\node [style=none] (0) at (1, 1.73) {$\bullet$};
		\node [style=none] (0) at (-1, 1.73) {$\bullet$};
		\node [style=none] (0) at (-2, 0) {$\bullet$};
		\node [style=none] (0) at (1, -1.73) {$\bullet$};
		\node [style=none] (0) at (-1, -1.73) {$\bullet$};
		\node [style=none] (1) at (-5, 0) {$\widehat G = \Spec(\mathbb C[G]) = $};
	\end{pgfonlayer}
\end{tikzpicture}
\end{figure}

In this case, $\Spec$ is taking the maximal ideals of rings. With $\mathbb C[G]\cong \mathbb C[x]/(x^n-1)$, it is equivalent to describe the elements in $\Hom_{\mathbb C\alg}(\mathbb C[x]/(x^n-1),\mathbb C)$, which are characterized by each of the $n$-th roots of unity.

We take a small detour to discuss Schur's lemma, which states that for any group $G$, that any $\mathbb C[G]$-module homomorphisms between simple $\mathbb C[G]$-modules are either $0$ or are isomorphisms of $\mathbb C[G]$-modules. This follows by investigating kernels and images since they are submodules of $V,W$ respectively. In particular, any $\mathbb C[G]$-module endomorphism of a simple $\mathbb C[G]$-module is a scalar multiple of the identity, since for any nonzero endomorphism $T$ we can consider $T-\lambda \id_V$, which is no longer an isomorphism and hence must be zero. Here we used the algebraic closedness of $\mathbb C$, and in fact we could have replaced $\mathbb C$ by any algebraically closed field $k$.

Schur's lemma is used to prove that if $G$ is Abelian then any finite-dimensional irrep of $G$ is one-dimensional. If such an irrep $V$ had finite dimension greater than or equal to $2$, then left multiplication $V\xrightarrow{g}V$ by any element $g\in G$ must be a scalar multiple of $\id_V$, which contradicts the irreducibility of $V$ (since we assumed $\dim V\geq 2$). 

The action of $G$ on the irrep $V$ is given by scalar multiplication by $\chi_V(g)\in \mathbb C^\ast$ since $V$ is one-dimensional. More importantly, the assignment $g\to\chi_V(g)$ is a group homomorphism $G\to \mathbb C^\ast$, called a character of $G$.

For $G$ Abelian, $\widehat G$ is the set of maximal ideals of $\mathbb C[G]$, which is in bijection with $\Hom_{\mathbb C\alg}(\mathbb C[G],\mathbb C)$. By the universal property of the group algebra, $\Hom_{\mathbb C\alg}(\mathbb C[G],\mathbb C)$ is in bijection with $ \Hom_{\Group}(G,\mathbb C^\ast)$:
\[\begin{tikzcd}[ampersand replacement=\&]
	G \& {\Aut(\mathbb C)\cong \mathbb C^\ast} \& {\End(\mathbb C)\cong \mathbb C} \\
	{\mathbb C[G]}
	\arrow["f", from=1-1, to=1-2]
	\arrow[hook, from=1-1, to=2-1]
	\arrow[hook, from=1-2, to=1-3]
	\arrow["{\overline f}"', dashed, from=2-1, to=1-3]
\end{tikzcd}\]
So in general we can also think of $\widehat G$ as the collection of irreps of $G$.

Given an irrep $V$ of $G$ (i.e. a simple $\mathbb C[G]$-module), we can try to form a sheaf out of $V$ on $\widehat G$. Since $V$ is irreducible, this sheaf has to be concentrated at only one of the points of $\widehat{G}$, and this point is the point corresponding to the irrep $V$ itself. So in the example where $G = \abr{x}$ has order $n$, the point at which an irrep $V$ lies on is the root of unity $\zeta$ for which the action of $x$ on an irrep $V$ is given by multiplication by $\zeta$.

\begin{figure}[h]
  \centering
  \begin{tikzpicture}
	\begin{pgfonlayer}{nodelayer}
		\node [style=none] (0) at (2, 0) {$\bullet$};
		\node [style=none] (99) at (1, 1.73) {$\bullet$};
		\node [style=none] (0) at (-1, 1.73) {$\bullet$};
		\node [style=none] (0) at (-2, 0) {$\bullet$};
		\node [style=none] (0) at (1, -1.73) {$\bullet$};
		\node [style=none] (0) at (-1, -1.73) {$\bullet$};
		\node [style=none] (1) at (1, 2.73) {};
		\node [style=none] (2) at (1, 0.73) {};
		\node [style=none] (3) at (1.25, 3) {$V$};
		\node [style=none] (4) at (2.25, 2.25) {$xv=\zeta v$};
	\end{pgfonlayer}
	\begin{pgfonlayer}{edgelayer}
		\draw [style=to] (99.center) to (1.center);
		\draw [style=to] (99.center) to (2.center);
	\end{pgfonlayer}
\end{tikzpicture}
\end{figure}

Since $\widehat G$ is in bijection with $\Hom_{\Group}(G,\mathbb C^\ast)$, we can equip $\widehat G$ with a group operation; by doing so, $G$ and $\widehat G$ are non-canonically isomorphic as groups. So in particular finite cyclic groups are (non-canonically) self-dual.

Next time, we will explore the Fourier transform, which for $G$ finite Abelian is a $\mathbb C[G]$-module isomorphism
\[\Fun(G)\xlongrightarrow{\widehat{-}}\Fun(\widehat G)\]
where $\Fun(G)$, $\Fun(\widehat G)$ are function spaces on $G,\widehat G$ and are given convolution and pointwise multiplication, respectively. Note that in this case $\Fun(G) \cong \mathbb C[G]$, so we give $\Fun(G)$ the corresponding action, which is given by $gf(x) = f(g^{-1}x)$. We will look at the action of $G$ on $\Fun(\hat G)\cong \mathbb C[\widehat G]$ next time. This isomorphism has some symmetry which is part of the statement of Pontryagin duality.

\newpage\section{09/02}
\subsection{More on the structure of $\widehat G$}
For $G$ a finite Abelian group, we saw/will see four equivalent descriptions of $\widehat G$:
\begin{align*}
	\widehat G &= \text{set of irreps of $G$}\\
	&= \Hom_{\Group}(G,\mathbb C^\ast)\\
	&= \text{unitary irreps of $G$} = \Hom_{\Group}(G,\U(1))\\
	&= \Spec(\mathbb C[G]) = \text{maximal ideals of $\mathbb C[G]$} = \Hom_{\mathbb C\alg}(\mathbb C[G],\mathbb C)
\end{align*}
Some of these descriptions for $\widehat G$ stop becoming equivalent to each other if we remove the adjectives finite or Abelian from $G$.

For any commutative ring $R$, recall that we think of $R$ as functions on $\Spec(R)$ with pointwise multiplication. In general there is no reason to expect $\Spec(R)$ to be a group, but we were able to give $\Spec(\mathbb C[G])$ a group operation when $G$ is Abelian. So in this situation something special is happening. 

The deeper lesson: Suppose now that $G$ is any group (not necessarily Abelian). If $V,W$ are representations of $G$ then we can form a new representation $V\otimes_{\mathbb C}W$ where $g(v\otimes w) = gv\otimes gw$. This comes from the diagonal embedding $\mathbb C[G]\xrightarrow{\Delta}\mathbb C[G]\otimes_{\mathbb C} \mathbb C[G]$ given by $\mathbb C$-linearly extending the assignment $g\mapsto g\otimes g$. By doing so $\Delta$ is a $\mathbb C$-algebra homomorphism; it suffices only to check that it respects the group multiplication:
\[\Delta(hg) = hg\otimes hg = (h\otimes h)(g\otimes g) = \Delta(h)\Delta(g)\]
The natural action of $\mathbb C[G]\otimes_{\mathbb C} \mathbb C[G]$ on $V\otimes_{\mathbb C} W$ given by $(f\otimes g)(v\otimes w) = fv\otimes gw$ can be pulled back along the diagonal embedding $\Delta$ to give $V\otimes_{\mathbb C} W$ the action of $G$ mentioned above.

If $A$ is a $\mathbb C$-algebra and $V,W$ are $A$-modules, there is a natural action of $A\otimes_\mathbb CA$ on $V\otimes_{\mathbb C}W$ given by $(a\otimes a^\prime)(v\otimes w) = av\otimes a^\prime w$. But the diagonal map $A\to A\otimes_{\mathbb C}A$ might not be a $\mathbb C$-algebra homomorphism, but only $\mathbb C$-linear. In this case there may not be a natural way for $A$ to act on $V\otimes_{\mathbb C}W$ like in the case of the group algebra. That the diagonal map $\mathbb C[G]\xrightarrow{\Delta}\mathbb C[G]\otimes_{\mathbb C} \mathbb C[G]$ for the group algebra is a $\mathbb C$-algebra homomorphism is part of the group algebra really being a Hopf algebra (\href{https://en.wikipedia.org/wiki/Hopf_algebra}{see Wikipedia}), and in this setting $\Delta$ is the comultiplication map.

If $V$ is a representation of $G$, another representation of $G$ we can form is the dual space $V^\ast\coloneqq \Hom_{\mathbb C}(V,\mathbb C)$, where the action of $G$ on $\mathbb C$ is the trivial action. The action of $G$ on $V^\ast$ is $(gf)(v) = f(g^{-1}v)$. We may not be able to  construct dual modules for an arbitrary $\mathbb C$-algebra $A$ and $A$-module $V$, since we need some way to invert the action of $A$ when ``passing the action inside the function''. This is related to one other datum that Hopf algebras have, which is their coinverse (also called antipode) map. In the case of the group algebra $\mathbb C[G]$, the coinverse map is $g\mapsto g^{-1}$ and this actually gives an isomorphism of $\mathbb C[G]$ and its opposite ring $\mathbb C[G]^{\op}$ (the ring where $g\cdot_{\op}h = hg$).

If $A$ is a Hopf algebra over $\mathbb C$ and $V,W$ are $A$-modules it is possible to define actions of $A$ on $V\otimes_{\mathbb C} W$ and on $V^\ast\coloneqq \Hom_{\mathbb C}(V,\mathbb C)$ in a manner analogous to the above by using the comultiplication and antipode maps on $A$, respectively.

If we want to think of the tensor product as giving a monoid structure to $\Rep_{\mathbb C}(G)$ (the category of dimensional representations of $G$), we might wonder if there is a way to obtain inverses in this category. So far, being able to take tensor products and duals of representations make $\Rep_{\mathbb C}(G)$ a rigid symmetric monoidal category; see the \href{https://ncatlab.org/nlab/show/rigid+monoidal+category}{nLab}. Another question is when the irreps of $G$ form a monoid or even a group; in the case that $G$ is not Abelian the tensor product of irreps need not be an irrep...

Returning to $G$ finite Abelian, in the description of $\widehat G$ as the collection of irreps of $G$, the group operation in $\widehat G$ is given by the tensor product of representations and inverses are given by taking dual spaces. Of course, the tensor product is associative and commutative in this case, and observe that $V\otimes_{\mathbb C}V^\ast \cong \End(V)\cong \mathbb C$ (since $V$ is one-dimensional). The action of $G$ on $V\otimes_{\mathbb C}V^\ast$ is also trivial: $g(v\otimes f) = gv\otimes gf = \chi_V(g) v\otimes \chi_V^{-1}(g)f = v\otimes f$.

\subsection{The Fourier transform on finite Abelian groups}
Let $G$ be a finite Abelian group. Then $G\times \widehat G$ has a distinguished complex-valued function $G\times \widehat G\xrightarrow{\chi}\mathbb C^\ast$ called the universal character, defined by 
\[\chi(g,\hat g) = \chi_{\hat g}(g)\quad \text{($=\hat g(g)$ if we think of $\hat g$ as a character)}\]
where $\chi_{\hat g}$ is the character corresponding to $\hat g\in \widehat G$ (here $\widehat G$ is thought of as a set of points). Observe that $\chi$ is multiplicative in each component; that is,
\[\chi(hg,\hat h\hat g) = \chi_{\hat h\hat g}(hg) = \chi_{\hat h\hat g}(h)\chi_{\hat h\hat g}(g) = \chi_{\hat h}(h)\chi_{\hat g}(h)\chi_{\hat h}(g)\chi_{\hat g}(g)\]

To each $g\in G$, the function $\chi_g\coloneqq \chi(g,-)\colon \widehat G\to\mathbb C^\ast$ is a group homomorphism: 
\[\chi_g(\hat h\hat g) = \chi(g,\hat h\hat g) = \chi_{\hat h}(g)\chi_{\hat g}(g) = \chi_g(\hat h)\chi_g(\hat g)\]
The assignment $g\mapsto \chi_g$ is a group homomorphism $G\xrightarrow{\chi_{-}} \Hom_{\Group}(\widehat G,\mathbb C^\ast)$ that informally speaking, takes an element $g$ to the map $\widehat G\to\mathbb C^\ast$ which evaluates a character at $g$:
\[\chi_{gh}(\hat g) = \chi(gh,\hat g) = \chi(g,\hat g)\chi(h,\hat g) = \chi_g(\hat g)\chi_h(\hat g)\quad \text{Thus $\chi_{gh} = \chi_g\cdot \chi_h$ (pointwise multiplication)}\]

Denote by $\dwidehat G$ the group $\Hom_{\Group}(\widehat G,\mathbb C^\ast)$. Since the kernel of $\chi_-$ is trivial and the image of $\chi_-$ is all of $\dwidehat G$, we obtain a (canonical) isomorphism of $G$ with $\dwidehat G$. This is the content of Pontryagin duality for finite Abelian groups. The same result is true for locally compact Abelian groups, but requires some more tools to prove.

An analyst cannot help but think of the universal character $\chi(-,-)$ as being similar to kernels of integral operators (which are usually denoted $K(-,-)$ for example), or in this finite setting it might be more accurate to think of the universal character as similar to a matrix. The similarity is no coincidence. The universal character $\chi(-,-)$ is the kernel for the Fourier transform.

The Fourier transform is a map $\Fun(G)\xrightarrow{\widehat{-}}\Fun(\widehat G)$ which we think of as a linear transformation with matrix $\chi(-,-)$. The Fourier transform is given by the formula
\[\hat f(\hat g) = \sum_{g\in G}\chi(g,\hat g)f(g)\]
For infinite groups that admit a Fourier transform like this, the sum is replaced by an integral of some kind and the kind of function spaces we consider may need adjustment.

One other perspective of this map is that it comes from a pullback and a pushforward. Consider the diagram 
% https://q.uiver.app/#q=WzAsMyxbMSwwLCJHXFx0aW1lcyBcXHdpZGVoYXQgRyJdLFswLDEsIkciXSxbMiwxLCJcXHdpZGVoYXQgRyJdLFswLDEsIlxccGlfMSIsMl0sWzAsMiwiXFxwaV8yIl1d
\[\begin{tikzcd}
	& {G\times \widehat G} \\
	G && {\widehat G}
	\arrow["{\pi_1}"', from=1-2, to=2-1]
	\arrow["{\pi_2}", from=1-2, to=2-3]
\end{tikzcd}\]
and consider a function $f\in \Fun(G)$. The Fourier transform $\hat f$ is given by pulling back $f$ along $\pi_1$, multiplying by $\chi(-,-)$, and then pushing forward along $\pi_2$. That is, 
\[\hat f = {\pi_2}_\ast(\pi_1^\ast f\cdot \chi)\]
The pushforward just means to sum (integrate) over fibers, and fiber of $\hat g$ is $\cbr{(g,\hat g)\mid g\in G}$. Pulling back $f$ along this projection does nothing but view $f$ as a function on $G\times \widehat G$; that is, $(\pi_1^\ast f)(g,\hat g) = f(g)$. This recovers the first formula above for the Fourier transform.

As a fun fact, we can think about matrix multiplication this way. An $n\times m$ matrix $A = (A_{ij})$ is a function on the set $\cbr{1,\dots,m}\times\cbr{1,\dots,n}$ and a vector $v$ in $\mathbb C^m$ is a function on $\cbr{1,\dots,m}$. There is a diagram 
% https://q.uiver.app/#q=WzAsMyxbMSwwLCJcXGNicnsxLFxcZG90cyxufVxcdGltZXNcXGNicnsxLFxcZG90cyxtfSJdLFswLDEsIlxcY2JyezEsXFxkb3RzLG19Il0sWzIsMSwiXFxjYnJ7MSxcXGRvdHMsbn0iXSxbMCwxLCJcXHBpXzEiLDJdLFswLDIsIlxccGlfMiJdXQ==
\[\begin{tikzcd}
	& {\cbr{1,\dots,m}\times\cbr{1,\dots,n}} \\
	{\cbr{1,\dots,m}} && {\cbr{1,\dots,n}}
	\arrow["{\pi_1}"', from=1-2, to=2-1]
	\arrow["{\pi_2}", from=1-2, to=2-3]
\end{tikzcd}\]
So $Av$ coincides with ${\pi_2}_\ast(\pi_1^\ast v\cdot A)$; that is, $(Av)_j = \sum_{i=1}^m A_{ij}v_i$ as expected. The same thing can be done for integral operators with kernel $K(-,-)$.

Denote by $\delta_g$ the Kronecker delta or the delta function(al) at $g$, or also the indicator function of $\cbr{g}$:
\[\delta_g(h) = \delta_{gh} = \begin{cases}
	1 & \text{if } g = h\\
	0 & \text{otherwise}
\end{cases}\]
Since $G$ is finite, these delta functions span $\Fun(G)$. The Fourier transform of the delta function is calculated by
\[\widehat{\delta_g}(\hat g) = \sum_{h\in G}\chi(h,\hat g)\delta_g(h) = \chi(g,\hat g), \quad\text{so }\widehat{\delta_g} = \chi_g\]
which we summarize by saying that Fourier transforms of delta functions are characters.

From the calculation 
\[(\delta_g\ast\delta_h)(x) = \sum_{y}\delta_g(y)\delta_h(y^{-1}x) = \delta_h(g^{-1}x) = \begin{cases}
	1 & \text{if } x = gh\\
	0 & \text{otherwise}
\end{cases}\]
we have $\delta_g\ast\delta_h = \delta_{gh}$. Recalling that $\chi_{gh} = \chi_g\cdot \chi_h$, conclude from the linearity of convolution that for any $f,h\in \Fun(G)$ that $\widehat{f\ast h} = \hat f\cdot \hat h$, which is summarized by saying that the Fourier transform turns convolution to pointwise multiplication. In other words, the Fourier transform diagonalizes the action of $G$ on $\Fun(G)$ since $gf = \delta_g\ast f$. So in order for the Fourier transform to be a $\mathbb C[G]$-module homomorphism, $G$ should act on $\Fun(\widehat G)$ by pointwise multiplication by $\chi_g$; that is, $gF(x) = \chi_g(x)F(x)$ for $F\in \Fun(\widehat G)$.

\subsection{Fourier inversion on finite Abelian groups}
We will show that the Fourier transform has a $G$-linear inverse and hence is an isomorphism. 

From some of the above calculations, we have that $g\delta_h = \delta_g\ast \delta_h = \delta_{gh}$. This equation shows that the action of $G$ on the delta functions is by pushforward via left multiplication. So in reality the delta function in this case is a distribution or measure as opposed to a function.

Let $G$ be any finite group. The left multiplication of $G$ on itself gives rise to a natural right action on functions by pullback along the left multiplication map and a natural left action on distributions (or measures) by pushforward via the right action map on $\Fun(G)$. That is, given a complex-valued function $G\xrightarrow{f}\mathbb C$, the right action of $G$ on $f$ is given by $fg = (g\cdot)^\ast f = f(g-)$:
% https://q.uiver.app/#q=WzAsMyxbMCwwLCJHIl0sWzEsMCwiRyJdLFsxLDEsIlxcbWF0aGJiIEMiXSxbMCwxLCJnXFxjZG90Il0sWzEsMiwiZiJdLFswLDIsImYoZy0pPShnXFxjZG90KV5cXGFzdCBmIiwyXV0=
\[\begin{tikzcd}[ampersand replacement=\&]
	G \& G \\
	\& {\mathbb C}
	\arrow["{g\cdot}", from=1-1, to=1-2]
	\arrow["{f(g-)=(g\cdot)^\ast f}"', from=1-1, to=2-2]
	\arrow["f", from=1-2, to=2-2]
\end{tikzcd} \quad \text{pulling back $f$ along $g\cdot$}\]
The right action on $\Fun(G)$ defines a map $\Fun(G)\xrightarrow{\cdot g}\Fun(G)$, by which we will pushforward distributions. Given a distribution on $G$; that is, a linear function $\Fun(G)\xrightarrow{T}\mathbb C$, the left action of $G$ on $T$ is given by $gT = (\cdot g)_\ast T = T(-g^{-1})$:
% https://q.uiver.app/#q=WzAsMyxbMCwwLCJcXEZ1bihHKSJdLFsxLDAsIlxcRnVuKEcpIl0sWzAsMSwiXFxtYXRoYmIgQyJdLFswLDEsIlxcY2RvdCBnIl0sWzAsMiwiVCIsMl0sWzEsMiwiKFxcY2RvdCBnKV9cXGFzdCBUPSBUKC1nXnstMX0pIl1d
\[\begin{tikzcd}[ampersand replacement=\&]
	{\Fun(G)} \& {\Fun(G)} \\
	{\mathbb C}
	\arrow["{\cdot g}", from=1-1, to=1-2]
	\arrow["T"', from=1-1, to=2-1]
	\arrow["{(\cdot g)_\ast T= T(-g^{-1})}", from=1-2, to=2-1]
\end{tikzcd} \quad \text{pushing forward $T$ by $\cdot g$}\]
Here, $(\cdot g)_\ast T = T(-g^{-1})$ because
\[(\cdot g)_\ast T(f) = \sum_{\substack{h\\hg=f}}T(h) = T(fg^{-1})\quad \text{(only one element in the fiber)}\]
Typically, for a function $f\colon X\to Y$ and a distribution $F\colon \Fun(X)\to \mathbb C$, the pushforward $(f^\ast-)_\ast F$ is denoted by $f_\ast F$, and this is what is meant by the pushforward of the distribution $F$ by $f$ (similarly for measures).

It is instructive to look at the left action of $G$ on the regular distributions, which are the distributions $T_f$ obtained by integration against $f\in\Fun(G)$:
\[T_f(h) = \sum_{g\in G}f(g)h(g)\]
and the action of $G$ on a regular distribution looks like
\begin{multline*}
	(gT_f)(h) = ((\cdot g)_\ast T_f)(h) = T_f(hg) = T_f(h(g-)) \\ = \sum_{x\in G} f(x)h(gx) = \sum_{x\in G} f(g^{-1}x)h(x) = \sum_{x\in G} (fg^{-1})(x)h(x) = T_{fg^{-1}}(h)
\end{multline*}
To summarize, for any $f\in \Fun(G)$, $gT_f = T_{fg^{-1}}$. For finite groups, every distribution on $G$ is a regular distribution. Given a distribution $T$ on $G$, define $f\in \Fun(G)$ by $f(g) = T(\delta_g)$. Then by linearity of $T$ the calculation
\[T_f(\delta_h) = \sum_{x\in G}f(x)\delta_h(x) = \sum_{x\in G}T(\delta_x)\delta_h(x) = T(\delta_h)\]
implies that $T = T_f$ (and that $f$ is unique).

Compare the original left action of $G$ (not the natural one defined by pullbacks above) on the function $\delta_h$ with the left action of $G$ on the regular distribution $T_{\delta_h}$:
\[g\delta_h = \delta_{gh}\quad \text{and}\quad gT_{\delta_h} = T_{\delta_hg^{-1}} = T_{\delta_{gh}}\]
So this is a reason for why we mentioned above that the delta function was really a distribution or measure. The original definition of the left action of $G$ on $\Fun(G)$ may be thought of as turning the natural right action of $G$ on $\Fun(G)$ into a left action by being clever with signs, or more interestingly as identifying all functions with their corresponding regular distributions.

Now return to the case when $G$ is finite Abelian. View $\chi_{\hat g}$ as an element of $\Fun(G)$. The calculation
\[g\chi_{\hat g}(h) = \chi_{\hat g}(g^{-1}h) = \chi_{\hat g}(g^{-1})\chi_{\hat g}(h) =\chi_{\hat g}(g)^{-1}\chi_{\hat g}(h) = \overline{\chi_{\hat g}(g)}\chi_{\hat g}(h)\]
shows that under the action of $G$ on $\Fun(G)$, $\chi_{\hat g}$ is an eigenvector with eigenvalue $\chi_{\hat g}(-)^{-1} = \overline{\chi_{\hat g}(-)}$. In the theory of Fourier series (which we will see more later), a character of $\U(1) = \mathbb R/\mathbb Z$ is of the form $x\mapsto \exp(2\pi inx)$. The action of $\mathbb R/\mathbb Z$ on $\Fun(\mathbb R/\mathbb Z)$ is given by $(x\cdot f)(\theta) = f(\theta-x)$, so
\[y\cdot\exp(2\pi inx) = \exp(2\pi in(x-y)) = \overline{\exp(2\pi i n y)}\exp(2\pi inx)\]
(See \href{https://terrytao.wordpress.com/2009/04/06/the-fourier-transform/#more-2015}{Terence Tao's blog}.)

Earlier we showed that $\widehat{\delta_g} = \chi_g$, and that extending this linearly produces the Fourier transform. We would like for the inverse Fourier transform to send $\delta_{\hat g}$, the indicator function of $\hat g$, to a character, to match the Fourier transform. (Note that the space $\Fun(\widehat G)$ has basis the indicator functions $\delta_{\hat g}$.) But the inverse Fourier transform should also respect the group action on the function spaces. Compared to the action of $G$ on $\Fun(G)$, the action of $G$ on $\Fun(\widehat G)$ is given by identifying $G$ with $\dwidehat{G}\subset \Fun(\widehat G)$ and using pointwise multiplication. That is, $gf = \chi_g\cdot f$. We should try another calculation to see if that provides a clue; in particular, if we want to have a chance at defining the inverse Fourier transform we should try to take Fourier transforms of characters. The Fourier transform of $\chi_{\hat g}\in \Fun(G)$ is given by 
\[\widehat{\chi_{\hat g}}(\hat h) = \sum_{h\in G}\chi(h,\hat h)\chi_{\hat g}(h) = \sum_{h\in G}\chi(h,\hat h\hat g) = \abs{G}\delta_{\hat g^{-1}},\]
where in the last equality we used the following result:
\[\sum_{g\in G}\chi_{g}(\hat h) = \begin{cases*}
	0 & \text{if } $\hat h \neq 1_{\widehat G}$\\
	\abs{G} & \text{otherwise}
\end{cases*}\]
Indeed, since $\dwidehat{G}$ is a group, if $\hat h\neq 1_{\widehat G}$, then choose $h\in G$ for which $\chi_{\hat h}(h) = \chi_h(\hat h)\neq 1$ (we can do this since $\chi_{\hat h}$ is not the trivial character). Then with $\chi_h\in \dwidehat{G}$, we have
\[\chi_h(\hat h)\sum_{g\in G}\chi_{g}(\hat h) = \sum_{g\in G}\chi_{hg}(\hat h) = \sum_{g\in G}\chi_{g}(\hat h),\]
but $\chi_h(\hat h)\neq 1$, so $\sum_{g\in G}\chi_{g}(\hat h)=0$. If $\hat h = 1_{\widehat G}$, then $\sum_{g\in G}\chi_{g}(\hat h) = \sum_{g\in G}1 = \abs{G}$. A similar argument may be done to evaluate the sum $\sum_{\hat g\in \widehat G}\chi_{\hat g}(h)$.

Based on the calculations above, one might try to calculate the Fourier transform of $\overline{\chi_{\hat g}}$, to obtain $\abs{G}\delta_{\hat g}$. So the inverse Fourier transform should send $\delta_{\hat g}$ to $\chi_{\hat g}/\abs{G}$, and by extending linearly we find that the inverse Fourier transform $\Fun(\widehat G)\xrightarrow{-^{\vee}}\Fun(G)$ is given by
\[F^\vee(g) = \frac{1}{\abs{G}}\sum_{\hat g \in \widehat G}\overline{\chi(g,\hat g)}F(\hat g)\]
One can calculate that for any $f\in \Fun(G)$ that 
\[f(g) = \frac{1}{\abs{G}}\sum_{\hat g\in \widehat G}\overline{\chi(g,\hat g)}\hat f(\hat g)\]
which is the Fourier inversion theorem for finite Abelian groups.

By endowing $\Fun(G)$ with the Hermitian $L^2$ inner product
\[\abr{f,h} = \sum_{g\in G}f(g)\overline{h(g)}\]
it turns out that the characters $\chi_{\hat g}$ form an orthonormal basis (use one of the earlier calculations to do this). A similar statement is true for $\Fun(\widehat G)$, and by scaling the Fourier transform, the inverse Fourier transform, and possibly the inner products correctly, we obtain an unitary isomorphism of the Hilbert spaces $\Fun(G)$ and $\Fun(\widehat G)$ (i.e., the inner product is preserved).

\subsection{A preview for locally compact Abelian groups}
Some examples of locally compact Abelian (LCA) groups are the finite Abelian groups, $\mathbb Z$, $\U(1)$, $\mathbb R$, $\mathbb Z_p$ (the $p$-adic integers), and $\mathbb Q_p$ (the $p$-adic numbers). In this setting $\widehat G$ is defined to be the set of unitary irreps, otherwise given by $\Hom_{\Group}(G,\U(1))$.

Pontryagin duality still holds in this setting. The result encapsulates the following statements:
\begin{enumerate}
	\item The canonical map $G\to\dwidehat{G}$ (where $\dwidehat G = \Hom_{\Group}(\widehat G,\U(1))$) is an isomorphism.
	\item The Fourier transform and its inverse are Hilbert space isomorphisms (i.e. unitary isomorphisms) of $L^2(G)$ with $L^2(\widehat G)$ (with the correct measures on $G$, $\widehat G$). The transforms interchange convolution with pointwise multiplication, and sends delta ``functions'' to characters (neither of which are $L^2$ functions, so this is meant in the sense of distributions).
	\item A spectral theorem: Representations of LCA groups are in correspondence with sheaves of vector spaces on $\widehat G$. Given a finite dimensional representation $V$ of $G$, decompose $V$ into its invariant subspaces (otherwise called isotypic components) $V = \bigoplus_{\hat g}V_{\hat g}$ where $V_{\hat g} = \cbr{v\in V\mid gv = \chi_{\hat g}(g)v}$. The Fourier transform in this setting takes the representation (sheaf) $\Fun(G)$ and spits out the representation (sheaf) $\Fun(\widehat G) = \bigoplus_{\hat g\in\widehat G}\mathbb C$...
\end{enumerate}
Soon we will look at the LCA groups $\U(1)$ and $\mathbb R$. There we can recover the usual Fourier series and real Fourier transform theory.

\newpage\section{09/04}
\subsection{Unitarizable representations and semisimplicity}
A category is called semisimple if every object is the direct sum of finitely many simple objects. We will show that $\Rep_{\mathbb C}(G)$ is semisimple for $G$ finite.

A complex representation of a group $G$ is unitarizable if there exists a Hermitian inner product (non-degenerate, sesquilinear, positive definite bilinear form) $\abr{-,-}$ on $V$ for which 
\[\abr{gv,gw} = \abr{v,w}\]
In other words, $V$ is unitarizable if it can be endowed with a Hermitian inner product that is invariant under the group action, or otherwise we can say that the group acts by unitary transformations on $V$ with this inner product. In the language of diagrams, given the representation $(V,\rho)$, there exists an inner product $\abr{-,-}$ and the map $G\xrightarrow{\rho} \U(V,\abr{-,-})$ (with the same definition as $G\xrightarrow{\rho} \GL(V)$) for which the following diagram commutes
% https://q.uiver.app/#q=WzAsMyxbMCwwLCJHIl0sWzEsMCwiXFxHTChWKSJdLFsxLDEsIlxcVShWLFxcYWJyey0sLX0pIl0sWzIsMSwiIiwwLHsic3R5bGUiOnsidGFpbCI6eyJuYW1lIjoiaG9vayIsInNpZGUiOiJ0b3AifX19XSxbMCwyLCJcXHJobyIsMix7InN0eWxlIjp7ImJvZHkiOnsibmFtZSI6ImRhc2hlZCJ9fX1dLFswLDEsIlxccmhvIl1d
\[\begin{tikzcd}
	G & {\GL(V)} \\
	& {\U(V,\abr{-,-})}
	\arrow["\rho", from=1-1, to=1-2]
	\arrow["\rho"', dashed, from=1-1, to=2-2]
	\arrow[hook, from=2-2, to=1-2]
\end{tikzcd}\]

A non-example of a unitarizable representation: If $\mathbb Z$ acts on a one-dimensional vector space $V$ and $1\in \mathbb Z$ acts by an operator outside of $\U(V)$, then there is no way to unitarize $V$.

Given a representation $(V,\rho)$, recall we can define the dual representation $V^\ast$ where $gf(v) = f(g^{-1}v)$. In this case we think of the action of $G$ as being inverted since we use the inversion map on $G$ to define the action on $V^\ast$. On the other hand, since the action of $G$ on $V$ is by complex-valued matrices, we can pointwise conjugate these matrices to obtain a representation $(\overline V,\overline \rho)$ where the underlying Abelian group $\overline V$ coincides with the Abelian group $V$ but the scalar multiplication is given by $c\cdot_{\overline V} v = \overline{c}\cdot_Vv$. Extending this, the group $G$ acts on $\overline V$ by $gv = \overline{\rho(g)}v$, and in this case we think of the action as being conjugated as opposed to inverted in the case with the dual vector space.

A Hermitian inner product $\abr{-,-}$ on any complex vector space $V$ corresponds to an isomorphism of $\overline V$ with $V^\ast$ given by $v\mapsto \abr{-,v}$, and from any isomorphism $\overline V\xrightarrow{H} V^\ast$, define the inner product $\abr{-,-}$ by $\abr{v,w} = H(v)(w)$.

That a representation $V$ is unitarizable is a property of $V$ as opposed to being part of the structure. On the other hand, if $W$ is a unitary representation, part of the data of $W$ is a fixed $G$-invariant inner product $\abr{-,-}$, as opposed to a unitarizable representation $V$ for which we do not specify any one $G$-invariant inner product. Of course, once we do specify a $G$-invariant inner product $\abr{-,-}_0$, we can speak of the unitary representation $(V,\abr{-,-}_0)$, which is different from the unitary representation $(V,\abr{-,-}_1)$ for a different $G$-invariant inner product $\abr{-,-}_1$.

If $V$ is a finite-dimensional unitarizable representation, then $V$ is a semisimple representation; that is, $V$ is the direct sum of irreps/simples. Assume $V$ is already not simple, and fix a $G$-invariant inner product $\abr{-,-}$ on $V$. Let $W\subset V$ be any $G$-invariant subspace, and see that $W^\perp = \cbr{v\in V\mid \abr{v,w} = 0\text{ for all }w\in W}$ is a $G$-invariant subspace of $V$ since the $W$ and the inner product $\abr{-,-}$ are $G$-invariant:
\[\abr{gv,w} = \abr{v,g^{-1}w}\quad\text{implies $gv\in W^\perp$ whenever $v\in W^\perp$}\]
Then induction proves the result (this amounts to iterating this procedure on $W^\perp$ and repeating on each of the resulting subspaces until it cannot be done, which happens because $V$ is finite dimensional).

Another nice result is that if $G$ is finite, then a finite dimensional representation $V$ is unitarizable and hence semisimple. The heart of this result is due to the existence of an averaging element $\av\in\mathbb C[G]$ given by
\[\av = \frac{1}{\abs{G}}\sum_{g\in G}g\]
Note that $\av\in Z(\mathbb C[G])$. To define this averaging element, it is crucial that $G$ is finite and that the characteristic of the field $\mathbb C$ (zero) does not divide the order of the group. The existence of an averaging operator is the content of Maschke's theorem from earlier, since we can use this averaging element to form a projector needed to form complements of $G$-stable subspaces, as we will see. 

Let $V$ be any finite dimensional representation. Then we can define the subspace of $G$-fixed points, $V^G$, by
\[V^G = \cbr{v\in V\mid gv = v\text{ for all }g\in G}\]
The subspace $V^G$ is the part of $V$ that $G$ acts trivially on, or in other words is the trivial representation part of $V$ when broken into irreps. For any $v\in V$, $av\cdot v\in V^G$ since for any $h\in G$ we have
\[h(\av\cdot v) = h\biggl(\frac{1}{\abs{G}}\sum_{g\in G}gv\biggr) = \biggl(\frac{1}{\abs{G}}\sum_{g\in G}hgv\biggr) = \biggl(\frac{1}{\abs{G}}\sum_{g\in G}gv\biggr) = \av\cdot v\]
Observe further that because the sum is normalized with the factor $1/\abs{G}$, multiplication by $\av$ defines a projection map to $V^G$, since $\av^2 = \av$, so $\av-$ is an idempotent. The complementary idempotent is $(1-\av)-$, and see that it squares to itself as well. It is complementary since $\av(1-\av) = 0$. It follows that $V = \av V\oplus (1-\av)V = V^G\oplus (1-\av)V$.

An aside: To prove Maschke's theorem, we do a similar trick. Given any $G$-invariant subspace $H$ of a representation $V$, consider the space $\Hom_{\mathbb C}(V,H)$, which contains many projections since short exact sequences in $\Vect_{\mathbb C}$ always split, or this is just true from linear algebra. What we want to find is a projector $\pi$ which is $G$-intertwining; that is, $(g\cdot)\pi(g^{-1}\cdot) = \pi$. Considering the action of $G$ on $\Hom_{\mathbb C}(V,H)$ by conjugation, finding a $G$-intertwining projector $\pi$ amounts to finding a projector that is fixed under conjugation. Pick any one projector $V\xrightarrow{\pi_0}H$, and average $\pi_0$ using $\av$ to get the projector
\[\pi = \av\cdot\pi_0 = \frac{1}{\abs{G}}\sum_{g\in G}(g\cdot)\pi_0(g^{-1}\cdot)\]
Hence $\pi$ is a fixed point under the conjugation action and thus is a desired $G$-equivariant projector.

Returning to showing that a finite dimensional representation $V$ is unitarizable and hence semisimple, it amounts to finding a $G$-invariant inner product $\abr{-,-}$. Pick any inner product $\abr{-,-}_0$ on $V$ and make it $G$-invariant by averaging it via $\av$. Define the inner product $\abr{-,-}$ by 
\[\abr{-,-} = ``\av(\abr{-,-}_0)\textrm{''} = \frac{1}{\abs{G}}\sum_{g\in G}\abr{g-,g-}_0\]
Observe that this new inner product is $G$-invariant as desired, and is really a Hermitian inner product because $\abr{-,-}_0$ was to begin with. (In other words, we have found an isomorphism $H\in \Hom_{\mathbb C}(\overline V, V^\ast)^G$ by applying the projection $\Hom_{\mathbb C}(\overline V, V^\ast)\to \Hom_{\mathbb C}(\overline V, V^\ast)^G$ obtained via averaging to any isomorphism $H_0\in \Hom_{\mathbb C}(\overline V, V^\ast)$.) Since $V$ is unitarizable, $V$ is semisimple. This shows that $\Rep_{\mathbb C}(G)$ is semisimple. The slickness of the argument above gives a compelling reason to use the group algebra $\mathbb C[G]$.

\subsection{Distributions and measures}
When $G$ is finite, then the group algbera $\mathbb C[G]$ can be identified with the algebra $\Fun(G)$ with convolution, and we identify $\av$ with $\mathbf{1}_G/\abs{G}$ ($\mathbf{1}_G$ is the indicator function on $G$) and identify $1_{\mathbb C[G]}$ with $\delta_{1_{G}}$, and everything in between. If $G$ is also Abelian, then $\Fun(G)$ with convolution is isomorphic to $\Fun(\widehat G)$ with pointwise multiplication via the Fourier transform. A short calculation shows that the Fourier transform of $\delta_{1_G}$ is $\mathbf{1}_{\widehat G} = \sum_{\hat g\in\widehat G}\delta_{\hat g}$, which can be thought of as some version of Plancherel's theorem.

So if $G$ acts on $V$, then we know that $V$ decomposes as $V = \bigoplus_{\hat g\in \widehat G}V_{\hat g}$, where $V_{\hat g} = \delta_{\hat g}V$, where we think of the delta functions as being projectors onto subspaces of $V$ for which $G$ acts by multiplication by characters. The notation is suggestive: $g\cdot \delta_{\hat g}V = \chi_{\hat g}(g)\delta_{\hat g}V$.

When we consider groups $G$ that are not finite, we may need to consider a smaller function space than all of $\Fun(G)$ in order for the theory to be nicer. In doing so, the functions $\delta_{1_G}$, $\mathbf{1}_G/\abs{G}$ cease to belong in these new function spaces or even in $\Fun(G)$ to begin with, and should be thought of as distributions or measures. For example, this will happen with $G = \U(1)$ or $\mathbb R$ since we will look at $L^2(G)$ primarily. For example, we think of $\mathbf{1}_G/\abs{G}$ more accurately as the distribution $1/\mu(G)\int_G-\dd\mu$ for a correctly chosen measure $\mu$ on $G$, for $G$ compact, and $\delta_{1_G}$ as the functional which evaluates functions at $1_G$.

Because the convolution of two functions is given by an integral or otherwise a pushforward, secretly somehow what is actually happening is that we are pushing forward a particular distribution or measure. ``You don't integrate functions, you integrate measures.'' So this is more evidence that we should see if we can do an analysis of the representations of infinite Abelian groups using distributions instead of functions.

% The convolution of two distributions $F,H$ in this setting is given by $F\ast H = m_\ast(F\boxtimes H)$ where $\Fun(G)\times \Fun(G)\xrightarrow{F\boxtimes H} \mathbb C$ is given by $(F\boxtimes H)(f,h) = Ff\cdot Hh$ and $m$ is the multiplication map $G\times G\to G$. Explicitly, 
% \[(F\ast H)(f) = m_\ast(F\boxtimes H)(f) = (m^\ast-)_\ast(F\boxtimes H)(f) = \]

Let $G$ be a locally compact Hausdorff topological group. Then Haar's theorem (see \href{https://en.wikipedia.org/wiki/Haar_measure}{Wikipedia}) states that there is a measure $\mu$ on the Borel subsets of $G$, called the Haar measure, which is:
\begin{enumerate}
	\item Left-translation invariant under the left multiplication action of $G$; that is, the pushforward measure $(g\cdot)_\ast \mu$ agrees with $\mu$.
	\item Finite on compact sets of $G$.
	\item Outer regular on Borel sets; that is, $\mu(S) = \inf\cbr{\mu(U)\mid S\subseteq U\text{ with $U$ open}}$ for $S$ a Borel set. 
	\item Inner regular on open sets; that is, $\mu(U) = \sup\cbr{\mu(K)\mid K\subseteq U\text{ with $K$ compact}}$ for $U$ an open set.
	\item Unique up to positive scaling.
\end{enumerate}

If $G$ is compact then we can normalize the Haar measure on $G$ to produce a unique Haar measure for which $\mu(G) = 1$. This measure is provided by $\av$, viewed as a measure by the formula $\av(E) = \frac{1}{\mu(G)}\int_E 1 \dd\mu$ where $\mu$ is any Haar measure, or as the regular distribution $T_\av$ where $T_\av(f) = \int_G f(g)\av(g)\dd\mu = \frac{1}{\mu(G)}\int_G f(g) \dd\mu$ where $\mu$ is any Haar measure. So for example, the normalized Haar measure on $\U(1)$ is $\dd\theta/2\pi$.

On a manifold, to be able to integrate we need to be able to find a top differential form. So if $G$ is a Lie group, consider the tangent space $\mathfrak g = T_G(1_G)$, which is the only one we need to consider since all other tangent spaces are obtained via translation of the tangent space at the identity. Then the space of top differential forms $\bigwedge^{\dim \mathfrak g}\mathfrak g^\ast$ is a $1$-dimensional vector space. The normalized Haar measure is the measure induced by the normalized top form in this vector space.

The requirement that $G$ is locally compact probably comes from wanting to be able to approximate integrals using compact exhaustions. The Hausdorff requirement is a consequence of the following result of topological groups: For $G$ a topological group, let $H$ be the closure of the set $\cbr{1_G}$. Then $H$ is a normal subgroup and the quotient $G/H$ is the largest quotient of $G$ that is Hausdorff. Furthermore, every continuous map of $G$ to a Hausdorff space factors through the quotient $G/H$. So in our setting, we will be considering continuous maps from $G$ to the complex numbers, so in this setting $G$ is indistinguishable from its largest Hausdorff quotient. As a consequence, there is no loss in defining LCA groups to be locally compact, Abelian, and Hausdorff.

A fun fact is that if $G$ is compact, then the Haar measure on $G$ is automatically both left and right-translation invariant, even if $G$ is not Abelian. For any Haar measure $\mu$ on $G$ and $g\in G$, the pushforward of $\mu$ by right multiplication by $g^{-1}$ produces another left invariant Haar measure $(\cdot g)_\ast\mu$. So $\mu$ is some multiple of $(\cdot g)_\ast\mu$, denote this multiple by $\bmod(g)$, called the modulus character of $G$. It turns out that this quantity does not depend on the choice of Haar measure $\mu$ we started with, and is also a group homomorphism $G\to \mathbb R_{>0}$, which measures how much the left and right-translation invariance of the Haar measure disagree. The modulus character is trivial if $G$ is Abelian. For $G$ compact, there are no compact subgroups of $\mathbb R_{>0}$ aside from $\cbr{1}$, so in this case the modulus character is also trivial, and hence Haar measures on compact groups are bi-invariant. In other words, the Haar measure is a $G\times G$-invariant measure on $G$ (where $(g,h)\in G\times G$ acts by left translation by $g$ and by right translation by $h^{-1}$). This is because the element $\av\in\mathbb C[G]$ is $G\times G$-invariant (where $(g,h)\in G\times G$ acts by left multiplication by $g$ and by right multiplication by $h^{-1}$); this is stronger than just being invariant under conjugation by $G$.

\subsection{A tiny preview for LCA groups}
Let $G$ be compact. Then finite dimensional representations of $G$ are semisimple. The proof follows the same ideas as in the finite case, but we replace all sums with integrals using the normalized Haar measure. Note that the normalization of the Haar measure is what will ensure that our desired projector actually is an idempotent as we did in the finite case.

For LCA groups, it is also true that finite dimensional representations are semisimple, and from before we know the irreps are one-dimensional. If $G = \U(1) = S^1$, then we obtain the theory of Fourier series, since the dual group $\widehat{S^1}$ is isomorphic to $\mathbb Z$; each character of $S^1$ is of the form $z\mapsto z^n$ (or $x\mapsto \exp(2\pi i nx)$ if we think of $S^1$ as $\mathbb R/\mathbb Z$) for some integer $n$.

\newpage\section{09/09}
\subsection{Fourier series}
Let $G = \U(1)$, which we know familiarly as the circle. Since $G$ is compact, representations of $G$ are semisimple, and since $G$ is Abelian, irreps of $G$ are one-dimensional. We could decide to stop with the theory here, since at this point we have classified all of the finite-dimensional representations using the dual group $\widehat G$ from before.

The dual group of the circle $\widehat{\U(1)}$ is $\mathbb Z$, where $n\in\mathbb Z$ corresponds to the character $z\mapsto z^n$, or if we think of $U(1)$ as $\mathbb R/\mathbb Z$, the character $x\mapsto \exp(2\pi inx)$. We will think of the circle as $\mathbb R/\mathbb Z$ usually, to appease the analysts (this is a joke). In this setting, the universal character $\chi(-,-)$ is the map $(x,n)\mapsto \exp(2\pi i nx )$ (so $\chi_x = \chi(x,-)$ and $\chi_n = \chi(-,n)$). Then the Fourier transform as we mentioned before is obtained from the maps in the diagram 
\[\begin{tikzcd}
	& {G\times \widehat G} \\
	G && {\widehat G}
	\arrow["{\pi_1}"', from=1-2, to=2-1]
	\arrow["{\pi_2}", from=1-2, to=2-3]
\end{tikzcd}\]
by pulling back $f\in \Fun(G)$ along $\pi_1$, multiplying by $\chi(-,-)$, and then pushing forward along $\pi_2$. That is, 
\[\hat f = {\pi_2}_\ast(\pi_1^\ast f\cdot \chi)\quad\text{so }\hat f(n) = \int_0^1 f(x)\exp(2\pi inx)\dd x\]

The Fourier inversion theorem in this setting is
\[f(x) \approx \sum_{n\in\mathbb Z}\hat f(n)\exp(-2\pi i nx)\]
where the $\approx$ means that we should worry about convergence. It ends up being the case that we get convergence in the $L^2$ norm, but more analysis is needed for better results.

Another point of view of the Fourier transform is to look at $G = \U(1)= \mathbb R/\mathbb Z$ acting on $L^2(G)$ by translation. The action is 
\[y\cdot f \eqqcolon \tau_yf = f((-)-y)\quad \text{for }y\in\mathbb R/\mathbb Z\]
The group action preserves each of the subspaces $\mathbb C\chi_n$ since like before, we have $\tau_y\chi_n(x) = \chi_{-n}(y)\chi_n(x)$. We can think of the Fourier transform in linear algebra terms, since $L^2(G)$ has the Hermitian inner product $\abr{f,g} = \int_0^1f(x)\overline{g(x)}\dd x$. Some analysis shows that the characters $\chi_n$ form an orthonormal basis of $L^2(G)$, so the Fourier transform finds the components of any $f\in L^2(G)$ using the inner product:
\[\hat f(n) = \abr{f,\chi_{-n}} = \int_0^1 f(x)\exp(2\pi inx)\dd x\]
The inverse Fourier transform reassembles $f$ from its projections onto each of the subspaces $\mathbb C\chi_n$, but as we noted before, we might need to worry about convergence:
\[f(x) \approx \sum_{n\in\mathbb Z}\abr{f,\chi_{-n}}\chi_{-n} = \sum_{n\in\mathbb Z}\hat f(n)\exp(-2\pi i nx)\]
Analysts are likely used to the characters being maps $x\mapsto \exp(inx)$ and the Fourier transform of $f\in L^2(\U(1))$ as $\frac{1}{2\pi}\int_0^{2\pi}f(x)\exp(-inx)\dd x$; to obtain this form we can present $U(1)$ not by $\mathbb R/\mathbb Z$ but as unimodular points in $\mathbb C$, with the Haar measure given by arclength measure, and by reindexing the characters by interchanging $n$ with $-n$.

\subsection{Function spaces}
An algebraic point of view is to view the algebra spanned by all of the characters as the algebraic functions on the circle; that is,
\[\bigoplus_{n\in\mathbb Z}\mathbb C\chi_n\cong\mathbb C[z,z^{-1}]\quad\text{where $z = \exp(2\pi ix)$ since $x\in \mathbb R/\mathbb Z$}\]
The coordinate ring of the affine space $\mathbb C$ is $\mathbb C[z]$. By localizing the coordinate ring at the maximal ideal $m_0 = \cbr{f\in \mathbb C[z]\mid f(0) = 0} = (z)$, we obtain the algebraic functions on $\mathbb C^\ast$, which agree with the algebraic functions on the circle if we restrict $z$ to $z = \exp(2\pi i x)$ for $x\in\mathbb R/\mathbb Z$.

An analytic point of view is to see this space of algebraic functions as being dense in function spaces like $L^2(G)$, $L^1(G)$, $C^\infty(G)$, $C^\omega(G)$ (analytic functions), $C^{-\infty}(G)$ (distributions), or even $C^{-\omega}(G)$ (hyperfunctions?). Let us suppress the $(G)$ for now. Each of these spaces has the algebraic functions above as an ``algebraic core'', for which we understand the Fourier transform on. The question is how the action of the Fourier transform extends to the rest of the function spaces, and this requires analysis. The Fourier transform exchanges the following spaces (this is not to say the Fourier transform is always an isomorphism of these spaces!):
\begin{align*}
	L^2 &\longleftrightarrow \ell^2 & L^1 &\longleftrightarrow c_0\\
	C^\infty &\longleftrightarrow \cbr{f\to 0 \text{ faster than } 1/\text{polynomial}} & C^\omega &\longleftrightarrow \cbr{f\to 0 \text{ faster than } 1/\text{exponential}}\\
	C^{-\infty} &\longleftrightarrow \cbr{f\to \infty \text{ polynomial order}} & C^{-\omega} &\longleftrightarrow \cbr{f\to \infty \text{ exponential order?}}
\end{align*}

It is actually the case that the Fourier transform gives an isomorphism of $L^2$ with $\ell^2$. Harish-Chandra studied similar function spaces for non-Abelian groups and found similar exchanges as above.

More than just an orthonormal basis, the algebraic core of characters is the basis in which the action of $G$ is diagonalized via the Fourier transform. This is due to the property that the Fourier transform interchanges convolution of functions with pointwise multiplication of functions. In the setting of the circle group, this is to recast the action of translation as convolution against a delta distribution and to use the Fourier transform to obtain pointwise multiplication by a scalar:
\[\tau_yf = \delta_y\ast f\xmapsto{\widehat{-}} \widehat{\delta_y\ast f} = \exp(2\pi i(-)y)\hat f.\]
If $f$ is the character $\chi_n$, then $\tau_y$ acts on $f$ by multiplication by the scalar $\exp(-2\pi iny)$, and this action commutes with the Fourier transform. In general we should worry about what functions it makes sense to convolve against; after all, we are convoluting functions with distributions, so some analysis must be done. If we restrict to continuous functions, then the analysis of $C^\ast$-algebras will appear, for example.

Since $\U(1)$ is thought of as continuous as opposed to discrete, there is an ``infinitesimal'' action of $\U(1)$ on functions given by taking the derivative. The Fourier transform takes differentiation to multiplication by a multiple of the identity function:
\[\widehat{\dv{x}f}(n) = -2\pi in\hat f\]
If $f$ is the character $\chi_m$, then the derivative of $f$ is $2\pi i m f$. The Fourier transform of $2\pi i m f$ is $2\pi i m \delta_{-m}$, so the action of the derivative commutes with the Fourier transform.

What is meant by the ``infinitesimal'' action $\dv{x}$? From the action of $\U(1) = \mathbb R/\mathbb Z$, we define an action of its Lie algebra $\mathbb R$ (the tangent space at the identity $0$ of $\mathbb R/\mathbb Z$) on a function by $y\cdot f = \dv{t}(\tau_{ty}f)|_{t=0}$. In this case, $\dv{x}$ agrees with the action of $-1$ on functions; that is, $\dv{x}f = \dv{t}(\tau_{-t}f)|_{t=0}$. If we want to insist on working in $\U(1)$ given by the unimodular complex numbers, then the Lie algebra of $\U(1)$ in this case is $i\mathbb R$. The action of $i\mathbb R$ on $g$ (here $g$ is a function of $\theta$; if $g$ is a function of $z$, put $z = \exp(i\theta)$) is given by $iy \cdot g = \dv{t}(\exp(iyt)\cdot g)|_{t=0}$. In this case $\dv{\theta}$ is given by the action of $-i$ on functions; that is, $\dv{\theta}g = \dv{t}(\exp(-it)\cdot g)|_{t=0}$.

Due to Pontryagin duality, we can mirror the above theory by considering taking Fourier transforms of functions on $\mathbb Z$ to get functions on $\widehat{\mathbb Z} = \U(1)$. The universal character $\chi(-,-)$ in this case is the same map (but we should flip the inputs) $(n,x)\mapsto \exp(2\pi inx)$ Even though $\mathbb Z$ is a discrete group, we can still think about difference operators in place of derivatives (these are just linear combinations of differences of translation operators). As desired, the action of $\mathbb Z$ on a function is given by translation, and the Fourier transform exchanges this action with multiplication by a character. That is, for $n\in\mathbb Z$
\[n\cdot g \eqqcolon \tau_ng = \delta_n\ast g = g((-)-n)\xmapsto{\widehat{-}}\widehat{\delta_n\ast g} = \exp(2\pi in(-)) \hat g\]
As before, we should also expect difference operators to correspond to a multiplier operator after taking the Fourier transform.

The Fourier transform on $\mathbb Z$, which we also denote by $\mathcal F_{\mathbb Z}$, is given by $\hat g(x) = \sum_{n\in \mathbb Z}g(n)\exp(2\pi i nx)$, which is different from the inverse Fourier transform of the Fourier transform $\mathcal F_{\U(1)}$ from $\U(1)$ to $\mathbb Z$. Each Fourier transform has order four. Note also that by composing the two Fourier series together, we get $\mathcal F_{\mathbb Z}\mathcal F_{\U(1)}f(x) = f(-x)$. Similarly, $\mathcal F_{\U(1)}\mathcal F_{\mathbb Z}g(n) = g(-n)$. 

Even though we understand well how the Fourier transform acts on delta distributions and characters, there is the issue of figuring out how the Fourier transforms extend to larger function spaces. There should be a nice way to package the theory together nicely, in a way that is agnostic to the exact functions appearing in the larger spaces. A nice way to package everything together is via the spectral theorem for the group $\U(1)$.

\subsection{The spectral theorem for $\U(1)$}
Let $\mathcal H$ be any unitary representation of $\U(1)$ (note that some of the function spaces we mentioned earlier are not Hilbert spaces!). Since we have diagonalized the action of $\U(1)$, it acts on $\mathcal H$ by compact operators (in this case the operators are approximated by finite-rank diagonal operators). By the spectral theorem for compact operators, we can find countably many eigenspaces of $\mathcal H$ whose direct sum is dense in $\mathcal H$. We find these eigenspaces explicitly using the characters $\chi_n$.

The first part of this is to find analogues of the group algebra in this setting. On one hand, compactly supported functions on $G = \mathbb R/\mathbb Z$ would be a good starting place, but we can choose even larger spaces for which the product, convolution, still makes sense. For example, suitable candidates for a group algebra might be the space of continuous or even $L^1$ functions on $G$. For now, consider the continuous functions on $G$ with convolution, denoted by $(C(G),\ast)$. A unitary representation $(\mathcal H,\rho)$ of $G$ is in correspondence with a non-degenerate, bounded ${}^\ast$-representation (whatever this means, but note that this new representation is not unitary) $\pi_\rho$ of the algebra $C(G)$. Define $\pi_\rho$ by the \href{https://en.wikipedia.org/wiki/Bochner_integral}{Bochner integral}
\[\pi_\rho(f) = \int_0^1 f(x)\rho(x)\dd x\]
In other words, $f$ acts on $v$ by ``convolution against $v$'':
\[f\cdot v = ``f\ast v\textrm{''} = \int_0^1 f(x)(x\cdot v)\dd x\]
In this setting the action of the character $\chi_n$ (which is an element of $C(G)$) on $\mathcal H$ is an orthogonal idempotent; that is, the map $\chi_n\ast -$ satisfies 
\[(\chi_m\ast -)(\chi_n\ast -) = (\chi_m\ast\chi_n)\ast - = \delta_{mn}\chi_n\ast -\]
To see this, use the Fourier transform: It suffices to see that $\widehat{\chi_m\ast\chi_n} = \delta_m\cdot \delta_n = \delta_{mn}\delta_n$ (here $\delta_{mn}$ is the usual Kronecker delta).

The subspaces $\mathcal H_n = \chi_n\ast \mathcal H$ of $H$ combine to form $\mathcal H^{\textrm{alg}} = \bigoplus_{n\in\mathbb Z}\mathcal H_n$. It turns out that $\mathcal H^{\textrm{alg}}$ is dense in $\mathcal H$. As expected, $\U(1)$ acts by pointwise multiplication by characters on the isotypic components of $\mathcal H$, the summands in $\mathcal H^{\textrm{alg}}$. Like before, we can get a sheaf picture where to each point $n$ of $\mathbb Z$ we place above it the vector space $\mathcal H_n$. The picture is a little different since the Hilbert space $\mathcal H$ is the closure of $\mathcal H^{\textrm{alg}}$.
\begin{align*}
  \mathcal H &= \overline{\mathcal H_1 \oplus \mathcal H_2 \oplus \cdots \oplus \mathcal H_\ell}\\
  \mathbb Z&~~~~~1~~\phantom{\oplus}~~2~~\phantom{\oplus} \cdots \phantom{\oplus}~\,\ell
\end{align*}

The density of $\mathcal H^{\textrm{alg}}$ is due to some version of Plancherel's theorem in this setting. The algebra $C(G)$ does not have a unit, but if it did, it should have been the delta function $\delta_{1_G}$ (which is not a continuous function, or much less a function at all). This is evidence that we should have taken some larger space to be the analogue of the group algebra in this setting; that is, perhaps some space of distributions (with convolution as product) would have been a better choice. If we believe that such an analogue exists, then $\delta_{1_G}$ acts as the identity operator on $\mathcal H$:
\[\id_{\mathcal H}v = v = \delta_{1_G}\ast v\]
But by the Fourier transform $\delta_{1_G}$ corresponds to the unit for multiplication in the analogue of the group algebra for $\mathbb Z$ with pointwise multiplication, which is the constant function $\mathbf 1_{\mathbb Z}$ ($\mathbf 1_{\mathbb Z}(n) = 1$, notably not finitely supported). Since $\mathbf 1_{\mathbb Z} = \sum_{n\in\mathbb Z}\delta_n$, taking the inverse Fourier transform shows that $\id_{\mathcal H} = \sum_{n\in\mathbb Z}\chi_n\ast -$; this shows that $\mathcal H^{\textrm{alg}}$ is dense in $\mathcal H$.

\subsection{Chebyshev polynomials of the first kind}
The Chebyshev polynomials (of the first kind) are special functions, like the characters given by either $n$ or $x$ maps to $\exp(2\pi i nx)$ that play nice with the Fourier theory.

\begin{figure}[h]
	\centering
	\begin{tikzpicture}
		\draw[-] (2,0) arc[
			start angle = 0,
			end angle = -330,
			x radius = 2,
			y radius = 2
		];
		\draw[->] (2,0) arc[
			start angle = 0,
			end angle = 30,
			x radius = 2,
			y radius = 2
		];
	\begin{pgfonlayer}{nodelayer}
		\node [style=none] (0) at (-2.5, -0.5) {$-1$};
		\node [style=none] (1) at (2.5, -0.5) {$1$};
		\node [style=none] (2) at (-2, 0) {};
		\node [style=none] (3) at (2, 0) {};
		\node [style=none] (5) at (0.5, -0.5) {$x$};
		\node [style=none] (6) at (0.5, 0) {};
		\node [style=none] (7) at (2.25, 1.25) {$\theta$};
	\end{pgfonlayer}
	\begin{pgfonlayer}{edgelayer}
		\draw [style=to] (2.center) to (6.center);
		\draw (6.center) to (3.center);
	\end{pgfonlayer}
	\end{tikzpicture}
\end{figure}
Comparing the real and imaginary parts in de Moivre's formula $(\cos(\theta) + i\sin(\theta))^n = \cos(n\theta) + i\sin(n\theta)$, one can prove that $\cos(n\theta)$ is a polynomial in $\cos(\theta)$. The change of coordinates $x = \cos(\theta) = (\exp(i\theta)+\exp(-i\theta))/2$ can now be used to turn power series in $\exp(i\theta),\exp(-i\theta)$ (Fourier series) representing even functions on the circle to power series in $x$. We can use a different change of coordinates $\sin(\theta) = (\exp(i\theta)-\exp(-i\theta))/2i$ to handle odd functions on the circle; these will lead to Chebyshev polynomials of the second kind.

The $n$-th Chebyshev polynomial of the first kind $T_n(x)$ is defined to be the polynomial for which
\[T_n(\cos(\theta)) = \cos(n\theta)\]
That $\cbr{\cos(n\theta)}$ is an orthogonal basis for a suitable space of even functions on the circle corresponds to $\cbr{T_n(x)}$ being an orthogonal basis of a suitable space of functions on $[-1,1]$ with weighted $L^2$ inner product $\abr{f,g} = \int_{-1}^1f(x)g(x)/\sqrt{1-x^2}\dd x$.
Some of the first few Chebyshev polynomials of the first kind are 
\begin{align*}
	T_0(x) &= 1 & T_1(x) &= x\\
	T_2(x) &= 2x^2-1& T_3(x) &=4x^3-3x \\
	T_4(x) &= 8x^4-8x^2+1& T_5(x) &= 16x^5-20x^3+5x
\end{align*}

We should think of the $T_n(x)$ as characters; in particular we can undo the change of coordinates $x = \cos(\theta)$ to obtain the universal character $(n,\theta)\mapsto T_n(\cos(\theta)) = \cos(n\theta) = (\exp(in\theta)-\exp(-in\theta))/2$.

These polynomials satisfy the recurrence relation
\[T_{n+1}(x) = 2xT_n(x) - T_{n-1}(x)\]
which should be thought of as a second order difference equation in $n$; that is, a discrete or otherwise integral version of a second order ODE. The Fourier transform takes shift operators to multiplier operators, so in this setting, undoing the change of coordinates $x = \cos(\theta)$ and applying a suitable Fourier transform to the resulting recurrence relation would return a second order algebraic equation (a quadratic equation) for $\cos(n\theta)$ in $\theta$.

These polynomials also satisfy the differential equation
\[(1-x^2)\dv[2]{x}T_n(x)-x\dv{x}T_n+ n^2T_n = 0\]
in $x$, so we should obtain a second order algebraic equation in $n$ after applying a suitable Fourier transform to the above differential equation. That the above difference equation and differential equation are of second order should not be surprising since the universal character is $\cos(n\theta)$ in this setting.

From the point of view of trigonometric polynomials, for some space of functions the set
\[\cbr{\underline{1},\underline{\cos(\theta), \sin(\theta)}, \underline{\cos(2\theta),\sin(2\theta)},\dots}\]
is an orthogonal basis. The underlined groups of functions are eigenspaces for the Laplacian $\Delta_{\U(1)} = \pdv[2]{\theta}$ on the circle. We can obtain these functions in a different way by first considering harmonic polynomials in the plane; that is, polynomials $p(x,y)$ for $x,y$ real which satisfy $\Delta_{\mathbb R^2}p = \bigl(\pdv[2]{x} + \pdv[2]{y}\bigr)p = 0$. Then consider the homogeneous polynomials in the plane; these are the polynomials $p$ for $p(kx,ky) = k^{\deg(p)}p(x,y)$; in polar coordinates these polynomials are separable with $p(r,\theta) = r^{\deg p}\tilde p(\theta)$ for some function $\tilde p(\theta)$ ($\deg p$ is the homogeneous degree of $p$). If $p$ is a harmonic homogeneous polynomial,
\[0 = \Delta_{\mathbb R}p(r,\theta) = \biggl(\dv[2]{r}+\frac{1}{r}\dv{r}+\frac{1}{r^2}\dv[2]{\theta}\biggr)r^{\deg p}\tilde p(\theta) = r^{\deg p - 2}\biggl((\deg p)^2\tilde p(\theta)+\dv[2]{\theta}\tilde p(\theta)\biggr)\]
By restricting to the circle $r = 1$, observe that $\tilde p$ is an eigenfunction of the Laplacian on the circle, which by the theory of ordinary differential equations tells us that $\tilde p$ is some linear combination of $\cos(n\theta)$ and $\sin(n\theta)$ for $n = \deg p$.

That the real eigenspaces (that aren't $\mathbb R\{1\}$) for the Laplacian on the circle are two-dimensional comes from the action of the Laplacian on a suitable function space on the circle. Other differential operators may have different eigenspaces, of possibly different dimensions. In a Lie algebra, there is no notion of squaring elements, only taking their bracket. The universal enveloping algebra for a Lie algebra is an algebra which makes it possible to multiply elements of the Lie algebra together, and from a representation of a Lie group we should obtain a module over the universal enveloping algebra in a similar way to how we did so with the group algebra. We saw before how to obtain the derivative $\dv{\theta}$ from the action of the Lie algebra of $\U(1)$ on functions. In the universal enveloping algebra it makes sense to square the element producing the derivative (it was $-1$ for the Lie algebra of $\mathbb R/\mathbb Z$ or $-i$ for the Lie algebra of $\U(1)$), which would then act on functions as the Laplacian.

\newpage\section{09/11}
\subsection{The Fourier transform on $\mathbb R$}
Let $G$ be the real line $\mathbb R$, which henceforth we will denote by $\mathbb R_x$ to indicate that the coordinate on $\mathbb R$ is $x$. Like before, because $G$ is an LCA group we know that the irreps of $G$ are its one-dimensional unitary representations, which are parameterized by $\widehat G = \mathbb R_t$ (again, this notation means the real line with coordinate $t$, to distinguish this real line with the real line defining $G$). The universal character $\chi$ in this setting is the map $(x,t)\mapsto \exp(ixt)$. 

Let $\mathcal H$ be a unitary representation of $G$; that is, $\mathcal H$ is a Hilbert space that the real line acts on by unitary operators. In this case, the image of $\mathbb R$ in $\U(\mathbb H)$ is a one-parameter subgroup $\cbr{U_x}_{x\in\mathbb R_x}$ of $\U(\mathbb H)$. By Stone's theorem on one-parameter unitary groups (\href{https://en.wikipedia.org/wiki/Stone%27s_theorem_on_one-parameter_unitary_groups}{Wikipedia}), it follows that $U_x = \exp(ixH)$ for some self-adjoint operator $H$ on $\mathcal H$. So the data of a unitary representation $\mathcal H$ of $G$ corresponds to the data of a single self-adjoint operator on $\mathcal H$. By differentiating this action, that is, passing to the action of the Lie algebra of $G$ on $\mathcal H$ we find that the Lie algebra of $G$ acts by the operator $iH$ on $\mathcal H$; this element is called the infinitesimal generator for the group $\cbr{U_x}_{x\in\mathbb R_x}$.

One example of this story is in quantum mechanics, where $\mathcal H$ is the space of states for a quantum mechanical system and $\mathbb R$ acts on $\mathcal H$ by time evolution. The infinitesimal generator of the one-parameter group $\cbr{U_x}_{x\in\mathbb R_x}$ in this setting is the system's Hamiltonian.

From the spectral theorem for self-adjoint operators on Hilbert spaces, we obtain a kind of dictionary between
\[\{\text{irreps of unitary reps}\}\longleftrightarrow\{\text{eigenvalues in $\mathbb R$}\}\]
where the left side lives in the world of representation theory and the right side lives in the world of self-adjoint operator theory. Specifically, for a fixed unitary representation $\mathcal H$ of $G$, decomposing $\mathcal H$ into its irreps amounts to finding the eigenvalues of the operator $H$ defining the representation. More generally, the decomposition of unitary representations $\mathcal H$ of $G$ corresponds to the spectrum of of the operator $H$ defining the representation. We will see this later.

The universal character $\chi$ is used to obtain the Fourier transform $\Fun(G)\to \Fun(\widehat G)$ as usual. It is given by $f\mapsto \hat f$ where
\[\hat f(t) = \int_{\mathbb R_x}f(x)\exp(itx)\dd x,\]
and the Fourier inversion theorem in this setting is 
\[f(x)\approx \int_{\mathbb R_t} \hat f(t)\exp(-itx)\dd t,\]
where as usual we should worry about convergence. We should also care about what functions we apply the Fourier transform to. For example, a nice result from analysis is that the Fourier transform defines a unitary isomorphism of $L^2(\mathbb R_x)$ with $L^2(\mathbb R_t)$ (that the isomorphism is unitary is usually known as Plancherel's theorem).

The Fourier transform diagonalizes the translation operators like before. The group $G$ acts naturally on $\Fun(G)$ by translation:
\[y\cdot f \eqqcolon \tau_yf = f((-)-y)\quad \text{for }y\in\mathbb R_x\]
and one can calculate the infinitesimal version of this as differentiation. The group algebra version of this action is given by convolution; that is, an element $f$ in the group algebra acts on $\Fun(G)$ by the operator $f\ast -$. By group algebra we might mean a nicer function space like continuous or smooth functions with compact support or even $L^1$ functions, so it makes sense to convolve, but these are matters of analysis. The point is that the Fourier transform takes these translation actions and simultaneously diagonalizes them, turning them into multiplication operators on $\Fun(\widehat G)$. In particular, translation $\tau_y$ is sent to pointwise multiplication by $\chi_y$, differentiation is sent to multiplication by $-it$, and convolution against $f$ is sent to pointwise multiplication by $\hat f$.

The statement that $L^2(\mathbb R_x)$ is isomorphic to $L^2(\mathbb R_t)$ can instead be thought of as a consequence of the spectral theory of the operator $H = -i\dv{x}$. This operator should be thought of as an infinitesimal translation operator, which under the Fourier transform is sent to multiplication by $-t$ (a diagonal operator). The exponential of $H$ yields a translation operator, and the exponential of $-t$??????????

% \newpage\section{XX/XX}

% \newpage\section*{Exercises}
% These were suggested problems for the class.

\newpage\section*{References}
The references below were from the course page.
{\footnotesize
\subsection*{\hspace{1em}Basic Lie theory}
 R.~Carter, G.~Segal and I.MacDonald: Lectures on Lie Groups and Lie Algebras. London Math. Society Student Texts 32. The lectures by Segal are a beautiful overview of the fundamental ideas of Lie groups and algebras, with geometry and examples emphasized.

W.~Fulton and J.~Harris: Representation Theory. Springer GTM 129. The canonical reference for representations, especially for the (Lie) algebraic point of view and basic algebro--geometric aspects.
\subsection*{\hspace{1em}Review articles}
G.~Mackey: Harmonic Analysis as the Exploitation of Symmetry. Bull. Amer. Math. Soc. 3/1 (1980) 543-699. Reprinted in: The Scope and History of Commutative and Noncommutative Harmonic Analysis. History of Math Vol.5, Amer. Math Soc./London Math Soc. 1992. My favorite (and first) introduction to representation theory, emphasizing its history and origins in probability, number theory and physics. The reprint appears in a volume devoted to Mackey's wonderful representation theory survey articles.

Other Mackey surveys (besides those in the book above): Unitary Group Representations in physics, probability and number theory (Addison-Wesley 1989) -- a book delving deeper into the ideas of the above survey article.

W.~Schmid: Representations of semi-simple Lie groups. In: Atiyah et al., Representation Theory of Lie Groups. Londom Math Society Lecture Notes 34. Cambridge U. Press 1979. Excellent overview by a master, in a volume full of useful reviews (also Mackey, Bott, Kostant, Kazhdan..)

W.~Schmid: Analytic and Geometric Realization of Representations. In: Tirao and Wallach (eds)., New Developments in Lie Theory and their Applications. Lecture notes overviewing the subject in the title, with a strong emphasis on $\SL_2$.

R.~Howe, \href{https://utexas.instructure.com/courses/1428938/files/86306747?wrap=1}{On the role of the Heisenberg group in harmonic analysis}.
\subsection*{\hspace{1em}More advanced, $\SL_2$ specific references}
K.~Vilonen: Representations of $\SL_2$. Course Notes, Northwestern University.

R.~Howe and E.C.~Tan: Non-abelian harmonic analysis - applications of $\SL(2,\mathbb R)$. Springer Universitext. A very nice introduction to representations of $\SL(2,\mathbb R)$, with interesting applications to classical analysis.

V.S.~Varadarajan: An Introduction to Harmonic Analysis on Semisimple Lie Groups. Cambridge Studies in advanced math 16. Nice textbook, with an emphasis on $\SL_2$ and good introductions to various topics.

S.~Lang: $\SL(2,\mathbb R)$. Springer GTM. Fairly analytical introduction, alas not very coherent -- read the \href{http://www.sunsite.ubc.ca/DigitalMathArchive/Langlands/miscellaneous.html#lang}{review} by \href{http://www.sunsite.ubc.ca/DigitalMathArchive/Langlands}{Robert Langlands}
\subsection*{\hspace{1em}Other}
D.~Ramakrishnan and R.~Valenza: Fourier Analysis on Number Fields. Springer GTM 186. Contains an introduction to the Pontrjagin duality theory for locally compact abelian groups, and its applications to number theory (Tate's thesis).

J.~Arthur: Harmonic Analysis and Group Representations. Notices of the AMS. Gorgeous introduction to the ideas of Harish-Chandra. Available off AMS website.

I.~Gelfand, M.~Graev and I.~Piatetskii-Shapiro: Representation Theory and Automorphic Functions. Academic Press. A classic thorough study of representations of SL2 over real, p-adic and adelic fields.

D.~Bump: Automorphic Forms and Representations. Cambridge Studs. Advanced Math. 55. A good and detailed introduction to Langlands program ideas focussed on representations of GL2.

T.~Bailey and A.~Knapp (eds): Representation Theory and Automorphic Forms. Proc. Symp. Pure Math 61. Proceedings of an instructional conference, with a variety of great introductory articles. (In particular Schmid, Knapp and Langlands).

J.~Bernstein and S.~Gelbart (eds): An Introduction to the Langlands Program. Birkhauser. A very timely collection of introductions to the basic constituents of the Langlands program.

A.~Borel and W.~Casselman (eds): Automorphic Forms, Representations and L-functions (the Corvallis volumes). The standard source of information about the Langlands program. Available off the AMS website (see e.g. Arinkin's webpage, below).

J.~Bernstein, Courses on Eisenstein Series and Representations of p-adic Groups. Beautiful expositions. Available at \href{http://www.math.uchicago.edu/~arinkin/langlands}{Dima Arinkin's Langlands page}

\href{http://www.math.ucsd.edu/~wgan}{Wee Teck Gan}, Automorphic Forms and Automorphic Representations: slides for a series of five lectures given in Hangzhou, China giving an excellent overview of the basic theory (available on his web page).}

% \newpage
% \begin{bibdiv}
% \begin{biblist}

% \bib{w}{book}{
%     title = {Algebraic Geometry I: Schemes},
%     author = {G\"ortz, Ulrich},
%     author = {Wedhorn, Torsten},
%     isbn = {978-3-658-30732-5},
%     series = {Springer Studium Mathematik - Master},
%     year = {2020},
%     publisher = {Springer Spektrum Wiesbaden}
% }

% \bib{gw}{book}{
%     title = {An Introduction to Homological Algebra},
%     author = {Weibel, Charles A.},
%     isbn = {9781139644136},
%     series = {Cambridge Studies in Advanced Mathematics},
%     year = {1994},
%     publisher = {Cambridge University Press}
% }

% \bib{h}{book}{
%     title = {Algebraic Geometry},
%     author = {Hartshorne, Robin},
%     isbn = {978-0-387-90244-9},
%     series = {Graduate Texts in Mathematics},
%     year = {1977},
%     publisher = {Springer New York, NY}
% }

% \end{biblist}
% \end{bibdiv}
\end{document}