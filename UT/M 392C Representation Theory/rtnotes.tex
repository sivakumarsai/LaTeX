\documentclass[11pt,leqno]{article}
\headheight=13.6pt

% packages
\usepackage[alphabetic]{amsrefs}
\usepackage{physics}
% margin spacing
\usepackage[top=1in, bottom=1in, left=0.5in, right=0.5in]{geometry}
\usepackage{amsfonts, amsmath, amssymb, amsthm}
\usepackage{fancyhdr}
\usepackage{graphicx}
\graphicspath{{./images/}}
\usepackage{float}
\usepackage{color}
\newcommand{\sai}[1]{\textcolor{red}{#1}}
\usepackage{mathrsfs}
\usepackage{hyperref}
\usepackage[noabbrev, capitalise]{cleveref}
\crefformat{equation}{equation~#2#1#3}
\crefformat{lemma}{\textrm{Lemma}~#2#1#3}
\usepackage{extarrows}
\usepackage{quiver}
\usepackage{tikzit}
\input{rt.tikzstyles}
\usepackage{titlesec}
\titleformat{\section}
  {\bfseries}{Lecture \thesection}{1em}{}
\titleformat{\subsection}
  {\bfseries}{}{1em}{}

% theorems
\theoremstyle{plain}
\newtheorem{lem}{Lemma}
\newtheorem{lemma}[lem]{Lemma}
\newtheorem{thm}[lem]{Theorem}
\newtheorem{theorem}[lem]{Theorem}
\newtheorem{prop}[lem]{Proposition}
\newtheorem{proposition}[lem]{Proposition}
\newtheorem{cor}[lem]{Corollary}
\newtheorem{corollary}[lem]{Corollary}

\theoremstyle{definition}
\newtheorem{defn}[lem]{Definition}
\newtheorem{definition/}[lem]{Definition}
\newenvironment{definition}
  {\renewcommand{\qedsymbol}{\textdagger}%
   \pushQED{\qed}\begin{definition/}}
  {\popQED\end{definition/}}
\newtheorem{example}[lem]{Example}
\newtheorem{remark}[lem]{Remark}

\numberwithin{equation}{section}
\numberwithin{lem}{section}

% header/footer formatting
\pagestyle{fancy}
\fancyhead{}
\fancyfoot{}
\fancyhead[L]{M 392C}
\fancyhead[C]{}
\fancyhead[R]{Representation Theory}
\fancyfoot[R]{\thepage}
\renewcommand{\headrulewidth}{1pt}

% paragraph indentation/spacing
\setlength{\parindent}{0cm}
\setlength{\parskip}{10pt}
\renewcommand{\baselinestretch}{1.25}

% operators and commands
\newcommand{\eq}[1]{\overset{(#1)}{=}}
\DeclareMathOperator{\im}{im}
\DeclareMathOperator{\GL}{GL}
\DeclareMathOperator{\Hom}{Hom}
\DeclareMathOperator{\id}{id}
\DeclareMathOperator{\Aut}{Aut}
\DeclareMathOperator{\End}{End}
\DeclareMathOperator{\SO}{SO}
\DeclareMathOperator{\SL}{SL}
\DeclareMathOperator{\PSL}{PSL}
\DeclareMathOperator{\Fun}{Fun}
\DeclareMathOperator{\Spec}{Spec}
\DeclareMathOperator{\U}{U}
\DeclareMathOperator{\Ann}{Ann}
\DeclareMathOperator{\Heis}{Heis}
\DeclareMathOperator{\SU}{SU}

% bracket notation for inner product
\usepackage{mathtools}
\DeclarePairedDelimiterX{\abr}[1]{\langle}{\rangle}{#1}

% categories
\newcommand{\catname}[1]{{\normalfont\mathbf{#1}}}
\renewcommand{\op}{\mathrm{op}} % where is \op defined?
\newcommand{\Rmod}{R\text{-}\catname{mod}}
\newcommand{\Vect}{\catname{Vect}}
\newcommand{\Rep}{\catname{Rep}}
\newcommand{\alg}{\text{-}\catname{alg}}
\newcommand{\Group}{\catname{Group}}
\newcommand{\Ring}{\catname{Ring}}
\newcommand{\av}{\mathrm{av}}

% double hat, widecheck macros
\makeatletter
\newcommand{\dwidehat}[1]{% 
\begingroup%
  \let\macc@kerna\z@%
  \let\macc@kernb\z@%
  \let\macc@nucleus\@empty%
  \hspace{0.165em}\widehat{\hspace{-0.165em}\raisebox{.3ex}{\vphantom{\ensuremath{#1}}}\smash{\widehat{#1}}}%
\endgroup%
}
\makeatother

\makeatletter
\newcommand{\dhat}[1]{% 
\begingroup%
  \let\macc@kerna\z@%
  \let\macc@kernb\z@%
  \let\macc@nucleus\@empty%
  \hat{\raisebox{.3ex}{\vphantom{\ensuremath{#1}}}\smash{\hat{#1}}}%
\endgroup%
}
\makeatother

% smileys frownies
\usepackage{wasysym}
\newcommand{\smallhappy}{\raisebox{-.14em}{\smiley}}
\newcommand{\happy}{\raisebox{-.24em}{\resizebox{1.2em}{!}{\smiley}}}
\newcommand{\smallsad}{\raisebox{-.14em}{\frownie}}
\newcommand{\sad}{\raisebox{-.24em}{\resizebox{1.2em}{!}{\frownie}}}
\DeclareMathOperator{\mathhappy}{\!\happy\!}
\DeclareMathOperator{\smallmathhappy}{\!\smallhappy\!}
\DeclareMathOperator{\mathsad}{\!\sad\!}
\DeclareMathOperator{\smallmathsad}{\!\smallsad\!}

% array column and row separation
\arraycolsep = 2pt
\renewcommand{\arraystretch}{.8}

% set page count index to begin from 1
\setcounter{page}{1}

% subfiles
\usepackage{subfiles}

\begin{document}
M 392C Representation Theory, Fall 2025: Dr.~David Ben-Zvi's lectures, \TeX ed by Sai Sivakumar. The main goal of these notes was to clarify and even expand on several of the points David made during the lectures. As a warning, I made minimal effort to reorder the content of the course, and many digressions in these notes have additional details not discussed in class.
\tableofcontents

\newpage\subfile{./sections/0826/0826.tex}
\newpage\subfile{./sections/0828/0828.tex}
\newpage\subfile{./sections/0902/0902.tex}
\newpage\subfile{./sections/0904/0904.tex}
\newpage\subfile{./sections/0909/0909.tex}
\newpage\subfile{./sections/0911/0911.tex}
\newpage\subfile{./sections/0916/0916.tex}
\newpage\subfile{./sections/0918/0918.tex}
% \newpage\subfile{./sections/0920/0920.tex}
% \newpage\subfile{./sections/XXXX/XXXX.tex}
% \newpage\subfile{./sections/XXXX/XXXX.tex}
% \newpage\subfile{./sections/XXXX/XXXX.tex}
% \newpage\subfile{./sections/XXXX/XXXX.tex}
% \newpage\subfile{./sections/XXXX/XXXX.tex}
% \newpage\subfile{./sections/XXXX/XXXX.tex}
% \newpage\subfile{./sections/XXXX/XXXX.tex}
% \newpage\subfile{./sections/XXXX/XXXX.tex}
% \newpage\subfile{./sections/XXXX/XXXX.tex}
% \newpage\subfile{./sections/XXXX/XXXX.tex}
% \newpage\subfile{./sections/XXXX/XXXX.tex}
% \newpage\subfile{./sections/XXXX/XXXX.tex}
% \newpage\subfile{./sections/XXXX/XXXX.tex}
% \newpage\subfile{./sections/XXXX/XXXX.tex}
% \newpage\subfile{./sections/XXXX/XXXX.tex}
% \newpage\subfile{./sections/XXXX/XXXX.tex}
% \newpage\subfile{./sections/XXXX/XXXX.tex}
% \newpage\subfile{./sections/XXXX/XXXX.tex}
% \newpage\subfile{./sections/XXXX/XXXX.tex}
% \newpage\subfile{./sections/XXXX/XXXX.tex}
% \newpage\subfile{./sections/XXXX/XXXX.tex}
% \newpage\subfile{./sections/XXXX/XXXX.tex}
% \newpage\subfile{./sections/XXXX/XXXX.tex}
% \newpage\subfile{./sections/XXXX/XXXX.tex}
% \newpage\subfile{./sections/XXXX/XXXX.tex}
% \newpage\subfile{./sections/XXXX/XXXX.tex}
% \newpage\subfile{./sections/XXXX/XXXX.tex}
% \newpage\subfile{./sections/XXXX/XXXX.tex}
% \newpage\subfile{./sections/XXXX/XXXX.tex}
% \newpage\subfile{./sections/XXXX/XXXX.tex}
% \newpage\subfile{./sections/XXXX/XXXX.tex}
% \newpage\subfile{./sections/XXXX/XXXX.tex}
% \newpage\subfile{./sections/XXXX/XXXX.tex}
% \newpage\subfile{./sections/XXXX/XXXX.tex}
% \newpage\subfile{./sections/XXXX/XXXX.tex}


% \newpage\subfile{./sections/XXXX/XXXX.tex}
\newpage\subfile{./sections/ref/ref.tex}
\end{document}