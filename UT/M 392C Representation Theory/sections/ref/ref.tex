\section*{References}
The references below were from the course page. (There are other references throughout the text that I should put in here?)
{\footnotesize
\subsection*{\hspace{1em}Basic Lie theory}
 R.~Carter, G.~Segal and I.MacDonald: Lectures on Lie Groups and Lie Algebras. London Math. Society Student Texts 32. The lectures by Segal are a beautiful overview of the fundamental ideas of Lie groups and algebras, with geometry and examples emphasized.

W.~Fulton and J.~Harris: Representation Theory. Springer GTM 129. The canonical reference for representations, especially for the (Lie) algebraic point of view and basic algebro--geometric aspects.

D. Milicic, \href{https://www.math.utah.edu/~milicic/Math_6260/weyl_character.pdf}{The Weyl Character Formula}. Explicit derivation for $\SU(2)$ not using anything ``external'' (Part of a course on representation theory.)
\subsection*{\hspace{1em}Review articles}
G.~Mackey: Harmonic Analysis as the Exploitation of Symmetry. Bull. Amer. Math. Soc. 3/1 (1980) 543-699. Reprinted in: The Scope and History of Commutative and Noncommutative Harmonic Analysis. History of Math Vol.5, Amer. Math Soc./London Math Soc. 1992. My favorite (and first) introduction to representation theory, emphasizing its history and origins in probability, number theory and physics. The reprint appears in a volume devoted to Mackey's wonderful representation theory survey articles.

Other Mackey surveys (besides those in the book above): Unitary Group Representations in physics, probability and number theory (Addison-Wesley 1989) -- a book delving deeper into the ideas of the above survey article.

W.~Schmid: Representations of semi-simple Lie groups. In: Atiyah et al., Representation Theory of Lie Groups. Londom Math Society Lecture Notes 34. Cambridge U. Press 1979. Excellent overview by a master, in a volume full of useful reviews (also Mackey, Bott, Kostant, Kazhdan..)

W.~Schmid: Analytic and Geometric Realization of Representations. In: Tirao and Wallach (eds)., New Developments in Lie Theory and their Applications. Lecture notes overviewing the subject in the title, with a strong emphasis on $\SL_2$.

R.~Howe, \href{https://utexas.instructure.com/courses/1428938/files/86306747?wrap=1}{On the role of the Heisenberg group in harmonic analysis}.
\subsection*{\hspace{1em}More advanced, $\SL_2$ specific references}
K.~Vilonen: Representations of $\SL_2$. Course Notes, Northwestern University.

R.~Howe and E.C.~Tan: Non-abelian harmonic analysis - applications of $\SL(2,\mathbb R)$. Springer Universitext. A very nice introduction to representations of $\SL(2,\mathbb R)$, with interesting applications to classical analysis.

V.S.~Varadarajan: An Introduction to Harmonic Analysis on Semisimple Lie Groups. Cambridge Studies in advanced math 16. Nice textbook, with an emphasis on $\SL_2$ and good introductions to various topics.

S.~Lang: $\SL(2,\mathbb R)$. Springer GTM. Fairly analytical introduction, alas not very coherent -- read the \href{http://www.sunsite.ubc.ca/DigitalMathArchive/Langlands/miscellaneous.html#lang}{review} by \href{http://www.sunsite.ubc.ca/DigitalMathArchive/Langlands}{Robert Langlands}
\subsection*{\hspace{1em}Other}
D.~Ramakrishnan and R.~Valenza: Fourier Analysis on Number Fields. Springer GTM 186. Contains an introduction to the Pontrjagin duality theory for locally compact abelian groups, and its applications to number theory (Tate's thesis).

J.~Arthur: Harmonic Analysis and Group Representations. Notices of the AMS. Gorgeous introduction to the ideas of Harish-Chandra. Available off AMS website.

I.~Gelfand, M.~Graev and I.~Piatetskii-Shapiro: Representation Theory and Automorphic Functions. Academic Press. A classic thorough study of representations of SL2 over real, p-adic and adelic fields.

D.~Bump: Automorphic Forms and Representations. Cambridge Studs. Advanced Math. 55. A good and detailed introduction to Langlands program ideas focussed on representations of GL2.

T.~Bailey and A.~Knapp (eds): Representation Theory and Automorphic Forms. Proc. Symp. Pure Math 61. Proceedings of an instructional conference, with a variety of great introductory articles. (In particular Schmid, Knapp and Langlands).

J.~Bernstein and S.~Gelbart (eds): An Introduction to the Langlands Program. Birkhauser. A very timely collection of introductions to the basic constituents of the Langlands program.

A.~Borel and W.~Casselman (eds): Automorphic Forms, Representations and L-functions (the Corvallis volumes). The standard source of information about the Langlands program. Available off the AMS website (see e.g. Arinkin's webpage, below).

J.~Bernstein, Courses on Eisenstein Series and Representations of p-adic Groups. Beautiful expositions. Available at \href{http://www.math.uchicago.edu/~arinkin/langlands}{Dima Arinkin's Langlands page}

\href{http://www.math.ucsd.edu/~wgan}{Wee Teck Gan}, Automorphic Forms and Automorphic Representations: slides for a series of five lectures given in Hangzhou, China giving an excellent overview of the basic theory (available on his web page).}