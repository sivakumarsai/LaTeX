\documentclass[../../rtnotes.tex]{subfiles}
\begin{document}
\section{October 02}
\subsection{More on Lie algebras}
The Lie bracket of vector fields (derivations) on a Lie group $G$ we defined last time as just the commutator of the operators they define; that is, if $\xi,\eta$ are vector fields, then their bracket is $[\xi,\eta] = \xi\eta-\eta\xi$, which is also a vector field. Two other ways we can think about the commutator are as follows:
\begin{enumerate}
    \item We can write $[\xi,\eta]$ as $L_\xi\eta$, where $L_\xi$ is the Lie derivative of $\xi$. The meaning of this notation: For any $g\in G$, start by solving the ODE that $\xi$ defines in a neighborhood of $g$ to obtain a flow $\Phi_{\xi,g}$ passing through $g$. If you do this for every $g\in G$, we obtain a self-diffeomorphism of $G$ which we call $\Phi_\xi$; we will suppress notating the $g$ in $\Phi_{\xi,g}$ to refer to the specific flow passing through $g\in G$. We want to ``differentiate`` $\eta$ along paths passing through each $g\in G$; that is, for each $g$ we want to compare the tangent vector $\eta|_g = \eta|_{\Phi_\xi(0)}\in T_gG$ with $\eta|_{\Phi_\xi(t)}$ for some small $t>0$. These vectors belong to different vector spaces, so we will need to have some notion of ``parallel transport'' to move $\eta|_{\Phi_\xi(t)}\in T_{\Phi_\xi(t)}G$ along the curve $\Phi_\xi$ (backwards in $t$) to a vector in $T_gG$. Then only will it make sense to take the difference of these vectors.
    \sai{picture}

    The formula for the Lie derivative $L_\xi\eta$ is
    \[\dv{t}(\dd\Phi_\xi(-t)\eta|_{\Phi_\xi(t)})|_{t=0}\]
    where $\dd\Phi_\xi(-t)$ is meant to be the map which transports tangent vectors in $T_{\Phi_\xi(t)}G$ to $T_gG$ ``backwards'' along the path defined by $\Phi_\xi$ passing through $g$. We will not think about this notion again, so refer to any text on smooth manifolds to see why the Lie derivative of vector fields coincides with the Lie bracket of vector fields.
    \item Let $\Phi_\xi$ and $\Phi_\eta$ be the flows that $\xi,\eta$ define. Then we can take the commutator (of diffeomorphisms) of these flows to get the flow defined by $\Phi_\eta^{-1}\circ\Phi_\xi^{-1}\circ\Phi_\eta\circ \Phi_\xi$. It turns out that the vector field this flow corresponds to is $[\xi,\eta]$; that is, $\Phi_{[\xi,\eta]} = \Phi_\eta^{-1}\circ\Phi_\xi^{-1}\circ\Phi_\eta\circ \Phi_\xi$, and at any $g\in G$ we have 
    \[[\xi,\eta]|_g = \frac{1}{2}\dv[2]{t}((\Phi_\eta^{-1}\circ\Phi_\xi^{-1}\circ\Phi_\eta\circ \Phi_\xi)(t))|_{t=0}\]
    where indeed the first derivative of the composite flow vanishes. Again, see other sources for the proofs.
    \sai{picture}
\end{enumerate}
Recall from last time that for any Lie group $G$ we defined the Lie algebra of $G$, denoted $\Lie(G) = \mathfrak g$, by three equivalent objects: The tangent space of $G$ at the identity, the space of left-invariant vector fields on $G$, and as $\Hom_{\LieGrp}(\mathbb R,G)$. We also stated last time that the Lie functor sending a Lie group to its Lie algebra defines an equivalence of categories between the category of simply connected Lie groups and the category of Lie algebras. This tells us that not every Lie algebra occurs as the Lie algebra of a Lie group. So what is another way to think of Lie algebras if not from Lie groups?

Lie algebras can also come from associative algebras over fields. If $A$ is an associative algebra over a field $k$, then define the Lie bracket on $A$ to be $[a,b] = ab-ba$. This defines a Lie algebra structure on $A$. For example, on a manifold $M$ we can consider the space of smooth functions $C^\infty(M)$ on $M$ and consider the associative algebra $\End(C^\infty(M))$, or equivalently the space of jets of functions on $M$. Equip this algebra with its natural Lie algebra structure. This is a huge, infinite-dimensional algebra, but the derivations (or for the space of jets, the vector fields) form a Lie subalgebra of $\End(C^\infty(M))$ (or jets) which we usually think about (these are the ``first-order'' elements). In a similar way, $A$ can be thought of as the ``first-order'' elements in $\End(A)$, where $\End(A)$ is also an associative algebra with Lie bracket the usual commutator of operators, and $A$ embeds as a Lie subalgebra of $\End(A)$ by $a\mapsto a\cdot-$. 

Another nice example: Let $V$ be a $n$-dimensional vector space over $\mathbb R$. Then $\End(V)$ with the usual commutator of operators is the Lie algebra of the associative algebra $\Aut(V) = \GL(V)$. The notation we use for this example is $\End(V) = \gl(V) = \Lie(\GL(V))$, or if we fix a basis of $V$, then we can also write $\Mat_n(\mathbb R) = \gl_n(\mathbb R) = \Lie(\GL_n(\mathbb R))$ (where $\Mat_n(\mathbb R)$ denotes $n\times n$ matrices over $\mathbb R$). Since $\GL(V)$ is a Lie group, we can ask about the exponential map $\exp\colon \gl(V)\to \GL(V)$, which in this case is given by the usual Taylor series for $\exp$ (and calculations from analysis make sense of such a series to converge for matrices). For $x\in T_{1_{\GL_n(\mathbb R)}}\GL_n(\mathbb R) = \gl_n(\mathbb R) = \Mat_n(\mathbb R)\cong \mathbb R^{n^2}$ and $t\in\mathbb R$, $\exp(tx) = 1+tx+\frac{t^2x^2}{2}+ \cdots$. If we don't want to think about convergence, we can do things the algebraic geometry way and only think about the truncated Taylor series for $\exp$ to arbitrary order. The commutator in $\gl_n(\mathbb R)$ is the usual matrix commutator.

As usual with objects in mathematics, we should study maps into and out of them. In this class we are concerned with representations, so we define what a representation of a Lie algebra is, starting with some motivation. Let $G$ be a Lie group and $\mathfrak g$ its Lie algebra. Then a finite-dimensional representation $V$ of $G$ is given by a map $G\xrightarrow{\rho}\Aut(V)$, where $\GL(V)$ is given some smooth structure. We differentiate $\rho$ to obtain a map $\mathfrak g \xrightarrow{\dd\rho}\gl(V)$ of Lie algebras. So in general define a representation of a Lie algebra $\mathfrak g$ to be a vector space $V$ and a Lie algebra map $\mathfrak g\to \gl(V)$ which defines how $\mathfrak g$ acts on $V$. Note that the action is by endomorphisms and not automorphisms.

We briefly mention a few well-known Lie algebras of Lie groups: The Lie group $\SL_n(\mathbb R)$ is the kernel of the determinant map $\GL_n(\mathbb R)\xrightarrow{\det}\GL_1(\mathbb R)\cong \mathbb R$. It is true that the Lie functor is left-exact (one way to see this is to recall that the Lie functor is represented by $\mathbb R$), so we can differentiate the determinant map and take the kernel of the resulting map to obtain a description of $\sl_n(\mathbb R)$. One can show that the derivative of the determinant map is the trace map (which is a map of Lie algebras $\gl_n(\mathbb R)\to\mathbb R$), so the kernel of the trace map is the collection of trace-free matrices. It is also true that for any matrix $A$, $\dv{t}(\det\exp(tA))|_{t=0} = \tr A$. Therefore $\sl_n(\mathbb R)$ is the collection of trace-free real-valued matrices with commutator the usual commutator of matrices. The calculation above can also be repeated for $\SL_n(\mathbb C)$ to find that $\sl_n(\mathbb C)$ is the collection of trace-free complex-valued matrices. This is because $\mathbb C\cong \mathbb R^2$, so we can do the appropriate calculations in real coordinates. 

The Lie group $\U(n)\subset \GL_n(\mathbb C)$ is characterized by the collection of complex-valued matrices $K$ such that $K\overline{K}^\top = 1_{\U(n)}$. Differentiate the characteristic equation $K\overline{K}^\top= 1_{\U(n)}$ to find that $\u(n) = \{A\in\gl_n(\mathbb C)\mid A+\overline{A}^\top\}$, the skew-Hermitian complex-valued matrices. It follows that for $\SU(n)$, its Lie algebra $\su(n)$ is the collection of trace-free and skew-Hermitian matrices. 

Two Lie groups can have the same Lie algebra. The Lie groups $\O_n(\mathbb R)$ and $\SO_n(\mathbb R)$ have Lie algebras the collection of skew-symmetric real-valued matrices. This is because the condition of being trace-free is subsumed in the skew-symmetry. Some Lie algebras are already familiar. The Lie algebra of $\SO_3(\mathbb R)$ can be identified with $\mathbb R^3$ equipped with the cross product as the Lie bracket. It is also no coincidence that the same identification can be made for the Lie algebra of $\SU(2)$, given by the $\mathbb R$-span of $\{i = \bigl(\!\begin{smallmatrix}
	i & 0 \\ 0 & -i
\end{smallmatrix}\!\bigr), j = \bigl(\!\begin{smallmatrix}
	0 & 1 \\ -1 & 0
\end{smallmatrix}\!\bigr), k = \bigl(\!\begin{smallmatrix}
	0 & i \\ i & 0
\end{smallmatrix}\!\bigr)\}$ with Lie bracket the ordinary product of matrices. It follows that if $V$ is a finite-dimensional Lie algebra representation of $\su(2)$; that is, there is a map of Lie algebras $\su(2)\xrightarrow{\varphi}\End(V)$ with $[\varphi(i), \varphi(j)] = \varphi(k)$, $[\varphi(j), \varphi(k)] = \varphi(i)$, and other relations coming from the multiplication law in $\su(2)$.

For $G$ simply connected, because we can recover Lie groups from their Lie algebras, it is further true that the finite-dimensional complex Lie algebra representations of $\Lie(G)$ are in bijection with the finite-dimensional complex Lie group representations of $G$. This is because the elements of each side are parameterized by maps $\Lie(G)\to \gl_n(\mathbb C)$ and $G\to\GL_n(\mathbb C)$ respectively, and we can pass between these collections by exponentiation and differentiation. A more general theorem is that if $G,H$ are Lie groups with Lie algebras $\mathfrak g,\mathfrak h$ respectively, a Lie algebra homomorphism $\mathfrak g\xrightarrow{\phi}\mathfrak h$ exponentiates uniquely to a Lie group homomorphism $G\xrightarrow{\Phi}H$ with $\Phi(e^x) = e^{\phi(x)}$ for all $x\in \mathfrak g$ if $G$ is simply connected.

\subsection{Weyl's unitary trick} 
The group $G = \SL_2(\mathbb C)$ is simply connected, since topologically $G$ is isomorphic to $S^3\times\mathbb R$ (the $S^3$ factor is $\SU(2)$ and the $\mathbb R$ is there to scale the determinant). Thus $\Rep(\SL_2(\mathbb C))$ is equivalent to $\Rep(\sl_2(\mathbb C))$, where in each $\Rep$ we absorb the adjectives we ought to mention once: $\Rep(\SL_2(\mathbb C))$ is the category of finite-dimensional complex Lie group representations of $\SL_2(\mathbb C)$, and $\Rep(\sl_2(\mathbb C))$ is the category of finite-dimensional complex Lie algebra representations of $\sl_2(\mathbb C)$.

The same is true for $G = \SU(2)$, where $\Rep(\SU(2))$ is equivalent to $\Rep(\su(2))$. Note that since $\SU(2)$ is compact, the representations in $\Rep(\SU(2))$ are unitarizable. For $\SU(2)$, the equivalence is more concrete. Every element of $\SU(2)$ belongs to some torus, but observe that every torus in $\SU(2)$ is a one-parameter subgroup; that is, every torus occurs as $\{\exp(tA)\mid t\in\mathbb R\}$ for $A\in\su(2)$ (this is the usual polar decomposition for $\SU(2)$). In other words, the exponential map $\su(2)\xrightarrow{\exp}\SU(2)$ is surjective. (In fact, the exponential map $\Lie(G)\xrightarrow{\exp} G$ is surjective for $G$ a connected compact Lie group.) So any Lie algebra representation $\su(2)\xrightarrow{\phi}\End(V)$ exponentiates uniquely to a Lie group representation $\SU(2)\xrightarrow{\Phi}\GL(V)$ and differentiation gives the correspondence in the other direction.

The unitary ``trick'' is the following theorem: that the collections of finite-dimensional representations of $\SU(2)$, $\SL_2(\mathbb C)$, $\SL_2(\mathbb R)$, $\su(2)$, $\sl_2(\mathbb C)$, and $\sl_2(\mathbb R)$ may be identified with each other.

To see why this theorem is true, we start with a $\mathfrak g$ a Lie algebra over $\mathbb R$. An $n$-dimensional complex representation of $\mathfrak g$ is given by an $\mathbb R$-linear map $\mathfrak g \xrightarrow{\phi} \Mat_n(\mathbb C)$. Since the target $\Mat_n(\mathbb C)$ is a $\mathbb C$-vector space, we may extend $\phi$ uniquely to a $\mathbb C$-linear map $\mathfrak g \otimes_\mathbb R\mathbb C\to \Mat_n(\mathbb C)$ by the universal property of the tensor product.
% https://q.uiver.app/#q=WzAsMyxbMCwwLCJcXG1hdGhmcmFrIGciXSxbMSwwLCJcXE1hdF9uKFxcbWF0aGJiIEMpIl0sWzAsMSwiXFxtYXRoZnJhayBnXFxvdGltZXNfXFxtYXRoYmIgUiBcXG1hdGhiYiBDIl0sWzAsMSwiXFxwaGkiXSxbMiwxLCIiLDIseyJzdHlsZSI6eyJib2R5Ijp7Im5hbWUiOiJkYXNoZWQifX19XSxbMCwyXV0=
\[\begin{tikzcd}
	{\mathfrak g} & {\Mat_n(\mathbb C)} \\
	{\mathfrak g\otimes_\mathbb R \mathbb C}
	\arrow["\phi", from=1-1, to=1-2]
	\arrow[from=1-1, to=2-1]
	\arrow[dashed, from=2-1, to=1-2]
\end{tikzcd}\]
The object $\mathfrak g \otimes_\mathbb R\mathbb C$ is denoted $\mathfrak g_{\mathbb C}$, and is called the complexification of the real Lie algebra $\mathfrak g$. The complex vector space $\mathfrak g_{\mathbb C}$ is given the structure of a complex Lie algebra by extending the bracket on $\mathfrak g$ to a bracket on $\mathfrak g_{\mathbb C}$ by $\mathbb C$-linearity, and $\mathfrak g$ embeds as a real Lie subalgebra of $\mathfrak g_{\mathbb C}$. If $\mathfrak g$ is the Lie algebra of a (real) Lie group $G$, then it is the case that $\mathfrak g_{\mathbb C}$ is the Lie algebra of the complex Lie group (i.e. viewed as a complex manifold) $G_{\mathbb C}$, which is the complexification of $G$. We will not think about $G_{\mathbb C}$ in what follows. In the above observation, we merely extended the Lie algebra homomorphism $\phi$ to a map of complex vector spaces, but because the bracket on $\mathfrak g_{\mathbb C}$ is obtained by extending the bracket on $\mathfrak g$ $\mathbb C$-linearly, the extended map $\mathfrak g_{\mathbb C}\to \Mat_n(\mathbb C)$ is a map of complex (more than just real!) Lie algebras. This shows that $\Rep(\mathfrak g)$ is equivalent to $\Rep(\mathfrak g_{\mathbb C})$ (where again, the maps defining representations of the complex Lie algebra $\mathfrak g_{\mathbb C}$ are complex Lie algebra maps) since the representations in these categories are complex vector spaces.

Recall that our favorite $\mathbb R$-basis for $\su(2)$ is given by $\{i = \bigl(\!\begin{smallmatrix}
	i & 0 \\ 0 & -i
\end{smallmatrix}\!\bigr), j = \bigl(\!\begin{smallmatrix}
	0 & 1 \\ -1 & 0
\end{smallmatrix}\!\bigr), k = \bigl(\!\begin{smallmatrix}
	0 & i \\ i & 0
\end{smallmatrix}\!\bigr)\}$, forming a real vector subspace of the complex vector space $\Mat_2(\mathbb C)$. By complexifying, it follows that $\su(2)\otimes_{\mathbb R}\mathbb C$ is isomorphic to $\sl_2(\mathbb C)$, the collection of trace-free complex-valued matrices (as complex Lie algebras). So $\Rep(\su(2))$ is equivalent to $\Rep(\sl_2(\mathbb C))$. A small calculation shows that $\sl_2(\mathbb R)$ is the collection of trace-free real-valued matrices, so its complexification is $\sl_2(\mathbb C)$. So $\Rep(\sl_2(\mathbb R))$ is equivalent to $\Rep(\sl_2(\mathbb C))$. We obtain the following diagram:
% https://q.uiver.app/#q=WzAsNixbMCwwLCJcXFJlcChcXHN1KDIpKSJdLFsxLDAsIlxcUmVwKFxcc2xfMihcXG1hdGhiYiBDKSkiXSxbMiwwLCJcXFJlcChcXHNsXzIoXFxtYXRoYmIgUikpIl0sWzAsMSwiXFxSZXAoXFxTVSgyKSkiXSxbMSwxLCJcXFJlcChcXFNMXzIoXFxtYXRoYmIgQykpIl0sWzIsMSwiXFxSZXAoXFxTTF8yKFxcbWF0aGJiIFIpKSJdLFswLDEsIlxcY29uZyIsMCx7InN0eWxlIjp7InRhaWwiOnsibmFtZSI6ImFycm93aGVhZCJ9fX1dLFsxLDIsIlxcY29uZyIsMCx7InN0eWxlIjp7InRhaWwiOnsibmFtZSI6ImFycm93aGVhZCJ9fX1dLFs0LDEsIlxcY29uZyIsMCx7InN0eWxlIjp7InRhaWwiOnsibmFtZSI6ImFycm93aGVhZCJ9fX1dLFszLDAsIlxcY29uZyIsMCx7InN0eWxlIjp7InRhaWwiOnsibmFtZSI6ImFycm93aGVhZCJ9fX1dLFs0LDMsIlxcdGV4dHtyZXN0cmljdH0iXSxbNCw1LCJcXHRleHR7cmVzdHJpY3R9IiwyXV0=
\[\begin{tikzcd}
	{\Rep(\su(2))} & {\Rep(\sl_2(\mathbb C))} & {\Rep(\sl_2(\mathbb R))} \\
	{\Rep(\SU(2))} & {\Rep(\SL_2(\mathbb C))} & {\Rep(\SL_2(\mathbb R))}
	\arrow["\cong", tail reversed, from=1-1, to=1-2]
	\arrow["\cong", tail reversed, from=1-2, to=1-3]
	\arrow["\cong", tail reversed, from=2-1, to=1-1]
	\arrow["\cong", tail reversed, from=2-2, to=1-2]
	\arrow["{\text{restrict}}", from=2-2, to=2-1]
	\arrow["{\text{restrict}}"', from=2-2, to=2-3]
\end{tikzcd}\]
The vertical arrows in the above diagram form equivalences since $\SL_2(\mathbb C)$ and $\SU(2)$ are simply connected. This establishes most of the equivalences we need, but we are missing the equivalence between $\Rep(\SL_2(\mathbb R))$ and the other five nodes above. It turns out that an equivalence between it and $\Rep(\SL_2(\mathbb C))$ can be verified directly as we will see much later on, but we can also obtain an equivalence with $\Rep(\SU(2))$ from what is really the unitary (or unitarian) trick, which states that despite $\SL_2(\mathbb R)$ not being simply connected, we may still obtain an equivalence between $\Rep(\sl_2(\mathbb R))$ and $\Rep(\SL_2(\mathbb R))$ by following the arrows in the diagram above (by taking a representation of $\sl_2(\mathbb R)$, passing to a representation of $\sl_2(\mathbb R)\otimes_{\mathbb R}\mathbb C\cong \sl_2(\mathbb C)$, exponentiating, then restricting the representation to one of the real Lie group $\SL_2(\mathbb R)$). 
\sai{find a source for why this works}

So as promised, we will start with the finite-dimensional representation theory of $\SU(2)$, and follow the arrows around to obtain the finite-dimensional representation theory of $\SL_2(\mathbb C)$. For us, it is the simplest non-Abelian group we care about, but from the point of view of harmonic analysis, the Peter-Weyl theorem justifies our choice in starting with the representation theory of $\SU(2)$, since it is a compact Lie group. There the finite-dimensional representation theory is completely reducible and we found all the irreps already.

\subsection{A tiny preview of $\sl_2(\mathbb C)$}
We can obtain a basis of $\sl_2(\mathbb C)$ by complexifying the basis $\{i = \bigl(\!\begin{smallmatrix}
	i & 0 \\ 0 & -i
\end{smallmatrix}\!\bigr), j = \bigl(\!\begin{smallmatrix}
	0 & 1 \\ -1 & 0
\end{smallmatrix}\!\bigr), k = \bigl(\!\begin{smallmatrix}
	0 & i \\ i & 0
\end{smallmatrix}\!\bigr)\}$ of $\su(2)$, but this is not a ``democratic'' basis for $\sl_2(\mathbb C)$ for reasons we will see in future lectures. A more democratic basis for $\sl_2(\mathbb C)$ is given by 
\[\{e = \bigl(\!\begin{smallmatrix}
	0 & 1 \\ 0 & 0
\end{smallmatrix}\!\bigr), h = \bigl(\!\begin{smallmatrix}
	1 & 0 \\ 0 & -1
\end{smallmatrix}\!\bigr), f = \bigl(\!\begin{smallmatrix}
	0 & 1 \\ 0 & 0
\end{smallmatrix}\!\bigr)\}\]
These elements satisfy the relations
\[[e,f] = h,\quad [h,e] = 2e,\quad [h,f] = -2f\]
Let $V$ be a finite-dimensional representation of $\SU(2)$. By differentiating, obtain a representation of $\su(2)$, and complexify to obtain a representation of $\sl_2(\mathbb C)$. A representation $V$ of $\sl_2(\mathbb C)$ amounts to specifying elements $E,H,F$ in $\End_{\mathbb C}(V)$ satisfying the same commutation relations as the ones above for $e,h,f$. Note that even though we started with a unitarizable representation $V$ of $\SU(2)$, the representation given by the action of $\sl_2(\mathbb C)$ on $V$ need not be unitarizable (after all, $E,H,F$ merely belong to $\End_\mathbb C(V)$).

Returning to $\SL_2(\mathbb C)$, there is a distinguished subgroup we will look at. Denote by $H$ the subgroup consisting of the diagonal matrices in $\SL_2(\mathbb C)$; that is, $H = \{\bigl(\!\begin{smallmatrix}
	a & 0 \\ 0 & a^{-1}
\end{smallmatrix}\!\bigr)\mid a\in\mathbb C^\times\}$. From this description see that $H\cong \mathbb C^\times$. By restricting to the copy of $\U(1)$ in $\mathbb C^\times$, we can use the representation theory we obtained earlier. In particular, see that $\Lie(H)$ as a Lie subalgebra of $\Lie(\SL_2(\mathbb C))$ is $\mathbb Ch$ (differentiate a path in $\SL_2(\mathbb C)$ taking values in $H$ passing through the identity matrix), which is also the complexification of the real Lie algebra $i\mathbb Rh$ of the copy of $\U(1)$ in $H$. From the $\U(1)$-representation theory from earlier, it follows that $h$ acts on a representation semi-simply (meaning the representation restricted to $i\mathbb Rh$, and hence also $\mathbb Ch$, is completely reducible).
\end{document}