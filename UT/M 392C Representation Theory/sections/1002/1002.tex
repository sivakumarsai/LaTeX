\documentclass[../../rtnotes.tex]{subfiles}
\begin{document}
\section{October 02}
\subsection{More on Lie algebras}
The Lie bracket of vector fields (derivations) on a Lie group $G$ we defined last time as just the commutator of the operators they define; that is, if $\xi,\eta$ are vector fields, then their bracket is $[\xi,\eta] = \xi\eta-\eta\xi$, which is also a vector field. Two other ways we can think about the commutator are as follows:
\begin{enumerate}
    \item We can write $[\xi,\eta]$ as $L_\xi\eta$, where $L_\xi$ is the Lie derivative of $\xi$. The meaning of this notation: For any $g\in G$, start by solving the ODE that $\xi$ defines in a neighborhood of $g$ to obtain a flow $\Phi_{\xi,g}$ passing through $g$. If you do this for every $g\in G$, we obtain a self-diffeomorphism of $G$ which we call $\Phi_\xi$; we will suppress notating the $g$ in $\Phi_{\xi,g}$ to refer to the specific flow passing through $g\in G$. We want to ``differentiate`` $\eta$ along paths passing through each $g\in G$; that is, for each $g$ we want to compare the tangent vector $\eta|_g = \eta|_{\Phi_\xi(0)}\in T_gG$ with $\eta|_{\Phi_\xi(t)}$ for some small $t>0$. These vectors belong to different vector spaces, so we will need to have some notion of ``parallel transport'' to move $\eta|_{\Phi_\xi(t)}\in T_{\Phi_\xi(t)}G$ along the curve $\Phi_\xi$ (backwards in $t$) to a vector in $T_gG$. Then only will it make sense to take the difference of these vectors.
    \sai{picture}

    The formula for the Lie derivative $L_\xi\eta$ is
    \[\dv{t}(\dd\Phi_\xi(-t)\eta|_{\Phi_\xi(t)})|_{t=0}\]
    where $\dd\Phi_\xi(-t)$ is meant to be the map which transports tangent vectors in $T_{\Phi_\xi(t)}G$ to $T_gG$ ``backwards'' along the path defined by $\Phi_\xi$ passing through $g$. We will not think about this notion again, so refer to any text on smooth manifolds to see why the Lie derivative of vector fields coincides with the Lie bracket of vector fields.
    \item Let $\Phi_\xi$ and $\Phi_\eta$ be the flows that $\xi,\eta$ define. Then we can take the commutator (of diffeomorphisms) of these flows to get the flow defined by $\Phi_\eta^{-1}\circ\Phi_\xi^{-1}\circ\Phi_\eta\circ \Phi_\xi$. It turns out that the vector field this flow corresponds to is $[\xi,\eta]$; that is, $\Phi_{[\xi,\eta]} = \Phi_\eta^{-1}\circ\Phi_\xi^{-1}\circ\Phi_\eta\circ \Phi_\xi$, and at any $g\in G$ we have 
    \[[\xi,\eta]|_g = \frac{1}{2}\dv[2]{t}((\Phi_\eta^{-1}\circ\Phi_\xi^{-1}\circ\Phi_\eta\circ \Phi_\xi)(t))|_{t=0}\]
    where indeed the first derivative of the composite flow vanishes. Again, see other sources for the proofs.
    \sai{picture}
\end{enumerate}
Recall from last time that for any Lie group $G$ we defined the Lie algebra of $G$, denoted $\Lie(G) = \mathfrak g$, by three equivalent objects: The tangent space of $G$ at the identity, the space of left-invariant vector fields on $G$, and as $\Hom_{\LieGrp}(\mathbb R,G)$. We also stated last time that the Lie functor sending a Lie group to its Lie algebra defines an equivalence of categories between the category of simply connected Lie groups and the category of Lie algebras. This tells us that not every Lie algebra occurs as the Lie algebra of a Lie group. So what is another way to think of Lie algebras if not from Lie groups?

Lie algebras can also come from associative algebras over fields. If $A$ is an associative algebra over a field $k$, then define the Lie bracket on $A$ to be $[a,b] = ab-ba$. This defines a Lie algebra structure on $A$. For example, on a manifold $M$ we can consider the space of smooth functions $C^\infty(M)$ on $M$ and consider the associative algebra $\End(C^\infty(M))$, or equivalently the space of jets of functions on $M$. Equip this algebra with its natural Lie algebra structure. This is a huge, infinite-dimensional algebra, but the derivations (or for the space of jets, the vector fields) form a Lie subalgebra of $\End(C^\infty(M))$ (or jets) which we usually think about (these are the ``first-order'' elements). In a similar way, $A$ can be thought of as the ``first-order'' elements in $\End(A)$, where $\End(A)$ is also an associative algebra with Lie bracket the usual commutator of operators, and $A$ embeds as a Lie subalgebra of $\End(A)$ by $a\mapsto a\cdot-$. 

Another nice example: Let $V$ be a $n$-dimensional vector space over $\mathbb R$. Then $\End(V)$ with the usual commutator of operators is the Lie algebra of the associative algebra $\Aut(V) = \GL(V)$. The notation we use for this example is $\End(V) = \gl(V) = \Lie(\GL(V))$, or if we fix a basis of $V$, then we can also write $\Mat_n(\mathbb R) = \gl_n(\mathbb R) = \Lie(\GL_n(\mathbb R))$ (where $\Mat_n(\mathbb R)$ denotes $n\times n$ matrices over $\mathbb R$). Since $\GL(V)$ is a Lie group, we can ask about the exponential map $\exp\colon \gl(V)\to \GL(V)$, which in this case is given by the usual Taylor series for $\exp$ (and calculations from analysis make sense of such a series to converge for matrices).
\end{document}