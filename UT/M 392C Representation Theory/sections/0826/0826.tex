\documentclass[../../rtnotes.tex]{subfiles}
\begin{document}
\section{08/26}
\subsection{Definitions}
Let $G$ be a group and $k$ be a field. A representation of $G$ is a vector space $V\in \Vect_k$ with a linear action of $G$ on $V$. (For convenience, let's assume representations are finite-dimensional vector spaces unless otherwise stated.)

This is to say there is a homomorphism $G\xrightarrow{\rho}\Aut(V)$ so that we may define the action of $G$ on $V$ to be $g\cdot v \coloneqq \rho(g)v$ or alternatively a product $G\times V\to V$ given by $(g,v)\mapsto \rho(g)v$ that is linear in $V$ and associative in $G$, among other properties. (We often suppress the $\,\cdot\,$ in the action/product.)

Note $\Aut(V) \eqqcolon \GL(V) \cong \GL_{\dim V}(k)$, so we could even think of the elements of $G$ as matrices if we fix a basis of $V$.

The (finite dimensional) representations $(V,\rho)$ of $G$ over a field $k$ form the objects of a category $\Rep_k(G)$ whose morphisms are linear transformations which commute with the action of $G$; i.e., $\Hom_G(V,W)$ consists of the linear transformations $T\colon V\to W$ for which $g(Tv) = T(gv)$ for all $g\in G$ and $v\in V$. (These are also called intertwining operators, $G$-linear maps, or $G$-equivariant maps, etc. Also note that $\Hom_G(V,W)$ is merely a vector space, but see \href{https://math.stackexchange.com/questions/2334514/whats-the-internal-hom-of-linear-representations-of-categories}{this MSE post}.) 

The field $k$ does matter, but for the majority of what follows $k$ will be $\mathbb C$. Sometimes it may be possible to replace $\mathbb C$ with any algebraically closed field, but the characteristic of the field is rather important (see \href{https://en.wikipedia.org/wiki/Modular_representation_theory}{modular representation theory} for example). 

\subsection{Variants of representations}
We can add adjectives in various places to the definitions above to get different kinds of representations. 

For example, if $V$ has an inner product we can ask about linear actions of $G$ on $V$ that preserve that inner product. If $V$ is a Hilbert space we can ask about linear actions of $G$ on $V$ for which $\rho(g)$ is a unitary operator; these are the unitary representations of $G$.

If $G$ has a topology and $V$ is a topological vector space, we say $(V,\rho)$ is a continuous representation if the product map $G\times V\to V$ defined before is continuous. If $G$ has a smooth structure (e.g. is a Lie group) or is an algebraic group, we can ask about smooth or algebraic representations, respectively.

\subsection{Aspects of representations we study}
\begin{enumerate}
    \item Irreducible representations (henceforth called ``irreps''): An irrep of $G$ are the representations $V$ which have no invariant (or stable some might say) subspaces under the group action; that is, if $W$ is a $G$-invariant subspace of $V$ then $W$ is either $0$ or $V$. These are like the ``atoms'' in representation theory.
    
    For a given group $G$ one goal of representation theory is to classify up to isomorphism its irreps. One approach is to try to attach numerical invariants to representations and see if they help to classify irreps, and generalizations of this idea lead to character theory. Sometimes there are ``isotopes'' that are not isomorphic but nevertheless cannot be distinguished by certain invariants.

    Another thing we do is to take a subgroup $H$ of $G$ and study how irreps of $G$ decompose when the group action is restricted to $H$. On the other hand, we can also look at how to build bigger representations or ``molecules'' out of the atomic irreps via extension. Of course, it is easy to take direct sums of irreps but depending on the context it may be possible to obtain indecomposable reps which are extensions of irreps but do not decompose into a direct sum of irreps (so to summarize, irreducible implies indecomposable but not the other way around in general).

    Maschke's theorem implies that indecomposable representations of finite groups over fields with characteristic not dividing the order of the group are irreducible. Alternatively, the theorem implies that in this setting, all short exact sequences of representations split.
    
    For example, the indecomposable representations over $\mathbb C$ coincide with the irreps when $G$ is finite. On the other hand, the shearing representation 
    \begin{align*}
      \mathbb Z &\to \GL_2(\mathbb R)\\
      1 &\mapsto \big(\!\begin{smallmatrix}
        1 & 1 \\ 0 & 1
      \end{smallmatrix}\!\big)
    \end{align*}
    preserves the horizontal axis, so it is not an irrep, but is not decomposable since $\big(\!\begin{smallmatrix}
        1 & 1 \\ 0 & 1
    \end{smallmatrix}\!\big)$ is not diagonalizable. The representations $(k^n,\rho)$ of $\mathbb Z$ are given by specifying an invertible $n\times n$ matrix, and any two such representations are isomorphic if the matrices specifying them are conjugate. If $k$ is algebraically closed, the indecomposable representations in this setting correspond to Jordan blocks. Of course, every representation is the direct sum of indecomposables, and in this setting the Jordan normal form of an $n\times n$ matrix would describe the decomposition of $k^n$ into indecomposables.

    \item Harmonic analysis is the study of naturally appearing ``large'' representations; we would like to perform some kind of ``spectroscopy'' to determine what representations appear within them.
    
    If a group $G$ acts on some object $X$ (which could be a set, manifold, or otherwise a ``geometric'' object), we say $X$ has some symmetries which we would like to ``linearize''. We achieve this by considering the $k$-valued functions on $X$ for some field $k$; if $X$ has additional structure we can restrict to functions on $X$ which interact with that structure (e.g. measurable/continuous/etc maps)

    From a right action of $G$ on $X$, function spaces on $X$ inherit a natural linear left action of $G$; one is given by $gf(x) = f(xg)$. These representations are typically very decomposable or reducible.

    \item Different groups give rise to different phenomena. Below are some examples of groups we will see again.
    \begin{table}[H]
\centering
\begin{tabular}{lllll}
\multicolumn{1}{l|}{}        & \multicolumn{1}{l|}{compact} & noncompact  &  &  \\ \cline{1-3}
\multicolumn{1}{l|}{Abelian} & \multicolumn{1}{l|}{$S^1$}   & $\mathbb R$ &  &  \\ \cline{1-3}
\multicolumn{1}{l|}{non-Abelian} & \multicolumn{1}{l|}{$\SO(3)$} & $\SL_2(\mathbb R)$, $\SL_2(\mathbb C)$ &  &  \\
                             &                              &             &  & 
\end{tabular}
\end{table}
    
    It is a theorem that irreps of compact groups coincide with their indecomposables, and that irreps of Abelian groups are all one-dimensional.
    
    The representation theory of $S^1$ leads to the theory of Fourier series, and in a similar way we can recover the Fourier transform from the representation theory of $\mathbb R$.

    The Lie group $\SO(3)$ can be thought of as the group of rotations of a $2$-sphere. One nice result is that the spherical harmonics are basis functions for the irreps of $\SO(3)$, which occur naturally as the atomic orbitals (see \href{https://en.wikipedia.org/wiki/Spherical_harmonics}{the Wikipedia article on spherical harmonics}).

    The representations of the groups $\SL_2(k)$ for various fields $k$ appear all over math.  We can study special functions like the Bessel or hypergeometric functions, or even modular forms. The hard Lefschetz theorem in algebraic geometry says that $\SL_2(\mathbb C)$ acts on $H^\ast(X,\mathbb C)$ for $X$ a nice enough smooth projective variety. In physics, the special linear group sort of appears in the Lorentz group $\SO(1,3)^+ = \PSL_2(\mathbb C)$.
\end{enumerate}

\subsection{Spectral theory}
A slogan for what is to come: ``commutativity implies geometry''.

Let $k = \mathbb C$ and $X$ a set. Then the complex-valued functions on $X$ form a commutative algebra. This is some example of a functor suggestively called $\mathcal O$ from the category of some kind of geometric objects to commutative algebras.

To expand on the previous idea, here is a motto originating from Gelfand and Grothendieck's work:
\begin{enumerate}
  \item Any commutative ring should be thought of as functions on some space.
  
  That is, there is some functor going from commutative rings to some category of geometric objects that realizes this idea. In particular we should be able to adjust the functor and its source/target to obtain an equivalence of categories.
  \[\text{commutative rings}\longleftrightarrow \text{geometry}\]
  \item Once we are in the situation where we have an equivalence of categories between commutative rings and geometric objects, we should further obtain a correspondence between the modules over a ring $R$ and sheaves (special families of vector spaces or Abelian groups) on the corresponding geometric object $X$ to $R$.
  \[R\text{-modules}\longleftrightarrow \text{sheaves on }X\]
\end{enumerate}
For example, if $X$ is a finite set, the corresponding ring $R$ is the finite-dimensional commutative algebra of complex-valued functions on $X$. This algebra is semisimple with $R = \bigoplus_i \mathbb Ce_i$ where $e_ie_j = \delta_{ij}$. We can think of the $e_i$ as delta/indicator functions on points of $X$.

An $R$-module $M$ has the decomposition $M = \oplus_i e_i M$, which corresponds to a sheaf on $X$ where at each point of $X$ we imagine the corresponding module $e_iM$ lying on it:
\begin{align*}
  M &~~~~ e_1M \phantom{\oplus} e_2M \phantom{\oplus} \cdots \phantom{\oplus} e_n M\\
  X&~~~~~\bullet~\phantom{\oplus}~~\bullet~~\phantom{\oplus} \cdots \phantom{\oplus}~~\bullet
\end{align*}

\subsection{A short word about the group algebra}
A representation $V$ of a group $G$ is given by a group homomorphism $G\xrightarrow{\rho}\Aut(V)$. Since $\Aut(V)$ is contained in $\End(V)$, a $k$-algebra, there is an object $k[G]$ called the group algebra for which the homomorphism $\rho$ factors through $k[G]$; that is, the diagram 
% https://q.uiver.app/#q=WzAsMyxbMCwwLCJHIl0sWzEsMCwiXFxFbmQoVikiXSxbMCwxLCJrW0ddIl0sWzAsMSwiXFxyaG8iXSxbMCwyLCIiLDIseyJzdHlsZSI6eyJ0YWlsIjp7Im5hbWUiOiJob29rIiwic2lkZSI6InRvcCJ9fX1dLFsyLDEsIlxcb3ZlcmxpbmUgXFxyaG8iLDIseyJzdHlsZSI6eyJib2R5Ijp7Im5hbWUiOiJkYXNoZWQifX19XV0=
\[\begin{tikzcd}
	G & {\End(V)} \\
	{k[G]}
	\arrow["\rho", from=1-1, to=1-2]
	\arrow[hook, from=1-1, to=2-1]
	\arrow["{\overline \rho}"', dashed, from=2-1, to=1-2]
\end{tikzcd}\]
commutes. One explicit description of $k[G]$ is the set of finite linear combinations of elements of $G$ with coefficients in $k$:
\[k[G] = \biggl\{\sum_{g\in G}c_gg \biggm\vert c_g\in k, \text{ all but finitely many $c_k$ are zero}\biggr\}.\]
The addition and multiplication in $k[G]$ are defined using the addition in $k$ and multiplication in $G$, respectively. We will return to the group algebra when investigating representations from a module-theoretic point of view.
\end{document}