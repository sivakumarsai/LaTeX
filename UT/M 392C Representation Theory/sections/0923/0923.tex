\documentclass[../../rtnotes.tex]{subfiles}
\begin{document}
\section{09/20}
\subsection{Prologue to representation theory of $\SU(2)$}
There are several ways to study representations of $\SU(2)$, but the approach we will take is to start by first using the representation theory of $\U(1)$ to extract interesting information about representations of $\SU(2)$ as opposed to constructing them explicitly via a robust structure theory or some other approach.

Let $V$ be a finite dimensional complex (continuous) representation of $\SU(2)$. Since $\SU(2)$ is not Abelian, we do not immediately get to use the nice theory of LCA groups we discussed earlier. So instead we look at the action of a copy of $\U(1)$ living inside $\SU(2)$ on $V$; that is, we think of $V$ first as a $\U(1)$-representation. There are lots of different copies of $\U(1)$ inside of $\SU(2)$, but for now think of the copy given by $\{\cos(\theta) + \sin(\theta)i\}\subset \SU(2)$ (the unit quaternions where the coefficients of $j,k$ are zero). A copy of $\U(1)$ in $\SU(2)$ is called a (maximal) torus, usually denoted $T$ (a torus in a compact Lie group $G$ is a compact, connected, Abelian Lie subgroup of $G$, and a maximal torus is a torus that is maximal under inclusion of tori).

From the earlier $\U(1)$-theory, $V$ is unitarizable under the action of $T$ and hence decomposes into isotypic components $V = \bigoplus_{n\in\mathbb Z}V_n$ where each $V_n$ is $\dim V_n$ many copies of the $n$-th irrep $\mathbb C_{\chi_n}$ of $T$ where $z\cdot w = \chi_n(z)w$ where $\chi_n$ is a characrer of $T$; that is,
\[V_n = \mathbb C^{\dim V_n}\otimes_{\mathbb C}\mathbb C_{\chi_n}\]
In other words, $V_n$ keeps track of the possibly repeated $1$-dimensional irreps. All but finitely many $\dim V_n$ are zero since $V$ is finite dimensional. This decomposition of $V$ is called the weight space decomposition with respect to the (maximal; we will also see this later) torus $T$ inside $\SU(2)$, and the $V_n$ are called the weight spaces.

One function we can get from this decomposition is a function $\mathbb Z\to\mathbb Z$ given by $n\mapsto \dim V_n$ (which is of course finitely supported since $V$ is finite dimensional). From this function form the Laurent polynomial $\chi_V(z)$ given by
\[z\mapsto \chi_V(z) = \sum_{n\in \mathbb Z}(\dim V_n)z^n\in\mathbb C[z,z^{-1}]\]

If we use the symbol $z$ for an element $\exp(i\theta)\in \U(1)\cong T$ (identifying $z$ with $\bigl(\!\begin{smallmatrix}
    z & 0 \\ 0 & \overline z
\end{smallmatrix}\!\bigr)$), then $z\cdot v$ for $v\in V_n$ is $z^nv$. The $\dim V_n\times \dim V_n$ matrix for the action of $z$ on $V_n = \mathbb C^{\dim V_n}\otimes_{\mathbb C}C_{\chi_n}$ is given by
\[z\cdot v = \begin{pmatrix}
    z^n & & \\
    & \mathbin{\rotatebox[origin=c]{-10}{$\ddots$}} & \\
    & & z^n
\end{pmatrix}v\]
Thus if $z\in\U(1)$, then $\Tr_{V_n}(z\cdot) = (\dim V_n)z^n$; similarly, $\chi_V(z) = \sum_{n\in\mathbb Z}(\dim V_n)z^n$ is the trace $\Tr_V(z\cdot)$. So think of the Laurent polynomial $\chi_V(z)$ as a function $\U(1)\to \mathbb C$, which we call the character of the $\U(1)$-representation $V$.

But we don't need to restrict $\chi_V$ to $T$; since $V$ is finite-dimensional, we could instead define more generally define the character $\chi_V$ on $G$ by $\chi_V(g) = \Tr_V(g\cdot)$; that is, by taking the trace of the group action on $V$. Another observation is that the character $\chi_V$ is invariant under conjugation since the trace is: $\chi_V(hgh^{-1}) = \Tr_V(hgh^{-1}\cdot) = \Tr_V(g\cdot) = \chi_V(g)$. Functions that are invariant under conjugation are called class functions. The approach we are heading towards to study the representation theory of $\SU(2)$ is called character theory.

\subsection{The Weyl group}
Let $V$ be an irrep of a group $G$. By Schur's lemma $Z(G)\subset G$ acts on $V$ by scalar multiples of $\id_V$ since acting on $V$ by an element of $Z(G)$ defines an intertwining operator $V\to V$ (because elements of $Z(G)$ commute with every element of $G$).

The center of $\SU(2)$ is $\{\pm 1\}$: any matrix $\bigl(\!\begin{smallmatrix}
	z & w \\ -\overline w & \overline z
\end{smallmatrix}\!\bigr)$ in the center of $\SU(2)$ must satisfy $\bigl(\!\begin{smallmatrix}
	0 & 1 \\ 1 & 0
\end{smallmatrix}\!\bigr)\bigl(\!\begin{smallmatrix}
	z & w \\ -\overline w & \overline z
\end{smallmatrix}\!\bigr)\bigl(\!\begin{smallmatrix}
	0 & 1 \\ 1 & 0
\end{smallmatrix}\!\bigr)^{-1} = \bigl(\!\begin{smallmatrix}
	z & w \\ -\overline w & \overline z
\end{smallmatrix}\!\bigr)$, so we observe first that $\Im z = 0$ and $\Re w  = 0$, then from $\bigl(\!\begin{smallmatrix}
	i & 0 \\ 0 & -i
\end{smallmatrix}\!\bigr)\bigl(\!\begin{smallmatrix}
	z & w \\ w & z
\end{smallmatrix}\!\bigr)\bigl(\!\begin{smallmatrix}
	i & 0 \\ 0 & -i
\end{smallmatrix}\!\bigr)^{-1} = \bigl(\!\begin{smallmatrix}
	z & w \\ w & z
\end{smallmatrix}\!\bigr)$, deduce that $\Im w = 0$. Then $\det \bigl(\!\begin{smallmatrix}
	z & 0 \\ 0 & z
\end{smallmatrix}\!\bigr) = 1$ so that $z = \pm 1$. As a side note, it follows that $\SU(2)$ is an extension of $\SU(2)/\{\pm 1\} \cong \SO_3(\mathbb R)$ by $Z(\SU(2)) = \{\pm 1\}\cong \mathbb Z/2\mathbb Z$, which is central but non-split. This is because a section $\SO_3(\mathbb R)\to \SU(2)$ must be continuous and hence preserve loops. Consider the element in $\SO_3(\mathbb R)$ given by fixing an axis $u\in\mathbb R^3$ (with $u$ of unit length) and rotating counterclockwise by an angle $\theta$, denote this element by $R_{u,\theta}$. The element $R_{u,\theta}$ should be sent to the element $g_\theta = \cos(\theta/2)+u\sin(\theta/2)$ in $\SU(2)$ (here we are identifying the unit pure quaternions with $S^2\subset \mathbb R^3$). So when $\theta = 0$, $g_\theta = 1$ and when $\theta = 2\pi$, $g_\theta = -1$. But the collection $\{R_{u,\theta}\mid 0\leq \theta\leq 2\pi\}$ forms a loop in $\SO_3(\mathbb R)$, but this loop is not sent to a loop in $\SU(2)$. So no section can exist.

The observations of the previous two paragraphs combine to reveal that $-1\in \SU(2)$ acts on an irrep $V$ of $\SU(2)$ by a scalar multiple of $\id_V$. Since $-1^2 = 1$, we are limited to the cases that either $-1$ can act as $\id V$ or $-\id_V$. Sort irreps of $\SU(2)$ by whether $-1$ acts trivially by $\id_V$ or ``genuinely'' by $-\id_V$. The irreps in the first collection are called even irreps of $\SU(2)$, and the irreps in the second collection are called odd irreps of $\SU(2)$. The even irreps of $\SU(2)$ may be viewed as an irrep of $\SO_3(\mathbb R)$, since last time we saw that $\SO_3(\mathbb R)\cong \SU(2)/\{\pm 1\}$ (if both $\pm 1$ act trivially on an even irrep $V$ of $\SU(2)$, then the action of the quotient $\SU(2)/\{\pm 1\}$ on $V$ is well-defined and $V$ will remain irreducible under the new action).

If $V$ is an even irrep of $\SU(2)$, its decomposition as a $\U(1)$-representation is $V = \bigoplus_{n\in\mathbb Z}V_n$ where $\dim V_n$ for $n$ odd is zero. This is because $-1$ is contained in $\U(1)$, and the the action of $-1$ on $\mathbb C_{\chi_n}$ is given by scalar multiplication by $(-1)^n$, so in order for the action of $-1$ on $V_n$ to be trivial for odd $n$, $\dim V_n$ must be zero.

Consider a representation $V$ of $\SU(2)$. As before, the weight space decomposition of $V$ with respect to $T$ is $V = \bigoplus_{n\in\mathbb Z}V_n$. Let $h$ be an element of the normalizer $N(T)\subset \SU(2)$ of the torus $T$. Observe that $\chi_n(h^{-1}(-)h)$ defines a character of the torus $T$, which we know is of the form $\chi_m$ for some $m$ (where it is of course possible that $m = n$).

Then for any $t\in T$ and $v\in V_n$, 
\[t\cdot (h\cdot v) = th\cdot v = hh^{-1}th\cdot v = h\cdot (\chi_n(h^{-1}th)v) = \chi_m(t)(h\cdot v)\]
This shows that the action of the normalizer of $T$ on elements of the weight spaces $V_n$ is to possibly move elements between weight spaces. Furthermore, $h\cdot V_n$ is isomorphic to $V_n$ since multiplication by $h$ is invertible. So the action of $h$ on the weight spaces is a permutation of the weight spaces. Observe that the action of $T\subset N(T)$ does not move elements of weight spaces out of their weight space. The action of $N(T)$ on elements of weight spaces descends to an action of $N(T)/T$ on elements of the weight spaces, but in order to know more we would need to calculate $N(T)$.

In $\SU(2)$ the copy of $T$ we started with was the one with $i$ in it; that is, we started with $T = \{\bigl(\!\begin{smallmatrix}
    z & 0 \\ 0 & \overline z
\end{smallmatrix}\!\bigr)\}\subset \SU(2)$ (note $\overline z = z^{-1}$ since $z\in \U(1)$). We calculate the normalizer of $T$ in $\SU(2)$: An element $g\in N(T)\subset \SU(2)$ will normalize all elements $\bigl(\!\begin{smallmatrix}
    t & 0 \\ 0 & \overline t
\end{smallmatrix}\!\bigr)\in T$; that is, for each $\bigl(\!\begin{smallmatrix}
    t & 0 \\ 0 & \overline t
\end{smallmatrix}\!\bigr)\in T$ there exists $\bigl(\!\begin{smallmatrix}
    s & 0 \\ 0 & \overline s
\end{smallmatrix}\!\bigr)\in T$ for which $g\bigl(\!\begin{smallmatrix}
    t & 0 \\ 0 & \overline t
\end{smallmatrix}\!\bigr)g^{-1} = \bigl(\!\begin{smallmatrix}
    s & 0 \\ 0 & \overline s
\end{smallmatrix}\!\bigr)$. Put $v_1 = g^{-1}e_1$, so that 
\[g\bigl(\!\begin{smallmatrix}
    t & 0 \\ 0 & \overline t
\end{smallmatrix}\!\bigr)v_1 = \bigl(\!\begin{smallmatrix}
    s & 0 \\ 0 & \overline s
\end{smallmatrix}\!\bigr)gg^{-1}e_1 = se_1\]
Then $\bigl(\!\begin{smallmatrix}
    t & 0 \\ 0 & \overline t
\end{smallmatrix}\!\bigr)v_1 = sv_1$, which in other words is to say $v_1$ is an eigenvector of $\bigl(\!\begin{smallmatrix}
    t & 0 \\ 0 & \overline t
\end{smallmatrix}\!\bigr)$ with eigenvalue $s$. The eigenvalue $s$ must be equal to one of $t$ or $\overline t$. In the case that $s = t$, the element $g$ must be of the form $\bigl(\!\begin{smallmatrix}
    z & 0 \\ 0 & \overline z
\end{smallmatrix}\!\bigr)$ for $z\in \U(1)$. In the case that $s = \overline t$, the element $g$ must be of the form $\bigl(\!\begin{smallmatrix}
    0 & 1 \\ -1 & 0
\end{smallmatrix}\!\bigr)\bigl(\!\begin{smallmatrix}
    z & 0 \\ 0 & \overline z
\end{smallmatrix}\!\bigr) = \bigl(\!\begin{smallmatrix}
    0 & \overline z \\ -z & 0
\end{smallmatrix}\!\bigr)$ for $z\in \U(1)$. We can obtain the outcome of either case by taking $t = i$ and solving a small system of equations obtained after conjugation by $\bigl(\!\begin{smallmatrix}
    a& b \\ c & d
\end{smallmatrix}\!\bigr)\in N(T)$ in each case. 

From the above calculations see that $N(T)$ is the group generated by the order $4$ element $\bigl(\!\begin{smallmatrix}
    0 & 1 \\ -1 & 0
\end{smallmatrix}\!\bigr)$ and $T$. The quotient $N(T)/T$ is isomorphic to $\mathbb Z/2\mathbb Z$ since the coset $\bigl(\!\begin{smallmatrix}
    0 & 1 \\ -1 & 0
\end{smallmatrix}\!\bigr)T$ in $N(T)$ has order $2$. Therefore there is a short exact sequence 
\[1\to T\to N(T)\to \mathbb Z/2\mathbb Z\to 0\]
but this short exact sequence does not split. There is no homomorphism $\mathbb Z/2\mathbb Z\to N(T)$ which when followed by the quotient map is the identity on $\mathbb Z/2\mathbb Z$. The generator $-1$ of $\mathbb Z/2\mathbb Z$ would need to be sent to an order $2$ element of $N(T)$, but $\bigl(\!\begin{smallmatrix}
    0 & 1 \\ -1 & 0
\end{smallmatrix}\!\bigr)$ has order $4$, which is why no section $\mathbb Z/2\mathbb Z\to N(T)$ exists. So $N(T)$ is a non-split extension of $\mathbb Z/2\mathbb Z$ by $T$. As a side remark, if we consider a maximal torus $\widetilde T$ inside $\SU(2)/\{\pm 1\}\cong \SO_3(\mathbb R)$, then the normalizer $N(\widetilde T)$ in $ \SO_3(\mathbb R)$ is a split extension of $\mathbb Z/2\mathbb Z$ by $\widetilde T$. A choice of torus in this case could be, for example, the group of rotations around a fixed axis.

% Therefore $N(T)\subset \SU(2)$ is the product of the subgroups $\{1,\bigl(\!\begin{smallmatrix}
%     0 & 1 \\ -1 & 0
% \end{smallmatrix}\!\bigr)\}\cong \mathbb Z/2\mathbb Z$ and $T$, which intersect trivially. It follows that 
% \[N(T) = T\rtimes \mathbb Z/2\mathbb Z\]
% (since by definition $T$ is normal in $N(T)$), and by identifying $T$ with $\SO_2(\mathbb R)$ (via $\bigl(\!\begin{smallmatrix}
%     x+iy & 0 \\ 0 & x-iy
% \end{smallmatrix}\!\bigr)\mapsto \bigl(\!\begin{smallmatrix}
%     x & y \\ -y & x
% \end{smallmatrix}\!\bigr)$), the split extension $1\to T \to N(T) \to \{1,\bigl(\!\begin{smallmatrix}
%     0 & 1 \\ -1 & 0
% \end{smallmatrix}\!\bigr)\} \to 1$ is isomorphic to the split extension
% \[1\to \SO_2(\mathbb R) \to \O_2(\mathbb R) \to \{1,\bigl(\!\begin{smallmatrix}
%     0 & 1 \\ -1 & 0
% \end{smallmatrix}\!\bigr)\}\ \to 1\]
% In other words, $N(T)$ is isomorphic to $\O_2(\mathbb R)$.

Interestingly, the quotient map $\SU(2)\to T\backslash\SU(2)$ viewed as a quotient map of topological spaces is the Hopf fibration $S^3\to S^2$, where $T\backslash\SU(2)\cong S^2$. Consider the map $\SU(2)\to \mathbb C^2\setminus \{0\}$ given by $\bigl(\!\begin{smallmatrix}
    z & w \\ -\overline w & \overline z
\end{smallmatrix}\!\bigr)\to (z,w)$. Then project onto $\mathbb{CP}^1$ by $(z,w)\mapsto [z:w]$. The fibers of the composite map $\SU(2)\to \mathbb{CP}^1$ are the right cosets $T\bigl(\!\begin{smallmatrix}
    z & w \\ -\overline w & \overline z
\end{smallmatrix}\!\bigr)$. Therefore we can identify $\mathbb{CP}^1$ with the right coset space $T\backslash\SU(2)$, and $\mathbb{CP}^1\cong S^2$. 

The group $N(T)$ acts on $\SU(2)$ by conjugation, and the action descends to an action $N(T)$ on $T\backslash\SU(2)$ by conjugation because $N(T)$ normalizes $T$. The action of conjugation by an element of $T$ fixes orbits in $T\backslash \SU(2)$, so we can further define an action of $T\backslash N(T) = N(T)/T$ on $T\backslash \SU(T)$ by conjugation. Identifying $T\backslash \SU(2)$ with $S^2$, the action of $T\backslash N(T)$ looks like reflection across a plane:
\begin{figure}[h]
    \centering
\begin{tikzpicture}
  \shade[ball color = green!40, opacity = 0.4] (0,0) circle (2cm);
  \draw (0,0) circle (2cm);
  \draw (-2,0) arc (180:360:2 and 0.6);
  \draw[dashed] (2,0) arc (0:180:2 and 0.6);
  \node [style=none] (0) at (0,2) {$\bullet$};
  \node [style=none] (0) at (0,-2) {$\bullet$};
  \node [style=none] (0) at (2,0) {$\bullet$};
  \node [style=none] (0) at (-2,0) {$\bullet$};
  \node [style=none] (0) at (0.57,0.57) {$\bullet$};
  \node [style=none] (0) at (-0.58,-0.58) {$\bullet$};
  \node [style=none] (0) at (0,2.5) {$T1$};
  \node [style=none] (0) at (0,-2.5) {$T\bigl(\!\begin{smallmatrix}
    0 & 1 \\ -1 & 0
\end{smallmatrix}\!\bigr)$};
  \node [style=none] (0) at (3.25,0) {$T\frac{1}{\sqrt{2}}\bigl(\!\begin{smallmatrix}
    1 & 1 \\ -1 & 1
\end{smallmatrix}\!\bigr)$};
  \node [style=none] (0) at (-3.25,0) {$T\frac{1}{\sqrt{2}}\bigl(\!\begin{smallmatrix}
    -1 & 1 \\ -1 & -1
\end{smallmatrix}\!\bigr)$};
  \node [style=none] (0) at (-0.5,-1) {$T\frac{1}{\sqrt{2}}\bigl(\!\begin{smallmatrix}
    -i & 1 \\ -1 & i
\end{smallmatrix}\!\bigr)$};
  \node [style=none] (0) at (0.5,1) {$T\frac{1}{\sqrt{2}}\bigl(\!\begin{smallmatrix}
    i & 1 \\ -1 & -i
\end{smallmatrix}\!\bigr)$};
\draw[dashed] (-2,0) -- (2,0);
\draw [fill=blue!70, opacity = 0.1] (-5,-3) rectangle (5,3);
%   \draw[->] (-2.5,1.767766) arc (135:225:2.5);
%   \draw[->] (2.5,-1.767766) arc (-45:45:2.5);
%   \node [style=none] (0) at (-4,1) {$j(-)\overline j$};
%   \node [style=none] (0) at (4,-1) {$j(-)\overline j$};
\end{tikzpicture}
\end{figure}
In the picture the reflection is across the light blue plane splitting the sphere in half through the middle (through the dashed diameter). Observe that the four points on the top, left, bottom, and right are fixed by the reflection, but the other two points are interchanged. In general, $T\bigl(\!\begin{smallmatrix}
    0 & t \\ -\overline t & 0
\end{smallmatrix}\!\bigr)\in T\backslash N(T)$ acts by conjugation on an element $T\bigl(\!\begin{smallmatrix}
    z & w \\ -\overline w & \overline z
\end{smallmatrix}\!\bigr)\in T\backslash \SU(2)$ by
\[\bigl(\!\begin{smallmatrix}
    0 & t \\ -\overline t & 0
\end{smallmatrix}\!\bigr)T\bigl(\!\begin{smallmatrix}
    z & w \\ -\overline w & \overline z
\end{smallmatrix}\!\bigr)\bigl(\!\begin{smallmatrix}
    0 & -t \\ \overline t & 0
\end{smallmatrix}\!\bigr) = T\bigl(\!\begin{smallmatrix}
    0 & t \\ -\overline t & 0
\end{smallmatrix}\!\bigr)\bigl(\!\begin{smallmatrix}
    z & w \\ -\overline w & \overline v
\end{smallmatrix}\!\bigr)\bigl(\!\begin{smallmatrix}
    0 & -t \\ \overline t & 0
\end{smallmatrix}\!\bigr) = T\bigl(\!\begin{smallmatrix}
    \overline z & \overline w \\ -w & z
\end{smallmatrix}\!\bigr).\] 
The corresponding action on $\mathbb{CP}^1$ is the involution $[z:w]\mapsto [\overline z:\overline w]$.

The quotient group $N(T)/T$ is called the Weyl group, denoted $W$, which in this setting is isomorphic to $\mathbb Z/2\mathbb Z$. We can also think of the Weyl group as the symmetric group on two elements. The action of $W$ on $T$ by conjugation is to swap the eigenvalues of elements of $T$, but there is another manifestation of this idea if we return to representations of $\SU(2)$.

Let $V$ be a representation of $\SU(2)$ and consider its weight space decomposition $V = \bigoplus_{n\in\mathbb Z}V_n$ with respect to $T$. Since $T$ centralizes $T$, the conjugation action of $N(T)$ on $T$ can be replaced by the faithful conjugation action of $N(T)/T$ on $T$. We saw before that for $h\in N(T)/T$, $\chi_n(h^{-1}(-)h)$ defines a character of the torus denoted $\chi_m$. Since the action of $h$ by conjugation is by complex conjugation, $\chi_m = \chi_{-n}$. Therefore the action of $h$ on $V$ interchanges $V_n$ with $V_{-n}$ (which also implies $V_n\cong V_{-n}$) for all $n$.

It follows that characters of representations of $\SU(2)$ viewed as polynomials in $z$ are palindromic; that is, if $cz^n$ appears as a term in $\chi_V(z)$, then $cz^{-n}$ must also appear as a term. In particular, for even irreps of $\SU(2)$, the only powers of $z$ that could appear in $\chi_V(z)$ are $z^n,z^{-n}$ for $n$ even. An example of a palindromic polynomial that might occur for an irrep $V$ (even or odd, this would just determine if $n$ was even or odd in the following expression) was 
\[z^n + 72z^{n-2} + 46z^{n-4} + 46z^{-n+4} + 72z^{-n+2} + z^{-n}\]
We will see much later that this Laurent polynomial actually could not be a character of an irrep $V$.

For $V = \mathbb C^2$ with $\SU(2)$ acting on $V$ by matrix multiplication, $\mathbb C^2$ has weight space decomposition $\mathbb C^2 = \mathbb Ce_1 + \mathbb Ce_2$ that is just the standard basis decomposition. The action of an element of the torus $T$ on the first summand is to scale by one eigenvalue (its top-left entry), and the action of an element of the torus $T$ on the second summand is to scale by the other eigenvalue. Therefore the character of this representation is $\chi_V(z) = z+z^{-1}$.

\subsection{Choice of torus?}
At the beginning we chose $T$ to be the maximal torus generated by $1,i$; that is, by the matrices $\bigl(\!\begin{smallmatrix}
    z & 0 \\ 0 & \overline z
\end{smallmatrix}\!\bigr)$ for $z\in\U(1)$. But there are several (infinitely many) other isomorphic copies of $\U(1)$ in $\SU(2)$ that could have been chosen in place of $T$. One might object and say that the above theory might have depended on this particular choice of $T$. We will see that it did not matter. 

The first fact we need is that any two maximal toruses in $\SU(2)$ are conjugate to each other. This does not constitute a proof, but we can see this visually:  A maximal torus of $\SU(2)$ is the same as the group of counterclockwise rotations of $S^2$ about a fixed axis (this picture also exhibits the fact that a maximal torus is isomorphic to $\U(1)^k$ if and only if $k = 1$). Any two maximal toruses then differ only by which axis we choose to rotate about. Obtain an isomorphism of any two maximal toruses by conjugation by a suitable element of $\SU(2)$ which rotates one axis to the other. In other words, the action of $\SU(2)$ on the set of maximal tori by conjugation is transitive. 

The $\SU(2)$-set of maximal tori (with the conjugation action) is isomorphic to the coset space $\SU(2)/\Stab(T)\cong \SU(2)/N(T)\cong (T\backslash \SU(2))/(T\backslash N(T))\cong S^2/(\mathbb Z/2\mathbb Z)$. The action of $\mathbb Z/2\mathbb Z$ on $S^2$ in this case is the antipodal map, so the set of maximal tori can be identified with $\mathbb{RP}^2$. Recall the action of $\SU(2)$ on pure quaternions by conjugation; we identified this with the representation $\mathbb R^3$ of $\SU(2)$. From this representation we obtained a two-to-one map $\SU(2)\to\SO_3(\mathbb R)$. The action of $\SO_3(\mathbb R)$ on $\mathbb R^3$ is transitive, and therefore the action of $\SU(2)$ on $\mathbb R^3$ is also. Furthermore, the actions of $\SU(2),\SO_3(\mathbb R)$ on $\mathbb R^3$ descend to transitive actions on $\mathbb{RP}^2$, since the actions move one-dimensional subspaces to one-dimensional subspaces. The stabilizer in $\SO_3(\mathbb R)$ of a point in $\mathbb{RP}^2$, or rather of a line passing through the origin in $\mathbb R^3$, is a copy of $\O_2(\mathbb R)$ in $\SO_3(\mathbb R)$. For example, the stabilizer of the line through the standard basis vector $e_1$ is the copy of $\O_2(\mathbb R)$ generated by 
\[\begin{pmatrix}
    1 & 0 & 0 \\
    0 & a & b \\
    0 & c & d
\end{pmatrix}, \begin{pmatrix}
    -1 & 0 & 0 \\
    0 & 1 & 0 \\
    0 & 0 & -1
\end{pmatrix}\]
where $\bigl(\!\begin{smallmatrix}
    a & b \\ c & d
\end{smallmatrix}\!\bigr)\in \SO_2(\mathbb R)$. So $\SO_3(\mathbb R)/\O_2(\mathbb R)$ is isomorphic to $\mathbb{RP}^2$. The preimage of this copy of $\O_2(\mathbb R)$ in $\SU(2)$ is $N(T)$, since the map $\SU(2)\to\SO_3(\mathbb R)$ is given by 
\[\begin{pmatrix}
    z & w \\
    -\overline w & \overline z
\end{pmatrix}\mapsto \begin{pmatrix}
    \abs{z}^2-\abs{w}^2 & iw\overline z-iz\overline w & w\overline z + z\overline w\\
    -izw+i\overline z\overline w & (w^2+\overline w^2 + z^2+\overline z^2)/2 & -i(w^2-\overline w^2-z^2+\overline z^2)/2\\
-zw-\overline z\overline w & -i(w^2-\overline w^2+z^2-\overline z^2)/2 & -(w^2+\overline w^2-z^2-\overline z^2)/2\end{pmatrix}\]
The representation $\SU(2)\to \SO_3(\mathbb R)$ above is sometimes called the adjoint representation of $\SU(2)$.

Maximal tori are similar to Sylow subgroups in that the maximal tori are all conjugate to each other and that every element of $\SU(2)$ belongs to a maximal torus since conjugation by an element of $\SU(2)$ fixes an axis in $\mathbb R^3$; that is, conjugation fixes some pure quaternion.

Since any element $g\in \SU(2)$ belongs to a maximal torus $T^\prime$, we may conjugate $g$ by a suitable element $h\in \SU(2)$ so that $hgh^{-1}\in T$. Then $\chi_V(g) = \chi_V(hgh^{-1})$ can be explicitly computed using the Laurent polynomial expression for $\chi_V$ obtained from the weight space decomposition of $V$ with respect to $T$. This shows that the choice of the torus $T$ does not change the form of the Laurent polynomial representing $\chi_V(z)$ (any two such Laurent polynomials would agree on an infinite number of points).

\subsection{Matrix elements}
We take a detour to discuss matrix elements, which will explain where characters come from, and will lead to the Peter-Weyl theorem.

From $v\in V$ and $w\in V^\ast$, define $f_{v,w}\colon G\to \mathbb C$ by $g\mapsto \abr{w,gv}\coloneqq w(gv)$. We call the functions $f_{v,w}$ for $v\in V$ and $w\in V^\ast$ matrix elements, and $f_{v,w}(g)$ the matrix elements of $g$. Note that this definition in terms of $V$ and $V^\ast$ is coordinate-free, but recovers what we think are ``matrix elements''. If we fix a basis $\{e_i\}$ of $V$ and give $V^\ast$ the dual basis, then $f_{e_j^\ast,e_k}(g) = \abr{e_j^\ast,ge_k}$ denotes the $(j,k)$-entry in the matrix for $g\cdot$ in the given basis for $V$. Collect all the matrix elements via a map $V\otimes V^\ast\xrightarrow{f_{-,-}} \Fun(G)$. 

A different perspective: Let $V$ be a finite-dimensional representation of a group $G$; that is, consider a map $G\xrightarrow{\rho}\GL(V)\subset \End(V)\cong V\otimes V^\ast$. Given any map $F$ in $(V\otimes V^\ast)^\ast$ we may pull back $F$ along $\rho$ to obtain a map $G\xrightarrow{\rho^\ast F}\mathbb C$. So $\rho^\ast$ defines a map from $(V\otimes V^\ast)^\ast$ to $\Fun(G)$. The identification $(V\otimes V^\ast)^\ast\cong V^\ast\otimes V^{\ast\ast}\cong V^\ast\otimes V\cong V\otimes V^\ast$ shows that the matrix elements map $f_{-,-}$ agrees with $\rho^\ast$, which is really some kind of adjoint map (i.e., an adjoint map of vector spaces) to the map $\rho$ from $G$ (or by linearly extending $\rho$, from a suitably chosen $\Fun(G)$) to $\End(V)\cong V\otimes V^\ast$.

We would like to investigate what kinds of functions the matrix elements are. For example, if $V$ is a continuous representation, then the matrix elements are continuous functions, and if $G$ is compact, then the matrix elements are also $L^2$-integrable. 

Since $V$ is a representation of $G$, the vector space $V\otimes V^\ast$ is a $G\times G$-representation by the action $(g,h)(v\otimes w) = hv\otimes gw$. The reason this looks backwards is because we identified $V^\ast\otimes V$ with $V\otimes V^\ast$ by the map $w\otimes v\to v\otimes w$; really the matrix elements map should have been defined on $V^\ast\otimes V$ to begin with. Furthermore, we can identify $V\otimes V^\ast$ with $\End(V)$ as vector spaces by the map $v\otimes w\mapsto w(-)v$, but we can upgrade the isomorphism to an isomorphism of $G\times G$-representations if we let $G\times G$ act on $\End(V)$ by $(g,h)f = hf(g^{-1}(-))$. The space $\Fun(G)$ is also a $G\times G$-representation where $(g,h)f = f(g^{-1}(-)h)$ (combining both the left and right actions of $G$ on $\Fun(G)$). Then the matrix elements map $V\otimes V^\ast\to \Fun(G)$ is a $G\times G$-equivariant map:
\[hv\otimes gw\mapsto f_{hv,gw} = \abr{gw,(-)hv} = \abr{w,g^{-1}(-)hv} = gf_{v,w}h\]

Now assume $V$ is an irrep of $G$. Then $V\otimes V^\ast$ is an irrep of $G\times G$ (See \href{https://math.stackexchange.com/questions/5013533/proof-that-boxtimes-of-irreducible-representations-is-again-irreducible}{this MSE post}). In this case the matrix element map $f_{-,-}$ is either zero or injective (maps out of simple modules are either zero or injective). If $V$ is not the zero space, then $V\otimes V^\ast\cong \End(V)$ is not trivial since it has $\id_V$. (This is a basis free way of finding $\id_V$ in $V\otimes V^\ast$; if we fix a basis $\{e_i\}$ of $V$, then $\id_V$ corresponds to the element $\sum_i e_i\otimes e_i^\ast$.) The element $\id_V\in V\otimes V^\ast$ is sent to the matrix element $g\mapsto \abr{e_i^\ast,ge_i} = \Tr_V(g\cdot)$, which is the character $\chi_V$ (of course, this means the whole subspace $\mathbb C\id_V\subset V\otimes V^\ast$ is sent to the subspace $\mathbb C\chi_V\subset \Fun(G)$). In particular, notice that $\chi_g(\id_G) = \dim V\neq 0$. Therefore the matrix elements map $f_{-,-}$ is a nonzero map and hence is injective in the case that $V$ is an irrep.

Since $V$ is an irrep, Schur's lemma says that the space $\End_G(V)$ of $G$-equivariant endomorphisms of $V$ is equal to $\mathbb C\id_V$. Consider the diagonal embedding of $G$ into $G\times G$ by $g\mapsto (g,g)$. Through this map we can think of $V\otimes V^\ast$ as a $G$-representation by the action $g(v\otimes w) = (g,g)(v\otimes w) = gv\otimes gw$, and similarly of $\Fun(G)$ by $gf = (g,g)f = f(g^{-1}(-)g)$. Observe that the subspace $(V\otimes V^\ast)^G$ of $G$-invariants in $V\otimes V$ is isomorphic to $\End_G(V)$, which is isomorphic to $\mathbb C\id_V$ since $V$ is an irrep. Therefore the image of $(V\otimes V^\ast)^G\cong \mathbb C\id_V$ under $f_{-,-}$, $\mathbb C\chi_V$, is contained in the space $\Fun(G)^G$ of (continuous/integrable/etc.) class functions as well, since intertwining maps preserve $G$-invariance. This is another way to see that the character $\chi_V$ is invariant under conjugation. If $V$ is not an irrep, then we can still see that the image of $(V\otimes V^\ast)^G\cong \End_G(V)$ under $f_{-,-}$, whatever it may be, is still contained in $\Fun(G)^G$.

There may be some confusion about why we call $\chi_V$ a character if it is not quite the same thing as the characters we looked at before in the Abelian group case. If $G$ is compact Abelian, then the character $\chi_V= \Tr_V$ of an irrep $V$ of $G$ will agree with whatever character $\chi$ the action of $G$ on $V$ is given by. That is, since $V$ is unitarizable and one-dimensional, $G$ will act on $V$ by multiplication by $\chi(g)$ for some character $\chi$, and the trace of $g\cdot$ is exactly $\chi(g)$ as a result. In general for non-Abelian groups we should not expect to match characters of representations with characters of their groups, but the previous example might explain why the name ``character'' is used for both. As an example, there are no nontrivial group homomorphisms $\SU(2)\to \U(1)$, but we can and should consider characters of representations of $\SU(2)$.

\subsection{Preview of the Peter-Weyl theorem}
Let $V,W$ be distinct irreps of a compact group $G$. The matrix elements map on $(V\oplus W)\otimes (V\oplus W)^\ast$ is $(v,w)\otimes(s,t)\mapsto f_{(v,w),(s,t)}$ where 
\[f_{(v,w),(s,t)}(g) = \abr{(s,t),(gv,gw)} = \abr{(s,0),(gv,0)} + \abr{(s,0),(0,gw)} + \abr{(0,t),(gv,0)} + \abr{(0,t),(0,gw)}\]
The middle two terms are zero, and using the identification $(V\oplus W)\otimes (V\oplus W)^\ast\cong (V\otimes V^\ast) \oplus (W\otimes V^\ast) \oplus(V\otimes W^\ast)\oplus(W\otimes W^\ast)$, we see that $f_{(v,w),(s,t)} = f_{v,s} + f_{w,t}$, where the $f_{-,-}$ appearing on the right side are on $V\otimes V^\ast$ and $W\otimes W^\ast$, respectively. So the matrix elements map on $(V\oplus W)\otimes (V\oplus W)^\ast$ really restricts to the direct sum of the matrix elements maps $f_{-,-}\oplus f_{-,-}$ on $(V\otimes V^\ast) \oplus (W\otimes W^\ast)\cong \End(V)\oplus \End(W)$. Since $V,W$ are irreps, $f_{-,-}$ on each of $(V\otimes V^\ast)$ and $(W\otimes W^\ast)$ is injective, so the direct sum $f_{-,-}\oplus f_{-,-}$ is injective and hence the matrix elements map $f_{-,-}$ on $(V\oplus W)\otimes (V\oplus W)^\ast$ is also injective. Here it was important that $W$ was a distinct irrep from $V$ since $GV$ is equal to $V$ and not a subset of $W$. 

We can be ambitious and try to move from two distinct irreps to every irrep of $G$ counted once. That is, we look at the matrix elements map on $\bigoplus_{V\text{ irrep}}V\otimes V^\ast$, and the same argument as above shows that the matrix elements map defined on this direct sum is also injective, mapping into $\Fun(G)$, or really into $C(G)$ or $L^2(G)$. The image of the subspace $\bigoplus_{V\text{ irrep}} \mathbb C\id_V$ in $L^2(G)$ is $\bigoplus_{V\text{ irrep}}\mathbb C\chi_V$. We can give $L^2(G)$ a nice inner product $\abr{-,-}$ given by $\abr{f,g} = \int_G f\overline g\dd \mu$ where $\mu$ is the normalized Haar measure for $G$; by doing so the characters $\chi_V$ for irreps $V$ form an orthogonal family in $L^2(G)$. Note that the Haar measure $\mu$ is $G\times G$-invariant (left and right-translation invariant), so the inner product on $L^2(G)$ is also $G\times G$-invariant as well.

Part of the statement of the Peter-Weyl theorem is as follows: The image of $\bigoplus_{V\text{ irrep}} V\otimes V^\ast$ in $L^2(G)$ under the matrix elements map is dense in $L^2(G)$. If $G$ is $\U(1)$, then the Peter-Weyl theorem states that the Fourier modes $\exp(inx)$ form a dense orthogonal basis of $L^2(\mathbb R/\mathbb Z)$.

The functions coming from the image of $\bigoplus_{V\text{ irrep}}V\oplus V^\ast$ in $L^2(G)$ are the algebraic functions on $G$; that is, we can identify $\bigoplus_{V\text{ irrep}}V\oplus V^\ast$ with $\mathbb C[G]$ (where this notation means the coordinate ring of $G$, and since the matrix elements map is injective we might as well just think about $\bigoplus_{V\text{ irrep}}V\oplus V^\ast$ instead of its image). In the case $G = \U(1)$, $\mathbb C[G]$ is $\mathbb C[z,z^{-1}]$.
\end{document}