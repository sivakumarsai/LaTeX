\documentclass[../../rtnotes.tex]{subfiles}
\begin{document}
\section{09/16}
\subsection{Central extensions}
We slowly transition away from Abelian groups by first discussing a slightly non-Abelian group, the Heisenberg group. It is only ``slightly'' non-Abelian because it is formed as a central extension of two Abelian groups. 

In general, a central extension of the group $H$ (which need not be Abelian) by an Abelian group $A$ is a group $\tilde H$ which fits in the short exact sequence of groups 
\[1\to A\to \widetilde H\to H \to 1\]
(so $A\to \widetilde H$ is injective and $\widetilde H \to H$ is surjective)  $A\hookrightarrow Z(\widetilde H)$. If $H$ is Abelian then the group $\widetilde H$ is close to being Abelian in the sense that it is a nilpotent group with nilpotency class $2$ (Abelian groups are class $1$).
\end{document}