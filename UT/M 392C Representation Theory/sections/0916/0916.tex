\documentclass[../../rtnotes.tex]{subfiles}
\begin{document}
\section{09/16}
\subsection{Central extensions}
We slowly transition away from Abelian groups by first discussing a slightly non-Abelian group, the Heisenberg group. It is only ``slightly'' non-Abelian because it is formed as a central extension of two Abelian groups. 

In general, a central extension of the group $H$ (which need not be Abelian) by an Abelian group $A$ is a group $\tilde H$ which fits in the short exact sequence of groups 
\[1\to A\to \widetilde H\to H \to 1\]
(so $A\to \widetilde H$ is injective and $\widetilde H \to H$ is surjective)  with $A\hookrightarrow Z(\widetilde H)$. If $H$ is Abelian then the group $\widetilde H$ is close to being Abelian in the sense that it is a nilpotent group with nilpotency class $2$ (Abelian groups have nilpotency class $1$). 

A nilpotent group $G$ of nilpotency class $n$ is a group whose central series of shortest length terminates in the whole group after $n$ many steps; that is, there is a series of normal subgroups $G_i$   
\[1=G_0\lhd G_1\lhd\cdots\lhd G_n = G\]
with $G_{i+1}/G_i\leq Z(G/G_i)$ (or equivalently $[G,G_{i+1}]\leq G_i$).
With this definition it is clear that if $H$ is Abelian, then $\widetilde H$ is of nilpotency class $2$ since its upper central series is
\[1\lhd A\lhd \widetilde H\]

To classify central extensions of $H$ (which need not be Abelian) by $A$, we turn to group cohomology. By some process in group cohomology, we can produce a set (really a group) denoted $H^2(H,A)$, which classifies central extensions of $H$ by $A$ up to isomorphism. As a set, $\widetilde H \cong H\times A$. The multiplication in $\widetilde H$ is given by 
\[(g,a)(h,b) = (gh,f(g,h)ab)\]
where $f\colon H\times H \to A$ is a set map satisfying the $2$-cocycle conditions
\[f(1_H,1_H) = 1_A\quad\text{and}\quad f(g,h)f(gh,j) = f(g,hj)f(h,j)\]
for $g,h,j\in H$. The first condition is needed if we want the restriction of the product to $A$ viewed as a subgroup of $\widetilde H$, as $\{1_H\}\times A \subseteq \widetilde H$, to agree with the product in $A$. The second condition is to ensure the multiplication is associative. (See \href{https://www.mathematik.uni-muenchen.de/~schotten/LNP-cft-pdf/03_978-3-540-68625-5_Ch03_23-08-08.pdf}{this MSE question}.)

It is important to note that a central extension is not necessarily a semidirect product of $H$ and $A$; the semidirect products of $H$ and $A$ are the split extensions; that is, the extensions
\[1\to A\to \widetilde H\xrightarrow p H\to 1\]
where we do not assume $A\hookrightarrow Z(\widetilde H)$ (or that $A$ is Abelian; i.e., a possibly non-central extension) but there is a homomorphism $s\colon H \to\widetilde H$ that is a section; that is, $ps=\id_H$. A central extension of $H$ by $A$ that is also a split extension makes $\widetilde H$ isomorphic to the direct product of $H$ and $A$; that is, $\widetilde H \cong H\times A$ (see this \href{https://math.stackexchange.com/questions/2386212/central-extensions-versus-semidirect-products}{MSE} question). The split central extensions correspond to the identity element in the group $H^2(H,A)$.

\subsection{The Heisenberg group}
We return to the Heisenberg group. Let $G$ be an LCA group and let $\widehat G = \Hom_{\Group}(G,\U(1))$. Then the universal character $\chi(-,-)$ is a map from $G\times \widehat G$ to $\U(1)$, and if $G$ is finite the universal character really maps into a finite set of roots of unity $\mu_n$ for some $n$. The Heisenberg group $\Heis$ is obtained as a central extension 
\[1\to \U(1)\to \Heis\to G\times \widehat G\to 1\]
(where if $G$ is finite, we could replace $\U(1)$ by $\mu_n$ if desired) and the $2$-cocycle $f\colon (G\times \widehat G)\times (G\times \widehat G)\to \U(1)$ appearing in the multiplication formula
\[(g,\hat g, z)(h,\hat h,w) = (gh,\hat g\hat h,f(g,\hat g,h,\hat h)zw)\]
in $\Heis$ (viewed as the set $(G\times \widehat G)\times \U(1)$) is given by 
\[f(g,\hat g,h,\hat h) = \overline{\chi(g,\hat h)}\]
% \[f(g,\hat g,h,\hat h) = \overline{\chi(h,\hat g)}\chi(g,\hat h)\] 
By comparing the results of the following calculations it follows that $f$ really is a $2$-cocycle:
\begin{align*}
    f(g,\hat g,h,\hat h)f(gh,\hat g\hat h,j,\hat j) &= \overline{\chi(g,\hat h)\chi(gh,\hat j)}\\
    f(g,\hat g,hj,\hat h \hat j)f(h,\hat h,j,\hat j) &= \overline{\chi(g,\hat h\hat j)\chi(h,\hat j)}
\end{align*}
We can obtain an isomorphic group by using the $2$-cocycle $\tilde f$ given by $\tilde f(g,\hat g,h,\hat h) = \chi(h,\hat g)$ instead of $f$ in the multiplication formula above. Results from group cohomology imply that we obtain isomorphic groups if $f$ and $\tilde f$ differ by a $2$-coboundary, which is the case:
\[\tilde f(g,\hat g,h,\hat h)^{-1} f(g,\hat g,h,\hat h) = \chi(h,\hat h)\overline{\chi(gh,\hat g\hat h)}\chi(g,\hat g)\]
The $2$-coboundary in this situation is the image of $\chi \colon G\times \widehat G\to \U(1)$ under a particular differential $d^1$ coming from a bar complex, that is, the $2$-coboundary is $d^1\chi \colon (G\times \widehat G)\times (G\times \widehat G)\to \U(1)$ given by $d^1\chi(g,\hat g,h,\hat h) = \chi(h,\hat h)\chi(gh,\hat g\hat h)^{-1}\chi(g,\hat g)$.
% \begin{align*}
%     f(g,\hat g,h,\hat h)f(gh,\hat g\hat h,j,\hat j) &= \overline{\chi(h,\hat g)}\chi(g,\hat h)\overline{\chi(j,\hat g \hat h)}\chi(gh,\hat j)\\
%     f(g,\hat g,hj,\hat h \hat j)f(h,\hat h,j,\hat j) &= \overline{\chi(hj,\hat g)}\chi(g,\hat h\hat j)\overline{\chi(j,\hat h)}\chi(h,\hat j)
% \end{align*}

As an aside, some Abelian subgroups of $\Heis$ are $G\times \{1_{\widehat G}\}\times \U(1)$, $\{1_G\}\times\widehat G\times \U(1)$, $G\cong G\times\{1_{\widehat G}\}\times \{1_{\U(1)}\}$, and $\widehat G\cong \{1_G\}\times \widehat G\times \{1_{\U(1)}\}$. The subgroups  $G\times \{1_{\widehat G}\}\times \U(1)$ and $\{1_G\}\times\widehat G\times \U(1)$ do not commute with each other!

In the past, we had three group actions on $\Fun(G)$. One was the action of $G$ by translation, another was the action of $\widehat G$ by multiplication by characters, and lastly the usual scalar multiplication by $\U(1)$. The action of $\U(1)$ commutes with the first two group actions (and of course with its own group action). We saw before that the actions of $G,\widehat G$ do not commute, which was why we formed the Heisenberg group in the first place. These three group actions can be thought of as living in $\Aut(\Fun(G))$. The Heisenberg group collects all the commutation relations between the different group actions, and so we can say that the Heisenberg group can be presented as generated by the group actions of $G$, $\widehat G$, and $\U(1)$ modulo their relations in $\Aut(\Fun(G))$. For example, we recover the commutation relation $[g\cdot,\hat g\cdot] = \overline{\chi(g,\hat g)}$ as the commutator of the corresponding elements in $\Heis$:
\begin{align*}
    [(g,1,1),(1,\hat g,1)] &= (g^{-1},1,1)(1,\hat g^{-1},1)(g,1,1)(1,\hat g,1)\\
    &= (g^{-1},\hat g^{-1},\overline{\chi(g,\hat g)})(g,\hat g,\overline{\chi(g,\hat g)})\\
    &= (1,1,\overline{\chi(g,\hat g)})
\end{align*}

One particular form of the Heisenberg group for $G = \mathbb R_x^n$ that may be familiar is as a matrix group. To make this identification, first see that the dual group $\widehat G = \mathbb R_t^n$ is really the dual vector space to $G$; that is, there is a natural isomorphism between $\widehat G = \Hom_{\Group}(\mathbb R_x^n,\U(1))$ and $G^\ast = \Hom_{\mathbb R}(\mathbb R_x^n,\mathbb R)$ that identifies $x\mapsto \exp(i\abr{x,t})$ with $x\mapsto \abr{x,t}$ for $t\in\mathbb R_t$. Then the Heisenberg group for $G = \mathbb R_x$ (i.e., $n=1$) is obtained from the central extension
\[1\to \U(1)\to \Heis\to \mathbb R_x\times\mathbb R_x^\ast\to 0\]
Further identifications of 
\begin{align*}
    \U(1)\quad&\text{with}\quad\Bigl\{\Bigl(\!\begin{smallmatrix}
    1 & 0 & z\\
    0 & 1 & 0\\
    0 & 0 & 1
\end{smallmatrix}\!\Bigr)\Bigm\vert z\in\mathbb R/\mathbb Z\Bigr\}
\intertext{and}
\mathbb R_x\times\mathbb R_x^\ast\quad&\text{with}\quad \Bigl\{\Bigl(\!\begin{smallmatrix}
    1 & x & 0\\
    0 & 1 & t\\
    0 & 0 & 1
\end{smallmatrix}\!\Bigr)\Bigm\vert x\in\mathbb R_x,t\in\mathbb R_t\Bigr\}
\end{align*}
imply that $\Heis$ can be identified with the matrix ``group''
\[\Bigl\{\Bigl(\!\begin{smallmatrix}
    1 & x & z\\
    0 & 1 & t\\
    0 & 0 & 1
\end{smallmatrix}\!\Bigr)\Bigm\vert x\in\mathbb R_x,t\in\mathbb R_t, z\in\mathbb R/\mathbb Z\Bigr\}\]
where the multiplication law is matrix multiplication but we reduce modulo $\mathbb Z$ in the top right entry. Somehow the $2$-cocycle appearing after multiplying two elements doesn't exactly match the sign conventions in the usual $2$-cocycle we use for the Heisenberg group, but nevertheless we recover the commutation relation
\[\Bigl[\Bigl(\!\begin{smallmatrix}
    1 & x & 0\\
    0 & 1 & 0\\
    0 & 0 & 1
\end{smallmatrix}\!\Bigr),\Bigl(\!\begin{smallmatrix}
    1 & 0 & 0\\
    0 & 1 & t\\
    0 & 0 & 1
\end{smallmatrix}\!\Bigr)\Bigr] = \Bigl(\!\begin{smallmatrix}
    1 & 0 & -xt\\
    0 & 1 & 0\\
    0 & 0 & 1
\end{smallmatrix}\!\Bigr)\]
Then in general the Heisenberg group for $G = \mathbb R_x^n$ can be thought of as the matrix ``group''
\[\Bigl\{\Bigl(\!\begin{smallmatrix}
    1 & x & z\\
    0 & 1 & t\\
    0 & 0 & 1
\end{smallmatrix}\!\Bigr)\Bigm\vert x\in\mathbb R_x,t\in\mathbb R_t, z\in\mathbb R/\mathbb Z\Bigr\}\] where the multiplication law is given by
\[\Bigl(\!\begin{smallmatrix}
    1 & x & z\\
    0 & 1 & t\\
    0 & 0 & 1
\end{smallmatrix}\!\Bigr)\Bigl(\!\begin{smallmatrix}
    1 & y & w\\
    0 & 1 & s\\
    0 & 0 & 1
\end{smallmatrix}\!\Bigr) = \Bigl(\!\begin{smallmatrix}
    1 & x+y & z+w+\abr{x,s}\\
    0 & 1 & t+s\\
    0 & 0 & 1
\end{smallmatrix}\!\Bigr)\]
where the top right entry is reduced modulo $\mathbb Z$. In any case, this point of view is very coordinate-ful and thus might not be a good way to think about the Heisenberg group in general.

\subsection{Representation theory of the Heisenberg group}
The representation theory of the Heisenberg group can be used to obtain slick proofs of results in Fourier analysis.

One important result in this vein is that $\Fun(G)$ is an irrep of $\Heis$ (it will not be exactly $\Fun(G)$ but some analysis is needed to get the function space correct; we need to have an inner product, for example, so we should think of $L^2(G)$ instead). This is to say that $\Fun(G)$ is a simple $\mathbb C[\Heis]$-module, and again what is meant by the group algebra is not clear, but it should include things like delta distributions and other odd elements that might not be nice functions. Contained inside this group algebra are the group algebras $(\mathbb C[G],\ast)$ and $(\mathbb C[\widehat G],\ast)$. The action of $\mathbb C[G]$ on $\Fun(G)$ is given by convolution; that is,
\[F\cdot f = F\ast f\]
for $F\in\mathbb C[G]$. Indeed, thinking of $g\in G$ as the element $\delta_g\in \mathbb C[G]$, we recover the usual action $g\cdot f = \delta_g\ast f = f(g^{-1}-)$. The action of $\mathbb C[\widehat G]$ on $\Fun(G)$ is given by first identifying $(\mathbb C[\widehat G],\ast)$ with $(\mathbb C[G],\cdot)$ by taking the inverse Fourier transform and then pointwise multiplying. Specifically this is
\[H\cdot f = (H\ast \hat f)^\vee = H^\vee \cdot f\]
for $H\in\mathbb C[\widehat G]$. Indeed, thinking of $\hat g\in \widehat G$ as the element $\delta_{\hat g}$, we recover the usual action $\hat g\cdot f = (\delta_{\hat g}\ast \hat f)^\vee = \chi(-,\hat g)f$.

To show that $\Fun(G)$ is an irrep of $\Heis$ we show that any nonzero function $f$ can be sent to $\delta_{1_G}$ by acting on $f$ by elements of the group algebra of $\Heis$ (and from the delta function we can reconstruct any other function in $\Fun(G)$). Indeed, given any $f\in \Fun(G)$ that is not the zero function, we can act on $f$ by elements of $\mathbb C[\Heis]$ which take $f$ to $\delta_{1_G}$. Since $f(g) \neq 0$ for some $g\in G$, translate and scale $f$ so that $f(1_G) = 1$. We would like to take the pointwise product $\delta_{1_G}\cdot f$ to obtain $\delta_{1_G}$, but we obtain this product by taking the inverse Fourier transform of convolution against the character $\chi(1_G,-)\in\mathbb C[\widehat G]$. That is,
\[\chi(1_G,-)\cdot f = (\chi(1_G,-)\ast\hat f)^\vee = \delta_{1_G}\cdot f = \delta_{1_G}\]
In general we need some analysis to approximate these delta distributions with nice enough functions to obtain the result, but this is a good enough sketch of a proof.

The action of the Heisenberg group on $L^2(\widehat G)$ is given by swapping the roles of $G$ and $\widehat G$ up to a sign since we want the action to commute with taking the Fourier transform. The element $(g,\hat g, z)\in \Heis$ acts on $f\in \Fun(G)$ by
\[(g,\hat g, z)\cdot f = z\chi(-,\hat g)f(g^{-1}-)\]
and on $F\in \Fun(\widehat G)$ by 
\[(g,\hat g, z)\cdot F = z\chi(g,\hat g)\chi(g,-)F(\hat g-)\]
(this is just interchanging the roles of $G$ and $\widehat G$ in some sense.) Then the Fourier transform (and hence also the inverse Fourier transform) intertwines with this action:
\[\widehat{(g,\hat g,z)\cdot f} = \widehat{z\chi(g,-)\delta_g\ast f} = z\delta_{\hat g^{-1}}\ast(\chi(g,-)\hat f) = z\chi(g,\hat g)\chi(g,-)\hat f(\hat g-) = (g,\hat g, z)\cdot \hat f\]
Since this action is not really any different than the original action we defined, $\Fun(\widehat G)$ is also an irrep of $\Heis$ (and we can repeat this as well for $\Fun(\dwidehat G)$).

Let $G = \mathbb R_x$. That the Fourier transform is a unitary isomorphism of $L^2(G)$ with $L^2(\widehat G)$ can also be proved knowing some kind of unitary Schur's lemma, which comes from analysis. Identify $G$ and $\widehat G$, as well as $L^2(G)$ with $\overline{L^2(G)}$ and $L^2(G)^\ast$. The argument in analysis is that the Fourier transform is densely defined and can be closed (we can take limits of Fourier transforms), so its adjoint is also densely defined so the Fourier transform must be unitary up to a constant multiple. Morally what is happening is that intertwining operators of unitary representations are typically unitary operators (they would ``have to be terrible not to be unitary'' according to \href{https://utexas.instructure.com/courses/1428938/files/86306747?wrap=1}{Howe}). This is usually known as Plancherel's theorem. (See p.452 in \href{https://59clc.wordpress.com/wp-content/uploads/2012/08/real-and-functional-analysis-lang.pdf}{Real and Functional Analysis} by Lang.) In other words, the Fourier transform takes an invariant Hermitian form $\abr{-,-}$ on $L^2(G)$; that is, an isomorphism of $\Heis$-representations $\overline{L^2(G)}\to L^2(G)^\ast$ to another invariant Hermitian form $\abr{\widehat -,\widehat -}$ on $L^2(G)$; that is, to another isomorphism $\overline{L^2(G)}\to L^2(G)^\ast$. But by Schur's lemma these two forms should only differ by a constant multiple and to find out which multiple it suffices to calculate the new Hermitian form on a convenient element to see that the Fourier transform is unitary.

To prove that $~\dhat{\!\!f}(g) = f(g^{-1})$, observe that applying the Fourier transform twice intertwines with correctly chosen actions of the Heisenberg groups. The reflection map $\Fun(G)\xrightarrow{R}\Fun(G)$ given by $Rf = f((-)^{-1})$ also intertwines with the action of the Heisenberg group, so applying the Fourier transform twice is a scalar multiple of $R$. To find out what the multiple is, it suffices to calculate the twice Fourier transform of a single element; usually we pick a convenient element. In the case when $G = \mathbb R$, the Gaussian $f(x) = \exp(-x^2/2)$ is self-dual, meaning $\hat f(t) = \exp(-t^2/2)$ from which it follows that taking the Fourier transform twice really does agree with the reflection map. But this is not too bad to do since $f$ satisfies the differential equation $\dv{x}u+xu = 0$. The Fourier transform of this differential equation is $-it\hat u - i\dv{t}\hat u = 0$; that is, $\dv{t}\hat u+t\hat u = 0$, from which it follows that the Fourier transform of the aforementioned Gaussian is itself.

One of the most important results about the Heisenberg group is the Stone-von Neumann theorem, which roughly says that $\Fun(G)$ is the only genuine irrep of the Heisenberg group up to isomorphism, where by a genuine representation $V$ we mean that the subgroup $\U(1)\subset \Heis$ acts on $V$ by the usual scalar multiplication of complex numbers on $V$.

A stronger result is that the categories $\Rep_{\textrm{genuine}}(\Heis)$ and $\Vect_{\mathbb C}$ are equivalent, which implies the Stone-von Neumann theorem since in this setting $\Fun(G)$ corresponds to the one-dimensional vector space $\mathbb C$. Let $W$ be a genuine representation of $\Heis$. Restricting to the action of $G,\widehat G$ in $\Heis$, view $W$ as a representation of $\widehat G$, which from before it follows that $W$ can be thought of as global sections of a sheaf over $\dwidehat{G}\cong G$. The following picture is slightly inaccurate but we will fix it shortly:
\begin{figure}[h]
    \centering
    \begin{tikzpicture}
        \node [style=none] (0) at (-2.25, 0) {$\bullet$};
		\node [style=none] (2) at (-0.75, 0) {$\bullet$};
		\node [style=none] (3) at (0.75, 0) {$\bullet$};
		\node [style=none] (4) at (2.25, 0) {$\bullet$};
		\node [style=none] (5) at (-3, 0) {};
		\node [style=none] (6) at (3, 0) {};
		\node [style=none] (7) at (-2.25, -0.5) {$g_{-1}$};
		\node [style=none] (9) at (-0.75, -0.5) {$1_G$};
		\node [style=none] (10) at (0.75, -0.5) {$g_1$};
		\node [style=none] (11) at (2.25, -0.5) {$g_2$};
		\node [style=none] (12) at (-2.25, 3) {};
		\node [style=none] (13) at (-2.25, 1) {};
		\node [style=none] (14) at (-0.75, 3) {};
		\node [style=none] (15) at (-1.25, 2.25) {};
		\node [style=none] (16) at (-0.75, 1) {};
		\node [style=none] (17) at (-0.25, 1.75) {};
		\node [style=none] (18) at (0.75, 3) {};
		\node [style=none] (19) at (0.75, 1) {};
		\node [style=none] (20) at (2.25, 3) {};
		\node [style=none] (21) at (1.75, 2.25) {};
		\node [style=none] (22) at (2.25, 1) {};
		\node [style=none] (23) at (2.75, 1.75) {};
		\node [style=none] (24) at (-2.25, 3.5) {$W_{g_{-1}}$};
		\node [style=none] (25) at (-0.75, 3.5) {$W_{1_G}$};
		\node [style=none] (26) at (0.75, 3.5) {$W_{g_{1}}$};
		\node [style=none] (27) at (2.25, 3.5) {$W_{g_{2}}$};
        \begin{pgfonlayer}{edgelayer}
		\draw [style=new edge style 0] (5.center) to (6.center);
		\draw [style=new edge style 0] (12.center) to (13.center);
		\draw [style=new edge style 0] (18.center) to (19.center);
        \draw [fill=gray!15] (14.center) -- (15.center) -- (16.center) -- (17.center) -- (14.center);
        \draw [fill=gray!15] (20.center) -- (21.center) -- (22.center) -- (23.center) -- (20.center);
	\end{pgfonlayer}
    \end{tikzpicture}
\end{figure}
(The inaccuracy is that the vector spaces $W_g$ have different dimensions; the picture is an example where the drawn vector spaces have dimension $1$ or $2$.)
\newpage
On $W_g$, the group $\widehat G$ acts by multiplication by $\chi(g,\hat g)$. Think of $W_g$ as the $\chi(g,-)$-eigenspace for the $\widehat G$-action on $W$. On the other hand, $G$ acts on $w\in W_g$ by translation, specifically by translating the eigenspace $w$ belongs to. That is, if $h\in G$ and $w\in W_g$, then by the commutation relation of $G$ and $\widehat G$ in $\Heis$ we have 
\[\hat g(hw) = (\hat g h)w = \chi(h,\hat g)(h\hat g)w = \chi(hg,\hat g)(gw)\]

So $h\in G$ sends $w\in W_g$ to $hw\in W_{hg}$. Since multiplication by group elements in $G$ is invertible, the eigenspaces $W_g$ are all isomorphic to each other (i.e. for any $h,g\in G$, $W_g\cong W_{hg}$ as $\widehat G$-representations), so the picture from earlier is more accurately depicted with each stalk the same dimension.
\begin{figure}[h]
    \centering
    \begin{tikzpicture}
        \node [style=none] (0) at (-2.25, 0) {$\bullet$};
		\node [style=none] (2) at (-0.75, 0) {$\bullet$};
		\node [style=none] (3) at (0.75, 0) {$\bullet$};
		\node [style=none] (4) at (2.25, 0) {$\bullet$};
		\node [style=none] (5) at (-3, 0) {};
		\node [style=none] (6) at (3, 0) {};
		\node [style=none] (7) at (-2.25, -0.5) {$g_{-1}$};
		\node [style=none] (9) at (-0.75, -0.5) {$1_G$};
		\node [style=none] (10) at (0.75, -0.5) {$g_1$};
		\node [style=none] (11) at (2.25, -0.5) {$g_2$};
		\node [style=none] (14) at (-0.75, 3) {};
		\node [style=none] (15) at (-1.25, 2.25) {};
		\node [style=none] (16) at (-0.75, 1) {};
		\node [style=none] (17) at (-0.25, 1.75) {};
		\node [style=none] (20) at (2.25, 3) {};
		\node [style=none] (21) at (1.75, 2.25) {};
		\node [style=none] (22) at (2.25, 1) {};
		\node [style=none] (23) at (2.75, 1.75) {};
		\node [style=none] (24) at (-2.25, 3.5) {$W_{g_{-1}}$};
		\node [style=none] (25) at (-0.75, 3.5) {$W_{1_G}$};
		\node [style=none] (26) at (0.75, 3.5) {$W_{g_{1}}$};
		\node [style=none] (27) at (2.25, 3.5) {$W_{g_{2}}$};
		\node [style=none] (28) at (-2.25, 3) {};
		\node [style=none] (29) at (-2.75, 2.25) {};
		\node [style=none] (30) at (-2.25, 1) {};
		\node [style=none] (31) at (-1.75, 1.75) {};
		\node [style=none] (32) at (0.75, 3) {};
		\node [style=none] (33) at (0.25, 2.25) {};
		\node [style=none] (34) at (0.75, 1) {};
		\node [style=none] (35) at (1.25, 1.75) {};
        \draw [fill=gray!15] (14.center) -- (15.center) -- (16.center) -- (17.center) -- (14.center);
        \draw [fill=gray!15] (20.center) -- (21.center) -- (22.center) -- (23.center) -- (20.center);
        \draw [fill=gray!15] (28.center) -- (29.center) -- (30.center) -- (31.center) -- (28.center);
        \draw [fill=gray!15] (32.center) -- (33.center) -- (34.center) -- (35.center) -- (32.center);
        \draw [style=new edge style 0] (5.center) to (6.center);
    \end{tikzpicture}
\end{figure}

The equivalence $\Rep_{\textrm{genuine}}(\Heis)\to \Vect_{\mathbb C}$ is given by taking the stalk of the sheaf we get by smearing a representation $W$ at $1_G$. The functor going the other way is given by taking a complex vector space $V$, letting $\widehat G$ act on it by multiplication by multiplication by $\chi(1_G,-)$; that is, thinking of $V$ as the stalk of a sheaf at $1_G$. Then ``translate'' $V_{1_G} = V$ by $G$ to form copies $V_g$ for which $\widehat G$ acts by multiplication by $\chi(g,-)$. The global section of the sheaf we obtain by translating $V$ this way is a genuine representation of $\Heis$ if throughout we let $\U(1)$ act by usual scalar multiplication. The only way to obtain an irrep of $\Heis$ through this functor is by starting with $V = \mathbb C$, which when translated over $G$ yields a sheaf whose global sections is exactly $\Fun(G)$.

So the role of $\widehat G$ in this picture is to spectrally decompose a representation $W$ and $G$ translates the eigenspaces around. Of course, there is a symmetric picture if we swap the roles of $G$ and $\widehat G$ around.

In physics, we look at the infinitesimal actions of translation from $G = \mathbb R_x$ and multiplication from $\widehat G = \mathbb R_t\cong \mathbb R_x$. From some calculations from earlier, we saw that these correspond to $\dv{x}$ and $x$; the observables of position and momentum (in physics some additional constants are present). These two operators do not commute, and hence cannot be simultaneously diagonalized, which is the content of the Heisenberg uncertainty principle. The Stone-von Neumann theorem says that this is pretty much the only case up to scale.

\subsection{A short preview of the Weyl algebra}
Let $G = \mathbb R_x$. The Weyl algebra is a non-commuting $\mathbb C$-algebra of polynomial differential operators:
\[D = \frac{\mathbb C\abr{x,\partial_x}}{(\partial_xx-x\partial_x - 1)}\]
The Weyl algebra naturally acts on $\Fun(G)$ or rather $L^2(G)$ by unbounded operators, and Hermann Weyl used this algebra and related algebras to study the Heisenberg uncertainty principle in quantum mechanics. It is related to the Heisenberg group from earlier since differentiation and multiplication by $x$ are the infinitesimal actions of $G, \widehat G$ on $L^2(G)$, but there are very interesting differences in the theories surrounding representations of the Weyl algebra and of the Heisenberg group.

The Weyl algebra appears in the theory of $D$-modules, which from one point of view is a way to study the theory of systems of linear partial differential equations using algebra and algebraic geometry. It is (one of) David's favorite algebras.
\end{document}