\documentclass[../../rtnotes.tex]{subfiles}
\begin{document}
\section{09/18}
\subsection{The Heisenberg group and the Weyl algebra}
The Heisenberg group is not a semidirect product of $\U(1)$ and $G\times\widehat G$, but by breaking the symmetry of the pair $G,\widehat G$, it is a semidirect product of $\widehat G\times \U(1)$ and $G$. That is, 
\[\Heis \cong G \ltimes_\varphi \widehat G\times \U(1)\]
where $\varphi\colon G\to \Aut(\widehat G\times \U(1))$ is given by $\phi(g)(\hat h,w) = (\hat h,\overline{\chi(g,\hat h)}w)$. Hence the group multiplication is given by 
\[(g,\hat g,z)(h,\hat h,w) = (gh,\hat g \hat h, \overline{\chi(g,\hat h)}zw)\] 
and agrees with the usual one on $\Heis$.

The observations we made last time about viewing genuine representations of $\Heis$ as a sheaf over $G$ in which $\widehat G, \U(1)$ act in a manner that preserve stalks and $G$ acts by moving vectors between stalks can be recast using the above isomorphism. Genuine irreps of $\widehat G\times \U(1)$ of a fixed dimension are in correspondence with (or live over) the elements of $G$, and $G$ acts on the set of irreps by changing the action of $\widehat G$ on them; that is, by exchanging irreps (last time, this was to exchange $W_g$ with $W_{hg}$).

Last time, we saw that the Weyl algebra $D$ on $\mathbb R_x$ was the noncommutative $\mathbb C$-algebra generated by $x$ and $\partial_x$, subject to the relation $\partial_xx-x\partial_x = 1$. It acts on $L^2(\mathbb R)$, but a better space for the Weyl algebra to act on is the Schwartz space $\mathcal S(\mathbb R)$. The Schwartz space contains the $C^\infty$ functions on $\mathbb R$ whose derivatives of all orders decay faster than the reciprocal of any polynomial as $x$ tends to $\pm \infty$. A precise definition of the Schwartz space and its topology is rather cumbersome, so it suffices to think of the functions living in the Schwartz space as ``rapidly decaying smooth functions''. For example, the Gaussian $\exp(-x^2/2)$ belongs to $\mathcal S(\mathbb R)$.

By design, the Schwartz space is a module for the Weyl algebra, since the functions in the Schwartz space and their derivatives rapidly decrease, multiplication by $x$ and differentiation really define an action on $\mathcal S(\mathbb R)$. By taking the continuous dual of the Schwartz space, we obtain the tempered distributions, denoted $\mathcal S'(\mathbb R)$. The tempered distributions $\mathcal S'(\mathbb R)$ is another candidate for the group algebra for $G = \mathbb R_x$. 

From analysis, the Fourier transform defines operators on $L^2(\mathbb R)$ and on $\mathcal S(\mathbb R)$. By taking adjoints, the Fourier transform also defines an operator on $\mathcal S'(\mathbb R)$. So the space of tempered distributions is a very nice place to do harmonic analysis on, and we would like to do something similar in the future for non-Abelian groups.

There are no nonzero finite-dimensional $D$-modules. Suppose $V$ is a finite-dimensional $D$-module. Then
\[\dim(V) = \Tr_V(1) = \Tr_V(\partial_xx-x\partial_x) = \Tr_V(\partial_xx)-\Tr_V(x\partial_x) = \Tr_V(\partial_xx)-\Tr(\partial_xx) = 0\]
This result may be thought of as an algebraic version of the Heisenberg uncertainty principle, since not only can we not simultaneously diagonalize $\partial_x$ and $x$, but we cannot have these operators act on finite dimensional vector spaces.

The exponentials $\exp(iy-)$ of $i\partial_x$ and of $x$ give the translation $\tau_{y}$ and multiplication by the character $\exp(iyx)$, and these generate the Heisenberg group. By starting with a $D$-module, we can obtain a module over $\Heis$ by differentiating the action of $G,\widehat G$ to obtain differentiation and multiplication by $x$. To be careful, we need analysis because $D$-modules are not finite-dimensional.

\subsection{Quick introduction to $D$-modules}
The data of a finitely presented $D$-module is the same as a system of linear partial differential equations with polynomial coefficients. We will come back to this claim after looking at small examples.

Let $M = D/D(\partial_x-\lambda)$ be a left $D$-module (left, right are important since $D$ is not commutative). Then for some other $D$-module $F$ which is typically a nice function space, $f\in \Hom_D(M,F)$ can be thought of as a solution to the differential equation $(\partial_x-\lambda)u = 0$ living in the function space $F$. This is because any $f\in \Hom_D(M,F)$ is zero on $D(\partial_x-\lambda)$ and any $f\in \Hom_D(D,F)$ which is zero on $D(\partial_x-\lambda)$ descends to a well defined homomorphism $f\in \Hom_D(M,F)$. That is, 
\[\Hom_D(M,F)\cong \{f\in \Hom_D(D,F)\mid f(\partial_x-\lambda) = 0\}\]
The map $f\mapsto f(1)$ defines an isomorphism of $\Hom_D(D,F)$ with $F$, so in particular 
\[\{f\in \Hom_D(D,F)\mid f(\partial_x-\lambda) = (\partial_x-\lambda)f(1) = 0\}\cong \{f\in F\mid (\partial_x-\lambda)f = 0\}\]
So if $F$ is a $D$-module that contains $\exp(\lambda x)$, then $\Hom_D(M,F)$ is $\mathbb C\exp(\lambda x)$, otherwise $\Hom_D(M,F)$ is zero. One example of an $F$ that has exponentials is the continuous dual of $C_c^\infty(\mathbb R)$, the usual space of distributions on $\mathbb R$ (analysis shows that $\exp(\lambda x)$ does not define a tempered distribution on $\mathbb R$).
\end{document}