\documentclass[../../rtnotes.tex]{subfiles}
\begin{document}
\section{09/25}
\subsection{The Peter-Weyl theorem}
Let $G$ be $\SU(2)$ or in general a compact Lie group and let $V$ be a finite-dimensional (continuous, we have been and will continue to assume this) representation of $G$. Since $V$ is a continuous representation of $G$, the matrix elements map $f_{-,-}$ defined last time has image lying in $C(G)$; that is, the matrix elements $f_{v,w}$ for $v\in V$ and $w\in V^\ast$ are continuous functions.

Last time we also saw that the matrix elements map $f_{-,-}$ was a $G\times G$-intertwining map by carefully defining the actions of each factor of $G$ on $V\otimes V^\ast$ and $\Fun(G)$. Furthermore, we saw that if $V$ was an irrep, then $V\otimes V^\ast$ was a $G\times G$-irrep so the matrix elements map $f_{-,-}$ was injective. And if $V,W$ were irreps, then the matrix elements map on $(V\oplus W)\otimes (V\oplus W)^\ast$ decomposed into the injective map $f_{-,-}\oplus f_{-,-}$ on $(V\otimes V^\ast)\oplus (W\otimes W^\ast)$, which implies the matrix elements map on $(V\oplus W)\otimes (V\oplus W)^\ast$ was injective to begin with. Then we defined the matrix elements map on $\bigoplus_{V\text{ irrep}}V\otimes V^\ast$, which maps injectively into $C(G)$. 

But the expression $\bigoplus_{V\text{ irrep}}V\otimes V^\ast$ maybe isn't so clear, since we would like for any irrep $W$ of $G$ to find $W\otimes W^\ast$ in the direct sum $\bigoplus_{V\text{ irrep}}V\otimes V^\ast$. For each irrep $V$, fix a basis of $V$ and give $V^\ast$ the corresponding dual basis. Then by Schur's lemma there is only one map taking $W\otimes W^\ast$ to $\bigoplus_{V\text{ irrep}}V\otimes V^\ast$ which preserves the scale of the elements in $W\otimes W^\ast$; that is, $\id_W$ is sent to $(\dots,0,\id_W,0,\dots)$. So it would not matter which isomorphism class of each irrep $V$ we take in the direct sum $\bigoplus_{V\text{ irrep}}V\otimes V^\ast$.

Each of the irreps $V\otimes V^\ast$ are unitarizable since $G$ is compact, so by fixing a suitable inner product on each summand $V\otimes V^\ast$ in $\bigoplus_{V\text{ irrep}}V\otimes V^\ast$ we can give $\bigoplus_{V\text{ irrep}}V\otimes V^\ast$ an inner product for which the elements $\id_V$ in $\bigoplus_{V\text{ irrep}}V\otimes V^\ast$ are orthogonal. This inner product can also be chosen so that the matrix elements map $f_{-,-}$ is a map of unitary representations where $C(G)$ is given a suitable weighted $L^2$ inner product. We will see later that this implies the orthogonality of the characters $\chi_V$ for irreps $V$.

We can give $\bigoplus_{V\text{ irrep}}V\otimes V^\ast$ a commutative ring structure that is not the noncommutative ring structure coming from function composition in $\bigoplus_{V\text{ irrep}}\End(V)$. This is done by pulling back the commutative ring structure on $\Fun(G)$. For $v\otimes w, u\otimes z\in V\otimes V^\ast$, since we defined the matrix elements map $f_{-,-}$ on simple tensors and extended by linearity, the addition of $v\otimes w, u\otimes z$ in $\bigoplus_{V\text{ irrep}}V\otimes V^\ast$ is just the usual addition in the summand $V\otimes V^\ast$.

Let $V,W$ be irreps of $G$. For $v\otimes w\in V\otimes V^\ast$ and $v'\otimes w'\in W\otimes W^\ast$, the addition of matrix elements is given pointwise by
\[(f_{v,w}+f_{v',w'})(g) = f_{v,w}(g)+f_{v',w'}(g) = \abr{w,gv}+\abr{w',gv'} = \abr{(w,w'),g(v,v')} = f_{(v,v'),(w,w')}(g)\]
and the multiplication of matrix elements is also given pointwise by
\[(f_{v,w}\cdot f_{v',w'})(g) = f_{v,w}(g)\cdot f_{v',w'}(g) = \abr{w,gv}\abr{w',gv'} = \abr{w\otimes w',g(v\otimes v')} = f_{v\otimes v',w\otimes w'}(g)\]
This is to say that the ring structure on $\bigoplus_{V\text{ irrep}}V\otimes V^\ast$ is given by direct sum and tensor product, where $(v\otimes w)+ (v'\otimes w') = (v,v')\otimes (w,w')$ and $(v\otimes w) \cdot (v'\otimes w') = (v\otimes v')\otimes (w\otimes w')$. Here we are identifying $(V\otimes W)\otimes (V\otimes W)^\ast$ with $(V\otimes V^\ast)\otimes (W\otimes W^\ast)$. Generally we should not expect either of $(V\oplus W)\otimes (V\oplus W)^\ast$ or $(V\otimes W)\otimes (V\otimes W)^\ast$ to be irreducible, but since $G$ is a compact Lie group, these will decompose into direct sums of irreps which do appear in $\bigoplus_{V\text{ irrep}}V\otimes V^\ast$. We don't need to break representations into direct sums of irreps to define the matrix elements map, but for the purposes of defining a ring structure on $\bigoplus_{V\text{ irrep}}V\otimes V^\ast$ this is worth observing. Call $\bigoplus_{V\text{ irrep}}V\otimes V^\ast$ the ring of matrix elements of $G$. By defining the ring of matrix elements this way, the matrix elements map $f_{-,-}$ is a ring homomorphism.

The ring structure on $\bigoplus_{V\text{ irrep}}V\otimes V^\ast\cong \bigoplus_{V\text{ irrep}}\End(V)$ coming from function composition is also relevant. The function space $C(G)$ with the convolution product is a non-commutative ring (for $G$ non-Abelian). Define the action map $\mathrm{Act}\colon C(G)\to \bigoplus_{V\text{ irrep}}\End(V)$ by 
\[h\mapsto \int_G h(g)g(-)\dd g\]
(this is the Bochner integral using the normalized Haar measure on $G$ we saw before in an earlier discussion about the spectral theorem on $\U(1)$). So in some sense the function space $C(G)$ is acting as a version of the group algebra, but is not quite there since we are missing notable elements like the delta distribution, which is the unit of the convolution product. If $h$ is the constant function $1$, then $\mathrm{Act}(h)$ is the averaging operator $\av\in\bigoplus_{V\text{ irrep}}\End(V)$, which defines a projector to the subspace $\bigoplus_{V\text{ irrep}}V^G$

The inner product of the matrix element $f_{v,v'} = \abr{v',(-)v}$ with a function $h$ is 
\[\abr{f_{v,v'},h} = \int_G\abr{v',gv} \overline {h(g)}\dd g = \Bigl\langle v',\int_G \overline{h(g)} gv\Bigr\rangle = \abr{v',\mathrm{Act}(\overline h)(v)}\]
It follows that $f_{v,-}$ and $\mathrm{Act}(\overline{-})(v)$ are adjoint to each other in some sense. Compare this with the earlier remark about how the matrix elements map $f_{-,-}$ is a kind of adjoint (pullback) of $\rho$ for a representation $(V,\rho)$ of $G$.

Let $V$ be a representation of $G$ which we do not assume is finite-dimensional. Then denote by $V^\fin$ the subspace of $V$ containing vectors which belong to some finite-dimensional subrepresentation of $V$ (we are not fixing a finite-dimensional subrepresentation of $V$ for this definition, we just insist that the vectors live in a finite-dimensional subrepresentation of $V$). Using this definition, we observe the following: The ring of matrix elements is equal to $C(G)^\fin$ as $G$-representations, where the action on \sai{more...} One containment is clear. Assume that $f$ is in $C(G)^\fin$, that is, $f\in W$ for some finite-dimensional subrepresentation $W$ of $C(G)$. Normalize $f$ so that we can form an orthonormal basis $\{f_i\}$ of $W$ where $f_1 = f$. Let $\{f_i^\ast\}$ be the corresponding dual basis for $W^\ast$. Then the action of $g$ on each $f_i$ is given by
\[gf_i = \sum_j\abr{f_j^\ast,gf_i}f_j\]
From this equation and $f_1=f$ we obtain 
\[f(g) = (g^{-1}f_1)(1) = \sum_j\abr{f_j^\ast,g^{-1}f_1}f_j(1) = \sum_j\abr{gf_j^\ast,f_1}f_j(1) = \sum_jf_j(1)f_{f_1,gf_j^\ast}\]

For $G= \U(1)$, $C(G)^\fin$ is the space of functions given by finite Fourier series.
\end{document}