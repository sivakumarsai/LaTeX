\documentclass[../../rtnotes.tex]{subfiles}
\begin{document}
\section{09/25}
\subsection{The Peter-Weyl theorem}
Let $G$ be $\SU(2)$ or in general a compact Lie group and let $V$ be a finite-dimensional (continuous, we have been and will continue to assume this) representation of $G$. Since $V$ is a continuous representation of $G$, the matrix elements map $f_{-,-}$ defined last time has image lying in $C(G)$; that is, the matrix elements $f_{v,w}$ for $v\in V$ and $w\in V^\ast$ are continuous functions.

Last time we also saw that the matrix elements map $f_{-,-}$ was a $G\times G$-intertwining map by carefully defining the actions of each factor of $G$ on $V\otimes V^\ast$ and $\Fun(G)$. Furthermore, we saw that if $V$ was an irrep, then $V\otimes V^\ast$ was a $G\times G$-irrep so the matrix elements map $f_{-,-}$ was injective. And if $V,W$ were irreps, then the matrix elements map on $(V\oplus W)\otimes (V\oplus W)^\ast$ decomposed into the injective map $f_{-,-}\oplus f_{-,-}$ on $(V\otimes V^\ast)\oplus (W\otimes W^\ast)$, which implies the matrix elements map on $(V\oplus W)\otimes (V\oplus W)^\ast$ was injective to begin with. Then we defined the matrix elements map on $\bigoplus_{V\text{ irrep}}V\otimes V^\ast$, which maps injectively into $C(G)$. 

But the expression $\bigoplus_{V\text{ irrep}}V\otimes V^\ast$ maybe isn't so clear, since we would like for any irrep $W$ of $G$ to find $W\otimes W^\ast$ in the direct sum $\bigoplus_{V\text{ irrep}}V\otimes V^\ast$. For each irrep $V$, fix a basis of $V$ and give $V^\ast$ the corresponding dual basis. Then by Schur's lemma there is only one $G\times G$-equivariant map of irreps $W\otimes W^\ast\to V\otimes V^\ast$ which sends $\id_W$ to $\id_V$. Furthermore, $\id_V$ and $\id_W$ both map to the same function under the matrix elements map, and in general the matrix elements map would agree on both $V\otimes V^\ast$ and $W\otimes W^\ast$. So it would not matter which isomorphism class of each irrep $V$ we take in the direct sum $\bigoplus_{V\text{ irrep}}V\otimes V^\ast$. Henceforth fix an enumeration of the irreps $V$ of $G$, and refer to these irreps only.

Each irrep $V$ is unitarizable, so fix a $G$-invariant inner product on each $V$. The (antilinear) Riesz map $V\xrightarrow{r} V^\ast$ given by $v \mapsto \abr{-,v}$ is $G$-equivariant:
\[gv\mapsto \abr{-,gv} = \abr{gg^{-1}(-),gv} = \abr{g^{-1}(-),v} = g\abr{-,v}\]
Therefore the inverse to the Riesz map is also $G$-equivariant, so we can give a natural $G$-invariant inner product on $V^\ast$:
\[\abr{-,-}_{V^\ast} = \abr{r^{-1}(-),r^{-1}(-)}_V\]
Later we will give $\bigoplus_{V\text{ irrep}}V\otimes V^\ast$ a $G\times G$-invariant inner product for which the elements $\id_V$ in $\bigoplus_{V\text{ irrep}}V\otimes V^\ast$ are orthogonal. This inner product can also be chosen so that the matrix elements map $f_{-,-}$ is a map of unitary representations where $C(G)$ is given a suitable weighted $L^2$ inner product. We will see later that this implies the orthogonality of the characters $\chi_V$ for irreps $V$.

We can give $\bigoplus_{V\text{ irrep}}V\otimes V^\ast$ a (commutative) ring structure that is not the (noncommutative) ring structure coming from function composition in $\bigoplus_{V\text{ irrep}}\End(V)$. This is done by pulling back the commutative ring structure on $\Fun(G)$. For $v\otimes w, u\otimes z\in V\otimes V^\ast$, since we defined the matrix elements map $f_{-,-}$ on simple tensors and extended by linearity, the addition of $v\otimes w, u\otimes z$ in $\bigoplus_{V\text{ irrep}}V\otimes V^\ast$ is just the usual addition in the summand $V\otimes V^\ast$.

Let $V,W$ be irreps of $G$. For $v\otimes v'\in V\otimes V^\ast$ and $w\otimes w'\in W\otimes W^\ast$, the addition of matrix elements is given pointwise by
\[(f_{v,v'}+f_{w,w'})(g) = f_{v,v'}(g)+f_{w,w'}(g) = \abr{v',gv}+\abr{w',gw} = \abr{(v',w'),g(v,w)} = f_{(v,w),(v',w')}(g)\]
and the multiplication of matrix elements is also given pointwise by
\[(f_{v,v'}\cdot f_{w,w'})(g) = f_{v,v'}(g)\cdot f_{w,w'}(g) = \abr{v',gv}\abr{w',gw} = \abr{v'\otimes w',g(v\otimes w)} = f_{v\otimes w,v'\otimes w'}(g)\]
This shows that the ring structure on $\bigoplus_{V\text{ irrep}}V\otimes V^\ast$ is given by basically taking the direct sum and tensor product, where $(v\otimes v')+ (w\otimes w') = (v,w)\otimes (v',w')$ and $(v\otimes v') \cdot (w\otimes w') = (v\otimes w)\otimes (v'\otimes w')$. But the elements $(v,w)\otimes (v',w')$ and $(v\otimes w)\otimes (v'\otimes w')$ are not necessarily elements of $\bigoplus_{V\text{ irrep}}V\otimes V^\ast$. So far, these addition and multiplication formulas have been formulated so that the matrix elements map yields the same function when applied to both sides of each formula (we don't need to break representations into direct sums of irreps to define the matrix elements map, but for the purposes of defining a ring structure on $\bigoplus_{V\text{ irrep}}V\otimes V^\ast$ this is worth observing). To define the ring operator properly in $\bigoplus_{V\text{ irrep}}V\otimes V^\ast$, the addition and multiplication formulas should return elements of that space.

We start with the addition of simple tensors. We had $(v\otimes v')+ (w\otimes w') = (v,w)\otimes (v',w')\in (V\oplus W)\otimes (V\oplus W)^\ast$. Decompose $(V\oplus W)\otimes (V\oplus W)^\ast$ into $(V\otimes V^\ast) \oplus (W\otimes V^\ast) \oplus(V\otimes W^\ast)\oplus(W\otimes W^\ast)$, on which we saw earlier that the matrix elements map was zero on the middle two summands, regardless of whether $V$ was distinct from $W$ or not. This decomposition is just $(v,w)\otimes (v',w') = (v,0)\otimes (v',0) + (0,w)\otimes (v',0) + (v,0)\otimes (0,w') + (0,w)\otimes (0,w')$. Because the matrix elements map is trivial on the middle two terms, we can ignore (delete) these terms. The first and fourth terms can be combined if $V = W$ (remember we have fixed an enumeration of every irrep of $G$, so equals means equals). If $V = W$, then send $(v,0)\otimes (v',0) + (0,w)\otimes (0,w')$ to $v\otimes v' + w\otimes w'\in V\otimes V^\ast$. Otherwise, view the first and fourth terms in $\bigoplus_{V\text{ irrep}}V\otimes V^\ast$ as $(v\otimes v') + (w\otimes w')$ (suppressing the notation for elements in a direct sum). To summarize, if $V$ and $W$ are distinct, we have defined a map $(V\oplus W)\otimes (V\oplus W)^\ast\to (V\otimes V^\ast)\oplus(W\otimes W^\ast)\hookrightarrow \bigoplus_{V\text{ irrep}}V\otimes V^\ast$. In the case that $V$ and $W$ coincide, we have defined a similar map $(V\oplus W)\otimes (V\oplus W)^\ast\to (V\otimes V^\ast)\hookrightarrow \bigoplus_{V\text{ irrep}}V\otimes V^\ast$. Denote both maps by $\underline{\mathrm{add}}$; that is, $\underline{\mathrm{add}}((v,w)\otimes (v',w')) = v\otimes v' + w\otimes w'$ where depending on whether $V$ is distinct from $W$ or not, we view $ v\otimes v' + w\otimes w'$ in $(V\otimes V^\ast)\oplus(W\otimes W^\ast)$ or $V\otimes V^\ast$. Then the following diagram commutes:
% https://q.uiver.app/#q=WzAsMyxbMCwwLCJcXGJpZ29wbHVzX3tWXFx0ZXh0eyBpcnJlcH19Vlxcb3RpbWVzIFZeXFxhc3QiXSxbMSwwLCJcXEZ1bihHKSJdLFswLDEsIihWXFxvcGx1cyBXKVxcb3RpbWVzIChWXFxvcGx1cyBXKV5cXGFzdCJdLFswLDEsImZfey0sLX0iXSxbMiwxLCJmX3stLC19IiwyXSxbMiwwLCJcXHVuZGVybGluZXtcXG1hdGhybXthZGR9fSJdXQ==
\[\begin{tikzcd}[ampersand replacement=\&]
	{\bigoplus_{V\text{ irrep}}V\otimes V^\ast} \& {\Fun(G)} \\
	{(V\oplus W)\otimes (V\oplus W)^\ast}
	\arrow["{f_{-,-}}", from=1-1, to=1-2]
	\arrow["{\underline{\mathrm{add}}}", from=2-1, to=1-1]
	\arrow["{f_{-,-}}"', from=2-1, to=1-2]
\end{tikzcd}\]
So the correct addition law in $\bigoplus_{V\text{ irrep}}V\otimes V^\ast$ is $(v\otimes v')+ (w\otimes w') = \underline{\mathrm{add}}((v,w)\otimes (v',w')) = v\otimes v' + w\otimes w'$, where the notation on the right hand side is confusingly not the same as the notation on the left hand side.

We do a similar thing with multiplication. On simple tensors, we had $(v\otimes v') \cdot (w\otimes w') = (v\otimes w)\otimes (v'\otimes w')\in (V\otimes W)\otimes (V\otimes W)^\ast$. Decompose $(V\otimes W)\otimes (V\otimes W)^\ast$ into 
\[\biggl(\bigoplus_{i}U_i\biggr)\otimes\biggl(\bigoplus_{j}U_j^\ast\biggr)\]
where $U_i$ and $U_j$ are irreps of $G$; more importantly, the irreps appearing the direct sums may repeat. We define a similar map to $\underline{\mathrm{add}}$ which coalesces the elements in repeated summands $U_i$ by addition, which yields an element of $\bigoplus_{V\text{ irrep}}V\otimes V^\ast$. The $\underline{\mathrm{coalesce}}$ map is defined in the same way as $\underline{\mathrm{add}}$ above, and we can even just use this map in place of $\underline{\mathrm{add}}$ above. We obtain the commutative diagram
% https://q.uiver.app/#q=WzAsMyxbMCwwLCJcXGJpZ29wbHVzX3tWXFx0ZXh0eyBpcnJlcH19Vlxcb3RpbWVzIFZeXFxhc3QiXSxbMSwwLCJcXEZ1bihHKSJdLFswLDEsIlxcYmlnZ2woXFxiaWdvcGx1c197aX1VX2lcXGJpZ2dyKVxcb3RpbWVzXFxiaWdnbChcXGJpZ29wbHVzX3tqfVVfal5cXGFzdFxcYmlnZ3IpIl0sWzAsMSwiZl97LSwtfSJdLFsyLDEsImZfey0sLX0iLDJdLFsyLDAsIlxcdW5kZXJsaW5le1xcbWF0aHJte2NvYWxlc2NlfX0iXV0=
\[\begin{tikzcd}[ampersand replacement=\&]
	{\bigoplus_{V\text{ irrep}}V\otimes V^\ast} \& {\Fun(G)} \\
	{\biggl(\bigoplus_{i}U_i\biggr)\otimes\biggl(\bigoplus_{j}U_j^\ast\biggr)}
	\arrow["{f_{-,-}}", from=1-1, to=1-2]
	\arrow["{\underline{\mathrm{coalesce}}}", from=2-1, to=1-1]
	\arrow["{f_{-,-}}"', from=2-1, to=1-2]
\end{tikzcd}\]
It follows that there is a law for multiplication on $\bigoplus_{V\text{ irrep}}V\otimes V^\ast$, but the explicit form of this law is unclear since we would need to know how the tensor product of irreps decompose as a direct sum of irreps to carry out the multiplication. In practice this entire technicality should not matter and we would not need to insist on obtaining elements of $\bigoplus_{V\text{ irrep}}V\otimes V^\ast$ after addition or multiplication.

Call $\bigoplus_{V\text{ irrep}}V\otimes V^\ast$ the ring of matrix elements of $G$. By defining the ring of matrix elements this way, the matrix elements map $f_{-,-}$ is a ring homomorphism.

The ring structure on $\bigoplus_{V\text{ irrep}}V\otimes V^\ast\cong \bigoplus_{V\text{ irrep}}\End(V)$ coming from function composition is also relevant. The function space $C(G)$ with the convolution product is a non-commutative ring (for $G$ non-Abelian). Define the action map $\mathrm{Act}\colon C(G)\to \bigoplus_{V\text{ irrep}}\End(V)$ by 
\[h\mapsto \int_G h(g)g(-)\dd g\]
(this is the Bochner integral using the normalized Haar measure on $G$ we saw before in an earlier discussion about the spectral theorem on $\U(1)$). So in some sense the function space $C(G)$ is acting as a version of the group algebra, but is not quite there since we are missing notable elements like the delta distribution, which is the unit of the convolution product. Observe that $\mathrm{Act}(\mathbf 1_G)$ is the averaging operator $\av\in\bigoplus_{V\text{ irrep}}\End(V)$, which defines a projector to the direct sum of trivial representations.

The inner product of the matrix element $f_{v,v'} = \abr{v',(-)v}$ with a function $h$ is 
\[\abr{f_{v,v'},h} = \int_G\abr{v',gv} \overline {h(g)}\dd g = \Bigl\langle v',\int_G \overline{h(g)} gv\Bigr\rangle = \abr{v',\mathrm{Act}(\overline h)(v)}\]
It follows that $f_{v,-}$ and $\mathrm{Act}(\overline{-})(v)$ are adjoint to each other in some sense. Compare this with the earlier remark about how the matrix elements map $f_{-,-}$ is a kind of adjoint (pullback) of $\rho$ for a representation $(V,\rho)$ of $G$. 

Let $V$ be a representation of $G$ which we do not assume is finite-dimensional. Then denote by $V^\fin$ the subspace of $V$ containing vectors which belong to some finite-dimensional subrepresentation of $V$ (we are not fixing a finite-dimensional subrepresentation of $V$ for this definition, we just insist that the vectors live in a finite-dimensional subrepresentation of $V$). Using this definition, we observe the following: The ring of matrix elements is isomorphic to $C(G)^\fin$ as $G$-representations (or rather the image of the ring of matrix elements under the matrix elements map is equal to $C(G)^\fin$), where the action of $G$ on matrix elements is given by $h(v\otimes v') = hv\otimes v'$ and the action of $G$ on $\Fun(G)^\fin$ is $hf = f((-)h)$. The matrix elements map is injective, so it suffices to see that the image of the matrix elements map is $C(G)^\fin$. Assume that $f$ is in $C(G)^\fin$, that is, $f\in W$ for some finite-dimensional subrepresentation $W$ of $C(G)$. Form a basis $\{f_i\}$ of $W$ where $f_1 = f$. Let $\{f_i^\ast\}$ be the corresponding dual basis for $W^\ast$. Then the action of $h$ on each $f_i$ is given by
\[hf_i = \sum_j\abr{f_j^\ast,hf_i}f_j\]
From this equation and $f_1=f$ we obtain 
\[f(h) = (hf_1)(1) = \sum_j\abr{f_j^\ast,hf_1}f_j(1) = \sum_jf_j(1)f_{f_1,f_j^\ast}(h)\]
It follows that any function in $C(G)^\fin$ can be obtained by a linear combination of matrix elements, which proves the proposition. As an example, if $G= \U(1)$, then $C(G)^\fin$ is the space of functions given by finite Fourier series.

Since $C(G)$ is an inner product space with the usual $L^2$ inner product, we may pull back this inner product to obtain some inner product on $\bigoplus_{V\text{ irrep}}V\otimes V^\ast$, which is a direct sum of inner products on each irrep $V\otimes V^\ast$, because a $G$-invariant inner product on $(V\otimes V^\ast)\oplus (W\otimes W^\ast)$ for distinct irreps $V,W$ of $G$ is an antilinear $G$-equivariant isomorphism of $(V\otimes V^\ast)\oplus (W\otimes W^\ast)$ with its dual. But by Schur's lemma such an isomorphism is really a direct sum of antilinear $G$-equivariant isomorphisms of $(V\otimes V^\ast)$ with its dual and $(W\otimes W^\ast)$ with its dual. The inner product on $V\otimes V^\ast$ is given on simple tensors by
\[\abr{v\otimes v',w\otimes w'}_{V\otimes V^\ast} = C_V\abr{v,w}_V\abr{v',w'}_{V^\ast}\]
where $C_V$ is a constant depending on $V$.
Let $\{e_i\}$ be a basis of $V$ and give $V^\ast$ the corresponding dual basis. Then to calculate $C_V$, calculate the $L^2$ inner product of the matrix elements $f_{e_i,e_j^\ast}$ and $f_{e_s,e_t^\ast}$. The inner product on $V\otimes V^\ast$ is obtained by pulling back this inner product, so
\[C_V\abr{e_i,e_s}_V\abr{e_j^\ast,e_t^\ast}_{V^\ast} = \abr{f_{e_i,e_j^\ast},f_{e_s,e_t^\ast}}_{L^2} = \int_G f_{e_i,e_j^\ast}(g)\overline{f_{e_s,e_t^\ast}(g)}\dd g = \int_G M_{ij}(g)\overline{M_{st}(g)}\dd g\]
where $M_{kl}(g)$ is the $(k,l)$-entry of the unitary matrix representing the action of $g$ on $V$. Since $g$ acts by a unitary matrix, $\overline{M_{st}(g)} = M_{ts}(g^{-1})$. Observe further that $\abr{e_i,e_s} = \delta_{is}$ and $\abr{e_j^\ast,e_t^\ast}_{V^\ast} = \abr{e_j,e_t}_V = \delta_{jt}$. Then 
\[C_V\delta_{is}\delta_{jt} = \int_G M_{ts}(g^{-1})M_{ij}(g)\dd g,\]
and by setting $i=s$ and summing over $i$, obtain 
\[C_V\delta_{jt}\dim V = \int_G\sum_{i=1}^{\dim V} M_{ti}(g^{-1})M_{ij}(g)\dd g = \int_G M_{tj}(g)\dd g = \delta_{tj}\]
Hence $C_V = 1/\dim V$. So the inner product on $\bigoplus_{V\text{ irrep}}V\otimes V^\ast$ is given by the direct sum of the inner products$(1/\dim V)\abr{-,-}_V\abr{-,-}_{V^\ast}$.

This immediately implies the orthonormality of the characters $\chi_V$ for irreps $V$, since they are obtained as the matrix elements of the identity maps $\id_V = \sum_i e_i\otimes e_i^\ast$ in each $V\oplus V^\ast$. Therefore any finite dimensional representation of $G$ is determined up to isomorphism by its character. We observed earlier that the characters $\chi_V$ actually belong to $C(G)^G$, the space of class functions (here $C(G)$ is a $G$-representation by the diagonal embedding $G\to G\times G$ which results in the conjugation action, and we may further observe that the characters belong to $(C(G)^\fin)^G$). It is a theorem that $(C(G)^\fin)^G$ is $L^2$-dense in $C(G)^G$ and hence also $L^2(G)^G$. Observe that the characters from an orthonormal basis of $C(G)^G$ (or of $L^2(G)^G$), since the $G$-invariant (conjugation-invariant) subspace of $\bigoplus_{V\text{ irrep}}V\otimes V^\ast$ is $\bigoplus_{V\text{ irrep}}\mathbb C\id_V$. In general, analysis shows that $C(G)^\fin$ is $L^2$-dense in $C(G)$ and hence also $L^2(G)$.

If $U,P$ are any representations of $G$, then by decomposing $U$ and $P$ into irreps (which may repeat), observe that $\abr{\chi_U,\chi_P}_{L^2} = \dim\Hom_G(U,P)$. The dimension of $\Hom_G(U,P)$ simply counts the number of ways irreps in $U$ map to irreps in $P$, which is also what the inner product of the characters is counting, by the orthonormality of characters of irreps.

As a final aside, it turns out that $\mathrm{Act}$ defines an isomorphism of algebras $C(G)^\fin$ with convolution and $\bigoplus_{V\text{ irrep}}V\otimes V^\ast\cong \bigoplus_{V\text{ irrep}}\End(V)$ with composition.
\subsection{Return to $\SU(2)$}

\end{document}