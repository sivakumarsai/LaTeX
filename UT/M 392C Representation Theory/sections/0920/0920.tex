\documentclass[../../rtnotes.tex]{subfiles}
\begin{document}
\section{09/20}
\subsection{Prologue to finding representations of $\SU_2$}
There are several ways to study representations of $\SU_2$, but the approach we will take is to start by first using the representation theory of $\U(1)$ to extract interesting information about representations of $\SU_2$ as opposed to constructing them explicitly via a robust structure theory or some other approach.

Let $V$ be a finite dimensional complex (continuous) representation of $\SU_2$. Since $\SU_2$ is not Abelian, we don't immediately get to use the nice theory of LCA groups we discussed earlier. So instead we look at the action of a $\U(1)\subset \SU_2$ on $V$; that is, we think of $V$ first as a $\U(1)$-representation. There are lots of different copies of $\U(1)$ inside of $\SU_2$, but for now think of the copy given by $\{\cos(\theta) + \sin(\theta)i\}\subset \SU_2$ (the unit quaternions where the coefficients of $j,k$ are zero).

From the earlier $\U(1)$-theory, $V$ decomposes into  $V = \bigoplus_{n\in\mathbb Z}V_n$
\end{document}