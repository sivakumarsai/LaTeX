\documentclass[11pt,leqno]{article}
\headheight=13.6pt

% packages
\usepackage[alphabetic]{amsrefs}
\usepackage{physics}
% margin spacing
\usepackage[top=1in, bottom=1in, left=0.5in, right=0.5in]{geometry}
\usepackage{amsfonts, amsmath, amssymb, amsthm}
\usepackage{extarrows}
\usepackage[none]{hyphenat}
\usepackage{fancyhdr}
\usepackage{graphicx}
\graphicspath{{./images/}}
\usepackage{float}
\usepackage{imakeidx}
    % \AtBeginDocument{\renewcommand\indexname{Index}}
    \makeindex
\indexsetup{level=\section*,toclevel=section,noclearpage,firstpagestyle=fancy}
\usepackage{color}
\newcommand{\sai}[1]{\textcolor{red}{#1}}
\usepackage{enumitem}
\usepackage{mathrsfs}
\usepackage{hyperref}
\usepackage[noabbrev, capitalise]{cleveref}
\crefformat{equation}{equation~#2#1#3}
\crefformat{lemma}{\textrm{Lemma}~#2#1#3}
\usepackage{quiver}
\usepackage{titlesec}
% \titleformat{\section}
%   {\bfseries}{Lecture \thesection}{1em}{}

% theorems
\theoremstyle{plain}
\newtheorem{lem}{Lemma}
\newtheorem{lemma}[lem]{Lemma}
\newtheorem{thm}[lem]{Theorem}
\newtheorem{theorem}[lem]{Theorem}
\newtheorem{prop}[lem]{Proposition}
\newtheorem{proposition}[lem]{Proposition}
\newtheorem{cor}[lem]{Corollary}
\newtheorem{corollary}[lem]{Corollary}
\newtheorem{conj}[lem]{Conjecture}
\newtheorem{fact}[lem]{Fact}
\newtheorem{form}[lem]{Formula}

\theoremstyle{definition}
\newtheorem{defn}[lem]{Definition}
\newtheorem{definition/}[lem]{Definition}
\newenvironment{definition}
  {\renewcommand{\qedsymbol}{\textdagger}%
   \pushQED{\qed}\begin{definition/}}
  {\popQED\end{definition/}}
\newtheorem{example}[lem]{Example}
\newtheorem{remark}[lem]{Remark}
\newtheorem{exercise}[lem]{Exercise}
\newtheorem{notation}[lem]{Notation}

\numberwithin{equation}{section}
\numberwithin{lem}{section}

% header/footer formatting
\pagestyle{fancy}
\fancyhead{}
\fancyfoot{}
\fancyhead[L]{M 396D}
\fancyhead[C]{}
\fancyhead[R]{Homological Algebra}
\fancyfoot[R]{\thepage}
\renewcommand{\headrulewidth}{1pt}

% paragraph indentation/spacing
\setlength{\parindent}{0cm}
\setlength{\parskip}{10pt}
\renewcommand{\baselinestretch}{1.25}

% extra commands defined here
\newcommand{\br}[1]{\left(#1\right)}
\newcommand{\sbr}[1]{\left[#1\right]}
\newcommand{\cbr}[1]{\left\{#1\right\}}
\newcommand{\eq}[1]{\overset{(#1)}{=}}
\newcommand{\idx}[1]{#1\index{#1}} % adds to index
\newcommand{\bidx}[1]{\textbf{#1\index{#1}}} % adds to index

% bracket notation for inner product
\usepackage{mathtools}

\DeclarePairedDelimiterX{\abr}[1]{\langle}{\rangle}{#1}

\DeclareMathOperator{\im}{im}
\DeclareMathOperator{\rad}{rad}
\DeclareMathOperator{\Hom}{Hom}
\DeclareMathOperator{\id}{id}

% categories
\newcommand{\catname}[1]{{\normalfont\mathbf{#1}}}
\newcommand{\Set}{\catname{Set}}
\newcommand{\Ring}{\catname{Ring}}
\newcommand{\Top}{\catname{Top}}
\newcommand{\Rmod}{R\text{-}\catname{mod}}
\newcommand{\modR}{\catname{mod}\text{-}R}
\newcommand{\Ch}{\catname{Ch}}


% smileys frownies
\usepackage{wasysym}
\newcommand{\smallhappy}{\raisebox{-.14em}{\smiley}}
\newcommand{\happy}{\raisebox{-.24em}{\resizebox{1.2em}{!}{\smiley}}}
\newcommand{\smallsad}{\raisebox{-.14em}{\frownie}}
\newcommand{\sad}{\raisebox{-.24em}{\resizebox{1.2em}{!}{\frownie}}}
\DeclareMathOperator{\mathhappy}{\!\happy\!}
\DeclareMathOperator{\smallmathhappy}{\!\smallhappy\!}
\DeclareMathOperator{\mathsad}{\!\sad\!}
\DeclareMathOperator{\smallmathsad}{\!\smallsad\!}

% set page count index to begin from 1
\setcounter{page}{1}
\setcounter{section}{-1}

\begin{document}
M 396D Conference (Reading) Course, Spring 2025: Sai Sivakumar's notes. Text references: \textit{An Introduction to Homological Algebra} by Weibel, and \sai{other}.
\tableofcontents\newpage

\sai{italicize nice words/phrases throughout tastefully}
\sai{fix enumerates? textups too in enumerates}
\sai{eventually cite all texts including achar.../fix all citations}
\section{Introduction}
\sai{to be done}
\section{The rudiments}\sai{quotes in each section?} In these notes we only insist that rings are associative unless indicated otherwise. The categories of rings and right $R$-modules for a ring $R$ are denoted $\Ring$ and $\modR$, respectively. We use right $R$-modules so that homomorphisms are written on the left; e.g., homomorphisms $R^n\to R^n$ may be given by left multiplication by an appropriate $m\times n$ $R$-valued matrix.

In this section we explore basic notions in homological algebra, from chain complexes of modules over commutative rings to Abelian categories. Throughout we collect many results and examples related to these topics, many of which are from algebraic topology.

\subsection{Chain complexes of \textit{R}-modules}

Let $R$ be a ring and consider the sequence of $R$-modules $A\xrightarrow{f}B\xrightarrow{g}C$. This sequence is \bidx{exact} at $B$ if $\ker g = \im f$. This is a stronger condition than requiring $gf = 0$, in which we have $\im f\subseteq \ker g$. A sequence of $R$-modules is exact if it is exact at each module (that is not the first or the last, if applicable) in the sequence. A short exact sequence is an exact sequence of the form $0\to A\xrightarrow{f}B\xrightarrow{g}C\to 0$.

\begin{definition}
    A \bidx{chain complex} $C_\bullet$ of $R$-modules is a family $\cbr{C_n}_{n\in\mathbb Z}$ of $R$-modules with morphisms $d_n\colon C_n\to C_{n-1}$ such that for all $n$, the composites $d_n d_{n+1}$ equal zero. 

    The maps $d_n$ are called the \bidx{differentials} of $C_\bullet$. The module of \bidx{$n$-cycles} of $C_\bullet$ is $Z_n(C_\bullet) = \ker d_n$ and the module of \bidx{$n$-boundaries} of $C_\bullet$ is $B_n(C_\bullet) = \im d_{n+1}$.
\end{definition}
We will suppress the $(C_\bullet)$ in e.g., $Z_n(C_\bullet)$ when there will be no confusion. The equations $d_n d_{n+1} = 0$ yield the containments $B_n\subseteq Z_n\subseteq C_n$ for all $n$.
\begin{definition}
    The \textbf{$n$-th \idx{homology} module} of $C_\bullet$ is the subquotient $H_n(C_\bullet) = Z_n/B_n = \ker d_n/\im d_{n+1}$.
\end{definition}

A morphism of chain complexes $u\colon C_\bullet\to D_\bullet$ is a collection of $R$-module homomorphisms $\cbr{u_n\colon C_n\to D_n}_{n\in\mathbb Z}$ such that the following diagram commutes
% https://q.uiver.app/#q=WzAsMTAsWzEsMCwiQ197bisxfSJdLFsyLDAsIkNfbiJdLFszLDAsIkNfe24tMX0iXSxbMSwxLCJEX3tuKzF9Il0sWzIsMSwiRF9uIl0sWzMsMSwiRF97bi0xfSJdLFswLDAsIlxcY2RvdHMiXSxbMCwxLCJcXGNkb3RzIl0sWzQsMSwiXFxjZG90cyJdLFs0LDAsIlxcY2RvdHMiXSxbNiwwLCJkX3tuKzJ9Il0sWzcsMywiZF97bisyfSJdLFswLDEsImRfe24rMX0iXSxbMyw0LCJkX3tuKzF9Il0sWzEsMiwiZF97bn0iXSxbNCw1LCJkX3tufSJdLFsyLDksImRfe24tMX0iXSxbNSw4LCJkX3tuLTF9Il0sWzAsMywidV97bisxfSJdLFsxLDQsInVfbiJdLFsyLDUsInVfe24tMX0iXV0=
\[\begin{tikzcd}[ampersand replacement=\&]
	\cdots \& {C_{n+1}} \& {C_n} \& {C_{n-1}} \& \cdots \\
	\cdots \& {D_{n+1}} \& {D_n} \& {D_{n-1}} \& \cdots
	\arrow["{d_{n+2}}", from=1-1, to=1-2]
	\arrow["{d_{n+1}}", from=1-2, to=1-3]
	\arrow["{u_{n+1}}", from=1-2, to=2-2]
	\arrow["{d_{n}}", from=1-3, to=1-4]
	\arrow["{u_n}", from=1-3, to=2-3]
	\arrow["{d_{n-1}}", from=1-4, to=1-5]
	\arrow["{u_{n-1}}", from=1-4, to=2-4]
	\arrow["{d_{n+2}}", from=2-1, to=2-2]
	\arrow["{d_{n+1}}", from=2-2, to=2-3]
	\arrow["{d_{n}}", from=2-3, to=2-4]
	\arrow["{d_{n-1}}", from=2-4, to=2-5]
\end{tikzcd}\] 
Therefore there is a category \idx{$\Ch(\modR)$} of chain complexes of $R$-modules with morphisms as defined above.

Morphisms of chain complexes send boundaries to boundaries and cycles to cycles: If $x\in B_n(C_\bullet)$, $x = d_{n+1}y$ for some $y\in C_{n+1}$. Then $u_nx = u_nd_{n+1}y = d_{n+1}(u_{n+1}y)\in B_n(D_\bullet)$. Similarly, if $x\in Z_n(C_\bullet)$, $d_nx = 0$ so that $0 = u_{n-1}d_nx = d_n(u_nx)$ as needed. Therefore there are well-defined induced maps of homology groups. In fact, taking homology defines functors $H_n\colon \Ch(\modR)\to \modR$.

\begin{example}
    Every short exact sequence of vector spaces is split exact. We will explore split-exactness in \sai{later?}, but recall that this implies that for any short exact sequence of vector spaces $0\to U\xrightarrow{j}V\xrightarrow{p}W\to 0$, there exists a section $s\colon W\to V$ such that $ps = \id_W$ and a retract $r\colon V\to U$ such that $rj = \id_U$.

    Let $V_\bullet = \cdots\to V_{n+1}\to V_n\to V_{n-1}\to\cdots$ be a chain complex of vector spaces. For each $n$, obtain the short exact sequences (of vector spaces) $0\to Z_n\to V_n\to B_{n-1}\to 0$ and $0\to B_n\to Z_n\to H_n\to 0$. Therefore there are retracts $r_n\colon V_n\to Z_n$ and $t_n\colon Z_n\to B_n$ such that $V_n$ is isomorphic to $C_n = B_n\oplus H_n\oplus B_{n-1}$ under the map $f_n\colon v\mapsto (t_nr_nv, \overline{r_nv}, d_nv)$. With $d_n\colon C_n\to C_{n-1}$ by $(a,b,c)\mapsto (c,0,0)$, we obtain a chain complex $C_\bullet$ out of the $C_n$. Observe that $d_nf_nv = (d_nv, 0, 0)$ and $f_{n-1}(d_nv) = (t_{n-1}r_{n-1}d_nv, \overline{r_{n-1}d_nv}, d_{n-1}d_nv) = (d_nv, 0, 0)$.
    
    Therefore $V_\bullet$ is isomorphic to $C_\bullet$.
\end{example}

\begin{definition}
    A morphism $C_\bullet\to D_\bullet$ of chain complexes is called a \bidx{quasi-isomorphism} if the induced maps  $H_n(C_\bullet)\to H_n(D_\bullet)$ are all isomorphisms.
\end{definition}

If a chain complex $C_\bullet$ is exact, then it follows that every homology group $H_n(C_\bullet)$ is trivial. In this case, the map of chain complexes $0\to C_\bullet$ (where $0\in \Ch(\modR)$ is the complex of all zero modules and zero maps) is a quasi-isomorphism. Lastly, these conditions are all equivalent since if $H_n(C_\bullet)\cong 0$, for every $n$, then $C_\bullet$ is exact.

\newpage\section{Long exact sequences}
\newpage\section{Mapping cones and cylinders}
\newpage\section{Abelian categories}

% \newpage\section{XX/XX}

\newpage
\printindex\newpage
\begin{bibdiv}
\begin{biblist}

\bib{w}{book}{
    title = {Algebraic Geometry I: Schemes},
    author = {G\"ortz, Ulrich},
    author = {Wedhorn, Torsten},
    isbn = {978-3-658-30732-5},
    series = {Springer Studium Mathematik - Master},
    year = {2020},
    publisher = {Springer Spektrum Wiesbaden}
}

\bib{gw}{book}{
    title = {An Introduction to Homological Algebra},
    author = {Weibel, Charles A.},
    isbn = {9781139644136},
    series = {Cambridge Studies in Advanced Mathematics},
    year = {1994},
    publisher = {Cambridge University Press}
}

\bib{h}{book}{
    title = {Algebraic Geometry},
    author = {Hartshorne, Robin},
    isbn = {978-0-387-90244-9},
    series = {Graduate Texts in Mathematics},
    year = {1977},
    publisher = {Springer New York, NY}
}

\end{biblist}
\end{bibdiv}
\end{document}