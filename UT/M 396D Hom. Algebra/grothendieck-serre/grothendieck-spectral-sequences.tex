\documentclass[11pt,leqno]{article}
\headheight=13.6pt

% packages
\usepackage[alphabetic]{amsrefs}
% margin spacing
\usepackage[top=1in, bottom=1in, left=0.5in, right=0.5in]{geometry}
\usepackage{amsfonts, amsmath, amssymb, amsthm}
\usepackage{extarrows}
\usepackage[none]{hyphenat}
\usepackage{fancyhdr}
\usepackage{graphicx}
\graphicspath{{./images/}}
\usepackage{color}
\newcommand{\sai}[1]{\textcolor{red}{#1}}
\usepackage{enumitem}
\usepackage{mathrsfs}
\usepackage{hyperref}
\usepackage[noabbrev, capitalise]{cleveref}
\crefformat{equation}{equation~#2#1#3}
\crefformat{lemma}{\textrm{Lemma}~#2#1#3}
\usepackage{quiver}

% theorems
\theoremstyle{plain}
\newtheorem{lem}{Lemma}
\newtheorem{lemma}[lem]{Lemma}
\newtheorem{thm}[lem]{Theorem}
\newtheorem{theorem}[lem]{Theorem}
\newtheorem{prop}[lem]{Proposition}
\newtheorem{proposition}[lem]{Proposition}
\newtheorem{cor}[lem]{Corollary}
\newtheorem{corollary}[lem]{Corollary}

\theoremstyle{definition}
\newtheorem{defn}[lem]{Definition}
\newtheorem{definition/}[lem]{Definition}
\newenvironment{definition}
  {\renewcommand{\qedsymbol}{\textdagger}%
   \pushQED{\qed}\begin{definition/}}
  {\popQED\end{definition/}}

\numberwithin{equation}{section}
\numberwithin{lem}{section}

% header/footer formatting
\pagestyle{fancy}
\fancyhead{}
\fancyfoot{}
\fancyhead[L]{Grothendieck spectral sequences}
\fancyhead[C]{}
\fancyhead[R]{\thepage}
\fancyfoot[R]{}
\renewcommand{\headrulewidth}{1pt}

% paragraph indentation/spacing
\setlength{\parindent}{0cm}
\setlength{\parskip}{10pt}
\renewcommand{\baselinestretch}{1.25}

% new commands
\newcommand{\eq}[1]{\overset{(#1)}{=}}
\DeclareMathOperator{\im}{im}
\DeclareMathOperator{\Tot}{Tot}

% smileys frownies
\usepackage{wasysym}
\newcommand{\smallhappy}{\raisebox{-.14em}{\smiley}}
\newcommand{\happy}{\raisebox{-.24em}{\resizebox{1.2em}{!}{\smiley}}}
\newcommand{\smallsad}{\raisebox{-.14em}{\frownie}}
\newcommand{\sad}{\raisebox{-.24em}{\resizebox{1.2em}{!}{\frownie}}}
\DeclareMathOperator{\mathhappy}{\!\happy\!}
\DeclareMathOperator{\smallmathhappy}{\!\smallhappy\!}
\DeclareMathOperator{\mathsad}{\!\sad\!}
\DeclareMathOperator{\smallmathsad}{\!\smallsad\!}

\begin{document}
These notes closely follow \textit{An Introduction to Homological Algebra} by Weibel.

In Abelian categories with enough injectives, there is a procedure for calculating right derived functors out of them by taking injective resolutions of objects, applying the left exact functor which is being derived, and taking cohomology. Let $G\colon \mathcal A\to\mathcal B$ and $F\colon \mathcal B\to\mathcal C$ be left exact functors of Abelian categories. We outline a procedure for calculating $R^i(FG)$ under mild conditions on the categories $\mathcal A$ and $\mathcal B$ and the functors $F$ and $G$. 

\begin{theorem}[Grothendieck spectral sequence]
  \label[theorem]{thm: gss}
  Let $\mathcal A$, $\mathcal B$, and $\mathcal C$ be Abelian categories such that $\mathcal A$ and $\mathcal B$ have enough injectives. Let $G\colon \mathcal A\to\mathcal B$ and $F\colon \mathcal B\to\mathcal C$ be left exact functors. 
  % https://q.uiver.app/#q=WzAsMyxbMCwwLCJcXG1hdGhjYWwgQSJdLFsyLDAsIlxcbWF0aGNhbCBCIl0sWzEsMSwiXFxtYXRoY2FsIEMiXSxbMCwxLCJHIl0sWzEsMiwiRiJdLFswLDIsIkZHIiwyLHsic3R5bGUiOnsiYm9keSI6eyJuYW1lIjoiZGFzaGVkIn19fV1d
  \[\begin{tikzcd}
	  {\mathcal A} && {\mathcal B} \\
	  & {\mathcal C}
	  \arrow["G", from=1-1, to=1-3]
	  \arrow["FG"', dashed, from=1-1, to=2-2]
	  \arrow["F", from=1-3, to=2-2]
  \end{tikzcd}\]
  In addition, if $G$ sends injective objects of $\mathcal A$ to $F$-acyclic objects of $B$ (i.e. to elements $B\in \mathcal B$ for which $R^iF(B) = 0$ for $i>0$), then for each $A\in \mathcal A$ there exists a convergent first quadrant cohomological spectral sequence
  \[\phantom{{}^I\!}E_2^{pq} = (R^pF)(R^qG)(A)\quad\text{converging to}\quad R^{p+q}(FG)(A).\]
  There are natural edge maps
  \[(R^pF)(GA)\to R^p(FG)(A)\quad \text{and}\quad R^q(FG)(A)\to F(R^qG(A))\] and an exact sequence of low degree terms
  \[0\to (R^1F)(GA)\to R^1(FG)(A)\to F(R^1G(A))\to (R^2F)(GA)\to R^2(FG)(A).\]
\end{theorem}

The main idea of the proof of \cref{thm: gss} relies extensively on spectral sequences arising from two particular filtrations associated to double complexes and Cartan-Eilenberg resolutions of cochain complexes.

Given a double complex $C = C^{\ast\ast}$, filter the (product or direct sum) total complex $\Tot(C)$ by the columns of $C$; that is, let ${}^I\!F^k\Tot(C)$ be given by
\[({}^I\!F^k\Tot(C))_n = \bigoplus_{\substack{p+q = n\\ p\geq k}} C^{pq}.\] Filter the total complex $\Tot(C)$ by the rows of $C$ so that ${}^{II}\!F^k\Tot(C)$ is given by
\[({}^{II}\!F^k\Tot(C))_n = \bigoplus_{\substack{p+q = n\\ q\geq k}} C^{pq}.\]
Filtering by columns yields a spectral sequence $\{{}^{I}\!E_r^{pq}\}$ with ${}^{I}\!E_0^{pq} = ({}^I\!F^p\Tot(C))_{p+q}/({}^I\!F^{p+1}\Tot(C))_{p+q} = C^{pq}$. The maps $d_0$ are the vertical differentials $d^v$ of $C$, so ${}^{I}\!E_1^{pq} = H_v^q(C^{p\ast})$ with differentials $d_1$ given by $d_1^{pq}\colon H_v^q(C^{p\ast})\to H_v^q(C^{(p+1)\ast})$, induced by the horizontal differentials $d_h$ on $C$. Then ${}^{I}\!E_2^{pq} = H_h^pH_v^q(C)$.

Filtering by rows yields a spectral sequence $\{{}^{II}\!E_r^{pq}\}$ with ${}^{II}\!E_0^{pq} = ({}^{II}\!F^p\Tot(C))_{p+q}/({}^{II}\!F^{p+1}\Tot(C))_{p+q} = C^{qp}$. The maps $d_0$ are the horizontal differentials $d^h$ of $C$ (but are vertical in this sheet), so the maps $d_1$ are induced by the vertical differentials of $C$. Then ${}^{II}\!E_2^{pq} = H_v^pH_h^q(C)$.

If $C$ is a first quadrant double complex, the filtrations ${}^I\!F^k\Tot(C)$ and ${}^{II}\!F^k\Tot(C)$ are canonically bounded (in this setting, this means ${}^{I,II}\!F^0\Tot(C) = \Tot(C)$ and $({}^{I,II}\!F^{n+1}\Tot(C))_n = \Tot(C)_n$ for each $n$), so
\[{}^{I}\!E_2^{pq} = H_h^pH_v^q(C) \quad \text{and}\quad {}^{II}\!E_2^{pq} = H_v^pH_h^q(C) \quad\text{converge to}\quad H^{p+q}(\Tot(C)).\]
This procedure is nice because it allows us to calculate homology in two different ways. We will use this technology on a Cartan-Eilenberg resolution of a particular cochain complex (yielding a double complex), to obtain the Grothendieck spectral sequence in \cref{thm: gss}.

Let $\mathcal A$ be an Abelian category with enough injectives. A (right) Cartan-Eilenberg resolution of a cochain complex $A^\ast$ in $\mathcal A$ is an upper half-plane complex $I = I^{\ast\ast}$ of injective objects of $\mathcal A$ and a chain map $A^\ast\xrightarrow{\epsilon} I^{\ast 0}$ (called the augmentation map) such that \begin{enumerate}
  \item the columns $I^{p\ast}$ are injective resolutions of $A^p$,
  \item the induced maps on coboundaries and cohomology 
  \begin{align*}
    Z^p(\epsilon)&\colon Z^p(A)\to Z^p(I,d_h),\\
    B^p(\epsilon)&\colon B^p(A)\to B^p(I,d_h),\\
    H^p(\epsilon)&\colon H^p(A)\to H^p(I,d_h)
  \end{align*}
  form injective resolutions, where $B^p(I,d_h)$ is the cochain complex given by $(B^p(I,d_h))^q = \im(d_h^{(p-1)q})$. The cochain complexes $Z^p(I,d_h)$ and $H^p(I,d_h) = Z^p(I,d_h)/B^p(I,d_h)$ are defined similarly.
\end{enumerate}
Every cochain complex $A = A^\ast$ in $\mathcal A$ has a Cartan-Eilenberg resolution $A\to I$. Let $F\colon\mathcal A\to\mathcal B$ be a left exact functor. We may define the (right) hyper-derived functor $\mathbb R^iF$ by $\mathbb R^iF(A) = H^i\Tot^{\Pi}(F(I))$ whenever $\Tot^{\Pi}(F(I))$ exists in $\mathcal B$. In the previous discussion on spectral sequences associated to filtrations of cochain complexes, let $A^p = 0$ for $p<0$ and take $C$ to be $F(I)$ to see that there are two spectral sequences converging to $\mathbb R^iF(A)$.

\begin{proof}[Proof of \cref{thm: gss}.]
  Let $A\in \mathcal A$ and choose an injective resolution $A\to I$ of $A$ in $\mathcal A$. Apply $G$ to obtain a cochain complex $G(I)$ in $\mathcal B$. Then consider a Cartan-Eilenberg resolution $J$ of $G(I)$. Since $G(I)$ has no terms in degrees less than $0$, $J$, and hence also $F(J)$, is a first quadrant double complex so that $\Tot^\Pi(F(J))$ exists in $\mathcal C$. Thus we may consider $\mathbb R^iF(G(I)) = H^i\Tot^\Pi(F(J))$, and observe that there are two spectral sequences converging to $\mathbb R^iF(G(I))$.

  The first spectral sequence converging to $\mathbb R^{p+q}F(G(I))$ is given by ${}^{I}\!E_2^{pq} = H_h^pH_v^q(F(J)) = H_h^p(R^qF[G(I)])$. But $G$ takes injective objects of $\mathcal A$ to $F$-acyclic objects of $\mathcal B$, so $R^qF[G(I^p)] = 0$ for $q>0$. Then this spectral sequence collapses, leaving only the terms on the horizontal axis ${}^{I}\!E_2^{p0} = H_h^p(FG(I)) = R^p(FG)(A)$, which are the cohomology groups $H^p\Tot^\Pi(F(J)) = \mathbb R^pF(G(I))$.

  On the other hand, the other spectral sequence converging to $\mathbb R^{p+q}F(G(I))$ is given by
  \[{}^{II}\!E_2^{pq} = H_v^pH_h^q(F(J)) \eq{\smallmathhappy} H_v^p(F(H_h^qJ)) = (R^pF)(H^q(GI)) = (R^pF)(R^qG)(A)\]
  The equality $(\mathhappy)$ involving commuting $F$ with taking cohomology requires some elaboration. Since $J = J^{\ast\ast}$ is a double complex, we obtain the short exact sequences 
  \[0\to Z^q(J,d_h)^\ast\to J^{q\ast}\to B^{q+1}(J,d_h)^\ast\to 0\quad\text{and}\quad 0\to B^q(J,d_h)^\ast\to Z^q(J,d_h)^\ast\to H^q(J,d_h)^\ast\to 0.\]
  Since $Z^q(J,d_h)^\ast$ and $B^q(J,d_h)^\ast$ are injective, the above short exact sequences split so that $J^{q\ast} = Z^q(J,d_h)^\ast\oplus B^{q+1}(J,d_h)^\ast$ and $Z^q(J,d_h)^\ast = B^q(J,d_h)^\ast \oplus H^q(J,d_h)^\ast$. Since $F$ is an additive functor, applying $F$ to the above short exact sequences preserves exactness. Thus $H_h^q(F(J)) = FZ^q(J,d_h)^\ast/FB^q(J,d_h)^\ast = (FB^q(J,d_h)^\ast\oplus FH^q(J,d_h)^\ast)/FB^q(J,d_h)^\ast = FH^q(J,d_h)^\ast$ as needed.

  Thus for any $A\in\mathcal A$, the spectral sequence $E_2^{pq} = (R^pF)(R^qG)(A)$ converges to $\mathbb R^{p+q}F(G(I)) = R^{p+q}(FG)(A)$. The edge maps are indeed the natural maps $(R^pF)(GA) = E_2^{p0}\to E_\infty^{p0}\subset H^p\Tot^\Pi(F(J)) = R^p(FG)(A)$ and $R^q(FG)(A) = H^q\Tot^\Pi(F(J))\to E_\infty^{0q}\subset E_2^{0q} = F(R^qG(A))$.

  A first quadrant spectral sequence $\{E_r^{pq}\}_{r\geq 2}$ calculating the cohomology of a total complex $T$ has $E_2^{10} = E_\infty^{10}$, since the differentials mapping to $E_r^{10}$ have zero modules as domain and differentials with $E_r^{10}$ as domain map to zero modules for $r\geq 2$. Then one edge map gives the first inclusion $E_2^{10}\subset H^1(T)$. Since $F^1H^1(T)\cong E_\infty^{10} = E_2^{10}$, the map $H^1(T)\to H^1(T)/F^1H^1(T)\cong E_\infty^{01}\subset E_2^{01}$ has kernel $F^1H^1(T) \cong E_2^{10}$. The differentials mapping to $E_r^{01}$ have zero modules as domain but only the differential $d_2^{01}$ maps to $E_2^{20}$, so the image of $H^1(T)\to H^1(T)/F^1H^1(T)\cong E_\infty^{01}\subset E_2^{01}$ is $\ker d_2^{10}$. Since $E_\infty^{20} = E_2^{20}/\im d_2^{10}$ is a quotient of $E_2^{20}$, the kernel of the edge map $E_2^{20}\to H^2(T)$ is $\im d_2^{10}$. Combining everything together, we obtain the exact sequence 
  \[0\to E_2^{10}\to H^1(T)\to E_2^{01}\xrightarrow{d_2}E_2^{20}\to H^2(T).\]
  Specializing this exact sequence to the Grothendieck spectral sequence yields the exact sequence of low degree terms $0\to (R^1F)(GA)\to R^1(FG)(A)\to F(R^1G(A))\to (R^2F)(GA)\to R^2(FG)(A)$.
\end{proof} 

An application of the Grothendieck spectral sequence in geometry is in sheaf cohomology. Let $f\colon X\to Y$ be a continuous map of topological spaces. Consider the pushforward functor $f_\ast$ on sheaves on $X$, which is right adjoint to the pullback functor $f^{-1}$ on sheaves on $Y$. Then $f_\ast$ is left exact. Because $f^{-1}$ is exact, $f_\ast$ preserves injectives. The category of sheaves on a topological space has enough injectives, so we are lead to consider composing $f_\ast$ with the global sections functor $\Gamma$ on sheaves on $Y$ and applying the Grothendieck spectral sequence. The derived functors $R^i\Gamma$ form sheaf cohomology, which is usually written as $H^i(Y,-)$. For a sheaf $\mathcal F$ on $X$, observe that $\Gamma(f_\ast \mathcal F) = (f_\ast\mathcal F)(Y) = \mathcal F(f^{-1}Y) = \mathcal F(X) = \Gamma(\mathcal F)$, which we suggestively summarize as $\Gamma f_\ast = \Gamma$, where the second $\Gamma$ refers to taking global sections of sheaves on $X$. Therefore for any sheaf $\mathcal F$ on $X$, the Grothendieck spectral sequence is $E_2^{pq} = H^p(Y,R^qf_\ast \mathcal F)$, which converges to $H^{p+q}(X,\mathcal F)$.

Pushing forward to a point is the same as taking global sections: The constant map $g\colon Y\to \{\ast\}$ has pushforward $g_\ast$ satisfying $(g_\ast \mathcal G)(\{\ast\}) = \mathcal G(g^{-1}\{\ast\}) = G(Y) = \Gamma(\mathcal G)$. So in general we might consider the composition of two continuous maps $f\colon X\to Y$ and $g\colon Y\to Z$ and the spectral sequence $E_2^{pq} = R^pg_\ast(R^qf_\ast\mathcal F)$ converging to $R^{p+q}(g_\ast f_\ast)\mathcal F = (R^{p+q}(gf)_\ast)\mathcal F$ for any sheaf $\mathcal F$ on $X$. These spectral sequences find much use in algebraic geometry.
\end{document}