\documentclass[11pt,leqno]{article}
\headheight=13.6pt

% packages
\usepackage[alphabetic]{amsrefs}
\usepackage{physics}
% margin spacing
\usepackage[top=1in, bottom=1in, left=0.5in, right=0.5in]{geometry}
\usepackage{hanging}
\usepackage{amsfonts, amsmath, amssymb, amsthm}
\usepackage{systeme}
\usepackage[none]{hyphenat}
\usepackage{fancyhdr}
\usepackage{graphicx}
\graphicspath{{./images/}}
\usepackage{float}
\usepackage{siunitx}
\usepackage{esint}
\usepackage{color}
\usepackage{enumitem}
\usepackage{mathrsfs}
\usepackage{hyperref}
\usepackage[noabbrev, capitalise]{cleveref}
\crefformat{equation}{equation~#2#1#3}
\crefformat{lemma}{\textrm{Lemma}~#2#1#3}

% theorems
\theoremstyle{plain}
\newtheorem{lem}{Lemma}
\newtheorem{lemma}[lem]{Lemma}
\newtheorem{thm}[lem]{Theorem}
\newtheorem{theorem}[lem]{Theorem}
\newtheorem{prop}[lem]{Proposition}
\newtheorem{proposition}[lem]{Proposition}
\newtheorem{cor}[lem]{Corollary}
\newtheorem{corollary}[lem]{Corollary}
\newtheorem{conj}[lem]{Conjecture}
\newtheorem{fact}[lem]{Fact}
\newtheorem{form}[lem]{Formula}

\theoremstyle{definition}
\newtheorem{defn}[lem]{Definition}
\newtheorem{definition/}[lem]{Definition}
\newenvironment{definition}
  {\renewcommand{\qedsymbol}{\textdagger}%
   \pushQED{\qed}\begin{definition/}}
  {\popQED\end{definition/}}
\newtheorem{example}[lem]{Example}
\newtheorem{remark}[lem]{Remark}
\newtheorem{exercise}[lem]{Exercise}
\newtheorem{notation}[lem]{Notation}

\numberwithin{equation}{section}
\numberwithin{lem}{section}

% header/footer formatting
\pagestyle{fancy}
\fancyhead{}
\fancyfoot{}
\fancyhead[L]{M 383C}
\fancyhead[C]{HW5}
\fancyhead[R]{Sai Sivakumar}
\fancyfoot[R]{\thepage}
\renewcommand{\headrulewidth}{1pt}

% paragraph indentation/spacing
\setlength{\parindent}{0cm}
\setlength{\parskip}{10pt}
\renewcommand{\baselinestretch}{1.25}

% extra commands defined here
\newcommand{\br}[1]{\left(#1\right)}
\newcommand{\sbr}[1]{\left[#1\right]}
\newcommand{\cbr}[1]{\left\{#1\right\}}
\newcommand{\eq}[1]{\overset{(#1)}{=}}

% bracket notation for inner product
\usepackage{mathtools}

\DeclarePairedDelimiterX{\abr}[1]{\langle}{\rangle}{#1}

\DeclareMathOperator{\Span}{span}
\DeclareMathOperator{\im}{im}
\DeclareMathOperator{\dist}{dist}
\DeclareMathOperator{\diam}{diam}
\DeclareMathOperator{\supp}{supp}
\DeclareMathOperator{\EV}{ev}
\DeclareMathOperator{\co}{co}
\newcommand{\res}[1]{\operatorname*{res}_{#1}}

% set page count index to begin from 1
\setcounter{page}{1}

\begin{document}
\subsection*{2.10 Exercises}
\begin{enumerate}
  \item[39. NOT DONE] Let $X$ be a reflexive nontrivial Banach space with $F\colon X\to X^{\ast\ast}$ be the canonical isometric isomorphism sending $x\in X$ to $E_x\in X^{\ast\ast}$ which evaluates a functional $f\in X^\ast$ at $x$. Then the adjoint $F^\ast\colon X^{\ast\ast\ast}\to X^\ast$ given by $g\mapsto gF$ of $F$ is an isometric isomorphism of Banach spaces: The adjoint $F^\ast$ is linear since functionals in dual spaces are linear and act pointwise. We show $\norm{F^\ast} = \norm{F}$: We have for $g\in X^{\ast\ast\ast}$ and $f\in X^{\ast\ast}$ that $\abs{F^\ast g(f)} = \abs{g(Ff)} \leq \norm{g}\norm{F}\norm{x}$ so that $\norm{F^\ast g}\leq \norm{F}\norm{g}$; that is, $F^\ast$ is bounded with $\norm{F^\ast}\leq \norm{F}$. For the reverse inequality, note that $\norm{F}>0$ since $F$ is not the zero map, so for any $\varepsilon>0$ there exists $x_0\in X$ with $\norm{x_0} = 1$ and $\norm{Fx_0}\geq \norm{F} - \varepsilon$. Using a corollary of the Hahn-Banach theorem, there exists a $g_0\in X^{\ast\ast\ast}$ with $\norm{g_0} = 1$ and $g_0(Fx_0) = \norm{Fx_0}$. It follows that $\norm{F^\ast}\geq \norm{F^\ast g_0} = \sup_{\norm{x} =1}\abs{T^\ast g_0(x)}\geq \abs{F^\ast g_0(x_0)} = \abs{g_0(Fx_0)} = \norm{Fx_0}\geq \norm{F} - \varepsilon$, and with $\varepsilon>0$ arbitrary, we have $\norm{F^\ast}\geq \norm{F}$.
  
  Isometries are injective. A right inverse of $F^\ast$ is the canonical map $E\colon X^\ast\to X^{\ast\ast\ast}$ sending $f\in X^\ast$ to $E_f\in X^{\ast\ast\ast}$ which evaluates a functional $h\in X^{\ast\ast}$ at $f$: Indeed, for any $f\in X^\ast$ and $x\in X$, we have $(F^\ast E)f(x) = E_fF_x = f(x)$. It follows that $F^\ast$ is an isometric isomorphism of Banach spaces, so the canonical map $E$ is also an isomorphism of Banach spaces, from which we deduce that $X^\ast$ is reflexive.

  The converse is true since $X$ is assumed to be Banach: The image of $X$ in $X^{\ast\ast}$ under the canonical map $F$ as above is closed: Given any convergent sequence $\cbr{Fx_i}\subset X^{\ast\ast}$, since $F$ is a linear isometry, $\cbr{x_i}$ is a Cauchy sequence in $X$ converging to some $x\in X$. Then necessarily $Fx_i$ converges to $Fx$ in $X^{\ast\ast}$, so that $\im F$ is closed in $X^{\ast\ast}$. So by restricting the range of $F$, we have an isometric isomorphism of $X$ with its image in $X^{\ast\ast}$. The canonical map $J\colon X^{\ast\ast}\to X^{\ast\ast\ast\ast}$ taking $f\in X^{\ast\ast}$ to $J_f\in X^{\ast\ast\ast\ast}$ which evaluates a functional at $f$ is an isomorphism because $X^{\ast}$ is reflexive (due to what we proved earlier). We check that $\im F$ is reflexive. Indeed, restricting the map $J$ to $\im F$, we obtain an isometric embedding of $\im F$ in $X^{\ast\ast\ast\ast}$, in particular in $(\im F)^\ast$. Let $E\colon X^{\ast}\to X^{\ast\ast\ast}$ be the canonical isomorphism, and consider its adjoint $E^\ast\colon X^{\ast\ast\ast\ast}\to X^{\ast\ast}$, which by a similar argument as before is a linear isometry. The map $E^\ast$ restricted to $(\im F)^{\ast\ast}$ is a right inverse to $J$ restricted to $\im F$: For any functional $f\in (\im F)^{\ast\ast}$ and $g\in X^{\ast\ast\ast}$, $(JE^\ast )f(g) = J_fE_g=f(g)$. Therefore $\im F$ is reflexive, so up to an isometric isomorphism, $X$ is reflexive.
  
  There are non-reflexive normed linear spaces with reflexive dual spaces. If $D$ is a dense subspace of a Banach space $X$, then $D^\ast $ is isometrically isomorphic to $ X^\ast$, since every bounded functional $D\to \mathbb F$ can be extended to a unique bounded functional $X\to \mathbb F$ with the same norm (Corollary 2.29 and Corollary 2.34 in the notes), so $D^\ast\hookrightarrow X^\ast$, and the reverse injection (a right inverse) is not hard to find since we can always restrict bounded functionals to subspaces.
  Consider the dense subspace $C_0(\Omega)$ in $L^p(\Omega)$ for $1<p<\infty$, and let $q$ be the conjugate exponent to $p$. Then $C_0(\Omega)^\ast\cong L^q(\Omega)\cong L^q(\Omega)^{\ast\ast}$ but $C_0(\Omega)\not\cong L^p(\Omega)$.
  
  \item[40.] Let $y = \cbr{y_i}$ be a sequence of complex numbers such that $\sum_ix_iy_i$ converges for every $x = \cbr{x_i}$ in $c_0 = \cbr{x\in \ell^\infty\mid \lim_{i\to\infty}x_i = 0}$. Consider the sequence of sequences $\cbr{y_n} = \cbr{\cbr{y_i}_{i=1}^n}\subset c_0$, and define the maps $T_{y_n}\colon c_0\to\mathbb C$ for $n\geq 1$ given by $x\mapsto \sum_{i=1}^nx_iy_i$. It is clear that the maps $T_{y_n}$ are linear for $n\geq 1$.
  
  There cannot exist an $x\in c_0$ such that $\sup_{n\geq 1}\abs{T_{y_n}x} = \infty$ due to the definition of $y$. We check quickly that $c_0$ is complete by showing it is closed in $\ell^\infty$: Consider the linear transformations $\limsup_i(\cdot)_i,\liminf_i(\cdot)_i\colon \ell^\infty\to\mathbb C$ that extract limit superiors and limit inferiors of sequences in $\ell^\infty$. These maps are clearly bounded. Then $c_0 = \ker\limsup_i(\cdot)_i\cap\ker \liminf_i(\cdot)_i$, which is closed since kernels of bounded linear transformations are closed. Then by the Uniform Boundedness Principle, the maps $T_{y_n}$ are uniformly bounded in $n$. Consider the sequence of sequences  $\cbr{\tilde y_n} = \cbr{\cbr{\overline{y_i}/\abs{y_i}}_{i=1}^n}\subset c_0$, and note that $\norm{\tilde y_n} = 1$ for all $n\geq 1$. Then $\sum_i \abs{y_i} = \lim_{n\to\infty}T_{y_n}\tilde y_n\leq \limsup_{n\geq 1}\norm{T_{y_n}}\norm{\tilde y_n} < \infty$; that is, $y\in \ell^1$.

  This result serves as a lemma that can be used to show that $c_0^\ast \cong \ell^1$. For any $y\in \ell^1$, define $T_y\colon c_0\to\mathbb C$ by $x\mapsto \sum_i x_iy_i$, where $\sum_i x_iy_i\leq \norm{x}\sum_i\abs{y_i}<\infty$. The maps $T_y$ are linear and bounded. The map $y\mapsto T_y$ is also linear and has norm $1$. So there is a copy of $\ell^1$ in $c_0^\ast$, an isometric embedding. Now let $f\in c_0^\ast$, and consider the sequence $\cbr{fe_i}$ of complex numbers. We show that $\cbr{fe_i}\in\ell^1$. Then for any $x\in c_0$, we have $\sum_ix_ife_i = \lim_{n\to\infty}\sum_{i=1}^nx_ife_i\leq \norm{x}\lim_{n\to\infty}\norm{f(\sum_{i=1}^ne_i)}\leq \norm{x}\norm{f} < \infty$, so from the previous result we have that $\cbr{fe_i}\in \ell^1$. The map $f\mapsto \cbr{fe_i}$ is linear and is also a right inverse to the map $y\mapsto T_y$. So the isometric embedding from earlier is actually an isomorphism.
  
  \item[41.] 
  
  \item[43.] 
  
\end{enumerate}
\end{document}