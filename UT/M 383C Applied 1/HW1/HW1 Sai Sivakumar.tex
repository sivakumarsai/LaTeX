\documentclass[11pt,leqno]{article}
\headheight=13.6pt

% packages
\usepackage[alphabetic]{amsrefs}
\usepackage{physics}
% margin spacing
\usepackage[top=1in, bottom=1in, left=0.5in, right=0.5in]{geometry}
\usepackage{hanging}
\usepackage{amsfonts, amsmath, amssymb, amsthm}
\usepackage{systeme}
\usepackage[none]{hyphenat}
\usepackage{fancyhdr}
\usepackage{graphicx}
\graphicspath{{./images/}}
\usepackage{float}
\usepackage{siunitx}
\usepackage{esint}
\usepackage{color}
\usepackage{enumitem}
\usepackage{mathrsfs}
\usepackage{hyperref}
\usepackage[noabbrev, capitalise]{cleveref}
\crefformat{equation}{equation~#2#1#3}
\crefformat{lemma}{\textrm{Lemma}~#2#1#3}

% theorems
\theoremstyle{plain}
\newtheorem{lem}{Lemma}
\newtheorem{lemma}[lem]{Lemma}
\newtheorem{thm}[lem]{Theorem}
\newtheorem{theorem}[lem]{Theorem}
\newtheorem{prop}[lem]{Proposition}
\newtheorem{proposition}[lem]{Proposition}
\newtheorem{cor}[lem]{Corollary}
\newtheorem{corollary}[lem]{Corollary}
\newtheorem{conj}[lem]{Conjecture}
\newtheorem{fact}[lem]{Fact}
\newtheorem{form}[lem]{Formula}

\theoremstyle{definition}
\newtheorem{defn}[lem]{Definition}
\newtheorem{definition/}[lem]{Definition}
\newenvironment{definition}
  {\renewcommand{\qedsymbol}{\textdagger}%
   \pushQED{\qed}\begin{definition/}}
  {\popQED\end{definition/}}
\newtheorem{example}[lem]{Example}
\newtheorem{remark}[lem]{Remark}
\newtheorem{exercise}[lem]{Exercise}
\newtheorem{notation}[lem]{Notation}

\numberwithin{equation}{section}
\numberwithin{lem}{section}

% header/footer formatting
\pagestyle{fancy}
\fancyhead{}
\fancyfoot{}
\fancyhead[L]{M 383C}
\fancyhead[C]{HW1}
\fancyhead[R]{Sai Sivakumar}
\fancyfoot[R]{\thepage}
\renewcommand{\headrulewidth}{1pt}

% paragraph indentation/spacing
\setlength{\parindent}{0cm}
\setlength{\parskip}{10pt}
\renewcommand{\baselinestretch}{1.25}

% extra commands defined here
\newcommand{\br}[1]{\left(#1\right)}
\newcommand{\sbr}[1]{\left[#1\right]}
\newcommand{\cbr}[1]{\left\{#1\right\}}
\newcommand{\eq}[1]{\overset{(#1)}{=}}

% bracket notation for inner product
\usepackage{mathtools}

\DeclarePairedDelimiterX{\abr}[1]{\langle}{\rangle}{#1}

\DeclareMathOperator{\Span}{span}
\DeclareMathOperator{\im}{im}
\DeclareMathOperator{\dist}{dist}
\DeclareMathOperator{\diam}{diam}
\DeclareMathOperator{\supp}{supp}
\DeclareMathOperator{\EV}{ev}
\DeclareMathOperator{\co}{co}
\newcommand{\res}[1]{\operatorname*{res}_{#1}}

% set page count index to begin from 1
\setcounter{page}{1}

\begin{document}
\subsection*{2.10 Exercises}
\begin{enumerate}
  \item[1.] Let $X$ be a vector space.
  \begin{enumerate}
    \item Let $A,B\subset X$ be convex, and let $A+B = \cbr{a+b\mid a\in A,b\in B}$. Let $x,y\in A+B$ so that $x = a_1+b_1$ and $y = a_2+b_2$ for some $a_1,a_2\in A$ and $b_1,b_2\in B$. Then for $t\in [0,1]$, we have $tx+(1-t)y = (ta_1 + (1-t)a_2) + (tb_1 + (1-t)b_2)\in A+B$, since $A$ and $B$ are convex.
    
    Now let $x,y\in A\cap B$. Then for $t\in [0,1]$, the vector $tx + (1-t)y$ is an element of $A$ since $A$ is convex, and is also an element of $B$ since $B$ is convex. Hence $tx + (1-t)y\in A\cap B$ from which it follows that $A\cap B$ is convex.

    For convex sets $A,B\subset X$, $A\cup B$ or $A\setminus B$ need not be convex. Let $X = \mathbb R^2$. Then with $A = B_1(-2,0)$ and $B = B_1(2,0)$, it is clear that $A,B$ are convex but $A\cup B$ is not convex: No line segment with endpoints in each ball is contained in the union of the balls. With $A = B_2(0,0)$ and $B = B_1(0,0)$, it is clear that $A,B$ are convex but $A\setminus B$ is not convex: The set $A\setminus B$ is an annulus.

    \item Let $A\subset X$, and let $cA = \cbr{ca\mid a\in A}$ for $c\in \mathbb R$. Then $2A = \cbr{2a\mid a\in A}$ is a subset of $A + A$, since for any $a\in 2A$, $a = 2b$ for some $b\in A$ so that $a = 2b = b+b$. It follows that $a \in A+A$ and hence $2A\subset A + A$. The reverse containment $A + A\subset 2A$ need not hold, for example if $X = \mathbb R^2$ and $A = \cbr{e_1,e_2}$, then $2A = \cbr{2e_1,2e_2}$ and $A + A =\cbr{2e_1, 2e_2, e_1 + e_2}$.
    
    The reverse containment $A + A\subset 2A$ holds whenever $A$ is convex: If $A\subset X$ is convex, $2A$ is also convex. Take two elements $x,y\in 2A$, so that $x = 2a_1$ and $y = 2a_2$ for $a_1,a_2\in A$. Then for $t\in [0,1]$, $tx+(1-t)y = 2(ta_1 + (1-t)a_2)\in 2A$ since $A$ is convex. An element $x \in A+A$ is of the form $a+b$ for $a,b\in A$, and so $x = (2a)/2 + (2b)/2\in 2A$ since $2A$ is convex. It follows in this case that $2A = A + A$.
  \end{enumerate}

  \item[3.] Let $(X,d)$ be a metric space and consider $A,B\subset X$.
  
  Suppose that $\diam(A)>\diam(B)$; that is, $\sup \cbr{d(x,y)\mid x,y\in A}> \sup \cbr{d(x,y)\mid x,y\in B}$. There exists $w,z\in A$ such that $\diam(A) \geq d(w,z)>\diam(B)$. We show that one of $w,z$ do not belong to $B$. If both $w,z$ belong to $B$, then $d(w,z)\in \cbr{d(x,y)\mid x,y\in B}$, so $\diam(B)\geq d(w,z)$, a contradiction. It follows that $A\not\subset B$. Hence $A\subset B$ implies that $\diam(A)\leq \diam(B)$. So if $B$ is bounded, $A$ is bounded since $\diam(A)\leq \diam(B)< \infty$.
  % The inequality $\diam(A)\leq \diam(B)$ holds when $A$ or $B$ is empty, and when $A$ or $B$ is unbounded. So assume that $A,B$ are bounded. Then $\cbr{d(x,y)\mid x,y\in A}\subset \cbr{d(x,y)\mid x,y\in B}\subset \mathbb R$, from which it follows that the suprema of these sets (the diameters of $A,B$) exist and $\diam(A) = \sup \cbr{d(x,y)\mid x,y\in A}\leq \sup \cbr{d(x,y)\mid x,y\in B} = \diam(B)$.
  
  \item[4.] Let $(X,d)$ be a metric space.
  \begin{enumerate}
    \item The function $\rho\colon X\times X\to \mathbb R$ given by $\rho(x,y) = \min\cbr{1,d(x,y)}$ defines a metric on $X$. For any $x,y\in X$, it is clear that $\rho(x,y)\geq 0$ since $d(x,y)\geq 0$. We have $\rho(x,y) = \min\cbr{1,d(x,y)} = 0$ if and only if $d(x,y) = 0$, which holds if and only if $x = y$. Furthermore, it is clear that $\rho(x,y) = \rho(y,x)$ since $d(x,y) = d(y,x)$.
    
    Before verifying the triangle inequality for $\rho$, we prove a small result. For nonnegative $a,b\in\mathbb R$, $\min\cbr{1,a+b}\leq \min\cbr{1,a} + \min\cbr{1,b}$. In the case that $\min\cbr{1,a+b}= a+b$, then both $a,b\leq 1$ so that $\min\cbr{1,a+b} = a+b = \min\cbr{1,a} + \min\cbr{1,b}$. 
    % If both $a,b<1/2$, then $a+b<1$. Hence if $a+b\geq 1$, $a$ or $b$ is greater than or equal to $1/2$. 
    Now suppose that $\min\cbr{1,a+b}= 1$ so that $a+b\geq 1$.
    % ; without loss of generality assume that $a\geq 1/2$. If $\min\cbr{1,b} = 1$ Then $\min\cbr{1,a+b}= 1 \leq 1/2 + $
    For each of the four values that $\min\cbr{1,a} + \min\cbr{1,b}$ can take, the inequality $\min\cbr{1,a+b}= 1\leq \min\cbr{1,a} + \min\cbr{1,b}$ holds. Thus for $x,y,z\in X$, \begin{multline*}
      \rho(x,y) = \min\cbr{1,d(x,y)}\leq \min\cbr{1,d(x,z) + d(z,y)}\\\leq \min\cbr{1,d(x,z)} + \min\cbr{1,d(z,y)} = \rho(x,z) + \rho(z,y).
    \end{multline*}

    \item To see that $U\subset X$ is open in $(X,d)$ if and only if $U$ is open in $(X,\rho)$, it suffices to show that the metric topologies $d$ and $\rho$ induce are the same. In particular, it is enough to show that for any $x\in X$ and $\varepsilon>0$, there exists $\eta, \nu>0$ such that $B_\eta^d(x)\subset B_\varepsilon^\rho(x)$ and $B_\nu^\rho(x)\subset B_\varepsilon^d(x)$. Indeed, for $\varepsilon>0$, choose $\eta < \varepsilon$ and $\nu< \min\cbr{1,\varepsilon}$. An element $y\in B_\eta^d(x)$ satisfies $d(x,y)<\eta<\varepsilon$, so that $\rho(x,y) = \min\cbr{1,d(x,y)} < \varepsilon$ and thus $y\in B_\varepsilon^\rho(x)$. An element $y\in B_\nu^\rho(x)$ satisfies $\min\cbr{1,d(x,y)}<\nu<\min\cbr{1,\varepsilon}$ If $\min\cbr{1,d(x,y)} = 1$, then $1<1$, impossible. Thus $\min\cbr{1,d(x,y)} = d(x,y)$, in which case $d(x,y)<\varepsilon$ and so $y\in B_\varepsilon^d(x)$.
    
    \item The function $\sigma\colon X\times X\to \mathbb R$ given by $\sigma(x,y) = d(x,y)/(1+d(x,y))$ defines a metric on $X$. For any $x,y\in X$, it is clear that $\sigma(x,y) \geq 0$ since $d(x,y)\geq 0$. We have $\sigma(x,y) = d(x,y)/(1+d(x,y)) = 0$ if and only if $d(x,y) = 0$, which holds if and only if $x=y$. Furthermore, it is clear that $\sigma(x,y) = \sigma(y,x)$ since $d(x,y) = d(y,x)$.
    
    Before verifying the triangle inequality for $\sigma$, we prove a small result. For nonnegative $a,b\in \mathbb R$, if $a/(1+a)< b/(1+b)$, then $a+ ab < b + ba$ implies that $a<b$. It follows that if $a\geq b$, then $a/(1+a)\geq b/(1+b)$. Thus for $x,y,z\in X$, \begin{multline*}
      \sigma(x,y) = d(x,y)/(1+d(x,y)) \leq (d(x,z) + d(z,y))/(1+ d(x,z) + d(z,y))\\ \leq d(x,z)/(1+ d(x,z)) + d(z,y)/(1+d(z,y)) = \sigma(x,z) + \sigma(z,y).
    \end{multline*}
    As in part (b) it is sufficient to show that for any $x\in X$ and $\varepsilon>0$, there exists $\eta, \nu>0$ such that $B_\eta^d(x)\subset B_\varepsilon^\sigma(x)$ and $B_\nu^\sigma(x)\subset B_\varepsilon^d(x)$. Indeed, for $\varepsilon>0$, choose $\eta< \varepsilon$ and $\nu <\varepsilon/(1+\varepsilon)$. An element $y\in B_\eta^d(x)$ satisfies $d(x,y)<\eta<\varepsilon$, so that $\sigma(x,y) = d(x,y)/(1+d(x,y))\leq d(x,y)/1 < \varepsilon$ and thus $y\in B_\varepsilon^\sigma(x)$. An element $y\in B_\nu^\sigma(x)$ satisfies $\sigma(x,y) = d(x,y)/(1+d(x,y))<\nu< \varepsilon/(1+\varepsilon)$. Then some algebra yields $d(x,y)<\varepsilon$ so that $y\in B_\varepsilon^d(x)$.
  \end{enumerate}
  
  \item[8.] Let $X$ be a vector space. In parts (b), (c), and (d), let $X$ be a normed linear space.
  \begin{enumerate}
    \item Let $A\subset X$. Then $\co(A)$ is convex: Let $x,y\in \co(A)$ so that $x = \sum_{i=1}^n t_ia_i$ for some $n\geq 1$, $t_i\in [0,1]$, $\sum_{i=1}^n t_i = 1$, and $a_i\in A$; similarly $y = \sum_{j=1}^m s_jb_j$ for some $m\geq 1$, $s_j\in [0,1]$, $\sum_{j=1}^n s_i = 1$, and $b_i\in A$. Then for $t\in [0,1]$, $tx + (1-t)y = \sum_{i=1}^n tt_ia_i + \sum_{j=1}^m (1-t)s_jb_j$. This constitutes a finite sum with $tt_i,(1-t)s_j\in [0,1]$ and $\sum_{i=1}^n tt_i + \sum_{j=1}^m (1-t)s_i = t + 1-t =1$. Since $a_i,b_j\in A$ for $1\leq i\leq n$ and $1\leq j\leq m$, it follows that $tx + (1-t)y\in A$.
    
    The convex hull $\co(A)$ is the intersection of all convex subsets of $X$ containing $A$: The set $A$ is contained in $\co(A)$, since any element $a\in A$ may be written as $\sum_{i=1}^n t_ia_i$ with $n = 1$ and $a_1 = a$, $t_1 = 1$. Since $\co(A)$ is convex, the intersection of all convex subsets of $X$ containing $A$ is a subset of $\co(A)$.
    
    Let $x\in \co(A)$, so that $x = \sum_{i=1}^n t_ia_i$ for some $n\geq 1$, $t_i\in [0,1]$, $\sum_{i=1}^n t_i = 1$, and $a_i\in A$; let $C$ be any convex subset of $X$ containing $A$. We prove by induction on $n$ that $x\in C$. Let $y = \sum_{i=1}^n t_ia_i \in \co(A)$. If $n = 1$, then $y = a_1\in A\subset C$. Let $n>1$, and assume that an element $\sum_{j=1}^ms_jb_j\in \co(A)$ (a convex combination of elements of $A$) with $m<n$ is an element of $C$. Then $y = t_1a_1 + (\sum_{i=2}^n t_i)\sum_{i=2}^{n}t_i/(\sum_{i=2}^n t_i)a_i$. We have that $\sum_{i=2}^{n}t_i/(\sum_{i=2}^n t_i)a_i\in \co(A)$, so by hypothesis is also an element of $C$. Then with $a_1\in C$, we have that $y\in C$ since $t_1 + \sum_{i=2}^n t_i = 1$ and $C$ is convex. Thus by induction, $x\in C$. Since $C$ was any convex subset of $X$ containing $A$, it follows that $\co(A)$ is contained in the intersection of all convex subsets of $X$ containing $A$.

    \item Let $A\subset X$ be an open set. Then $\co(A)$ is open: Let $x\in \co(A)$, so that $x = \sum_{i=1}^n t_ia_i$ for some $n\geq 1$, $t_i\in [0,1]$, $\sum_{i=1}^n t_i = 1$, and $a_i\in A$. Then for each $a_i\in A$, there exists $\varepsilon_i>0$ for which $B_{\varepsilon_i}(a_i)\subset A$ since $A$ is open. Let $\varepsilon = \min\cbr{\varepsilon_1,\dots,\varepsilon_n}$ and let $p\in B_\varepsilon(0)$. Then $x+p = \sum_{i=1}^n t_ia_i + p = \sum_{i=1}^n t_i(a_i + p)$ since $\sum_{i=1}^n t_i = 1$. But $a_i+ p\in A$ for each $i$, so $x+p\in \co(A)$. Since $p$ was arbitrary, $B_\varepsilon(x)\subset \co(A)$, from which it follows that $\co(A)$ is open.
    
    \item The convex hull of a closed set need not be closed.
    Consider taking the convex hull of the subset $A = \cbr{(t,0)\mid t\in \mathbb R}\cup \cbr{(0,1)}\subset \mathbb R^2$. The intersection of all convex subsets of $\mathbb R^2$ containing $A$ is the strip $\cbr{(t,s)\mid t\in\mathbb R, s\in [0,1)}\cup \cbr{(0,1)}$, which is not closed. A limit point $(s,1)$ for $s\neq 0$ is not attainable as a convex combination of points in $A$: Let $(s,1) = x = t_1(0,1) + \sum_{i= 2}^nt_i(s_i,0)\in \co(A)$. We must have $t_1 = 1$, so that $\sum_{i= 2} t_i = 0$ and hence each $t_i = 0$ for $i\geq 2$, so the equality $x = (s,1)$ could not hold unless $s = 0$.
    % Consider taking the convex hull of the subset $A = \cbr{(t,0)\mid t\in \mathbb R}\cup \cbr{(0,t)\mid t\in \mathbb R}$ of $X = \mathbb R^2$, the union of the horizontal and vertical axes. The only convex subset of $X$ containing $A$ is $X$ itself, so $\co(A) = X$, which is open. (Alternatively, for any $x\in X$, one may explicitly find a convex combination of points in $A$ producing $x$.)
    \item The convex hull of a bounded set is bounded. Let $A\subset X$ be a bounded set, so that $A\subset B_R(0)$ for some $R>0$. Proposition 2.4 in the course notes states that open balls in normed linear spaces are convex. It follows that $\co(A)$ is also contained in $B_R(0)$, since $B_R(0)$ is a convex subset of $X$ containing $A$.
  \end{enumerate}
\end{enumerate}
\end{document}