\documentclass[11pt,leqno]{article}
\headheight=13.6pt

% packages
\usepackage[alphabetic]{amsrefs}
\usepackage{physics}
% margin spacing
\usepackage[top=1in, bottom=1in, left=0.5in, right=0.5in]{geometry}
\usepackage{hanging}
\usepackage{amsfonts, amsmath, amssymb, amsthm}
\usepackage{systeme}
\usepackage[none]{hyphenat}
\usepackage{fancyhdr}
\usepackage{graphicx}
\graphicspath{{./images/}}
\usepackage{float}
\usepackage{siunitx}
\usepackage{esint}
\usepackage{cancel}
\usepackage{enumitem}
\usepackage{mathrsfs}
\usepackage{hyperref}
\usepackage[noabbrev, capitalise]{cleveref}
\crefformat{equation}{equation~#2#1#3}
\crefformat{lemma}{\textrm{Lemma}~#2#1#3}

% theorems
\theoremstyle{plain}
\newtheorem{lem}{Lemma}
\newtheorem{lemma}[lem]{Lemma}
\newtheorem{thm}[lem]{Theorem}
\newtheorem{theorem}[lem]{Theorem}
\newtheorem{prop}[lem]{Proposition}
\newtheorem{proposition}[lem]{Proposition}
\newtheorem{cor}[lem]{Corollary}
\newtheorem{corollary}[lem]{Corollary}
\newtheorem{conj}[lem]{Conjecture}
\newtheorem{fact}[lem]{Fact}
\newtheorem{form}[lem]{Formula}

\theoremstyle{definition}
\newtheorem{defn}[lem]{Definition}
\newtheorem{definition/}[lem]{Definition}
\newenvironment{definition}
  {\renewcommand{\qedsymbol}{\textdagger}%
   \pushQED{\qed}\begin{definition/}}
  {\popQED\end{definition/}}
\newtheorem{example}[lem]{Example}
\newtheorem{remark}[lem]{Remark}
\newtheorem{exercise}[lem]{Exercise}
\newtheorem{notation}[lem]{Notation}

\numberwithin{equation}{section}
\numberwithin{lem}{section}

% header/footer formatting
\pagestyle{fancy}
\fancyhead{}
\fancyfoot{}
\fancyhead[L]{M 383C}
\fancyhead[C]{HW1}
\fancyhead[R]{Sai Sivakumar}
\fancyfoot[R]{\thepage}
\renewcommand{\headrulewidth}{1pt}

% paragraph indentation/spacing
\setlength{\parindent}{0cm}
\setlength{\parskip}{10pt}
\renewcommand{\baselinestretch}{1.25}

% extra commands defined here
\newcommand{\br}[1]{\left(#1\right)}
\newcommand{\sbr}[1]{\left[#1\right]}
\newcommand{\cbr}[1]{\left\{#1\right\}}
\newcommand{\eq}[1]{\overset{(#1)}{=}}

% bracket notation for inner product
\usepackage{mathtools}

\DeclarePairedDelimiterX{\abr}[1]{\langle}{\rangle}{#1}

\DeclareMathOperator{\Span}{span}
\DeclareMathOperator{\im}{im}
\DeclareMathOperator{\dist}{dist}
\DeclareMathOperator{\diam}{diam}
\DeclareMathOperator{\supp}{supp}
\DeclareMathOperator{\EV}{ev}
\newcommand{\res}[1]{\operatorname*{res}_{#1}}

% set page count index to begin from 1
\setcounter{page}{1}

\begin{document}
\subsection*{2.10 Exercises}
\begin{enumerate}
  \item[1.] Let $X$ be a vector space.
  \begin{enumerate}
    \item Let $A,B\subset X$ be convex, and let $A+B \coloneqq \cbr{a+b\mid a\in A,b\in B}$. Let $x,y\in A+B$ so that $x = a_1+b_1$ and $y = a_2+b_2$ for some $a_1,a_2\in A$ and $b_1,b_2\in B$. Then for $t\in [0,1]$, we have $tx+(1-t)y = (ta_1 + (1-t)a_2) + (tb_1 + (1-t)b_2)\in A+B$, since $A$ and $B$ are convex.
    
    Now let $x,y\in A\cap B$. Then for $t\in [0,1]$, the vector $tx + (1-t)y$ is an element of $A$ since $A$ is convex, and is also an element of $B$ since $B$ is convex. Hence $tx + (1-t)y\in A\cap B$ from which it follows that $A\cap B$ is convex.

    For convex sets $A,B\subset X$, $A\cup B$ or $A\setminus B$ need not be convex. Let $X = \mathbb R^2$. Then with $A = B_1(-2,0)$ and $B = B_1(2,0)$, it is clear that $A,B$ are convex but $A\cup B$ is not convex: No line segment with endpoints in each ball is contained in the union of the balls. With $A = B_2(0,0)$ and $B = B_1(0,0)$, it is clear that $A,B$ are convex but $A\setminus B$ is not convex: The set $A\setminus B$ is an annulus.

    \item Let $A\subset X$, and let $cA = \cbr{ca\mid a\in A}$ for $c\in \mathbb R$. Then $2A = \cbr{2a\mid a\in A}$ is a subset of $A + A$, since for any $a\in 2A$, $a = 2b$ for some $b\in A$ so that $a = 2b = b+b$. It follows that $a \in A+A$ and hence $2A\subset A + A$. The reverse containment $A + A\subset 2A$ need not hold, for example if $X = \mathbb R^2$ and $A = \cbr{e_1,e_2}$, then $2A = \cbr{2e_1,2e_2}$ and $A + A =\cbr{2e_1, 2e_2, e_1 + e_2}$.
    
    The reverse containment $A + A\subset 2A$ holds whenever $A$ is convex: If $A\subset X$ is convex, it is clear that $2A$ is also convex. An element $x \in A+A$ is of the form $a+b$ for $a,b\in A$, and so $x = (2a)/2 + (2b)/2\in 2A$ since $2A$ is convex. It follows in this case that $2A = A + A$.
  \end{enumerate}

  \item[3.] Let $(X,d)$ be a metric space and consider $A,B\subset X$ with $A\subset B$. The inequality $\diam(A)\leq \diam(B)$ holds when $A$ or $B$ is empty, and when $A$ or $B$ is unbounded. So assume that $A,B$ are bounded. Then $\cbr{d(x,y)\mid x,y\in A}\subset \cbr{d(x,y)\mid x,y\in B}\subset \mathbb R$, from which it follows that the suprema of these sets (the diameters of $A,B$) exist and $\diam(A) = \sup \cbr{d(x,y)\mid x,y\in A}\leq \sup \cbr{d(x,y)\mid x,y\in B} = \diam(B)$.
  
  \item[4. incomplete] Let $(X,d)$ be a metric space.
  \begin{enumerate}
    \item The function $\rho\colon X\times X\to \mathbb R$ given by $\rho(x,y) = \min\cbr{1,d(x,y)}$ defines a metric on $X$. For any $x,y\in X$, it is clear that $\rho(x,y)\geq 0$ since $d(x,y)\geq 0$. We have $\rho(x,y) = \min\cbr{1,d(x,y)} = 0$ if and only if $d(x,y) = 0$, which holds if and only if $x = y$.
    
    Before verifying the triangle inequality for $\rho$, we prove a small result. For nonnegative $a,b\in\mathbb R$, $\min\cbr{1,a+b}\leq \min\cbr{1,a} + \min\cbr{1,b}$. In the case that $\min\cbr{1,a+b}= a+b$, then both $a,b\leq 1$ so that $\min\cbr{1,a+b} = a+b = \min\cbr{1,a} + \min\cbr{1,b}$. 
    % If both $a,b<1/2$, then $a+b<1$. Hence if $a+b\geq 1$, $a$ or $b$ is greater than or equal to $1/2$. 
    Now suppose that $\min\cbr{1,a+b}= 1$ so that $a+b\geq 1$.
    % ; without loss of generality assume that $a\geq 1/2$. If $\min\cbr{1,b} = 1$ Then $\min\cbr{1,a+b}= 1 \leq 1/2 + $
    For each of the four values that $\min\cbr{1,a} + \min\cbr{1,b}$ can take, the inequality $\min\cbr{1,a+b}= 1\leq \min\cbr{1,a} + \min\cbr{1,b}$ holds. Thus for $x,y,z\in X$, \begin{multline*}
      \rho(x,y) = \min\cbr{1,d(x,y)}\leq \min\cbr{1,d(x,z) + d(z,y)}\\\leq \min\cbr{1,d(x,z)} + \min\cbr{1,d(z,y)} = \rho(x,z) + \rho(z,y).
    \end{multline*}

    \item It suffices to show that \dots
    \item The function $\sigma\colon X\times X\to \mathbb R$ given by $\sigma(x,y) = d(x,y)/(1+d(x,y))$ defines a metrix on $X$. For any $x,y\in X$, it is clear that $\sigma(x,y) \geq 0$ since $d(x,y)\geq 0$. We have $\sigma(x,y) = d(x,y)/(1+d(x,y)) = 0$ if and only if $d(x,y) = 0$, which holds if and only if $x=y$.
    
    Before verifying the triangle inequality for $\sigma$, we prove a small result. For nonnegative $a,b\in \mathbb R$, if $a/(1+a)< b/(1+b)$, then $a+ ab < b + ba$ implies that $a<b$. It follows that if $a\geq b$, then $a/(1+a)\geq b/(1+b)$. Thus for $x,y,z\in X$, \begin{multline*}
      \sigma(x,y) = d(x,y)/(1+d(x,y)) \leq (d(x,z) + d(z,y))/(1+ d(x,z) + d(z,y))\\ \leq d(x,z)/(1+ d(x,z)) + d(z,y)/(1+d(z,y)) = \sigma(x,z) + \sigma(z,y).
    \end{multline*}

    To see \dots, it suffices to show that \dots
  \end{enumerate}
  
  \item[8. incomplete] \dots
\end{enumerate}
\end{document}