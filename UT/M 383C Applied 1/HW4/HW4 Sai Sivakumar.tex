\documentclass[11pt,leqno]{article}
\headheight=13.6pt

% packages
\usepackage[alphabetic]{amsrefs}
\usepackage{physics}
% margin spacing
\usepackage[top=1in, bottom=1in, left=0.5in, right=0.5in]{geometry}
\usepackage{hanging}
\usepackage{amsfonts, amsmath, amssymb, amsthm}
\usepackage{systeme}
\usepackage[none]{hyphenat}
\usepackage{fancyhdr}
\usepackage{graphicx}
\graphicspath{{./images/}}
\usepackage{float}
\usepackage{siunitx}
\usepackage{esint}
\usepackage{color}
\usepackage{enumitem}
\usepackage{mathrsfs}
\usepackage{hyperref}
\usepackage[noabbrev, capitalise]{cleveref}
\crefformat{equation}{equation~#2#1#3}
\crefformat{lemma}{\textrm{Lemma}~#2#1#3}

% theorems
\theoremstyle{plain}
\newtheorem{lem}{Lemma}
\newtheorem{lemma}[lem]{Lemma}
\newtheorem{thm}[lem]{Theorem}
\newtheorem{theorem}[lem]{Theorem}
\newtheorem{prop}[lem]{Proposition}
\newtheorem{proposition}[lem]{Proposition}
\newtheorem{cor}[lem]{Corollary}
\newtheorem{corollary}[lem]{Corollary}
\newtheorem{conj}[lem]{Conjecture}
\newtheorem{fact}[lem]{Fact}
\newtheorem{form}[lem]{Formula}

\theoremstyle{definition}
\newtheorem{defn}[lem]{Definition}
\newtheorem{definition/}[lem]{Definition}
\newenvironment{definition}
  {\renewcommand{\qedsymbol}{\textdagger}%
   \pushQED{\qed}\begin{definition/}}
  {\popQED\end{definition/}}
\newtheorem{example}[lem]{Example}
\newtheorem{remark}[lem]{Remark}
\newtheorem{exercise}[lem]{Exercise}
\newtheorem{notation}[lem]{Notation}

\numberwithin{equation}{section}
\numberwithin{lem}{section}

% header/footer formatting
\pagestyle{fancy}
\fancyhead{}
\fancyfoot{}
\fancyhead[L]{M 383C}
\fancyhead[C]{HW4}
\fancyhead[R]{Sai Sivakumar}
\fancyfoot[R]{\thepage}
\renewcommand{\headrulewidth}{1pt}

% paragraph indentation/spacing
\setlength{\parindent}{0cm}
\setlength{\parskip}{10pt}
\renewcommand{\baselinestretch}{1.25}

% extra commands defined here
\newcommand{\br}[1]{\left(#1\right)}
\newcommand{\sbr}[1]{\left[#1\right]}
\newcommand{\cbr}[1]{\left\{#1\right\}}
\newcommand{\eq}[1]{\overset{(#1)}{=}}

% bracket notation for inner product
\usepackage{mathtools}

\DeclarePairedDelimiterX{\abr}[1]{\langle}{\rangle}{#1}

\DeclareMathOperator{\Span}{span}
\DeclareMathOperator{\im}{im}
\DeclareMathOperator{\dist}{dist}
\DeclareMathOperator{\diam}{diam}
\DeclareMathOperator{\supp}{supp}
\DeclareMathOperator{\EV}{ev}
\DeclareMathOperator{\co}{co}
\newcommand{\res}[1]{\operatorname*{res}_{#1}}

% set page count index to begin from 1
\setcounter{page}{1}

\begin{document}
\subsection*{2.10 Exercises}
\begin{enumerate}
  \item[29.] Let $X,Y,Z$ be Banach spaces and $T\colon X\times Y\to Z$ be bilinear.
  \begin{enumerate}
    \item Let $T$ be continuous. Then for fixed $y\in Y$, the map $T(\cdot,y)\colon X\to Z$ is continuous: Let $i_y\colon X\to X\times Y$ be the inclusion map sending $x\in X$ to $(x,y)$, which is \textit{not} linear. The map $i_y$ is continuous since preimages under $i_y$ of the open sets $U\times V$ in $X\times Y$ for $U\subset X$, $V\subset Y$ open are either the empty set or $U$. It follows that $T(\cdot,y) = T\circ i_y$ is continuous, hence bounded so that $\norm{T(x,y)}\leq \norm{T(\cdot,y)}\norm{x}$ for all $x\in X$. Similarly, for fixed $x\in X$, the map $T(x,\cdot)\colon Y\to Z$ is bounded with $\norm{T(x,y)}\leq \norm{T(x,\cdot)}\norm{y}$.
    
    By bilinearity $T(0,y) = T(x,0) = T(0,0) = 0$ for any $x\in X$ and $y\in Y$. So let $x\in X$ and $y\in Y$ be nonzero. Then $\norm{T(x,y)}\leq \norm{T(\cdot,y)}\norm{x} = \norm{T(\cdot,y/\norm{y})}\norm{x}\norm{y}$. Therefore it suffices to show that the family of bounded linear functionals $\cbr{T(\cdot,y)}_{\norm{y} = 1}$ is uniformly bounded. Suppose that there exists $w\in X$ with a sequence $\cbr{y_n}\subset Y$ such that $\norm{T(w,y_n)}$ diverges. This is impossible since $\norm{T(w,y_n)}\leq \norm{T(w,\cdot)}\norm{y_n} = \norm{T(w,\cdot)}$. Thus by the Uniform Boundedness Principle, we must have that $\cbr{T(\cdot,y)}_{\norm{y} = 1}$ is uniformly bounded, from which it follows that there is a constant $M$ such that $\norm{T(x,y)}\leq M\norm{x}\norm{y}$ for all $x\in X$ and $y\in Y$.

    Only completeness of $X$ was used in the above argument; by symmetry only one of $X,Y$ needs to be complete, and $Z$ does not need to be complete.

    \item It remains to prove that continuity of $T$ at the origin $(0,0)$ implies continuity at any point $(x,y)\in X\times Y$. Let $T$ be continuous at $(0,0)$, so there exists $\delta>0$ such that if $\norm{(x,y)} = \max\cbr{\norm{x},\norm{y}} <\leq \delta$, then $\norm{T(x,y)} \leq 1$. Since $\norm{(\delta x/\norm{x},\delta y/\norm{y})} = \delta$, we have $\norm{T(x,y)} = \norm{\norm{x}\norm{y}T(\delta x/\norm{x},\delta y/\norm{y})/\delta^2} = \norm{x}\norm{y}\norm{T(\delta x/\norm{x},\delta y/\norm{y})}/\delta^2\leq \norm{x}\norm{y}/\delta^2$.
    
    \item There exists a constant $M$ such that $\norm{T(x,y)}\leq M\norm{x}\norm{y}$ for all $x\in X$ and $y\in Y$. Fix $(x,y)\in X\times Y$. Let $\varepsilon>0$ be given, and choose $\delta = \varepsilon/M$. If $\norm{(x,y) - (a,b)} = \max\cbr{\norm{x-a},\norm{y-b}}\leq\delta$, then \begin{multline*}
      \norm{T(x,y) - T(a,b)} \leq \norm{T(x,y) - T(a,y)} + \norm{T(a,y) - T(a,b)} = \norm{T(x-a,y)} + \norm{T(a,y-b)}\\ \leq M(\norm{x-a}\norm{y} + \norm{a}\norm{y-b})\leq \varepsilon(\norm{y} + \norm{a})\leq \varepsilon(\norm{x} + \norm{y} + \delta).
    \end{multline*}
    It follows that $T$ is continuous at $(x,y)\in X\times Y$, hence is continuous on $X\times Y$.
  \end{enumerate}
  
  \item[30.] 
  
  \item[31.] 
  
  \item[34.] 
\end{enumerate}
\end{document}