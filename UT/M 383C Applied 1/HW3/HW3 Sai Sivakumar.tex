\documentclass[11pt,leqno]{article}
\headheight=13.6pt

% packages
\usepackage[alphabetic]{amsrefs}
\usepackage{physics}
% margin spacing
\usepackage[top=1in, bottom=1in, left=0.5in, right=0.5in]{geometry}
\usepackage{hanging}
\usepackage{amsfonts, amsmath, amssymb, amsthm}
\usepackage{systeme}
\usepackage[none]{hyphenat}
\usepackage{fancyhdr}
\usepackage{graphicx}
\graphicspath{{./images/}}
\usepackage{float}
\usepackage{siunitx}
\usepackage{esint}
\usepackage{color}
\usepackage{enumitem}
\usepackage{mathrsfs}
\usepackage{hyperref}
\usepackage[noabbrev, capitalise]{cleveref}
\crefformat{equation}{equation~#2#1#3}
\crefformat{lemma}{\textrm{Lemma}~#2#1#3}

% theorems
\theoremstyle{plain}
\newtheorem{lem}{Lemma}
\newtheorem{lemma}[lem]{Lemma}
\newtheorem{thm}[lem]{Theorem}
\newtheorem{theorem}[lem]{Theorem}
\newtheorem{prop}[lem]{Proposition}
\newtheorem{proposition}[lem]{Proposition}
\newtheorem{cor}[lem]{Corollary}
\newtheorem{corollary}[lem]{Corollary}
\newtheorem{conj}[lem]{Conjecture}
\newtheorem{fact}[lem]{Fact}
\newtheorem{form}[lem]{Formula}

\theoremstyle{definition}
\newtheorem{defn}[lem]{Definition}
\newtheorem{definition/}[lem]{Definition}
\newenvironment{definition}
  {\renewcommand{\qedsymbol}{\textdagger}%
   \pushQED{\qed}\begin{definition/}}
  {\popQED\end{definition/}}
\newtheorem{example}[lem]{Example}
\newtheorem{remark}[lem]{Remark}
\newtheorem{exercise}[lem]{Exercise}
\newtheorem{notation}[lem]{Notation}

\numberwithin{equation}{section}
\numberwithin{lem}{section}

% header/footer formatting
\pagestyle{fancy}
\fancyhead{}
\fancyfoot{}
\fancyhead[L]{M 383C}
\fancyhead[C]{HW3}
\fancyhead[R]{Sai Sivakumar}
\fancyfoot[R]{\thepage}
\renewcommand{\headrulewidth}{1pt}

% paragraph indentation/spacing
\setlength{\parindent}{0cm}
\setlength{\parskip}{10pt}
\renewcommand{\baselinestretch}{1.25}

% extra commands defined here
\newcommand{\br}[1]{\left(#1\right)}
\newcommand{\sbr}[1]{\left[#1\right]}
\newcommand{\cbr}[1]{\left\{#1\right\}}
\newcommand{\eq}[1]{\overset{(#1)}{=}}

% bracket notation for inner product
\usepackage{mathtools}

\DeclarePairedDelimiterX{\abr}[1]{\langle}{\rangle}{#1}

\DeclareMathOperator{\Span}{span}
\DeclareMathOperator{\im}{im}
\DeclareMathOperator{\dist}{dist}
\DeclareMathOperator{\diam}{diam}
\DeclareMathOperator{\supp}{supp}
\DeclareMathOperator{\EV}{ev}
\DeclareMathOperator{\co}{co}
\newcommand{\res}[1]{\operatorname*{res}_{#1}}

% set page count index to begin from 1
\setcounter{page}{1}

\begin{document}
\subsection*{2.10 Exercises}
\begin{enumerate}
  \item[16.] Let $\Omega\subset \mathbb R^d$ be measurable with finite (of course nonzero) measure, and let $1\leq p\leq q\leq \infty$. 
  \begin{enumerate}
    \item Let $f\in L^q(\Omega)$. If $q = \infty$, then it is clear that $f\in L^p(\Omega)$, since $(\int_\Omega \abs{f}^p)^{1/p}\leq (\int_\Omega \norm{f}_\infty^p)^{1/p} = \mu(\Omega)^{1/p}\norm{f}_\infty$. If $q\neq \infty$, then $\int_\Omega \abs{f}^p = \int_{\Omega\cap\cbr{\abs{f}\leq 1}}\abs{f}^p + \int_{\Omega\cap\cbr{\abs{f} > 1}}\abs{f}^p \leq \mu(\Omega) + \int_\Omega \abs{f}^q$. In both cases it follows that $f\in L^p(\Omega)$.
    
    If $q = \infty$, we saw above that $\norm{f}_p\leq \mu(\Omega)^{1/p-1/q}\norm{f}_q = \mu(\Omega)^{1/p}\norm{f}_\infty$. It is also clear that the inequality $\norm{f}_p\leq \mu(\Omega)^{1/p-1/q}\norm{f}_q$ holds when $p = q$. So let $1\leq p < q <\infty$.
    
    There exists $r>p$ such that $p/q + p/r = 1$; that is, that $1/q + 1/r = 1/p$. Then from Young's inequality, we have $(ab)^p \leq (a^p)^{q/p}/(q/p) + (b^p)^{r/p}/(r/p)$ for $a,b$ nonnegative (so $a^p,b^p$ are also nonnegative). Thus $(ab)^p/p\leq a^q/q + b^r/r$ for $a,b$ nonnegative. Let $f\in L^r(E)$ and $g\in L^q(E)$ for any measurable set $E\subset \mathbb R^d$. It is clear that if $\norm{f}_r = 0$ or $\norm{g}_q=0$, then $fg$ is almost everywhere the zero function, so that the inequality $\int_E\abs{fg}^p \leq \norm{f}_r^p\norm{g}_q^p$ holds. So let $\norm{f}_r > 0$ and $\norm{g}_q > 0$; by choosing $a = \abs{f}/\norm{f}_r$ and $b = \abs{g}/\norm{g}_q$ in the inequality $(ab)^p/p\leq a^q/q + b^r/r$ and integrating, we obtain $\int_E\abs{fg}^p \leq \norm{f}_r^p\norm{g}_q^p$. Hence for $f\in L^r(E)$, $g\in L^q(E)$ for any measurable set $E\subset \mathbb R^d$, $fg\in L^p(E)$ with $\norm{fg}_p \leq \norm{f}_r\norm{g}_q$.

    By specializing to $E = \Omega$, $1\in L^r(\Omega)$, and $f\in L^q(\Omega)$, we have $\norm{f}_p\leq \mu(\Omega)^{1/p-1/q}\norm{f}_q$.
    
    \item If $f\in L^\infty(\Omega)$, then $f\in L^p(\Omega)$ with $\norm{f}_p\leq \mu(\Omega)^{1/p}\norm{f}_\infty$ for all $1\leq p< \infty$. Then take $p$ to infinity to deduce that $\limsup_{p\to\infty}\norm{f}_p\leq \norm{f}_\infty$. Let $N>0$. Then for any $p>0$, $\mu(\cbr{\abs{f}>N}) = \int_{\Omega\cap\cbr{\abs{f}>N}} 1\leq \int_{\Omega\cap\cbr{\abs{f}>N}} \abs{f}^p/N^p \leq \norm{f}_p^p/N^p$. Now let $N$ be in $(0,\norm{f}_\infty)$ so that $\mu(\cbr{\abs{f}>N})$ has finite nonzero measure; thus obtain the inequality $\norm{f}_p\geq N\mu(\cbr{\abs{f}>N})^{1/p}$. For fixed $N>0$, take $p$ to infinity to deduce that $\liminf_{p\to\infty} \norm{f}_p\geq N$; then take $N$ to $\norm{f}_\infty$ and obtain $\liminf_{p\to\infty} \norm{f}_p\geq \norm{f}_\infty$. Hence $\lim_{p\to\infty} \norm{f}_p = \norm{f}_\infty$.
    
    \item Let $f\in L^p(\Omega)$ for all $p$ with $1\leq p <\infty$, but let $f\not\in L^\infty(\Omega)$; that is, $\norm{f}_\infty >K$ for any $K>0$. In other words, for any $K>0$ the set $\cbr{\abs{f}>K}$ has positive finite measure. Fix $K>0$. Then for any $p>0$, $\mu(\cbr{\abs{f}>K})\leq \norm{f}_p^p/K^p$; hence $\norm{f}_p\geq K\mu(\cbr{\abs{f}>K})^{1/p}$. Thus $\liminf_{p\to\infty}\norm{f}_p \geq K$. Now take $K$ arbitrarily large to deduce that $\norm{f}_p$ has no uniform upper bound in $p$.
  \end{enumerate}
  
  \item[17.] Let $1\leq p<\infty$.
  \begin{enumerate}
    \item For each $r\in \mathbb R^d$, the translation operator $\tau_r$ is linear and maps into $L^p(\mathbb R^d)$: For $f,g\in L^p(\mathbb R^d)$ and $r,x\in \mathbb R^d$, $\tau_r(f+g)(x) = (f+g)(x+r) = f(x+r)+ g(x+r) = \tau_rf + \tau_rg$. Furthermore, $\norm{\tau_r f}_p^p = \int \abs{f(\cdot+r)}^p = \int \abs{f}^p = \norm{f}_p^p$ since the Lebesgue integral is translation invariant (since the Lebesgue measure is translation invariant, approximate by translated simple functions to see this). This computation shows that $\tau_r$ is bounded for any $r\in\mathbb R^d$, and that $\norm{\tau_r} = 1$. 
    
    \item Because the translation operators are bounded, for $f_n\to f$ in $L^p(\mathbb R^d)$, we have $\tau_rf_n\to \tau_rf$ for any $r\in \mathbb R^d$. By density of $C_c^0(\mathbb R^d)$ in $L^p(\mathbb R^d)$, choose $f_n$ to be compactly supported and continuous. Then $\norm{\tau_rf - \tau_s f}_p\leq \norm{\tau_rf - \tau_rf_n}_p + \norm{\tau_rf_n - \tau_sf_n}_p + \norm{\tau_sf_n - \tau_s f}_p$, and the quantities $\norm{\tau_rf - \tau_rf_n}_p$, $\norm{\tau_sf_n - \tau_s f}_p$ may be made arbitrarily small with large $n$.
    
    By the dominated convergence theorem and continuity of $f$ (i.e., of $\tau_rf,\tau_sf$), $\lim_{r\to s} \norm{\tau_sf_n - \tau_s f}_p^p = \lim_{r\to s} \int \abs{\tau_sf_n - \tau_s f}^p = \int \abs{\lim_{r\to s}(\tau_sf_n - \tau_s f)}^p = 0$. Thus $\lim_{r\to s}\norm{\tau_rf - \tau_s f}_p = 0$.
  \end{enumerate}
  
  \item[19.] Let $Y$ be a subspace of a vector space $X$.
  \begin{enumerate}
    \item Let $\cbr{x_i + Y}_{i\in I}$ be the set of all distinct cosets of $Y$ in $X$; that is for $i\neq j$, $x_i+ Y\neq x_j + Y$, and this set is maximal (no further cosets may be added). Then for $i\neq j$, $(x_i + Y)\cap (x_j + Y) = \emptyset$, since otherwise there exist $y_{x_i},y_{x_j}\in Y$ such that $x_i + y_{x_i} = x_j + y_{x_j}$, from which it follows that $x_i = x_j + y_{x_j} - y_{x_i}\in x_j + Y$; similarly $x_j\in x_i + Y$ and so $x_i+ Y = x_j + Y$, impossible. The union $\bigcup_{i\in I}(x_i + Y)$ must be equal to $X$, otherwise, there is some $x\in X$ not in any coset $x_i + Y$ for all $i\in I$. Then $x + Y$ is necessarily a distinct coset from each coset $x_i + Y$ for each $i\in I$, otherwise $x +Y = x_i+Y$ for some $i$ implies that $x\in x_i + Y$. But $\cbr{x_i + Y}_{i\in I}$ was chosen maximally; that is, no further distinct cosets exist.
    
    We show that the addition and scalar multiplication of cosets of $Y$ is well defined. 
  \end{enumerate}
  
  \item[20.] 
\end{enumerate}
\end{document}