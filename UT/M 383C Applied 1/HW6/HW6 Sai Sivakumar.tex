\documentclass[11pt,leqno]{article}
\headheight=13.6pt

% packages
\usepackage[alphabetic]{amsrefs}
\usepackage{physics}
% margin spacing
\usepackage[top=1in, bottom=1in, left=0.5in, right=0.5in]{geometry}
\usepackage{hanging}
\usepackage{amsfonts, amsmath, amssymb, amsthm}
\usepackage{systeme}
\usepackage[none]{hyphenat}
\usepackage{fancyhdr}
\usepackage{graphicx}
\graphicspath{{./images/}}
\usepackage{float}
\usepackage{siunitx}
\usepackage{esint}
\usepackage{color}
\usepackage{enumitem}
\usepackage{mathrsfs}
\usepackage{hyperref}
\usepackage[noabbrev, capitalise]{cleveref}
\crefformat{equation}{equation~#2#1#3}
\crefformat{lemma}{\textrm{Lemma}~#2#1#3}

% theorems
\theoremstyle{plain}
\newtheorem{lem}{Lemma}
\newtheorem{lemma}[lem]{Lemma}
\newtheorem{thm}[lem]{Theorem}
\newtheorem{theorem}[lem]{Theorem}
\newtheorem{prop}[lem]{Proposition}
\newtheorem{proposition}[lem]{Proposition}
\newtheorem{cor}[lem]{Corollary}
\newtheorem{corollary}[lem]{Corollary}
\newtheorem{conj}[lem]{Conjecture}
\newtheorem{fact}[lem]{Fact}
\newtheorem{form}[lem]{Formula}

\theoremstyle{definition}
\newtheorem{defn}[lem]{Definition}
\newtheorem{definition/}[lem]{Definition}
\newenvironment{definition}
  {\renewcommand{\qedsymbol}{\textdagger}%
   \pushQED{\qed}\begin{definition/}}
  {\popQED\end{definition/}}
\newtheorem{example}[lem]{Example}
\newtheorem{remark}[lem]{Remark}
\newtheorem{exercise}[lem]{Exercise}
\newtheorem{notation}[lem]{Notation}

\numberwithin{equation}{section}
\numberwithin{lem}{section}

% header/footer formatting
\pagestyle{fancy}
\fancyhead{}
\fancyfoot{}
\fancyhead[L]{M 383C}
\fancyhead[C]{HW6}
\fancyhead[R]{Sai Sivakumar}
\fancyfoot[R]{\thepage}
\renewcommand{\headrulewidth}{1pt}

% paragraph indentation/spacing
\setlength{\parindent}{0cm}
\setlength{\parskip}{10pt}
\renewcommand{\baselinestretch}{1.25}

% extra commands defined here
\newcommand{\br}[1]{\left(#1\right)}
\newcommand{\sbr}[1]{\left[#1\right]}
\newcommand{\cbr}[1]{\left\{#1\right\}}
\newcommand{\eq}[1]{\overset{(#1)}{=}}

% bracket notation for inner product
\usepackage{mathtools}

\DeclarePairedDelimiterX{\abr}[1]{\langle}{\rangle}{#1}

\DeclareMathOperator{\Span}{span}
\DeclareMathOperator{\im}{im}
\DeclareMathOperator{\dist}{dist}
\DeclareMathOperator{\diam}{diam}
\DeclareMathOperator{\supp}{supp}
\DeclareMathOperator{\EV}{ev}
\DeclareMathOperator{\co}{co}
\newcommand{\res}[1]{\operatorname*{res}_{#1}}

% set page count index to begin from 1
\setcounter{page}{1}

\begin{document}
\subsection*{2.10 Exercises}
\begin{enumerate}
    \item[49.] Let $X$ be a normed linear space and let $Y$ be a closed (in norm) linear subspace. Let $\cbr{y_i}$ be a weakly convergent sequence in $Y\subseteq X$; that is, for any functional $f\in X^\ast$, $\cbr{fy_i}$ converges in $\mathbb F$ to $fx$ for some $x\in X$. Suppose that $x\not\in Y$, from which it follows that $\dist(x,Y)>0$. Choose a functional $f\in X^\ast$ with $f(x) = \dist(x,Y)>0$ and $f(y) = 0$ for all $y\in Y$ by the Mazur Separation Lemma I. Thus $\cbr{fy_i} = \cbr{0}$ converges to $fx\neq 0$, a contradiction. Hence $x\in Y$.
    \item[51.] Let $X$ be a Banach space and let $T\in X^\ast$. Let $\mathbb F = \Span_{\mathbb F}\cbr{1}$ and $\mathbb F^\ast = \Span_{\mathbb F}\cbr{1^\ast}$. Then for any $f = c1^\ast\in \mathbb F^\ast$, $T^\ast f = cT^\ast 1^\ast = c1^\ast T = cT\in X^\ast$, where the last equality holds since for any $x\in X$, $1^\ast Tx = 1^\ast(Tx1) = Tx1$. Since $c$ was arbitrary, the range of $T^\ast$ is $\Span_{\mathbb F}\cbr{T}\subseteq X^\ast$.
    \item[52.] Let $X$ be a Banach space and let $K\subset X^\ast$ be a weak-$\ast$ compact subset of $X$. The weak-$\ast$ topology is Hausdorff, and since compact subsets of Hausdorff spaces are closed, $K$ is closed.
    % A basic weak-$\ast$ open set containing $0\in X^\ast$ is given by $V = \cbr{f\in X^\ast\mid \abs{fx_i}<\varepsilon_i, 1\leq i\leq m} = \bigcap_{i=1}^mE_{x_i}^{-1}B_{\varepsilon_i}(0)$ for some $m$, $\varepsilon_i>0$, and $x_i\in X$. Then an open cover of $K$ is given by $\cbr{f + V\mid f\in K}$, so there exists $\cbr{f_1,\dots,f_n}$ such that $K\subset\bigcup_{i=1}^n f_i+V$. 
    For any $x\in X$, the evaluation map $E_x\in X^{\ast\ast}$ is continuous by definition of the weak-$\ast$ topology, so $E_xK = Kx$ is compact in $\mathbb F$ (the continuous image of a compact set is compact). Compact sets in Euclidean spaces are closed and bounded, so $K$ is pointwise bounded and hence uniformly bounded in norm by the Uniform Boundedness Principle.
\end{enumerate}
\subsection*{3.6 Exercises}
\begin{enumerate}
    \item[3.] Let $H$ be a real Hilbert space and let $S\subset H$ be a closed (nonempty) convex subset. Fix $x\in H$ and $y\in S$.
    Let $\norm{x-y} = \inf_{z\in S}\norm{x-z}$. Then for $z\in S$ and $\alpha\in[0,1]$, $\alpha y + (1-\alpha)z\in S$. We have by assumption that $0\leq \norm{x-(\alpha y + (1-\alpha)z)}^2 - \norm{x-y}^2 = \norm{(x-y)-((1-\alpha)(z-y))} = -2(1-\alpha)(x-y,z-y) + (1-\alpha)^2\norm{z-y}^2$. Dividing through by $-2(1-\alpha)$ and taking $\alpha\to 1$ yields $0\geq (x-y,z-y)$.
    Assume that for any $z\in S$, $0\geq (x-y,z-y)$. Then $\norm{x-z}^2 = (x-z,x-z) = ((x-y)-(z-y),(x-y)-(z-y)) = \norm{x-y}^2 -2(x-y,z-y) + \norm{z-y}^2\geq \norm{x-y}^2 + \norm{z-y}^2$. Therefore $\norm{x-y}\leq \norm{x-y}\sqrt{1+\norm{z-y}^2/\norm{x-y}^2}\leq \norm{x-z}$ for any $z\in S$. By taking the infimum over $z\in S$, we have $\norm{x-y} \leq \inf_{z\in S}\norm{x-z}$; since $\norm{x-y} \geq \inf_{z\in S}\norm{x-z}$ is true, we have $\norm{x-y} = \inf_{z\in S}\norm{x-z}$.
\end{enumerate}
\end{document}