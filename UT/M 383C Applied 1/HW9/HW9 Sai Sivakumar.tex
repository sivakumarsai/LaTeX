\documentclass[11pt,leqno]{article}
\headheight=13.6pt

% packages
\usepackage[alphabetic]{amsrefs}
\usepackage{physics}
% margin spacing
\usepackage[top=1in, bottom=1in, left=0.5in, right=0.5in]{geometry}
\usepackage{hanging}
\usepackage{amsfonts, amsmath, amssymb, amsthm}
\usepackage{systeme}
\usepackage[none]{hyphenat}
\usepackage{fancyhdr}
\usepackage{graphicx}
\graphicspath{{./images/}}
\usepackage{float}
\usepackage{siunitx}
\usepackage{esint}
\usepackage{color}
\usepackage{enumitem}
\usepackage{mathrsfs}
\usepackage{hyperref}
\usepackage[noabbrev, capitalise]{cleveref}
\crefformat{equation}{equation~#2#1#3}
\crefformat{lemma}{\textrm{Lemma}~#2#1#3}

% theorems
\theoremstyle{plain}
\newtheorem{lem}{Lemma}
\newtheorem{lemma}[lem]{Lemma}
\newtheorem{thm}[lem]{Theorem}
\newtheorem{theorem}[lem]{Theorem}
\newtheorem{prop}[lem]{Proposition}
\newtheorem{proposition}[lem]{Proposition}
\newtheorem{cor}[lem]{Corollary}
\newtheorem{corollary}[lem]{Corollary}
\newtheorem{conj}[lem]{Conjecture}
\newtheorem{fact}[lem]{Fact}
\newtheorem{form}[lem]{Formula}

\theoremstyle{definition}
\newtheorem{defn}[lem]{Definition}
\newtheorem{definition/}[lem]{Definition}
\newenvironment{definition}
  {\renewcommand{\qedsymbol}{\textdagger}%
   \pushQED{\qed}\begin{definition/}}
  {\popQED\end{definition/}}
\newtheorem{example}[lem]{Example}
\newtheorem{remark}[lem]{Remark}
\newtheorem{exercise}[lem]{Exercise}
\newtheorem{notation}[lem]{Notation}

\numberwithin{equation}{section}
\numberwithin{lem}{section}

% header/footer formatting
\pagestyle{fancy}
\fancyhead{}
\fancyfoot{}
\fancyhead[L]{M 383C}
\fancyhead[C]{HW9}
\fancyhead[R]{Sai Sivakumar}
\fancyfoot[R]{\thepage}
\renewcommand{\headrulewidth}{1pt}

% paragraph indentation/spacing
\setlength{\parindent}{0cm}
\setlength{\parskip}{10pt}
\renewcommand{\baselinestretch}{1.25}

% extra commands defined here
\newcommand{\br}[1]{\left(#1\right)}
\newcommand{\sbr}[1]{\left[#1\right]}
\newcommand{\cbr}[1]{\left\{#1\right\}}
\newcommand{\eq}[1]{\overset{(#1)}{=}}

% bracket notation for inner product
\usepackage{mathtools}

\DeclarePairedDelimiterX{\abr}[1]{\langle}{\rangle}{#1}

\DeclareMathOperator{\Span}{span}
\DeclareMathOperator{\im}{im}
\DeclareMathOperator{\dist}{dist}
\DeclareMathOperator{\diam}{diam}
\DeclareMathOperator{\supp}{supp}
\DeclareMathOperator{\EV}{ev}
\DeclareMathOperator{\co}{co}
\newcommand{\res}[1]{\operatorname*{res}_{#1}}
\DeclareMathOperator{\id}{id}

% set page count index to begin from 1
\setcounter{page}{1}

\begin{document}
\subsection*{4.8 Exercises}
\begin{enumerate}
    \item[13.] Let $X$ be a Banach space and $S,T\in B(X,X)$ with $T$ compact.
    \begin{enumerate}
        \item The operator $TS$ is bounded and linear. Since $S$ is bounded, for any bounded $M\subset X$, $SM$ is bounded. With $T$ compact, $(TS)M = T(SM)$ is precompact, so $TS$ is compact.
        
        The operator $ST$ is bounded and linear. Since $T$ is compact, for any bounded $M\subset X$, $TM$ is precompact. Since $S$ is continuous, $S(\overline{TM})\subset \overline{S(TM)}$ and so $\overline{S(\overline{TM})}\subset \overline{S(TM)}$. Similarly, $S(TM)\subset S(\overline{TM})$ implies $\overline{S(TM)}\subset \overline{S(\overline{TM})}$, from which $\overline{S(TM)}= \overline{S(\overline{TM})}$ follows. Since $\overline{S(\overline{TM})}$ is compact, $S(TM)$ is precompact, so $ST$ is compact.
        \item Let $S$ be invertible and assume that $S+T$ is injective. Then $A= S^{-1}(S+T) = I+S^{-1}T$ is an injective compact perturbation of the identity. Injectivity of $A$ precludes the conclusion of the Fredholm Alternative that if the equation $Ax=y$ has a solution $x$ for given $y$, there are infinitely many solutions. Therefore, we must have for any $y$ a unique solution $x$ to the equation $Ax=y$. It follows that $S+T$ is invertible on all of $X$.
    \end{enumerate}
    \item[19.] Let $H$ be a complex Hilbert space and $T\colon H\to H$ a bounded linear operator. We are told that $T^\ast T$ and $TT^\ast$ are positive, self-adjoint operators (this is not hard to see). The residual spectra of both $T^\ast T$ and $TT^\ast$ are empty, and the spectra of $T^\ast T$ and $TT^\ast$ are contained in $[0,\infty)$. In particular, $\sigma(T^\ast T)\subset [r,R]\subset [0,\infty)$ where $r = \inf_{\norm{x} =1}\abr{T^\ast Tx, x} = \inf_{\norm{x} =1}\abr{Tx, Tx} = \inf_{\norm{x} =1}\norm{Tx}^2$ and $R = \sup_{\norm{x}=1}\abr{T^\ast Tx,x} = \sup_{\norm{x}=1}\norm{Tx}^2 = \norm{T}^2$. I am not sure how to determine $r$. Similarly, $\sigma(TT^\ast)\subset [s,S]\subset [0,\infty)$ where $s = \inf_{\norm{x} =1}\abr{TT^\ast x, x} = \inf_{\norm{x} =1}\abr{T^\ast x, T^\ast x} = \inf_{\norm{x} =1}\norm{T^\ast x}^2$ and $S = \sup_{\norm{x}=1}\abr{TT^\ast x,x} = \sup_{\norm{x}=1}\norm{T^\ast x}^2 = \norm{T^\ast }^2 = \norm{T}^2$. I am not sure how to determine $s$. We have $(T^\ast T)_{-1} = T^\ast T - (-1)I = I+T^\ast T$ and similarly $(TT^\ast)_{-1} = I+TT^\ast$. Because $-1$ is not contained in the spectrum of the operators $T^\ast T$ and $TT^\ast$, we must have that $(T^\ast T)_{-1}$ and $(TT^\ast)_{-1}$ are boundedly invertible.
    \item[20.] Let $H$ be a complex Hilbert space and let $A$ be a bounded linear operator on $H$. Define $\abs{A}$ to be the positive square root of the positive self-adjoint operator $A^\ast A$, written $(A^\ast A)^{1/2}$.
    \begin{enumerate}
        \item 
    \end{enumerate}
    \item[22.] 
\end{enumerate}
\end{document}