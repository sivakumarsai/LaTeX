\documentclass[11pt,leqno]{article}
\headheight=13.6pt

% packages
\usepackage[alphabetic]{amsrefs}
\usepackage{physics}
% margin spacing
\usepackage[top=1in, bottom=1in, left=0.5in, right=0.5in]{geometry}
\usepackage{hanging}
\usepackage{amsfonts, amsmath, amssymb, amsthm}
\usepackage{systeme}
\usepackage[none]{hyphenat}
\usepackage{fancyhdr}
\usepackage{graphicx}
\graphicspath{{./images/}}
\usepackage{float}
\usepackage{siunitx}
\usepackage{esint}
\usepackage{color}
\usepackage{enumitem}
\usepackage{mathrsfs}
\usepackage{hyperref}
\usepackage[noabbrev, capitalise]{cleveref}
\crefformat{equation}{equation~#2#1#3}
\crefformat{lemma}{\textrm{Lemma}~#2#1#3}

% theorems
\theoremstyle{plain}
\newtheorem{lem}{Lemma}
\newtheorem{lemma}[lem]{Lemma}
\newtheorem{thm}[lem]{Theorem}
\newtheorem{theorem}[lem]{Theorem}
\newtheorem{prop}[lem]{Proposition}
\newtheorem{proposition}[lem]{Proposition}
\newtheorem{cor}[lem]{Corollary}
\newtheorem{corollary}[lem]{Corollary}
\newtheorem{conj}[lem]{Conjecture}
\newtheorem{fact}[lem]{Fact}
\newtheorem{form}[lem]{Formula}

\theoremstyle{definition}
\newtheorem{defn}[lem]{Definition}
\newtheorem{definition/}[lem]{Definition}
\newenvironment{definition}
  {\renewcommand{\qedsymbol}{\textdagger}%
   \pushQED{\qed}\begin{definition/}}
  {\popQED\end{definition/}}
\newtheorem{example}[lem]{Example}
\newtheorem{remark}[lem]{Remark}
\newtheorem{exercise}[lem]{Exercise}
\newtheorem{notation}[lem]{Notation}

\numberwithin{equation}{section}
\numberwithin{lem}{section}

% header/footer formatting
\pagestyle{fancy}
\fancyhead{}
\fancyfoot{}
\fancyhead[L]{M 383C}
\fancyhead[C]{HW9}
\fancyhead[R]{Sai Sivakumar}
\fancyfoot[R]{\thepage}
\renewcommand{\headrulewidth}{1pt}

% paragraph indentation/spacing
\setlength{\parindent}{0cm}
\setlength{\parskip}{10pt}
\renewcommand{\baselinestretch}{1.25}

% extra commands defined here
\newcommand{\br}[1]{\left(#1\right)}
\newcommand{\sbr}[1]{\left[#1\right]}
\newcommand{\cbr}[1]{\left\{#1\right\}}
\newcommand{\eq}[1]{\overset{(#1)}{=}}

% bracket notation for inner product
\usepackage{mathtools}

\DeclarePairedDelimiterX{\abr}[1]{\langle}{\rangle}{#1}

\DeclareMathOperator{\Span}{span}
\DeclareMathOperator{\im}{im}
\DeclareMathOperator{\dist}{dist}
\DeclareMathOperator{\diam}{diam}
\DeclareMathOperator{\supp}{supp}
\DeclareMathOperator{\EV}{ev}
\DeclareMathOperator{\co}{co}
\newcommand{\res}[1]{\operatorname*{res}_{#1}}
\DeclareMathOperator{\id}{id}

% set page count index to begin from 1
\setcounter{page}{1}

\begin{document}
\subsection*{4.8 Exercises}
\begin{enumerate}
    \item[13.] Let $X$ be a Banach space and $S,T\in B(X,X)$ with $T$ compact.
    \begin{enumerate}
        \item The operator $TS$ is bounded and linear. Since $S$ is bounded, for any bounded $M\subset X$, $SM$ is bounded. With $T$ compact, $(TS)M = T(SM)$ is precompact, so $TS$ is compact.
        
        The operator $ST$ is bounded and linear. Since $T$ is compact, for any bounded $M\subset X$, $TM$ is precompact. Since $S$ is continuous, $S(\overline{TM})\subset \overline{S(TM)}$ and so $\overline{S(\overline{TM})}\subset \overline{S(TM)}$. Similarly, $S(TM)\subset S(\overline{TM})$ implies $\overline{S(TM)}\subset \overline{S(\overline{TM})}$, from which $\overline{S(TM)}= \overline{S(\overline{TM})}$ follows. Since $\overline{S(\overline{TM})}$ is compact, $S(TM)$ is precompact, so $ST$ is compact.
        \item Let $S$ be invertible and assume that $S+T$ is injective. Then $A= S^{-1}(S+T) = I+S^{-1}T$ is an injective compact perturbation of the identity. Injectivity of $A$ precludes the conclusion of the Fredholm Alternative that if the equation $Ax=y$ has a solution $x$ for given $y$, there are infinitely many solutions. Therefore, we must have for any $y$ a unique solution $x$ to the equation $Ax=y$. It follows that $S+T$ is invertible on all of $X$.
    \end{enumerate}
    \item[19.] Let $H$ be a complex Hilbert space and $T\colon H\to H$ a bounded linear operator. We are told that $T^\ast T$ and $TT^\ast$ are positive, self-adjoint operators (this is not hard to see). The residual spectra of both $T^\ast T$ and $TT^\ast$ are empty, and the spectra of $T^\ast T$ and $TT^\ast$ are contained in $[0,\infty)$. In particular, $\sigma(T^\ast T)\subset [r,R]\subset [0,\infty)$ where $r = \inf_{\norm{x} =1}\abr{T^\ast Tx, x} = \inf_{\norm{x} =1}\abr{Tx, Tx} = \inf_{\norm{x} =1}\norm{Tx}^2$ and $R = \sup_{\norm{x}=1}\abr{T^\ast Tx,x} = \sup_{\norm{x}=1}\norm{Tx}^2 = \norm{T}^2$. I am not sure how to determine $r$. Similarly, $\sigma(TT^\ast)\subset [s,S]\subset [0,\infty)$ where $s = \inf_{\norm{x} =1}\abr{TT^\ast x, x} = \inf_{\norm{x} =1}\abr{T^\ast x, T^\ast x} = \inf_{\norm{x} =1}\norm{T^\ast x}^2$ and $S = \sup_{\norm{x}=1}\abr{TT^\ast x,x} = \sup_{\norm{x}=1}\norm{T^\ast x}^2 = \norm{T^\ast }^2 = \norm{T}^2$. I am not sure how to determine $s$. We have $(T^\ast T)_{-1} = T^\ast T - (-1)I = I+T^\ast T$ and similarly $(TT^\ast)_{-1} = I+TT^\ast$. Because $-1$ is not contained in the spectrum of the operators $T^\ast T$ and $TT^\ast$, we must have that $(T^\ast T)_{-1}$ and $(TT^\ast)_{-1}$ are boundedly invertible.
    \item[20.] Let $H$ be a complex Hilbert space and let $A$ be a bounded linear operator on $H$. Define $\abs{A}$ to be the positive square root of the positive self-adjoint operator $A^\ast A$, written $(A^\ast A)^{1/2}$.
    \begin{enumerate}
        \item (Paraphrasing the argument from class and specializing to $A^\ast A$.) We use the power series for the square root given by $\sqrt{1-z} = \sum_{n=0}^\infty c_nz^n$ converging absolutely for all $\abs{z}\leq 1$ with $c_0 = 1$, $c_n\leq 0$ for $n\geq 1$. In particular, $\abs{A} = (A^\ast A)^{1/2} = \norm{A^\ast A}^{1/2}(A^\ast A/\norm{A^\ast A})^{1/2} = \norm{A^\ast A}^{1/2}\sum_{n=0}^\infty c_n(I-A^\ast A/\norm{A^\ast A})^n$, since $\norm{I-A^\ast A/\norm{A^\ast A}}\leq 1$. We have $(\norm{A^\ast A}^{1/2}\sum_{n=0}^\infty c_n(I-A^\ast A/\norm{A^\ast A})^n)^2 = \norm{A^\ast A}\sum_{n=0}^\infty d_n(I-A^\ast A/\norm{A^\ast A})^n = \norm{A^\ast A}(I-(I-A^\ast A/\norm{A^\ast A})) = A^\ast A$, where $1-z = \sum_{n=0}^\infty d_nz^n$. We have for $n\geq 1$ that $(I-A^\ast A/\norm{A^\ast A})^n\geq 0$ and $\norm{I-A^\ast A/\norm{A^\ast A}}^n\leq 1$. Then for any $x\in H$ with $\norm{x} = 1$, $\abr{\norm{A^\ast A}^{1/2}\sum_{n=0}^N c_n(I-A^\ast A/\norm{A^\ast A})^nx,x} = \norm{A^\ast A}^{1/2}\sum_{n=0}^N c_n\abr{(I-A^\ast A/\norm{A^\ast A})^nx,x}\geq \norm{A^\ast A}^{1/2}(1+\sum_{n=1}^N c_n\abr{(I-A^\ast A/\norm{A^\ast A})^nx,x})\geq \norm{A^\ast A}^{1/2}(1+\sum_{n=1}^N c_n) = 0$. Taking $N$ arbitrarily large, we have that $\abs{A}$ is a positive operator.
        
        It follows that $\abs{A}$ is a positive square root of $A^\ast A$ (hence is well-defined, bounded linear, and self-adjoint). One can also show that $\abs{A}$ is unique.
        \item We have $\norm{\abs{A}x}^2 = \abr{\abs{A}x,\abs{A}x} = \abr{\abs{A}^2x,x} = \abr{A^\ast A x,x} = \abr{Ax,Ax} = \norm{Ax}^2$; that is, $\norm{\abs{A}x} = \norm{Ax}$, for all $x\in H$.
        \item In general, if $T\in B(H,H)$, the orthogonal complement of $\overline{\im T}$ is $\ker T^\ast$. I sketched this argument on the last homework, and it is attached verbatim: For fixed $x\in \ker T^\ast$ and any $y = Ty^\prime \in \im T$, then $\abr{x,y} = \abr{x,Ty^\prime} = \abr{T^\ast x,y^\prime}=\abr{0,y^\prime} = 0$. Thus $\ker T^\ast\subseteq (\overline{\im T})^\perp$. Conversely, let $x\in (\overline{\im T})^\perp$ so that $\abr{x,Ty}= \abr{T^\ast x,y} = 0$ for all $y\in H$. But by taking $y = T^\ast x$, we have $\abr{T^\ast x, T^\ast x} = 0$ so that $T^\ast x =0$; that is, $x\in \ker T^\ast$, providing the reverse inclusion as needed. Specializing to $T = \abs{A}$, we have $H = \overline{\im \abs{A}} \oplus \ker \abs{A}$. 

        Furthermore, since $\norm{\abs{A}x} = \norm{Ax}$, for all $x\in H$, we have that $\ker \abs{A} = \ker A$: That $x\in \ker \abs{A}$ is equivalent to $\norm{\abs{A}x} = 0$, which by (b) is equivalent to, $\norm{Ax} = 0$, which is equivalent to $x\in \ker A$.
        \item Let $U\colon \im \abs{A}\to \im A$ be given by $\abs{A}x\mapsto Ax$. This is well-defined since if $x,y\in H$ with $\abs{A}x = \abs{A}y$, then $x-y\in \ker \abs{A} = \ker A$; that is $Ax = Ay$. The map $U$ is linear since $\abs{A},A$ are linear. By (b), we have that $U$ is an isometry on $\im \abs{A}$. Let $y\in \overline{\im \abs{A}}$ so that there is a sequence $\cbr{\abs{A}x_n}$ converging to $y$. By defining $Uy = \lim_{n\to\infty}U\abs{A}x_n = \lim_{n\to\infty}Ax_n$ we obtain an extension of $U$ to a well-defined isometry $\overline{\im \abs{A}}\to\overline{\im A}$: If $\cbr{\abs{A}x_n},\cbr{\abs{A}x^\prime_n}$ converge to $y\in\overline{\im \abs{A}}$, then $\norm{U(\abs{A}x_n-\abs{A}x^\prime_n)} = \norm{\abs{A}x_n-\abs{A}x^\prime_n}$ can be made arbitrarily small as needed, and if $\cbr{\abs{A}x_n},\cbr{\abs{A}x^\prime_n}$ converge to $y,y^\prime\in \overline{\im \abs{A}}$, then $\norm{U(y-y^\prime)} = \lim_{n\to\infty}\norm{U(\abs{A}x_n-\abs{A}x^\prime_n)} = \lim_{n\to\infty}\norm{\abs{A}x_n-\abs{A}x^\prime_n} = \norm{y-y^\prime}$.
        
        Then extend $U$ to an operator on $H$ by defining it to be the zero operator on $\ker \abs{A}$ (since $H = \overline{\im \abs{A}} \oplus \ker \abs{A}$). Thus $U$ becomes a partial isometry on $H$ (note that $Ux = 0$ for $x\in\overline{\im \abs{A}}$ if and only $x = 0$, so $\ker U = \ker \abs{A}$ and $(\ker U)^\perp = \overline{\im \abs{A}}$). We have that $A = U\abs{A}$ (as $\ker \abs{A}$ is killed by $\abs{A}$).
    \end{enumerate}
    \item[22.] Let $H$ be a separable Hilbert space and let $\cbr{e_n}_{n=1}^\infty$ be a maximal orthonormal set; that is, a Hilbert basis, of $H$. Let $\cbr{\lambda_n}_{n=1}^\infty$ be a bounded sequence of real numbers. For any $x\in H$, define $A\colon H\to H$ by $Ax= \sum_n\lambda_n\abr{x,e_n}e_n$.
    \begin{enumerate}
      \item The quantity $Ax$ for any $x\in H$ exists since $\cbr{\lambda_n}_{n=1}^\infty\subset B_M(0)$: We have $\norm{\sum_{n=1}^N\lambda_n\abr{x,e_n}e_n}^2 = \sum_{n=1}^N \abs{\lambda_n\abr{x,e_n}}^2\leq M^2\sum_{n=1}^N\abs{\abr{x,e_n}}^2$, and by taking $N$ arbitrarily large it follows that $Ax$ exists. The function $A$ is well-defined: If $\sum_n a_ne_n ,\sum_n b_ne_n\in H$ with $\sum_n a_ne_n =\sum_n b_ne_n$, then the Fourier coefficients $a_n = b_n$ for every $n$ by continuity of the inner product. It follows that $\sum_n\lambda_na_ne_n  = \lim_{N\to\infty} \sum_{n=1}^N\lambda_na_ne_n = \lim_{N\to\infty} \sum_{n=1}^N\lambda_nb_ne_n = \sum_n\lambda_nb_ne_n$ as needed. The map $A$ is linear due to linearity of the inner product (and continuity of addition): $A(x+cy) = \sum_n\lambda_n\abr{x+cy,e_n}e_n = \sum_n\lambda_n(\abr{x,e_n} + c\abr{y,e_n})e_n = \sum_n \lambda_n\abr{x,e_n}e_n + c\sum_n\lambda_n\abr{y,e_n}e_n = Ax + cAy$. We have $\sup_{\norm{x} = 1}\norm{Ax} \leq \sup_{\norm{x} = 1}\sqrt{M^2\norm{x}^2} = M$, so $A$ is bounded. Self-adjointness of $A$ follows by the continuity of the inner product and since $\cbr{\lambda_n}\subset\mathbb R$: For any $x,y\in H$, we have $\abr{Ax,y}= \lim_{N\to\infty}\sum_{n=1}^N \lambda_n\abr{x,e_n}\overline{\abr{y,e_n}} = \lim_{N\to\infty}\sum_{n=1}^N \abr{x,e_n}\overline{\lambda_n\abr{y,e_n}}= \abr{x,Ay}$.
      \item The point spectrum of $A$ is given by $\cbr{\lambda_n}_{n=1}^\infty$, since $Ax=\lambda x$ has nonzero solutions if and only if $\lambda\in \cbr{\lambda_n}_{n=1}^\infty$. In particular, if $\lambda\not\in \cbr{\lambda_n}_{n=1}^\infty$ and there exists a nonzero $x$ for which $Ax = \lambda x$, by selecting $m\geq 1$ for which $\abr{x,e_m}\neq 0$, we have by taking inner products that $\lambda_m = \lambda$, impossible. The corresponding eigenspaces to each $\lambda_n$ are evidently the one dimensional subspaces given by $\Span\cbr{e_n}$, again since $\cbr{e_n}$ is an orthonormal basis of $H$.
      \item If each $\lambda_n$ is nonzero and $0\not \in\overline{\cbr{\lambda_n}}$ (that is, $\cbr{\lambda_n}$ is bounded away from zero), then $A$ is surjective. Specifically, for any $x = \sum_n\abr{x,e_n}e_n\in H$, a suitable preimage is $\sum_n\frac{1}{\lambda_n}\abr{x,e_n}e_n$ (which exists since $0\not \in\overline{\cbr{\lambda_n}}$).
      \item Consider the operators $A_N$ given by $A_Nx = \sum_{n=1}^N\lambda_n\abr{x,e_n}e_n$. The $A_N$ are compact operators since the image of any bounded set is a bounded set in a finite dimensional vector space, hence precompact. The $A_N$ form a Cauchy sequence: For $N\geq M$, we have $\norm{A_N-A_M} = \sup_{\norm{x} = 1}\norm{(A_N-A_M)x} = \sup_{\norm{x} = 1}\norm{\sum_{n=M+1}^N\lambda_n\abr{x,e_n}e_n} \leq \sup_{\norm{x} = 1}\max_{M+1\leq n\leq N}\abs{\lambda_n}$, which can be made arbitrarily small as $M$ is taken large. Since $C(H,H)$ is a closed subspace of $B(H,H)$, $\lim_{N\to\infty}A_N$ is compact also. But the pointwise limit of the $A_N$ is $A$, so we must have $\lim_{N\to\infty}A_N = A$ and so $A$ is compact.
    \end{enumerate}
\end{enumerate}
\end{document}