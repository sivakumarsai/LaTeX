\documentclass[11pt,leqno]{article}
\headheight=13.6pt

% packages
\usepackage[alphabetic]{amsrefs}
\usepackage{physics}
% margin spacing
\usepackage[top=1in, bottom=1in, left=0.5in, right=0.5in]{geometry}
\usepackage{hanging}
\usepackage{amsfonts, amsmath, amssymb, amsthm}
\usepackage{systeme}
\usepackage[none]{hyphenat}
\usepackage{fancyhdr}
\usepackage{graphicx}
\graphicspath{{./images/}}
\usepackage{float}
\usepackage{siunitx}
\usepackage{esint}
\usepackage{color}
\usepackage{enumitem}
\usepackage{mathrsfs}
\usepackage{hyperref}
\usepackage[noabbrev, capitalise]{cleveref}
\crefformat{equation}{equation~#2#1#3}
\crefformat{lemma}{\textrm{Lemma}~#2#1#3}

% theorems
\theoremstyle{plain}
\newtheorem{lem}{Lemma}
\newtheorem{lemma}[lem]{Lemma}
\newtheorem{thm}[lem]{Theorem}
\newtheorem{theorem}[lem]{Theorem}
\newtheorem{prop}[lem]{Proposition}
\newtheorem{proposition}[lem]{Proposition}
\newtheorem{cor}[lem]{Corollary}
\newtheorem{corollary}[lem]{Corollary}
\newtheorem{conj}[lem]{Conjecture}
\newtheorem{fact}[lem]{Fact}
\newtheorem{form}[lem]{Formula}

\theoremstyle{definition}
\newtheorem{defn}[lem]{Definition}
\newtheorem{definition/}[lem]{Definition}
\newenvironment{definition}
  {\renewcommand{\qedsymbol}{\textdagger}%
   \pushQED{\qed}\begin{definition/}}
  {\popQED\end{definition/}}
\newtheorem{example}[lem]{Example}
\newtheorem{remark}[lem]{Remark}
\newtheorem{exercise}[lem]{Exercise}
\newtheorem{notation}[lem]{Notation}

\numberwithin{equation}{section}
\numberwithin{lem}{section}

% header/footer formatting
\pagestyle{fancy}
\fancyhead{}
\fancyfoot{}
\fancyhead[L]{M 383C}
\fancyhead[C]{HW12}
\fancyhead[R]{Sai Sivakumar}
\fancyfoot[R]{\thepage}
\renewcommand{\headrulewidth}{1pt}

% paragraph indentation/spacing
\setlength{\parindent}{0cm}
\setlength{\parskip}{10pt}
\renewcommand{\baselinestretch}{1.25}

% extra commands defined here
\newcommand{\br}[1]{\left(#1\right)}
\newcommand{\sbr}[1]{\left[#1\right]}
\newcommand{\cbr}[1]{\left\{#1\right\}}
\newcommand{\eq}[1]{\overset{(#1)}{=}}

% bracket notation for inner product
\usepackage{mathtools}

\DeclarePairedDelimiterX{\abr}[1]{\langle}{\rangle}{#1}

\DeclareMathOperator{\Span}{span}
\DeclareMathOperator{\im}{im}
\DeclareMathOperator{\dist}{dist}
\DeclareMathOperator{\diam}{diam}
\DeclareMathOperator{\supp}{supp}
\DeclareMathOperator{\EV}{ev}
\DeclareMathOperator{\co}{co}
\newcommand{\res}[1]{\operatorname*{res}_{#1}}
\DeclareMathOperator{\id}{id}
\let\PV\relax
\DeclareMathOperator{\PV}{PV}

% set page count index to begin from 1
\setcounter{page}{1}

\begin{document}
\subsection*{5.8 Exercises}
\begin{enumerate}
    \item[14.] \begin{enumerate}
      \item Let $\phi\in \mathcal D(\mathbb R)$ with $\supp \phi\subseteq [-R,R]$, and decompose $\phi$ into its even and odd parts by $\phi = \phi_e + \phi_o$.
      For fixed $n$, we have $\abr{\cos(nx)\PV(1/x), \phi(x)}= \lim_{\varepsilon\to 0^+}\int_{\abs{x}>\varepsilon}\phi(x)\cos(nx)/x \eq{1} \lim_{\varepsilon\to 0^+}\int_{\abs{x}>\varepsilon}\phi_o(x)\allowbreak\cos(nx)/x \eq{2}\int_{-R}^{R}\phi_o(x)\cos(nx)/x$.
      Equality (1) holds since $\cos(nx)/x$ is an odd function, and equality (2) holds since $\phi_o(0) = 0$ (so $\lim_{x\to 0}\phi_0(x)/x = \phi_o^\prime(0)$). Then by the dominated convergence theorem, $\lim_{n\to\infty}\int_{-R}^{R}\phi_o(x)\cos(nx)/x = 0$, so $\lim_{n\to\infty}\cos(nx)\PV(1/x)= 0$.
      \item Let $\phi\in \mathcal D(\mathbb R)$ with $\supp \phi\subseteq [-R,R]$, and decompose $\phi$ into its even and odd parts by $\phi = \phi_e + \phi_o$.
      For fixed $n$, we have $\abr{\sin(nx)\PV(1/x), \phi(x)}= \lim_{\varepsilon\to 0^+}\int_{\abs{x}>\varepsilon}\phi(x)\sin(nx)/x\eq{1} \lim_{\varepsilon\to 0^+}\int_{\abs{x}>\varepsilon}\phi_e(x)\allowbreak\sin(nx)/x\eq{2} \int_{-R}^{R}\phi_e(x)\sin(nx)/x$. Equality (1) holds since $\sin(nx)/x$ is an even function, and equality (2) holds since $\phi_e(0) = \phi(0)$. By Taylor's theorem or the mean value theorem, $\phi_e(x) = \phi(0) + [\phi_e^\prime(0)+r(x)]x$ where $r$ is differentiable with $\lim_{x\to 0}r(x) = 0$. Then $\int_{-R}^{R}\phi_e(x)\sin(nx)/x = \int_{-R}^{R}\phi(0)\sin(nx)/x + \int_{-R}^{R}r(x)\sin(nx)$ and by the dominated convergence theorem, $\lim_{n\to\infty}\int_{-R}^{R}r(x)\sin(nx) = 0$. We have $\lim_{n\to\infty}\int_{-R}^{R}\phi(0)\sin(nx)/x = \phi(0)\lim_{n\to\infty}\int_{-nR}^{nR}\sin(x)/x = \pi\phi(0)$, so $\lim_{n\to\infty}\sin(nx)\PV(1/x)= \pi\delta_0$.
      \item Let $\phi\in \mathcal D(\mathbb R)$. Then $\abr{e^{inx}\PV(1/x),\phi(x)} = \abr{\cos(nx)\PV(1/x),\phi(x)} + i\abr{\sin(nx)\PV(1/x),x}$, and by parts (a) and (b) above, it follows that $\lim_{n\to\infty}e^{inx}\PV(1/x) = i\pi\delta_0$.
    \end{enumerate}
    \item[20.] The functional $\delta_0$ is defined on $X$ by the following: Let $\cbr{\phi_i}$ be a Cauchy sequence from $\mathcal D(-1,1)$ converging to $f\in X$. Let $\delta_0f \coloneqq \lim_{i\to\infty}\phi_i(0)$, which we need to show exists. We have $\phi_i(0) = \phi_i(x) - \int_0^x\phi_i^\prime(y)\dd y$ for any $x\in(-1,1)$ by differentiability of $\phi_i$, so $\abs{\phi_i(0) - \phi_j(0)}\leq \abs{\phi_i(x) - \phi_j(x)} + \int_0^x\big|\phi_i^\prime(y) - \phi_j^\prime(y)\big|\dd y\leq \abs{\phi_i(x) - \phi_j(x)} + \int_{-1}^1\big|\phi_i^\prime(y) - \phi_j^\prime(y)\big|\dd y$. Then integrate both sides with respect to $x$ to obtain $2\abs{\phi_i(0) - \phi_j(0)}\leq\norm{\phi_i-\phi_j} +2\norm*{\phi_i^\prime - \phi_j^\prime}$, which may be made arbitrarily small since $\cbr{\phi_i}$ converges to $f$ in the norm $\norm{\cdot}_1$. Therefore $\delta_0f$ exists for any $f\in X$.
    
    Let $\cbr{\phi_i},\cbr{\psi_j}$ be two Cauchy sequences from $\mathcal D(-1,1)$ converging to $f\in X$. By a similar argument as before, $2\abs{\phi_i(0) - \psi_j(0)}\leq \norm{\phi_i-\psi_j} + 2\norm*{\phi_i^\prime-\psi_j^\prime}$ may be made arbitrarily small, so $\delta_0$ is well defined on the equivalence class of $f$.
    \item[22.] Let $u\in \mathcal D^\prime(\mathbb R^d)$ be compactly supported on $K$ and let $\phi\in\mathcal D(\mathbb R^d)$. Then $u\ast\phi\in C^\infty$ with $\supp(u\ast\phi)\subseteq K+\supp\phi$, from which it follows that $u\ast\phi$ is compactly supported and hence belongs to $\mathcal D(\mathbb R^d)$. Indeed, for $y\not\in K+\supp\phi$, $\supp(\tau_yR\phi) = -\supp\phi + y$, so $\supp(\tau_yR\phi)\cap K = \emptyset$ and $(u\ast\phi)(y) = \abr{u,\tau_yR\phi} = 0$.
    
    Since $u\ast \phi$ is a test function, the convolution $v\ast(u\ast \phi)$ is defined for any $v\in\mathcal D^\prime(\mathbb R^d)$.

    For any $v\in\mathcal D^\prime(\mathbb R^d)$, define $v\ast u$ by $\abr{v\ast u,\phi}\coloneqq \abr{v,\abr{\tau_yu,\phi}}$, and note that $\abr{\tau_yu,\phi} = [R(u\ast R\phi)](y)$ is a test function. Let $\phi,\psi\in\mathcal D(\mathbb R^d)$. Then $\abr{v\ast u,\phi+c\psi} = \abr{v,\abr{\tau_yu,\phi+c\psi}} = \abr{v,\abr{\tau_yu,\phi} + c\abr{\tau_yu,\psi}}$, so $v\ast u$ is linear. Let $\cbr{\phi_i}$ converge to $\phi$. Then $\lim_{i\to\infty}\abr{v\ast u,\phi_i} = \lim_{i\to\infty}\abr{v,\abr{\tau_yu,\phi_i}} = \abr{v,\abr{\tau_yu,\lim_{i\to\infty} \phi_i}} = \abr{v,\abr{\tau_yu,\phi}} = \abr{v\ast u,\phi}$, so $v\ast u\in\mathcal D^\prime(\mathbb R^d)$.
    
    We have $(v\ast u)\ast \phi = \abr{v\ast u,\tau_xR\phi} = \abr{v,\abr{\tau_yu,\tau_xR\phi}} = \abr{v,\abr{u,\tau_{-y}\tau_xR\phi}} = \abr{v,\abr{u,\tau_{x-y}R\phi}} = \abr{v,(u\ast\phi)(x-y)} = \abr{v,\tau_xR(u\ast\phi)} = v\ast(u\ast \phi)$.
    \item[23.] \begin{enumerate}
      \item Let $v = Du$ and solve $Dv = 0$ to find solutions $v = c$ for constant $c$. Corollary 5.26 implies that solutions to $Du = c$ in $\mathcal D^\prime(\mathbb R)$ are of the form $cx+d$ for constants $c,d$. Thus solutions to $D^2u = 0$ are of the form $cx+d$ for constants $c,d$.
      \item Solutions to $xD^2u = 0$ are linear combinations of solutions to the equations $D^2u = 0$ and $D^2u = c\delta_0$ for constant $c$, since $xw = 0$ has solutions of the form $c\delta_0$ for constant $c$. A first primitive of $c\delta_0$ is of the form $c_1 + c_2H$ for constants $c_1,c_2$, and a second primitive is of the form $c_1 + c_2x + c_3xH$ for constants $c_1,c_2,c_3$, since the distributional derivative of $xH$ is indeed $H$: $D(xH) = (Dx)H + xDH = H + x\delta_0 = H$. Therefore solutions to $xD^2u = 0$ are of the form $c_1+c_2x+c_3xH$ for constants $c_1,c_2,c_3$.
    \end{enumerate}
\end{enumerate}
\end{document}