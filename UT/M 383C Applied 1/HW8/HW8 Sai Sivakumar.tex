\documentclass[11pt,leqno]{article}
\headheight=13.6pt

% packages
\usepackage[alphabetic]{amsrefs}
\usepackage{physics}
% margin spacing
\usepackage[top=1in, bottom=1in, left=0.5in, right=0.5in]{geometry}
\usepackage{hanging}
\usepackage{amsfonts, amsmath, amssymb, amsthm}
\usepackage{systeme}
\usepackage[none]{hyphenat}
\usepackage{fancyhdr}
\usepackage{graphicx}
\graphicspath{{./images/}}
\usepackage{float}
\usepackage{siunitx}
\usepackage{esint}
\usepackage{color}
\usepackage{enumitem}
\usepackage{mathrsfs}
\usepackage{hyperref}
\usepackage[noabbrev, capitalise]{cleveref}
\crefformat{equation}{equation~#2#1#3}
\crefformat{lemma}{\textrm{Lemma}~#2#1#3}

% theorems
\theoremstyle{plain}
\newtheorem{lem}{Lemma}
\newtheorem{lemma}[lem]{Lemma}
\newtheorem{thm}[lem]{Theorem}
\newtheorem{theorem}[lem]{Theorem}
\newtheorem{prop}[lem]{Proposition}
\newtheorem{proposition}[lem]{Proposition}
\newtheorem{cor}[lem]{Corollary}
\newtheorem{corollary}[lem]{Corollary}
\newtheorem{conj}[lem]{Conjecture}
\newtheorem{fact}[lem]{Fact}
\newtheorem{form}[lem]{Formula}

\theoremstyle{definition}
\newtheorem{defn}[lem]{Definition}
\newtheorem{definition/}[lem]{Definition}
\newenvironment{definition}
  {\renewcommand{\qedsymbol}{\textdagger}%
   \pushQED{\qed}\begin{definition/}}
  {\popQED\end{definition/}}
\newtheorem{example}[lem]{Example}
\newtheorem{remark}[lem]{Remark}
\newtheorem{exercise}[lem]{Exercise}
\newtheorem{notation}[lem]{Notation}

\numberwithin{equation}{section}
\numberwithin{lem}{section}

% header/footer formatting
\pagestyle{fancy}
\fancyhead{}
\fancyfoot{}
\fancyhead[L]{M 383C}
\fancyhead[C]{HW8}
\fancyhead[R]{Sai Sivakumar}
\fancyfoot[R]{\thepage}
\renewcommand{\headrulewidth}{1pt}

% paragraph indentation/spacing
\setlength{\parindent}{0cm}
\setlength{\parskip}{10pt}
\renewcommand{\baselinestretch}{1.25}

% extra commands defined here
\newcommand{\br}[1]{\left(#1\right)}
\newcommand{\sbr}[1]{\left[#1\right]}
\newcommand{\cbr}[1]{\left\{#1\right\}}
\newcommand{\eq}[1]{\overset{(#1)}{=}}

% bracket notation for inner product
\usepackage{mathtools}

\DeclarePairedDelimiterX{\abr}[1]{\langle}{\rangle}{#1}

\DeclareMathOperator{\Span}{span}
\DeclareMathOperator{\im}{im}
\DeclareMathOperator{\dist}{dist}
\DeclareMathOperator{\diam}{diam}
\DeclareMathOperator{\supp}{supp}
\DeclareMathOperator{\EV}{ev}
\DeclareMathOperator{\co}{co}
\newcommand{\res}[1]{\operatorname*{res}_{#1}}
\DeclareMathOperator{\id}{id}

% set page count index to begin from 1
\setcounter{page}{1}

\begin{document}
\subsection*{3.6 Exercises}
\begin{enumerate}
    \item[21.] Let $\mathcal P$ be the set of polynomials in $C([0,1])$.
    \begin{enumerate}
        \item Since $[0,1]$ is compact, Proposition 2.24 implies that $C([0,1])$ is dense in $L^p([0,1])$ for $1\leq p <\infty$. The Weierstrass theorem states that $\mathcal P$ is dense in $C([0,1])$ (and uniform convergence implies $L^p$ convergence on sets of finite measure). Thus for $f\in L^p([0,1])$, there exists $g\in C([0,1])$ within $\varepsilon$ of $f$, and there exists $h\in \mathcal P$ within $\varepsilon$ of $g$ so that $\norm{f-h}\leq \norm{f-g} + \norm{g-h}\leq 2\varepsilon$. Thus $\mathcal P$ is dense in $L^p([0,1])$.
        \item The Legendre polynomials exist. One can construct them explicitly using the Gram-Schmidt process and induction; we will just use induction. That is, we show that there is a (countably infinite) orthonormal set of polynomials containing polynomials of each degree.
        
        Observe that $\cbr{p_i(x)}_{i=0}^0 = \cbr{1}$ is an orthonormal set of polynomials of degree at most $0$ where $p_i$ has strict degree $i$. Assume that there exists an orthonormal set $\cbr{p_i(x)}_{i=0}^n$ of polynomials of degree at most $n$ where $p_i$ has strict degree $i$. Then $\mathcal P_n = \Span \cbr{p_i(x)}_{i=0}^n$ is a closed subspace of $L^2([0,1])$. Let $\pi f(x)$ be the orthogonal projection of $f(x) = x^{n+1}$ onto $\mathcal P_n$. Then $p_{n+1}(x) = (f(x)-\pi f(x))/\norm{f(x)-\pi f(x)}$ is of unit norm and is orthogonal to all elements of $\mathcal P_n$, hence also $\cbr{p_i(x)}_{i=0}^n$. The polynomial $p_{n+1}(x)$ must also have strict degree $n+1$ since $\pi f(x)$ has degree at most $n$. Thus $\cbr{p_i(x)}_{i=0}^{n+1}$ is an orthonormal set of polynomials of degree at most $n+1$ where $p_i$ has strict degree $i$. Thus the Legendre polynomials $\cbr{p_i(x)}_{i=0}^\infty$ is an orthonormal set of polynomials where $p_i(x)$ has strict degree $i$.

        Theorem 3.22 implies that the Legendre polynomials form an orthonormal basis of $L^p([0,1])$, since $\mathcal P$ is spanned by the Legendre polynomials and $\mathcal P$ is dense in $L^p([0,1])$.
        \item Let $f\in L^p([0,1])$. By Theorem 3.12, an explicit expression for the polynomial $q(x)$ of degree $n$ minimizing $\norm{f-p}$ over all $p(x)$ a polynomial of degree $n$ is the orthogonal projection of $f$ onto $\mathcal P_n$ given by $\sum_{i=0}^n \abr{f,p_n}p_n(x)$. (An orthonormal basis of $\mathcal P_n$ is $\cbr{p_i(x)}_{i=0}^n$, which extends to an orthonormal basis $\cbr{p_i(x)}_{i=0}^\infty$ of $L^p([0,1])$; expanding $f$ into a Fourier series and truncating the series to the first $n$ terms is the orthogonal projection of $f$ onto $\mathcal P_n$.)
    \end{enumerate}
    \item[24.] Let $V$ and $W$ be real Hilbert spaces, $A\in B(V,V)$, and $B\in B(W,V)$. Assume that $A$ is self-adjoint, $B^\ast\in B(V,W)$ is surjective, and that there are constants $\alpha,\gamma>0$ for which 
    \begin{alignat*}{2}
        \abr{Av,v}&\geq \alpha \norm{v}^2 &&\quad\text{for all }v\in \ker B^\ast,\\
        \sup_{v\in V}\frac{\abr{B^\ast v,w}}{\norm{v}}&\geq \gamma\norm{w} &&\quad\text{for all }w\in W.
    \end{alignat*}
    Assume also that the following system 
    \begin{equation*}
        \begin{cases*}
            Au+Bp = f\in V \\
            B^\ast u = g\in W
        \end{cases*}
    \end{equation*}
    has a solution $(u,p)$ with $u\in V$ and $p\in W$.
    \begin{enumerate}
        \item We have $\norm{p}\leq \sup_{v\in V}\abr{B^\ast v, p}/\gamma\norm{v} = \sup_{v\in V}\abr{v, Bp}/\gamma\norm{v} = \sup_{v\in V}\abr{v, f-Au}/\gamma\norm{v} = \norm{f-Au}/\gamma\leq (\norm{f}+ \norm{A}\norm{u})/\gamma$.
        \item Assume that $(u^\prime,p^\prime)$ is another solution, so that $B^\ast(u-u^\prime) = 0$ and $A(u-u^\prime)+B(p-p^\prime) = 0$. Then $0 = \abr{A(u-u^\prime) + B(p-p^\prime), u-u^\prime} = \abr{A(u-u^\prime) , u-u^\prime}  + \abr{p-p^\prime, B^\ast(u-u^\prime)}\geq \alpha\norm{u-u^\prime}^2$, which implies that $\norm{u-u^\prime} = 0$; that is, $u = u^\prime$. We have $\gamma\norm{p-p^\prime}\leq \sup_{v\in V}\abr{B^\ast v, p-p^\prime}/\norm{v} = \sup_{v\in V}\abr{v,B(p-p^\prime)}/\norm{v} = 0$, which implies that $\norm{p-p^\prime} = 0$, so that $p =p^\prime$.
        \item Observe that the orthogonal complement of $\overline{\im B}$ is $\ker B^\ast$. Indeed, for fixed $x\in \ker B^\ast$ and any $y = By^\prime \in \im B$, then $\abr{x,y} = \abr{x,By^\prime} = \abr{B^\ast x,y^\prime}=\abr{0,y^\prime} = 0$. Thus $\ker B^\ast\subseteq (\overline{\im B})^\perp$. Conversely, let $x\in (\overline{\im B})^\perp$ so that $\abr{x,By}= \abr{B^\ast x,y} = 0$ for all $y\in H$. But by taking $y = B^\ast x$, we have $\abr{B^\ast x, B^\ast x} = 0$ so that $B^\ast x =0$; that is, $x\in \ker B^\ast$, providing the reverse inclusion as needed.
        
        With $V = \ker B^\ast \oplus \overline{\im B}$, decompose $u$ into $u = u_1 + u_2$ with $u_1\in \ker B^\ast$ and $u_2\in \overline{\im B}$. We have $\abr{f,u_1} = \abr{Au+Bp,u_1} = \abr{Au_1,u_1} +\abr{Au_2,u_1}+ \abr{Bp,u_1}\geq \alpha\norm{u_1}^2 +\abr{Au_2,u_1}+ \abr{Bp,u_1} = \alpha\norm{u_1}^2 +\abr{Au_2,u_1}+ \abr{p,B^\ast u_1}= \alpha\norm{u_1}^2 +\abr{Au_2,u_1}$. This implies that $\norm{u_1}^2\leq \abs{\abr{f,u_1} - \abr{Au_2,u_1}}/\alpha\leq (\norm{f}\norm{u_1} + \norm{A}\norm{u_2}\norm{u_1})/\alpha$, so that $\norm{u_1}\leq (\norm{f} + \norm{A}\norm{u_2})/\alpha$. By continuity of the inner product, we have $\norm{u_2} = \sup_{v\in \im B}\abr{u_2,v}/\norm{v} = \sup_{w\in W}\abr{u_2,Bw}/\norm{Bw} = \sup_{w\in W}\abr{B^\ast u_2,w}/\norm{Bw}=\sup_{w\in W}\abr{B^\ast u_2 + B^\ast u_1,w}/\norm{Bw} = \sup_{w\in W}\abr{B^\ast u,w}/\norm{Bw} = \sup_{w\in W}\abr{g,w}/\norm{Bw}$, which should be finite, but I cannot immediately see this. Let $C = \sup_{w\in W}\abr{g,w}/\norm{Bw}$. Then $\norm{u} = (\norm{u_1}^2 + \norm{u_2}^2)^{1/2}\leq ((\norm{f} + C\norm{A})^2/\alpha^2 + C^2)^{1/2}$.
    \end{enumerate}
\end{enumerate}
\subsection*{4.8 Exercises}
\begin{enumerate}
    \item[2.] Let $X = C([0,1])$.
    \begin{enumerate}
        \item Fix $g\in X$ and let $T\colon X\to X$ be given by $Tf = gf$. By Corollary 4.3, $\lambda\in \rho(T)$ if and only if $T_\lambda = T-\lambda I$ is injective and surjective on $X$. Since $T_\lambda f= (g-\lambda)f$, we find that $T_\lambda$ is injective whenever $g$ is not identically $\lambda$, and $T_\lambda$ is surjective whenever $0$ is not an element of $\im (g-\lambda)$. Therefore $\lambda$ is in the spectrum of $T$ if and only if $0\in \im(g-\lambda)$ (i.e. the spectrum is the image of $g$).
        \item Let $g$ be defined by $g(x) = (b-a)x + a$. Then $g-\lambda$ attains zero if and only if $\lambda\in [a,b]$. Thus $T\colon X\to X$ given by $Tf = gf$ has spectrum $[a,b]$ (in brief, $\im g = [a,b]$).
    \end{enumerate}
    \item[7.] Let $H$ be a Hilbert space and $P\in B(H,H)$ a projection.
    \begin{enumerate}
        \item Let $P$ be a projection, and $P^\prime = I-P$. That $P$ is an orthogonal projection is to say that $\im P$ is orthogonal to $\im P^\prime$. This implies that for any $x,y\in H$, $\abr{Px,P^\prime y} = 0 = \abr{P^\prime x, Py}$. Conversely, if $\abr{Px,P^\prime y} = \abr{P^\prime x, Py}$ for all $x,y\in H$, then by choosing $P^\prime y$ for $y$ we obtain $\abr{Px,P^\prime y} = \abr{P^\prime x, Py} = \abr{P^\prime x, 0} = 0$ for all $x,y\in H$ (similarly choose $P^\prime x$ in place of $x$ to obtain $0 = \abr{0,P^\prime y} = \abr{Px,P^\prime y} = \abr{P^\prime x, Py}$ for all $x,y\in H$). We have shown that $\im P$ being orthogonal to $\im P^\prime$ is equivalent to $\abr{Px,P^\prime y} = \abr{P^\prime x, Py}$ for all $x,y\in H$.
        
        The statement $\abr{Px,P^\prime y} = \abr{P^\prime x, Py}$ for all $x,y\in H$ is equivalent to $\abr{Px,P^\prime y + Py} = \abr{Px + P^\prime x, Py}$ for all $x,y\in H$; that is, for all $x,y\in H$, $\abr{Px,y} = \abr{x,Py}$. This is equivalent to the statement that for all $x,y\in H$, $\abr{x,(P^\ast-P)y} = 0$, which is equivalent to $P^\ast = P$.
        \item Let $P$ be an orthogonal projection. If $\ker P$ is nontrivial (i.e., if $P\neq \id$), then $0\in \sigma_p(P)$. For $\lambda\neq 0$, the map $P-\lambda I$ is injective if $\lambda\neq 1$ since if $x\not\in\ker P$ and $Px=\lambda x$, then with $P$ idempotent we have $Px = \lambda Px$, which implies that $\lambda = 1$. Thus the point spectrum for $P$ is $\cbr{0,1}$, where $0$ is omitted if $P = \id$.
        
        Let $\lambda\in \sigma(P)\setminus \sigma_p(P)$. Then $\lambda$ is in $\rho(P)$. Indeed, for $P\neq \id$ and $\lambda\neq 0,1$, then $P_\lambda$ is injective as discussed earlier and has a two-sided inverse given by $P_\lambda^{-1}$ given by $(1-\lambda)^{-1}P + \lambda^{-1}(P-I)$ (indeed, $(P-\lambda I)[(1-\lambda)^{-1}P + \lambda^{-1}(P-I)] = I = [(1-\lambda)^{-1}P + \lambda^{-1}(P-I)](P-\lambda I)$). The inverse $(1-\lambda)^{-1}P + \lambda^{-1}(P-I)$ is bounded since $\norm{(1-\lambda)^{-1}P + \lambda^{-1}(P-I)}\leq (1-\lambda)^{-1} + 2\lambda^{-1}<\infty$. (One can repeat the same argument for $P = \id$.) It follows that $\lambda\in \rho(P)$, so $\sigma_c(P)$ is empty.

        By the spectral theorem for self-adjoint operators (Theorem 4.23), $\sigma_r(P)$ is empty.
    \end{enumerate}
\end{enumerate}
\end{document}