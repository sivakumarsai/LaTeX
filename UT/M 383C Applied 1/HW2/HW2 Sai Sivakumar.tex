\documentclass[11pt,leqno]{article}
\headheight=13.6pt

% packages
\usepackage[alphabetic]{amsrefs}
\usepackage{physics}
% margin spacing
\usepackage[top=1in, bottom=1in, left=0.5in, right=0.5in]{geometry}
\usepackage{hanging}
\usepackage{amsfonts, amsmath, amssymb, amsthm}
\usepackage{systeme}
\usepackage[none]{hyphenat}
\usepackage{fancyhdr}
\usepackage{graphicx}
\graphicspath{{./images/}}
\usepackage{float}
\usepackage{siunitx}
\usepackage{esint}
\usepackage{color}
\usepackage{enumitem}
\usepackage{mathrsfs}
\usepackage{hyperref}
\usepackage[noabbrev, capitalise]{cleveref}
\crefformat{equation}{equation~#2#1#3}
\crefformat{lemma}{\textrm{Lemma}~#2#1#3}

% theorems
\theoremstyle{plain}
\newtheorem{lem}{Lemma}
\newtheorem{lemma}[lem]{Lemma}
\newtheorem{thm}[lem]{Theorem}
\newtheorem{theorem}[lem]{Theorem}
\newtheorem{prop}[lem]{Proposition}
\newtheorem{proposition}[lem]{Proposition}
\newtheorem{cor}[lem]{Corollary}
\newtheorem{corollary}[lem]{Corollary}
\newtheorem{conj}[lem]{Conjecture}
\newtheorem{fact}[lem]{Fact}
\newtheorem{form}[lem]{Formula}

\theoremstyle{definition}
\newtheorem{defn}[lem]{Definition}
\newtheorem{definition/}[lem]{Definition}
\newenvironment{definition}
  {\renewcommand{\qedsymbol}{\textdagger}%
   \pushQED{\qed}\begin{definition/}}
  {\popQED\end{definition/}}
\newtheorem{example}[lem]{Example}
\newtheorem{remark}[lem]{Remark}
\newtheorem{exercise}[lem]{Exercise}
\newtheorem{notation}[lem]{Notation}

\numberwithin{equation}{section}
\numberwithin{lem}{section}

% header/footer formatting
\pagestyle{fancy}
\fancyhead{}
\fancyfoot{}
\fancyhead[L]{M 383C}
\fancyhead[C]{HW2}
\fancyhead[R]{Sai Sivakumar}
\fancyfoot[R]{\thepage}
\renewcommand{\headrulewidth}{1pt}

% paragraph indentation/spacing
\setlength{\parindent}{0cm}
\setlength{\parskip}{10pt}
\renewcommand{\baselinestretch}{1.25}

% extra commands defined here
\newcommand{\br}[1]{\left(#1\right)}
\newcommand{\sbr}[1]{\left[#1\right]}
\newcommand{\cbr}[1]{\left\{#1\right\}}
\newcommand{\eq}[1]{\overset{(#1)}{=}}

% bracket notation for inner product
\usepackage{mathtools}

\DeclarePairedDelimiterX{\abr}[1]{\langle}{\rangle}{#1}

\DeclareMathOperator{\Span}{span}
\DeclareMathOperator{\im}{im}
\DeclareMathOperator{\dist}{dist}
\DeclareMathOperator{\diam}{diam}
\DeclareMathOperator{\supp}{supp}
\DeclareMathOperator{\EV}{ev}
\DeclareMathOperator{\co}{co}
\newcommand{\res}[1]{\operatorname*{res}_{#1}}

% set page count index to begin from 1
\setcounter{page}{1}

\begin{document}
\subsection*{2.10 Exercises}
\begin{enumerate}
  \item[10.] Let $X,Y$ be normed linear spaces, let $X$ be finite dimensional with basis $\cbr{e_1,\dots,e_d}$, and let $T\colon X\to Y$ be linear. We show that $T$ is bounded and that the dual space $X^\ast = B(X,\mathbb F)$ is isomorphic and homeomorphic to $\mathbb F^d$.
  
  All norms on $X$ are equivalent since $X$ is finite dimensional. Therefore it suffices to show that $T\colon X\to Y$ is bounded when $X$ is topologized by the norm $\norm{\cdot}_X\coloneqq \norm{c(\cdot)}_{\ell^1}$, where $c\colon X\to \mathbb F^d$ is the coordinate mapping that takes $x = \sum_ix_ie_i\in X$ to $(x_1,\dots,x_d)\in \mathbb F^d$. Let $x = \sum_i x_ie_i\in B_1(0)$, so that $\sum_i \abs{x_i} \leq 1$. Then $\norm{Tx} = \norm{\sum_i x_iTe_i}\leq \sum_i \abs{x_i} \norm{Te_i}\leq \max_{1\leq i\leq d}\norm{Te_i}< \infty$. Therefore $T$ is bounded.

  To show that $X^\ast\cong \mathbb F^d$, it suffices to write down a basis of size $d$ for $X^\ast$. For $1\leq i\leq d$, consider the map $e_i^\ast\colon X\to \mathbb F$ defined by $e_i^\ast(e_j) = \delta_{ij}$, where $\delta_{ij}$ is the Kronecker delta. Extending by linearity, $e_i^\ast$ is a linear map, and hence bounded. Let $f\in X^\ast$ be any functional. Then $f = \sum_i f(e_i)e_i^\ast$, so $\cbr{e_1^\ast,\dots,e_d^\ast}$ spans $X^\ast$. The set $\cbr{e_1^\ast,\dots,e_d^\ast}$ is linearly independent since $\sum_i c_ie_i^\ast$ is the zero functional if and only if $c_i = 0(e_i) = 0$ (indeed, $(\sum_i c_ie_i^\ast)(e_j) = c_j$). Thus $X^\ast$ has dimension $d$ and hence is isomorphic and homeomorphic to $\mathbb F^d$.
  
  \item[12.] Consider the space $(\ell^p, \norm{\cdot}_p)$.
  \begin{enumerate}
    \item The function $\norm{\cdot}_1$ is a norm. For any $x = \cbr{x_i}\in \ell^1$, the quantity $\norm{x}_1 = \sum_i \abs{x_i}$ is clearly nonnegative. The quantity $\norm{x}_1$ is zero if and only if each of the terms $\abs{x_i}$ are zero; that is, if each $x_i$ is zero and so $x = 0$. Let $\lambda\in \mathbb F$. Then $\lambda x = \cbr{\lambda x_i}$ so that $\norm{\lambda x}_1 = \sum_i \abs{\lambda x_i} = \abs{\lambda}\sum_i \abs{x_i} = \abs{\lambda}\sum_i \abs{x_i} = \abs{\lambda}\norm{x}_1$ (we factor $\abs{\lambda}$ out of the sum since the infinite sum is a limit of partial sums). For $x,y\in \ell^p$, $\norm{x+y}_1 = \sum_i\abs{x_i-y_i}\leq \sum_i \abs{x_i} + \sum_i\abs{y_i} = \norm{x}_1 + \norm{y}_1$.
    
    The function $\norm{\cdot}_\infty$ is a norm: For any $x = \cbr{x_i}\in \ell^\infty$, the quantity $\norm{x}_\infty = \sup_i\abs{x_i}$ is clearly nonnegative. The quantity $\norm{x}_\infty$ is zero if and only if $\abs{x_i}$ is zero for each $i$, otherwise the supremum $\sup_i\abs{x_i}$ is positive. Let $\lambda\in \mathbb F$. Then $\lambda x = \cbr{\lambda x_i}$ so that $\norm{\lambda x}_\infty = \sup_i\abs{\lambda x_i} = \sup_i \abs{\lambda}\abs{x_i} = \abs{\lambda}\sup_i\abs{x_i} = \abs{\lambda}\norm{x}_\infty$. For $x,y\in\ell^\infty$, $\norm{x+y}_\infty = \sup_i\abs{x_i+y_i}\leq \sup_i(\abs{x_i} + \abs{y_i}) \leq \sup_i\abs{x_i} + \sup_i\abs{y_i} = \norm{x}_\infty + \norm{y}_\infty$.
    
    The function $\norm{\cdot}_p$ is a norm for $1 < p < \infty$: For any $x = \cbr{x_i}\in \ell^p$, the quantity $\norm{x}_p = (\sum_i \abs{x_i}^p)^{1/p}$ is clearly nonnegative. The quantity $\norm{x}_p$ is zero if and only if each of the terms $\abs{x_i}^p$ are zero; that is, if each $x_i$ is zero and so $x = 0$. Let $\lambda\in \mathbb F$. Then $\lambda x = \cbr{\lambda x_i}$ so that $\norm{\lambda x}_p = (\sum_i \abs{\lambda x_i}^p)^{1/p} = (\abs{\lambda}^p\sum_i \abs{x_i}^p)^{1/p} = \abs{\lambda}(\sum_i \abs{x_i}^p)^{1/p} = \abs{\lambda}\norm{x}_p$ (we factor $\abs{\lambda}$ out of the sum since the infinite sum is a limit of partial sums). To show that $\norm{\cdot}_p$ satisfies the triangle inequality, we use H\"older's inequality in $\ell^p$ (Theorem 2.14 in the course notes). Let $x,y\in \ell^p$, and let $q$ be the conjugate exponent to $p$; that is, $1/p+ 1/q =1$. Then (by applying H\"older's inequality twice) $\norm{x+y}_p^p = \sum_i\abs{x_i + y_i}^p \leq \sum_i \abs{x_i + y_i}^{p-1}(\abs{x_i} + \abs{y_i}) \leq (\sum_i \abs{x_i + y_i}^{(p-1)q})^{1/q}(\norm{x}_p + \norm{y}_p)$. Since $(p-1)q = p$ and $1/q = 1-1/p$, we have $\norm{x+y}_p = \norm{x+y}_p^p (\sum_i \abs{x_i + y_i}^{(p-1)q})^{-1/q}\leq \norm{x}_p + \norm{y}_p$.
    
    \item Let $\cbr{x_n} = \cbr{\cbr{x_{n,i}}}$ be a Cauchy sequence in $\ell^p$ for $1\leq p \leq \infty$ (here $x_{n,i}$ is the $i$-th component of the $n$-th sequence). Then for fixed $j$, $\cbr{x_{n,j}}$ is a Cauchy sequence in $\mathbb F$ for each $n$: Suppose not; that is, suppose that $\abs{x_{n,j} - x_{m,j}}$ tends towards a positive number as $n,m$ grow unboundedly. Then there exists $\varepsilon>0 $ such that $\abs{x_{n,j} - x_{m,j}} > \varepsilon$ for all but finitely many $n,m$. If $1\leq p < \infty$, then it follows that $\norm{x_n - x_m}_p = (\sum_i\abs{x_{n,j} - x_{m,j}}^p)^{1/p}\geq \abs{x_{n,j} - x_{m,j}} > \varepsilon$ for all but finitely many $n,m$, so the sequence $\cbr{x_n}$ could not be Cauchy (the functions $(\cdot)^p,(\cdot)^{1/p}$ are monotonic functions). Similarly, if $p = \infty$, it follows that $\norm{x_n-x_m}\infty = \sup_i\abs{x_{n,i} - x_{m,i}}\geq \abs{x_{n,j} - x_{m,j}} > \varepsilon$ for all but finitely many $n,m$, so the sequence $\cbr{x_n}$ could not be Cauchy. 
    
    Let $x_{0,i} = \lim_{n\to\infty} x_{n,i}$ for each $i$, and let $x_0 = \cbr{x_{0,i}}$. We show that $x_0$ is the limit of $x_n$ as $n$ grows unboundedly: Let $1\leq p < \infty$. Because $\cbr{x_n}$ is a Cauchy sequence, we may make the quantity $\norm{x_n-x_m}_p$ as small as we like for $n,m$ sufficiently large. In particular, for given $\varepsilon>0$, there is a large enough natural number $K$ such that for $n,m\geq K$ and any natural number $N$, $(\sum_{i\leq N}\abs{x_{n,i} - x_{m,i}}^p)^{1/p}\leq \norm{x_n-x_m}_p\leq \varepsilon$. But then by letting $m$ grow unboundedly, we have for all $n>K$ and any natural number $N$ that $(\sum_{i\leq N}\abs{x_{n,i} - x_{0,i}}^p)^{1/p} \leq \varepsilon$. Take $N$ arbitrarily large to deduce that $\norm{x_n-x_0}_p \leq \varepsilon$. 
    
    Let $p = \infty$. We may make the quantity $\norm{x_n-x_m}_\infty$ as small as we like for $n,m$ large enough. Then for any $i$ and any $n,m$ sufficiently large, $\abs{x_{n,i}-x_{m,i}}\leq \norm{x_n-x_m}_\infty\leq \varepsilon$ for some prescribed $\varepsilon>0$. Take $m$ arbitrarily large to find $\abs{x_{n,i}-x_{0,i}}\leq\varepsilon$. Take the supremum over $i$ to obtain $\norm{x_n-x_0}_\infty\leq \varepsilon$ as needed.

    To see that $x_0\in \ell^p$ for $1\leq p \leq \infty$, let $n$ be large enough so that $|\norm{x_n}_p-\norm{x_0}_p|\leq \norm{x_n-x_0}_p\leq 1$. Thus $\norm{x_0}_p\leq 1 + \norm{x_n}_p$. Since convergent sequences are bounded, for all $n$ large enough, $\norm{x_n}_p\leq M$ so that $\norm{x_0}_p$ is finite. It follows that the sequence spaces $\ell^p$ are Banach spaces.
    
    \item The function $\norm{\cdot}_p$ is not a norm for $0<p<1$: Let $e_1 = (1,0,0,\cdots)$ and $e_2 = (0,1,0,\cdots)$ be elements of $\ell^p$. Then $\norm{e_1+e_2}_p = 2^{1/p}>2 = \norm{e_1} + \norm{e_2}$ since $1/p>1$.
    
    \item The unit ball for $\ell^\infty$ is the unit square: The norm extracts the maximum component in absolute value of a vector $(x,y)\in \mathbb R^2$, so $x$ or $y$ is $\pm 1$ when $(x,y)\in B_1^\infty(0)$.

    The unit ball in $(\mathbb R^2,\ell^p)$ for $1\leq p <\infty$ is a symmetric convex body that fits inside of the unit square, because $(\abs{x}^p+\abs{x}^p)^{1/p} = 1$ implies that $\abs{x}$ or $\abs{y}$ is less than or equal to 1, and for any points $(x,y),(r,s)\in B_1^p(0)$, the convex combination $t(x,y) + (1-t)(r,s)$ is contained in the unit ball: $\norm{t(x,y) + (1-t)(r,s)}_p\leq t\norm{(x,y)}_p + (1-t)\norm{(r,s)}_p\leq 1$ (in general, unit balls in normed linear spaces are convex).

    The unit ball in $(\mathbb R^2,\ell^p)$ for $0<p<1$ is a symmetric body that is not convex, due to the triangle inequality failing (see above).

    A fun fact is that the unit balls in $\ell^1$ and $\ell^\infty$ are polytopes that are dual to each other, since the spaces are dual to each other.
  \end{enumerate}\newpage
  
  \item[14.]
  \begin{enumerate}
      \item A Schauder basis for $\ell^p$, $1\leq p<\infty$ is $\cbr{e_i}$, where $e_j = \cbr{\delta_{ij}}$, where $\delta_{ij}$ is the Kronecker delta ($e_j$ is the sequence whose entries are zero except for $1$ in the $j$-th component). For any $x = \cbr{x_i}\in \ell^p$, let $x_n = \sum_{i\leq n} x_ie_i$. Then $\cbr{x_n}$ converges to $x$: The quantity $\norm{x-x_n}_p = (\sum_{i>n}\abs{x_i}^p)^{1/p}$ tends to zero since $x\in \ell^p$; the tails of the sum $\sum_i\abs{x_i}^p = \norm{x}_p^p$ must tend to zero since the sum itself converges.

      \item A normed linear space $X$ with a Schauder basis $\cbr{e_i}$ is separable: The subspace of all finite rational linear combinations $A = \{\sum_{i\leq n}c_ie_i\mid c_i\in \mathbb Q +i\mathbb Q, n\in \mathbb N\}$ (or $c_i\in \mathbb Q$ if $\mathbb F = \mathbb R$) of elements of the Schauder basis $\cbr{e_i}$, is dense in $X$. This space is evidently countable and is dense in $X$. For any $x \in X$ there is a unique sequence of scalars $\cbr{c_i}$ such that $x_n = \sum_{i\leq n} c_ie_i$ converges to $x$. There exists $r_{i,n}\in\mathbb Q + i\mathbb Q$ (or $r_{i,n}\in \mathbb Q$ if $\mathbb F = \mathbb R$) such that for fixed $n$, $y_n = \sum_{i\leq n}r_{i,n}e_i$ is very close to $x_n$ (e.g., pick $r_{i,n}$ within $1/2^{ni}$ of $c_i$). Thus $\cbr{y_n}\subset A$ converges to $x$: we have $\norm{x-y_n}_p\leq \norm{x-x_n}_p + \norm{x_n-y_n}_p$, and both quantities may be made arbitrarily small.
  \end{enumerate}
\end{enumerate}
\end{document}