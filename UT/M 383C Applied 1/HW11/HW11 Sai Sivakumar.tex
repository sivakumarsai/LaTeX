\documentclass[11pt,leqno]{article}
\headheight=13.6pt

% packages
\usepackage[alphabetic]{amsrefs}
\usepackage{physics}
% margin spacing
\usepackage[top=1in, bottom=1in, left=0.5in, right=0.5in]{geometry}
\usepackage{hanging}
\usepackage{amsfonts, amsmath, amssymb, amsthm}
\usepackage{systeme}
\usepackage[none]{hyphenat}
\usepackage{fancyhdr}
\usepackage{graphicx}
\graphicspath{{./images/}}
\usepackage{float}
\usepackage{siunitx}
\usepackage{esint}
\usepackage{color}
\usepackage{enumitem}
\usepackage{mathrsfs}
\usepackage{hyperref}
\usepackage[noabbrev, capitalise]{cleveref}
\crefformat{equation}{equation~#2#1#3}
\crefformat{lemma}{\textrm{Lemma}~#2#1#3}

% theorems
\theoremstyle{plain}
\newtheorem{lem}{Lemma}
\newtheorem{lemma}[lem]{Lemma}
\newtheorem{thm}[lem]{Theorem}
\newtheorem{theorem}[lem]{Theorem}
\newtheorem{prop}[lem]{Proposition}
\newtheorem{proposition}[lem]{Proposition}
\newtheorem{cor}[lem]{Corollary}
\newtheorem{corollary}[lem]{Corollary}
\newtheorem{conj}[lem]{Conjecture}
\newtheorem{fact}[lem]{Fact}
\newtheorem{form}[lem]{Formula}

\theoremstyle{definition}
\newtheorem{defn}[lem]{Definition}
\newtheorem{definition/}[lem]{Definition}
\newenvironment{definition}
  {\renewcommand{\qedsymbol}{\textdagger}%
   \pushQED{\qed}\begin{definition/}}
  {\popQED\end{definition/}}
\newtheorem{example}[lem]{Example}
\newtheorem{remark}[lem]{Remark}
\newtheorem{exercise}[lem]{Exercise}
\newtheorem{notation}[lem]{Notation}

\numberwithin{equation}{section}
\numberwithin{lem}{section}

% header/footer formatting
\pagestyle{fancy}
\fancyhead{}
\fancyfoot{}
\fancyhead[L]{M 383C}
\fancyhead[C]{HW11}
\fancyhead[R]{Sai Sivakumar}
\fancyfoot[R]{\thepage}
\renewcommand{\headrulewidth}{1pt}

% paragraph indentation/spacing
\setlength{\parindent}{0cm}
\setlength{\parskip}{10pt}
\renewcommand{\baselinestretch}{1.25}

% extra commands defined here
\newcommand{\br}[1]{\left(#1\right)}
\newcommand{\sbr}[1]{\left[#1\right]}
\newcommand{\cbr}[1]{\left\{#1\right\}}
\newcommand{\eq}[1]{\overset{(#1)}{=}}

% bracket notation for inner product
\usepackage{mathtools}

\DeclarePairedDelimiterX{\abr}[1]{\langle}{\rangle}{#1}

\DeclareMathOperator{\Span}{span}
\DeclareMathOperator{\im}{im}
\DeclareMathOperator{\dist}{dist}
\DeclareMathOperator{\diam}{diam}
\DeclareMathOperator{\supp}{supp}
\DeclareMathOperator{\EV}{ev}
\DeclareMathOperator{\co}{co}
\newcommand{\res}[1]{\operatorname*{res}_{#1}}
\DeclareMathOperator{\id}{id}

% smileys frownies
\usepackage{wasysym}
\newcommand{\smallhappy}{\raisebox{-.14em}{\smiley}}
\newcommand{\happy}{\raisebox{-.24em}{\resizebox{1.2em}{!}{\smiley}}}
\newcommand{\smallsad}{\raisebox{-.14em}{\frownie}}
\newcommand{\sad}{\raisebox{-.24em}{\resizebox{1.2em}{!}{\frownie}}}
\DeclareMathOperator{\mathhappy}{\!\happy\!}
\DeclareMathOperator{\smallmathhappy}{\!\smallhappy\!}
\DeclareMathOperator{\mathsad}{\!\sad\!}
\DeclareMathOperator{\smallmathsad}{\!\smallsad\!}

% set page count index to begin from 1
\setcounter{page}{1}

\begin{document}
\subsection*{4.8 Exercises}
\begin{enumerate}
    \item[38.] Solve the system $p = x^2w, p^\prime = xw$ by observing that $p^\prime$ is also equal to $2xw + x^2w^\prime$. Then $xw + x^2w^\prime = 0$, which has solutions in $\Span\cbr{1/x}$. The exact solution we choose does not matter since $p = x^2w$, so take $w = 1/x$, $p = x$, and $q =0$. The Euler operator $L = x^2D^2 + xD$ with boundary conditions $u(1) = u(e) = 0$ on $[1,e]$ is the regular SL problem 
    \begin{equation*}
      \begin{cases*}
        x[(xu^\prime)^\prime + 0u] = f\\
        1u(1) + 0u^\prime(1) = 0\\
        1u(e) + 0u^\prime(e) = 0
      \end{cases*}\quad\text{on $[1,e]$.}
    \end{equation*}
    Let $\lambda\in\mathbb R$. Solutions to the Cauchy-Euler equation $(\mathhappy) ~ x^2u^{\prime\prime}+xu^\prime-\lambda u = 0$ satisfying the above boundary conditions are the eigenfunctions of $L$. 

    For $\lambda = 0$, a general solution to $(\mathhappy)$ is of the form $c_1 + c_2\ln x$, but enforcing the boundary conditions yields the zero solution, so $\lambda = 0$ is not an eigenvalue of $L$. For $\lambda = \nu^2 > 0$ where $\nu>0$, a general solution to $(\mathhappy)$ is of the form $c_1x^\nu + c_2x^{-\nu}$. Enforcing the boundary conditions yields $\nu = 0$, so $\lambda = 0$, impossible. So no positive eigenvalues of $L$ exist. For $\lambda = -\nu^2 < 0$ where $\nu>0$, a general solution to $(\mathhappy)$ is of the form $c_1\cos(\nu\ln x) + c_2\sin(\nu\ln x)$. Enforcing the boundary conditions yields $\nu = n\pi$ for integers $n > 0$, so $\lambda = -n^2\pi^2$ for $n > 0$. The corresponding eigenfunctions are of the form $c\sin(\nu\ln x)$ for $c\neq 0$.
\end{enumerate}
\subsection*{5.8 Exercises}
\begin{enumerate}
    \item[6.] Let $T\colon\mathcal D(\mathbb R)\to \mathbb R$.
    \begin{enumerate}
      \item The function $T$ is not linear.
      \item The function $T$ is well-defined since any test function has compact support: $T\phi = \sum_n\phi(n) = \sum_{n=0}^N\phi(n)$ for some $N\geq 1$. It follows that $T$ is linear. If $\cbr{\phi_i}$ converges to $\phi$, then $\cbr{\phi_i}$ converges uniformly to $\phi$ and $\cbr{\phi_i}$ have compact supports lying in a fixed bounded set not depending on $i$. Thus $\cbr{T\phi_i} = \big\{\sum_{n=0}^N\phi_i(n)\big\}$ converges to $T\phi = \sum_{n=0}^N\phi(n)$. Thus $T$ is continuous.
      \item The function $T$ is well defined for the same reason as the function in (b), and hence is linear. Continuity of $T$ follows by repeating the argument in (b), but using instead the fact that $\cbr{D^n\phi_i}$ converges uniformly to $D^n\phi$ for all $n\geq 0$, and that the support of $\supp(D^n\phi_i)=\supp{\phi_i}$ for all $i$.
    \end{enumerate}
    \item[10.] Vacuously, $\emptyset$ is open. It is also obvious that $\mathcal D(\Omega)$ is also open since any tail of any sequence converging to a test function is contained in $\mathcal D(\Omega)$. Let $U_i\subset \mathcal D(\Omega)$ be open for all $i\in I$. Let $\phi\in \bigcup_{i\in I}U_i$ and let $\cbr{\phi_j}$ be a sequence converging to $\phi$. Since $\phi$ belongs to $U_k$ for some $k\in I$, there exists $N$ large enough so that $\phi_n\in U_k\subset \bigcup_{i\in I}U_i$ for all $n\geq N$. Hence $\bigcup_{i\in I}U_i$ is open. Let $U_i\subset \mathcal D(\Omega)$ be open for $1\leq i\leq n$. Let $\phi\in \bigcap_{i=1}^nU_i$, and let $\cbr{\phi_j}$ be a sequence converging to $\phi$. There exists $N_i$ such that $\phi_k\in U_i$ for $k\geq N_i$, for $1\leq i\leq n$. Thus $\phi_k\in \bigcap_{i=1}^nU_i$ for $k\geq \min\cbr{N_i\mid 1\leq i\leq n}$. Therefore we obtain a topology.
    
    A set $V\subset \mathcal D(\Omega)$ is closed if for any $\phi\in V$, any tail of any sequence converging to $\phi$ contains elements not in $V$; that is, if $\cbr{\phi_n}$ converges to $\phi$, then for every $N\geq 1$, $\cbr{\phi_n}_{n\geq N}\not\subset V$. (I am not sure what is being asked of me in this part; it seems that this whole problem has nothing to do with $\mathcal D(\Omega)$.)

    Suppose that $T\colon \mathcal D(\Omega)\to\mathbb F$ is sequentially continuous and that $U\subset \mathbb F$ is open. Then for $\phi\in T^{-1}U$, let $\cbr{\phi_n}$ be a sequence converging to $\phi$. By sequential continuity of $T$, $\cbr{T\phi_n}$ converges to $T\phi\in U$. Therefore there exists $N\geq 1$ so that $T\phi_n\in U$ for $n\geq N$. In other words, $\phi_n\in T^{-1}U$ for $n\geq N$. It follows that $T^{-1}U$ is open.

    Suppose that $T$ is continuous and that $\cbr{\phi_n}$ converges to $\phi\in\mathcal D(\Omega)$. Let $U\subset \mathbb F$ be an open set containing $T\phi$, so that $T^{-1}U$ is an open set containing $\phi$. There exists $N\geq 1$ so that $\phi_n\in T^{-1}U$ for $n\geq N$; by applying $T$ it follows that $T\phi_n\in U$ for $n\geq N$. Since $U$ was arbitrary, we may take $U$ to be open balls of arbitrarily small radius, so that $\abs{T\phi_n - T\phi}$ may be made arbitrarily small. Hence $T$ is sequentially continuous.
    \item[13.] Let $T_h\colon \mathcal D(R)\to\mathcal D(R)$ be the translation operator sending $\phi$ to $\phi(\cdot-h)$. Observe that $\cbr{(\phi-T_h\phi)/h}_{0\leq h}$ converges pointwise to $\phi^\prime$ as $h\to 0$, and that $D^n$ commutes with $T_h$ for any $n\geq 0$. Observe also that $\cbr{(\phi-T_h\phi)/h}_{0\leq h<1}\cup $ has compact supports lying in a compact set $\Omega$ containing $\supp \phi$, independent of $h$. Let $h,k<1$ in what follows. For $i\geq 0$, $\sup_{x\in \Omega}\abs{D^i(\phi(x)-\phi(x-h))/h -D^i(\phi(x)-\phi(x-k))/k} = \sup_{x\in \Omega}\abs{(D^i\phi(x)-D^i\phi(x-h))/h -(D^i\phi(x)-D^i\phi(x-k))/k} = \sup_{x\in \Omega}\abs{D^{i+1}\phi(x_h) -D^{i+1}\phi(x_k)}$ for $x_h$ $\in (x-h,x)$ and $x_k\in (x-k,x)$, by the mean value theorem. A subsequent application of the mean value theorem yields $\sup_{x\in \Omega}\abs{D^i(\phi(x)-\phi(x-h))/h -D^i(\phi(x)-\phi(x-k))/k} = \sup_{x\in \Omega}\abs{x_h-x_k}\abs{D^{i+2}\phi(x_{hk})} \leq \sup_{x\in \Omega}C\abs{x_h-x_k}$, for $x_{hk}\in (\min\{x_h,x_k\},\max\{x_h,x_k\})$. Choose $h,k$ small to make $\sup_{x\in \Omega}C\abs{x_h-x_k}$ uniformly small (we should have $\sup_{x\in \Omega}C\abs{x_h-x_k}\leq \sup_{x\in \Omega}C\max\cbr{h,k}$). Thus for fixed $n\geq 0$, we may make the quantity $\norm{(\phi-T_h\phi)/h - (\phi-T_k\phi)/k} = \sum_{i\leq n}\norm{D^i(\phi-T_h\phi)/h - D^i(\phi-T_k\phi)/k}$ arbitrarily small (e.g. less than $n\varepsilon$ for given $\varepsilon>0$). By completeness of the space of test functions we have $\cbr{(\phi-T_h\phi)/h}_{0\leq h}$ converges to $\phi^\prime$.
\end{enumerate}
\end{document}