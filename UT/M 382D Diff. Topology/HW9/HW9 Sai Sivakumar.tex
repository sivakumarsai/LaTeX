\documentclass[11pt,leqno]{article}
\headheight=13.6pt

% packages
\usepackage[alphabetic]{amsrefs}
\usepackage{physics}
% margin spacing
\usepackage[top=1in, bottom=1in, left=0.5in, right=0.5in]{geometry}
\usepackage{hanging}
\usepackage{amsfonts, amsmath, amssymb, amsthm}
\usepackage{systeme}
\usepackage[none]{hyphenat}
\usepackage{fancyhdr}
\usepackage{graphicx}
\graphicspath{{./images/}}
\usepackage{float}
\usepackage{siunitx}
\usepackage{esint}
\usepackage{color}
\usepackage{enumitem}
\usepackage{mathrsfs}
\usepackage{hyperref}
\usepackage[noabbrev, capitalise]{cleveref}
\crefformat{equation}{equation~#2#1#3}
\crefformat{lemma}{\textrm{Lemma}~#2#1#3}
\usepackage{hanging}

% theorems
\theoremstyle{plain}
\newtheorem{lem}{Lemma}
\newtheorem{lemma}[lem]{Lemma}
\newtheorem{thm}[lem]{Theorem}
\newtheorem{theorem}[lem]{Theorem}
\newtheorem{prop}[lem]{Proposition}
\newtheorem{proposition}[lem]{Proposition}
\newtheorem{cor}[lem]{Corollary}
\newtheorem{corollary}[lem]{Corollary}
\newtheorem{conj}[lem]{Conjecture}
\newtheorem{fact}[lem]{Fact}
\newtheorem{form}[lem]{Formula}

\theoremstyle{definition}
\newtheorem{defn}[lem]{Definition}
\newtheorem{definition/}[lem]{Definition}
\newenvironment{definition}
  {\renewcommand{\qedsymbol}{\textdagger}%
   \pushQED{\qed}\begin{definition/}}
  {\popQED\end{definition/}}
\newtheorem{example}[lem]{Example}
\newtheorem{remark}[lem]{Remark}
\newtheorem{exercise}[lem]{Exercise}
\newtheorem{notation}[lem]{Notation}

\numberwithin{equation}{section}
\numberwithin{lem}{section}

% header/footer formatting
\pagestyle{fancy}
\fancyhead{}
\fancyfoot{}
\fancyhead[L]{M 382D}
\fancyhead[C]{HW9}
\fancyhead[R]{Sai Sivakumar}
\fancyfoot[R]{\thepage}
\renewcommand{\headrulewidth}{1pt}

% paragraph indentation/spacing
\setlength{\parindent}{0cm}
\setlength{\parskip}{10pt}
\renewcommand{\baselinestretch}{1.25}

% extra commands defined here
\newcommand{\br}[1]{\left(#1\right)}
\newcommand{\sbr}[1]{\left[#1\right]}
\newcommand{\cbr}[1]{\left\{#1\right\}}
\newcommand{\eq}[1]{\overset{(#1)}{=}}

% bracket notation for inner product
\usepackage{mathtools}

\DeclarePairedDelimiterX{\abr}[1]{\langle}{\rangle}{#1}

\DeclareMathOperator{\Span}{span}
\DeclareMathOperator{\im}{im}
\newcommand{\res}[1]{\operatorname*{res}_{#1}}
\DeclareMathOperator{\id}{id}
\DeclareMathOperator{\Hom}{Hom}
\DeclareMathOperator{\Adj}{Adj}
\DeclareMathOperator{\Ad}{Ad}
\DeclareMathOperator{\End}{End}
\DeclareMathOperator{\codim}{codim}
\DeclareMathOperator{\Int}{int}
\DeclareMathOperator{\sgn}{sgn}
\newcommand{\GL}{\mathrm{GL}}
\newcommand{\SO}{\mathrm{SO}}
\newcommand{\Mat}{\mathrm{M}}
\newcommand{\Sp}{\mathrm{Sp}}
\newcommand{\AS}{\mathrm{AntiSym}}

% set page count index to begin from 1
\setcounter{page}{1}

\begin{document}
\begin{enumerate}
    \item Let $X$ have a smooth structure given by an atlas $\cbr{(U_\alpha,\varphi_\alpha)}$. On each chart $(U_\alpha,\varphi_\alpha)$, we can trivialize $TX$ via $h_\alpha\colon \pi^{-1}(U_\alpha)\xrightarrow{\cong} U_\alpha\times \mathbb R^n$. For any $\alpha$, define the map $\tilde\varphi_\alpha = (\varphi_\alpha\times \id)h_\alpha \colon \pi^{-1}(U_\alpha) \to \mathbb R^n\times \mathbb R^n$. By definition each $\tilde\varphi_\alpha$ is a homeomorphism. We show that $\cbr{(\pi^{-1}(U_\alpha), \tilde\varphi_\alpha)}$ constitutes an atlas of $TX$. Certainly the collection $\cbr{\pi^{-1}(U_\alpha)}$ covers $TX$. For charts $(\pi^{-1}(U_\alpha), \tilde\varphi_\alpha)$ and $(\pi^{-1}(U_\beta), \tilde\varphi_\beta)$, the transition function $\tilde\varphi_\beta\tilde\varphi_\alpha^{-1} = (\varphi_\beta\times \id)h_\beta h_\alpha^{-1}(\varphi_\alpha^{-1}\times\id)$ acts by $(x,v)\mapsto (\varphi_\alpha^{-1}(x),v)\mapsto [(U_\alpha,\varphi_\alpha),v] = [(U_\beta,\varphi_\beta),D(\varphi_\beta\varphi_\alpha^{-1})_{x}v]\mapsto (\varphi_\alpha^{-1}(x),D(\varphi_\beta\varphi_\alpha^{-1})_{x}v)\mapsto ((\varphi_\beta\varphi_\alpha^{-1})(x),D(\varphi_\beta\varphi_\alpha^{-1})_{x}v)$. Note that $[(U_\alpha,\varphi_\alpha),v] = [(U_\beta,\varphi_\beta),D(\varphi_\beta\varphi_\alpha^{-1})_{x}v]\in T_{\varphi_\alpha^{-1}(x)}X$. Therefore the transition functions are smooth, so $TX$ is a smooth manifold.
    
    The projection map is smooth since for any point $[(U_\alpha,\varphi_\alpha),v]\in T_{\varphi_\alpha^{-1}(x)}X\subset TX$, we have $\varphi_\alpha \pi \tilde\varphi_\alpha^{-1}(x,v) = \varphi_\alpha\pi h_\alpha^{-1}(\varphi_\alpha^{-1}\times \id)(x,v) = \varphi_\alpha\pi h_\alpha^{-1}(\varphi_\alpha^{-1}(x),v) = \varphi_\alpha\pi([(U_\alpha,\varphi_\alpha),v]) = \varphi_\alpha\varphi_\alpha^{-1}(x) = x$. Therefore $\pi$ in coordinates is the projection map $\mathbb R^n\times \mathbb R^n\to \mathbb R^n$ taking $(x,v)$ to $x$, which is smooth, so $\pi\colon TX\to X$ is smooth.
    \item \begin{enumerate}
        \item The bundle transition function $g_{01} = h_1|_{\pi^{-1}(\cdot)}h_0|_{\pi^{-1}(\cdot)}^{-1}$ at a point $[a+bi:1]\in U_0\cap U_1$ is given by $a+bi\mapsto \big(\!\begin{smallmatrix}
            b^2-a^2 & 2ab \\ -2ab & b^2-a^2
        \end{smallmatrix}\!\big)\in \GL_2(\mathbb R)$ for $v\in \mathbb R^2 \cong T_{[z:1]}\mathbb{CP}^1$ since in complex coordinates we have $D(\varphi_1\varphi_0^{-1})_{1/(a+bi)}$ is given by multiplication by $ -(a+bi)^2$, which is realized in real coordinates as multiplication by the $2\times 2$ matrix $\big(\!\begin{smallmatrix}
            b^2-a^2 & 2ab \\ -2ab & b^2-a^2
        \end{smallmatrix}\!\big)$. 
        
        If $a+bi = e^{it}$ (so $[a+bi:1]$ is an element of the equator of $\mathbb{CP}^1$), then $\big(\!\begin{smallmatrix}
            b^2-a^2 & 2ab \\ -2ab & b^2-a^2
        \end{smallmatrix}\!\big) = \big(\!\begin{smallmatrix}
            -\cos(2t) & \sin(2t) \\ -\sin(2t) & -\cos(2t)
        \end{smallmatrix}\!\big)$, which is the element $-2$ in the fundamental group of $\SO_2(\mathbb R)$.
        \item The usual norm on the quaternions $\mathbb H$ is given by $\norm{a+bi+cj+dk}_2 = \sqrt{a^2+b^2+c^2+d^2}$. A lengthy calculation shows that the unit sphere $S^3\subset \mathbb H$ inherits the multiplication and inversion in $\mathbb H$. The group operations are smooth in the usual atlas for $S^3$ since smooth maps of manifolds restrict to smooth maps on submanifolds. So $S^3$ is a Lie group, and the tangent space at any point of $S^3$ may be obtained by translating the tangent space at $1$. Since the tangent space at $1$ has three linearly independent basis vectors, all other tangent spaces have three basis vectors, from which it follows that there are three linearly independent sections of the tangent bundle $TS^3$, so $TS^3$ is trivial (in other words, $S^3$ is parallelizable).
    \end{enumerate}
    \item Suppose $\cbr{v_1,\dots,v_k}$ are linearly dependent so that $v_1 = c_2v_2 + \cdots c_kv_k$ for $c_i$ not all zero. Then $v_1\wedge\cdots \wedge v_k = (c_2v_2 + \cdots c_kv_k)\wedge\cdots v_k = \sum_{i=2}^k c_i(v_i\wedge v_2\wedge \cdots\wedge v_k) = 0$ since $v_i\wedge v_2\wedge \cdots\wedge v_k$ contains repeated vectors for all $2\leq i\leq n$.

    If $\dim V < k$, then the vectors $\cbr{v_1,\dots,v_k}$ appearing in any element $v_1\wedge\cdots \wedge v_k\in \bigwedge^k V$ must be linearly dependent. Hence $\bigwedge^k V = 0$.

    Let $\dim V\geq k$. For any generic $n\times k$ matrix $M$ in $\mathbb R^{n\times k}$, let $m$ be any $k\times k$ minor of $M$. Define the function $\det_m\colon V^k\to \mathbb R$ by sending $(v_1,\dots,v_k)$ to the determinant of the $m$-minor of the matrix $(v_1\,\cdots\,v_k)$ whose columns are the coordinate vectors of $v_1,\dots,v_k$ in a fixed basis for $V$. Since the determinant is alternating and multilinear, there exists a unique linear map $\widetilde{\det}_m\colon \bigwedge^kV\to \mathbb R$ for each $m$ such that $\det_m = \widetilde{\det}_m w$ where $w\colon V^k\to \bigwedge^kV$ sends $(v_1,\dots,v_k)$ to $v_1\wedge\cdots\wedge v_n$. If $v_1\wedge\cdots \wedge v_k = 0$, then $\det_m(v_1,\dots,v_n) = 0$ for all $m$, which implies that the matrix $(v_1\,\cdots\,v_k)$ in any basis of $V$ does not have full rank; that is, $\cbr{v_1,\dots,v_l}$ are linearly dependent.
    \item Let $A$ be given by multiplication by the matrix $(A_{ij})$ in the given bases for $V$ and $W$. Directly computing yields
    \begin{multline*}
        \bigwedge^nA(v_1\wedge \cdots\wedge v_n) = (Av_1)\wedge \cdots \wedge(Av_n) = \Big(\sum_{i_1}A_{i_11}w_{i_1}\Big)\wedge\cdots \wedge \Big(\sum_{i_n}A_{i_nn}w_{i_n}\Big) \\ = \sum_{\mathclap{\substack{i_1,\dots,i_n\\ = 0,\dots,n}}}A_{i_11}\cdots A_{i_nn}(w_{i_1}\wedge\cdots w_{i_n}) = \sum_{\sigma\in S_n}A_{\sigma(1)1}\cdots A_{\sigma(n)n}(w_{\sigma(1)}\wedge\cdots \wedge w_{\sigma(n)}) \\= \Big(\sum_{\sigma\in S_n}\sgn(\sigma)A_{\sigma(1)1}\cdots A_{\sigma(n)n}\Big)(w_{1}\wedge\cdots \wedge w_{n}) = (\det A)(w_{1}\wedge\cdots \wedge w_{n}) 
    \end{multline*}
    as needed.
    \item \begin{enumerate}
        \item A quick summary of the coordinates on $\mathbb{CP}^1 = \cbr{[z_0,z_1]\mid z_0,z_1\in\mathbb C\text{ not both zero}}\cong S^2$: they are given by the usual atlas $\cbr{(U_0,\varphi_0),(U_1,\varphi_1)}$ where $\varphi_0$ sends $[z_0,z_1]$ to $z_1/z_0\in \mathbb C$; this copy of $\mathbb C$ is identified with $\mathbb R^2 = \cbr{(x,y)\mid x,y\in\mathbb R}$. Similarly $\varphi_1$ sends $[z_0,z_1]$ to $z_0/z_1\in \mathbb C$; this copy of $\mathbb C$ is identified with $\mathbb R^2 = \cbr{(a,b)\mid a,b\in\mathbb R}$. Then avoiding $(x,y) = (0,0)$ and $(a,b) = (0,0)$, we have $x = a/(a^2 + b^2)$, $y = -b/(a^2 + b^2)$, $dx = [(b^2 - a^2)/(a^2 + b^2)^2]da - [2ab/(a^2 + b^2)^2]db$, and $dy = [2ab/(a^2 + b^2)^2]da + [(b^2-a^2)/(a^2 + b^2)^2]db$. It follows that 
        \begin{align*}
            \frac{dx}{1 + x^2 + y^2} &= \frac{1}{1 + (a^2 + b^2)^{-1}}\bigg(\frac{b^2 - a^2}{(a^2 + b^2)^2}da - \frac{2ab}{(a^2 + b^2)^2}db\bigg)\\
            &= \frac{1}{a^2 + b^2}\bigg(\frac{b^2 - a^2}{1 + a^2 + b^2}da - \frac{2ab}{1 + a^2 + b^2}\bigg)\quad\text{and}\\
            \frac{dy}{1 + x^2 + y^2} &= \frac{1}{1 + (a^2 + b^2)^{-1}}\bigg(\frac{2ab}{(a^2 + b^2)^2}da + \frac{b^2-a^2}{(a^2 + b^2)^2}db\bigg)\\
            &= \frac{1}{a^2 + b^2}\bigg(\frac{2ab}{1 + a^2 + b^2}da + \frac{b^2 - a^2}{1 + a^2 + b^2}\bigg).
        \end{align*}
        Since we cannot smoothly extend the above expressions to $(a,b) = (0,0)$, the forms $dx/(1 + x^2 + y^2)$ and $dy/(1 + x^2 + y^2)$ do not individually extend to smooth $1$-forms on all of $\mathbb{CP}^1$.
        \item Finite-rank vector bundles are noncanonically self-dual. If there exist $1$-forms $\alpha$ and $\beta$ defined on all of $\mathbb{CP}^1$ such that $\omega = \alpha\wedge \beta$, then $\alpha$ and $\beta$ are linearly independent sections of $T^\ast\mathbb{CP}^1$, so $T^\ast\mathbb{CP}^1\cong T \mathbb{CP}^1\cong TS^2$ is trivial, which is impossible.
    \end{enumerate}
    \item 
\end{enumerate}
\end{document}