\documentclass[11pt,leqno]{article}
\headheight=13.6pt

% packages
\usepackage[alphabetic]{amsrefs}
\usepackage{physics}
% margin spacing
\usepackage[top=1in, bottom=1in, left=0.5in, right=0.5in]{geometry}
\usepackage{hanging}
\usepackage{amsfonts, amsmath, amssymb, amsthm}
\usepackage{systeme}
\usepackage[none]{hyphenat}
\usepackage{fancyhdr}
\usepackage{graphicx}
\graphicspath{{./images/}}
\usepackage{float}
\usepackage{siunitx}
\usepackage{esint}
\usepackage{color}
\usepackage{enumitem}
\usepackage{mathrsfs}
\usepackage{hyperref}
\usepackage[noabbrev, capitalise]{cleveref}
\crefformat{equation}{equation~#2#1#3}
\crefformat{lemma}{\textrm{Lemma}~#2#1#3}

% theorems
\theoremstyle{plain}
\newtheorem{lem}{Lemma}
\newtheorem{lemma}[lem]{Lemma}
\newtheorem{thm}[lem]{Theorem}
\newtheorem{theorem}[lem]{Theorem}
\newtheorem{prop}[lem]{Proposition}
\newtheorem{proposition}[lem]{Proposition}
\newtheorem{cor}[lem]{Corollary}
\newtheorem{corollary}[lem]{Corollary}
\newtheorem{conj}[lem]{Conjecture}
\newtheorem{fact}[lem]{Fact}
\newtheorem{form}[lem]{Formula}

\theoremstyle{definition}
\newtheorem{defn}[lem]{Definition}
\newtheorem{definition/}[lem]{Definition}
\newenvironment{definition}
  {\renewcommand{\qedsymbol}{\textdagger}%
   \pushQED{\qed}\begin{definition/}}
  {\popQED\end{definition/}}
\newtheorem{example}[lem]{Example}
\newtheorem{remark}[lem]{Remark}
\newtheorem{exercise}[lem]{Exercise}
\newtheorem{notation}[lem]{Notation}

\numberwithin{equation}{section}
\numberwithin{lem}{section}

% header/footer formatting
\pagestyle{fancy}
\fancyhead{}
\fancyfoot{}
\fancyhead[L]{M 382D}
\fancyhead[C]{HW4}
\fancyhead[R]{Sai Sivakumar}
\fancyfoot[R]{\thepage}
\renewcommand{\headrulewidth}{1pt}

% paragraph indentation/spacing
\setlength{\parindent}{0cm}
\setlength{\parskip}{10pt}
\renewcommand{\baselinestretch}{1.25}

% extra commands defined here
\newcommand{\br}[1]{\left(#1\right)}
\newcommand{\sbr}[1]{\left[#1\right]}
\newcommand{\cbr}[1]{\left\{#1\right\}}
\newcommand{\eq}[1]{\overset{(#1)}{=}}

% bracket notation for inner product
\usepackage{mathtools}

\DeclarePairedDelimiterX{\abr}[1]{\langle}{\rangle}{#1}

\DeclareMathOperator{\Span}{span}
\DeclareMathOperator{\im}{im}
\newcommand{\res}[1]{\operatorname*{res}_{#1}}
\DeclareMathOperator{\id}{id}
\DeclareMathOperator{\Hom}{Hom}
\DeclareMathOperator{\Adj}{Adj}
\DeclareMathOperator{\Ad}{Ad}
\DeclareMathOperator{\End}{End}
\newcommand{\GL}{\mathrm{GL}}
\newcommand{\Mat}{\mathrm{M}}
\newcommand{\Sp}{\mathrm{Sp}}
\newcommand{\AS}{\mathrm{AntiSym}}

% set page count index to begin from 1
\setcounter{page}{1}

\begin{document}
I worked on this problem set with Sudharshan KV.
\begin{enumerate}
    \item \begin{enumerate}
      \item At each $f(p)\in f(U)$ for $U$ open, there exist open sets $V$ containing $p$ and $W$ containing $f(p)$ such that $f|_V\colon V\to W$ is surjective. By passing to charts, choose $\tilde W\subseteq W$ such that $\tilde V = f^{-1}(\tilde W)$ is contained in $U$. Then $f(\tilde V)$ is open and is contained in $f(U)$, so $f$ is an open map.
      \item The set $f(X)$ is open since $f$ is an open map by part (a). The set $f(X)$ is closed by compactness of $X$, $Y$ being Hausdorff, and continuity of $f$. Since $f(X)$ is nonempty, connectedness of $Y$ implies $f(X) = Y$.
    \end{enumerate}
    \item From the preimage theorem deduce that $f^{-1}(\cbr{y})$ is a discrete set (i.e., a dimension zero submanifold) in $X$; by compactness of $X$ we must have that $f^{-1}(\cbr{y})$ is a finite set of points $\cbr{x_1,\dots,x_n}$. Since $X$ and $Y$ have the same dimension, $Df_{x_i}$ are each invertible. Therefore there exist neighborhoods $\tilde U_i$ containing $x_i$ and $\tilde V_i$ containing $y$ such that $f$ maps $\tilde U_i$ diffeomorphically to $\tilde V_i$. Choose the $\tilde U_i$ to be connected sets, and choose them to be disjoint from each other since $X$ is Hausdorff. With $V^\prime = \bigcap_i \tilde V_i$, the sets $U_i^\prime = (f|_{\tilde U_i})^{-1}(V^\prime)$ also map diffeomorphically onto $V$. 
    
    The set $X\setminus (\bigcup_iU_i^\prime)$ is compact so its image $f(X\setminus (\bigcup_iU_i^\prime))$ is closed in $Y$. The set $f(X\setminus (\bigcup_iU_i^\prime))$ does not contain $y$ since $f^{-1}(\cbr{y})\subseteq \bigcup_iU_i^\prime$. So further refine $V^\prime$ by choosing an open subset $V$ of $V^\prime$ that does not intersect $f(X\setminus (\bigcup_iU_i^\prime))$, and let $U_i = (f|_{\tilde U_i})^{-1}(V)$. Thus $V$ is an open neighborhood of $y$ such that $f^{-1}(V) = \bigsqcup_i U_i$ and $f$ maps each of the $U_i$ diffeomorphically onto $V$.
    \item \begin{enumerate}
      \item Since $S$ is a smooth $k$-dimensional submanifold of $M$, there exists a chart $(U,\phi)$ of $M$ around $p$ such that $\phi$ maps $S\cap U$ homeomorphically onto $\im \phi\cap \iota(\mathbb R^k)$ where $\iota\colon \mathbb R^k\to \mathbb R^n$ is the canonical inclusion $x\mapsto (x,0)$. Let $f = \pi\phi$, where $\pi\colon \im \phi \to \mathbb R^{n-k}$ is the projection $(x,y)\mapsto y$. Therefore $f^{-1}(0) = \phi^{-1}(\cbr{x,0}) = S\cap U$.
    
      The derivative $Df_x$ for $x\in U$ is surjective since $\phi$ is a diffeomorphism and $\pi$ has surjective derivative, so $Df$ is surjective along $S\cap U$ also.

      We check that the inclusion of $S$ in $M$ is an immersion. In coordinates, the inclusion is given by the canonical inclusion $x\mapsto (x,0)$; so its derivative $\big(\!\begin{smallmatrix}
        I \\ 0
      \end{smallmatrix}\!\big)$ is injective.
      \item Consider the inclusion $i\colon S\to \mathbb R^2$ and the identity chart $\id = \pi_x\oplus \pi_y\colon \mathbb R^2\to\mathbb R^2$. Since $i$ is an immersion, its derivative at $p\in S$ in the identity chart given by $(D\pi_x|_S,D\pi_y|_S)_p$ is injective so at least one of $(D\pi_x|_S)_p,(D\pi_y|_S)_p$ is nonzero. Since $S$ is one-dimensional, one of $(D\pi_x|_S)_p,(D\pi_y|_S)_p$ is invertible and hence one of $\pi_x|_S, \pi_y|_S$ is a local diffeomorphism around any $p$.
      \item The square $S = \cbr{(x,y)\mid \max\cbr{\abs{x},\abs{y}} = 1}$ cannot be a smooth one-dimensional submanifold of $\mathbb R^2$ since the projection maps $\pi_x|_S, \pi_y|_S$ are both not locally injective at any of the corners of the square.
    \end{enumerate}
    \item \begin{enumerate}
      \item On $U_0$, the map $f_0\colon U_0\to \mathbb R$ given by $[1:x:y:z]\mapsto z-xy$ is smooth and $H\cap U_0 = f_0^{-1}(0)$. The derivative of $f_0$ along $f^{-1}(0)$ in coordinates is $(-y~-x~1)$, which is surjective. Therefore $H\cap U_0$ is a submanifold of $\mathbb{RP}^3$. Similar arguments on the remaining charts $U_i$ show that $H\cap U_i$ is a submanifold of $\mathbb{RP}^3$ for all $i$, so $H$ is a submanifold of $\mathbb{RP}^3$.
      \item The product of smooth manifolds $X$ and $Y$ is a smooth manifold: The set $X\times Y$ is Hausdorff, is second countable, and charts are given by taking the products of charts on $X$ and $Y$.
      
      The map $\sigma$ in coordinates in the charts $U_0\times U_0$ and $U_0$ is $(x,y)\mapsto (y,x,xy)$, which is smooth and has injective derivative $\Big(\!\begin{smallmatrix}
        0 & 1 \\
        1 & 0 \\
        y & x
      \end{smallmatrix}\!\Big)$. By repeating the argument on $U_i\times U_i\to U_i$, deduce that $\sigma$ is an immersion.
      
      Furthermore, $\sigma$ is injective since, for example, if $[1:y:x:xy] = [1:y^\prime:x^\prime:x^\prime y^\prime]$, then $x = x^\prime$ and $y = y^\prime$. A similar argument can be made on the other charts of $\mathbb{RP}^1\times\mathbb{RP}^1$. The image of $\sigma$ is indeed $H$ (it is clear that $\im \sigma\subseteq H$) since for $[1:z_1:z_2:z_3]\in H$, we may take a preimage to be $([1,z_2],[1,z_1])$, since $z_3 = z_1z_2$. On other charts of $\mathbb{RP}^3$, the argument is similar. Since $\mathbb{RP}^1$ is diffeomorphic to $S^1$, which is compact, and $\sigma$ maps into a Hausdorff space, it follows that $\sigma$ is an embedding. Therefore $\sigma$ is a smooth embedding onto $H$.
      \item Since $\mathbb{RP}^1\times\mathbb{RP}^1$ is diffeomorphic to the torus $S^1\times S^1$, it follows from (b) that $H$ is diffeomorphic to the torus.
    \end{enumerate}
    \item If $A^T\Omega A = \Omega$, then $A$ is invertible since $\Omega$ is invertible, and $A^T\Omega A$ is antisymmetric since $(A^T\Omega A)^T = A^T\Omega^T A = -A^T\Omega A$. Let $\AS_{2n}(\mathbb R)$ denote the antisymmetric matrices of size $2n\times 2n$. Let $f\colon \GL_{2n}(\mathbb R)\to\AS_{2n}(\mathbb R)$ be given by $A\mapsto A^T\Omega A$. Since matrix multiplication is a polynomial in the entries of the matrices, $f$ is smooth. The tangent space of $\GL_{2n}$ is $\Mat_{2n}(\mathbb R)$. The dimension of $\AS_{2n}(\mathbb R)$ is $(2n)(2n-1)/2$ since antisymmetric matrices are determined by the strictly lower-triangular entries. This also implies that the tangent space of $\AS_{2n}(\mathbb R)$ is itself.
    
    The derivative of $f$ is $Df_A(B) = \lim_{t\to 0}\dv{t}(f(A+tB)) = \lim_{t\to 0}\dv{t}(t(A^T\Omega B + B^T\Omega A) + O(t^2)) = A^T\Omega B + B^T\Omega A$. This map is surjective on $\Sp_{2n}(\mathbb R) = f^{-1}(\Omega)$ since above we may take $B = AC$ and for any $C = \big(\!\begin{smallmatrix}
      C_{11} & C_{12} \\
      C_{21} & C_{22}
    \end{smallmatrix}\!\big)$ as a block matrix, we have $Df_A(C) = A^T\Omega AC + C^TA^{T}\Omega A = \Omega C +C^T\Omega = \Big(\!\begin{smallmatrix}
      C_{21} - C_{21}^T & C_{22}+C_{11}^T \\
      -C_{22}-C_{11}^T & -C_{12} + C_{12}^T
    \end{smallmatrix}\!\Big)$. For any antisymmetric matrix $M$, we may choose the block matrices $C_{12},C_{21},C_{11},C_{22}$ so that $Df_A(C) = M$ (since the block matrices of $M$ satisfy $M_{21}^T = -M_{12}$, $M_{12}^T = -M_{21}$, $M_{22}^T = -M_{22}$, and $M_{11}^T = -M_{11}$).

    By the preimage theorem, $\Sp_{2n}(\mathbb R)$ is a $4n^2 - (2n)(2n-1)/2 = (2n)(2n+1)/2$-dimensional submanifold of $\GL_{2n}(\mathbb R)$. The tangent space at the identity is the kernel of $Df_{\id}$ given by all matrices $C$ for which by the above calculation requires $C_{12},C_{21}$ to be symmetric and $C_{11} + C_{22}^T = 0$. By the following problem on Lie groups, it follows that the tangent space at any $A$ is given by multiplying all such matrices $C$ on the left by $A$.
    \item \begin{enumerate}
      \item The multiplication map on $\GL_n(\mathbb R)$ is a polynomial in the entries of the matrices, hence smooth. The inversion map on $\GL_n(\mathbb R)$ is given by $\det^{-1}(-)\cdot\Adj(-)$, where $\Adj(-)$ is the adjugate of a matrix, which is a polynomial in the entries of the matrix. Since $\det$ does not vanish on $\GL_n(\mathbb R)$, the inversion map is smooth. Hence $\GL_n(\mathbb R)$ is a Lie group.
      
      Let $i\colon Z\to Y$ be an inclusion of a submanifold $Z$ in $Y$, and let $f\colon X\to Z$ be a smooth map. If $f$ is smooth, then $if$ is evidently smooth. If $if$ is smooth, then we may choose coordinates around $f(p)$ for any fixed $p\in X$ so that in coordinates $i$ is the canonical inclusion of a lower-dimensional Euclidean space into a higher-dimensional one, since $i$ is an immersion. Then for $\pi $ the left inverse of the canonical inclusion above, $f$ in coordinates is just $\pi i f$, which is smooth since $\pi, if$ are smooth. This proves the following:

      If $G$ is a subgroup and submanifold of $\GL_n(\mathbb R)$, the restriction of the smooth multiplication map is $G\times G\to G$, which is smooth because $G\times G\to \GL_n(\mathbb R)\times \GL_n(\mathbb R)\to \GL_n(\mathbb R)$ is smooth. Similarly the inversion map on $G$ is smooth, so $G$ is a Lie group.
      \item Since $L_g$ is linear, the derivative should be left multiplication by $g$. Indeed, for any curve $\gamma\in T_a\GL_n(\mathbb R)$, the derivative of left multiplication by $g$ is acts by $\gamma\mapsto g\gamma$. By taking the standard charts of $\GL_n(\mathbb R)$ around $a$ and $ga$, we manifest $DL_g$ by left multiplication by $g$.
      \item Consider the inclusion $i\colon G\to \GL_n(\mathbb R)$ and observe that since $G$ is a submanifold, we may identify the tangent space $T_pG$ with a subspace of $T_p\GL_n(\mathbb R)$ for any $p$; this inclusion is the derivative of $i$ at $p$. Then by identifying $T_p\GL_n(\mathbb R)$ with $\Mat_n(\mathbb R)$, we find that the derivative of $L_g$ is given by left multiplication by $g$ on $\Mat_n(\mathbb R)$. The composite $(DL_g)_e Di_e$ agrees with left multiplying by $g$ on $T_eG$ and then applying $Di_g$; that is, as subspaces of $\Mat_n(\mathbb R)$, $T_gG$ is obtained by left multiplying the tangent vectors of $T_eG$ by $g$.
      \item For $R_g$ given by right multiplication by $g$, $C_g = L_gR_{g^{-1}}$ so $(DC_g)_e$ is given by $(DL_g)_{g^{-1}}(DR_{g^{-1}})_e$. By similar reasoning as before, $DR_{g^{-1}}$ is right multiplication by $g^{-1}$. Therefore $DC_g$ is given by conjugation by $g$, restricted to the subspace $\mathfrak g$ of $\Mat_n(\mathbb R)$. The adjoint representation is a linear representation; that is, a homomorphism, since $DC_{gh}$ is conjugation by $gh$, which is given by conjugation by $h$ first and then $g$. So $DC_{gh} = DC_gDC_h$.
      \item Identify the Lie algebra of $\GL_n(\mathbb R)$ with $\Mat_n(\mathbb R)$ so that $D\Ad_e\colon \Mat_n(\mathbb R)\to \End(\Mat_n(\mathbb R))$ is (roughly) computed as $\lim_{t\to 0}\dv{t}(Y\mapsto\Ad(e+tX)Y) = Y\mapsto \lim_{t\to 0}\dv{t}((e+tX)Y(e+tX)^{-1}) = Y\mapsto XY-YX$ since $(e+tX)^{-1} = e-tX + O(t^2)$ (and in spirit we are using the fact that the derivative of a matrix is the derivative of its entries). Hence $D\Ad_e(Y) = XY-YX$, the Lie bracket on matrices. Restricting this calculation to $\mathfrak g$ yields the same derivative of the adjoint representation on $\mathfrak g$ as needed (it should come as restriction of $D\Ad_e$ to $\mathfrak g$).
    \end{enumerate}
\end{enumerate}
\end{document}