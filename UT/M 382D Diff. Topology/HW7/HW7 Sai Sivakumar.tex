\documentclass[11pt,leqno]{article}
\headheight=13.6pt

% packages
\usepackage[alphabetic]{amsrefs}
\usepackage{physics}
% margin spacing
\usepackage[top=1in, bottom=1in, left=0.5in, right=0.5in]{geometry}
\usepackage{hanging}
\usepackage{amsfonts, amsmath, amssymb, amsthm}
\usepackage{systeme}
\usepackage[none]{hyphenat}
\usepackage{fancyhdr}
\usepackage{graphicx}
\graphicspath{{./images/}}
\usepackage{float}
\usepackage{siunitx}
\usepackage{esint}
\usepackage{color}
\usepackage{enumitem}
\usepackage{mathrsfs}
\usepackage{hyperref}
\usepackage[noabbrev, capitalise]{cleveref}
\crefformat{equation}{equation~#2#1#3}
\crefformat{lemma}{\textrm{Lemma}~#2#1#3}
\usepackage{hanging}

% theorems
\theoremstyle{plain}
\newtheorem{lem}{Lemma}
\newtheorem{lemma}[lem]{Lemma}
\newtheorem{thm}[lem]{Theorem}
\newtheorem{theorem}[lem]{Theorem}
\newtheorem{prop}[lem]{Proposition}
\newtheorem{proposition}[lem]{Proposition}
\newtheorem{cor}[lem]{Corollary}
\newtheorem{corollary}[lem]{Corollary}
\newtheorem{conj}[lem]{Conjecture}
\newtheorem{fact}[lem]{Fact}
\newtheorem{form}[lem]{Formula}

\theoremstyle{definition}
\newtheorem{defn}[lem]{Definition}
\newtheorem{definition/}[lem]{Definition}
\newenvironment{definition}
  {\renewcommand{\qedsymbol}{\textdagger}%
   \pushQED{\qed}\begin{definition/}}
  {\popQED\end{definition/}}
\newtheorem{example}[lem]{Example}
\newtheorem{remark}[lem]{Remark}
\newtheorem{exercise}[lem]{Exercise}
\newtheorem{notation}[lem]{Notation}

\numberwithin{equation}{section}
\numberwithin{lem}{section}

% header/footer formatting
\pagestyle{fancy}
\fancyhead{}
\fancyfoot{}
\fancyhead[L]{M 382D}
\fancyhead[C]{HW7}
\fancyhead[R]{Sai Sivakumar}
\fancyfoot[R]{\thepage}
\renewcommand{\headrulewidth}{1pt}

% paragraph indentation/spacing
\setlength{\parindent}{0cm}
\setlength{\parskip}{10pt}
\renewcommand{\baselinestretch}{1.25}

% extra commands defined here
\newcommand{\br}[1]{\left(#1\right)}
\newcommand{\sbr}[1]{\left[#1\right]}
\newcommand{\cbr}[1]{\left\{#1\right\}}
\newcommand{\eq}[1]{\overset{(#1)}{=}}

% bracket notation for inner product
\usepackage{mathtools}

\DeclarePairedDelimiterX{\abr}[1]{\langle}{\rangle}{#1}

\DeclareMathOperator{\Span}{span}
\DeclareMathOperator{\im}{im}
\newcommand{\res}[1]{\operatorname*{res}_{#1}}
\DeclareMathOperator{\id}{id}
\DeclareMathOperator{\Hom}{Hom}
\DeclareMathOperator{\Adj}{Adj}
\DeclareMathOperator{\Ad}{Ad}
\DeclareMathOperator{\End}{End}
\DeclareMathOperator{\codim}{codim}
\DeclareMathOperator{\Int}{int}
\newcommand{\GL}{\mathrm{GL}}
\newcommand{\Mat}{\mathrm{M}}
\newcommand{\Sp}{\mathrm{Sp}}
\newcommand{\AS}{\mathrm{AntiSym}}

% set page count index to begin from 1
\setcounter{page}{1}

\begin{document}
\begin{enumerate}
    \setcounter{enumi}{3}
    \item Let $z\in \mathbb R^n\setminus X$. Let $x\in X$ and let $U$ be an open set containing $X$. Let $V$ be a tubular neighborhood of $X$; e.g., some $\varepsilon$-fattening of $X$, such that $\varepsilon>0$ is smaller than the diameter of $U$. Choose any path from $z$ to any point $v\in V$ and pick any point $\tilde v \in U\cap V$. Since $V$ is path connected, there is a path from $v$ to $\tilde v$, so $z$ is path connected to $\tilde v$ as needed. This procedure works since $X$ is compact and connected.
    \item Let $x\in X$ and choose a small ball $B$ centered at $x$ so that $B\setminus X$ has two connected components (here $X$ has nonempty interior so we can choose the radius small enough so that this happens). From the previous problem, every point in $\mathbb R^n\setminus X$ may be joined to a point in one of the connected components by a curve not intersecting $X$. Therefore there are at most two (path) connected components of $\mathbb R^n\setminus X$ (the class of points path connected to one component of $B\setminus X$ may agree with the class of points path connected to the other component of $B\setminus X$).
    \item The winding number modulo $2$ is homotopy invariant. If $z_1,z_2$ are path connected via a path $\gamma(t)$, then the homotopy $u_t(x) = (x-\gamma(t))/\abs{x-\gamma(t)}$ from $u_0 $ to $u_1$. The winding numbers of $u_0$ and $u_1$ give the winding numbers of $z_0$ and $z_1$ respectively; by homotopy invariance of the winding number we have that the winding numbers of $z_0$ and $z_1$ agree mod $2$.
    \item That the ray $r$ is transversal to $X$ is to say that $X$ is transverse to the preimage of $\vec v$ under $g\colon \mathbb R^n\setminus\cbr{z}\to S^{n-1}$ given by $g(y) = (y-z)/\abs{y-z}$. There is the composition of smooth maps $u\colon X\hookrightarrow \mathbb R^n\setminus\cbr{z}\xrightarrow{g}S^{n-1}$ which by Exercise 1.5.7 implies that $u$ is transverse to $\cbr{\vec v}$ if and only if $X$ is transverse to $g^{-1}(\vec v)$; that is, $r$ is transverse to $X$ if and only if $\vec v$ is a regular value of $u$ (and Sard's theorem says that regular values are of full measure so almost every ray is transverse to $X$).
    
    Exercise 1.5.7: Let $f$ be transverse to $g^{-1}W$ so that $\im(Df_x) + T_{fx}(g^{-1}W) = \im(Df_x) + Dg_{fx}^{-1}(T_{gfx}Z) = T_{fx}Y$. As $g$ is transverse to $W$, $\im(Dg_{fx}) + T_{gfx}W = T_{gfx}Z$. Then $T_{gfx}Z = Dg_{fx}(\im(Df_x) + Dg_{fx}^{-1}(T_{gfx}Z)) + T_{gfx}W = \im(Dg_{fx}Df_x) + T_{gfx}W$, so $g\circ f$ is transverse to $W$. Conversely, assume $g\circ f$ is transverse to $W$ so that $T_{gfx}Z = \im(Dg_{fx}Df_x) + T_{gfx}W$. For $\gamma\in T_yY$ there exists $\alpha\in T_xX$ and $\beta\in T_{gfx}W$ such that $Dg_{fx}\gamma = Dg_{fx}Df_x\alpha + \beta$. That $\beta = Dg_{fx}\gamma - Dg_{fx}Df_x\alpha$ is in $T_{gfx}Z$ is due to transversality of $g\circ f$ and $W$ so taking preimages under $Dg_{fx}$ yields the result.
    \item We saw that the winding number around $z$ is equal to the number of times a transverse ray starting from $z$ intersects $X$ in class. Therefore considering $r$ and another ray with the same direction starting from $z_1$, we have that $W_2(X,z_0) = W_2(X,z_1)+l\pmod 2$.
    \item Let $x\in X$ and some $z_0\in \mathbb R^n\setminus X$ and a ray $r$ starting from $z_0$. Pick any point $z_1$ on a transverse ray $r$ such that the line segment in between $z_0$ and $z_1$ intersects $X$ once. Then the winding numbers of $z_0$ and $z_1$ differ by $1\pmod 2$ by the previous part. Since winding number is a homotopy invariant, the connected components of $z_0$ and $z_1$ are nonempty and distinct; and so we obtain that there are exactly two distinct components of $\mathbb R^n\setminus X$ given by $D_0$ and $D_1$.
    \item Since $X$ is compact, there is a half-space (or a large ball) $C$ completely containing $X$. Taking $z$ in the complement of this set $C$, we immediately obtain that the winding number of $z$ is zero since we may choose a ray that misses $X$ starting at $z$. This shows that $D_1$ is bounded.
    \item Since $D_1$ is bounded and $\mathbb R^n$ is connected, $\overline D_1\cup D_0 = \mathbb R^n$ so that the boundary of $D_1$ must be $X$. As an open subset of $\mathbb R^n$ $D_1$ has natural charts, but the charts on the boundary are obtained by using components of balls centered at points on $X$ contained in $D_1$; and since $X$ is a submanifold we may arrange for these half-balls to be homeomorphic to half-spaces, which gives $\overline D_1$ its smooth structure.
    \item[1.] \begin{enumerate}
        \item Let $\dim X = n$, $\dim Y = m$, $\dim Z = k$. Let $N_x(S;X)$ and $M_x(S;X)$ be complementary subspaces of $T_xX$ to $T_xS$. We also have (from a previous homework problem) that $Df_x(N_x(S;X)) \oplus T_{f(x)}Z = T_{f(x)}Y$ and $Df_x(M_x(S;X)) \oplus T_{f(x)}Z = T_{f(x)}Y$; this is due to injectivity of $Df_x$ on $N_x(S;X)$ and $M_x(S;X)$ and transversality of $f$ and $Z$. Let $\gamma$ be an oriented basis of $T_{f(x)}Z$ and choose bases $\alpha,\tilde\alpha$ of $Df_x(N_x(S;X))$ and $Df_x(M_x(S;X))$ such that we obtain two positive orientations of $T_{f(x)}Y$. The block matrix for the change of basis from $\alpha\cup\gamma$ to $\tilde\alpha\cup\gamma$ is of the form $\big(\!\begin{smallmatrix}
            A & \ast \\ 0 & I
        \end{smallmatrix}\!\big)$; here $A$ has positive determinant. Pull back $\alpha$ and $\tilde\alpha$ along $Df_x$ to bases $\beta$ and $\tilde\beta$ of $N_x(S;X)$ and $M_x(S;X)$, and choose a basis $\delta$ of $T_xS$ so that $\beta\cup\delta$ and $\tilde\beta\cup\delta$ are bases of $T_xX$. It suffices to show that the two orientations given to $T_xX$ in this manner are in the same equivalence class; that is, it is enough to see that the change of basis matrix from $\beta\cup\delta$ to $\tilde\beta\cup\delta$ has positive determinant. Indeed, the change of basis matrix is also of the form $\big(\!\begin{smallmatrix}
            A & \ast \\ 0 & I
        \end{smallmatrix}\!\big)$ by construction (the size of $I$ may differ of course). Since $A$ has positive determinant, the change of basis matrix from $\beta\cup\delta$ to $\tilde\beta\cup\delta$ has positive determinant as needed. Thus the two orientations that are induced on $T_xS$ are in the same equivalence class as well.
        \item If $Z$ is zero-dimensional, then $Df_xN_x(S;X) + 0^\pm = T_{f(x)}Y$ so that $Df_xN_x(S;X)$ has the opposite orientation as $T_{f(x)}Y$ if the orientation of $T_{f(x)}Z$ is negative. Pull back the orientation on $Df_xN_x(S;X)$ to $N_x(S;X)$ via $Df_x$. Then if $S$ is zero-dimensional, the orientation of $T_xS$ is determined by comparing the orientations of $T_xX$ and $N_x(S;X)$; if they agree (disagree), $T_xS$ is positively (negatively) oriented. If $S$ has positive dimension, determine the orientation on $T_xS$ via the orientations on $N_x(S;X)$ and $T_xX$ as usual. 
        
        If $Z$ has positive dimension and $S$ has zero dimension, then choose $N_x(S;X)$ to be $T_xX$. Then $Df_xN_x(S;X)$ has an orientation determined by $T_{f(x)}Z$ and $T_{f(x)}Y$ which we pull back to $N_x(S;X)$. Then if the orientation on $N_x(S;X)$ agrees (disagrees) with the orientation on $T_xX$ then $T_xS$ has positive (negative) orientation.
        \item Let $y\in X\cap Z = S$ and again let $N_y(S;X) = T_yX$ so that an orientation on $T_yS$ is determined by whether $Df_yN_y(S;X) \oplus T_yZ = T_yX \oplus T_yZ = T_yY$ as oriented vector spaces. In particular, if $T_yX \oplus T_yZ = T_yY$ then $T_yS$ must have positive orientation since $N_y(S;X) \oplus T_yS = T_yX \oplus T_yS = T_yX$; if not, then $T_yS$ gets the opposite orientation.
    \end{enumerate}
    \item[2.] Since $X$ and $Z$ are transverse, the orientation of $S = X\cap Z$ is specified by $N_y(S;X) \oplus T_yS = T_yX$ for $y\in S$, from which we obtain also $N_y(S;X)\oplus T_yZ = T_yY$. We also have $N_y(S;Z)\oplus T_yS = T_yZ$ and $N_y(S;Z)\oplus T_yX = T_yY$. Then $N_y(S;X)\oplus [N_y(S;Z)\oplus T_yS] = [N_y(S;X)\oplus  N_y(S;Z)] \oplus T_yS = T_yY$. This equation determines the orientation of $T_yS$, and if we reverse the roles of $X$ and $Z$ we obtain the equation $[N_y(S;Z)\oplus  N_y(S;X)] \oplus T_yS = T_yY$, which determines the orientation of $T_yS$ with $X$ and $Z$ interchanged. To exchange $N_y(S;X)$ and $N_y(S;Z)$ in the equality $[N_y(S;X)\oplus  N_y(S;Z)] \oplus T_yS = T_yY$, we do so on their bases. Each transposition of basis vectors reverses orientation, so we reverse the orientation some number of times equal to the codimension of $S$ in $X$ times the codimension of $S$ in $Z$; equivalently we reverse the orientation $(\codim X) (\codim Z)$ times. So $X\cap Z = (-1)^{(\codim X)(\codim Z)}Z\cap X$.
    \item[3.] Consider $\mathbb {RP}^n$ using the charts $U_i$ for $0\leq i\leq n$ and the usual chart maps $\varphi_i$ sending $[x_0:\cdots:x_n]$ to $(x_i/x_j)_{i\neq j}$. For $j>i$, the determinant of $D(\varphi_j\varphi_i^{-1})_{(x_1,\dots,x_ n)}$ is $(-1)^{j-i}/x_j^{n+1}$. If $n$ is even, then $(-1)^{j-i}/x_j^{n+1}$ always changes sign from $x_j<0$ to $x_j>0$. Since the $(U_i,\varphi_i)$ are connected charts generating the atlas on $\mathbb {RP}^n$, for $n$ even there is no way to orient $\mathbb {RP}^n$.
    
    For $n$ odd, let charts on $\mathbb {RP}^n$ be given by the usual charts $(U_i,\phi_i = \varphi_i)$ for $i$ even and $(U_j, \phi_j = -\varphi_j)$ for $j$ odd, where $-\varphi_j$ is the pointwise negation of $\varphi_j$. Then for $j>i$, the determinant of $D(\phi_j\phi_i^{-1})_{(x_1,\dots,x_n)}$ is $1/x_j^{n+1}$, which is nonnegative for $x_j<0$ and $x_j>0$. (For $x_j = 0$, use different charts.) Hence $\mathbb {RP}^n$ is orientable for odd $n$.

    The Jordan-Brouwer separation theorem implies compact, connected hypersurfaces in $\mathbb R^{n+1}$ are orientable (we discussed in class that oriented manifolds with boundary have oriented boundaries). Let $n\geq 2$ be even. If $\mathbb {RP}^{n}$ occurs as a hypersurface (it is compact and connected) in $\mathbb R^{n+1}$, then it is orientable, which is impossible.
\end{enumerate}
\end{document}