\documentclass[11pt,leqno]{article}
\headheight=13.6pt

% packages
\usepackage[alphabetic]{amsrefs}
\usepackage{physics}
% margin spacing
\usepackage[top=1in, bottom=1in, left=0.5in, right=0.5in]{geometry}
\usepackage{hanging}
\usepackage{amsfonts, amsmath, amssymb, amsthm}
\usepackage{systeme}
\usepackage[none]{hyphenat}
\usepackage{fancyhdr}
\usepackage{graphicx}
\graphicspath{{./images/}}
\usepackage{float}
\usepackage{siunitx}
\usepackage{esint}
\usepackage{color}
\usepackage{enumitem}
\usepackage{mathrsfs}
\usepackage{hyperref}
\usepackage[noabbrev, capitalise]{cleveref}
\crefformat{equation}{equation~#2#1#3}
\crefformat{lemma}{\textrm{Lemma}~#2#1#3}
\usepackage{hanging}

% theorems
\theoremstyle{plain}
\newtheorem{lem}{Lemma}
\newtheorem{lemma}[lem]{Lemma}
\newtheorem{thm}[lem]{Theorem}
\newtheorem{theorem}[lem]{Theorem}
\newtheorem{prop}[lem]{Proposition}
\newtheorem{proposition}[lem]{Proposition}
\newtheorem{cor}[lem]{Corollary}
\newtheorem{corollary}[lem]{Corollary}
\newtheorem{conj}[lem]{Conjecture}
\newtheorem{fact}[lem]{Fact}
\newtheorem{form}[lem]{Formula}

\theoremstyle{definition}
\newtheorem{defn}[lem]{Definition}
\newtheorem{definition/}[lem]{Definition}
\newenvironment{definition}
  {\renewcommand{\qedsymbol}{\textdagger}%
   \pushQED{\qed}\begin{definition/}}
  {\popQED\end{definition/}}
\newtheorem{example}[lem]{Example}
\newtheorem{remark}[lem]{Remark}
\newtheorem{exercise}[lem]{Exercise}
\newtheorem{notation}[lem]{Notation}

\numberwithin{equation}{section}
\numberwithin{lem}{section}

% header/footer formatting
\pagestyle{fancy}
\fancyhead{}
\fancyfoot{}
\fancyhead[L]{M 382D}
\fancyhead[C]{HW12}
\fancyhead[R]{Sai Sivakumar}
\fancyfoot[R]{\thepage}
\renewcommand{\headrulewidth}{1pt}

% paragraph indentation/spacing
\setlength{\parindent}{0cm}
\setlength{\parskip}{10pt}
\renewcommand{\baselinestretch}{1.25}

% extra commands defined here
\newcommand{\br}[1]{\left(#1\right)}
\newcommand{\sbr}[1]{\left[#1\right]}
\newcommand{\cbr}[1]{\left\{#1\right\}}
\newcommand{\eq}[1]{\overset{(#1)}{=}}

% bracket notation for inner product
\usepackage{mathtools}

\DeclarePairedDelimiterX{\abr}[1]{\langle}{\rangle}{#1}

\DeclareMathOperator{\Span}{span}
\DeclareMathOperator{\im}{im}
\newcommand{\res}[1]{\operatorname*{res}_{#1}}
\DeclareMathOperator{\id}{id}
\DeclareMathOperator{\Hom}{Hom}
\DeclareMathOperator{\Adj}{Adj}
\DeclareMathOperator{\Ad}{Ad}
\DeclareMathOperator{\End}{End}
\DeclareMathOperator{\codim}{codim}
\DeclareMathOperator{\Int}{int}
\DeclareMathOperator{\sgn}{sgn}
\newcommand{\GL}{\mathrm{GL}}
\newcommand{\SO}{\mathrm{SO}}
\newcommand{\Mat}{\mathrm{M}}
\newcommand{\Sp}{\mathrm{Sp}}
\newcommand{\AS}{\mathrm{AntiSym}}

% set page count index to begin from 1
\setcounter{page}{1}

\begin{document}
\begin{enumerate}
    \item If $\omega$ is exact, then $\int_{T^2}\omega = \int_{T^2} d\tilde \omega = 0$ by Stokes' theorem (the torus has no boundary). On the other hand, if $\int_{T^2}\omega = \int_0^{2\pi}\int_0^{2\pi}h(x,y)\,dxdy = 0$, then the function $I(y) = \int_0^{2\pi}h(x,y)\,dx$ satisfies $\int_0^{2\pi}[h(x,y)-\frac{1}{2\pi}I(y)]\,dx = \int_0^{2\pi}h(x,y)\,dx-I(y) = 0$ for any fixed $y$. Let $h^\prime(x,y) = h(x,y) - \frac{1}{2\pi}I(y)$. It follows that $h^\prime$ has an antiderivative in $x$; that is, $h^\prime(x,y) = \pdv{H}{x}\/(x,y) = -\pdv{(-H)}{x}\/(x,y) + \pdv{0}{y}\/(x,y)$ for all $x,y$. So $h^\prime dx\wedge dy$ is exact. Since $\int_0^{2\pi}I(y)\,dy = \int_0^{2\pi}\int_0^{2\pi}h(x,y)\,dx dy = 0$, $I(y)$ has the antiderivative $I(y) = \pdv{y}\int_0^yI(t)\,dt$ and so $hdx\wedge dy$ is exact as well. It follows that any closed form $hdx\wedge dy$ is cohomologous to $(\int_0^{2\pi}\int_0^{2\pi}h(x,y)\,dx dy)dx\wedge dy$; hence $H^2(T^2) = \mathbb R$.
    \item \begin{enumerate}
      \item Since $\mathbb R^3$ is homotopic to a point, $H^i(\mathbb R^3)$ is $\mathbb R$ if $i = 0$, and is $0$ otherwise. It follows that (1) every closed $1$-form is exact, and (2) every closed $2$-form is exact.
      \item The $1$-form $\omega = (\frac{-y}{x^2 + y^2}) dx + (\frac{x}{x^2 + y^2} ) dy$ defined on $\mathbb R^3\setminus \cbr{(0,0,z)\mid z\in\mathbb R}$ is closed but not exact by a similar argument as in HW10, problem 9. The $2$-form $\omega = (x^2+y^2+z^2)^{-3/2}[xdy\wedge dz + ydz\wedge dx\wedge zdx\wedge dy]$ defined on $\mathbb R^3\setminus \cbr{(0,0,0)}$ was closed but not exact as we saw in HW11, problem 2.
      \item The $1$-form $dx$ (periodic on $\mathbb R^3$) is closed since it is a constant $1$-form, but not exact since $\int_0^{2\pi}1\,dx \neq 0$. Similarly, the $2$-form $dx\wedge dy$ is closed but not exact since $\int_0^{2\pi}\int_0^{2\pi}1\,dxdy \neq 0$; via a similar argument as in problem 1.
    \end{enumerate}
    \item We show that any closed $1$-form $\omega$ is exact. Let $p\in X$ and $f\colon X\to\mathbb R$ be given by $f(q) = \int_0^1 \gamma^\ast\omega$ for any $\gamma\colon [0,1]\to X$ with $\gamma(0) = p$ and $\gamma(1) = q$. That $f$ is well defined is due to $X$ being simply-connected and the homotopy invariance of integration of closed forms. One may choose $t$ close to $1$ so that $f(q) = \int_0^t\gamma^\ast\omega + \int_t^1\gamma^\ast \omega$ with $\gamma(t)$ in a coordinate chart of $q$. So without loss of generality we only consider $q$ close to $p$; that is, $p,q$ belong to a common coordinate chart. Choose coordinates $x_1,\dots,x_n$ around $p,q$ and $\gamma$ a path which in coordinates is a path along coordinate axes so that $f$ is given by a path integral in $\mathbb R^n$ along the coordinate axes from the points $p$ and $q$ in coordinates. The fundamental theorem of calculus implies that $df = \omega$ in these coordinates and hence also in general.
    % Thus for any path $\gamma\colon [0,1]\to X$ with $\gamma(0) = p$ and $\gamma(1) = q$, $\gamma^\ast f(x) = \int_0^x\gamma^\ast\omega$, so $d\gamma^\ast f = [\pdv{x}\int_0^x\gamma^\ast\omega]dx = \gamma^\ast\omega$. Since the exterior derivative commutes with pullback, $\gamma^\ast(df-\omega) = 0$; since pullback is injective (the pullback of a nonzero form is nonzero), $df = \omega$. 
    \item \begin{enumerate}
      \item Write $f\sim g$ if $f$ is chain homotopic to $g$. Evidently $\sim$ is reflexive and symmetric; choose the chain homotopy $H$ to be the identity in the first case and negate the provided chain homotopy in the second case. If $f\sim g$ and $g\sim h$ with $f-g = dH-Hd$ and $g-h = dG-Gd$, then $f-h = d(H+G)-(H+G)d$.
      \item Let $h\colon B\to C$ and $j\colon D\to A$ be chain maps. If $f\sim g$ with $f,g\colon A\to B$, then $hf,hg\colon A\to C$ with $hf-hg = h(dH-Hd) = d(hH)-(hH)d$ and $fj,gj\colon D\to B$ with $fj-gj = (dH-Hd)j = d(Hj)-(Hj)d$ since $h,j$ are chain maps.
      \item Let $g,g^\prime$ be chain homotopy inverses of $f$ so that $gf-\id_A = dH-Hd$, $fg-\id_B = dG-Gd$, $g^\prime f-\id_A = dH^\prime -H^\prime d$, $fg^\prime -\id_B = dG^\prime -G^\prime d$. Then $gfg^\prime - g^\prime = d(Hg^\prime)-(Hg^\prime)d$ and $gfg^\prime - g = d(gG)-(gG)d$, so that $g-g^\prime = d(Hg^\prime-gG)-(Hg^\prime-gG)d$ as needed.
      \item Write $A\sim B$ if $A$ and $B$ are homotopy equivalent. Evidently $\sim$ is reflexive and symmetric; choose $f,g$ to be $\id_A$ in the first case and swap $f,g$ in the second case. If $A\sim B$ and $B\sim C$, then there exist $f\colon A\to B$, $g\colon B\to A$ with $gf-\id_A = dH-Hd$ and $fg-\id_B = dG-Gd$ and $h\colon B\to C$, $j\colon C\to B$ with $jh-\id_B = dL-Ld$ and $hj-\id_C = dM-Md$. Then $hfgj - hj = d(hGj) - (hGj)d$ so that $hfgj-\id_C = d(hGj+N) - (hGj+N)d$ and similarly $gjhf - \id_A = d(gLf+H) - (gLf+H)d$ as needed.
    \end{enumerate}
\end{enumerate}
\end{document}