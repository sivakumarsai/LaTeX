\documentclass[11pt,leqno]{article}
\headheight=13.6pt

% packages
\usepackage[alphabetic]{amsrefs}
\usepackage{physics}
% margin spacing
\usepackage[top=1in, bottom=1in, left=0.5in, right=0.5in]{geometry}
\usepackage{hanging}
\usepackage{amsfonts, amsmath, amssymb, amsthm}
\usepackage{systeme}
\usepackage[none]{hyphenat}
\usepackage{fancyhdr}
\usepackage{graphicx}
\graphicspath{{./images/}}
\usepackage{float}
\usepackage{siunitx}
\usepackage{esint}
\usepackage{color}
\usepackage{enumitem}
\usepackage{mathrsfs}
\usepackage{hyperref}
\usepackage[noabbrev, capitalise]{cleveref}
\crefformat{equation}{equation~#2#1#3}
\crefformat{lemma}{\textrm{Lemma}~#2#1#3}

% theorems
\theoremstyle{plain}
\newtheorem{lem}{Lemma}
\newtheorem{lemma}[lem]{Lemma}
\newtheorem{thm}[lem]{Theorem}
\newtheorem{theorem}[lem]{Theorem}
\newtheorem{prop}[lem]{Proposition}
\newtheorem{proposition}[lem]{Proposition}
\newtheorem{cor}[lem]{Corollary}
\newtheorem{corollary}[lem]{Corollary}
\newtheorem{conj}[lem]{Conjecture}
\newtheorem{fact}[lem]{Fact}
\newtheorem{form}[lem]{Formula}

\theoremstyle{definition}
\newtheorem{defn}[lem]{Definition}
\newtheorem{definition/}[lem]{Definition}
\newenvironment{definition}
  {\renewcommand{\qedsymbol}{\textdagger}%
   \pushQED{\qed}\begin{definition/}}
  {\popQED\end{definition/}}
\newtheorem{example}[lem]{Example}
\newtheorem{remark}[lem]{Remark}
\newtheorem{exercise}[lem]{Exercise}
\newtheorem{notation}[lem]{Notation}

\numberwithin{equation}{section}
\numberwithin{lem}{section}

% header/footer formatting
\pagestyle{fancy}
\fancyhead{}
\fancyfoot{}
\fancyhead[L]{M 382D}
\fancyhead[C]{HW2}
\fancyhead[R]{Sai Sivakumar}
\fancyfoot[R]{\thepage}
\renewcommand{\headrulewidth}{1pt}

% paragraph indentation/spacing
\setlength{\parindent}{0cm}
\setlength{\parskip}{10pt}
\renewcommand{\baselinestretch}{1.25}

% extra commands defined here
\newcommand{\br}[1]{\left(#1\right)}
\newcommand{\sbr}[1]{\left[#1\right]}
\newcommand{\cbr}[1]{\left\{#1\right\}}
\newcommand{\eq}[1]{\overset{(#1)}{=}}

% bracket notation for inner product
\usepackage{mathtools}

\DeclarePairedDelimiterX{\abr}[1]{\langle}{\rangle}{#1}

\DeclareMathOperator{\Span}{span}
\DeclareMathOperator{\im}{im}
\newcommand{\res}[1]{\operatorname*{res}_{#1}}
\DeclareMathOperator{\id}{id}
\DeclareMathOperator{\Hom}{Hom}
\newcommand{\Gr}{\mathrm{Gr}}

% set page count index to begin from 1
\setcounter{page}{1}

\begin{document}
I worked on this problem set with Michael Han and Sudharshan KV.
\begin{enumerate}
    \item \begin{enumerate}
        \item Let $\mathcal A_1,\mathcal A_2$ be atlases on $X$. Recall that if $\mathcal A$ is an atlas on $X$, then $\mathcal A^{\max}$ is given by the set of all charts on $X$ smoothly compatible with the charts in $\mathcal A$. Suppose that each chart in $\mathcal A_1$ is compatible with every chart in $\mathcal A_2$, and vice-versa. Since every chart in $\mathcal A_2$ is compatible with $\mathcal A_1$, then $\mathcal A_2 \subseteq \mathcal A_1^{\max}$. We showed in class that any two charts in a maximal atlas are smoothly compatible; therefore, $\mathcal A_1^{\max} \subseteq \mathcal A_2^{\max}$. By symmetry $\mathcal A_2^{\max} \subseteq \mathcal A_1^{\max}$, so $\mathcal A_1^{\max} = \mathcal A_2^{\max}$. Conversely, suppose that $\mathcal A_1^{\max} = \mathcal A_2^{\max}$. Then $\mathcal A_1\subseteq \mathcal A_2^{\max}$ and so every chart in $\mathcal A_2$ is smoothly compatible with every chart in $\mathcal A_1$ and vice-versa.
        \item That $\mathcal A_1$ and $\mathcal A_2$ are different maximal atlases is to say that there is a chart $(U,\phi_U)\in \mathcal A_1$ not compatible with some chart $(V,\phi_V)\in\mathcal A_2$. Therefore some component of $\phi$, say $(\phi_U)_j\colon U\to\mathbb R$, is not smooth with respect to $(V,\phi_V)$ since $(\phi_U)_j\phi_V^{-1} = \pi_j\phi_U\phi_V^{-1} = (\phi_U\phi_V^{-1})_j$ could not be smooth. Thus by smooth compatibility of all charts in $\mathcal A_2$, $(\phi_U)_j$ is not smooth with respect to $\mathcal A_2$.
    \end{enumerate}
    \item \begin{enumerate}
      \item If $\mathcal A_1=\mathcal A_2$, then for any $(U_{1,i},\phi_{1,i})\in\mathcal A_1$ and $(U_{2,j},\phi_{2,j})\in\mathcal A_2$, $\phi_{2,j}\id\phi_{1,i}^{-1} = \phi_{2,j}\phi_{1,i}^{-1}$ is a diffeomorphism, and hence $\id\colon (X,\mathcal A_1)\to (X,\mathcal A_2)$ is also a diffeomorphism. If $\mathcal A_1\neq \mathcal A_2$, then one of $\phi_{2,j}\id\phi_{1,i}^{-1} = \phi_{2,j}\phi_{1,i}^{-1}$ is not a diffeomorphism and hence $\id\colon (X,\mathcal A_1)\to (X,\mathcal A_2)$ could not be a diffeomorphism.
      \item Let $X_{p} = (\mathbb R,\{(\mathbb R, x\mapsto x^{1/p})\})$, where $p$ is an odd prime. By part (a), the identity map $\id\colon X_q\to X_p$ is not a diffeomorphism since $x\mapsto x^{q/p}$ and $x\mapsto x^{p/q}$ are both not smooth. Therefore $\{(\mathbb R, x\mapsto x^{1/p})\}$ define distinct smooth structures on $\mathbb R$ but $X_q$ is diffeomorphic to $X_p$ since $x\mapsto x^{p/q}$ is a diffeomorphism; in coordinates it becomes the identity map on $\mathbb R$.
    \end{enumerate}
    \item The map $f$ is well-defined since each component of $f$ is defined by a homogeneous polynomial: $f(\lambda[X:Y]) = [\lambda^2X^2:\lambda^2XY:\lambda^2Y^2] = [X^2:XY:Y^2]$. We have  
    \[\phi_jf\phi_i^{-1}(t)=\begin{cases}
      (t,t^2) \text{ for } (i,j) = (0,0) & (t^2,t^1) \text{ for } (i,j) = (1,2)\\
      (t^{-1},t) \text{ for } (i,j) = (0,1), ~t\neq 0 & (t,t^{-1}) \text{ for } (i,j) = (1,1),~ t\neq 0\\
      (t^{-2},t^{-1}) \text{ for } (i,j) = (0,2),~ t\neq 0 & (t^{-1},t^{-2}) \text{ for } (i,j) = (1,0),~ t\neq 0.
    \end{cases}\]
    Therefore $f$ is smooth everywhere.
    \item Let $S^1$ have charts $(S^1\setminus\cbr{n},\phi_n\colon(x,y)\mapsto -2x/(y-1))$ and $(S^1\setminus\cbr{s},\phi_s\colon (x,y)\mapsto 2x/(y+1))$. Then let $f\colon S^1\to \mathbb{RP}^1$ be given by 
    \[f(x,y) = \begin{cases}
      [y-1:-2x]& \text{for }(x,y) \neq (0,1)\\
      [0: 1] & \text{for }(x,y) = (0,1).
    \end{cases}\]
    Indeed, for $t\in\mathbb R$ we have 
    \[\phi_if\phi_\ell^{-1}(t) = \begin{cases}
       t & \text{for } (i,\ell) = (0,n)\\
       t/4 & \text{for } (i,\ell) = (1,s),
    \end{cases}\] so $f$ is smooth.
    \item If $\deg f < 1$, let $\tilde f$ be defined by $[a:b]\mapsto [1:f(0)]$, which satisfies $\phi_0\tilde f\phi_0^{-1}(t) = \phi_0\tilde f([1:t]) = \phi_0([1:f(t)]) = f(t)$. Constant maps are automatically smooth since they are constant maps when written in coordinates. Let $\deg f \geq 1$ and define $\tilde f$ by $[1:t]\mapsto [1:f(t)]$ and $[0:1]\mapsto [0:1]$. Note for $t$ nonzero $\tilde f([1:t]) = [1:f(t)] = \tilde f([t^{-1}:1])$. Indeed, $\phi_0\tilde f\phi_0^{-1}(t) = f(t)$ and $\phi_1\tilde f\phi_1^{-1}(t) = f(1/t)^{-1}$. The map $\tilde f$ is smooth since $f(t) = \sum_{i=0}^n a_it^i$ is differentiable and $f(1/t)^{-1} = t^n/\sum_{i=0}^n a_ix^{n-i}$ is differentiable in a suitably chosen open set containing $t=0$. 
    \item Let $V = \mathbb R^4$. Let $Q$ be any point of $\Gr(4,2)$ and $U_Q = \{\cbr{(x,Ax)\mid x\in Q}\mid A\colon Q\to Q^\perp\}$, which contains $Q$ since we may choose $A = 0$ (here $(x,Ax)\in Q\oplus Q^\perp\cong \mathbb R^4$ and identify $Q$ with $Q\oplus 0$). Let $\psi_{U_Q}^{-1}\colon \Hom(Q,Q^\perp)\to U_Q$ be given by $A\mapsto \cbr{(x,Ax)\mid x\in Q}$. The map $\psi_{U_Q}\colon U_Q\to\Hom(Q,Q^\perp)$ is given by $\cbr{(x,y)\mid x\in Q}\mapsto A\colon x\mapsto y$. To obtain actual charts, let $\phi_{U_Q} = T_Q\psi_{U_Q}$ where $T_Q\colon \Hom(Q,Q^\perp)\to\mathbb R^4$ is any isomorphism taking $A$ to a matrix of $A$ in some preferred basis of $Q\oplus Q^\perp$. In the calculations below we use nice isomorphisms since our choices of $Q,P$ are very simple.
    
    Let $Q,P\in \Gr(4,2)$ be given by $Q = \Span\cbr{e_1,e_2}$ (and $Q^\perp = \Span\cbr{e_3,e_4}$) and $P = \Span\cbr{e_1,e_3}$ (and $P^\perp = \Span\cbr{e_2,e_4}$). Then $\phi_{U_P}\phi_{U_Q}^{-1}$ sends 
    \[\begin{pmatrix}
      a & b \\ c & d
    \end{pmatrix}\mapsto \begin{pmatrix}
      -a/b & 1/b \\ c - da/b & d/b
    \end{pmatrix}\]
    since the graph $\cbr{(x,Ax)\mid x\in Q} = \cbr{(c_1e_1,c_2e_2,(ac_1+bc_2)e_3, (cc_1+dc_2)e_4)}\subseteq Q\oplus Q^\perp$ is equivalent to the graph $\cbr{(k_1e_1,k_2e_3, (\alpha k_1 + \beta k_2)e_2,(\gamma k_1 + \delta k_2)e_4)}\subseteq P\oplus P^\perp$ with $\alpha = -a/b$, $\beta = 1/b$, $\gamma = c-da/b$, and $\delta = d/b$. For this transition map we have $b\neq 0$ since the graph $\cbr{(c_1e_1,(ac_1+bc_2)e_3,c_2e_2,(cc_1+dc_2)e_4)}$ should have trivial intersection with $P$ and $P^\perp$.

    Similarly one may compute the transition map for $Q = \Span\cbr{e_1,e_2} = P^\perp$ and $P = \Span\cbr{e_3,e_4} = Q^\perp$ to find that $\phi_{U_P}\phi_{U_Q}^{-1}\colon A\mapsto A^{-1}$.
    
    The dimension of $\Gr(4,2)$ is $4$.
    \item Let $(W,\phi_W)$ be a chart containing $p$. Let $V^\prime\subseteq \im\phi_W$ be an open ball of sufficiently small radius $R$ containing $\phi_W(p)$ and $U^\prime\subseteq V^\prime$ be any open ball of radius $r<R$ also containing $\phi_W(p)$. With $f(t) = \exp(-1/t^2)\chi_{(0,\infty)}(t)$, $g(t) = f(x-r)f(R-x)$, and $h(t) = (\int_r^xg(t)\dd t)/(\int_r^Rg(t)\dd t)$, let 
    \[b(x) = \begin{cases}
      1 & \text{for }x\in U^\prime\\
      1-h(\abs{x-\phi_W(p)}) & \text{for }x\in V^\prime\setminus U^\prime\\
      0 & \text{for } x\in \mathbb R^n\setminus V^\prime.
    \end{cases}\]
    Then $p$ is contained in open neighborhoods $U = \phi_W^{-1}U^\prime\subseteq V = \phi_W^{-1}V^\prime$ and $\beta = b\phi_W$, extended to the zero function outside of $W$, defines a smooth function (since $\beta\phi_W^{-1}$ is smooth) on $X$ that is $1$ on $U$, $0$ outside of $V$, and $0\leq \beta(x)\leq 1$ everywhere. 

    Let $f\colon O\to \mathbb R$ be as in Problem 1(b) and let $p\in O$. Then choose a chart $(W,\phi_W)$ on which $f$ is smooth containing $p$ and construct $\beta$ using the chart $(O\cap W,\phi_W|_{O\cap W})$. Then the product $f\cdot\beta\colon X\to\mathbb R$ is compactly supported around $p$ and smooth (in coordinates the product is $f\phi_W^{-1}\cdot b$, which is smooth).
\end{enumerate}
\end{document}