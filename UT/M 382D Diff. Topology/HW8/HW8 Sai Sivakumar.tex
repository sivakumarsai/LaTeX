\documentclass[11pt,leqno]{article}
\headheight=13.6pt

% packages
\usepackage[alphabetic]{amsrefs}
\usepackage{physics}
% margin spacing
\usepackage[top=1in, bottom=1in, left=0.5in, right=0.5in]{geometry}
\usepackage{hanging}
\usepackage{amsfonts, amsmath, amssymb, amsthm}
\usepackage{systeme}
\usepackage[none]{hyphenat}
\usepackage{fancyhdr}
\usepackage{graphicx}
\graphicspath{{./images/}}
\usepackage{float}
\usepackage{siunitx}
\usepackage{esint}
\usepackage{color}
\usepackage{enumitem}
\usepackage{mathrsfs}
\usepackage{hyperref}
\usepackage[noabbrev, capitalise]{cleveref}
\crefformat{equation}{equation~#2#1#3}
\crefformat{lemma}{\textrm{Lemma}~#2#1#3}
\usepackage{hanging}

% theorems
\theoremstyle{plain}
\newtheorem{lem}{Lemma}
\newtheorem{lemma}[lem]{Lemma}
\newtheorem{thm}[lem]{Theorem}
\newtheorem{theorem}[lem]{Theorem}
\newtheorem{prop}[lem]{Proposition}
\newtheorem{proposition}[lem]{Proposition}
\newtheorem{cor}[lem]{Corollary}
\newtheorem{corollary}[lem]{Corollary}
\newtheorem{conj}[lem]{Conjecture}
\newtheorem{fact}[lem]{Fact}
\newtheorem{form}[lem]{Formula}

\theoremstyle{definition}
\newtheorem{defn}[lem]{Definition}
\newtheorem{definition/}[lem]{Definition}
\newenvironment{definition}
  {\renewcommand{\qedsymbol}{\textdagger}%
   \pushQED{\qed}\begin{definition/}}
  {\popQED\end{definition/}}
\newtheorem{example}[lem]{Example}
\newtheorem{remark}[lem]{Remark}
\newtheorem{exercise}[lem]{Exercise}
\newtheorem{notation}[lem]{Notation}

\numberwithin{equation}{section}
\numberwithin{lem}{section}

% header/footer formatting
\pagestyle{fancy}
\fancyhead{}
\fancyfoot{}
\fancyhead[L]{M 382D}
\fancyhead[C]{HW8}
\fancyhead[R]{Sai Sivakumar}
\fancyfoot[R]{\thepage}
\renewcommand{\headrulewidth}{1pt}

% paragraph indentation/spacing
\setlength{\parindent}{0cm}
\setlength{\parskip}{10pt}
\renewcommand{\baselinestretch}{1.25}

% extra commands defined here
\newcommand{\br}[1]{\left(#1\right)}
\newcommand{\sbr}[1]{\left[#1\right]}
\newcommand{\cbr}[1]{\left\{#1\right\}}
\newcommand{\eq}[1]{\overset{(#1)}{=}}

% bracket notation for inner product
\usepackage{mathtools}

\DeclarePairedDelimiterX{\abr}[1]{\langle}{\rangle}{#1}

\DeclareMathOperator{\Span}{span}
\DeclareMathOperator{\im}{im}
\newcommand{\res}[1]{\operatorname*{res}_{#1}}
\DeclareMathOperator{\id}{id}
\DeclareMathOperator{\Hom}{Hom}
\DeclareMathOperator{\Adj}{Adj}
\DeclareMathOperator{\Ad}{Ad}
\DeclareMathOperator{\End}{End}
\DeclareMathOperator{\codim}{codim}
\DeclareMathOperator{\Int}{int}
\newcommand{\GL}{\mathrm{GL}}
\newcommand{\Mat}{\mathrm{M}}
\newcommand{\Sp}{\mathrm{Sp}}
\newcommand{\AS}{\mathrm{AntiSym}}

% set page count index to begin from 1
\setcounter{page}{1}

\begin{document}
\begin{enumerate}
    \item 
    \item \begin{enumerate}
      \item It suffices to see that a homotopy of $f$ or $g$ induces a homotopy of $f\times g$; by symmetry, consider just a homotopy of $f$. Let $H\colon X\times I\to Y$ be a homotopy of $f$ with $\tilde f$. Then the homotopy $\mathcal H\colon X\times Z\times I\to Y\times Y$ given by $\mathcal H(x,z,t) = H(x,t)\times g(z)$ is indeed continuous with $\mathcal H(x,z,0) = f\times g$ and $\mathcal H(x,z,1) = \tilde f\times g$. We may even simultaneously homotope $f$ and $g$ to obtain a homotopy of $f\times g$ by taking the product of their homotopies.
      \item By part (a), we may assume without loss of generality that $f$ is transversal to $Z$. We show that $f\times i_Z$ is transversal to $\Delta$: For any point $(x,z)$ in $(f\times i_Z)^{-1}(\Delta) = \cbr{(x,z)\in X\times Z\mid f(x) = z}$, we have $D(f\times i_Z)_{(x,z)}T_{(x,z)}(X\times Z) + T_{(f(x),z)}\Delta \cong Df_xT_xX \times T_zZ + \Delta(T_{z}Y)$, where $\Delta(T_z(Y)^2)$ is the diagonal of the space $T_z(Y)^2$. For any $(v,w)\in T_zY\times T_zY$, choose  Then $I(f\times i_Z, \Delta)$ is the signed count of the points belonging to 
    \end{enumerate}
    \item 
    \item 
    \item 
    \item 
    \item 
\end{enumerate}
\end{document}