\documentclass[11pt,leqno]{article}
\headheight=13.6pt

% packages
\usepackage[alphabetic]{amsrefs}
\usepackage{physics}
% margin spacing
\usepackage[top=1in, bottom=1in, left=0.5in, right=0.5in]{geometry}
\usepackage{hanging}
\usepackage{amsfonts, amsmath, amssymb, amsthm}
\usepackage{systeme}
\usepackage[none]{hyphenat}
\usepackage{fancyhdr}
\usepackage{graphicx}
\graphicspath{{./images/}}
\usepackage{float}
\usepackage{siunitx}
\usepackage{esint}
\usepackage{color}
\usepackage{enumitem}
\usepackage{mathrsfs}
\usepackage{hyperref}
\usepackage[noabbrev, capitalise]{cleveref}
\crefformat{equation}{equation~#2#1#3}
\crefformat{lemma}{\textrm{Lemma}~#2#1#3}
\usepackage{hanging}

% theorems
\theoremstyle{plain}
\newtheorem{lem}{Lemma}
\newtheorem{lemma}[lem]{Lemma}
\newtheorem{thm}[lem]{Theorem}
\newtheorem{theorem}[lem]{Theorem}
\newtheorem{prop}[lem]{Proposition}
\newtheorem{proposition}[lem]{Proposition}
\newtheorem{cor}[lem]{Corollary}
\newtheorem{corollary}[lem]{Corollary}
\newtheorem{conj}[lem]{Conjecture}
\newtheorem{fact}[lem]{Fact}
\newtheorem{form}[lem]{Formula}

\theoremstyle{definition}
\newtheorem{defn}[lem]{Definition}
\newtheorem{definition/}[lem]{Definition}
\newenvironment{definition}
  {\renewcommand{\qedsymbol}{\textdagger}%
   \pushQED{\qed}\begin{definition/}}
  {\popQED\end{definition/}}
\newtheorem{example}[lem]{Example}
\newtheorem{remark}[lem]{Remark}
\newtheorem{exercise}[lem]{Exercise}
\newtheorem{notation}[lem]{Notation}

\numberwithin{equation}{section}
\numberwithin{lem}{section}

% header/footer formatting
\pagestyle{fancy}
\fancyhead{}
\fancyfoot{}
\fancyhead[L]{M 382D}
\fancyhead[C]{HW10}
\fancyhead[R]{Sai Sivakumar}
\fancyfoot[R]{\thepage}
\renewcommand{\headrulewidth}{1pt}

% paragraph indentation/spacing
\setlength{\parindent}{0cm}
\setlength{\parskip}{10pt}
\renewcommand{\baselinestretch}{1.25}

% extra commands defined here
\newcommand{\br}[1]{\left(#1\right)}
\newcommand{\sbr}[1]{\left[#1\right]}
\newcommand{\cbr}[1]{\left\{#1\right\}}
\newcommand{\eq}[1]{\overset{(#1)}{=}}

% bracket notation for inner product
\usepackage{mathtools}

\DeclarePairedDelimiterX{\abr}[1]{\langle}{\rangle}{#1}

\DeclareMathOperator{\Span}{span}
\DeclareMathOperator{\im}{im}
\newcommand{\res}[1]{\operatorname*{res}_{#1}}
\DeclareMathOperator{\id}{id}
\DeclareMathOperator{\Hom}{Hom}
\DeclareMathOperator{\Adj}{Adj}
\DeclareMathOperator{\Ad}{Ad}
\DeclareMathOperator{\End}{End}
\DeclareMathOperator{\codim}{codim}
\DeclareMathOperator{\Int}{int}
\DeclareMathOperator{\sgn}{sgn}
\newcommand{\GL}{\mathrm{GL}}
\newcommand{\SO}{\mathrm{SO}}
\newcommand{\Mat}{\mathrm{M}}
\newcommand{\Sp}{\mathrm{Sp}}
\newcommand{\AS}{\mathrm{AntiSym}}

% set page count index to begin from 1
\setcounter{page}{1}

\begin{document}
\begin{enumerate}
    \item \begin{enumerate}
      \item Let $\cbr{U_i}$ and $\cbr{V_j}$ be a collection of charts covering $X$, and to each collection of charts let $\cbr{\rho_i}$ and $\cbr{\phi_j}$ be partitions of unity subordinate to $\cbr{U_i}$ and $\cbr{V_j}$ respectively. Then by a common refinement argument, we have $\sum_i\int_{U_i}\rho_i\omega = \sum_i\sum_j\int_{U_i\cap V_j}\rho_i\phi_j\omega = \sum_j\sum_i\int_{V_j\cap U_i}\phi_j\rho_i\omega = \sum_j\int_{V_j}\phi_j\omega$. Therefore the integral $\int_X\omega$ is well defined.
      \item Let $\cbr{(U_\alpha,\varphi_\alpha)}$ be compatible charts on $X$ giving $X$ a positive orientation. Then modify the charts by negating the first coordinates in each chart to obtain negatively oriented charts $\cbr{(U_\alpha,\tilde \varphi_\alpha)}$; these charts are the charts on $-X$. We have for each $\alpha$ that $\int_{U_\alpha}\omega = \int_{\mathbb R^n}f(x_1,\dots,x_n)dx_1\wedge\cdots\wedge dx_n = \int_{\mathbb R^n} f(x_1,\dots,x_n)dx_1\cdots dx_n = - \int_{\mathbb R^n} f(-y_1,y_2,\dots,y_n)dy_1\cdots dy_n = -\int_{U_\alpha} f(y_1,\dots,y_n)dy_1\wedge \cdots \wedge dy_n = -\int_{-U_\alpha}\omega$. Therefore $\int_X \omega = -\int_{-X}\omega$.
    \end{enumerate}
    \item If $X$ admits a volume form $\omega$, then in local coordinates on a chart $U$, the form $\omega$ is of the form $fdx_1\wedge \cdots\wedge dx_n$ with $f$ never evaluating to zero; that is, $f$ is either positive or negative on the coordinates in the coordinates on $U$. Changing coordinates to coordinates $y_1,\dots,y_n$ on $V$ yields $f(x(y))\det(D)dy_1,\dots,dy_n$, where $D$ is the inverse of the derivative of the change of coordinates map from $V$ to $U$. Since $f$ takes on only one sign, the determinant of $D$ must also have the same sign. Since $U$ and $V$ were arbitrary, it follows that for any two charts on $X$ the transition function between them has derivative with determinant of a fixed sign (across all charts), so $X$ is orientable. 
    
    If $X$ is orientable, choose a covering of $X$ by charts $U_i$ that have compatible orientation and choose a locally finite partition of unity (or otherwise just a partition of unity if $X$ is compact) $\{\phi_i\}$ subordinate to the charts $U_i$. Then in each chart $U_i$ with coordinates $x_{i,1},\dots,x_{i,n}$, consider the form given by $\phi_idx_{i,1}\wedge\cdots\wedge dx_{i,n}$, which since $\phi_i$ has compact support, globally defines a form $\omega_i$ on $X$. Then the sum $\omega = \sum_i\omega_i$ is an $n$-form on $X$ (and this sum is well defined since we chose a locally finite partition of unity). That $\omega$ is nonvanishing is due to the partition of unity being given by nonnegative functions and due to our charts being chosen with compatible orientations (the determinants of the derivatives of the change of coordinate maps are all of one sign).
    \item Choose the chart covering $S^1\setminus\cbr{(1,0)}$, parameterized by the interval $(0,2\pi)$ via the map $t\mapsto (\cos(t),\sin(t))$. Then $\int_{S^1}(xdy-ydx) = \int_0^{2\pi}[\cos^2(\theta) + \sin^2(\theta)]\,d\theta = 2\pi$.
    \item Calculating this integral in a chart given by sterographic projection is slightly miserable but reduces due to the symmetry of the sphere; this suggests that the integral may be easier to calculate in a chart given by the usual spherical change of coordinates. We will calculate the integral in spherical coordinates.
    
    % Via the stereographic projection $\varphi^{-1}\colon \mathbb R^2\to\mathbb R^3$\dots

    Via the spherical change of coordinates $\varpi^{-1}\colon \mathbb R^2\to\mathbb R^3$ given by $(\varphi,\theta)\mapsto (\cos(\theta)\sin(\varphi), \sin(\theta)\sin(\varphi), \cos(\varphi))$ (this is orientation-preserving), the subset of $S^2$ missed by the image of $\varpi^{-1}$ is a half-meridian with boundary the north and south poles of $S^2$, which has measure zero. Then $\int_{S^2}(x dy \wedge dz + y dz \wedge dx + z dx \wedge dy) = \int_0^{2\pi}\int_0^\pi [\cos^2(\theta)\sin^3(\varphi) + \sin^2(\theta)\sin^3(\varphi) + \sin(\varphi)\cos^2(\varphi)] \,d\varphi d\theta = 2\pi\int_0^\pi\sin(\varphi)\,d\varphi = 4\pi$.
    \item The standard copy of $\mathbb{CP}^1 = \cbr{[w_0:w_1:0]}$ inside $\mathbb{CP}^2$, has a subset of full measure ($\mathbb{CP}^1\setminus\cbr{[0:1:0]}$) that belongs to the chart $(U_0,\varphi_0)$, which is given complex coordinates $z_1,z_2$. The points in $\mathbb{CP}^1 = \cbr{[w_0:w_1:0]}$ belonging to $U_0$ in the coordinates $(z_1,z_2)$ are those with $z_2 = 0$. By identifying $x+iy$ with $(x,y)$, parameterize the points $(z_1,0)$ by the map $\mathbb R^2\to \mathbb C^2$ via $(x,y)\mapsto (x+iy,0)$. Pulling back the Fubini-Study form $\omega_{FS}$ along this map yields the form
    \[\bigg[\frac{1}{(1+x^2+y^2)^2}\bigg]dx\wedge dy,\]
    which when integrated on $\mathbb R^2$ yields $\int_0^\infty\int_0^\infty (1+x^2+y^2)^{-2} \,dxdy = \int_0^{2\pi}\int_0^\infty (1+r^2)^{-2}r\,drd\theta = \pi\int_1^\infty u^{-2}\, du = \pi$.
    \item That the map $\ell$ is well defined comes down to showing that it is alternating and multilinear as a map on $V^k$, since those are the properties by which we quotient the tensor powers of $V$ by to obtain the exterior powers of $V$. Indeed, for any element $\tau\in S_n$, $\ell(v_{\tau(1)}\wedge\cdots\wedge v_{\tau(k)}) = \sum_{\sigma\in S_n}\sgn(\sigma)\prod_{i=1}^k\alpha_i(v_{\sigma\tau(i)}) = \sgn(\tau)\sum_{\sigma\in S_n}\sgn(\sigma)\prod_{i=1}^k\alpha_i(v_{\sigma(i)}) = \sgn(\tau)\ell(v_1\wedge\cdots\wedge v_k)$; we have that 
    \begin{multline*}
      \ell(v_1+w_1\wedge\cdots\wedge v_k) = \sum_{\sigma\in S_n}\sgn(\sigma)\alpha_{\sigma^{-1}(1)}(v_1+w_1)\prod_{i=2}^k\alpha_{\sigma^{-1}(i)}(v_{i}) = \\\sum_{\sigma\in S_n}\sgn(\sigma)\alpha_{\sigma^{-1}(1)}(v_1)\prod_{i=2}^k\alpha_{\sigma^{-1}(i)}(v_{i})+ \sum_{\sigma\in S_n}\sgn(\sigma)\alpha_{\sigma^{-1}(1)}(W_1)\prod_{i=2}^k\alpha_{\sigma^{-1}(i)}(v_{i}) =\\ \ell(v_1\wedge\cdots\wedge v_k) + \ell(w_1\wedge v_2\wedge\cdots \wedge v_k).
    \end{multline*}
    Hence $\ell$ is well-defined. A similar calculation shows that $\eta$ is also well defined since it is also alternating and multilinear in a similar sense as before. That $\eta$ is an isomorphism is due to $\eta$ being a linear and injective map of vector spaces of the same dimension. Since $V$ has finite dimension, $\bigwedge^k(V^\ast)$ has the same dimension as $(\bigwedge^k V)^\ast$. Let $\alpha_1\wedge\cdots\wedge\alpha_k$ be nonzero so that the $\alpha_i$ are linearly independent in $V^\ast$. Extend $\cbr{\alpha_i}$ to a basis on $V^\ast$ and obtain a dual basis $\cbr{v_i}$ of $V$. Then $\ell$ is nonzero since $\ell(v_1\wedge\cdots\wedge v_k) = 1$; that is, $\ell$ is zero if and only if $\alpha_1\wedge\cdots\alpha_k$ is nonzero, so $\eta$ is injective and hence an isomorphism.
    
    Choose local coordinates $x_1,\dots,x_n$ in a chart containing $U$ in $X$ and choose coordinates on $U\subset Z$ for which the inclusion of $Z$ into $X$ in these coordinates is the canonical inclusion of $\mathbb R^k$ into $\mathbb R^n$, so the differential $Di_Z$ acts trivially at each point on the derivations $\pdv{x_i}|_p$. With $\omega = gdx_1\wedge\cdots\wedge dx_k$ on $U$, we have $\int_U f(p)\,dx_1\cdots dx_k = \int_U \eta(\omega|_p)(\pdv{x_1}|_p\wedge\cdots\wedge \pdv{x_k}|_p)\,dx_1\cdots dx_k = \int_U g(p)(1)\,dx_1\cdots dx_k = \int_U\omega$.
    \item \begin{enumerate}
      \item 
      % Let $\omega\in \Omega^0(X)$ be a $0$-form, and choose two charts $(U_\alpha,\varphi_\alpha)$ and $(U_\beta,\varphi_\beta)$ with coordinates $x_1,\dots,x_n$ and $y_1,\dots,y_n$ respectively. In the coordinates $x_1,\dots,x_n$, the form $\omega$ is given by a function $f = f(x_1,\dots,x_n)$. Then change coordinates via the transition function $\varphi_\alpha\varphi_\beta^{-1}$ and then take the exterior derivative to obtain 
      % \begin{multline*}
      %   df(x_1(y_1,\dots,y_n),\dots,x_n(y_1,\dots,y_n)) = \\\sum_j\bigg(\sum_i\pdv{f}{x_i}\/(x_1(y_1,\dots,y_n),\dots,x_n(y_1,\dots,y_n))\pdv{x_i}{y_j}\/(y_1,\dots,y_n)\bigg)dy_j.
      % \end{multline*}
      % On the other hand, taking the exterior derivative of $f$ first and then changing coordinates yields 
      % \begin{multline*}
      %   (df)(x_1(y_1,\dots,y_n),\dots,x_n(y_1,\dots,y_n)) = \\\sum_i\pdv{f}{x_i}\/(x_1(y_1,\dots,y_n),\dots,x_n(y_1,\dots,y_n))\bigg(\sum_j \pdv{x_i}{y_j}\/(y_1,\dots,y_n)dy_j\bigg).
      % \end{multline*}
      % Evidently both sums coincide, so for $0$-forms, the exterior derivative is well-defined.
      
      Let $\omega\in \Omega^k(X)$ be a $k$-form, which in the coordinates $x_1,\dots,x_n$ is $\sum_{i_1<\cdots<i_k}f_{i_1,\dots,i_k}dx_{i_1}\wedge\cdots\wedge dx_{i_k}$; we will prove the result for just one summand, which we denote by $f dx_{i_1}\wedge\cdots\wedge dx_{i_k}$, and assume that the exterior derivative is well-defined for $(k-1)$-forms. Then changing coordinates followed by taking the exterior derivative, we obtain from the product rule and by the anti-symmetry of the wedge product
      \[df(x(y))=\sum_{j_1,\dots,j_k}\sum_\ell\bigg[\pdv{f}{y_\ell}\pdv{x_{i_1}}{y_{j_1}}\cdots \pdv{x_{i_k}}{y_{j_k}}\bigg]dy_\ell\wedge dy_{j_1}\wedge\cdots\wedge dy_{j_k},\] which we compare to first taking the exterior derivative and then changing coordinates, which looks like
      \[\sum_\ell\bigg(\pdv{f}{x_\ell}\/(x(y))\bigg)\bigg(\sum_j\pdv{x_\ell}{y_j}dy_j\bigg)\wedge\bigg(\sum_j\pdv{x_{i_1}}{y_j}dy_j\bigg)\wedge\cdots\wedge\bigg(\sum_j\pdv{x_{i_k}}{y_j}dy_j\bigg).\] Both expressions are equal if you expand everything out (by antisymmetry of the wedge product), so the exterior derivative is well-defined.
      
      % . The exterior derivative of $\omega$ in these coordinates is $\sum_{i_1<\cdots<i_k}\sum_{\ell}f_{i_1,\dots,i_k}dx_{i_1}\wedge\cdots\wedge dx_{i_k}$
      \item We could proceed in a similar manner to the previous part, but maybe an inductive proof is cleaner. On zero forms $g\in\Omega^0(Y)$, write $g$ in local coordinates and take the pullback along $f\colon X\to Y$ followed by the exterior derivative to obtain $d(f^\ast g) = \sum_i\sum_j\pdv{g}{y_j}\/(f(x))\pdv{f}{x_i}\/(x)dx_i$; by taking the exterior derivative first and then pulling back along $f$ we obtain $f^\ast(dg) = \sum_j\pdv{g}{y_j}\/(f(x))\sum_i\pdv{f}{x_i}\/(x)dx_i$, which agrees with the previous expression. Since $f^\ast(\omega_1\wedge \omega_2) = f^\ast\omega_1\wedge f^\ast\omega_2$, we may induct and use the properties of the wedge product and multivariable calculus to deduce the same result for $k$-forms.
    \end{enumerate}
    \item Using the result of Problem 7, we have $\int_{[a,b]} \gamma^\ast\omega = \int_{[a,b]}\gamma^\ast df = \int_{[a,b]}d(\gamma^\ast f) = (\gamma^\ast f)(b) - (\gamma^\ast f)(a) = f(\gamma(b)) - f(\gamma(a)) = f(q) - f(p)$.
    \item \begin{enumerate}
      \item Directly calculating yields 
      \begin{align*}
        d\omega_{(x,y)} &= \bigg[\pdv{x}\bigg(\frac{-y}{x^2+y^2}\bigg)dx + \pdv{y}\bigg(\frac{-y}{x^2+y^2}\bigg)dy\bigg]\wedge dx + \bigg[\pdv{x}\bigg(\frac{x}{x^2+y^2}\bigg)dx + \pdv{y}\bigg(\frac{x}{x^2+y^2}\bigg)dy\bigg]\wedge dy\\
        &= \bigg(\frac{x^2-y^2}{(x^2+y^2)^2}\bigg) dx\wedge dy + \bigg(\frac{y^2-x^2}{(x^2+y^2)^2}\bigg)dx\wedge dy\\
        &= 0.
      \end{align*}
      \item If $df = \omega$ on $x>0$, then $f$ must satisfy 
      \[\pdv{f}{x}\/(x,y) = \frac{-y}{x^2+y^2} \quad\text{and}\quad \pdv{f}{y}\/(x,y) = \frac{x}{x^2+y^2}.\]
      Then $f(x,y) = \int_0^x -y/(t^2+y^2)\,dt + f(0,y) = -\arctan(x/y) + f(0,y)$. Differentiating the previous expression with respect to $y$ yields $\pdv{f}{y}\/(x,y) = \pdv{y}[-\arctan(x/y) + f(0,y)] = x/(x^2+y^2) + \pdv{f}{y}\/(0,y) = x/(x^2+y^2)$, which implies $f(0,y)$ is some constant $c$. Therefore for $x>0$, $f$ may be given by $f(x,y) = -\arctan(x/y) + c$ for any constant $c$.
      \item If there was a function $f$ defined on all of $\mathbb R^2\setminus\cbr{(0,0)}$ for which $\omega = df$, then Stokes' theorem implies $0 = \int_{\partial S^1}f = \int_{S^1}\omega = \int_0^{2\pi}[\sin^2(\theta) + \cos^2(\theta)]\,d\theta = 2\pi$, impossible.
    \end{enumerate}
\end{enumerate}
\end{document}