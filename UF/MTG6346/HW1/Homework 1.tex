\documentclass[11pt]{article}

% packages
\usepackage{physics}
% margin spacing
\usepackage[top=1in, bottom=1in, left=0.5in, right=0.5in]{geometry}
\usepackage{hanging}
\usepackage{amsfonts, amsmath, amssymb, amsthm}
\usepackage{systeme}
\usepackage[none]{hyphenat}
\usepackage{fancyhdr}
\usepackage[nottoc, notlot, notlof]{tocbibind}
\usepackage{graphicx}
\graphicspath{{./images/}}
\usepackage{float}
\usepackage{siunitx}
\usepackage{esint}
\usepackage{cancel}
\usepackage{enumitem}
\usepackage{tikz-cd}

% permutations (second line is for spacing)
\usepackage{permute}
\renewcommand*\pmtseparator{\,}

% colors
\usepackage{xcolor}
\definecolor{p}{HTML}{FFDDDD}
\definecolor{g}{HTML}{D9FFDF}
\definecolor{y}{HTML}{FFFFCF}
\definecolor{b}{HTML}{D9FFFF}
\definecolor{o}{HTML}{FADECB}
%\definecolor{}{HTML}{}

% \highlight[<color>]{<stuff>}
\newcommand{\highlight}[2][p]{\mathchoice%
  {\colorbox{#1}{$\displaystyle#2$}}%
  {\colorbox{#1}{$\textstyle#2$}}%
  {\colorbox{#1}{$\scriptstyle#2$}}%
  {\colorbox{#1}{$\scriptscriptstyle#2$}}}%

% header/footer formatting
\pagestyle{fancy}
\fancyhead{}
\fancyfoot{}
\fancyhead[L]{MTG6346 Topology}
\fancyhead[C]{Homework 1}
\fancyhead[R]{Sai Sivakumar}
\fancyfoot[R]{\thepage}
\renewcommand{\headrulewidth}{1pt}

% paragraph indentation/spacing
\setlength{\parindent}{0cm}
\setlength{\parskip}{10pt}
\renewcommand{\baselinestretch}{1.25}

% extra commands defined here
\newcommand{\br}[1]{\left(#1\right)}
\newcommand{\sbr}[1]{\left[#1\right]}
\newcommand{\cbr}[1]{\left\{#1\right\}}

\newcommand{\catname}[1]{{
  %\normalfont
  \textbf{#1}}}
\newcommand{\Set}{\catname{Set}}
\newcommand{\Top}{\catname{Top}}

% bracket notation for inner product
\usepackage{mathtools}

\DeclarePairedDelimiterX{\abr}[1]{\langle}{\rangle}{#1}
\DeclarePairedDelimiter{\ceil}{\lceil}{\rceil}
\DeclarePairedDelimiter{\floor}{\lfloor}{\rfloor}

% set page count index to begin from 1
\setcounter{page}{1}

\begin{document}
\begin{enumerate}
    \item Show that if a category $\mathbf{C}$ has two terminal objects $t_1,t_2$, then they are isomorphic. (In fact, there is a unique isomorphism $t_1\to t_2$.)
    
    \textit{Remark}. The same argument shoes that if a category $\mathbf{C}$ has two initial objects, then they are isomorphic. Furthermore, essentially the same argument shows that if a diagram in $\mathbf{C}$ has two limits/colimits then they are isomorphic. It is a convention to speak of ``the terminal object/initial object/limit/colimit'' instead of ``some terminal object/initial object/limit/colimit'' or ``the isomorphism class of terminal objects/initial objects/limits/colimits''. \begin{proof}
      Let $\mathbf{C}$, $t_1,t_2$ be as above. Since both $t_1,t_2$ are terminal there exist unique morphisms $f\colon t_1\to t_2$ and $g\colon t_2\to t_1$.

      Composing $f,g$ both ways gives us morphisms $g\circ f\colon t_1\to t_1$, $f\circ g\colon t_2\to t_2$. But for $i = 1,2$, the unique morphism from $t_i$ to itself is the identity $1_{t_i}$, so $g\circ f = 1_{t_1}$ and $f\circ g = 1_{t_2}$ so that $f,g$ are both injective and surjective. It follows that $f,g$ are isomorphisms of $t_1$ with $t_2$ as desired.
    \end{proof}
    \item Consider the categories $(\mathbb{Z},\leq)$ and $(\mathbb{R},\leq)$. Let $i\colon(\mathbb{Z},\leq)\to (\mathbb{R},\leq)$ be the functor given by inclusion. Find functors $L,R\colon (\mathbb{R},\leq)\to (\mathbb{Z},\leq)$ such that $L\dashv i$ and $i\vdash R$.
    
    Choose $L=\ceil{\cdot}$ and $R=\floor{\cdot}$, the standard ceiling and floor functions. [That is, for $x\in \mathbb{R}$, the quantity $\ceil{x}$ is the least integer greater than $x$; similarly $\floor{x}$ is the greatest integer less than $x$. These maps are functors since if $a,b,c\in\mathbb{R}$ with $a\leq b\leq c$ and $a\leq a$ then $\ceil a\leq \ceil b\leq \ceil c$ and $\ceil a \leq \ceil a$ hold, similarly $\floor a\leq \floor b\leq \floor c$ and $\floor a\leq \floor a$ hold.] \begin{proof}
      The functor $L=\ceil{\cdot}$ is a left adjoint of the inclusion map $i$ if for all $z\in\mathbb{Z}$ and $r\in \mathbb{R}$, we have $\mathbb{Z}(\ceil r, z)\cong \mathbb{R}(r,z)$. Indeed, $\ceil r\leq z$ if and only if $r\leq z$.

      Similarly, the functor $R = \floor{\cdot}$ is a right adjoint of the inclusion map $i$ if for all $r\in\mathbb{R}$ and $z\in\mathbb{Z}$, we have $\mathbb{R}(z,r)\cong \mathbb{Z}(z,\floor r)$. Indeed, $z\leq r$ if and only if $z\leq \floor r$.
    \end{proof}
    \item (1.3.3) A space $X$ is separated if and only if the diagonal $D = \cbr{(x,x)\mid x\in X}$ is closed in $X\times X$. \begin{proof}
      Let $X$ be a topological space, and write $D^c = (X\times X)\setminus D = \{(x_1,x_2)\mid x_1,x_2\in X, x_1\neq x_2\}$.
      
      Suppose $X$ is Hausdorff. We show that $D^c$ is open. Take any $(x_1,x_2)\in D^c$ so that $x_1\neq x_2$ and so there exist disjoint open sets $U_{x_1},U_{x_2}\subset X$ containing $x_1,x_2$ respectively. Then $U_{x_1}\times U_{x_2}$ is an open set in $X\times X$ containing $(x_1,x_2)$ which is contained in $D^c$. Hence $D$ is closed.

      Conversely, supposed $D$ is closed so that $D^c$ is open. For any points $x_1,x_2\in X$ with $x_1\neq x_2$, we have that $(x_1,x_2)\in D^c$. But then there exists a basis element $U_{x_1}\times U_{x_2}$ in the product topology on $X\times X$ containing $(x_1,x_2)$ which is contained in $D^c$. It follows that $U_{x_1}, U_{x_2}$ are disjoint open sets containing $x_1,x_2$ respectively. Since $x_1,x_2$ were arbitrary it follows that $X$ is Hausdorff.
    \end{proof}
    
    Let $f,g\colon X\to Y$ be continuous maps into a Hausdorff space. Then the \textit{coincidence set} $A = \cbr{x\mid f(x) = g(x)}$ is closed in $X$. \begin{proof}
      Define $h\colon X\to Y\times Y$ by $h(x) = (f(x),g(x))$, which is continuous because $f,g$ are. Then since $Y$ is Hausdorff, the diagonal $D = \cbr{(y,y)\mid y\in Y}$ is closed in $Y\times Y$, so that its preimage under $h$ is also closed by continuity.

      We have that $h^{-1}(D) = \cbr{x\mid h(x)\in D}$, but $h(x) = (f(x),g(x))\in D$ if and only if $f(x) = g(x)$. So $h^{-1}(D) = A = \cbr{x\mid f(x) = g(x)}$ is a closed set as desired.
    \end{proof}
    \item (1.8.1) Let $H$ be a normal subgroup of $G$ and $X$ a $G$-space. Restricting the group action to $H$, we obtain an $H$-space $X$. The orbit space $H\backslash X$ carries then an induced $G/H$-action. \begin{proof}
      Let $H$ be a normal subgroup of $G$ and $X$ a $G$-space with left action $\ell\colon G\times X\to X\in \Top$ as above. We check that the restriction $\ell\vert_H\colon H\times X \to H$ given by $(h,x)\mapsto \ell(h,x) = hx$ is a left action of $H$ on $X$. Let $x\in X$. Since $1_H = 1_G$, we have that $1_Hx = 1_Gx = x$. Then let $h_1,h_2\in H$. Since $H\subset G$ we use the left action specified by $\ell$ to find that $h_2(h_1x) = (h_2h_1)x$, and since $H$ is a group, $h_2h_1\in H$ as needed. The continuity of this group action comes from the continuity of $\ell$.

      The orbit space $H/X$ is given by equivalence classes of the relation $\sim$ where if $x\in X$, then $x\sim hx$ for all $h\in H$. For $x\in X$ denote its equivalence class by $[x]$. Give $H/X$ the quotient topology by the quotient map $\pi_H$ sending $x$ to $[x]$.

      Define the induced left action on $H/X$ by $G/H$ (which is a topological group since $H$ is normal in $G$ and we give $G/H$ the quotient topology also; call the quotient map $\pi_G$): $\rho\colon (G/H)\times (H/X)\to H/X $ where $(gH,[x])\mapsto [gx]$ where $gx = \ell(g,x)$.
      
      Check that $\rho$ is indeed a left action: Let $x\in X$ and $g_1,g_2\in G$. Then $1_GH[x] = [1_Gx] = [x]$ and $g_2H(g_1H[x]) = g_2H[g_1x] = [g_2(g_1x)] = [(g_2g_1)x] = (g_2g_1)H[x] = (g_1Hg_2H)[x]$ as expected.
      
      We check that this map is well defined: Let $gH = g^\prime H$, so that there exists $h_1\in H$ such that $gh_1 = g^\prime$. Similarly, let $[x] = [x^\prime]$, so that there exists $h_2\in H$ such that $h_2 x = x^\prime$. Then \[g^\prime H[x^\prime] = (gh_1)H[h_2 x] = [gh_1h_2 x] = [g(h_1h_2 x)] = gH((h_1h_2)H[x]) =  gH(1_GH[x]) = (gH1_GH)[x] = gH[x]\] as desired.

      The following diagram commutes: % https://q.uiver.app/?q=WzAsNCxbMCwwLCIoRy9IKVxcdGltZXMoSC9YKSJdLFsyLDAsIkgvWCJdLFswLDIsIkdcXHRpbWVzIFgiXSxbMiwyLCJYIl0sWzIsMywiXFxlbGwiLDJdLFswLDEsIlxccmhvIl0sWzMsMSwiXFxwaV9IIiwyXSxbMiwwLCJcXHBpX0dcXHRpbWVzXFxwaV9IIl1d
      \[\begin{tikzcd}
        {(G/H)\times(H/X)} && {H/X} \\
        \\
        {G\times X} && X
        \arrow["\ell"', from=3-1, to=3-3]
        \arrow["\rho", from=1-1, to=1-3]
        \arrow["{\pi_H}"', from=3-3, to=1-3]
        \arrow["{\text{``$\pi_G\times\pi_H$''}}", from=3-1, to=1-1]
      \end{tikzcd}\] Note here that $\ell, \pi_H$ are continuous so if we take an open set $U\subseteq H/X$ and take the preimage of $U$ under $\pi_H$ and then $\ell$, we obtain an open set $V$ in $G\times X$. This same set should be obtained if we took preimages in the other direction. So the preimage of $\rho$ is a set whose preimage under what should be the ``product'' quotient map ``$\pi_G\times\pi_H$'' is open, but by definition of the quotient topology it follows that the preimage of $U$ under $\rho$ must be open. Hence $\rho\in \Top$ as desired and it is the induced $G/H$ action on $H/X$.
    \end{proof}
\end{enumerate}
\end{document}