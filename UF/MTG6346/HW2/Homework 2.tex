\documentclass[11pt]{article}

% packages
\usepackage{physics}
% margin spacing
\usepackage[top=1in, bottom=1in, left=0.5in, right=0.5in]{geometry}
\usepackage{hanging}
\usepackage{amsfonts, amsmath, amssymb, amsthm}
\usepackage{systeme}
\usepackage[none]{hyphenat}
\usepackage{fancyhdr}
\usepackage[nottoc, notlot, notlof]{tocbibind}
\usepackage{graphicx}
\graphicspath{{./images/}}
\usepackage{float}
\usepackage{siunitx}
\usepackage{esint}
\usepackage{cancel}
\usepackage{enumitem}
\usepackage{tikz-cd}

% permutations (second line is for spacing)
\usepackage{permute}
\renewcommand*\pmtseparator{\,}

% colors
\usepackage{xcolor}
\definecolor{p}{HTML}{FFDDDD}
\definecolor{g}{HTML}{D9FFDF}
\definecolor{y}{HTML}{FFFFCF}
\definecolor{b}{HTML}{D9FFFF}
\definecolor{o}{HTML}{FADECB}
%\definecolor{}{HTML}{}

% \highlight[<color>]{<stuff>}
\newcommand{\highlight}[2][p]{\mathchoice%
  {\colorbox{#1}{$\displaystyle#2$}}%
  {\colorbox{#1}{$\textstyle#2$}}%
  {\colorbox{#1}{$\scriptstyle#2$}}%
  {\colorbox{#1}{$\scriptscriptstyle#2$}}}%

% header/footer formatting
\pagestyle{fancy}
\fancyhead{}
\fancyfoot{}
\fancyhead[L]{MTG6346 Topology}
\fancyhead[C]{Homework 2}
\fancyhead[R]{Sai Sivakumar}
\fancyfoot[R]{\thepage}
\renewcommand{\headrulewidth}{1pt}

% paragraph indentation/spacing
\setlength{\parindent}{0cm}
\setlength{\parskip}{10pt}
\renewcommand{\baselinestretch}{1.25}

% extra commands defined here
\newcommand{\br}[1]{\left(#1\right)}
\newcommand{\sbr}[1]{\left[#1\right]}
\newcommand{\cbr}[1]{\left\{#1\right\}}

\newcommand{\catname}[1]{{
  %\normalfont
  \textbf{#1}}}
\newcommand{\Set}{\catname{Set}}
\newcommand{\Top}{\catname{Top}}
\DeclareMathOperator{\Int}{Int}
\DeclareMathOperator{\Bd}{Bd}
\DeclareMathOperator{\id}{id}
\DeclareMathOperator{\im}{im}

% bracket notation for inner product
\usepackage{mathtools}

\DeclarePairedDelimiterX{\abr}[1]{\langle}{\rangle}{#1}
\DeclarePairedDelimiter{\ceil}{\lceil}{\rceil}
\DeclarePairedDelimiter{\floor}{\lfloor}{\rfloor}

% set page count index to begin from 1
\setcounter{page}{1}

\begin{document}
\begin{enumerate}
    \item Read Section 2.8. State that you've read Section 2.8 or part of Section 2.8 and give yourself a score out of 5 based on how much you have read.
    
    I read Section 2.8; 5/5.
    \item (2.8.1) Let $S^1\subset \mathbb{R}^2\times 0\subset \mathbb{R}^3$ be the standard circle. Let $D = \cbr{(0,0,t)\mid -2\leq t\leq 2}$ and $S^2(2) = \cbr{x\in\mathbb{R}^3\mid \norm{x} = 2}$. Then $S^2(2)\cup D$ is a deformation retract of $X = \mathbb{R}^3\setminus S^1$. The space $X$ is h-equivalent to $S^2\vee S^1$. \begin{proof}
      We describe the deformation retraction via the following pictures: \vspace*{5cm}

      For each $\theta\in [0,2\pi)$ we specify the deformation retract within the closed half plane $H^+_\theta$ given in green above. Points on $S^2(2)\cup D$ remain fixed throughout the homotopy. For points outside of $S^2(2)$, we follow the standard deformation retract and draw a line from such points to the origin and drag them along via the straight line homotopy until they hit $S^2(2)$. For points inside the sphere minus the point $p$ where $S^1$ intersected $H^+_\theta$, draw a line from $p$ towards such points and drag those points along the straight line homotopy until they hit either $D$ or $S^2(2)$ and map them there.

      To see that $X$ is h-equivalent to the wedge of $S^2$ and $S^1$ we use the fact that deformation retracts are h-equivalences and that composing h-equivalences are still h-equivalences. So follow the deformation retract outlined earlier by the homotopy which drags the top point of $D$ to its bottom point, outlined in the below pictures: \vspace*{5cm}
    \end{proof}
    \item (2.8.4) Let $i_{0\ast}$ in (2.6.2) be an isomorphism. Then $j_{1\ast}$ is an isomorphism. This statement is a
    general formal property of pushouts. If $i_{0\ast}$ is surjective, then $j_{1\ast}$ is surjective. \begin{proof} We prove the above statements for pushouts in general; that is, that pushouts of isomorphisms (epimorphisms) are isomorphisms (epimorphisms). We start with the following pushout: % https://q.uiver.app/?q=WzAsNCxbMCwwLCJaIl0sWzIsMCwiWCJdLFswLDIsIlkiXSxbMiwyLCJQIl0sWzAsMSwiZiJdLFswLDIsImciLDJdLFsyLDMsImpfMiIsMl0sWzEsMywial8xIl1d
      \[\begin{tikzcd}
        Z && X \\
        \\
        Y && P
        \arrow["f", from=1-1, to=1-3]
        \arrow["g"', from=1-1, to=3-1]
        \arrow["{j_2}"', from=3-1, to=3-3]
        \arrow["{j_1}", from=1-3, to=3-3]
      \end{tikzcd}\]

    For the first, observe that when $g$ is an isomorphism the following diagram commutes, using the universal property of pushouts:
      % https://q.uiver.app/?q=WzAsNixbMCwwLCJaIl0sWzIsMCwiWCJdLFswLDIsIlkiXSxbMiwyLCJQIl0sWzMsMywiWCJdLFszLDQsIlAiXSxbMCwxLCJmIl0sWzAsMiwiZyIsMl0sWzIsMywial8yIiwyXSxbMSwzLCJqXzEiXSxbMyw0LCJ1IiwwLHsic3R5bGUiOnsiYm9keSI6eyJuYW1lIjoiZGFzaGVkIn19fV0sWzIsNCwiZlxcY2lyYyBnXnstMX0iLDIseyJjdXJ2ZSI6Mn1dLFs0LDUsImpfMSJdLFsxLDQsIlxcaWRfWCIsMCx7ImN1cnZlIjotMn1dLFszLDUsIlxcaWRfUCIsMCx7ImN1cnZlIjotNX1dXQ==
\[\begin{tikzcd}
	Z && X \\
	\\
	Y && P \\
	&&& X \\
	&&& P
	\arrow["f", from=1-1, to=1-3]
	\arrow["g"', from=1-1, to=3-1]
	\arrow["{j_2}"', from=3-1, to=3-3]
	\arrow["{j_1}", from=1-3, to=3-3]
	\arrow["u", dashed, from=3-3, to=4-4]
	\arrow["{f\circ g^{-1}}"', bend right=24, from=3-1, to=4-4]
	\arrow["{j_1}", from=4-4, to=5-4]
	\arrow["{\id_X}", bend left=24, from=1-3, to=4-4]
	\arrow["{\id_P}", bend left=90, from=3-3, to=5-4]
\end{tikzcd}\] It follows that $j_1$ has left and right inverses, so it is an isomorphism.

For the second, assume $g$ is an epimorphism and let $h_1,h_2\colon P\to P^\prime$ be morphisms such that $h_1j_1 = h_2j_1$. Then $h_1j_1f = h_1j_2g = h_2j_2g = h_2j_1f$, and since $g$ is an epimorphism, $h_1j_2 = h_2j_2$. Now apply the universal property of the pushout to obtain the diagram % https://q.uiver.app/?q=WzAsNSxbMCwwLCJaIl0sWzIsMCwiWCJdLFswLDIsIlkiXSxbMiwyLCJQIl0sWzMsMywiUF5cXHByaW1lIl0sWzAsMSwiZiJdLFswLDIsImciLDJdLFsyLDMsImpfMiIsMl0sWzEsMywial8xIl0sWzMsNCwiaCIsMCx7InN0eWxlIjp7ImJvZHkiOnsibmFtZSI6ImRhc2hlZCJ9fX1dLFsxLDQsImhqXzEgPSBoXzFqXzEgPSBoXzJqXzEiLDAseyJjdXJ2ZSI6LTJ9XSxbMiw0LCJoal8yID0gaF8xal8yID0gaF8yal8yIiwyLHsiY3VydmUiOjJ9XV0=
\[\begin{tikzcd}
	Z && X \\
	\\
	Y && P \\
	&&& {P^\prime}
	\arrow["f", from=1-1, to=1-3]
	\arrow["g"', from=1-1, to=3-1]
	\arrow["{j_2}"', from=3-1, to=3-3]
	\arrow["{j_1}", from=1-3, to=3-3]
	\arrow["h", dashed, from=3-3, to=4-4]
	\arrow["{hj_1 = h_1j_1 = h_2j_1}", bend left=24, from=1-3, to=4-4]
	\arrow["{hj_2 = h_1j_2 = h_2j_2}"', bend right=24, from=3-1, to=4-4]
\end{tikzcd}\] where $h$ is unique so that $h_1 = h_2 = h$ as desired. It follows that $j_1$ is an epimorphism.

  Since (2.6.2) is a pushout the above is true for $j_{1\ast}$ whenever $i_{0\ast}$ is an isomorphism (epimorphism).
    \end{proof}
    \item (2.8.5, last statement) ... we obtain $\pi_1(P^2)\cong \mathbb{Z}/2$. \begin{proof}
      We use the following pushout diagram which tells us we can obtain $P^2$ from $S^1\cong P^1$ by attaching a $2$-cell: % https://q.uiver.app/?q=WzAsNCxbMCwwLCJTXjEiXSxbMiwwLCJQXjEiXSxbMCwyLCJZIl0sWzIsMiwiUF4yIl0sWzAsMSwiXFx2YXJwaGkiXSxbMCwyLCJqIl0sWzIsMywiXFxQaGkiXSxbMSwzLCJKIl1d
      \[\begin{tikzcd}
        {S^1} && {P^1} \\
        \\
        D^2 && {P^2}
        \arrow["\varphi", from=1-1, to=1-3]
        \arrow["j", from=1-1, to=3-1]
        \arrow["\Phi", from=3-1, to=3-3]
        \arrow["J", from=1-3, to=3-3]
      \end{tikzcd}\] We use the Seifert-van Kampen theorem due to the discussion in (2.8.10), and so we obtain the following diagram: % https://q.uiver.app/?q=WzAsNCxbMCwwLCJcXHBpXzEoU14xKVxcY29uZyBcXG1hdGhiYntafSJdLFsyLDAsIlxccGlfMShQXjEpXFxjb25nXFxtYXRoYmJ7Wn0iXSxbMCwyLCJcXHBpXzEoRF4yKVxcY29uZyAxIl0sWzIsMiwiXFxwaV8xKFBeMikiXSxbMCwxLCJcXHZhcnBoaV9cXGFzdCJdLFswLDIsImpfXFxhc3QiXSxbMiwzLCJcXFBoaV9cXGFzdCJdLFsxLDMsIkpfXFxhc3QiXV0=
      \[\begin{tikzcd}
        {\pi_1(S^1)\cong \mathbb{Z}} && {\pi_1(P^1)\cong\mathbb{Z}} \\
        \\
        {\pi_1(D^2)\cong 1} && {\pi_1(P^2)}
        \arrow["{\varphi_\ast}", from=1-1, to=1-3]
        \arrow["{j_\ast}", from=1-1, to=3-1]
        \arrow["{\Phi_\ast}", from=3-1, to=3-3]
        \arrow["{J_\ast}", from=1-3, to=3-3]
      \end{tikzcd}\] and so the fundamental group of $P^2$ is isomorphic to $\pi_1(P^1)/\langle\varphi\rangle$ where $\langle\varphi\rangle$ denotes the normal subgroup generated by the image of $\varphi_\ast$. In terms of generators and relations, we take the amalgamated free product of $\langle a \rangle = \pi_1(S)$ and $1$ to obtain $\langle a\mid \varphi_\ast(a) = e \rangle$. We deduce $\varphi_\ast(a)$ by taking a generator of $S^1$ (the identity $e^{i\theta}\mapsto e^{i\theta}$, a loop) and seeing that its image under $\varphi$ is $[\cos(\theta),\sin(\theta)]$, but to go back to $S^1$ (from which we obtained the fundamental group) we apply the specified homeomorphism (given in problem statement) to see that $[\cos(\theta),\sin(\theta)]$ maps to $e^{i(2\theta)}$. (If $\theta\in[0,\pi)$ then there is no need to choose a representative; otherwise take the representative to be $[\cos(\theta-\pi),\sin(\theta-\pi)]$ and map this to $e^{i(2[\theta-\pi])} = e^{i(2\theta)}$) It follows that in the fundamental groups, $a\mapsto a^2$ so that $\pi_1(P^2) = \langle a\mid a^2 = e \rangle$, which is isomorphic to $\mathbb{Z}/2\mathbb{Z}$.
    \end{proof}
    \item (2.8.6) The Klein bottle $K$ can be obtained from two M\"obius bands $M$ by an identification of their boundary curves with a homeomorphism, $K = M\cup_{\partial M} M$. 
    
    Apply the theorem of Seifert and van Kampen and obtain the presentation $\pi_1(K) = \langle a,b\mid a^2 = b^2 \rangle$. The elements $a^2,ab$ generate a free abelian subgroup of rank $2$ and of index $2$ in the fundamental group. The element $a^2$ generates the center of this group, it is represented by the central loop $\partial M$. The quotient by the center is isomorphic to $\mathbb{Z}/2\ast \mathbb{Z}/2$. \begin{proof}
      To show that the fundamental group of the Klein bottle $K$ is given by $G = \pi_1(K) = \langle a,b\mid a^2 = b^2 \rangle$, we first view $K$ as the pushout of two inclusions of $\partial M$ into $M$: % https://q.uiver.app/?q=WzAsNCxbMCwwLCJcXHBhcnRpYWwgTSJdLFsyLDAsIk0iXSxbMCwyLCJNIl0sWzIsMiwiTVxcY3VwX3tcXHBhcnRpYWwgTX0gTSJdLFswLDEsIiIsMCx7InN0eWxlIjp7InRhaWwiOnsibmFtZSI6Imhvb2siLCJzaWRlIjoidG9wIn19fV0sWzAsMiwiIiwyLHsic3R5bGUiOnsidGFpbCI6eyJuYW1lIjoiaG9vayIsInNpZGUiOiJ0b3AifX19XSxbMiwzLCIiLDIseyJzdHlsZSI6eyJ0YWlsIjp7Im5hbWUiOiJob29rIiwic2lkZSI6InRvcCJ9fX1dLFsxLDMsIiIsMCx7InN0eWxlIjp7InRhaWwiOnsibmFtZSI6Imhvb2siLCJzaWRlIjoidG9wIn19fV1d
      \[\begin{tikzcd}
        {\partial M} && M \\
        \\
        M && {M\cup_{\partial M} M}
        \arrow[hook, from=1-1, to=1-3]
        \arrow[hook, from=1-1, to=3-1]
        \arrow[from=3-1, to=3-3]
        \arrow[from=1-3, to=3-3]
      \end{tikzcd}\] With $\partial M$ homeomorphic to $S^1$ and $M$ homotopic to $S^1$, we apply Seifert-van Kampen and obtain another pushout: % https://q.uiver.app/?q=WzAsNCxbMCwwLCJcXHBpXzEoXFxwYXJ0aWFsIE0pXFxjb25nIFxcYWJye2N9Il0sWzIsMCwiXFxwaV8xKE0pXFxjb25nIFxcYWJye2J9Il0sWzAsMiwiXFxwaV8xKE0pXFxjb25nIFxcYWJye2F9Il0sWzIsMiwiXFxwaV8xKE1cXGN1cF97XFxwYXJ0aWFsIE19IE0pIl0sWzAsMV0sWzAsMl0sWzIsM10sWzEsM11d
      \[\begin{tikzcd}
        {\pi_1(\partial M)\cong \abr{c}} && {\pi_1(M)\cong \abr{b}} \\
        \\
        {\pi_1(M)\cong \abr{a}} && {\pi_1(M\cup_{\partial M} M)}
        \arrow[from=1-1, to=1-3]
        \arrow[from=1-1, to=3-1]
        \arrow[from=3-1, to=3-3]
        \arrow[from=1-3, to=3-3]
      \end{tikzcd}\] The generator $c$ represents going along the boundary of $M$, which amounts to going around the center line of $M$ twice. As a result the image of $c$ under the induced maps from $\pi_1(\partial M)$ into $\pi_1(M)$ are $a^2$ and $b^2$. Hence the fundamental group of $K = M\cup_{\partial M}M$ is presented as $\abr{a,b\mid a^2b^{-2} = e}$ as desired.

      The subgroup $H$ generated by $a^2,ab$ is torsion free because the only relation imposed on these generators is that $a^2 =b^2$, which could not cause a finite word to be the neutral element. To show commutativity, it suffices to show it for the generators: $a^2ab = aa^2b = ab^2b = abb^2 = aba^2$. It follows that $H$ is free and abelian. Observe also that since $H$ is generated by $a^2 = b^2$ and $ab$, it follows that $H$ contains only the words of even length in $G$, and exactly those (if $g\in G$ has even length, then by inserting in an even number of symbols and using the above identities to collect $g$ into a product of generators of $H$, we find that $g\in H$.) Then the remaining words of $G$ of even length may be obtained by prepending $a$ to elements of $H$. Hence $H,aH$ are the only two cosets of $H$ in $G$, so $H$ has index $2$ in $G$.

      The center $Z$ is given by elements which commute with every element of $G$, in particular with the generators of $G$. Observe that $ab\neq ba$, so that if an element $p$ were to commute with $a$ or $b$, it must not be $a$ or $b$. But observe that $a^2 = b^2$ will commute with $a$ and $b$. So an element $p$ commutes with $a$ and $b$ if it is the product of finitely many $a^2$. If $p$ is of odd length then at some point $p$ will cease to commute with $a$ and $b$. Hence $Z = \abr{a^2}$. In terms of generators and relations, $G/Z$ is given by $\abr{a,b\mid a^2 = b^2 = e}$, which is isomorphic to any presentation of $\mathbb{Z}/2\ast \mathbb{Z}/2$.
    \end{proof}

    The space $M/\partial M$ is homeomorphic to the projective plane $P^2$. If we identify the central $\partial M$ to a point, we obtain a map $q\colon K = M \cup_{\partial M} M \to P^2\vee P^2$. The induced map on the fundamental group is the homomorphism onto $\mathbb{Z}/2\ast \mathbb{Z}/2$. \begin{proof}
      Pictographically, we take $M$ and identify its boundary to a point, and see that we obtain a sphere with antipodal points identified, which is the definition of $P^2$: \vspace*{5cm}

      Then in $K$ if we identify the central $\partial M$ to a point, it is the same as taking two M\"obius strips $M$ and quotienting out by $\partial M$, and then taking their wedge at the point $\partial M$ was identified with. Using the previous result, it follows that quotienting out the central $\partial M$ from $K$ yields a space homeomorphic to $P^2\vee P^2$; thus there is a map $q\colon K = M \cup_{\partial M} M \to P^2\vee P^2$ which is the quotient map for the above composed with the appropriate homeomorphism, still a quotient map of spaces (it is surjective). Thus by Seifert-van Kampen, it follows that the induced homomorphism of fundamental groups is also surjective, and concretely the effect is to quotient out by the center $Z$ of $G$. So $q_\ast\colon G\to G/Z\cong \mathbb{Z}/2\ast \mathbb{Z}/2$ is the (surjective) quotient map of groups.
    \end{proof}
\end{enumerate}
\end{document}