\documentclass[10pt]{amsart}

\usepackage{amsmath,amsthm,amsfonts,amssymb,epsfig,graphicx,soul,tikz}
\usepackage{xskak}


\usepackage{color}
\usepackage{graphics,subfigure}

\bibliographystyle{plain}

%\textheight 24cm
%\textwidth 17cm
%\topmargin -3cm
%\oddsidemargin -0.1cm
%\evensidemargin 5cm

\newcommand{\reals}{\mathbb{R}}
\newcommand{\naturals}{\mathbb{N}}
\newcommand{\complex}{\mathbb{C}}
\newcommand{\integers}{\mathbb{Z}}
\newcommand{\banach}{\mathbb{B}}
\newcommand{\prob}{\mathbb{P}}
\newcommand{\expect}{\mathbb{E}}
\newcommand{\exponent}{\operatorname{e}}
\newcommand{\md}{\mathrm{d}}
\newcommand{\p}{\partial}
\newcommand{\e}{\varepsilon}
\newcommand{\no}{\noindent}
\newcommand{\UU}{\mathcal U}

%\theoremstyle{plain}
%\newtheorem{defi}{Definition}
%\newtheorem{prop}{Proposition}
\newtheorem{thm}{Theorem}
\newtheorem{corol}{Corollary}
\newtheorem{lemma}{Lemma}
\theoremstyle{remark}
\newtheorem{opm}{Remark}

% \newcommand{\comment}[1]{\textbf{[#1]}}

\begin{document}

\title{STEAM NIGHT AT NORTON ELEMENTARY}


\maketitle

\section{Eight queens}
\noindent {\bf Goal:} Place {\bf eight} chess queens on an $\mathbf {8\times 8}$ chessboard so that no two queens threaten each other. 

Recall that a queen can move any number of squares in a straight line vertically, horizontally, or diagonally: 

\begin{center}
    \chessboard[showmover=false,setpieces={qe5},pgfstyle=color,
    opacity=0.3,
    color=green,
    markfield={a5, b5, c5, d5, f5, g5, h5, e1, e2, e3, e4, e6, e7, e8, b8, c7, d6, f4, g3, h2, a1, b2, c3, d4, f6, g7, h8}]
\end{center}

So, the following chessboard is {\bf NOT} a valid solution:

\begin{center}
    \chessboard[showmover=false,setpieces={qa8, qc7, qb5, qc3, qe4, qg6, qf1, qh2}]
\end{center}

Indeed, the queen on square \textsf{a8} threatens the queen on square \textsf{e4}, for example. 

Consider a related problem where you attempt to place {\bf four} queens on a $\mathbf{4\times 4}$ chessboard so that no queens threaten each other.

\begin{center}
    \chessboard[maxfield=d4,showmover=false]
\end{center}

How many different ways can you solve this problem? How many different ways can you solve the problem for {\bf six} queens on a $\mathbf{6\times 6}$ chessboard? How about for our original {\bf eight} queens on an $\mathbf{8\times 8}$ chessboard? Maybe consider even more queens on an even bigger chessboard?

How many knights can you place on an $8\times 8$ chessboard that do not threaten each other? How about for other pieces?

A blank chessboard:

\begin{center}
    \chessboard[showmover=false]
\end{center}

\end{document}





 




