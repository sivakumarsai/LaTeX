\documentclass[11pt]{article}

% packages
\usepackage{physics}
% margin spacing
\usepackage[top=1in, bottom=1in, left=0.5in, right=0.5in]{geometry}
\usepackage{hanging}
\usepackage{amsfonts, amsmath, amssymb, amsthm}
\usepackage[none]{hyphenat}
\usepackage{fancyhdr}
\usepackage[nottoc, notlot, notlof]{tocbibind}
\usepackage{graphicx}
\graphicspath{{./images/}}
\usepackage{float}
\usepackage{siunitx}
\usepackage{esint}
\usepackage{cancel}

% header/footer formatting
\pagestyle{fancy}
\fancyhead{}
\fancyfoot{}
\fancyhead[L]{MAS4105 Dr. Zhang}
\fancyhead[C]{HW5}
\fancyhead[R]{Sai Sivakumar}
\fancyfoot[R]{\thepage}
\renewcommand{\headrulewidth}{0pt}

% paragraph indentation/spacing
\setlength{\parindent}{0cm}
\setlength{\parskip}{5pt}
\renewcommand{\baselinestretch}{1.25}

% extra commands defined here
\newcommand{\ihat}{\boldsymbol{\hat{\textbf{\i}}}}
\newcommand{\jhat}{\boldsymbol{\hat{\textbf{\j}}}}
\newcommand{\dr}{\vec{r}~^{\prime}(t)}
\newcommand{\dx}{x^{\prime}(t)}
\newcommand{\dy}{y^{\prime}(t)}

\newcommand{\br}[1]{\left(#1\right)}
\newcommand{\sbr}[1]{\left[#1\right]}
\newcommand{\cbr}[1]{\{#1\}}

\newcommand{\dprime}{\prime\prime}
\newcommand{\lap}[2]{\mathcal{L}[#1](#2)}

% bracket notation for inner product
\usepackage{mathtools}

\DeclarePairedDelimiterX{\abr}[1]{\langle}{\rangle}{#1}

\DeclareMathOperator{\Span}{span}

% set page count index to begin from 1
\setcounter{page}{1}

\begin{document}

Page 40: 5, 9, 12, 13

Page 40:

5. Show that the set $\cbr{1,x,x^2,\dots,x^n}$ is linearly independent in $\mathsf{P}_n(\mathbb{F})$.
\begin{proof}
One equivalent condition for vectors to be linearly independent is that the \textit{only} linear combination that produces the zero vector is the one where all the coefficients in the combination are taken to be zero (because if not, then you can represent one vector as a linear combination of the other ones and reach a contradiction).

So consider the following linear combination equal to zero: 
$$a_0x^0 + a_1x^2 + a_2x^2 + \cdots + a_nx^n = 0$$

This holds for all $x\mathbb{F}$, so we may take $x$ to be some nonzero value. Then we can divide through by $x^n$ to find the following:
$$a_0x^{-n} + a_1x^{1-n} + a_2x^{2-n} + \cdots + a_n = 0$$

Take the limit as $x$ tends to any infinity:
$$\lim_{x\to\infty} \br{a_0x^{-n} + a_1x^{1-n} + a_2x^{2-n} + \cdots + a_n} = a_n = 0$$

So then we may rewrite the polynomial like so:
$$a_0x^0 + a_1x^2 + a_2x^2 + \cdots + (0)x^n = a_0x^0 + a_1x^2 + a_2x^2 + \cdots + a_{n-1}x^{n-1} = 0$$

Without any complications we may repeat the exact same process to deduce that in fact \textit{all} of the coefficients $a_0,a_1,a_2,\dots, a_n$ are $0$. This then means that the set $\cbr{1,x,x^2,\dots,x^n}$ is linearly independent.
\end{proof}

9. Let $u$ and $v$ be distinct vectors in a vector space $\mathsf{V}$. Show that $\cbr{u, v}$ is linearly dependent if and only if $u$ or $v$ is a multiple of the other.

\begin{proof}
    Consider the case when either $u$ or $v$ is the zero vector. Then we can produce a nontrivial linear combination of $u$ and $v$ that is equal to the zero vector. Then consider the case $u$ and $v$ are both \textit{not} the zero vector.

    Let $\cbr{u,v}$ is linearly dependent. Then by definition there is a nontrivial combination (for $a,b\in\mathbb{F}$, not all zero) where $au+bv = 0$. Without loss of generality take $a\neq0$, so that we may rearrange the equation into $u = \frac{-b}{a}v$. We have that $u$ is a multiple of $v$.

    In the other direction, let $c\in\mathbb{F}$ so that $u = cv$. Then we may rearrange the equation into: $u-cv = 0$. 
    
    Since the coefficient of $u$ is $1$, we have a nontrivial combination of $u$ and $v$ that produces the zero vector, so by definition $\cbr{u,v}$ is linearly dependent.
\end{proof}

12. \textbf{Theorem 1.6.} Let $\mathsf{V}$ be a vector space, and let $S_1\subseteq S_2 \subseteq \mathsf{V}$. If $S_1$ is linearly dependent, then $S_2$ is linearly dependent.

\begin{proof} Let $S_1 = \cbr{v_1,v_2,\dots,v_m}$ and $S_2 = \cbr{v_1,v_2,\dots,v_n}$ where $n\geq m$ so that $S_1\subseteq S_2$.

    If $S_1$ is linearly dependent, then that means that there exist a choice of scalars $a_1,a_2,\dots,a_m\in\mathbb{F}$ \textit{not all zero} such that $$a_1v_1+a_2v_2+\cdots+a_mv_m = 0$$

    From this equality we may simply add to both sides the linear combination of $v_{m+1},\dots,v_n$ where all the coefficients are chosen to be zero to find:
    $$a_1v_1+a_2v_2+\cdots+a_mv_m + (0)v_{m+1} + \cdots + (0)v_n = 0$$

    Since we know that at least one of $a_1,a_2,\dots,a_m$ is nonzero and that $S_1\subseteq S_2$, the above equation constitutes a nontrivial linear combination of vectors in $S_2$ and so by definition $S_2$ is linearly dependent.
\end{proof}

\textbf{Corollary.} Let $\mathsf{V}$ be a vector space, and let $S_1\subseteq S_2 \subseteq \mathsf{V}$. If $S_2$ is linearly independent, then $S_1$ is linearly independent.

\begin{proof}
    This statement is logically equivalent by the contrapositive to Theorem 1.6., so no further proof is required.
\end{proof}

13. Let $\mathsf{V}$ be a vector space over a field of characteristic not equal to two.

\textbf{(a)} Let $u$ and $v$ be distinct vectors in $\mathsf{V}$. Prove that $\cbr{u,v}$ is linearly independent if and only if $\cbr{u+v,u-v}$ is linearly independent.

\begin{proof}
    Suppose $\cbr{u,v}$ is linearly independent. Then by way of contradiction suppose that $\cbr{u+v,u-v}$ is not linearly independent, that is, we have scalars $a,b\in\mathbb{F}$ not all zero such that: $$a\br{u+v}+b\br{u-v} = 0$$ With some algebra we have: $$\br{a+b}u+\br{a-b}v = 0$$ and since $a,b$ are not both zero by closure we have that at least one of $a+b$ or $a-b$ is nonzero, which contradicts the hypothesis that $\cbr{u,v}$ is linearly independent.

    In the other direction, take the contrapositive. So we wish to show that if $\cbr{u,v}$ is linearly dependent, then $\cbr{u+v,u-v}$ is also linearly dependent.

    Suppose $\cbr{u,v}$ is linearly dependent. Then there exist $a,b\in\mathbb{F}$ not both zero such that $au+bv = 0$. Observe that $2u = \br{u+v} + \br{u-v}$ and $2v = \br{u+v} - \br{u-v}$. Then we may rewrite the first linear combination as: $$2au+2bv = a\br{\br{u+v} + \br{u-v}} + b\br{\br{u+v} - \br{u-v}} = \br{a+b}\br{u+v} + \br{a-b}\br{u-v} = 0$$

    Since the last equality is a nontrivial linear combination (because $a-b$ or $a+b$ is nonzero) of $\cbr{u+v,u-v}$, we have that $\cbr{u+v,u-v}$ is linearly dependent as a result.

    Therefore $\cbr{u,v}$ is linearly independent if and only if $\cbr{u+v,u-v}$ is linearly independent.
\end{proof}

\textbf{(b)} Let $u$, $v$, and $w$ be distinct vectors in $\mathsf{V}$. Prove that $\cbr{u,v,w}$ is linearly independent if and only if $\cbr{u+v,u+w,v+w}$ is linearly independent.

\begin{proof}
    Suppose $\cbr{u,v,w}$ is linearly independent. Then by way of contradiction suppose that $\cbr{u+v,u+w,v+w}$ is linearly dependent, so that there exist scalars $a,b,c\in\mathbb{F}$, not all zero, such that $a(u+v)+b(u+w)+c(v+w) = 0$. Then some algebra reveals that $\br{a+b}u + \br{a+c}v + \br{b+c}w = 0$. However, at least one of $\br{a+b}, \br{a+c}, \br{b+c}$ is nonzero, which is a contradiction. Hence $\cbr{u+v,u+w,v+w}$ is linearly independent.

    In the other direction we again use the contrapositive. Suppose $\cbr{u,v,w}$ is linearly dependent. Then there exist scalars $a,b,c\in\mathbb{F}$ not all zero such that $au+bv+cw = 2au+2bv+2cw = 0$. Notice that 
    \begin{align*}
        2u &= \br{u+v}-\br{v+w}+\br{u+w} \\
        2v &= \br{u+v}+\br{v+w}-\br{u+w} \\
        2w &= -\br{u+v}+\br{v+w}+\br{u+w},
    \end{align*}
    so that we also have
    \begin{multline*}
        2au+2bv+2cw = a\br{\br{u+v}-\br{v+w}+\br{u+w}} \\+ b\br{\br{u+v}+\br{v+w}-\br{u+w}} + c\br{-\br{u+v}+\br{v+w}+\br{u+w}} 
    \end{multline*}
    \begin{align*}
        &= \br{a+b-c}\br{u+v} + \br{-a+b+c}\br{v+w} + \br{a-b+c}\br{u+w}\\
        &= 0
    \end{align*}

    Since at least one of $\br{a+b-c}, \br{-a+b+c}, \br{a-b+c}$ is not zero, we can deduce that $\cbr{u+v,u+w,v+w}$ is linearly dependent.

Thus $\cbr{u,v,w}$ is linearly independent if and only if $\cbr{u+v,u+w,v+w}$ is linearly independent.
\end{proof}

\end{document}