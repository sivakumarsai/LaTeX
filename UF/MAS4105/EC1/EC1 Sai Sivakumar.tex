\documentclass[11pt]{article}

% packages
\usepackage{physics}
% margin spacing
\usepackage[top=1in, bottom=1in, left=0.5in, right=0.5in]{geometry}
\usepackage{hanging}
\usepackage{amsfonts, amsmath, amssymb, amsthm}
\usepackage{systeme}
\usepackage[none]{hyphenat}
\usepackage{fancyhdr}
\usepackage[nottoc, notlot, notlof]{tocbibind}
\usepackage{graphicx}
\graphicspath{{./images/}}
\usepackage{float}
\usepackage{siunitx}
\usepackage{esint}
\usepackage{cancel}
\usepackage{color}

% header/footer formatting
\pagestyle{fancy}
\fancyhead{}
\fancyfoot{}
\fancyhead[L]{MAS4105 Dr. Zhang}
\fancyhead[C]{EC1 (Revised)}
\fancyhead[R]{Sai Sivakumar}
\fancyfoot[R]{\thepage}
\renewcommand{\headrulewidth}{0pt}

% paragraph indentation/spacing
\setlength{\parindent}{0cm}
\setlength{\parskip}{5pt}
\renewcommand{\baselinestretch}{1.25}

% extra commands defined here
\newcommand{\ihat}{\boldsymbol{\hat{\textbf{\i}}}}
\newcommand{\jhat}{\boldsymbol{\hat{\textbf{\j}}}}
\newcommand{\dr}{\vec{r}~^{\prime}(t)}
\newcommand{\dx}{x^{\prime}(t)}
\newcommand{\dy}{y^{\prime}(t)}

\newcommand{\br}[1]{\left(#1\right)}
\newcommand{\sbr}[1]{\left[#1\right]}
\newcommand{\cbr}[1]{\{#1\}}

\newcommand{\dprime}{\prime\prime}
\newcommand{\lap}[2]{\mathcal{L}[#1](#2)}

% bracket notation for inner product
\usepackage{mathtools}

\DeclarePairedDelimiterX{\abr}[1]{\langle}{\rangle}{#1}

\DeclareMathOperator{\Span}{span}

% set page count index to begin from 1
\setcounter{page}{1}

\begin{document}

1.5: 13 (b) , and 1.6: 22 \\

1.5 13 (b). Let $\mathsf{V}$ be a vector space over a field of characteristic not equal to two.

Let $\vec{u}$, $\vec{v}$, and $\vec{w}$ be distinct vectors in $\mathsf{V}$. Prove that $\cbr{\vec{u},\vec{v},\vec{w}}$ is linearly independent if and only if $\cbr{\vec{u}+\vec{v},\vec{u}+\vec{w},\vec{v}+\vec{w}}$ is linearly independent.

\begin{proof}
    Suppose $\cbr{\vec{u},\vec{v},\vec{w}}$ is linearly independent. Then by way of contradiction suppose that $\cbr{\vec{u}+\vec{v},\vec{u}+\vec{w},\vec{v}+\vec{w}}$ is linearly dependent, so that there exist scalars $a,b,c\in\mathbb{F}$, not all zero, such that $a(\vec{u}+\vec{v})+b(\vec{u}+\vec{w})+c(\vec{v}+\vec{w}) = 0$. Then by some algebra $\br{a+b}\vec{u} + \br{a+c}\vec{v} + \br{b+c}\vec{w} = 0$. At least one of $\br{a+b}, \br{a+c}, \br{b+c}$ is nonzero (since one of $a,b,c$ is nonzero). This is in contradiction with the assumption that $\cbr{\vec{u},\vec{v},\vec{w}}$ was linearly independent. Hence $\cbr{\vec{u}+\vec{v},\vec{u}+\vec{w},\vec{v}+\vec{w}}$ is linearly independent.

    For the converse, use the contrapositive. Suppose $\cbr{\vec{u},\vec{v},\vec{w}}$ is linearly dependent. Then there exist scalars $a,b,c\in\mathbb{F}$ not all zero such that $a\vec{u}+b\vec{v}+c\vec{w} = 2a\vec{u}+2b\vec{v}+2c\vec{w} = 0$. Notice that 
    \begin{align*}
        2\vec{u} &= \br{\vec{u}+\vec{v}}-\br{\vec{v}+\vec{w}}+\br{\vec{u}+\vec{w}} \\
        2\vec{v} &= \br{\vec{u}+\vec{v}}+\br{\vec{v}+\vec{w}}-\br{\vec{u}+\vec{w}} \\
        2\vec{w} &= -\br{\vec{u}+\vec{v}}+\br{\vec{v}+\vec{w}}+\br{\vec{u}+\vec{w}},
    \end{align*}
    so that we also have
    \begin{multline*}
        2a\vec{u}+2b\vec{v}+2c\vec{w} = a\br{\br{\vec{u}+\vec{v}}-\br{\vec{v}+\vec{w}}+\br{\vec{u}+\vec{w}}} \\+ b\br{\br{\vec{u}+\vec{v}}+\br{\vec{v}+\vec{w}}-\br{\vec{u}+\vec{w}}} + c\br{-\br{\vec{u}+\vec{v}}+\br{\vec{v}+\vec{w}}+\br{\vec{u}+\vec{w}}} 
    \end{multline*}
    \begin{align*}
        &= \br{a+b-c}\br{\vec{u}+\vec{v}} + \br{-a+b+c}\br{\vec{v}+\vec{w}} + \br{a-b+c}\br{\vec{u}+\vec{w}}\\
        &= 0
    \end{align*}

    \color{red}{\textbf{Revision:}} \color{black}{}At least one of $\br{a+b-c}, \br{-a+b+c}, \br{a-b+c}$ is not zero (since at least one of $a,b,c$ is not zero). Suppose by way of contradiction that each quantity $\br{a+b-c}, \br{-a+b+c}, \br{a-b+c}$ was equal to zero, which forms the following system of equations:
    \begin{equation*}
        \sysdelim..\systeme{
            a+b-c = 0,
            -a+b+c = 0,
            a-b+c = 0
        }\to
        \sysdelim..\systeme{
            a+0b+0c = 0,
            0a+b+0c = 0,
            0a+0b+c = 0
        }
    \end{equation*}
    
    This is a contradiction with the fact that at least one of $a,b,c$ is not zero. Deduce then that $\cbr{\vec{u}+\vec{v},\vec{u}+\vec{w},\vec{v}+\vec{w}}$ is linearly dependent.

Thus $\cbr{\vec{u},\vec{v},\vec{w}}$ is linearly independent if and only if $\cbr{\vec{u}+\vec{v},\vec{u}+\vec{w},\vec{v}+\vec{w}}$ is linearly independent.
\end{proof}

1.6 22. Let $\mathsf{W}_1$ and $\mathsf{W}_2$ be subspaces of a finite-dimensional vector space $\mathsf{V}$. Determine necessary and sufficient conditions on $\mathsf{W}_1$ and $\mathsf{W}_2$ so that $\dim(\mathsf{W}_1\cap\mathsf{W}_2) = \dim(\mathsf{W}_1)$.

We have $\dim(\mathsf{W}_1\cap\mathsf{W}_2) = \dim(\mathsf{W}_1)$ if and only if $\mathsf{W}_1$ is a subspace of $\mathsf{W}_2$.

\begin{proof}
    Forwards direction. Suppose $\dim(\mathsf{W}_1\cap\mathsf{W}_2) = \dim(\mathsf{W}_1)$. Then assume by way of contradiction that $\mathsf{W}_1$ is not a subspace of $\mathsf{W}_2$.

    Then we can find some vector $\vec{v}$ in $\mathsf{W}_1$ that is not in $\mathsf{W}_2$. Let $\alpha = \cbr{a_1,a_2,\dots,a_n}$ be a basis for $\mathsf{W}_1$ and $\beta = \cbr{b_1,b_2,\dots,b_m}$ be a basis for $\mathsf{W}_2$. Then we can also say that $\vec{v}\notin \Span\br{\beta}$.

    Then let $\gamma$ be a basis for $\mathsf{W}_1\cap\mathsf{W}_2$. It is also true that $\vec{v}\notin \Span\br{\gamma}$, since $\vec{v}\notin \Span\br{\beta}$ - thus $\gamma \cup \cbr{\vec{v}}$ is linearly independent, furthermore, it is a linearly independent subset of $\mathsf{W}_1$. 
    
    We may deduce the following: $\dim(\mathsf{W}_1)\geq \abs{\gamma \cup\cbr{\vec{v}}}$, but $\abs{\gamma \cup\cbr{\vec{v}}} = \abs{\gamma} + 1 = \dim(\mathsf{W}_1\cap\mathsf{W}_2) + 1$. So $\dim(\mathsf{W}_1)\geq \dim(\mathsf{W}_1\cap\mathsf{W}_2) + 1$ implies that $\dim(\mathsf{W}_1)\neq \dim(\mathsf{W}_1\cap\mathsf{W}_2)$, which is in contradiction to the assumption that $\dim(\mathsf{W}_1\cap\mathsf{W}_2) = \dim(\mathsf{W}_1)$. Thus we must have that $\mathsf{W}_1$ is a subspace of $\mathsf{W}_2$.

    To prove the converse we may do so directly. Suppose $\mathsf{W}_1$ is a subspace (and therefore a subset) of $\mathsf{W}_2$. Then we may simplify $\dim(\mathsf{W}_1\cap\mathsf{W}_2)$ into $\dim(\mathsf{W_1})$ because $\mathsf{W}_1\subseteq \mathsf{W}_2$. Therefore we automatically have that $\dim(\mathsf{W}_1\cap\mathsf{W}_2) = \dim(\mathsf{W}_1)$.

    Hence $\dim(\mathsf{W}_1\cap\mathsf{W}_2) = \dim(\mathsf{W}_1)$ if and only if $\mathsf{W}_1$ is a subspace of $\mathsf{W}_2$.
\end{proof}

\end{document}