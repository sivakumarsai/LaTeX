\documentclass[11pt]{article}

% packages
\usepackage{physics}
% margin spacing
\usepackage[top=1in, bottom=1in, left=0.5in, right=0.5in]{geometry}
\usepackage{hanging}
\usepackage{amsfonts, amsmath, amssymb, amsthm}
\usepackage{systeme}
\usepackage[none]{hyphenat}
\usepackage{fancyhdr}
\usepackage[nottoc, notlot, notlof]{tocbibind}
\usepackage{graphicx}
\graphicspath{{./images/}}
\usepackage{float}
\usepackage{siunitx}
\usepackage{esint}
\usepackage{cancel}

% colors
\usepackage{xcolor}
\definecolor{p}{HTML}{FFDDDD}
\definecolor{g}{HTML}{D9FFDF}
\definecolor{y}{HTML}{FFFFCF}
\definecolor{b}{HTML}{D9FFFF}
\definecolor{o}{HTML}{FADECB}
%\definecolor{}{HTML}{}

% \highlight[<color>]{<stuff>}
\newcommand{\highlight}[2][p]{\mathchoice%
  {\colorbox{#1}{$\displaystyle#2$}}%
  {\colorbox{#1}{$\textstyle#2$}}%
  {\colorbox{#1}{$\scriptstyle#2$}}%
  {\colorbox{#1}{$\scriptscriptstyle#2$}}}%

% header/footer formatting
\pagestyle{fancy}
\fancyhead{}
\fancyfoot{}
\fancyhead[L]{MAS4105 Dr. Zhang}
\fancyhead[C]{HW10}
\fancyhead[R]{Sai Sivakumar}
\fancyfoot[R]{\thepage}
\renewcommand{\headrulewidth}{0pt}

% paragraph indentation/spacing
\setlength{\parindent}{0cm}
\setlength{\parskip}{5pt}
\renewcommand{\baselinestretch}{1.25}

% extra commands defined here
\newcommand{\ihat}{\boldsymbol{\hat{\textbf{\i}}}}
\newcommand{\jhat}{\boldsymbol{\hat{\textbf{\j}}}}
\newcommand{\dr}{\vec{r}~^{\prime}(t)}
\newcommand{\dx}{x^{\prime}(t)}
\newcommand{\dy}{y^{\prime}(t)}

\newcommand{\br}[1]{\left(#1\right)}
\newcommand{\sbr}[1]{\left[#1\right]}
\newcommand{\cbr}[1]{\left\{#1\right\}}

\newcommand{\dprime}{\prime\prime}
\newcommand{\lap}[2]{\mathcal{L}[#1](#2)}

% bracket notation for inner product
\usepackage{mathtools}

\DeclarePairedDelimiterX{\abr}[1]{\langle}{\rangle}{#1}

\DeclareMathOperator{\Span}{span}
\DeclareMathOperator{\nullity}{nullity}

% set page count index to begin from 1
\setcounter{page}{1}

% theorem/corollary/lemma
\newtheorem{theorem}{Theorem}[section]
\newtheorem{corollary}{Corollary}[theorem]
\newtheorem{lemma}[theorem]{Lemma}

\begin{document}
2.5: 2(a), 3(a), 7, 8, 13

3.1: 12 \\

2.5:

2. Find the change of coordinate matrix that changes $\beta^{\prime}$-coordinates into $\beta$-coordinates. (This amounts to finding $Q = \sbr{\mathsf{I}}_{\beta^{\prime}}^{\beta}$ such that for $v\in\mathbb{R}^2$, $\sbr{v}_{\beta} = \sbr{\mathsf{I}}_{\beta^{\prime}}^{\beta}\sbr{v}_{\beta^{\prime}}$.)

(a) Note that $(a_1,a_2) = a_1e_1 + a_2e_2$ and $(b_1,b_2) = b_1e_1+b_2e_2$. Then $$Q = \begin{pmatrix}
    a_1 & b_1 \\
    a_2 & b_2
\end{pmatrix}.$$

3. Find the change of coordinate matrix that changes $\beta^{\prime}$-coordinates into $\beta$-coordinates. (This amounts to finding $Q = \sbr{\mathsf{I}}_{\beta^{\prime}}^{\beta}$ such that for $v\in\mathsf{P}_2(\mathbb{R})$, $\sbr{v}_{\beta} = \sbr{\mathsf{I}}_{\beta^{\prime}}^{\beta}\sbr{v}_{\beta^{\prime}}$.)

(a) Note that $a_2x^2+a_1x+a_0 = a_2(x^2)+a_1(x)+a_0(1)$, $b_2x^2+b_1x+b_0 = b_2(x^2)+b_1(x)+b_0(1)$, and $c_2x^2+c_1x+c_0 = c_2(x^2)+c_1(x)+c_0(1)$. Then $$Q = \begin{pmatrix}
    a_2 & b_2 & c_2 \\
    a_1 & b_1 & c_1 \\
    a_0 & b_0 & c_0
\end{pmatrix}.$$

7. In $\mathbb{R}^2$, let $L$ be the line $y=mx$, where $m\neq 0$. Find and expression for $\mathsf{T}(x,y)$, where

(a) $\mathsf{T}$ is the reflection of $\mathbb{R}^2$ about $L$.

Find the change of coordinate matrix $Q$ to the standard basis $\beta = \cbr{e_1,e_2}$ from the basis given by $$\beta^{\prime} = \cbr{\begin{pmatrix} 1 \\ m \end{pmatrix}, \begin{pmatrix} -m \\ 1 \end{pmatrix}},$$ which is $$Q = \begin{pmatrix}
    1 & -m \\
    m & 1
\end{pmatrix}, Q^{-1} = \frac{1}{1+m^2}\begin{pmatrix}
    1 & m \\
    -m & 1
\end{pmatrix}$$

We also know that the reflection is given by $$\sbr{\mathsf{T}}_{\beta^{\prime}} = \begin{pmatrix}
    1 & 0 \\
    0 & -1
\end{pmatrix}.$$ Then by change of coordinates, we have that $Q^{-1}\sbr{\mathsf{T}}_{\beta}Q = \sbr{\mathsf{T}}_{\beta^{\prime}} \implies \sbr{\mathsf{T}}_{\beta} = Q\sbr{\mathsf{T}}_{\beta^{\prime}}Q^{-1}$. So $$\sbr{\mathsf{T}}_{\beta} = \begin{pmatrix}
    1 & -m \\
    m & 1
\end{pmatrix} \begin{pmatrix}
    1 & 0 \\
    0 & -1
\end{pmatrix} \br{\frac{1}{1+m^2}\begin{pmatrix}
    1 & m \\
    -m & 1
\end{pmatrix}} = \frac{1}{1+m^2}\begin{pmatrix}
    1 & -m \\
    m & 1
\end{pmatrix} \begin{pmatrix}
    1 & m \\
    m & -1
\end{pmatrix}$$
$$\sbr{\mathsf{T}}_{\beta} = \frac{1}{1+m^2}\begin{pmatrix}
    1-m^2 & 2m \\
    2m & -(1-m^2)
\end{pmatrix}$$
$$\implies \mathsf{T} \begin{pmatrix} x \\ y \end{pmatrix} = \frac{1}{1+m^2}\begin{pmatrix}
    1-m^2 & 2m \\
    2m & -(1-m^2)
\end{pmatrix} \begin{pmatrix}
    x \\ y
\end{pmatrix} = \frac{1}{1+m^2}\begin{pmatrix}
    (1-m^2)x + 2my \\
    2mx  -(1-m^2)y
\end{pmatrix}$$

(b) $\mathsf{T}$ is the projection on $L$ along the line perpendicular to $L$. 

Use the same bases and change of coordinate matrices as in part (a). Then we have $$\sbr{T}_{\beta^{\prime}} = \begin{pmatrix}
    1 & 0 \\
    0 & 0
\end{pmatrix}.$$

Then by change of coordinates, we have that $Q^{-1}\sbr{\mathsf{T}}_{\beta}Q = \sbr{\mathsf{T}}_{\beta^{\prime}} \implies \sbr{\mathsf{T}}_{\beta} = Q\sbr{\mathsf{T}}_{\beta^{\prime}}Q^{-1}$. So $$\sbr{\mathsf{T}}_{\beta} = \begin{pmatrix}
    1 & -m \\
    m & 1
\end{pmatrix} \begin{pmatrix}
    1 & 0 \\
    0 & 0
\end{pmatrix} \br{\frac{1}{1+m^2}\begin{pmatrix}
    1 & m \\
    -m & 1
\end{pmatrix}} = \frac{1}{1+m^2}\begin{pmatrix}
    1 & -m \\
    m & 1
\end{pmatrix} \begin{pmatrix}
    1 & m \\
    0 & 0
\end{pmatrix}$$
$$\sbr{\mathsf{T}}_{\beta} = \frac{1}{1+m^2}\begin{pmatrix}
    1 & m \\
    m & m^2
\end{pmatrix}$$
$$\implies \mathsf{T} \begin{pmatrix}
    x \\ y
\end{pmatrix} = \frac{1}{1+m^2}\begin{pmatrix}
    1 & m \\
    m & m^2
\end{pmatrix}\begin{pmatrix}
    x \\ y
\end{pmatrix} = \frac{1}{1+m^2}\begin{pmatrix}
    x+my \\ mx+m^2y 
\end{pmatrix}$$

8. Let $\mathsf{T} : \mathsf{V} \to \mathsf{W}$ be a linear transformation from a finite-dimensional vector space $\mathsf{V}$ to a finite-dimensional vector space $\mathsf{W}$. Let $\beta$ and $\beta^{\prime}$ be ordered bases for $\mathsf{V}$ and let $\gamma$ and $\gamma^{\prime}$ be ordered bases for $\mathsf{W}$. Then $\sbr{\mathsf{T}}_{\beta^{\prime}}^{\gamma^{\prime}} = P^{-1}\sbr{\mathsf{T}}_{\beta}^{\gamma}Q$, where $Q$ is the matrix that changes $\beta^{\prime}$-coordinates into $\beta$-coordinates and $P$ is the matrix that changes $\gamma^{\prime}$-coordinates into $\gamma$-coordinates.

\begin{proof}
    Write $\mathsf{T}$ as $\mathsf{T}\mathsf{I_V}$ and also as $\mathsf{I_W}\mathsf{T}$. Then $$\highlight{P\sbr{\mathsf{T}}_{\beta^{\prime}}^{\gamma^{\prime}}} = \sbr{\mathsf{I_W}}_{\gamma^{\prime}}^{\gamma}\sbr{\mathsf{T}}_{\beta^{\prime}}^{\gamma^{\prime}} = \sbr{\mathsf{I_W T}}_{\beta^{\prime}}^{\gamma} = \sbr{TI_V}_{\beta^{\prime}}^{\gamma} = \sbr{\mathsf{T}}_{\beta}^{\gamma} \sbr{I_V}_{\beta^{\prime}}^{\beta} = \highlight{\sbr{\mathsf{T}}_{\beta}^{\gamma}Q}$$

    Thus $\sbr{\mathsf{T}}_{\beta^{\prime}}^{\gamma^{\prime}} = P^{-1}\sbr{\mathsf{T}}_{\beta}^{\gamma}Q.$
\end{proof}

13.$^{\dagger}$ Let $\mathsf{V}$ be a finite-dimensional vector space over a field $\mathbb{F}$, and let $\beta = \cbr{x_1,x_2,\dots,x_n}$ be an ordered basis for $\mathsf{V}$. Let $Q$ be an $n\times n$ invertible matrix with entries from $\mathbb{F}$. Define $$x_j^{\prime} = \sum_{i=1}^n  Q_{ij}x_i \text{ for } 1\leq j \leq n$$ and set $\beta^{\prime} = \cbr{x_1^{\prime},x_2^{\prime}, \dots, x_n^{\prime}}$. Prove that $\beta^{\prime}$ is a basis for $\mathsf{V}$ and hence that $Q$ is the change of coordinate matrix changing $\beta^{\prime}$-coordinates into $\beta$-coordinates.

\begin{proof}
    Since the dimension of $\mathsf{V}$ is $n$, it suffices to show that $\beta^{\prime}$ contains $n$ linearly independent vectors. 

    The columns of $Q$ are $[x_j^{\prime}]_{\beta}$, and so $Q$ takes vectors represented as $n$-tuples with the $\beta^{\prime}$ basis to vectors represented as $n$-tuples with the $\beta$ basis. Since $Q$ is invertible, it follows that there is an automorphism $\mathsf{L}_Q$ from $\mathbb{F}^n$ to itself. Furthermore, there are isomorphisms $\phi_{\beta}$ and $\phi_{\beta^{\prime}}$ from $\mathsf{V}$ to $\mathbb{F}^n$. So let $\mathsf{T}: \mathsf{V} \to \mathsf{V}$ be an isomorphism given by $\mathsf{T} = \phi_{\beta}^{-1}\mathsf{L}_Q\phi_{\beta^{\prime}}$. 
    
    Since $\mathsf{T}$ and $\mathsf{T}^{-1}$ are isomorphisms, they will carry linearly independent subsets of $\mathsf{V}$ into linearly independent subsets of $\mathsf{V}$. So then $\mathsf{T}^{-1}(\beta) = \beta^{\prime}$ and so $\beta^{\prime}$ is linearly independent and contains $n = \dim(\mathsf{V})$ vectors. Therefore $\beta^{\prime}$ is a basis for $\mathsf{V}$ and so $Q$ is the change of coordinate matrix changing $\beta^{\prime}$-coordinates into $\beta$-coordinates.
\end{proof}

\newpage
3.1:

12. We may only interchange two rows/columns of $A$ (type 1) or we can add any scalar multiple of a row/column of $A$ to another row/column of $A$ (type 3).

(1) So consider starting with the entry in the first row and column, $A_{11}$. If it is zero, interchange this column with another column with a nonzero entry in the first row (type 1). If no such column exists, interchange this first row of zeros with the last row (below this row) which has a nonzero element, and interchange columns if needed to move a nonzero element into the $A_{11}$ position. If there is no such row with a nonzero element below this first row, then terminate.

(2) We now have a nonzero element in the $A_{11}$ position. Using a finite sequence of type 3 row operations we may eliminate all of the $A_{i1}$ entries where $1<i\leq m$. So the first column is filled with zeros outside of the first row's entry. 

So consider the entries $A_{ij}$ where $i,j \geq > 1$ if they exist, i.e. those entries not in the first row/column. These entries form a $(m-1) \times (n-1)$ matrix $A^{\prime}$, and so similarly, we may eliminate the first column of such a matrix in a finite number of type 1 and type 3 operations. So for this ``nested" matrix, eliminate entries under the first entry in the top left (i.e. $A_{22}$, then with the next nested matrix, $A_{33}$, and so on) using steps similar to (1) and (2) where the indices are adjusted to match those of the remaining entries. 

Essentially, the goal is to continue iteratively in this manner until each $A_{kk}$, for $1 \leq k \leq \min(m,n)$ contains only zeroes in each entry below it, forming an upper triangular matrix. This process terminates since $m,n$ are finite.
\end{document}