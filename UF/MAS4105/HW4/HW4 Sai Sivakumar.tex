\documentclass[11pt]{article}

% packages
\usepackage{physics}
% margin spacing
\usepackage[top=1in, bottom=1in, left=0.5in, right=0.5in]{geometry}
\usepackage{hanging}
\usepackage{amsfonts, amsmath, amssymb, amsthm}
\usepackage{systeme}
\usepackage[none]{hyphenat}
\usepackage{fancyhdr}
\usepackage[nottoc, notlot, notlof]{tocbibind}
\usepackage{graphicx}
\graphicspath{{./images/}}
\usepackage{float}
\usepackage{siunitx}
\usepackage{esint}
\usepackage{cancel}

% header/footer formatting
\pagestyle{fancy}
\fancyhead{}
\fancyfoot{}
\fancyhead[L]{MAS4105 Dr. Zhang}
\fancyhead[C]{HW4}
\fancyhead[R]{Sai Sivakumar}
\fancyfoot[R]{\thepage}
\renewcommand{\headrulewidth}{0pt}

% paragraph indentation/spacing
\setlength{\parindent}{0cm}
\setlength{\parskip}{5pt}
\renewcommand{\baselinestretch}{1.25}

% extra commands defined here
\newcommand{\ihat}{\boldsymbol{\hat{\textbf{\i}}}}
\newcommand{\jhat}{\boldsymbol{\hat{\textbf{\j}}}}
\newcommand{\dr}{\vec{r}~^{\prime}(t)}
\newcommand{\dx}{x^{\prime}(t)}
\newcommand{\dy}{y^{\prime}(t)}

\newcommand{\br}[1]{\left(#1\right)}
\newcommand{\sbr}[1]{\left[#1\right]}
\newcommand{\cbr}[1]{\{#1\}}

\newcommand{\dprime}{\prime\prime}
\newcommand{\lap}[2]{\mathcal{L}[#1](#2)}

% bracket notation for inner product
\usepackage{mathtools}

\DeclarePairedDelimiterX{\abr}[1]{\langle}{\rangle}{#1}

\DeclareMathOperator{\Span}{span}

% set page count index to begin from 1
\setcounter{page}{1}

\begin{document}

Page 33: 2 (a,c,e), 3 (a,c,e), 4 (a,e), 6, 7, 10, 11, 13, 15

Page 33: \\

2.

(a). Swap the first and third rows to begin with:
\begin{equation*}
    \sysdelim..\systeme{
    x_1 - x_2 - 2x_3 -  x_4 = -3 ,
    3x_1 - 3x_2 - 2x_3 + 5x_4 = 7,
    2x_1 - 2x_2 - 3x_3 + 0x_4 = -2    
    }
\end{equation*}

Eliminate for $x_1$:
\begin{equation*}
    \sysdelim..\systeme{
    x_1 - x_2 - 2x_3 -  x_4 = -3 ,
    0x_1 + 0x_2 + 4x_3 + 8x_4 = 16,
    0x_1 + 0x_2 + 1x_3 + 2x_4 = 4    
    }
\end{equation*}

Divide the second row through by $4$ and then eliminate for $x_3$:
\begin{equation*}
    \sysdelim..\systeme{
    x_1 - x_2 + 0x_3 +  3x_4 = 5 ,
    0x_1 + 0x_2 + 1x_3 + 2x_4 = 4,
    0x_1 + 0x_2 + 0x_3 + 0x_4 = 0    
    }
\end{equation*}

The third row was really just a linear combination of the other two rows. The variables $x_2$ and $x_4$ are free so we may rewrite the system into the following equations where $x_2,x_4$ may take on any values:
\begin{align*}
    x_1 &= 5+x_2-3x_4 \\
    x_3 &= 4-2x_4
\end{align*}

(c). Eliminate for $x_1$:
\begin{equation*}
    \sysdelim..\systeme{
    x_1 + 2x_2 - x_3 + x_4 = 5,
    0x_1 + 2x_2 - 2x_3 - 4x_4 = 1,
    0x_1 - 1x_2 + 1x_3 + 2x_4 = -2
    }
\end{equation*}

Eliminate for $x_2$, swap the bottom two rows:
\begin{equation*}
    \sysdelim..\systeme{
    x_1 + 0x_2 + 1x_3 + 5x_4 = 4,
    0x_1 - 1x_2 + 1x_3 + 2x_4 = -2,
    0x_1 + 0x_2 + 0x_3 + 0x_4 = -3    
    }
\end{equation*}

Immediately it is evident that this system has no solution because of the inconsistency in the third row.

(e). Eliminate for $x_1$:
\begin{equation*}
    \sysdelim..\systeme{
    x_1 + 2x_2 - 4x_3  -x_4  +x_5 = 7,
    0x_1  +2x_2  +6x_3  -4x_4  -3x_5 = -9,
    0x_1  +x_2  +3x_3  -2x_4  -3x_5 = -12,
    0x_1  +x_2  +3x_3  -2x_4  +0x_5 = 3 
    }
\end{equation*}

Eliminate for $x_2$ and in the same process solve for $x_5$ and swap the second and fourth rows (since the third row becomes mostly empty):
\begin{equation*}
    \sysdelim..\systeme{
    x_1 + 0x_2  -10x_3  +3x_4  +4x_5 = 16,
    0x_1  +x_2  +3x_3  -2x_4  +0x_5 = 3,
    0x_1  +0x_2  +0x_3  +0x_4  +x_5 = 5,
    0x_1  +0x_2  +0x_3  +0x_4  -3x_5 = -15
    }
\end{equation*}

Of course, clean up the last row, and remove four multiples of the fourth row from the first row and we should get:
\begin{equation*}
    \sysdelim..\systeme{
    x_1 + 0x_2  -10x_3  +3x_4  +0x_5 = -4,
    0x_1  +x_2  +3x_3  -2x_4  +0x_5 = 3,
    0x_1  +0x_2  +0x_3  +0x_4  +x_5 = 5,
    0x_1  +0x_2  +0x_3  +0x_4  +0x_5 = 0
    }
\end{equation*}

Thus we have a few free variables, so we may give the solution to the system as:
\begin{align*}
    x_1 &= -4+10x_3-3x_4 \\
    x_2 &= 3-3x_3+2x_4 \\
    x_5 &= 5
\end{align*}

The quantities $x_3,x_4$ may take on any value.

3. If the first vector $A$ can be expressed as a linear combination of the other two vectors $B,C$, then there exist real numbers $x,y$ such that $A = Bx+Cy$.

(a). \textit{Yes.} Express the system like so and solve:
\begin{equation*}
    \sysdelim..\systeme{
        1x + 2y = -2,
        3x + 4y = 0,
        0x - 1y = 3
    }
\end{equation*}

Deduce that $y=-3$, and then reduce the system to the following equations:
\begin{align*}
    x &= 4 \\
    3x &= 12
\end{align*}

This remaining system is obviously true, so the vectors are linearly dependent.

(c). \textit{No.} Express the system like so and solve:
\begin{equation*}
    \sysdelim..\systeme{
        1x -2y = 3,
        -2x -1y = 4,
        1x +1y = 1
    }
\end{equation*}

Then eliminate for $x$:
\begin{equation*}
    \sysdelim..\systeme{
        1x -2y = 3,
        0x -5y = 10,
        0x +3y = -2
    }
\end{equation*}

Simplify the second row and eliminate for $y$ in the third row:
\begin{equation*}
    \sysdelim..\systeme{
        1x -2y = 3,
        0x -1y = 2,
        0x +0y = 4
    }
\end{equation*}

Immediately there is an impossibility that $0=4$, so the system is inconsistent and so the vectors are not linearly dependent.

(e). \textit{No.} Express the system like so and solve:
\begin{equation*}
    \sysdelim..\systeme{
        1x  -2y = 5,
        -2x  +3y = 1,
        -3x  -4y = -5
    }
\end{equation*}

Eliminate for $x$:
\begin{equation*}
    \sysdelim..\systeme{
        1x  -2y = 5,
        0x  -1y = -9,
        0x  -10y = -10
    }
\end{equation*}

Simplify the second row and eliminate for $y$:
\begin{equation*}
    \sysdelim..\systeme{
        1x  -2y = 5,
        0x + 1y = 9,
        0x + 0y = 80
    }
\end{equation*}

Again, this system is obviously inconsistent and so the vectors are not linearly dependent.

4. We wish to find real numbers $a,b$ such that for polynomials $P(x),Q(x),R(x)$, $P(x) = aQ(x)+bR(x)$. For brevity, I will not show the steps where I collect the coefficients of the polynomials and equate them, instead going directly to the linear system of equations that forms from doing so. The first row will have the coefficients of the highest degree term, and the last row will have the constant terms.

(a). \textit{Yes.} The linear system of equations that determines if the first polynomial can be written as a linear combination of the other two is:
\begin{equation*}
    \sysdelim..\systeme{
        a  +b =  1,
        2a  +3b =  0,
        -a  +0b =  -3,
        a  -b =  5
    }
\end{equation*}

Eliminate for $a$:
\begin{equation*}
    \sysdelim..\systeme{
        a  +b =  1,
        0a  +b =  -2,
        0a  +b =  -2,
        0a  -2b =  4
    }
\end{equation*}

Eliminate for $b$:
\begin{equation*}
    \sysdelim..\systeme{
        a  +0b =  3,
        0a  +b =  -2,
        0a  +0b =  0,
        0a  +0b =  0        
    }
\end{equation*}

Evidently the answer is yes, with the choice of $a=3$ and $b=-2$.

(e). \textit{No.} Similarly form the system like so:
\begin{equation*}
    \sysdelim..\systeme{
        1a +1b =  1,
        -2a +0b =  -8,
        3a  -2b =  4,
        -1a +3b =  0
    }
\end{equation*}

Eliminate for $a$:
\begin{equation*}
    \sysdelim..\systeme{
        1a +1b =  1,
        0a +2b =  -6,
        0a  -5b =  1,
        0a +4b =  1
    }
\end{equation*}

Simplify the second row and eliminate for $b$:
\begin{equation*}
    \sysdelim..\systeme{
        1a +0b =  4,
        0a +1b =  -3,
        0a  +0b =  -14,
        0a +0b =  13 
    }
\end{equation*}

Evidently the system is inconsistent and so we cannot form the first polynomial as a linear combination of the other two.

6. Take their linear combination ($r,s,t\in\mathbb{F}$) like so:
\begin{align*}
    r\br{1,1,0} + s\br{1,0,1} + t\br{0,1,1} &= \br{r, r, 0} + \br{s, 0, s} + \br{0, t, t} \\
    &=  \br{r + s, r + t, s + t}  
\end{align*}
We demand that this vector should be equal to any arbitrary vector in $\mathbb{F}^3$. So $\forall a_1,a_2,a_3\in\mathbb{F}$, the following is demanded:
$$\br{r + s, r + t, s + t} = \br{a_1,a_2,a_3}$$

This equality admits the following system:
\begin{equation*}
    \sysdelim..\systeme{
        r +s +0t = a_1,
        r +0s +t = a_2,
        0r +s +t = a_3
    }
\end{equation*}

Perform the elimination:
\begin{equation*}
    \sysdelim..\systeme{
        r +0s +0t = \frac{1}{2}\br{a_1+a_2-a_3},
        0r +s +0t = \frac{1}{2}\br{a_1-a_2+a_3},
        0r +0s +t = \frac{1}{2}\br{-a_1+a_2+a_3}
    }
\end{equation*}

So to form any arbitrary vector $\br{a_1,a_2,a_3}$ we choose as follows:
\begin{align*}
    r &= \frac{1}{2}\br{a_1+a_2-a_3}\\
    s &= \frac{1}{2}\br{a_1-a_2+a_3}\\
    t &= \frac{1}{2}\br{-a_1+a_2+a_3}
\end{align*}

Thus the three vectors generate $\mathbb{F}^3$.

7. The set $\cbr{e_1,e_2,\dots,e_n}$ generates $\mathbb{F}^n$.
\begin{proof} We must show that $\Span\br{\cbr{e_1,e_2,\dots,e_n}} = \mathbb{F}^n$. 

Take any vector in $\mathbb{F}^n$, say $\br{a_1,a_2,\dots,a_n}$. Then we can construct this vector out of a linear combination of $e_1,e_2,\dots,e_n$, like so:

$$\br{a_1,a_2,\dots,a_n} = \br{a_1,0,\dots,0} + \br{0,a_2,\dots,0} + \cdots + \br{0,0,\dots,a_n}$$
$$ = \sum_{k=1}^n a_ke_k$$

So from this we can deduce that $\mathbb{F}^n\subseteq \Span\br{\cbr{e_1,e_2,\dots,e_n}}$

Then take vectors produced by taking the span of $\cbr{e_1,e_2,\dots,e_n}$. For $a_1,a_2,\dots,a_n\in \mathbb{F}$, the span of the set of vectors is $$\cbr{\sum_{k=1}^n a_ke_k} = \cbr{\br{a_1,0,\dots,0} + \br{0,a_2,\dots,0} + \cdots + \br{0,0,\dots,a_n}}$$
$$ = \cbr{\br{a_1,a_2,\dots,a_n}}$$

These vectors will always be in $\mathbb{F}^n$, so $\Span\br{\cbr{e_1,e_2,\dots,e_n}} \subseteq \mathbb{F}^n$

Hence $\Span\br{\cbr{e_1,e_2,\dots,e_n}} = \mathbb{F}^n$ and so we have that $\cbr{e_1,e_2,\dots,e_n}$ generates $\mathbb{F}$.
\end{proof}

10. The set $\cbr{M_1,M_2,M_3}$ spans all symmetric $2\times 2$ matrices.
\begin{proof}
    We can decompose any symmetric $2\times 2$ matrix into a linear combination of $M_1,M_2,M_3$. A symmetric matrix takes on the form $$\begin{pmatrix}
        a_1 & a_3 \\
        a_3 & a_2
    \end{pmatrix}$$ for any choice of $a_1,a_2,a_3\in\mathbb{F}$ since the transpose of such a matrix is equivalent to itself. Then we can decompose the matrix like so:
    $$\begin{pmatrix}
        a_1 & a_2 \\
        a_2 & a_3
    \end{pmatrix} = \begin{pmatrix}
        a_1 & 0 \\
        0 & 0
    \end{pmatrix} + \begin{pmatrix}
        0 & 0 \\
        0 & a_2
    \end{pmatrix} + \begin{pmatrix}
        0 & a_3 \\
        a_3 & 0
    \end{pmatrix}$$
    $$ = a_1M_1 + a_2M_2 + a_3M_3$$

    This matrix is in the span of $\cbr{M_1,M_2,M_3}$, so the set of all symmetric $2\times 2$ matrices are a subset of $\Span\br{\cbr{M_1,M_2,M_3}}$

    Similarly the span of $\cbr{M_1,M_2,M_3}$ is contained in the set of all symmetric matrices. For all $a_1,a_2,a_3\in\mathbb{F}$, the span of $\cbr{M_1,M_2,M_3}$ is:
    $$\cbr{a_1M_1 + a_2M_2 + a_3M_3}$$
    $$ = \cbr{\begin{pmatrix}
        a_1 & 0 \\
        0 & 0
    \end{pmatrix} + \begin{pmatrix}
        0 & 0 \\
        0 & a_2
    \end{pmatrix} + \begin{pmatrix}
        0 & a_3 \\
        a_3 & 0
    \end{pmatrix}} = \cbr{\begin{pmatrix}
        a_1 & a_2 \\
        a_2 & a_3
    \end{pmatrix}}$$

    It is evident that the set of these matrices is symmetric, so then indeed the set of all symmetric $2\times 2$ matrices are a subset of $\Span\br{\cbr{M_1,M_2,M_3}}$.

    Hence the set of all symmetric $2\times 2$ matrices is equal to $\Span\br{\cbr{M_1,M_2,M_3}}$.

\end{proof}

11. The span of $\cbr{\vec{x}}$ is equivalent to $\cbr{a\vec{x} : a\in\mathbb{F}}$.
\begin{proof}
    By definition. The span of a set is defined to be the set of all linear combinations of vectors in the set.
    $$\Span\br{\cbr{\vec{x}}} := \cbr{a\vec{x} : a\in\mathbb{F}}$$
\end{proof}

Geometrically for a vector in $\mathbb{R}^3$ this is like enumerating a point set that forms a line extending infinitely in both directions in space. The slope of the line is parallel to the vector whose linear combinations generated the line, and the line passes through the origin.

13. Show that if $S_1$ and $S_2$ are subsets of a vector space $\mathsf{V}$ such that $S_1\subseteq S_2$, then $\Span(S_1)\subseteq \Span(S_2)$.

\begin{proof}
    Let $S_1 = \cbr{v_1,v_2,\dots,v_m}$ and $S_2 = \cbr{v_1,v_2,\dots,v_m,v_{m+1},\dots,v_n}$ where $m \leq n$. Then observe that every linear combination of vectors in $S_1$ also appears in $S_2$. For $a_1,a_2,\dots,a_n\in\mathbb{F}$ we have the following:
    \begin{align*}
        \Span\br{S_1} &= \cbr{a_1v_1 + a_2v_2 + \cdots + a_mv_m} \\
        \Span\br{S_2} &= \cbr{a_1v_1 + a_2v_2 + \cdots + a_nv_n}
    \end{align*}

    Notice that for any choice of $a_1,a_2,\dots,a_m$, we can choose $a_m+1,\dots,a_n$ to all be zero so that the following equality is achieved:
    $$\sum_{k=1}^m a_kv_k = \sum_{k=1}^m a_kv_k + \sum_{k=m+1}^n (0)v_k = \sum_{k=1}^m a_kv_k + \sum_{k=m+1}^n a_kv_k$$

    The point is that all of the vectors in the span of $S_1$ are just special choices of vectors in the span of $S_2$, and so we have $\Span(S_1)\subseteq \Span(S_2)$.
\end{proof}

In particular, if $S_1\subseteq S_2$ and $\Span(S_1) = \mathsf{V}$, deduce that $\Span(S_2) = \mathsf{V}$.

\begin{proof}
    All we need to do is show that $\Span(S_2) \subseteq \mathsf{V}$, since then $\Span(S_2) \subseteq \Span(S_1)$ and so $\Span(S_2) = \Span(S_1) = \mathsf{V}$. First observe that $S_2$ is a subset of $\mathsf{V}$, so then we can deduce that vectors $v_{m+1},\dots,v_n\in\mathsf{V}$. Then since we know that $S_1$ generates $\mathsf{V}$, $v_{m+1},\dots,v_n\in \Span(S_1)$.
    
    This means that each of $v_{m+1},\dots,v_n$ can be expressed as some linear combination of $v_1,v_2,\dots,v_m$. So we may rewrite the span of $S_2$ as
    $$\Span\br{S_2} = \cbr{\br{\sum_{k=1}^m a_kv_k} +\br{\sum_{k=1}^m b(m+1)_{k}v_k} + \cdots + \br{\sum_{k=1}^m b(n)_{k}v_k}}$$

    for all $a_1,a_2,\dots,a_m \in \mathbb{F}$ and for special choices of $b(m+1)_i,\dots,b(n)_i\in \mathbb{F}$ such that 
    \begin{align*}
        v_{m+1} &= \sum_{k=1}^m b(m+1)_{k}v_k \\
        &~\vdots \\
        v_{n} &= \sum_{k=1}^m b(n)_{k}v_k
    \end{align*}

    Then $$\Span\br{S_2} = \cbr{\br{\sum_{k=1}^m \br{a_k+b(m+1)_k+\cdots+b(n)_k}v_k}} \subseteq \Span\br{S_1}$$

    Hence $\Span\br{S_2} \subseteq \mathsf{V}$ and so $\Span\br{S_2} = \mathsf{V}$.
\end{proof}

15. Let $S_1$ and $S_2$ be subsets of a vector space $\mathsf{V}$. Prove that $\Span\br{S_1\cap S_2}\subseteq \Span(S_1)\cap\Span(S_2)$.
\begin{proof}
    Consider any vector $w\in \Span\br{S_1\cap S_2}$. Then $w = a_1v_1 + a_2+v_2 + \cdots + a_nv_n$, where $a_1,a_2,\dots,a_n\in\mathbb{F}$ and $v_1,v_2,\dots,v_n\in S_1\cap S_2$. Then by definition $v_1,v_2,\dots,v_n\in S_1$ and so $w\in \Span\br{S_1}$. Similarly $v_1,v_2,\dots,v_n\in S_2$ is also true and so $w\in \Span\br{S_2}$.

    Therefore for any $w$, we have $w\in \Span(S_1)\cap\Span(S_2)$, which means that $\Span\br{S_1\cap S_2}\subseteq \Span(S_1)\cap\Span(S_2)$.
\end{proof}

Give an example in which $\Span\br{S_1\cap S_2}$ and $\Span(S_1)\cap\Span(S_2)$ are equal and one in which they are unequal.

We have equality is whenever $S_1$ and $S_2$ are equal, which means their spans are also equal. This also happens when the sets $S_1$ and $S_2$ are disjoint. A concrete example of this might be literally the case where $S_1 = \cbr{v} = S_2 = S_1\cap S_2$, so that $\Span(S_1) = \cbr{av : a\in\mathbb{F}} = \Span(S_2) = \Span(S_1)\cap \Span(S_2)$.

We have inequality when $S_2$ is a proper subset of the span of $S_1$. Suppose we have $S_1 = \cbr{v_1,v_2}$ and $S_2 = {w}$, where $w = v_1+v_2$. So $S_2 \subsetneq \Span(S_1)$. Then $S_1\cap S_2 = \varnothing$, so the span of that is the zero vector. But then we can see that the span of $S_2$ is contained within the span of $S_1$, so then the intersection is not only the zero vector (there are special choices of vectors in each span that \textit{are} common to each span). For instance, since $\Span(S_1) = \cbr{a_1v_1 + a_2v_2 : a_1,a_2\in\mathbb{F}}$ and $\Span(S_2) = \cbr{aw : a\in\mathbb{F}} = \cbr{av_1+av_2 : a\in\mathbb{F}}$ we can simply choose values where $a_1=a_2=a \neq 0$ and find nontrivial vectors in the intersection of the spans.

\end{document}