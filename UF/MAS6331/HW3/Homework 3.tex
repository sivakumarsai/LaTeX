\documentclass[11pt]{article}

% packages
\usepackage{physics}
% margin spacing
\usepackage[top=1in, bottom=1in, left=0.5in, right=0.5in]{geometry}
\usepackage{hanging}
\usepackage{amsfonts, amsmath, amssymb, amsthm}
\usepackage{systeme}
\usepackage[none]{hyphenat}
\usepackage{fancyhdr}
\usepackage[nottoc, notlot, notlof]{tocbibind}
\usepackage{graphicx}
\graphicspath{{./images/}}
\usepackage{float}
\usepackage{siunitx}
\usepackage{esint}
\usepackage{cancel}
\usepackage{enumitem}

% permutations (second line is for spacing)
\usepackage{permute}
\renewcommand*\pmtseparator{\,}

% colors
\usepackage{xcolor}
\definecolor{p}{HTML}{FFDDDD}
\definecolor{g}{HTML}{D9FFDF}
\definecolor{y}{HTML}{FFFFCF}
\definecolor{b}{HTML}{D9FFFF}
\definecolor{o}{HTML}{FADECB}
%\definecolor{}{HTML}{}

% \highlight[<color>]{<stuff>}
\newcommand{\highlight}[2][p]{\mathchoice%
  {\colorbox{#1}{$\displaystyle#2$}}%
  {\colorbox{#1}{$\textstyle#2$}}%
  {\colorbox{#1}{$\scriptstyle#2$}}%
  {\colorbox{#1}{$\scriptscriptstyle#2$}}}%

% header/footer formatting
\pagestyle{fancy}
\fancyhead{}
\fancyfoot{}
\fancyhead[L]{MAS6331 Algebra}
\fancyhead[C]{Homework 3}
\fancyhead[R]{Sai Sivakumar}
\fancyfoot[R]{\thepage}
\renewcommand{\headrulewidth}{1pt}

% paragraph indentation/spacing
\setlength{\parindent}{0cm}
\setlength{\parskip}{10pt}
\renewcommand{\baselinestretch}{1.25}

% extra commands defined here
\newcommand{\br}[1]{\left(#1\right)}
\newcommand{\sbr}[1]{\left[#1\right]}
\newcommand{\cbr}[1]{\left\{#1\right\}}

\newcommand{\dprime}{\prime\prime}

% bracket notation for inner product
\usepackage{mathtools}

\DeclarePairedDelimiterX{\abr}[1]{\langle}{\rangle}{#1}

\DeclareMathOperator{\Span}{span}
\DeclareMathOperator{\nullity}{nullity}
\DeclareMathOperator\Aut{Aut}
\DeclareMathOperator\Inn{Inn}
\DeclareMathOperator{\Orb}{Orb}
\DeclareMathOperator{\lcm}{lcm}
\DeclareMathOperator{\Hol}{Hol}
\DeclareMathOperator{\Jac}{Jac}
\DeclareMathOperator{\rad}{rad}
\DeclareMathOperator{\Tor}{Tor}
\DeclareMathOperator{\End}{End}
\DeclareMathOperator{\Gal}{Gal}

% set page count index to begin from 1
\setcounter{page}{1}

\begin{document}
\subsection*{Graded}
\begin{enumerate}
    \item (13.6.4) Prove that if $n = p^km$ where $p$ is a prime and $m$ is relatively prime to $p$ then there are precisely $m$ distinct roots of unity over a field of characteristic $p$. \begin{proof}
        Let $F$ be a field of characteristic $p$ and let $n = p^km$ with $m$ relatively prime to $p$ as above. Then the roots of unity are the roots of the polynomial $f(x) = x^n - 1 = x^{p^km}-1$. Since $D_x(f(x)) = p^kmx^{p^km-1} = 0$, $f(x)$ is inseparable and so we show that there are $m$ distinct roots which are repeated.
        
        Since $F$ is characteristic $p$, write $f(x)$ as $(x^m)^{p^k} - 1^{p^k} = (x^m-1)^{p^k}$, so that the roots which are repeated are indeed the roots of $x^m-1$. But $x^m-1$ is separable since its derivative is $mx^{m-1}$ ($\neq 0$ since $p\nmid m$), which shares no roots with $x^m-1$ (if there were such a root, its minimal polynomial would divide both $x^m-1$ and $mx^{m-1}$ so that $\gcd(x^m-1,mx^{m-1})\neq 0$). Hence $x^m-1$ carries with it exactly $m$ distinct roots; as a result $f(x)$ does also.
    \end{proof}
    \item (An Infinite Extension) Let $K$ be a field of characteristic zero. Recall that $K(x)$ is the field of rational functions over $K$ (the field of fractions of $K[x]$). \begin{enumerate}
        \item Show that $\sigma_n\colon K(x)\to K(x)$ defined by sending $x$ to $x+n$ is an automorphism of $K(x)$. \begin{proof}
            Let $f(x),g(x)\in K(x)$%, with $f(x) = p(x)/q(x)$ and $g(x) = r(x)/s(x)$, for $p(x),q(x)$,$r(x),s(x)\in K[x]$
            . Then \begin{align*}
                \sigma_n(f(x)\pm g(x)) &= f(x+n) \pm g(x+n)\\
                &= \sigma_n(f(x)) \pm \sigma_n(g(x))\\
                \sigma(f(x)g(x)) &= f(x+n)g(x+n)\\
                &= \sigma_n(f(x))\sigma_n(g(x)),
            \end{align*} and for $g(x)$ not zero, \begin{align*}
                \sigma_n(f(x)/g(x)) &= f(x+n)/g(x+n)\\
                &= \sigma_n(f(x))/\sigma_n(g(x)).
            \end{align*}
            Furthermore, $\sigma_{-n}\colon K(x)\to K(x)$ sending $x$ to $x-n$ is a left and right inverse to $\sigma_n$. It follows $\sigma_n$ is an automorphism of $K(x)$ Since $K$ is characteristic zero, each $\sigma_n$ is not the identity.
        \end{proof}
        \item Let $G = \cbr{\sigma_n\mid n\in \mathbb{Z}}\subset \Aut(K(x))$. What is the fixed field of $G$ (i.e. what is $K(x)^G$)?
        
        The fixed field $K(x)^G$ is $K$. \begin{proof}
            Let $f(x) = p(x)/g(x)$ be an element of $K(x)$ with $p(x),g(x)\in K[x]$ irreducible and coprime with $g(x)$ nonzero. Then for any $\sigma_n$, we have $\sigma_n(f(x)) = p(x+n)/g(x+n)$; for $f(x)$ to be in $K(x)^G$, we must have $p(x)/g(x) = p(x+n)/g(x+n)$ for all $n\in\mathbb{Z}$.

            Since $p(x)$ and $g(x)$ are coprime (so that $p(x+n)$ and $g(x+n)$ are also), the rational functions $p(x)/g(x)$ and $p(x+n)/g(x+n)$ are equivalent if the roots of $p(x)$ and $g(x)$ coincide with the roots of $p(x+n)$ and $g(x+n)$, respectively. (Note these polynomials are separable since $K$ is characteristic $0$, so we do not need to worry about comparing multiplicities of roots.)

            Let $\alpha$ be a root of $p(x)$ and $\beta$ a root of $g(x)$. For each $n\in\mathbb{Z}$, if $\sigma_n$ fixes $f(x) = p(x)/g(x)$, then $p(x+n)$ shares the same roots with $p(x)$ and $g(x+n)$ shares the same roots with $g(x)$. Thus for each $n\in\mathbb{Z}$ $\alpha-n$ is a root of $p(x)$ and $\beta-n$ is a root of $g(x)$. With $\alpha\neq \alpha-n$ and $\beta\neq \beta-n$ for each $n$, we have that $p(x)$ and $g(x)$ have infinitely many roots, which is impossible unless $p(x)$ and $g(x)$ had no roots to begin with. Hence $p(x),g(x)$ are degree $0$ and so any element in $K(x)^G$ is of the form $k_1/k_2$ for $k_1,k_2\in K$. It follows that $K(x)^G = K$.
        \end{proof}
        \item What is $[K(x)\colon K(x)^G]$?
        
        The degree of $K(x)$ over $K(x)^G = K$ is infinite. \begin{proof}
            Since $K(x)$ is a field extension of $K$, we have that $K(x)$ is a $K$-vector space. To show that the basis is not finite, we exhibit an infinite set of linearly independent elements of $K(x)$ over $K$: consider $\cbr{x^n\mid n\in\mathbb{Z}}$; this is one such infinite linearly independent set. Hence the basis contains at least infinitely many elements, meaning the degree of the extension $K(x)$ over $K(x)^G = K$ is not finite.
        \end{proof}
    \end{enumerate}
\end{enumerate}
\subsection*{Additional Problems}
\begin{enumerate}
    \item (13.5.2) Find all irreducible polynomials of degrees $1$, $2$ and $4$ over $\mathbb{F}_2$ and prove that their product is $x^{16}-x$. \begin{proof}
        It is clear that the degree $1$ irreducible polynomials in $\mathbb{F}_2[x]$ are $x$, $x+1$.

        The degree $2$ polynomials are of the form $x^2+a_1x + a_0$; to be irreducible $a_0$ must be $1$ and from there it follows that $a_1$ must be $1$ so that $x^2+x+1$ is the only degree $2$ polynomial with no roots in $\mathbb{F}_2$.

        Degree $4$ polynomials are of the form $x^4+a_3x^3+a_2x^2+a_1x + a_0$; to be irreducible, we first demand $a_0$ be $1$. Then we would not want $1$ to be a root of such a polynomial, so we need there to be an odd number of terms (so two of $a_3,a_2,a_1$ must be zero or they are all $1$). Further note that $x^4+x^2+1 = x^4+2x^3+3x^2+2x+1 = (x^2+x+1)^2$ is not irreducible.

        We check that the only remaining polynomials $x^4+x^3+x^2+x+1$, $x^4+x^3+1$, and $x^4+x+1$ are irreducible by seeing that $x^2+x+1$ does not divide any of these three (by long division in $\mathbb{F}_2[x]$). Hence the above are the only irreducible degree $4$ polynomials.

        Computing, we have \begin{align*}
            &(x)[(x+1)(x^4+x^3+x^2+x+1)](x^2+x+1)[(x^4+x+1)(x^4+x^3+1)] \\
            =&(x)(x^5+1)[(x^2+x+1)(x^8+x^7+x^5+x^4+x^3+x+1)]\\
            =&(x)[(x^5+1)(x^{10}+x^5+1)]\\
            =&(x)(x^{15}+1)\\
            =&~x^{16}+x
        \end{align*} as desired.
    \end{proof}
    \item (13.5.9) Show that the binomial coefficient $\binom{pn}{pi}$ is the coefficient of $x^{pi}$ in the expansion of $(1+x)^{pn}$.
    
    Working over $\mathbb{F}_p$ show that this is the coefficient of $(x^p)^i$ in $(1+x^p)^n$ and hence prove that $\binom{pn}{pi}\equiv \binom{n}{i}\pmod p$. \begin{proof}
        By the binomial theorem, $(1+x)^{pn} = \sum_{k=0}^{pn}\binom{pn}{k}x^k$ so that the coefficient of $x^{pi}$ is $\binom{pn}{pi}$ as expected. 

        Working in $\mathbb{F}_p$, we write $(1+x)^{pn}$ as $[(1+x)^{p}]^n = (1^p + x^p)^n = (1+x^p)^n = \sum_{k=0}^n\binom{n}{k}(x^p)^k$ so that the coefficient for $x^{pi} = (x^p)^i$ is $\binom{n}{i}$. Since both binomial expansions are valid for $(1+x)^{pn}$ in $\mathbb{F}_p$ it follows that $\binom{pn}{pi}\equiv \binom{n}{i}\pmod p$.
    \end{proof}
    \item (13.6.6) Prove that for $n$ odd, $n>1$, $\varPhi_{2n}(x) = \varPhi_n(-x)$. \begin{proof}
        Observe that $\varPhi_{2n}(x)$ is the minimal polynomial for any $2n$-th root of unity. Note also that $\varPhi_n(x)$ is a minimal polynomial, so that $\varPhi_n(-x)$ is irreducible also; it is also monic since its degree is $\varphi(n)$, which for odd $n>1$ will be even since $\varphi$ is even for prime powers. Hence both polynomials $\varPhi_{2n}(x), \varPhi_n(-x)$ are minimal polynomials over any of its roots. If they share a common root, they must be equal by uniqueness of the minimal polynomial.

        Claim: $\zeta_2\zeta_n = -\zeta_n$ is a common root: Clearly it is a root of $\varPhi_n(-x)$, but to see that it is a root of $\varPhi_{2n}(x)$ we show that $-\zeta_n$ has order $2n$ in the group of $2n$-th roots of unity. Observe that $(-\zeta_n)^{2n} = \zeta_n^{2n} = 1$ so that the order of $-\zeta_n$ divides $2n$. But the order of $-1$ is $2$ and the order of $\zeta_n$ is $n$, and since $\gcd(2,n)=1$ it follows that the order of their product cannot be any lower than $2n$; that is, it must be exactly $2n$ as desired. Hence $-\zeta_n$ is a common root of the above two minimal polynomials so they must be equivalent.
    \end{proof}
    \item (13.6.8) Let $\ell$ be a prime and let $\varPhi_\ell(x) = \frac{x^\ell-1}{x-1} = x^{\ell-1} + x^{\ell-2} + \cdots + x + 1\in \mathbb{Z}[x]$ be the $\ell^\text{th}$ cyclotomic polynomial, which is irreducible by Theorem 41. This exercise determines the factorization of $\varPhi_\ell(x)$ modulo $p$ for any prime $p$. Let $\zeta$ denote any fixed primitive $\ell^\text{th}$ root of unity. \begin{enumerate}
        \item Show that if $p = \ell$ then $\varPhi_\ell(x) = (x-1)^{\ell-1}\in \mathbb{F}_\ell[x]$. \begin{proof}
            In $\mathbb{F}_\ell[x]$, write $x^\ell-1$ as $(x-1)^\ell$ so that $\varPhi_\ell(x) = \frac{x^\ell-1}{x-1} = \frac{(x-1)^\ell}{x-1} = (x-1)^{\ell-1}$ as desired.
        \end{proof}
        \item Suppose $p\neq \ell$ and let $f$ denote the order of $p$ mod $\ell$, i.e., $f$ is the smallest power of $p$ with $p^f \equiv 1 $ mod $\ell$. Use the fact that $\mathbb{F}_{p^n}^\times$ is a cyclic group to show that $n = f$ is the smallest power $p^n$ of $p$ with $\zeta \in \mathbb{F}_{p^n}$. Conclude that the minimal polynomial of $\zeta$ over $\mathbb{F}_p$ has degree $f$. \begin{proof}
            We show that $\zeta$ satisfies $x^{p^f-1}-1$ in $\mathbb{F}_{p^f}^\times$: since $p^f\equiv 1\pmod \ell$, it follows that there is an integer $m\geq 1$ such that $p^f-1 = m\ell$. Then $\zeta^{p^f-1}-1 = \zeta^{m\ell}-1 = 1^m-1 = 0$ as desired. No smaller power $p^n$ will admit this inclusion: if $n<f$, then $p^n\not\equiv 1\pmod \ell$ so that $p^n-1 = m\ell + r$ for $0<r < \ell$. Then $\zeta^{p^n-1}-1 = \zeta^{m\ell+r}-1 = \zeta^r-1\neq 0$.

            Since $\mathbb{F}_{p^f}$ is a degree $f$ extension of $\mathbb{F}_p$ and $\zeta\in \mathbb{F}_{p^f}$, it follows that the minimal polynomial of $\zeta$ over $\mathbb{F}_p$ is degree $f$.
        \end{proof}
        \item Show that $\mathbb{F}_p(\zeta) = \mathbb{F}_p(\zeta^a)$ for any integer $a$ not divisible by $\ell$. [One inclusion is obvious. For the other, note that $\zeta = (\zeta^a)^b$ where $b$ is the multiplicative inverse of $a$ mod $\ell$.] Conclude using (b) that, in $\mathbb{F}_p[x]$, $\varPhi_\ell(x)$ is the product of $\frac{\ell-1}{f}$ distinct irreducible polynomials of degree $f$. \begin{proof}
            The inclusion $\mathbb{F}_p(\zeta^a)\subseteq \mathbb{F}_p(\zeta)$ is clear since $\zeta^a$ is a power of $\zeta$. The reverse inclusion is shown similarly, that $\zeta$ is a power of $\zeta^a$. The power desired is the multiplicative inverse of $a$ viewed as an element of $\mathbb{F}_\ell^\times$ (recall $\ell$ is prime and $\gcd(a,\ell) = 1$). Call this multiplicative inverse $b$ so that $(\zeta^a)^b = \zeta^{ab} = \zeta^{1+m\ell} = \zeta$ for some integer $m$; it follows that $\zeta$ is a power of $\zeta^a$ so that the reverse inclusion holds.

            Since the minimal polynomial of $\zeta$ over $\mathbb{F}_{p}$ has degree $f$ it follows that $\mathbb{F}_{p^f} = \mathbb{F}_p(\zeta) = \mathbb{F}_p(\zeta^a)$ for $1\leq a\leq \ell-1$. But each of the $\ell-1$ roots $\zeta^a$ are roots of $\varPhi_\ell(x)$ viewed as an element of $\mathbb{F}_p$, so each of their minimal polynomials (and hence their product) divides $\varPhi_\ell(x)$. In fact since $\varPhi_\ell(x)$ is separable (it has no repeated roots so we consider degrees) and has degree $\ell-1$, it follows that it has exactly $\frac{\ell-1}{f}$ distinct irreducible factors over $\mathbb{F}_p$ of degree $f$.
        \end{proof}
        \item In particular, prove that, viewed in $\mathbb{F}_p[x]$, $\varPhi_7(x) = x^6+x^5+\cdots + x+1$ is $(x+1)^6$ for $p = 7$, a product of distinct linear factors for $p\equiv 1$ mod $7$, a product of $3$ irreducible quadratics for $p \equiv 6$ mod $7$, a product of $2$ irreducible cubics for $p\equiv 2,4$ mod $7$, and is irreducible for $p\equiv 3,5$ mod $7$. \begin{proof}
            Use part (a) to see that $\varPhi_7(x) = \frac{x^7-1}{x-1} = (x+1)^6\in\mathbb{F}_7[x]$. When $p\equiv 1 \pmod 7$, $p$ has order $1$ so that by the formula in (c), $\varPhi_7(x)$ is the product of $(7-1)/1 = 6$ degree $1$ factors. Similarly, the order of $6$ is $2$, the order of $2$ and $4$ is $3$, and the order of $3$ and $5$ is $6$.

            Therefore:
            
            When $p \equiv 6 \pmod 7$, $\varPhi_7(x)$ is the product of $(7-1)/2 = 3$ degree $2$ factors.
            
            When $p \equiv 6 \pmod 7$, $\varPhi_7(x)$ is the product of $(7-1)/2 = 3$ degree $2$ factors.
            
            When $p \equiv 2,4 \pmod 7$, $\varPhi_7(x)$ is the product of $(7-1)/3 = 2$ degree $3$ factors.
            
            When $p \equiv 3,5 \pmod 7$, $\varPhi_7(x)$ is the product of $(7-1)/6 = 1$ degree $6$ factor (i.e. it is irreducible).
        \end{proof}
    \end{enumerate}
    \item (14.1.1) \begin{enumerate}
        \item Show that if the field $K$ is generated over $F$ by the elements $\alpha_1,\dots, \alpha_n$ then an automorphism $\sigma$ of $K$ fixing $F$ is uniquely determined by $\sigma(\alpha_1),\dots, \sigma(\alpha_n)$. In particular show that an automorphism fixes $K$ if and only if it fixes a set of generators for $K$. \begin{proof}
            Let $\sigma$ be an automorphism of $K = F(\alpha_1,\dots,\alpha_n)$ which fixes $F$. Any element of $K$ is of the form $k = f(\alpha_1,\dots,\alpha_n)/g(\alpha_1,\dots,\alpha_n)$ for polynomials $f,g$. Since $\sigma$ is an automorphism that fixes $F$ it follows that $\sigma(k) = f(\sigma(\alpha_1),\dots,\sigma(\alpha_n))/g(\sigma(\alpha_1),\dots,\sigma(\alpha_n))$. Hence the action of $\sigma$ is determined by its action on each $\alpha_i$; that is, $\sigma_1$ agrees with $\sigma_2$ if and only if for every $k$ in the above form, $\sigma_1(k) = \sigma_2(k)$ -- if and only if each automorphism agrees on each $\alpha_i$.

            In particular it follows that $\sigma$ above is the identity map on $K$ if $\sigma(\alpha_i) = \alpha_i$ for each $i$.
        \end{proof}
        \item Let $G\leq \Gal(K/F)$ be a subgroup of the Galois group of the extension $K/F$ and suppose $\sigma_1,\dots, \sigma_k$ are generators for $G$. Show that the subfield $E/F$ is fixed by $G$ if and only if it is fixed by the generators $\sigma_1,\dots,\sigma_k$. \begin{proof}
            Since $K$ is Galois over $F$, it is a finite extension; hence $E$ is also. Let $E = F(\alpha_1,\dots,\alpha_n)$.

            Suppose $E/F$ is fixed by each of the generators $\sigma_1,\dots,\sigma_k$ for $G$. It follows that every automorphism $\sigma\in G$ fixes $E/F$ since $\sigma$ may be represented as a finite product (compositions) of the $\sigma_i$, and each of those fix $E/F$.

            Conversely, suppose $G$ fixes $E/F$; it follows immediately that $\sigma_i\in G$ fix $E/F$.
        \end{proof}
    \end{enumerate}
    \item (14.1.7) This exercise determines $\Aut(\mathbb{R}/\mathbb{Q})$. \begin{enumerate}
        \item Prove that any $\sigma\in \Aut(\mathbb{R}/\mathbb{Q})$ takes squares to squares and takes positive reals to positive reals. Conclude that $a<b$ implies $\sigma a < \sigma b$ for every $a,b\in\mathbb{R}$. \begin{proof}
            Since $\sigma$ is an automorphism, if $s = r^2$ for $r\in\mathbb{R}$ is a square of a real number then $\sigma(s) = \sigma(r^2) = \sigma(r)^2$ and $\sigma(r)\in \mathbb{R}$ as desired. In particular every positive real number is the square of some real number, so every positive real number is sent to a positive real number.

            Then if $a,b\in \mathbb{R}$ with $0<b-a$ then $0<\sigma(b-a) = \sigma(b)-\sigma(a)$, so $\sigma$ is order preserving.
        \end{proof}
        \item Prove that $-\frac{1}{m}< a-b <\frac{1}{m}$ implies $-\frac{1}{m}< \sigma a- \sigma b <\frac{1}{m}$ for every positive integer $m$. Conclude that $\sigma$ is a continuous map on $\mathbb{R}$. \begin{proof}
            For any positive integer $m$ if $-\frac{1}{m}< a-b < \frac{1}{m}$, then since $\sigma$ fixes $\mathbb{Q}$ and is order preserving, it follows that $\sigma\left(-\frac{1}{m}\right) = -\frac{1}{m} < \sigma(a-b) = \sigma(a) - \sigma(b) < \frac{1}{m} = \sigma\left(\frac{1}{m}\right)$.

            We show that $\sigma$ is continuous at every real number $r$. Let $\varepsilon>0$ be given, and choose $m$ large enough so that $\frac{1}{m}<\varepsilon$. If $\abs{x-r}<\frac{1}{m}$, then by the above argument $\abs{\sigma(x) - \sigma(r)}< \frac{1}{m} < \varepsilon$ so that $\sigma$ is continuous at $r$. Since $r$ was arbitrary, $\sigma$ is continuous on $\mathbb{R}$.
        \end{proof}
        \item Prove that any continuous map on $\mathbb{R}$ which is the identity on $\mathbb{Q}$ is the identity map, hence $\Aut(\mathbb{R}/\mathbb{Q}) = 1$. \begin{proof}
            Let $r$ be any real number. Then since $\mathbb{R}$ is a complete space, there is a (Cauchy) sequence $(a_n)$ converging to $r$ from $\mathbb{Q}$. Since any $\sigma\in \Aut(\mathbb{R}/\mathbb{Q})$ is continuous, it preserves convergence of the sequence $(a_n)$ in that $(\sigma(a_n))$ converges to $\sigma(r)$. But $\sigma$ fixes $\mathbb{Q}$ so that $(\sigma(a_n)) = (a_n)$; hence $\sigma(r) = r$. It follows that every $\sigma$ agrees with the identity map, so $\Aut(\mathbb{R}/\mathbb{Q}) = 1$.
        \end{proof}
    \end{enumerate}
\end{enumerate}
\subsection*{Feedback}
\begin{enumerate}
    \item 14.1.7, thanks.
    \item Things are okay as usual; just eager to learn more Galois theory because it has been a while since I thought about groups like symmetric groups or automorphism groups. It's nice to see things return like that.
\end{enumerate}
\end{document}