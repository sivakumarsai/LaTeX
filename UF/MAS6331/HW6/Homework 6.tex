\documentclass[11pt]{article}

% packages
\usepackage{physics}
% margin spacing
\usepackage[top=1in, bottom=1in, left=0.5in, right=0.5in]{geometry}
\usepackage{hanging}
\usepackage{amsfonts, amsmath, amssymb, amsthm}
\usepackage{systeme}
\usepackage[none]{hyphenat}
\usepackage{fancyhdr}
\usepackage[nottoc, notlot, notlof]{tocbibind}
\usepackage{graphicx}
\graphicspath{{./images/}}
\usepackage{float}
\usepackage{siunitx}
\usepackage{esint}
\usepackage{cancel}
\usepackage{enumitem}

% permutations (second line is for spacing)
\usepackage{permute}
\renewcommand*\pmtseparator{\,}

% colors
\usepackage{xcolor}
\definecolor{p}{HTML}{FFDDDD}
\definecolor{g}{HTML}{D9FFDF}
\definecolor{y}{HTML}{FFFFCF}
\definecolor{b}{HTML}{D9FFFF}
\definecolor{o}{HTML}{FADECB}
%\definecolor{}{HTML}{}

% \highlight[<color>]{<stuff>}
\newcommand{\highlight}[2][p]{\mathchoice%
  {\colorbox{#1}{$\displaystyle#2$}}%
  {\colorbox{#1}{$\textstyle#2$}}%
  {\colorbox{#1}{$\scriptstyle#2$}}%
  {\colorbox{#1}{$\scriptscriptstyle#2$}}}%

% header/footer formatting
\pagestyle{fancy}
\fancyhead{}
\fancyfoot{}
\fancyhead[L]{MAS6331 Algebra}
\fancyhead[C]{Homework 6}
\fancyhead[R]{Sai Sivakumar}
\fancyfoot[R]{\thepage}
\renewcommand{\headrulewidth}{1pt}

% paragraph indentation/spacing
\setlength{\parindent}{0cm}
\setlength{\parskip}{10pt}
\renewcommand{\baselinestretch}{1.25}

% extra commands defined here
\newcommand{\br}[1]{\left(#1\right)}
\newcommand{\sbr}[1]{\left[#1\right]}
\newcommand{\cbr}[1]{\left\{#1\right\}}

\newcommand{\dprime}{\prime\prime}

% bracket notation for inner product
\usepackage{mathtools}

\DeclarePairedDelimiterX{\abr}[1]{\langle}{\rangle}{#1}

\DeclareMathOperator{\Span}{span}
\DeclareMathOperator{\nullity}{nullity}
\DeclareMathOperator\Aut{Aut}
\DeclareMathOperator\Inn{Inn}
\DeclareMathOperator{\Orb}{Orb}
\DeclareMathOperator{\lcm}{lcm}
\DeclareMathOperator{\Hol}{Hol}
\DeclareMathOperator{\Jac}{Jac}
\DeclareMathOperator{\rad}{rad}
\DeclareMathOperator{\Tor}{Tor}
\DeclareMathOperator{\End}{End}
\DeclareMathOperator{\Gal}{Gal}
\DeclareMathOperator{\id}{id}

% set page count index to begin from 1
\setcounter{page}{1}

\begin{document}
\subsection*{Graded}
\begin{enumerate}
    \item (14.4.6) Prove that $\mathbb{F}_p(x,y)/\mathbb{F}_p(x^p,y^p)$ is not a simple extension by explicitly exhibiting an infinite number of intermediate subfields. \begin{proof}
        The extension $\mathbb{F}_p(x,y)/\mathbb{F}_p(x^p,y^p)$ is degree $p^2$ as the minimal polynomial for $x$ is $t^p-x^p$ and the minimal polynomial for $y$ is $t^p-y^p$, so the extension is degree $p^2$ by the tower law. We show that there are infinitely many distinct intermediate subfields which are degree $p$ extensions of $F = \mathbb{F}_p(x^p,y^p)$. 

        Consider the family of extensions $F(x+cy)/F$ for $c\in F$, which is infinite since $F$ is infinite (as $x,y$ are indeterminates). Each extension is degree $p$ since the minimal polynomial for $x+cy$ is $t^p-(x+cy)^p$ and by Frobenius we have $(x+cy)^p = x^p+c^py^p\in F$.
        
        If two extensions $F(x+cy)$ and $F(x+dy)$ were equivalent, then the difference $(c-d)y\in F(x+cy)$, and so also $x\in F(x+cy)$ so that $F(x,y) = \mathbb{F}_p(x,y)\subseteq F(x+cy)$. This is impossible by degree considerations, so it follows that any two extensions in the given family are distinct. 

        Since there are infinitely many distinct intermediate subfields of the extension $\mathbb{F}_p(x,y)/\mathbb{F}_p(x^p,y^p)$, it cannot be simple.
    \end{proof}
    \item (14.5.7) Show that complex conjugation restricts to the automorphism $\sigma_{-1}\in \Gal(\mathbb{Q}(\zeta_n)/\mathbb{Q})$ of the cyclotomic field of $n^\text{th}$ roots of unity. Show that the field $K^+ = \mathbb{Q}(\zeta_n+\zeta_n^{-1})$ is the subfield of real elements in $K = \mathbb{Q}(\zeta_n)$, called the \textit{maximal real subfield of $K$}. \begin{proof}
        Recall that complex conjugation respects sums and products so that for any rational function $P(\zeta_n)/Q(\zeta_n)$ of $\zeta_n$ over $\mathbb{Q}$, its complex conjugate is $P(\overline{\zeta_n})/Q(\overline{\zeta_n})$. But $\zeta_n = \exp(2\pi i /n)$ so that $\overline{\zeta_n} = \exp(-2\pi i /n) = \zeta_n^{-1}$. Furthermore, $\zeta_n^{-1}$ is also a primitive $n$-th root of unity since $-1$ is coprime with $n$. So complex conjugation restricted to $\mathbb{Q}(\zeta_n)$ agrees with the automorphism $\sigma_{-1}$ which sends $\zeta_n$ to $\zeta_n^{-1}$.

        Observe that elements fixed by complex conjugation are real and conversely. We find the fixed field of the subgroup generated by $\cbr{\sigma_{-1}}$ and show it is $K^+$ as above. Observe that elements of $K^+$ are rational functions of $\zeta_n+\zeta_n^{-1}$, and so $\sigma_{-1}$ fixes these rational functions since $\sigma_{-1}(\zeta_n+\zeta_n^{-1}) = \zeta_n^{-1}+\zeta_n$ and $\sigma_{-1}$ fixes real elements. Hence $K^+$ is contained in the fixed field.

        The degree of $K$ over $K^+$ is $2$: The monic polynomial $x^2-(\zeta_n+\zeta_n^{-1})x+1$ is irreducible since its roots $\zeta_n,\zeta_n^{-1}$ are complex, so it is the minimal polynomial for $\zeta_n$ over $K^+$. The extension $K/\mathbb{Q}$ is Galois with Abelian Galois group $(\mathbb{Z}/n\mathbb{Z})^\times$ (so every intermediate subfield is Galois over $\mathbb{Q}$), which has order $\phi(n)$. The degree of the fixed field $K^H$ of $H = \cbr{1,\sigma_{-1}}\cong \mathbb{Z}/2\mathbb{Z}$ over $\mathbb{Q}$ must be $\phi(n)/2$ by the Galois correspondence. Hence the degree of $K$ over $K^H$ is $2$ also.

        But the fixed field contains $K^+$ as a subfield, and $K^H$ is a proper subfield of $K$ (as $K$ contains complex elements while $K^H$ only contains real elements). Since $2$ is prime it follows that $[K^H\colon K^+] = 1$. Hence $K^+$ is the fixed field of $H$, the maximal real subfield of $K$.
    \end{proof}
\end{enumerate}
\subsection*{Additional Problems}
\begin{enumerate}
    \item (14.4.1) Determine the Galois closure of the field $\mathbb{Q}(\sqrt{1+\sqrt{2}})$ over $\mathbb{Q}$. \begin{proof}
        We show that the Galois closure of $F = \mathbb{Q}(\sqrt{1+\sqrt{2}})$ is $K=\mathbb{Q}(\sqrt{1+\sqrt{2}},\sqrt{1-\sqrt{2}})$, the splitting field of the separable polynomial $p(x) = x^4-2x^2-1$ (with roots are $\pm\sqrt{1\pm\sqrt{2}}$). We first show that $p(x)$ is irreducible so that it is the minimal polynomial of $\sqrt{1+\sqrt{2}}$: The extension $F$ over $\mathbb{Q}(\sqrt{2})$ is degree $2$: consider the polynomial $x^2-1-\sqrt{2}$. We show that $\alpha = \sqrt{1+\sqrt{2}}$ is not a square in $\mathbb{Q}(\sqrt{2})$ to show that the polynomial is irreducible. Suppose $\alpha$ is a square of an element $c = a+b\sqrt{2}\in\mathbb{Q}$. Then $a^2+2b^2 = -1$ and $2ab = -1$, so $b$ is determined by $a$. We show that $a$ could not be rational to obtain the contradiction. With $b = -1/2a$ we find with some algebra that $a$ needs to satisfy $2a^4+2a^2+1 = 0$. By the rational root theorem $a$ could not be rational so by contradiction $\alpha$ is not a square of an element of $\mathbb{Q}(\sqrt{2})$, so the polynomial $x^2-1-\sqrt{2}$ is irreducible hence the minimal polynomial of $\sqrt{1+\sqrt{2}}$ over $\mathbb{Q}(\sqrt{2})$. It follows by the tower law that the degree of $F$ over $\mathbb{Q}$ is $4$, and since $\sqrt{1+\sqrt{2}}$ satisfied the degree $4$ polynomial $p(x)$ over $\mathbb{Q}$, it follows that $p(x)$ was irreducible. Hence $p(x)$ is the minimal polynomial of $\sqrt{1+\sqrt{2}}$ over $\mathbb{Q}$.

        Then consider any other Galois extension $L$ over $\mathbb{Q}$ containing $F$. We show that $K$ is contained in $L$. Any irreducible polynomial with a root in $L$ splits completely over $L$; in particular since $L$ contains $F$ the minimal polynomial $p(x)$ of $\sqrt{1+\sqrt{2}}$ splits completely, so $L$ contains $K$. Since $L$ was arbitrary it follows that $K$ is the smallest Galois extension of $\mathbb{Q}$ containing $F$, the Galois closure of $F$ as desired.
    \end{proof}
    The other two I could have done but I got home too late.
    \item (14.5.10) Prove that $\mathbb{Q}(\sqrt[3]{2})$ is not a subfield of any cyclotomic field over $\mathbb{Q}$. \begin{proof}
        Cyclotomic field extensions have Abelian Galois groups, so every intermediate subfield is Galois over $\mathbb{Q}$. But we saw earlier that $\mathbb{Q}(\sqrt[3]{2})$ is not Galois over $\mathbb{Q}$ so it could not be a subfield of a cyclotomic field extension.
    \end{proof}
\end{enumerate}
\subsection*{Feedback}
\begin{enumerate}
    \item None.
    \item Things are okay so far; same as usual I think.
\end{enumerate}
\end{document}

\item (14.5.8) Let $K_n = \mathbb{Q}(\zeta_{2^{n+2}})$ be the cyclotomic field of $2^{n+2}$-th roots of unity, $n\geq 0$. Set $\alpha_n = \zeta_{2^{n+2}} + \zeta_{2^{n+2}}^{-1}$ and $K_n^+ = \mathbb{Q}(\alpha_n)$, the maximal real subfield of $K_n$. \begin{enumerate}
    \item Show that for all $n\geq 0$, $[K_n\colon \mathbb{Q}] = 2^{n+1}$, $[K_n\colon K_n^+] = 2$, $[K_n^+\colon\mathbb{Q}] = 2^n$, and $[K_{n+1}^+\colon K_n^+] = 2$.
    \item Determine the quadratic equation satisfied by $\zeta_{2^{n+2}}$ over $K_n^+$ in terms of $\alpha_n$.
    \item Show that for $n\geq 0$, $\alpha_{n+1}^2 = 2+\alpha_n$ and hence show that \[\alpha_n = \sqrt{2+\sqrt{2+\sqrt{\cdots + \sqrt{2}}}}\quad\text{($n$ times)},\] giving an explicit formula for the (constructible) $2^{n+2}$-th roots of unity.
\end{enumerate}
\item (14.5.9) Notation as in the previous exercise. \begin{enumerate}
    \item Prove that $K_n^+$ is a cyclic extension of $\mathbb{Q}$ of degree $2^n$. [Use an explicit isomorphism $(\mathbb{Z}/2^{n+2}\mathbb{Z})^\times \cong \mathbb{Z}/2\mathbb{Z}\times\mathbb{Z}/2^n\mathbb{Z}$ as Abelian groups (i.e., $(\mathbb{Z}/2^{n+2}\mathbb{Z})^\times$ is isomorphic to a cyclic group of order 2 and a cyclic group of order $2^n$ --- cf. Exercises 22 and 23 of section 2.3).]
    \item Prove that $K_n$ is a biquadratic extension of $K_{n-1}^+$ and that two of the three intermediate subfields are $K_n^+$ and $K_{n-1}$. Prove that the remaining field intermediate between $K_{n-1}^+$ and $K_n$ is a cyclic extension of $\mathbb{Q}$ of degree $2^n$.
\end{enumerate}