\documentclass[11pt]{article}

% packages
\usepackage{physics}
% margin spacing
\usepackage[top=1in, bottom=1in, left=0.5in, right=0.5in]{geometry}
\usepackage{hanging}
\usepackage{amsfonts, amsmath, amssymb, amsthm}
\usepackage{systeme}
\usepackage[none]{hyphenat}
\usepackage{fancyhdr}
\usepackage[nottoc, notlot, notlof]{tocbibind}
\usepackage{graphicx}
\graphicspath{{./images/}}
\usepackage{float}
\usepackage{siunitx}
\usepackage{esint}
\usepackage{cancel}
\usepackage{enumitem}

% permutations (second line is for spacing)
\usepackage{permute}
\renewcommand*\pmtseparator{\,}

% colors
\usepackage{xcolor}
\definecolor{p}{HTML}{FFDDDD}
\definecolor{g}{HTML}{D9FFDF}
\definecolor{y}{HTML}{FFFFCF}
\definecolor{b}{HTML}{D9FFFF}
\definecolor{o}{HTML}{FADECB}
%\definecolor{}{HTML}{}

% \highlight[<color>]{<stuff>}
\newcommand{\highlight}[2][p]{\mathchoice%
  {\colorbox{#1}{$\displaystyle#2$}}%
  {\colorbox{#1}{$\textstyle#2$}}%
  {\colorbox{#1}{$\scriptstyle#2$}}%
  {\colorbox{#1}{$\scriptscriptstyle#2$}}}%

% header/footer formatting
\pagestyle{fancy}
\fancyhead{}
\fancyfoot{}
\fancyhead[L]{MAS6331 Algebra}
\fancyhead[C]{Homework 5}
\fancyhead[R]{Sai Sivakumar}
\fancyfoot[R]{\thepage}
\renewcommand{\headrulewidth}{1pt}

% paragraph indentation/spacing
\setlength{\parindent}{0cm}
\setlength{\parskip}{10pt}
\renewcommand{\baselinestretch}{1.25}

% extra commands defined here
\newcommand{\br}[1]{\left(#1\right)}
\newcommand{\sbr}[1]{\left[#1\right]}
\newcommand{\cbr}[1]{\left\{#1\right\}}

\newcommand{\dprime}{\prime\prime}

% bracket notation for inner product
\usepackage{mathtools}

\DeclarePairedDelimiterX{\abr}[1]{\langle}{\rangle}{#1}

\DeclareMathOperator{\Span}{span}
\DeclareMathOperator{\nullity}{nullity}
\DeclareMathOperator\Aut{Aut}
\DeclareMathOperator\Inn{Inn}
\DeclareMathOperator{\Orb}{Orb}
\DeclareMathOperator{\lcm}{lcm}
\DeclareMathOperator{\Hol}{Hol}
\DeclareMathOperator{\Jac}{Jac}
\DeclareMathOperator{\rad}{rad}
\DeclareMathOperator{\Tor}{Tor}
\DeclareMathOperator{\End}{End}
\DeclareMathOperator{\Gal}{Gal}
\DeclareMathOperator{\id}{id}

% set page count index to begin from 1
\setcounter{page}{1}

\begin{document}
\subsection*{Graded}
\begin{enumerate}
    \item (14.2.28) Let $f(x)\in F[x]$ be an irreducible (and separable) polynomial of degree $n$ over the field $F$, let $L$ be the splitting field of $f(x)$ over $F$ and let $\alpha$ be a root of $f(x)$ in $L$. If $K$ is any Galois extension of $F$ contained in $L$, show that the polynomial $f(x)$ splits into a product of $m$ irreducible polynomials each of degree $d$ over $K$, where $m = [F(\alpha)\cap K\colon F]$ and $d = [K(\alpha)\colon K]$. [If $H$ is the subgroup of the Galois group of $L$ over $F$ corresponding to $K$ then the factors of $f(x)$ over $K$ correspond to the orbits of $H$ on the roots of $f(x)$. Then use Exercise 9 of Section 4.1] \begin{proof}
        Let $G = \Gal(L/F)$ and let $H = \Gal(L/K)$. Observe $H$ is normal in $G$ since $K$ is Galois over $F$. Furthermore, observe that the claim to prove is clear when $\alpha\in K$: since $K$ is Galois over $F$ and $f(x)$ has a root in $K$ then all of its roots are in $K$. So suppose further that $\alpha\not\in K$.
        
        There is a transitive group action of $G$ on the set of roots of $f(x)$ (all belonging to $L$) since $L$ is Galois over $F$. Since $H$ is a subgroup of $G$ it also acts on the set of roots of $f(x)$, and denote its distinct orbits by $\mathcal{O}_1,\dots,\mathcal{O}_r$; let $\alpha$ be in $\mathcal{O}_1$ by relabeling if needed.

        We show that these orbits are in correspondence with the irreducible monic factors of $f(x)$ over $K$: Observe that each orbit $\mathcal{O}_i$ is a collection of distinct Galois conjugates $\beta_j$ of roots under $H$; it follows that $\prod (x-\beta_j)$ is irreducible (if it were not then we could partition $\mathcal{O}_i$ into two sets which are orbits of $H$, which is impossible by minimality of the orbit $\mathcal{O}_i$) and divides $f(x)$. Similarly, if $g_i(x)$ is an irreducible monic factor of $f(x)$ over $K$ it is the minimal polynomial of one of its roots $\beta_k$. It follows that the other roots of $g_i(x)$ are Galois conjugates of $\beta_k$ under $H$ and thus the roots form one of the orbits of $H$. The degree of each of these factors is exactly the size of their corresponding orbit (i.e. the number of roots it has).

        Then apply Exercise 9 of Section 4.1 to see that $G$ permutes the orbits of $H$ transitively and each orbit has the same cardinality. In particular with $\alpha\in \mathcal{O}_1$ we have $\abs{\mathcal{O}_1} = \abs{H\colon H\cap G_\alpha}$, where $G_\alpha\leq G$ is the stabilizer of $\alpha$ in $G$, and $r = \abs{G\colon HG_a}$.

        Observe that the fixed field of $G_\alpha$ is $F(\alpha)$: every element of $F(\alpha)$ is fixed by $G_\alpha$ since $\alpha$ is fixed by $G_\alpha$, and since $L$ is the field $F$ adjoined with every root of $f(x)$, any element of $L$ fixed by $G_\alpha$ is a rational function of only $\alpha$ over $F$. Use the Galois correspondence to see that $\abs{\mathcal{O}_1} = \abs{H\colon H\cap G_\alpha} = [KF(\alpha)\colon K] = [K(\alpha)\colon K]$ ($F\subseteq K$) and the number of orbits $r$ is equal to $\abs{G\colon HG_\alpha} = [K\cap F(\alpha)\colon F]$. Thus $f(x)$ splits into the product of $[K\cap F(\alpha)\colon F]$ many irreducible monic factors times a unit, where each factor has degree $[K(\alpha)\colon K]$.
    \end{proof}
    \item (14.4.4) For \textit{any} Galois extension $K$ of $F$, show the irreducible (and separable) polynomial $f(x)\in F[x]$ factors in $K[x]$ as in Exercise 28 of Section 2 (whether or not $K$ is contained in the Galois closure $L$ of $f(x)$). [Show first that the factorization of $f(x)$ over $K$ is the same as its factorization over $L\cap K$. Then show the factors of $f(x)$ over $L\cap K$ correspond to the orbits of $H = \Gal(L/L\cap K)$ on the roots of $f(x)$ and use Exercise 9 of Section 4.1.] \begin{proof}
       We show first that the factorization of $f(x)$ over $K$ is the same as its factorization over $L\cap K$, where $L$ is the splitting field of $f(x)\in F[x]$. We show that if $f(x)$ has factorization $q_1(x)\cdots q_r(x)$ of irreducibles over $K$ then the same factorization holds over $L\cap K$. Take any $q_i(x)$ and observe that any root of $q_i(x)$ is a root of $f(x)$ so it is found in the splitting field $L$. As a result we can factorize $q_i(x)$ as the product $c\prod(x-\beta_j)$ in $L[x]$, where $c$ is a unit and $\beta_j$ is a distinct (since $f(x)$ is separable) root of $q_i(x)$. By expanding the product it follows that $q_i(x)$ is in $L[x]$, and so $q_i(x)$ is in $(L\cap K)[x]$. It follows by unique factorization that the factorization of $f(x)$ into $q_1(x)\cdots q_r(x)$ is the same in $K$ and in $L\cap K$.

       Let $G = \Gal(L/F)$ and $H = \Gal(L/L\cap K)$. Observe $G$ acts transitively on the roots of $f(x)$ by permuting them, and the subgroup $H$ acts on the roots of $f(x)$ as well. We show that $H$ is normal in $G$ by showing that $L\cap K$ is Galois over $F$: Since $L,K$ are Galois over $F$ they are both finite, separable, and normal extensions of $F$. Then the extension $L\cap K$ over $F$ is finite since $L$ and $K$ are, and it is separable since $L$ is separable (Subfields of separable extensions are seperable: Take any element $\alpha\in L\cap K$ and view it as an element of $L$, then the minimal polynomial of $\alpha$ over $F$ is separable as desired.). The extension $L\cap K$ is normal since $L$ and $K$ are normal extensions (Intersections of normal extensions are normal: Take any irreducible polynomial $p(x)\in F[x]$ with root $\beta\in L\cap K$. Then with $\beta\in L$, $f(x)$ splits into linear factors over $L$, similarly over $K$. But by unique factorization of polynomials these factorizations must be the same so $f(x)$ splits into linear factors over $L\cap K$.).
       
       Thus $L\cap K$ is a Galois extension over $F$ contained in $L$ as desired, and so $H$ is normal in $G$.
       Use the previous exercise with $L\cap K$ in place of $K$ and obtain the desired factorization of $f(x)$ over $(L\cap K)[x]$, which is the same factorization over $K[x]$ as shown above.
    \end{proof}
\end{enumerate}
\subsection*{Additional Problems}
\begin{enumerate}
    \item (14.2.27) Let $\alpha = \sqrt{(2+\sqrt{2})(3+\sqrt{3})}$ (positive square roots for concreteness) and consider the extension $E = \mathbb{Q}(\alpha)$. \begin{enumerate}
        \item Show that $a = (2+\sqrt{2})(3+\sqrt{3})$ is not a square in $F = \mathbb{Q}(\sqrt{2}, \sqrt{3})$. [If $a = c^2$, $c\in F$, then $a\varphi(a) = (2+\sqrt{2})^2(6) = (c\varphi(c))^2$ for the automorphism $\varphi\in \Gal(F/\mathbb{Q})$ fixing $\mathbb{Q}(\sqrt{2})$. Since $c\varphi(c) = N_{F/\mathbb{Q}(\sqrt{2})}(c)\in \mathbb{Q}(\sqrt{2})$ conclude that $\sqrt{6}\in \mathbb{Q}(\sqrt{2})$, a contradiction.]
        \item Conclude from (a) that $[E\colon \mathbb{Q}] = 8$. Prove that the roots of the minimal polynomial over $\mathbb{Q}$ for $\alpha$ are the $8$ elements $\pm\sqrt{(2\pm\sqrt{2})(3\pm\sqrt{3})}$.
        \item Let $\beta = \sqrt{(2-\sqrt{2})(3+\sqrt{3})}$. Show that $\alpha\beta = \sqrt{2}(3+\sqrt{3})\in F$ so that $\beta\in E$. Show similarly that the other roots are also elements of $E$ so that $E$ is a Galois extension of $\mathbb{Q}$. Show that the elements of the Galois group are precisely the maps determined by sending $\alpha$ to one of the eight elements in (b).
        \item Let $\sigma\in \Gal(E/\mathbb{Q})$ be the automorphism which maps $\alpha$ to $\beta$. Show that since $\sigma(\alpha^2) = \beta^2$ that $\sigma(\sqrt{2}) = -\sqrt{2}$ and $\sigma(\sqrt{3}) = \sqrt{3}$. From $\alpha\beta = \sqrt{2}(3+\sqrt{3})$ conclude that $\sigma(\alpha\beta) = -\alpha\beta$ and hence $\sigma(\beta) = -\alpha$. Show that $\sigma$ is an element of order $4$ in $\Gal(E/\mathbb{Q})$.
        \item Show similarly that the map $\tau$ defined by $\tau(\alpha) = \sqrt{(2+\sqrt{2})(3-\sqrt{3})}$ is an element of order $4$ in $\Gal(E/\mathbb{Q})$. Prove that $\sigma$ and $\tau$ generate the Galois group, $\sigma^4=\tau^4=1$, $\sigma^2=\tau^2$ and that $\sigma\tau = \tau\sigma^3$.
        \item Conclude that $\Gal(E/\mathbb{Q})\cong Q_8$, the quaternion group of order $8$.
    \end{enumerate}
    \begin{proof}
        (a) Suppose that $a = c^2$ for some $c\in F$. Then $a\varphi(a) = c^2\varphi(c^2) = (c\varphi(c))^2 = (2+\sqrt{2})(3+\sqrt{3})(2+\sqrt{2})(3-\sqrt{3}) = (2+\sqrt{2})^2(6)$; since $c\varphi(c)$ is equal to $N_{F/\mathbb{Q}(\sqrt{2})}(c)\in \mathbb{Q}(\sqrt{2})$ (the Galois group of $F/\mathbb{Q}(\sqrt{2})$ only has two elements), it follows that $(2+\sqrt{2})\sqrt{6}\in \mathbb{Q}(\sqrt{2})$, meaning $\sqrt{6}\in\mathbb{Q}(\sqrt{2})$, which is impossible. So $a$ is not a square in $F$.

        (b) The field extension $F/\mathbb{Q}$ is degree $4$, and the field extension $F(\alpha)/F$ is degree $2$ since $\alpha$ is not an element of $F$ by (a) and $\alpha$ has minimal polynomial $x^2-a\in F[x]$. Hence $\mathbb{Q}(\sqrt{2},\sqrt{3},\alpha)$ is a degree $8$ extension of $\mathbb{Q}$. Observe that $\alpha^2/(2+\sqrt{2}) - 3 = \sqrt{3}$ so that $\mathbb{Q}(\sqrt{2},\sqrt{3},\alpha) = \mathbb{Q}(\sqrt{2},\alpha) = E(\sqrt{2})$ We show that $E(\sqrt{2}) = E$ by contradiction. Suppose the degree of the extension is $2$ (the minimal polynomial is $x^2-2$) and apply the automorphism sending $\sqrt{2}\mapsto -\sqrt{2}$ which fixes $E$ to see that $\alpha^2\mapsto (2-\sqrt{2})(3+\sqrt{3})\neq \alpha^2$, which is a contradiction. Hence $E = E(\sqrt{2})$, and so $E$ is a degree $8$ extension of $\mathbb{Q}$. 

        Observe that $f(x) = \prod \left(x \pm\sqrt{(2\pm \sqrt{2})(3\pm \sqrt{3})}\right) = x^8 - 24 x^6 + 144 x^4 - 288 x^2 + 144$ is a monic degree eight polynomial over $\mathbb{Q}$ with the eight roots $\pm\sqrt{(2\pm \sqrt{2})(3\pm \sqrt{3})}$ as desired. Since $[E\colon \mathbb{Q}] = 8$, it follows that this polynomial is irreducible, and so is the minimal polynomial for $\alpha$.

        (c) With $\alpha\beta = \sqrt{(2+\sqrt{2})(3+\sqrt{3})}\sqrt{(2-\sqrt{2})(3+\sqrt{3})} = \sqrt{2(3+\sqrt{3})^2} = \sqrt{2}(3+\sqrt{3})$, we have that $\beta = \sqrt{2}(3+\sqrt{3})/\alpha\in E$. We have similarly that $\gamma = \sqrt{(2+\sqrt{2})(3-\sqrt{3})} = (2+\sqrt{2})\sqrt{6}/\alpha$ and $\omega = \sqrt{(2-\sqrt{2})(3-\sqrt{3})} = [\alpha\sqrt{2}(3-\sqrt{3})]/[(2+\sqrt{2})\sqrt{6}]$. It follows that $E$ is the splitting field for the irreducible separable polynomial $f(x)$, so $E$ is a Galois extension of $\mathbb{Q}$. Since we can write each of the eight roots in terms of $\alpha$, every automorphism in $\Gal(E/\mathbb{Q})$ permuting the roots of $f(x)$ is determined by where $\alpha$ is sent.

        (d) With $\sigma(\alpha^2) = (2-\sqrt{2})(3+\sqrt{3}) = \beta^2$, observe that (since $\sigma$ fixes $\mathbb{Q}$) $\sigma(\sqrt{2}) = -\sqrt{2}$ and $\sigma$ fixes $\sqrt{3}$. Then with $\alpha\beta = \sqrt{2}(3+\sqrt{3})$ we have that $\sigma(\alpha\beta) = \sigma(\alpha)\sigma(\beta) = \beta\sigma(\beta) = -\sqrt{2}(3+\sqrt{3}) = -\alpha\beta$; by cancellation $\sigma(\beta) = -\alpha$. It follows that $\sigma$ has order $4$ ($\alpha\mapsto\beta\mapsto -\alpha\mapsto -\beta \mapsto \alpha$).
        
        (e) The map $\tau$ sending $\alpha$ to $\gamma$ is similarly of order $4$: With $\alpha\gamma = (2+\sqrt{2})\sqrt{6}$ and $\tau(\alpha^2) = \gamma^2$, we have that $\tau$ sends $\sqrt{3}$ to $-\sqrt{3}$ and fixes $\sqrt{2}$. Thus $\tau(\alpha\gamma) = \tau(\alpha)\tau(\gamma) = \gamma\tau(\gamma) = -(2+\sqrt{2})\sqrt{6} = -\alpha\gamma$, so by cancellation $\tau(\gamma) = -\alpha$. Hence $\tau$ is of order $4$ as desired.

        Then we check that \begin{align*}
            &\sigma\colon\alpha\mapsto\beta,\quad \tau\colon\alpha\mapsto\gamma,\\
            &\sigma^2 = \tau^2\colon\alpha\mapsto-\alpha,\\
            &\sigma^3\colon\alpha\mapsto-\beta,\quad\tau^3\colon\alpha\mapsto-\gamma,\\
            &\tau\sigma\colon\alpha\mapsto\omega,\\
            &\tau\sigma^{-1} = \tau\sigma^3 = \sigma\tau\colon\alpha\mapsto-\omega,\\
            &\id_E=\tau^4= \sigma^4\colon\alpha\mapsto\alpha
        \end{align*} form the Galois group.

        (e) Observe that the group above is given by the presentation $\abr{\sigma,\tau\mid \sigma^4=\id_E,\sigma^2=\tau^2,\sigma\tau = \tau\sigma^{-1}}$, which is $Q_8$, the quaternion group of order $8$, up to isomorphism.
        \end{proof}
    \item (14.3.6) Suppose $K = \mathbb{Q}(\theta) = \mathbb{Q}(\sqrt{D_1},\sqrt{D_2})$ with $D_1,D_2\in\mathbb{Z}$, is a biquadratic extension and that $\theta = a+b\sqrt{D_1} + c\sqrt{D_2} + d\sqrt{D_1D_2}$ where $a,b,c,d\in\mathbb{Z}$ are integers. Prove that the minimal polynomial $m_\theta(x)$ for $\theta$ over $\mathbb{Q}$ is irreducible of degree $4$ over $\mathbb{Q}$ but is reducible modulo every prime $p$. In particular show that the polynomial $x^4-10x^2+1$ is irreducible in $\mathbb{Z}[x]$ but is reducible modulo every prime. [Use the fact that there are no biquadratic extensions over finite fields.] \begin{proof}
        Since $\mathbb{Q}(\theta)$ is a degree four extension of $\mathbb{Q}$, it follows that $m_\theta(x)$ is irreducible of degree $4$. There are no biquadratic extensions of finite fields, since extensions of finite fields are necessarily cyclic. If we view $m_\theta(x)$ as an element of $\mathbb{F}_p[x]$ and suppose it is irreducible, then its splitting field is an extension of $\mathbb{F}_p$, which is $\mathbb{F}_{p^4}$ (if $f(x)\in \mathbb{F}_p[x]$ is irreducible of degree $n$ then $\mathbb{F}_p[x]/(f(x))\cong \mathbb{F}_{p^n}$). But $\Gal(\mathbb{F}_{p^4}/\mathbb{F}_p)\cong \mathbb{Z}/4\mathbb{Z}$, and we cannot find a suitable automorphism permuting the roots of $m_\theta(x)$ which has order $4$ (the Klein $4$-group has elements of order $2$ and $1$), which means the Galois group could not be cyclic as desired. Hence $m_\theta(x)$ is reducible mod $p$ for any prime $p$.
    \end{proof}
    \item (14.3.9) Let $q = p^m$ be a power of the prime $p$ and let $\mathbb{F}_q = \mathbb{F}_{p^m}$ be the finite field with $q$ elements. Let $\sigma_q = \sigma_p^m$ be the $m^{\text{th}}$ power of the Frobenius automorphism $\sigma_p$, called the $q$-Frobenius automorphism. \begin{enumerate}
        \item Prove that $\sigma_q$ fixes $\mathbb{F}_q$.
        \item Prove that every finite extension of $\mathbb{F}_q$ of degree $n$ is the splitting field of $x^{q^n}-x$ over $\mathbb{F}_q$, hence is unique.
        \item Prove that every finite extension of $\mathbb{F}_q$ of degree $n$ is cyclic with $\sigma_q$ as generator.
        \item Prove that the subfields of the unique extension of $\mathbb{F}_q$ of degree $n$ are in bijective correspondence with the divisors $d$ of $n$.
    \end{enumerate}
    \begin{proof}
        (a) For any element $a\in\mathbb{F}_q$, observe that $a$ satisfies the polynomial $x^q-x$. It follows that $\sigma_q(a) = a^q = a$, so $\sigma_q$ fixes $\mathbb{F}_q$ as desired.

        (b) Observe that $x^{q^n}-x$ over $\mathbb{F}_q$ is separable (its derivative is $-1$), and that for any roots $\alpha,\beta$ we have that $\alpha\beta,\alpha^{-1}, (\alpha\pm \beta)$ are also roots (the first two are clear, for the third use the binomial theorem and the fact that we are working in characteristic $p$, or apply the Frobenius endomorphism directly). Hence the set $\mathbb{F}$ of these $q^n$ roots form a field which is a subfield of the splitting field, meaning $\mathbb{F}$ is the splitting field of $x^{q^n}-x$ over $\mathbb{F}_q$. It follows that the degree of the extension is $n$ since $\mathbb{F}$ has $q^n$ elements. Conversely, let $\mathbb{F}$ be a degree $n$ extension of $\mathbb{F}_q$ so that $\mathbb{F}$ has $q^n$ elements, and its multiplicative group is cyclic with order $q^n-1$. So each nonzero element of $\mathbb{F}$ satisfies the polynomial $x^{q^n-1}-1$, and so every element of $\mathbb{F}$ is a root of $x^{q^n}-x$; this polynomial is separable and has exactly $q^n$ roots so $\mathbb{F}$ is the splitting field for this polynomial. Hence every finite extension of $\mathbb{F}_q$ of degree $n$ is the unique splitting field of $x^{q^n}-x$ over $\mathbb{F}_q$.

        (c) For $\mathbb{F}$ a degree $n$ extension of $\mathbb{F}_q$, observe that $\sigma_q$ is injective, hence surjective since $\mathbb{F}$ is finite. So $\sigma_q$ is an automorphism of $\mathbb{F}$ fixing $\mathbb{F}_q$.

        We expect there to be $n$ automorphism of this form. We show that the cyclic group generated by $\sigma_q$ is the Galois group: Observe that each power of $\sigma_q$ is a distinct automorphism, and that $\sigma_q^n$ is the identity map. Suppose the order of $\sigma_q$ was less than $n$ -- this means that for some $i<n$, for every element $a\in \mathbb{F}$ we would have $a^{q^i} = a$, which means that every element of $\mathbb{F}$ satisfied the polynomial $x^{q^i}-x$ which only has $q^i$ roots at most, while $\mathbb{F}$ has $q^n$ elements, impossible. Hence $\sigma_q$ generates the Galois group, which is cyclic of order $n$.
        
        (d) By the Galois correspondence, subfields of the degree $n$ extension $\mathbb{F}$ of $\mathbb{F}_q$ are in bijective correspondence with the subgroups of the Galois group, which is isomorphic to $\mathbb{Z}/n\mathbb{Z}$. But the subgroups of $\mathbb{Z}/n\mathbb{Z}$ are unique and cyclic of order $d$ where $d$ divides $n$. So each divisor $d$ of $n$ determines a unique subgroup, which by the Galois correspondence determines a unique subfield, and vice versa.
    \end{proof}
    \item (Conjugate Fields) Let $K/F$ be Galois, and $E = K^H$ be the intermediate field corresponding to $H\subset\Gal(K/F)$ under the Galois correspondence. \begin{enumerate}
        \item For $\tau\in \Gal(K/F)$, show that $\tau(E)$ is the fixed field of $\tau H\tau^{-1}$. \begin{proof}
            Observe that $\tau(E)$ is a field since $\tau$ is a field isomorphism of $E$ with $\tau(E)$ ($\tau$ restricted to $E$ is a nonzero field homomorphism with left and right inverses). We have that any element of $\tau(E)$ is of the form $\tau(e)$ for $e\in E$, and this element is fixed by any element $\tau h\tau^{-1}\in \tau H\tau^{-1}$: $(\tau h\tau^{-1})(\tau(e)) = \tau(h(e)) = \tau(e)$ ($H$ fixes $E$). Conversely, if $k\in K$ is fixed by $\tau h\tau^{-1}$, then $h\tau^{-1}(k) = \tau^{-1}(k) = e$ for some $e\in E$. Thus $k = \tau(e)$. Hence $\tau(E)$ is the fixed field of $\tau H\tau^{-1}$.
        \end{proof}
        \item Take $F = \mathbb{Q}$, $K$ the splitting field of $x^3-2$ over $\mathbb{Q}$, and $H$ be a subgroup of order $2$ in $\Gal(K/F)\cong S_3$. What are the conjugate fields of the fixed field $E$, i.e. what is \[\cbr{\tau(K^H)\colon \tau\in \Gal(K/F)}?\] \begin{proof}
            Conjugate fields are in correspondence with conjugate subgroups. If $H$ is generated by a $2$-cycle $\tau$, then the other conjugate subgroups are given by $\sigma H \sigma^{-1}$ and $\sigma^2 H \sigma^{-2}$ where $\sigma$ is any $3$-cycle (these are the other two subgroups of order $2$ in $S_3$). It follows that the conjugate fields are $\mathbb{Q}(\sqrt[3]{2}), \mathbb{Q}(\zeta_3\sqrt[3]{2}),\mathbb{Q}(\zeta_3^2\sqrt[3]{2})$ where $\zeta_3$ is a primitive third root of unity (These are the subfields of $K =\mathbb{Q}(\sqrt[3]{2},\zeta_3)$ which are fixed by the order two subgroups of $\Gal(K/F)\cong S_3$; see pages 546 and 568 for the explicit automorphisms and field diagrams involved).
        \end{proof}
    \end{enumerate}
\end{enumerate}
\subsection*{Feedback}
\begin{enumerate}
    \item None.
    \item Things are okay so far; same as usual I think.
\end{enumerate}
\end{document}