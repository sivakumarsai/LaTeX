\documentclass[11pt]{article}
\headheight = 14pt

% packages
\usepackage{physics}
% margin spacing
\usepackage[top=1in, bottom=1in, left=0.5in, right=0.5in]{geometry}
\usepackage{hanging}
\usepackage{amsfonts, amsmath, amssymb, amsthm}
\usepackage{systeme}
\usepackage[none]{hyphenat}
\usepackage{fancyhdr}
\usepackage[nottoc, notlot, notlof]{tocbibind}
\usepackage{graphicx}
\graphicspath{{./images/}}
\usepackage{float}
\usepackage{siunitx}
\usepackage{esint}
\usepackage{cancel}
\usepackage{enumitem}

% colors
\usepackage{xcolor}
\definecolor{p}{HTML}{FFDDDD}
\definecolor{g}{HTML}{D9FFDF}
\definecolor{y}{HTML}{FFFFCF}
\definecolor{b}{HTML}{D9FFFF}
\definecolor{o}{HTML}{FADECB}
%\definecolor{}{HTML}{}

% \highlight[<color>]{<stuff>}
\newcommand{\highlight}[2][p]{\mathchoice%
  {\colorbox{#1}{$\displaystyle#2$}}%
  {\colorbox{#1}{$\textstyle#2$}}%
  {\colorbox{#1}{$\scriptstyle#2$}}%
  {\colorbox{#1}{$\scriptscriptstyle#2$}}}%

% header/footer formatting
\pagestyle{fancy}
\fancyhead{}
\fancyfoot{}
\fancyhead[L]{PHZ3113}
\fancyhead[C]{HW24}
\fancyhead[R]{Sai Sivakumar}
\fancyfoot[R]{\thepage}
\renewcommand{\headrulewidth}{1pt}

% paragraph indentation/spacing
\setlength{\parindent}{0cm}
\setlength{\parskip}{5pt}
\renewcommand{\baselinestretch}{1.25}

% extra commands defined here
\newcommand{\ihat}{\boldsymbol{\hat{\textbf{\i}}}}
\newcommand{\jhat}{\boldsymbol{\hat{\textbf{\j}}}}
\newcommand{\khat}{\boldsymbol{\hat{\textbf{k}}}}
\newcommand{\dr}{\vec{r}~^{\prime}(t)}
\newcommand{\dx}{x^{\prime}(t)}
\newcommand{\dy}{y^{\prime}(t)}

\newcommand{\br}[1]{\left(#1\right)}
\newcommand{\sbr}[1]{\left[#1\right]}
\newcommand{\cbr}[1]{\left\{#1\right\}}

\newcommand{\dprime}{\prime\prime}
\newcommand{\lap}[2]{\mathcal{L}[#1](#2)}

% bracket notation for inner product
\usepackage{mathtools}

\DeclarePairedDelimiterX{\abr}[1]{\langle}{\rangle}{#1}

\DeclareMathOperator{\Span}{span}
\DeclareMathOperator\Arg{Arg}
\DeclareMathOperator\Log{Log}

% set page count index to begin from 1
\setcounter{page}{1}

\begin{document}
\begin{enumerate}[label=24.\arabic*]
    \item Let $\ket{V} = (3-4i)\ket{1} + (5-6i)\ket{2}$ and $\ket{W} = (1-i)\ket{1} + (2-3i)\ket{2}$ where $\ket{1}$ and $\ket{2}$ form an orthonormal basis. Find $\bra{V}\ket{V}$, $\bra{W}\ket{W}$, and $\bra{V}\ket{W}$.
    
    We have \begin{align*}
        \bra{V}\ket{V} &= (3+4i)(3-4i)\bra{1}\ket{1} + (5+6i)(5-6i)\bra{2}\ket{2}\\
        &= 25 + 61 = \boxed{86},\\
        \bra{W}\ket{W} &= (1+i)(1-i)\bra{1}\ket{1} + (2+3i)(2-3i)\bra{2}\ket{2}\\
        &= 2 + 13 = \boxed{15},\\
        \bra{V}\ket{W} &= (3+4i)(1-i)\bra{1}\ket{1} + (5+6i)(2-3i)\bra{2}\ket{2}\\
        &= 7 + i + 28 + -3i = \boxed{35 - 2i}.
    \end{align*}
    \item (a) Show that $\ket{1} = \frac{1}{\sqrt{2}}\begin{pmatrix}
        1 \\ i
    \end{pmatrix}$ and $\ket{2} = \frac{1}{\sqrt{2}}\begin{pmatrix}
        1\\ -i
    \end{pmatrix}$ form an orthonormal basis.
    \begin{proof}
        Directly computing gives \begin{align*}
            \bra{1}\ket{1} &= \frac{1}{2}[1\cdot 1 + (-i)\cdot i] = 1\\
            \bra{2}\ket{2} &= \frac{1}{2}[1\cdot 1 + i\cdot (-i)] = 1\\
            \bra{1}\ket{2} &= \frac{1}{2}[1\cdot 1 + (-i)\cdot (-i)] = 0,
        \end{align*} which implies that the vectors given form an orthonormal basis.
    \end{proof}
    (b) Expand $\ket{V} = \begin{pmatrix}
        1+i \\ \sqrt{3} + i
    \end{pmatrix}$ in this basis. What is $\bra{V}\ket{V}$ in this basis?

    Using the inner product, \begin{align*}
        \bra{1}\ket{V} &= \frac{1}{\sqrt{2}}[(1+i)(1) + (\sqrt{3}+ i )(-i)] = \frac{1}{\sqrt{2}}(2 + i(1-\sqrt{3}))\\
        \bra{2}\ket{V} &= \frac{1}{\sqrt{2}}[(1+i)(1) + (\sqrt{3}+ i )(i)] = \frac{1}{\sqrt{2}}i(1+\sqrt{3}),
    \end{align*} which implies \[\ket{V} = \frac{1}{\sqrt{2}}(2 + i(1-\sqrt{3}))\ket{1} +\frac{1}{\sqrt{2}}i(1+\sqrt{3})\ket{2} .\] Then \begin{align*}
        \bra{V}\ket{V} &= \frac{1}{2}(2 + i(1-\sqrt{3}))(2 - i(1-\sqrt{3}))\bra{1}\ket{1} - \frac{1}{2}i(1+\sqrt{3})i(1+\sqrt{3})\bra{2}\ket{2}\\
        &= \frac{1}{2}[4 + (\sqrt{3}-1)^2 + (\sqrt{3}+1)^2] = 6,
    \end{align*} which matches with the computation in (6).
\end{enumerate}

\end{document}