\documentclass[11pt]{article}
\headheight = 14pt

% packages
\usepackage{physics}
% margin spacing
\usepackage[top=1in, bottom=1in, left=0.5in, right=0.5in]{geometry}
\usepackage{hanging}
\usepackage{amsfonts, amsmath, amssymb, amsthm}
\usepackage{systeme}
\usepackage[none]{hyphenat}
\usepackage{fancyhdr}
\usepackage[nottoc, notlot, notlof]{tocbibind}
\usepackage{graphicx}
\graphicspath{{./images/}}
\usepackage{float}
\usepackage{siunitx}
\usepackage{esint}
\usepackage{cancel}

% colors
\usepackage{xcolor}
\definecolor{p}{HTML}{FFDDDD}
\definecolor{g}{HTML}{D9FFDF}
\definecolor{y}{HTML}{FFFFCF}
\definecolor{b}{HTML}{D9FFFF}
\definecolor{o}{HTML}{FADECB}
%\definecolor{}{HTML}{}

% \highlight[<color>]{<stuff>}
\newcommand{\highlight}[2][p]{\mathchoice%
  {\colorbox{#1}{$\displaystyle#2$}}%
  {\colorbox{#1}{$\textstyle#2$}}%
  {\colorbox{#1}{$\scriptstyle#2$}}%
  {\colorbox{#1}{$\scriptscriptstyle#2$}}}%

% header/footer formatting
\pagestyle{fancy}
\fancyhead{}
\fancyfoot{}
\fancyhead[L]{PHZ3113}
\fancyhead[C]{HW2}
\fancyhead[R]{Sai Sivakumar}
\fancyfoot[R]{\thepage}
\renewcommand{\headrulewidth}{1pt}

% paragraph indentation/spacing
\setlength{\parindent}{0cm}
\setlength{\parskip}{5pt}
\renewcommand{\baselinestretch}{1.25}

% extra commands defined here
\newcommand{\ihat}{\boldsymbol{\hat{\textbf{\i}}}}
\newcommand{\jhat}{\boldsymbol{\hat{\textbf{\j}}}}
\newcommand{\dr}{\vec{r}~^{\prime}(t)}
\newcommand{\dx}{x^{\prime}(t)}
\newcommand{\dy}{y^{\prime}(t)}

\newcommand{\br}[1]{\left(#1\right)}
\newcommand{\sbr}[1]{\left[#1\right]}
\newcommand{\cbr}[1]{\left\{#1\right\}}

\newcommand{\dprime}{\prime\prime}
\newcommand{\lap}[2]{\mathcal{L}[#1](#2)}

% bracket notation for inner product
\usepackage{mathtools}

\DeclarePairedDelimiterX{\abr}[1]{\langle}{\rangle}{#1}

\DeclareMathOperator{\Span}{span}
\DeclareMathOperator{\nullity}{nullity}
\DeclareMathOperator\Arg{Arg}
\DeclareMathOperator\Log{Log}


% set page count index to begin from 1
\setcounter{page}{1}

\begin{document}

\begin{enumerate}
    \item For the double pendulum system, the given Lagrangian was \[\mathcal{L} = \frac{1}{2}(m_1+m_2)\ell_1^2\dot{\theta_1}^2 + \frac{1}{2}m_2\ell_2^2\dot{\theta_2}^2 + m_2\ell_1\ell_2\cos(\theta_1-\theta_2)\dot{\theta_1}\dot{\theta_2} + (m_1+m_2)g\ell_1\cos(\theta_1) + m_2g\ell_2\cos(\theta_2).\] Then \begin{align*}
        \pdv{\mathcal{L}}{\theta_1} &= -m_2\ell_1\ell_2\sin(\theta_1-\theta_2)\dot{\theta_1}\dot{\theta_2} - (m_1+m_2)g\ell_1\sin(\theta_1)\\
        \pdv{\mathcal{L}}{\dot{\theta_1}} &= (m_1+m_2)\ell_1^2\dot{\theta_1} + m_2\ell_1\ell_2\cos(\theta_1-\theta_2)\dot{\theta_2}\\
        \dv{t}\pdv{\mathcal{L}}{\dot{\theta_1}} &= (m_1+m_2)\ell_1^2\ddot{\theta_1} - m_2\ell_1\ell_2\sin(\theta_1-\theta_2)(\dot{\theta_1}-\dot{\theta_2})\dot{\theta_2} + m_2\ell_1\ell_2\cos(\theta_1-\theta_2)\ddot{\theta_2}
    \end{align*} and \begin{align*}
        \pdv{\mathcal{L}}{\theta_2} &= m_2\ell_1\ell_2\sin(\theta_1-\theta_2)\dot{\theta_1}\dot{\theta_2} - m_2g\ell_2\sin(\theta_2)\\
        \pdv{\mathcal{L}}{\dot{\theta_2}} &= m_2\ell_2^2\dot{\theta_2} + m_2\ell_1\ell_2\cos(\theta_1-\theta_2)\dot{\theta_1}\\
        \dv{t}\pdv{\mathcal{L}}{\dot{\theta_2}} &= m_2\ell_2^2\ddot{\theta_2} - m_2\ell_1\ell_2\sin(\theta_1-\theta_2)(\dot{\theta_1}-\dot{\theta_1})\dot{\theta_1} + m_2\ell_1\ell_2\cos(\theta_1-\theta_2)\ddot{\theta_1}.
    \end{align*}
    Hence the Euler-Lagrange system of equations for the double pendulum can be expressed as \begin{multline*}
        0 = (m_1+m_2)\ell_1^2\ddot{\theta_1} - m_2\ell_1\ell_2\sin(\theta_1-\theta_2)(\dot{\theta_1}-\dot{\theta_2})\dot{\theta_2} + m_2\ell_1\ell_2\cos(\theta_1-\theta_2)\ddot{\theta_2}\\ + m_2\ell_1\ell_2\sin(\theta_1-\theta_2)\dot{\theta_1}\dot{\theta_2} + (m_1+m_2)g\ell_1\sin(\theta_1)
    \end{multline*} \begin{multline*}
        0 = m_2\ell_2^2\ddot{\theta_2} - m_2\ell_1\ell_2\sin(\theta_1-\theta_2)(\dot{\theta_1}-\dot{\theta_2})\dot{\theta_1} + m_2\ell_1\ell_2\cos(\theta_1-\theta_2)\ddot{\theta_1}\\ -m_2\ell_1\ell_2\sin(\theta_1-\theta_2)\dot{\theta_1}\dot{\theta_2} + m_2g\ell_2\sin(\theta_2)
    \end{multline*}
    \item Define a Lagrangian density \[\mathcal{L} = \mathcal{L}(\phi, \dot{\phi}, \phi^{\prime}; t),\] such that the action $S$ is given by \[S[\phi(x,t)] = \int_{-\infty}^{\infty}\int_{t_i}^{t_f} \mathcal{L}(\phi, \dot{\phi}, \phi^{\prime}; t)\dd{t}\dd{x} .\]
    
    (a) In minimizing the action, $\delta S = 0$. Thus \begin{align*}0 = \delta S &= \int_{-\infty}^{\infty}\int_{t_i}^{t_f} \br{\mathcal{L}(\phi+\dd{\phi}, \dot{\phi}+ \dot{\dd{\phi}}, \phi^{\prime}+\dd{\phi}^{\prime}; t)-\mathcal{L}(\phi, \dot{\phi}, \phi^{\prime}; t)}\dd{t}\dd{x} \\
    &= \int_{-\infty}^{\infty}\int_{t_i}^{t_f} \br{\pdv{\mathcal{L}}{\phi}\dd{\phi} + \pdv{\mathcal{L}}{\dot{\phi}}\dot{\dd{\phi}} + \pdv{\mathcal{L}}{\phi^{\prime}}\dd{\phi}^{\prime}}\dd{t}\dd{x} \\
    &= \int_{-\infty}^{\infty} \br{\eval{\pdv{\mathcal{L}}{\dot{\phi}}\dd{\phi}}_{t_i}^{t_f} + \eval{\pdv{\mathcal{L}}{\phi^{\prime}}\dd{\phi}}_{-\infty}^{\infty} + \int_{t_i}^{t_f} \br{\pdv{\mathcal{L}}{\phi}\dd{\phi} - \dv{t}\pdv{\mathcal{L}}{\dot{\phi}}\dd{\phi} - \dv{x}\pdv{\mathcal{L}}{\phi^{\prime}}\dd{\phi}} \dd{t} }\dd{x} \\
    &= \int_{-\infty}^{\infty}\int_{t_i}^{t_f}\br{\pdv{\mathcal{L}}{\phi} - \dv{t}\pdv{\mathcal{L}}{\dot{\phi}} - \dv{x}\pdv{\mathcal{L}}{\phi^{\prime}}} \dd{\phi}\dd{t}\dd{x},\end{align*} where $\dd{\phi}(t_i) = \dd{\phi}(t_f) = 0$ (fixed ends) and $\lim_{x\to -\infty}\dd{\phi} = \lim_{x\to \infty}\dd{\phi} = 0$ (it would be sad if this did not vanish). Then the last integral can only vanish when \[\pdv{\mathcal{L}}{\phi} - \dv{t}\pdv{\mathcal{L}}{\dot{\phi}} - \dv{x}\pdv{\mathcal{L}}{\phi^{\prime}} = 0,\] which is the modified Euler-Lagrange equation.

    (b) Let \[\mathcal{L}(\phi,\dot{\phi}, \phi^{\prime};t) = \frac{1}{2}\rho\dot{\phi}^2 - \frac{1}{2}\tilde{\mathcal{K}}{\phi^{\prime}}^2.\]Taking derivatives, \begin{align*}
        \pdv{\mathcal{L}}{\phi} &= 0\\
        \pdv{\mathcal{L}}{\dot{\phi}} &= \rho\dot{\phi}\\
        \dv{t}\pdv{\mathcal{L}}{\dot{\phi}} &= \rho\ddot{\phi}\\
        \pdv{\mathcal{L}}{\phi^{\prime}} &= -\tilde{\mathcal{K}}\phi^{\prime}\\
        \dv{x}\pdv{\mathcal{L}}{\phi^{\prime}} &= -\tilde{\mathcal{K}}\phi^{\prime \prime},
    \end{align*} which by the modified Euler-Lagrange equation, yields the famous partial differential equation, the wave equation, \[\tilde{\mathcal{K}}\phi^{\prime \prime} = \rho\ddot{\phi}.\]
\end{enumerate}
\end{document}