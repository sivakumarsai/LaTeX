\documentclass[11pt]{article}
\headheight = 14pt

% packages
\usepackage{physics}
% margin spacing
\usepackage[top=1in, bottom=1in, left=0.5in, right=0.5in]{geometry}
\usepackage{hanging}
\usepackage{amsfonts, amsmath, amssymb, amsthm}
\usepackage{systeme}
\usepackage[none]{hyphenat}
\usepackage{fancyhdr}
\usepackage[nottoc, notlot, notlof]{tocbibind}
\usepackage{graphicx}
\graphicspath{{./images/}}
\usepackage{float}
\usepackage{siunitx}
\usepackage{esint}
\usepackage{cancel}

% colors
\usepackage{xcolor}
\definecolor{p}{HTML}{FFDDDD}
\definecolor{g}{HTML}{D9FFDF}
\definecolor{y}{HTML}{FFFFCF}
\definecolor{b}{HTML}{D9FFFF}
\definecolor{o}{HTML}{FADECB}
%\definecolor{}{HTML}{}

% \highlight[<color>]{<stuff>}
\newcommand{\highlight}[2][p]{\mathchoice%
  {\colorbox{#1}{$\displaystyle#2$}}%
  {\colorbox{#1}{$\textstyle#2$}}%
  {\colorbox{#1}{$\scriptstyle#2$}}%
  {\colorbox{#1}{$\scriptscriptstyle#2$}}}%

% header/footer formatting
\pagestyle{fancy}
\fancyhead{}
\fancyfoot{}
\fancyhead[L]{PHZ3113}
\fancyhead[C]{HW3}
\fancyhead[R]{Sai Sivakumar}
\fancyfoot[R]{\thepage}
\renewcommand{\headrulewidth}{1pt}

% paragraph indentation/spacing
\setlength{\parindent}{0cm}
\setlength{\parskip}{5pt}
\renewcommand{\baselinestretch}{1.25}

% extra commands defined here
\newcommand{\ihat}{\boldsymbol{\hat{\textbf{\i}}}}
\newcommand{\jhat}{\boldsymbol{\hat{\textbf{\j}}}}
\newcommand{\dr}{\vec{r}~^{\prime}(t)}
\newcommand{\dx}{x^{\prime}(t)}
\newcommand{\dy}{y^{\prime}(t)}

\newcommand{\br}[1]{\left(#1\right)}
\newcommand{\sbr}[1]{\left[#1\right]}
\newcommand{\cbr}[1]{\left\{#1\right\}}

\newcommand{\dprime}{\prime\prime}
\newcommand{\lap}[2]{\mathcal{L}[#1](#2)}

% bracket notation for inner product
\usepackage{mathtools}

\DeclarePairedDelimiterX{\abr}[1]{\langle}{\rangle}{#1}

\DeclareMathOperator{\Span}{span}
\DeclareMathOperator{\nullity}{nullity}
\DeclareMathOperator\Arg{Arg}
\DeclareMathOperator\Log{Log}


% set page count index to begin from 1
\setcounter{page}{1}

\begin{document}

\begin{enumerate}
    \item A particle of mass $m$ moves under the influence of gravity on the inner surface of a frictionless paraboloid of revolution $x^2+y^2 = az$.
    
    (a) In cylindrical coordinates the constraint is written as $r^2 - az = 0$. The coefficient for the Lagrange multipliers will be $(r^2-az)$.

    (b) To find the kinetic energy of this mass we must differentiate position, but to do so we can take time derivatives of $x= r\cos(\theta)$ and $y = r\sin(\theta)$. Then the kinetic energy is given by $\frac{1}{2}m\dot{z}^2 + \frac{1}{2}m(\dot{r}\cos(\theta)-r\sin(\theta)\dot{\theta})^2 + \frac{1}{2}m(\dot{r}\sin(\theta)+ r\cos(\theta)\dot{\theta})^2 = \frac{1}{2}m\dot{z}^2 + \frac{1}{2}m\dot{r}^2 + \frac{1}{2}mr^2\dot{\theta}^2$ (this is like using angular velocity and radial velocity and extracting kinetic energy from these coordinates) and the potential energy of this mass is given by $mgz$. 
    
    (c) Then $\mathcal{L} = \frac{1}{2}m\dot{z}^2 + \frac{1}{2}m\dot{r}^2 + \frac{1}{2}mr^2\dot{\theta}^2 - mgz + \lambda(r^2-az)$, so \begin{align*}
        S &= \int_{t_i}^{t_f}\br{\frac{1}{2}m\dot{z}^2 + \frac{1}{2}m\dot{r}^2 + \frac{1}{2}mr^2\dot{\theta}^2 - mgz + \lambda(r^2-az)}\dd{t} \\
        &\hspace*{6pt}\vdots \\
        \implies& \pdv{\mathcal{L}}{r}-\dv{t}\pdv{\mathcal{L}}{\dot{r}} = mr\dot{\theta}^2 + 2\lambda r -m\ddot{r} = 0\\
        & \pdv{\mathcal{L}}{\theta}-\dv{t}\pdv{\mathcal{L}}{\dot{\theta}} = -2mr\dot{r}\dot{\theta}-mr^2\ddot{\theta} = 0\\
        & \pdv{\mathcal{L}}{z}-\dv{t}\pdv{\mathcal{L}}{\dot{z}} = -\lambda a-mg-m\ddot{z} = 0\\
        & \pdv{\mathcal{L}}{\lambda}-\dv{t}\pdv{\mathcal{L}}{\dot{\lambda}} = r^2-az = 0
    \end{align*} 

    \item The frictional force is given by $f$, and so the work done by this force is given by the negative of displacement times $f$. So say an object with mass $m$ moves along the $x$ axis, where its displacement is given by $\phi$ (the potential energy does not change so along this axis so we can say it is $0$). Then to ``retrofit'' the Lagrangian, we take off from the kinetic energy the energy lost to friction at a given position, so \[\mathcal{L}(\phi,\dot{\phi};t) = \frac{1}{2}m\dot{\phi}^2 - f\phi.\] Then in taking derivatives, see that \[\pdv{\mathcal{L}}{\phi} = -f \text{ and } \dv{t}\pdv{\mathcal{L}}{\dot{\phi}} = m\ddot{\phi},\] so the equation of motion matches with what we would expect from Newtonian mechanics, \[m\ddot{\phi} = -f.\] This does not work if the mass changes direction.
\end{enumerate}
\end{document}