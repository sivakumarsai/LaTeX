\documentclass[11pt]{article}
\headheight = 14pt

% packages
\usepackage{physics}
% margin spacing
\usepackage[top=1in, bottom=1in, left=0.5in, right=0.5in]{geometry}
\usepackage{hanging}
\usepackage{amsfonts, amsmath, amssymb, amsthm}
\usepackage{systeme}
\usepackage[none]{hyphenat}
\usepackage{fancyhdr}
\usepackage[nottoc, notlot, notlof]{tocbibind}
\usepackage{graphicx}
\graphicspath{{./images/}}
\usepackage{float}
\usepackage{siunitx}
\usepackage{esint}
\usepackage{cancel}

% colors
\usepackage{xcolor}
\definecolor{p}{HTML}{FFDDDD}
\definecolor{g}{HTML}{D9FFDF}
\definecolor{y}{HTML}{FFFFCF}
\definecolor{b}{HTML}{D9FFFF}
\definecolor{o}{HTML}{FADECB}
%\definecolor{}{HTML}{}

% \highlight[<color>]{<stuff>}
\newcommand{\highlight}[2][p]{\mathchoice%
  {\colorbox{#1}{$\displaystyle#2$}}%
  {\colorbox{#1}{$\textstyle#2$}}%
  {\colorbox{#1}{$\scriptstyle#2$}}%
  {\colorbox{#1}{$\scriptscriptstyle#2$}}}%

% header/footer formatting
\pagestyle{fancy}
\fancyhead{}
\fancyfoot{}
\fancyhead[L]{PHZ3113}
\fancyhead[C]{HW5}
\fancyhead[R]{Sai Sivakumar}
\fancyfoot[R]{\thepage}
\renewcommand{\headrulewidth}{1pt}

% paragraph indentation/spacing
\setlength{\parindent}{0cm}
\setlength{\parskip}{5pt}
\renewcommand{\baselinestretch}{1.25}

% extra commands defined here
\newcommand{\ihat}{\boldsymbol{\hat{\textbf{\i}}}}
\newcommand{\jhat}{\boldsymbol{\hat{\textbf{\j}}}}
\newcommand{\dr}{\vec{r}~^{\prime}(t)}
\newcommand{\dx}{x^{\prime}(t)}
\newcommand{\dy}{y^{\prime}(t)}

\newcommand{\br}[1]{\left(#1\right)}
\newcommand{\sbr}[1]{\left[#1\right]}
\newcommand{\cbr}[1]{\left\{#1\right\}}

\newcommand{\dprime}{\prime\prime}
\newcommand{\lap}[2]{\mathcal{L}[#1](#2)}

% bracket notation for inner product
\usepackage{mathtools}

\DeclarePairedDelimiterX{\abr}[1]{\langle}{\rangle}{#1}

\DeclareMathOperator{\Span}{span}
\DeclareMathOperator{\nullity}{nullity}
\DeclareMathOperator\Arg{Arg}
\DeclareMathOperator\Log{Log}


% set page count index to begin from 1
\setcounter{page}{1}

\begin{document}
\begin{enumerate}
    \item Let $A$ be an $n\times n$ rotation matrix whose entries are given by $a_{ij}$. Then for some vector $v = (v_1,v_2,\dots,v_n)$, we demand $\abs{v} = \abs{Av}$. This means \begin{align*}
        \abs{v}^2 = a_ia_i &\overset{!}{=} \abs{Av}^2 \\
        &= (Av)_i(Av)_i \\
        &= (a_{ij}v_j)(a_{ik}v_k)\\
        &= (a_{ij}a_{ik})v_jv_k,
    \end{align*} and because $0\leq i,j,k \leq n$, we can match terms when $j=k$ with the left hand side by enforcing $a_{ij}a_{ik} = 1$. All of the terms where $j\neq k$ must be zero. Hence $a_{ij}a_{ik} = \delta_{jk}$.

    \item Observe that the condition $a_{ij}a_{ik} = \delta_{jk}$ is symmetric in $j$ and $k$, so out of all $n^2$ ways to select $j$ and $k$, for the cases when $j\neq k$ (there are $n$ cases where $j=k$), there are $n^2-n-\frac{n^2-n}{2}$ redundant equations (due to commutativity of multiplication, we can pair each equation where $j\neq k$ with another one). Then by adding back on the number of equations where $j=k$, the number of equations which are not redundant is $n^2-\frac{n^2-n}{2}$. Hence there are $n^2-\left(n^2-\frac{n^2-n}{2}\right) = \frac{n(n-1)}{2} = \binom{n}{2}$ degrees of freedom, same as the number of degrees of freedom of rotation in $n$ dimensions.
\end{enumerate}

\end{document}