\documentclass[11pt]{article}
\headheight = 14pt

% packages
\usepackage{physics}
% margin spacing
\usepackage[top=1in, bottom=1in, left=0.5in, right=0.5in]{geometry}
\usepackage{hanging}
\usepackage{amsfonts, amsmath, amssymb, amsthm}
\usepackage{systeme}
\usepackage[none]{hyphenat}
\usepackage{fancyhdr}
\usepackage[nottoc, notlot, notlof]{tocbibind}
\usepackage{graphicx}
\graphicspath{{./images/}}
\usepackage{float}
\usepackage{siunitx}
\usepackage{esint}
\usepackage{cancel}
\usepackage{enumitem}

% colors
\usepackage{xcolor}
\definecolor{p}{HTML}{FFDDDD}
\definecolor{g}{HTML}{D9FFDF}
\definecolor{y}{HTML}{FFFFCF}
\definecolor{b}{HTML}{D9FFFF}
\definecolor{o}{HTML}{FADECB}
%\definecolor{}{HTML}{}

% \highlight[<color>]{<stuff>}
\newcommand{\highlight}[2][p]{\mathchoice%
  {\colorbox{#1}{$\displaystyle#2$}}%
  {\colorbox{#1}{$\textstyle#2$}}%
  {\colorbox{#1}{$\scriptstyle#2$}}%
  {\colorbox{#1}{$\scriptscriptstyle#2$}}}%

% header/footer formatting
\pagestyle{fancy}
\fancyhead{}
\fancyfoot{}
\fancyhead[L]{PHZ3113}
\fancyhead[C]{HW14}
\fancyhead[R]{Sai Sivakumar}
\fancyfoot[R]{\thepage}
\renewcommand{\headrulewidth}{1pt}

% paragraph indentation/spacing
\setlength{\parindent}{0cm}
\setlength{\parskip}{5pt}
\renewcommand{\baselinestretch}{1.25}

% extra commands defined here
\newcommand{\ihat}{\boldsymbol{\hat{\textbf{\i}}}}
\newcommand{\jhat}{\boldsymbol{\hat{\textbf{\j}}}}
\newcommand{\khat}{\boldsymbol{\hat{\textbf{k}}}}
\newcommand{\dr}{\vec{r}~^{\prime}(t)}
\newcommand{\dx}{x^{\prime}(t)}
\newcommand{\dy}{y^{\prime}(t)}

\newcommand{\br}[1]{\left(#1\right)}
\newcommand{\sbr}[1]{\left[#1\right]}
\newcommand{\cbr}[1]{\left\{#1\right\}}

\newcommand{\dprime}{\prime\prime}
\newcommand{\lap}[2]{\mathcal{L}[#1](#2)}

% bracket notation for inner product
\usepackage{mathtools}

\DeclarePairedDelimiterX{\abr}[1]{\langle}{\rangle}{#1}

\DeclareMathOperator{\Span}{span}
\DeclareMathOperator\Arg{Arg}
\DeclareMathOperator\Log{Log}

% set page count index to begin from 1
\setcounter{page}{1}

\begin{document}
Two particles, each of mass $m$, are coupled to each other by a spring of spring constant $k^{\prime}$ and also to walls on both sides by springs of spring constant $k$ each. The kinetic and the potential energies of the system are given by 
\[T = \frac{1}{2}m(\dot{x}_1^2 + \dot{x}_2^2);\quad V = \frac{1}{2}[k(x_1^2+x_2^2) + k^{\prime}(x_1 - x_2)^2].\] This means $\mathcal{L} = \frac{1}{2}m(\dot{x}_1^2 + \dot{x}_2^2) - \frac{1}{2}[k(x_1^2+x_2^2) + k^{\prime}(x_1 - x_2)^2]$.
\begin{enumerate}[label=(\alph*)]
    \item Use the Euler-Lagrange equations to obtain the equations of motion for each mass.
    
    From the Euler-Lagrange equations, we have that \begin{align*}
        \dv{t}\pdv{\mathcal{L}}{\dot{x}_1} = m\ddot{x}_1 &= -kx_1 - k^{\prime}(x_1-x_2) = \pdv{\mathcal{L}}{x_1} \\
        \dv{t}\pdv{\mathcal{L}}{\dot{x}_2} = m\ddot{x}_2 &= -kx_2 - k^{\prime}(x_2-x_1) = \pdv{\mathcal{L}}{x_2}.
    \end{align*}
    \item Find the normal mode frequencies. Check that (i) when $k^{\prime} = 0$, the two frequencies are equal and (ii) when $k = 0$, one of the frequencies is zero.
    
    Assuming the solution is of the form $x_i(t) = x_{i0}\cos(\omega t)$, the system becomes
    \begin{align*}
        \omega^2 x_1 = \frac{1}{m}[(k+k^{\prime})x_1 - k^{\prime} x_2] \\
        \omega^2 x_2 = \frac{1}{m}[(k+k^{\prime})x_2 - k^{\prime} x_1] 
    \end{align*}
    
    In matrix form the system is 
    \[\begin{pmatrix}
        k+k^{\prime} & -k^{\prime} \\
        -k^{\prime} & k+k^{\prime}
    \end{pmatrix}\begin{pmatrix}
        x_1 \\ x_2
    \end{pmatrix} = m\omega^2 \begin{pmatrix}
        x_1 \\ x_2
    \end{pmatrix}.\]
    The eigenvalues are the solutions to the secular equation given by
    \begin{align*}
        \det\begin{pmatrix}
            k+k^{\prime}-\lambda & -k^{\prime} \\
            -k^{\prime} & k+k^{\prime}-\lambda
        \end{pmatrix} = 0 &\iff (k-\lambda)(k+2k^{\prime}-\lambda) = 0 \\
        &\iff \lambda = k, k+2k^{\prime},
    \end{align*} 
    and from the system we can determine the normal mode frequencies from $m\omega^2 = k, k+2k^{\prime}$, so that $\omega = \sqrt{k/m}, \sqrt{(k+2k^{\prime})/m}$. When $k^{\prime} = 0$ indeed the  frequencies are equal, and when $k = 0$, the first frequency given vanishes.
    \item Find the corresponding normalized eigenvectors. Check that they are orthogonal to each other. Discuss the patterns of oscillations for each normal mode.
    
    With the eigenvalues, solve (by inspection)
    \[\begin{pmatrix}
        k^{\prime} & -k^{\prime} \\
        -k^{\prime} & k^{\prime}
    \end{pmatrix}\begin{pmatrix}
        x_1 \\ x_2
    \end{pmatrix} = \vec{0}\quad \text{and}\quad \begin{pmatrix}
        -k^{\prime} & -k^{\prime} \\
        -k^{\prime} & -k^{\prime}
    \end{pmatrix}\begin{pmatrix}
        x_1 \\ x_2
    \end{pmatrix} = \vec{0}\]
    to find the eigenvectors \[\begin{pmatrix}
        1 \\ 1
    \end{pmatrix}\quad \text{and} \quad \begin{pmatrix}
        1 \\ -1
    \end{pmatrix}\]
    which we normalize to find 
    \[\ket{\omega_1} = \frac{1}{\sqrt{2}}\begin{pmatrix}
        1 \\ 1
    \end{pmatrix}\quad \text{and}\quad \ket{\omega_2} = \frac{1}{\sqrt{2}}\begin{pmatrix}
        1 \\ -1
    \end{pmatrix}.\]
    They are indeed orthogonal since the inner product $\frac{1}{2}(1-1) = 0$. These eigenvectors represent special initial conditions where the motion of each mass is uncoupled. In the first case with $\ket{\omega_1}$, this represents the motion where both masses move in the same direction (both masses oscillate left and right in parallel with the same angular frequencies). In the second case with $\ket{\omega_2}$, we have motion where both masses move towards each other and then oscillate away from each other (they move in opposite directions with the same angular frequencies).
\end{enumerate}
\end{document}