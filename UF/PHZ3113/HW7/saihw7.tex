\documentclass[11pt]{article}
\headheight = 14pt

% packages
\usepackage{physics}
% margin spacing
\usepackage[top=1in, bottom=1in, left=0.5in, right=0.5in]{geometry}
\usepackage{hanging}
\usepackage{amsfonts, amsmath, amssymb, amsthm}
\usepackage{systeme}
\usepackage[none]{hyphenat}
\usepackage{fancyhdr}
\usepackage[nottoc, notlot, notlof]{tocbibind}
\usepackage{graphicx}
\graphicspath{{./images/}}
\usepackage{float}
\usepackage{siunitx}
\usepackage{esint}
\usepackage{cancel}
\usepackage{enumitem}

% colors
\usepackage{xcolor}
\definecolor{p}{HTML}{FFDDDD}
\definecolor{g}{HTML}{D9FFDF}
\definecolor{y}{HTML}{FFFFCF}
\definecolor{b}{HTML}{D9FFFF}
\definecolor{o}{HTML}{FADECB}
%\definecolor{}{HTML}{}

% \highlight[<color>]{<stuff>}
\newcommand{\highlight}[2][p]{\mathchoice%
  {\colorbox{#1}{$\displaystyle#2$}}%
  {\colorbox{#1}{$\textstyle#2$}}%
  {\colorbox{#1}{$\scriptstyle#2$}}%
  {\colorbox{#1}{$\scriptscriptstyle#2$}}}%

% header/footer formatting
\pagestyle{fancy}
\fancyhead{}
\fancyfoot{}
\fancyhead[L]{PHZ3113}
\fancyhead[C]{HW7}
\fancyhead[R]{Sai Sivakumar}
\fancyfoot[R]{\thepage}
\renewcommand{\headrulewidth}{1pt}

% paragraph indentation/spacing
\setlength{\parindent}{0cm}
\setlength{\parskip}{5pt}
\renewcommand{\baselinestretch}{1.25}

% extra commands defined here
\newcommand{\ihat}{\boldsymbol{\hat{\textbf{\i}}}}
\newcommand{\jhat}{\boldsymbol{\hat{\textbf{\j}}}}
\newcommand{\khat}{\boldsymbol{\hat{\textbf{k}}}}
\newcommand{\dr}{\vec{r}~^{\prime}(t)}
\newcommand{\dx}{x^{\prime}(t)}
\newcommand{\dy}{y^{\prime}(t)}

\newcommand{\br}[1]{\left(#1\right)}
\newcommand{\sbr}[1]{\left[#1\right]}
\newcommand{\cbr}[1]{\left\{#1\right\}}

\newcommand{\dprime}{\prime\prime}
\newcommand{\lap}[2]{\mathcal{L}[#1](#2)}

% bracket notation for inner product
\usepackage{mathtools}

\DeclarePairedDelimiterX{\abr}[1]{\langle}{\rangle}{#1}

\DeclareMathOperator{\Span}{span}
\DeclareMathOperator\Arg{Arg}
\DeclareMathOperator\Log{Log}


% set page count index to begin from 1
\setcounter{page}{1}

\begin{document}
\begin{enumerate}%[label=(\alph*)]
    \item Prove the following product rules
    \begin{align}
        \vec{\nabla}(\psi\phi) &= \phi\vec{\nabla}\psi + \psi\vec{\nabla}\phi \\
        \vec{\nabla}\cdot (\psi\vec{A}) &= \psi\vec{\nabla}\cdot\vec{A} + \vec{\nabla}\psi \cdot \vec{A} \\
        \vec{\nabla}\times (\psi\vec{A}) &= \psi\vec{\nabla}\times\vec{A} + \vec{\nabla}\psi\times\vec{A}
    \end{align}
    \begin{enumerate}[label=(\arabic*)]
        \item $(\vec{\nabla}(\psi\phi))_i = \partial_i(\psi\phi) = (\partial_i\psi)\phi + \psi(\partial_i\phi) \implies \vec{\nabla}(\psi\phi) = \phi\vec{\nabla}\psi + \psi\vec{\nabla}\phi$
        \item $\vec{\nabla}\cdot (\psi\vec{A}) = \partial_i (\psi\vec{A})_i = \partial_i (\psi A_i) = (\partial_i A_i)\psi + (\partial_i\psi) A_i \implies \vec{\nabla}\cdot (\psi\vec{A}) = \psi\vec{\nabla}\cdot\vec{A} + \vec{\nabla}\psi \cdot \vec{A}$
        \item $(\vec{\nabla}\times (\psi\vec{A}))_i = \epsilon_{ijk}\partial_j(\psi\vec{A})_k = \epsilon_{ijk}\partial_j(\psi A_k) = \epsilon_{ijk}[(\partial_j A_k)\psi + (\partial_j\psi) A_k] = \epsilon_{ijk}(\partial_j A_k)\psi + \epsilon_{ijk}(\partial_j\psi)A_k$ $\implies \vec{\nabla}\times (\psi\vec{A}) = \psi\vec{\nabla}\times\vec{A} + \vec{\nabla}\psi\times\vec{A}$
    \end{enumerate}
    \item Quantum mechanical orbital angular momentum(?) is given by $\vec{L} = \vec{r}\times -i\vec{\nabla}$.\begin{enumerate}[label=(\alph*)]
        \item Show that $\vec{L}$ has components $L_x = -i(y\partial_z - z\partial_y), L_y = -i(z\partial_x-x\partial_z), L_z = -i(x\partial_y - y\partial_x)$.
        
        Keep in mind that the subscript $i$ is different from the complex number $i$. We have that $\vec{r} = (x,y,z)$. Thus $L_i = (\vec{r}\times -i\vec{\nabla})_i = -i\epsilon_{ijk}r_j\partial_k = -i(r_j\partial_k - r_k\partial_j)$. Thus by selecting $i,j,k$ cyclically in $x,y,z$ we can obtain the components as stated above.
        \item Show that these components satisfy the ``commutation relation'' $[L_x,L_y]\equiv  L_xL_y - L_yL_x = iL_z$.
        
        Directly, we have that \begin{multline*}[L_x,L_y]\equiv  L_xL_y - L_yL_x = -i(y\partial_z - z\partial_y)\cdot-i(z\partial_x-x\partial_z) - -i(z\partial_x-x\partial_z)\cdot -i(y\partial_z - z\partial_y) \\  = -(y\partial_zz\partial_x - z\partial_yz\partial_x-y\partial_zx\partial_z + z\partial_yx\partial_z) + (z\partial_xy\partial_z - x\partial_zy\partial_z - z\partial_xz\partial_y + x\partial_zz\partial_y) \\ = -(y\partial_x + yz\partial_z\partial_x-z^2\partial_y\partial_x - yx\partial_z\partial_z + zx\partial_y\partial_z) + (zy\partial_x\partial_z-xy\partial_z\partial_z - z^2\partial_x\partial_y + x\partial_y + xz\partial_z\partial_y)\\ = -y\partial_x+x\partial_y = i(-i(x\partial_y - y\partial_x)) = iL_z,\end{multline*}
        where mixed second partial derivative operators are equivalent when they act on twice continuously differentiable functions.
    \end{enumerate}
\end{enumerate}

\end{document}