\documentclass[11pt]{article}
\headheight = 14pt

% packages
\usepackage{physics}
% margin spacing
\usepackage[top=1in, bottom=1in, left=0.5in, right=0.5in]{geometry}
\usepackage{hanging}
\usepackage{amsfonts, amsmath, amssymb, amsthm}
\usepackage{systeme}
\usepackage[none]{hyphenat}
\usepackage{fancyhdr}
\usepackage[nottoc, notlot, notlof]{tocbibind}
\usepackage{graphicx}
\graphicspath{{./images/}}
\usepackage{float}
\usepackage{siunitx}
\usepackage{esint}
\usepackage{cancel}
\usepackage{enumitem}

% colors
\usepackage{xcolor}
\definecolor{p}{HTML}{FFDDDD}
\definecolor{g}{HTML}{D9FFDF}
\definecolor{y}{HTML}{FFFFCF}
\definecolor{b}{HTML}{D9FFFF}
\definecolor{o}{HTML}{FADECB}
%\definecolor{}{HTML}{}

% \highlight[<color>]{<stuff>}
\newcommand{\highlight}[2][p]{\mathchoice%
  {\colorbox{#1}{$\displaystyle#2$}}%
  {\colorbox{#1}{$\textstyle#2$}}%
  {\colorbox{#1}{$\scriptstyle#2$}}%
  {\colorbox{#1}{$\scriptscriptstyle#2$}}}%

% header/footer formatting
\pagestyle{fancy}
\fancyhead{}
\fancyfoot{}
\fancyhead[L]{PHZ3113}
\fancyhead[C]{HW25}
\fancyhead[R]{Sai Sivakumar}
\fancyfoot[R]{\thepage}
\renewcommand{\headrulewidth}{1pt}

% paragraph indentation/spacing
\setlength{\parindent}{0cm}
\setlength{\parskip}{5pt}
\renewcommand{\baselinestretch}{1.25}

% extra commands defined here
\newcommand{\ihat}{\boldsymbol{\hat{\textbf{\i}}}}
\newcommand{\jhat}{\boldsymbol{\hat{\textbf{\j}}}}
\newcommand{\khat}{\boldsymbol{\hat{\textbf{k}}}}
\newcommand{\dr}{\vec{r}~^{\prime}(t)}
\newcommand{\dx}{x^{\prime}(t)}
\newcommand{\dy}{y^{\prime}(t)}

\newcommand{\br}[1]{\left(#1\right)}
\newcommand{\sbr}[1]{\left[#1\right]}
\newcommand{\cbr}[1]{\left\{#1\right\}}

\newcommand{\dprime}{\prime\prime}
\newcommand{\lap}[2]{\mathcal{L}[#1](#2)}

% bracket notation for inner product
\usepackage{mathtools}

\DeclarePairedDelimiterX{\abr}[1]{\langle}{\rangle}{#1}

\DeclareMathOperator{\Span}{span}
\DeclareMathOperator\Arg{Arg}
\DeclareMathOperator\Log{Log}

% set page count index to begin from 1
\setcounter{page}{1}

\begin{document}
\begin{enumerate}[label=25.\arabic*]
    \item Consider a matrix operator \[\Omega \equiv \begin{pmatrix}
        1 & e^{i\theta} & 0 \\
        e^{-i\theta} & 1 & 0 \\
        0 & 0 & 1
    \end{pmatrix}.\] The operator $\Omega$ satisfies the eigenvalue equation $\Omega\ket{\lambda_i} = \lambda_i\ket{\lambda_i}$, with eigenvalues $\lambda_1 = 0$, $\lambda_2 = 1$, and $\lambda_3 = 2$, and the corresponding set of orthonormal eigenvectors are \[\ket{\lambda_1} = \frac{1}{\sqrt{2}}\begin{pmatrix}
        e^{i\theta/2}\\-e^{-i\theta/2}\\0
    \end{pmatrix}, \quad \ket{\lambda_2} = \begin{pmatrix}
        0\\0\\1
    \end{pmatrix},\ket{\lambda_3} = \frac{1}{\sqrt{2}}\begin{pmatrix}
        e^{i\theta/2}\\e^{-i\theta/2}\\0
    \end{pmatrix},\] such that $\bra{\lambda_i}\ket{\lambda_j} = \delta_{ij}$.\begin{enumerate}[label=(\alph*)]
        \item Given a vector $\ket{V} \equiv \begin{pmatrix}
            e^{i\theta}\\e^{-i\theta}\\1
        \end{pmatrix}$, find the expectation value\[\abr{\Omega} = \bra{V}\Omega\ket{V}.\]

        We have \begin{align*}
            \abr{\Omega} = \bra{V}\Omega\ket{V} &= \begin{pmatrix}
                e^{-\theta} & e^{i\theta} & 1
            \end{pmatrix}\begin{pmatrix}
                1 & e^{i\theta} & 0 \\
                e^{-i\theta} & 1 & 0 \\
                0 & 0 & 1
            \end{pmatrix}\begin{pmatrix}
                e^{i\theta}\\e^{-i\theta}\\1
            \end{pmatrix}\\
            &= \begin{pmatrix}
                e^{-\theta} & e^{i\theta} & 1
            \end{pmatrix}\begin{pmatrix}
                e^{i\theta}+1\\e^{-i\theta}+1\\1
            \end{pmatrix}\\
            &= (e^{-i\theta}+1) + (e^{i\theta}+1) + 1 = e^{i\theta}+e^{-i\theta}+3.
        \end{align*}
        \item Expand the vector $\ket{V}$ defined in (a) in terms of the eigenvectors $\ket{\lambda_i}$; i.e. find the expansion parameters $a_i$ in the expansion $\ket{V} = \sum_i a_i\ket{\lambda_i}$.
        
        Taking inner products, we have \begin{align*}
            \bra{\lambda_1}\ket{V} &= \frac{1}{\sqrt{2}}(e^{i\theta/2} - e^{-i\theta/2})\\
            \bra{\lambda_2}\ket{V} &= 1\\
            \bra{\lambda_3}\ket{V} &= \frac{1}{\sqrt{2}}(e^{i\theta/2} + e^{-i\theta/2})
        \end{align*} so that \[\ket{V} = \frac{1}{\sqrt{2}}(e^{i\theta/2} - e^{-i\theta/2})\ket{\lambda_1} + \ket{\lambda_2} + \frac{1}{\sqrt{2}}(e^{i\theta/2} + e^{-i\theta/2})\ket{\lambda_3}.\]
        \item Using the expansion $\ket{V} = \sum_i a_i\ket{\lambda_i}$, show that the expectation value $\abr{\Omega}$ van also be written as $\bra{V}\Omega\ket{V} = \sum_i \lambda_i\abs{a_i}^2$. Using $a_i$ as obtained in (b), show that your result agrees with (a).
        
        Since $\ket{\lambda_i}$ form an orthonormal eigenbasis, we have \begin{align*}
            \abr{\Omega} = \bra{V}\Omega\ket{V} &= \sum_i a_i\overline{a_i}\bra{\lambda_i}\Omega\ket{\lambda_i}\\
            &= \sum_i \lambda_i\abs{a_i}^2\bra{\lambda_i}\ket{\lambda_i}\\
            &= \sum_i \lambda_i\abs{a_i}^2\\
            &= (0)\cdot\frac{1}{2}(2-(e^{i\theta} + e^{-i\theta})) + (1)\cdot 1 + (2)\cdot\frac{1}{2}(2+e^{i\theta} + e^{-i\theta})\\
            &= e^{i\theta}+e^{-i\theta}+3,
        \end{align*} which matches with the computation in (a).
    \end{enumerate}
\end{enumerate}

\end{document}