\documentclass[11pt]{article}

\usepackage{physics}
\usepackage[top=1in, bottom=1in, left=0.5in, right=0.5in]{geometry}
\usepackage{hanging}
\usepackage{amsfonts, amsmath, amssymb}
\usepackage[none]{hyphenat}
\usepackage{fancyhdr}
\usepackage[nottoc, notlot, notlof]{tocbibind}
\usepackage{graphicx}
\graphicspath{{./images/}}
\usepackage{float}
\usepackage{siunitx}
\usepackage{esint}

\pagestyle{fancy}
\fancyhead{}
\fancyfoot{}
\fancyhead[L]{MAP2302 Professor Jury}
\fancyhead[R]{Sai Sivakumar }
\fancyfoot[R]{\thepage}
\renewcommand{\headrulewidth}{0pt}

\setlength{\parindent}{0cm}
\setlength{\parskip}{5pt}
\renewcommand{\baselinestretch}{1.25}

\newcommand{\ihat}{\boldsymbol{\hat{\textbf{\i}}}}
\newcommand{\jhat}{\boldsymbol{\hat{\textbf{\j}}}}
\newcommand{\dr}{\vec{r}~^{\prime}(t)}
\newcommand{\dx}{x^{\prime}(t)}
\newcommand{\dy}{y^{\prime}(t)}

\newcommand{\br}[1]{\left(#1\right)}
\newcommand{\sbr}[1]{\left[#1\right]}
\newcommand{\cbr}[1]{\{#1\}}

\newcommand{\dprime}{\prime\prime}

\usepackage{mathtools}

\DeclarePairedDelimiterX{\abr}[1]{\langle}{\rangle}{#1}

\setcounter{page}{1}

\begin{document}

The Laplace transform (starred lines are important)
$$\star ~\mathcal{L}[f(t)](s) = \lim_{b\to \infty}\int_0^{b}e^{-st}f(t)\dd{t}$$

Derivation of Laplace transform for $e^{at}$, $\cos(at)$, $\sin(at)$:
$$\star ~\mathcal{L}[e^{at}](s) = \lim_{b\to \infty}\int_0^b e^{-st}e^{at}\dd{t} = \frac{1}{s-a}$$
$$\star ~\mathcal{L}[\cos(at)](s) = \frac{s}{s^2-a^2}$$
$$\star ~\mathcal{L}[\sin(at)](s) = \frac{a}{s^2-a^2}$$

For the exponential:
$$\mathcal{L}[e^{at}](s) = \lim_{b\to \infty}\int_0^b e^{-st}e^{at}\dd{t} = \lim_{b\to \infty}\sbr{ \frac{e^{(a-s)b}}{a-s}-\frac{1}{a-s}} = \frac{1}{s-a}$$

This only happens when $a<s$. A remark to make is that the case when $a = s$ will also fail to converge.
As for the sine and cosine, to find their Laplace transforms we want to observe the following:
$$e^{iat} = \cos(at) + i\sin(at)\implies  \Re\br{e^{iat}} = \cos(at), ~ \Im\br{e^{iat}} = \sin(at)$$

Also know that since integration is a linear operation (it acts termwise on a sum), it is true that
$$\int \Re\br{f(z)}\dd{z} = \Re\br{\int f(z)\dd{z}}$$
$$\mathcal{L}[\Re\br{f(t)}](s) = \lim_{b\to \infty}\int_0^{b}e^{-st}\Re{f(t)}\dd{t} = \Re\br{\lim_{b\to \infty}\int_0^{b}e^{-st}f(t)\dd{t}} = \Re\br{\mathcal{L}[f(t)](s)}$$

We can use this fact to find the following Laplace transforms:
$$\mathcal{L}[\cos(at)](s) = \mathcal{L}[\Re\br{e^{iat}}](s) = \Re\br{ \mathcal{L}[e^{iat}](s)}$$
$$\mathcal{L}[\sin(at)](s) = \mathcal{L}[\Im\br{e^{iat}}](s) = \Im\br{ \mathcal{L}[e^{iat}](s)}$$

We must find the Laplace transform of $e^{iat}$ first (not rigorously but it will do):
$$\mathcal{L}[e^{iat}](s) = \lim_{b\to \infty}\int_{0}^{b}e^{-st} \br{e^{iat}}\dd{t} = \frac{1}{s-ia} = \frac{s+ia}{s^2-a^2}$$
$$\mathcal{L}[\cos(at)](s) = \Re\br{\frac{s+ia}{s^2-a^2}} = \frac{s}{s^2-a^2}$$
$$\mathcal{L}[\sin(at)](s) = \Im\br{\frac{s+ia}{s^2-a^2}} = \frac{a}{s^2-a^2}$$

\end{document}