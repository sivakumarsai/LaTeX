\documentclass[11pt]{article}

% packages
\usepackage{physics}
% margin spacing
\usepackage[top=1in, bottom=1in, left=0.5in, right=0.5in]{geometry}
\usepackage{hanging}
\usepackage{amsfonts, amsmath, amssymb, amsthm}
\usepackage{systeme}
\usepackage[none]{hyphenat}
\usepackage{fancyhdr}
\usepackage[nottoc, notlot, notlof]{tocbibind}
\usepackage{graphicx}
\graphicspath{{./images/}}
\usepackage{float}
\usepackage{siunitx}
\usepackage{esint}
\usepackage{cancel}
\usepackage{enumitem}

% permutations (second line is for spacing)
\usepackage{permute}
\renewcommand*\pmtseparator{\,}

% colors
\usepackage{xcolor}
\definecolor{p}{HTML}{FFDDDD}
\definecolor{g}{HTML}{D9FFDF}
\definecolor{y}{HTML}{FFFFCF}
\definecolor{b}{HTML}{D9FFFF}
\definecolor{o}{HTML}{FADECB}
%\definecolor{}{HTML}{}

% \highlight[<color>]{<stuff>}
\newcommand{\highlight}[2][p]{\mathchoice%
  {\colorbox{#1}{$\displaystyle#2$}}%
  {\colorbox{#1}{$\textstyle#2$}}%
  {\colorbox{#1}{$\scriptstyle#2$}}%
  {\colorbox{#1}{$\scriptscriptstyle#2$}}}%

% header/footer formatting
\pagestyle{fancy}
\fancyhead{}
\fancyfoot{}
\fancyhead[L]{MAS5312}
\fancyhead[C]{Assignment 4}
\fancyhead[R]{Sai Sivakumar}
\fancyfoot[R]{\thepage}
\renewcommand{\headrulewidth}{1pt}

% paragraph indentation/spacing
\setlength{\parindent}{0cm}
\setlength{\parskip}{10pt}
\renewcommand{\baselinestretch}{1.25}

% extra commands defined here
\newcommand{\ihat}{\boldsymbol{\hat{\textbf{\i}}}}
\newcommand{\jhat}{\boldsymbol{\hat{\textbf{\j}}}}
\newcommand{\dr}{\vec{r}~^{\prime}(t)}
\newcommand{\dx}{x^{\prime}(t)}
\newcommand{\dy}{y^{\prime}(t)}

\newcommand{\br}[1]{\left(#1\right)}
\newcommand{\sbr}[1]{\left[#1\right]}
\newcommand{\cbr}[1]{\left\{#1\right\}}

\newcommand{\dprime}{\prime\prime}
\newcommand{\lap}[2]{\mathcal{L}[#1](#2)}

\newcommand{\divides}{\mid}

% bracket notation for inner product
\usepackage{mathtools}

\DeclarePairedDelimiterX{\abr}[1]{\langle}{\rangle}{#1}

\DeclareMathOperator{\Span}{span}
\DeclareMathOperator{\nullity}{nullity}
\DeclareMathOperator\Aut{Aut}
\DeclareMathOperator\Inn{Inn}
\DeclareMathOperator{\Orb}{Orb}
\DeclareMathOperator{\lcm}{lcm}
\DeclareMathOperator{\Hol}{Hol}

% set page count index to begin from 1
\setcounter{page}{1}

\begin{document}
\begin{enumerate}
    \item (DF7.5.2) Let $R$ be an integral domain and let $D$ be a nonempty subset of $R$ that is closed under multiplication. Prove that the ring of fractions $D^{-1}R$ is isomorphic to a subring of the quotient field of $R$ (hence is also an integral domain).
    \begin{proof} Let $R$ be an integral domain and let $D$ be a nonempty subset of $R$ that is closed under multiplication as given.

      We check that the ring of fractions $D^{-1}R$ is a well defined ring, defining elements of this ring in the same way that the quotient field is defined, but with any $D$ closed under multiplication.

      If $D$ contains $0\in R$, then every element in $D^{-1}{R}$ is equal to the zero element, so that $D^{-1}{R}$ is the zero ring: For any fraction $a/b$ and any nonzero $d\in D$, we have \[\frac{a}{b} = \frac{0}{0} = \frac{0}{d},\] since $a0 = b0 = 0d = 0$. Hence in this case we get the zero ring which is automatically a subring of the quotient field of $R$ (also note that the zero ring is not an integral domain; perhaps the formulation of the problem was not meant to include this case). So suppose $D$ does not contain zero (it also will not contain any zero divisors since $R$ is an integral domain)

      For fractions $a/b = a^{\prime}/b^{\prime}$ and $c/d = c^{\prime}/d^{\prime}$, we check that addition and multiplication is well defined. With $ab^{\prime} = a^{\prime}b$ and $cd^{\prime} = c^{\prime}d$, we have \begin{align*}
        \frac{a}{b} + \frac{c}{d} &= \frac{ad + bc}{bd} &\quad \frac{a}{b}\cdot\frac{c}{d} &= \frac{ac}{bd}\\
        \frac{a^{\prime}}{b^{\prime}}  + \frac{c^{\prime}}{d^{\prime}} &= \frac{a^{\prime}d^{\prime} + b^{\prime}c^{\prime}}{b^{\prime}d^{\prime}} &\quad \frac{a^{\prime}}{b^{\prime}}\cdot\frac{c^{\prime}}{d^{\prime}} &= \frac{a^{\prime}c^{\prime}}{b^{\prime}d^{\prime}},
      \end{align*} and we check that \[adb^{\prime}d^{\prime}+bcb^{\prime}d^{\prime} = a^{\prime}d^{\prime}bd + b^{\prime}c^{\prime}bd \quad \text{and}\quad acb^{\prime}d^{\prime} = a^{\prime}c^{\prime}bd.\] Observe that $adb^{\prime}d^{\prime}+bcb^{\prime}d^{\prime} = a^{\prime}dbd^{\prime}+bc^{\prime}b^{\prime}d = a^{\prime}d^{\prime}bd + b^{\prime}c^{\prime}bd$ and that $acb^{\prime}d^{\prime} = a^{\prime}cbd^{\prime} = a^{\prime}c^{\prime}bd$ as desired. Hence the operations of addition and multiplication are well defined.
      
      Note the additive identity is $0/d$ for any $d\in D$ since $a/b + 0/d = ad/bd = a/b$, and the additive inverse of $c/d$ is $-c/d$ since $c/d + -c/d = (cd-cd)/d^2 = 0/d$. We check that $D^{-1}R$ is closed under addition and closed under multiplication, and that these operations are commutative (since $R$ is a commutative ring): \[\frac{a}{b} + \frac{c}{d} = \frac{ad+bc}{bd} = \frac{cb + da}{db} = \frac{c}{d} + \frac{a}{b}, \text{ and } \frac{a}{b}\frac{c}{d} = \frac{ac}{bd}\in D^{-1}{R},\] and since $bd = db\in D$ as $D$ is closed under multiplication, it follows that $D^{-1}{R}$ is closed under its operations. The multiplicative identity of $D^{-1}{R}$ is $d/d$ for any $d\in D$, since $(a/b)(d/d) = ad/bd = a/b = da/db = (d/d)(a/b)$ (since $adb = bda$).
      
      The multiplication and addition is associative because the multiplication in $R$ is associative:
      \[\left(\frac{a}{b} \cdot \frac{c}{d}\right) \cdot \frac{e}{f} = \frac{ac}{bd}\cdot \frac{e}{f} = \frac{ace}{bdf} = \frac{a}{b}\cdot \frac{ce}{df} = \frac{a}{b}\cdot \left(\frac{c}{d}\cdot \frac{e}{f}\right)\]
      \[\left(\frac{a}{b} + \frac{c}{d}\right) + \frac{e}{f} = \frac{ad + bc}{bd} + \frac{e}{f} = \frac{adf + bcf + bde}{bdf} = \frac{a}{b} + \frac{cf + de}{df} = \frac{a}{b} + \left(\frac{c}{d} + \frac{e}{f}\right).\] The multiplication also distributes: \[\frac{a}{b}\left(\frac{c}{d} + \frac{e}{f}\right) = \frac{a}{b}\left(\frac{cf + de}{df}\right) = \frac{acf + ade}{bdf} = \frac{acfb + adeb}{b^2df} = \frac{ac}{bd} + \frac{ae}{bf}\]\[\left(\frac{c}{d} + \frac{e}{f}\right)\frac{a}{b} = \left(\frac{cf + de}{df}\right)\frac{a}{b} = \frac{cfa + dea}{dfb} = \frac{cfab + deab}{dfb^2} = \frac{ca}{db} + \frac{ea}{fb}\]

      It follows that $D^{-1}{R}$ is a commutative ring with identity. 

      To show that $D^{-1}{R}$ is isomorphic to a subring of the quotient field $Q$ of $R$, we use the inclusion homomorphism $\iota \colon D^{-1}{R}\xhookrightarrow{}Q$ which sends $a/b$ to $a/b$, interpreting an element of $D$ to also be an element of $R-\cbr{0}$ since we assumed in this case that $D$ did not contain $0$. This map is clearly injective, so by the first isomorphism theorem, $D^{-1}{R}$ is isomorphic to its image under $\iota$, which is a subgroup of $Q$.
    \end{proof}
    \item (DF7.6.3) Let $R$ and $S$ be rings with identities. Prove that every ideal of $R\times S$ is of the form $I\times J$ where $I$ is an ideal of $R$ and $J$ is an ideal of $S$.
    \begin{proof}
      Let $A$ be an ideal of $R\times S$, and let $I$ be the set of elements $r\in R$ such that for each $r\in I$ there exists an element $s\in S$ such that $(r,s)\in A$. Similarly, let $J$ be the set of elements $s\in S$ such that for each $s\in J$, there exists an element $r\in R$ such that $(r,s)\in A$. Essentially, $I$ and $J$ are the sets which contain the elements which appear in the first and second components of elements of $A$, respectively. We show that these are ideals of $R$ and $S$ each, and show that $A$ is isomorphic to $I\times J$.

      Note since $A$ is an ideal it contains an additive identity, which from the definition of the addition on $R\times S$ it follows that the additive identity is $(0_R,0_S)$, so that $0_R\in I$ and $0_S\in J$. Then let $r,r^{\prime}$ be elements of $I$. Then there exist $s,s^{\prime}\in S$ such that $(r,s),(r^{\prime},s^{\prime})\in A$ and so since $A$ is an ideal, we have for any $(a,b)\in R\times S$ (so $a\in R$ and $b\in S$) that \[(r,s) - (r^{\prime},s^{\prime}) = (r-r^{\prime}, s-s^{\prime})\in A, \text{ and } (r,s)(r^{\prime},s^{\prime}) = (rr^{\prime}, ss^{\prime})\in A\]\[(a,b)(r,s) = (ar,bs)\in A, \text{ and } (r,s)(a,b) = (ra,sb)\in A\] so that $I$ is closed under subtraction and multiplication and is a nonempty subset of $R$, and multiplication by elements of $R$ on the right and left. It follows that $I$ is an ideal of $R$. Similarly, let $s,s^{\prime}$ be elements of $J$, so that there exist $r,r^{\prime}\in R$ such that $(r,s),(r^{\prime},s^{\prime})\in A$. Then for any $(a,b)\in R\times S$ we have \[(r,s) - (r^{\prime},s^{\prime}) = (r-r^{\prime}, s-s^{\prime})\in A, \text{ and } (r,s)(r^{\prime},s^{\prime}) = (rr^{\prime}, ss^{\prime})\in A\]\[(a,b)(r,s) = (ar,bs)\in A, \text{ and } (r,s)(a,b) = (ra,sb)\in A,\] and it follows similarly that $J$ is an ideal of $S$.

      We show that $I\times J$ is an ideal of $R\times S$. It is clear that $I\times J$ is a subring of $R\times S$ since $I,J$ are subrings of $R,S$ respectively. Then for any $(r,s)\in R\times S$ and $(a,b)\in I\times J$ we have that $(r,s)(a,b) = (ra + sb) \in I\times J$ and $(a,b)(r,s) = (ar,bs)\in I\times J$ since $I,J$ are ideals. Hence $I\times J$ is an ideal of $R\times S$. What remains is to show that $I\times J = S$.

      An element of $A$ is of the form $(a,b)$, and automatically $a\in I$ and $b\in J$ since for $a\in R$, we have that $b$ is an element of $S$ such that $(a,b)\in A$, and similarly for $b\in S$, we have that $a$ is an element of $R$ such that $(a,b)\in A$. Hence $(a,b)\in I\times J$ so that $A\subseteq I\times J$. Then any element $(r,s)$ of $I\times J$ can be decomposed into $(r,0_S) + (0_R, s)$, and since there are elements $s^{\prime}\in S$ and $r^{\prime}\in R$ such that $(r,s^{\prime}),(r^{\prime}, s)\in A$. But since $R,S$ have identities, we can write $(r,s)$ as $(1_r,0_S)(r,s^{\prime}) + (0_R,1_S)(r^{\prime}, s)$, and this combination is in $A$ since $A$ is an ideal of $R\times S$. Hence $I\times J\subseteq A$, so that $A = I\times J$.

      Since $A$ was an arbitrary ideal of $R\times S$, it follows that every ideal of $R\times S$ is of the form $I\times J$ where $I$ is an ideal of $R$ and $J$ is an ideal of $S$.
    \end{proof}
    \item (DF8.1.7) Find a generator for the ideal $(85, 1+13i)$ in $\mathbb{Z}[i]$, i.e., a greatest common divisor for $85$ and $1+13i$, by the Euclidean Algorithm. Do the same for the ideal $(47-13i, 53+56i)$.
    
    Generators of these ideals are greatest common divisors of the two numbers which generate the ideal. To find a greatest common divisor, we use the Euclidean algorithm. In each division we will choose the quotient to be any closest (with respect to the standard $\mathbb{C}$ Euclidean metric) element of $\mathbb{Z}[i]$ viewed as an element of $\mathbb{C}$ to the quotient computed in $\mathbb{Q}[i]$ viewed as an element of $\mathbb{C}$. That is, given elements $a,b\in\mathbb{Z}[i]$, we compute the quotient $q = a+bi$ in $\mathbb{Q}[i]$, and round $a,b$ to the nearest integer to obtain a quotient $q^{\prime}$ in $\mathbb{Z}[i]$ (this is the algorithm suggested by the text). Starting with $N(85) = 85^2 > N(1+13i) = 1^2+13^2$ and $N(53 + 56i) = 53^2+56^2 > N(47 -13i) = 47^2+13^2$, we have:
    \begin{align*}
      (85) &= (1-7i)(1+13i) + \boxed{(-7 - 6 i)} &\quad &\begin{cases}
        \frac{85}{1+13i} =\frac{1}{2} + \frac{-13}{2}i \approx 1-7i\\
        (85) - (1-7i)(1+13i) = (-7 - 6 i)
      \end{cases} \\
      (1+13i) &= (-1-i)(-7 - 6 i) + (0) &\quad &\begin{cases}
        \frac{1+13i}{-7 - 6 i} =-1-i \\
        \text{no remainder}
      \end{cases} \\
    \end{align*} and \begin{align*}
      (53 + 56i) &= (1+1i)(47 -13i) + (-7 + 22 i) &\quad &\begin{cases}
        \frac{53 + 56i}{47 -13i} =\frac{43}{58} + \frac{81}{58}i \approx 1+1i\\
        (53 + 56i) - (1+1i)(47 -13i) = (-7 + 22 i)
      \end{cases} \\
      (47 -13i) &= (-1-2i)(-7 + 22 i) + \boxed{(-4-5i)} &\quad &\begin{cases}
        \frac{47 -13i}{-7 + 22 i} =\frac{-15}{13} + \frac{-23}{13}i \approx -1-2i\\
        (47 -13i) - (-1-2i)(-7 + 22 i) = (-4-5i)
      \end{cases} \\
      (-7 + 22 i) &= (-2-3i)(-4-5i) + (0) &\quad &\begin{cases}
        \frac{-7 + 22 i}{-4-5i} = -2-3i\\
        \text{no remainder}.
      \end{cases} \\
    \end{align*}
    A greatest common factor can be multiplied by a unit to obtain another greatest common factor, so multiply the remainders $-7-6i$ and $-4-5i$ by $-1$ to find that $(85, 1+13i) = (7+6i)$ and $(47-13i, 53+56i) = (4+5i)$.
    \item (DF8.1.8) Let $F = \mathbb{Q}[\sqrt{D}]$ be a quadratic field with associated quadratic integer ring $\mathcal{O}$ and field norm $N$ as in section $7.1$. \begin{enumerate}[label=(\alph*)]
        \item Suppose $D$ is $-1, -2, -3, -7,$ or $-11$. Prove that $\mathcal{O}$ is a Euclidean Domain with respect to $N$. [Modify the proof for $\mathbb{Z}[i]$ ($D=-1$) in the text. For $D = -3, -7, -11$ prove that every element of $F$ differs from an element in $\mathcal{O}$ by an element whose norm is at most $(1+\abs{D})^2/(16\abs{D})$, which is less than $1$ for these values of $D$. Plotting the points of $\mathcal{O}$ in $\mathbb{C}$ may be helpful.]
        \begin{proof} Note the field norm on $F$ is a norm when $D < 0$, so we omit taking the absolute value of the field norms henceforth.

        For $D = -1,-2$ ($D\not\equiv 1 \mod 4$):
        Let $A = a+b\sqrt{D}$ and $B = c+d\sqrt{D}$ be any two elements of $\mathbb{Z}[\sqrt{D}]$ with $B\neq 0$. Then in $\mathbb{Q}[\sqrt{D}]$, the quotient $A/B$ is given by $r+s\sqrt{D}$ for some rational numbers $r,s$; in particular $r = (ac-bdD)/(c^2-d^2D)$ and $s = (bc-ad)/(c^2-d^2D)$. Then round $r$ to the nearest integer $p$ and round $s$ to the nearest integer $q$ so that $\abs{r-p},\abs{s-q}$ are at most $1/2$.
        
        We take the quotient $A/B$ in $\mathbb{Z}[\sqrt{D}]$ to be $p+q\sqrt{D}$. The remainder we choose is given by $A- (p+q\sqrt{D})B = \theta B$ (and $\theta B$ is computed in $\mathbb{Q}[\sqrt{D}]$), where $\theta = (r-p) + (s-q)\sqrt{D}$. This ensures that the remainder $\theta B$ is also an element of $\mathbb{Z}[\sqrt{D}]$, and since $D$ is equal to $-1$ or $-2$, it follows that \[N(\theta B) = N(\theta)N(B) = [(r-p)^2 - (s-q)^2D]N(B) \leq [1/4 + 2/4]N(B) = (3/4)N(B).\] Hence there is a division algorithm for $\mathbb{Z}[\sqrt{D}]$, meaning it is a Euclidean Domain.

        For $D = -3, -7, -11$ ($D\equiv 1 \mod 4$): Let $A = a+b[(1+\sqrt{D})/2]$ and $B = c+d[(1+\sqrt{D})/2]$ be any two elements of $\mathbb{Z}[(1+\sqrt{D})/2]$ with $B\neq 0$. Then in $\mathbb{Q}[\sqrt{D}]$, the quotient $A/B$ is given by $r+s\sqrt{D}$ for some rational numbers $r,s$; in particular $r = [(a+b/2)(c+d/2)-(b/2)(d/2)D]/[(c+d/2)^2-(d/2)^2D]$ and $s = [(b/2)(c+d/2)-(a+b/2)(d/2)]/[(c+d/2)^2-(d/2)^2D]$. We first rewrite $r+s\sqrt{D}$ into its ``cartesian'' form (we are changing coordinates in $\mathbb{C}$). There exist rational numbers $n,m$ such that $r+s\sqrt{D} = (n+m/2) + (m/2)\sqrt{D}$; that is, $m = 2s$ and $n = r-s$.
        
        We round in a particular order: First round $m = 2s$ to the nearest integer $q$, so that $\abs{2s-q}\leq 1/2$. Then round the rational number $r-q/2$ to the nearest integer $p$ so that $\abs{r-p-q/2}\leq 1/2$. The motivation for this rounding comes from finding the closest element of $\mathbb{Z}[(1+\sqrt{D})/2]$ to any element $u$ of $\mathbb{Q}[\sqrt{D}]$ by geometric means. Embedding $\mathbb{Z}[(1+\sqrt{D})/2]$ in $\mathbb{C}$ yields a lattice of points arranged in a manner such that $\mathbb{C}$ is tiled by parallelograms whose vertices are elements of $\mathbb{Z}[(1+\sqrt{D})/2]$. The parallelograms give a change of basis for $\mathbb{C}$ which is essentially given by basis vectors parallel to the sides of the parallelograms tiling $\mathbb{C}$. Slide $u$ ``vertically'' along a line parallel to the slanted edge of the parallelograms until it hits the closest horizontal side edge of the parallelogram $u$ was found in. This represents the first rounding to obtain $q$. Then slide the point horizontally to the closest element of $\mathbb{Z}[(1+\sqrt{D})/2]$ lying on the same line as it; this corresponds to the rounding used to obtain $p$. Graphically this might look like:
        \vspace*{5cm}

        Then take the quotient to be $p+q[(1+\sqrt{D})/2]$, and the remainder to be $A - (p+q[(1+\sqrt{D})/2])B = \theta B$ (note that it follows that this remainder is in $\mathbb{Z}[(1+\sqrt{D})/2]$), where $\theta = (r-p-q/2) + (s-q/2)\sqrt{D}$. Then we have \begin{align*}
          N(\theta B) = N(\theta)N(B) &= [(r-p-q/2)^2 - (s-q/2)^2D]N(B) \\
          &= [(r-p-q/2)^2 - (2s-q)^2D/4]N(B)\\ 
          &\leq [1/4 - D/16]N(B)\\ 
          &= [(4-D)/16]N(B) \\ 
          &\leq (15/16)N(B) < N(B),
        \end{align*} and the final inequalities follow since $D$ takes on the values $-3,-7,-11$. Thus we have chosen an appropriate quotient and remainder such that $\mathbb{Z}[(1+\sqrt{D})/2]$ has a division algorithm; that is, it is a Euclidean Domain.

        Thus for $D = -1,-2,-3,-7,$ or $-11$, the associated quadratic integer ring $\mathcal{O}$ of the quadratic field $F = \mathbb{Q}[\sqrt{D}]$ is a Euclidean Domain with respect to the field norm $N$.
        \end{proof}
        \item Suppose that $D = -43, -67,$ or $-163$. Prove that $\mathcal{O}$ is not a Euclidean Domain with respect to any norm. [Apply the same proof as for $D = -19$ in the text.]
        \begin{proof} Let $\omega = [(1+\sqrt{D})]/2$.
          Previously in the text it was determined that the only units of $\mathcal{O}$ for $D < -3$ were $\pm 1$ (since if the field norm of $a+b\omega$ was $1$, then we sought to choose integers $a,b$ satisfying $(2a+b)^2 + \abs{D}b^2 = 4$; it follows that $b=0$, so that $a= \pm 1$, so that the units are $\pm 1$.) Hence $\widetilde{\mathcal{O}} = \cbr{0,\pm 1}$.

          Suppose that $u$ is a universal side divisor in $\mathcal{O}$. Observe that for any element $a+b\omega$, its field norm $a^2+ab+[(1-D)/4]b^2 = (a+b/2)^2 + (\abs{D}/4)b^2 \geq (1-D)/4$ whenever $b\neq 0$. So the smallest values for the field norm on $\mathcal{O}$ are attained whenever $b = 0$. They are, in cases:

          $D = -43$, so $(1-D)/4 = 11$: Smallest norms are $1, 4, 9$.
          
          $D = -67$, so $(1-D)/4 = 17$: Smallest norms are $1,4,9,16$.

          $D = -163$, so $(1-D)/4 = 41$: Smallest norms are $1,4,9,16, 25, 36$.

          For the first case when $D = -43$, let $x$ take on $2,3$ ($x$ need not take on $1$ since $1$ is a unit). It follows that $u$ should divide one of $2$ or $3$, and that $u$ should divide one of $3$ or $4$ (the value $3-1 = 2$ will be handled in the first computation anyways). But observe that if $2 = ab$, then $N(2) = 4 = N(a)N(b)$ so that the only non-unit divisors of $2$ are $\cbr{\pm 2}$. Similarly, if $3 = ab$, since $3$ is not a possible norm value, we have that $N(3) = 9 = N(a)N(b)$. Hence the only non-unit divisors of $3$ are $\cbr{\pm 3}$. Again in a similar fashion, observing that $2$ and $8$ are not possible values the norm takes on, the only non-unit factors of $4$ are $\cbr{\pm 2, \pm 4}$. So $u$ can only take on values in $\cbr{\pm 2, \pm 3, \pm 4}$. But observe that if $x = \omega = (1+\sqrt{-43})/2$, none of the possible values for $u$ divide $\omega, \omega \pm 1$ since the quotient computed in the quadratic field would not contain integer coefficents anyways.

          For the second case when $D = -67$, repeat the above argument for $x = 2,3$, and additionally for $x = 4$. So $u$ has to either divide $4$ or $5$; if $5 = ab$ then $N(5) = 25 = N(a)N(b)$ yields that only $\pm 5$ are the only non-unit divisors of $5$. Still we find that any values in $\cbr{\pm 2. \pm 3, \pm 4, \pm 5}$ do not divide $\omega$ in this case.
          
          For the last case when $D = -163$ an additional two more values of $x$ must be considered: $x = 5,6$ so that we must find non-unit factors of $6,7$: if $6 = ab$ then $36 = N(a)N(b)$, and note that $2,3,12,18$ are not norms so that the only additional values that $u$ may take on is $\pm 6$. Similarly, $49$ is a square of a prime so that the only additional values that $u$ may take on is $\pm 7$. And once again with $x = \omega$, none of the values in $\cbr{\pm n \mid 2\leq n \leq 7}$ divide $\omega$.

          In all cases, any of the restrictions to the values that any universal divisor could take on were all nullified by the incapability of dividing $\omega$, so there are no universal side divisors in each case. Hence $\omega$ is not a Euclidean Domain.
        \end{proof}
    \end{enumerate}
\end{enumerate}
\end{document}