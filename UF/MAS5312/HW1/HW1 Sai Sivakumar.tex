\documentclass[11pt]{article}

% packages
\usepackage{physics}
% margin spacing
\usepackage[top=1in, bottom=1in, left=0.5in, right=0.5in]{geometry}
\usepackage{hanging}
\usepackage{amsfonts, amsmath, amssymb, amsthm}
\usepackage{systeme}
\usepackage[none]{hyphenat}
\usepackage{fancyhdr}
\usepackage[nottoc, notlot, notlof]{tocbibind}
\usepackage{graphicx}
\graphicspath{{./images/}}
\usepackage{float}
\usepackage{siunitx}
\usepackage{esint}
\usepackage{cancel}
\usepackage{enumitem}

% permutations (second line is for spacing)
\usepackage{permute}
\renewcommand*\pmtseparator{\,}

% colors
\usepackage{xcolor}
\definecolor{p}{HTML}{FFDDDD}
\definecolor{g}{HTML}{D9FFDF}
\definecolor{y}{HTML}{FFFFCF}
\definecolor{b}{HTML}{D9FFFF}
\definecolor{o}{HTML}{FADECB}
%\definecolor{}{HTML}{}

% \highlight[<color>]{<stuff>}
\newcommand{\highlight}[2][p]{\mathchoice%
  {\colorbox{#1}{$\displaystyle#2$}}%
  {\colorbox{#1}{$\textstyle#2$}}%
  {\colorbox{#1}{$\scriptstyle#2$}}%
  {\colorbox{#1}{$\scriptscriptstyle#2$}}}%

% header/footer formatting
\pagestyle{fancy}
\fancyhead{}
\fancyfoot{}
\fancyhead[L]{MAS5312}
\fancyhead[C]{Assignment 1}
\fancyhead[R]{Sai Sivakumar}
\fancyfoot[R]{\thepage}
\renewcommand{\headrulewidth}{1pt}

% paragraph indentation/spacing
\setlength{\parindent}{0cm}
\setlength{\parskip}{10pt}
\renewcommand{\baselinestretch}{1.25}

% extra commands defined here
\newcommand{\ihat}{\boldsymbol{\hat{\textbf{\i}}}}
\newcommand{\jhat}{\boldsymbol{\hat{\textbf{\j}}}}
\newcommand{\dr}{\vec{r}~^{\prime}(t)}
\newcommand{\dx}{x^{\prime}(t)}
\newcommand{\dy}{y^{\prime}(t)}

\newcommand{\br}[1]{\left(#1\right)}
\newcommand{\sbr}[1]{\left[#1\right]}
\newcommand{\cbr}[1]{\left\{#1\right\}}

\newcommand{\dprime}{\prime\prime}
\newcommand{\lap}[2]{\mathcal{L}[#1](#2)}

\newcommand{\divides}{\mid}

% bracket notation for inner product
\usepackage{mathtools}

\DeclarePairedDelimiterX{\abr}[1]{\langle}{\rangle}{#1}

\DeclareMathOperator{\Span}{span}
\DeclareMathOperator{\nullity}{nullity}
\DeclareMathOperator\Aut{Aut}
\DeclareMathOperator\Inn{Inn}
\DeclareMathOperator{\Orb}{Orb}
\DeclareMathOperator{\lcm}{lcm}
\DeclareMathOperator{\Hol}{Hol}

% set page count index to begin from 1
\setcounter{page}{1}

\begin{document}
\begin{enumerate}
    \item (DF7.1.13) An element $x$ in $R$ is called \textit{nilpotent} if $x^m = 0 $ for some $m\in \mathbb{Z}^+$. \begin{enumerate}[label=\textbf{(\alph*)}]
        \item Show that if $n = a^kb$ for some integers $a$ and $b$ then $\overline{ab}$ is a nilpotent element of $\mathbb{Z}/n\mathbb{Z}$.
        \begin{proof}
            Suppose that $n = a^kb$ for some integers $a$ and $b$. Then in the commutative ring $\mathbb{Z}/n\mathbb{Z} = \mathbb{Z}/a^kb\mathbb{Z}$, the element $\overline{ab}$ is nilpotent if there exists a positive integer $m$ such that $\overline{ab}^m = \overline{a^mb^m} = \overline{0}$, which is equivalent to requiring that $a^mb^m\equiv 0 \pmod{a^kb}$. Then we should have that $a^kb\mid a^mb^m$, and of course we can choose $m\geq k$ so that $a^kb\mid a^mb^m$. Since a suitable $m$ does exist such that $(\overline{ab})^m = \overline{0}$, $\overline{ab}$ is nilpotent in $\mathbb{Z}/n\mathbb{Z}$.
        \end{proof}
        \item If $a\in\mathbb{Z}$ is an integer, show that the element $\overline{a}\in\mathbb{Z}/n\mathbb{Z}$ is nilpotent if and only if every prime divisor of $n$ is also a divisor of $a$. In particular, determine the nilpotent elements of $\mathbb{Z}/72\mathbb{Z}$ explicitly.
        \begin{proof}
            Let $a,n$ be integers, and let $n = p_1^{\alpha_1}p_2^{\alpha_2}\cdots p_s^{\alpha_s}$ be the prime factorization of $n$ for primes $p_i$.

            Suppose that every prime divisor of $n$ is also a divisor of $a$. Observe that $p_1p_2\cdots p_s\mid a$, and let $\alpha = \max\cbr{\alpha_i\mid 1\leq i\leq s}$. Then $ p_1^{\alpha}p_2^{\alpha}\cdots p_s^{\alpha}\mid a^\alpha$, but due to our choice of $\alpha$, $n\mid p_1^{\alpha}p_2^{\alpha}\cdots p_s^{\alpha}$. It follows that $n\mid a^\alpha$, so that $\overline{a^\alpha} = \overline{0}$, meaning that $\overline{a}$ is nilpotent in $\mathbb{Z}/n\mathbb{Z}$.

            Conversely, suppose that $\overline{a}$ is nilpotent in $\mathbb{Z}/n\mathbb{Z}$; that is, there exists a positive integer $\alpha$ such that $\overline{a^\alpha} = \overline{0}$ so that $n\mid a^\alpha$. Since $a\in\mathbb{Z}$ and $p_i\mid n$, we must have that $p_i\mid a$, for $1\leq i \leq s$. (If $p_i\nmid a$, then we arrive at a contradiction with the fact that $n\mid a^\alpha$ by taking $\alpha = \max\cbr{\alpha_i\mid 1\leq i\leq s}$ and observing that $n\mid p_1^{\alpha}p_2^{\alpha}\cdots p_s^{\alpha}$ but $p_1^{\alpha}p_2^{\alpha}\cdots p_s^{\alpha} \nmid a^\alpha$.) 

            Hence $\overline{a}\in\mathbb{Z}/n\mathbb{Z}$ is nilpotent if and only if every prime divisor of $n$ is also a divisor of $a$.
        \end{proof}

        In $\mathbb{Z}/72\mathbb{Z} = \mathbb{Z}/2^33^2\mathbb{Z}$ it follows that every nilpotent element is of the form $\overline{2^i3^ja}$ for positive integers $i,j, a$, since $2$ and $3$ divide $2^i3^j$. Explicitly, these are the elements whose integer representative is even and divisible by three; i.e., a multiple of 6: 
        \[
            \overline{0}, \overline{6}, \overline{12}, \overline{18}, \overline{24}, \overline{30}, \overline{36}, \overline{42}, \overline{48}, \overline{54}, \overline{60}, \overline{66}
        \]
        \item Let $R$ be the ring of functions from a nonempty set $X$ to a field $F$. Prove that $R$ contains no nonzero nilpotent elements.
        \begin{proof}
            Let $R$ be the ring of functions from a nonempty set $X$ to a field $F$ as given. Suppose by way of contradiction that $R$ contains a nonzero nilpotent element $g$. 
            
            Because $g$ is a nilpotent element of $R$, there exists a positive integer $k$ such that $g^k$ is the zero function $0_R\colon X\to F$ with $0_R(x) = 0_F$ for all $x\in X$.

            We have that $g$ is not the zero function $0_R$, so that there exists $y\in X$ such that $g(y)\neq 0_F$. Then $g^k(y) = [g(y)]^k = 0_F$. But $g(y)\neq 0_F$ so that $F$ contains a nonzero zero divisor, which is a contradiction.

            Hence $R$ does not contain a nonzero nilpotent element $g$.
        \end{proof}
    \end{enumerate}
    \item (DF7.1.21) Let $X$ be any nonempty set and let $\mathcal{P}(X)$ be the set of all subsets of $X$ (the \textit{power set} of $X$). Define addition and multiplication on $\mathcal{P}(X)$ by \[A+B = (A-B)\cup (B-A)\quad \text{and} \quad A\times B = A\cap B\] i.e., addition is symmetric difference and multiplication is intersection. \begin{enumerate}[label=\textbf{(\alph*)}]
        \item Prove that $\mathcal{P}(X)$ is a ring under these operations ($\mathcal{P}(X)$ and its subrings are often referred to as \textit{rings of sets}).
        \begin{proof}
            Let $X$ be a nonempty set and let $\mathcal{P}(X)$ be the power set of $X$ as given with the operations of addition and multiplication as given above. Observe that the symmetric difference and intersection of subsets of $X$ return subsets of $X$, so they are valid choices of binary operations.

            Under the addition (symmetric difference) operation, $\mathcal{P}(X)$ is an abelian group. The additive identity is the empty set $\emptyset$: For any subset $A$ of $X$, \[\emptyset + A = (\emptyset - A)\cup (A - \emptyset) = \emptyset \cup A = A = A \cup \emptyset = (A - \emptyset)\cup (\emptyset - A) = A + \emptyset.\] Addition is also associative: For any subsets $A,B,C$ of $X$ we have by lots of set algebra (writing $S^c$ to mean the complement of $S$ in $X$) that \begin{align*}
                A+ (B+C) &= A + ((B- C)\cup (C-B))\\
                &= [A - ((B- C)\cup (C-B))] \cup [((B- C)\cup (C-B)) - A] \\ &= [(A\cap B^c \cap C^c) \cup (A\cap B\cap C)] \cup [(A^c\cap B\cap C^c) \cup (A^c\cap B^c\cap C)]\\
                &= [(A^c\cap B\cap C^c)\cup (A\cap B^c \cap C^c)]\cup [(A^c\cap B^c\cap C)\cup (A\cap B\cap C)] \\
                &= [((A-B)\cup (B-A)) - C] \cup [C - ((A-B)\cup (B-A))]\\
                &= ((A-B)\cup (B-A)) + C\\
                &= (A+B)+C.
            \end{align*} 
            Each subset of $X$ is its own additive inverse: For any subset $A$ of $X$, \[A + A = (A - A) \cup (A - A) = \emptyset.\] 
            Addition is also commutative: For any subsets $A,B$ of $X$, \[A + B = (A-B)\cup (B-A) = (B-A)\cup(A-B)  = B+ A.\]
            With the power set of $X$ being an abelian group under addition, the remaining ring axioms are checked for the multiplication given by intersection. Associativity of multiplication is immediate since set intersection is already associative; that is, for any subsets $A,B,C$ of $X$, we have $(A\times B)\times C = (A\cap B)\cap C = A\cap(B\cap C) = A\times(B\times C)$.

            The distributive laws hold: For any subsets $A,B,C$ of $X$, we have \begin{align*}
                (A+B)\times C &= [(A-B)\cup (B-A)]\times C\\
                &= (A\cap B^c \cap C)\cup (B\cap A^c\cap C)\\
                &= [(A\cap C \cap B^c)\cup (A\cap C \cap C^c)]\cup [(B\cap C\cap A^c)\cup (B\cap C\cap C^c)]\\
                &= [(A\cap C)\cap (B\cap C)^c] \cup [(B\cap C)\cap (A\cap C)^c]\\
                &= (A\cap C - B\cap C)\cup (B\cap C- A\cap C)\\
                &= (A\times C) + (B\times C)
            \end{align*} and \begin{align*}
                A\times (B + C) &= A\times [(B-C)\cup (C-B)]\\
                &= (A \cap B\cap C^c) \cup (A\cap C \cap B^c)\\
                &= [(A \cap B\cap C^c) \cup (A\cap B\cap A^c)]\cup [(A\cap C \cap B^c)\cup (A\cap C\cap A^c)]\\
                &= [(A\cap B)\cap(A\cap C)^c]\cup [(A\cap C)\cap(A\cap B)^c]\\
                &= (A\cap B - A\cap C) \cup (A\cap C - A\cap B)\\
                &= (A\times B)+ (A\times C).
            \end{align*}
            Hence $\mathcal{P}(X)$ is a ring under the operations of addition and multiplication given above.
        \end{proof}
        \item Prove that this ring is commutative, has an identity and is a Boolean ring.
        \begin{proof}
            The ring $\mathcal{P}(X)$ is commutative because set intersection is commutative; that is, $A\times B = A\cap B = B\cap A = B\times A$ for any subsets $A,B$ of $X$.

            The multiplicative identity in this ring is the subset $X$, since for any subset $A$ of $X$, we have $A\times X = A\cap X = A = X\cap A = X\times A$.

            Then for any subset $A$ of $X$, we have $A^2 = A\times A = A\cap A = A$, from which it follows that $\mathcal{P}(X)$ is a Boolean ring.
        \end{proof}
    \end{enumerate}
    \item (DF7.1.23) Let $D$ be a squarefree integer, and let $\mathcal{O}$ be the ring of integers in the quadratic field $\mathbb{Q}(\sqrt{D})$. For any positive integer $f$ prove that the set $\mathcal{O}_f = \mathbb{Z}[f\omega] = \cbr{a+bf\omega\mid a, b\in\mathbb{Z}}$ is a subring of $\mathcal{O}$ containing the identity. Prove that $[\mathcal{O}\colon\mathcal{O}_f] = f$ (index as additive abelian groups). Prove conversely that a subring of $\mathcal{O}$ containing the identity and having finite index $f$ in $\mathcal{O}$ (as additive abelian groups) is equal to $\mathcal{O}_f$. (The ring $\mathcal{O}_f$ is called the \textit{order of conductor} $f$ in the field $\mathbb{Q}(\sqrt{D})$. The ring of integers $\mathcal{O}$ is called the \textit{maximal order} in $\mathbb{Q}(\sqrt{D})$.)
    \begin{proof}
        Let $\mathcal{O}_f$ be as given. It is clear that $\mathcal{O}_f$ is a nonempty subset of $\mathcal{O}$ because $f$ is an integer. We check that this subset is closed under subtraction and multiplication:

        For integers $a,b,c,d$ we have $(a+bf\omega) - (c+df\omega) = (a-c) + (b-d)f\omega$, which is clearly an element of $\mathcal{O}_f$. Similarly, $(a+bf\omega)(c+df\omega) = ac + (ad+bc)f\omega + bdf^2\omega^2$, where \[\omega^2 = \begin{cases}
            D & \text{if $D\not\equiv 1 \pmod 4$}\\
            \br{\frac{D-1}{4}} + \omega & \text{if $D\equiv 1 \pmod 4$},
        \end{cases}\] and in either case we have an element of $\mathcal{O}_f$ (when $D\equiv 1 \pmod 4$ the product is the element $(ac + bdf^2(D-1)/4) + (ad+bc+bdf)f\omega $).

        Hence $\mathcal{O}_f$ is a subring of $\mathcal{O}$, and it also inherits the same $1 = 1 + 0f\omega$ from $\mathcal{O}$ which behaves the same way: $1(a+bf\omega) = (a+bf\omega)1 = a+bf\omega$.

        To show that the index of $\mathcal{O}_f$ in $\mathcal{O}$ is $f$, we use the group homomorphism given by projection onto $\mathbb{Z}/f\mathbb{Z}$ which maps $a+bf\omega$ to $\overline{b}$. Since $b$ was an arbitary integer, this map is surjective. We check that this map is a group homomorphism: For integers $a,b,c,d$ we have \[(a+bf\omega) + (c+df\omega) = (a+c) + (b+d)f\omega \mapsto \overline{(b+d)} = \overline{b} + \overline{d},\] where in the last equality the addition inside of the parenthesis is the addition in $\mathbb{Z}$ and the addition on the right hand side is the addition in $\mathbb{Z}/f\mathbb{Z}$. The kernel of this homomorphism is $\mathcal{O}_f$ (it is clear that elements of $\mathcal{O}_f$ are mapped to $\overline{0}$): For integers $a,b$, asserting the element $a+b\omega$ is in the kernel is equivalent to saying that $f$ divides $b$, meaning $a+b\omega = a+b^{\prime}f\omega$ where $b^{\prime}$ is the quotient $b/f$. Since $b$ was arbitrary, $b^{\prime}$ is also arbitrary.

        Thus by the first isomorphism theorem for groups we have \[[\mathcal{O}\colon \mathcal{O}_f] = \abs{\mathbb{Z}/f\mathbb{Z}} = f.\]

        Suppose that we are given a subring $\mathcal{O}^{\prime}$ of $\mathcal{O}$ of index $f$, containing $1$. Since $\mathcal{O}^{\prime}$ is closed under addition, $\mathbb{Z}$ is contained in $\mathcal{O}^{\prime}$, so that for any integer $a$, the quantity $(f-1)a$ is an element of $\mathcal{O}^{\prime}$. 

        We have that the quotient group $\mathcal{O}/\mathcal{O}^{\prime}$ has order $f$, so that for any element $a+b\omega$ of $\mathcal{O}$, the coset $f(a+b\omega) + \mathcal{O}^{\prime}$ (where $f(a+b\omega)$ denotes the $f$-fold sum given by $af + bf\omega$) is equivalent to the coset $\mathcal{O}^{\prime}$. Furthermore, since $(f-1)a\in\mathcal{O}^{\prime}$, it follows that $(af + bf\omega) +\mathcal{O}^{\prime} = ((f-1)a + 0f\omega) + \mathcal{O}^{\prime}$, which is equivalent to saying that \[(af + bf\omega) - ((f-1)a + 0f\omega) = a+bf\omega \in \mathcal{O}^{\prime n}.\] Since $a,b$ were arbitrary, it follows that $\mathcal{O}_f$ is contained in $\mathcal{O}^{\prime}$. Then \[f = [\mathcal{O}\colon\mathcal{O}^{\prime}] = [\mathcal{O}\colon\mathcal{O}_f][\mathcal{O}_f\colon \mathcal{O}^{\prime}] = f [\mathcal{O}_f\colon \mathcal{O}^{\prime}],\] so that $[\mathcal{O}_f\colon \mathcal{O}^{\prime}] = 1$ and so $\mathcal{O}^{\prime} = \mathcal{O}_f$. Hence any subring of $\mathcal{O}$ containing the identity and having finite index $f$ in $\mathcal{O}$ is equal to $\mathcal{O}_f$.
    \end{proof}
    \item (DF7.1.25) Let $I$ be the ring of integral Hamilton Quaternions and define \[N\colon I\to \mathbb{Z}\quad \text{by} \quad N(a+bi+cj+dk) = a^2 + b^2 + c^2 + d^2\](the map $N$ is called a \textit{norm}).\begin{enumerate}[label=\textbf{(\alph*)}]
        \item Prove that $N(\alpha) = \alpha\overline{\alpha}$ for all $\alpha\in I$, where if $\alpha = a+bi+cj+dk$ then $\overline{\alpha} = a-bi-cj-dk$.
        \item Prove that $N(\alpha\beta) = N(\alpha)N(\beta)$ for all $\alpha, \beta \in I$.
        \item Prove that an element of $I$ is a unit if and only if it has norm $+1$. Show that $I^{\times}$ is isomorphic to the quaternion group of order $8$. [The inverse in the ring of rational quaternions of a nonzero element $\alpha$ is $\overline{\alpha}/N(\alpha)$.]
    \end{enumerate}
    \begin{proof}
        Let $I$ be the ring of integral Hamilton Quaternions as given and define $N\colon I\to \mathbb{Z}$ as above. Then for any integral quaternion $\alpha = a+bi+cj+dk$, we have \begin{align*}
            \alpha\overline{\alpha} = (a+bi+cj+dk)(a-bi-cj-dk) &= (a^2+b^2+c^2+d^2)\\ &\hspace*{1cm}+ (-ab+ab-cd+cd)i\\ &\hspace*{1cm}+(-ac+ac-bd+bd)j \\ &\hspace*{1cm}+(-ad+ad-bc+bc)k\\
            &= a^2+b^2+c^2+d^2\\
            &= N(\alpha).
        \end{align*}
        The function $N$ is also multiplicative. Given any two integral quaternions $\alpha = a+bi+cj+dk$ and $\beta = e + fi + gj + hk$, we have $\alpha\beta = (ae-bf-cg-dh) + (af+be+ch-dg)i + (ag+ce+df-bh)j+ (ah+de+bg-cf)k$. Then (with many cancellations between the first equality and the second equality), we have\begin{align*}
            N(\alpha\beta) &= (ae-bf-cg-dh)^2\\ &\hspace*{1cm}+ (af+be+ch-dg)^2 \\ &\hspace*{1cm}+ (ag+ce+df-bh)^2 \\ &\hspace*{1cm}+ (ah+de+bg-cf)^2\\
            &= a^2e^2+b^2f^2+c^2g^2+d^2h^2 \\ &\hspace*{1cm}+ a^2f^2+b^2e^2+c^2h^2+d^2g^2 \\ &\hspace*{1cm}+a^2g^2+c^2e^2+d^2f^2+b^2h^2 \\ &\hspace*{1cm}+ a^2h^2+d^2e^2+b^2g^2+c^2f^2\\
            &= (a^2+b^2+c^2+d^2)(e^2+f^2+g^2+f^2)\\
            &= N(\alpha)N(\beta).
        \end{align*}

        An integral quaternion is a unit if and only if its norm is $1$. If a quaternion $\alpha$ is a unit then it has an inverse $\alpha^{-1}$ with $\alpha\alpha^{-1} = \alpha^{-1}\alpha = 1$. Then $N(1) = N(\alpha\alpha^{-1}) = N(\alpha)N(\alpha^{-1})$, so that $N(\alpha) = N\alpha^{-1} = \pm 1$ (they both need to have the same sign). But $N(\alpha)$ and $N(\alpha^{-1})$ are sums of squared integers, meaning that their values are necessarily positive. This forces $N(\alpha) = N(\alpha^{-1}) = 1$, which means that units have a norm of $1$.

        If an integral quaternion $\alpha = a+bi+cj+dk$ has a norm of $1$, we have that $a^2+b^2+c^2+d^2 = 1$. In the integers, there are only eight solutions, which we write as tuples $(a,b,c,d)$: \[(\pm1, 0,0,0), (0,\pm1, 0,0), (0,0,\pm1, 0), (0,0,0, \pm1).\] Any other combination of integers will not satisfy the equality $a^2+b^2+c^2+d^2 = 1$. The corresponding integral quaternions (suppressing the zero-coefficient terms in each one) which have a norm of $1$ are \[\pm 1, \pm i, \pm j, \pm k,\] and these are units (observe that we multiply by the conjugate to obtain a multiplicative inverse): \[1 = (1)(1) = i(-i) = (-i)i = j(-j) = (-j)j = k(-k) = (-k)k.\]

        Hence the units of the integral quaternions are those with norm $1$.
        
        It follows that these eight units form a group under multiplication, $I^{\times}$, which is isomorphic to $Q_8$. The isomorphism needed is just the relabeling \begin{align*}
            \pm 1 + 0i+0j+0k &\mapsto \pm 1\\
            0+ \pm i + 0j+0k &\mapsto \pm i\\
            0+ 0i+ \pm j +0k &\mapsto \pm j\\
            0+ 0i+0j + \pm k &\mapsto \pm k,
        \end{align*} which is an isomorphism because these units as a group obey the same multiplication rule as $Q_8$ due to the multiplication on the integral Hamilton Quaternions being defined in the same manner (meaning the multiplication tables are identical under the relabeling given above).
    \end{proof}
\end{enumerate}
\end{document}