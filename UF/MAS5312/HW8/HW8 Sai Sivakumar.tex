\documentclass[11pt]{article}

% packages
\usepackage{physics}
% margin spacing
\usepackage[top=1in, bottom=1in, left=0.5in, right=0.5in]{geometry}
\usepackage{hanging}
\usepackage{amsfonts, amsmath, amssymb, amsthm}
\usepackage{systeme}
\usepackage[none]{hyphenat}
\usepackage{fancyhdr}
\usepackage[nottoc, notlot, notlof]{tocbibind}
\usepackage{graphicx}
\graphicspath{{./images/}}
\usepackage{float}
\usepackage{siunitx}
\usepackage{esint}
\usepackage{cancel}
\usepackage{enumitem}

% permutations (second line is for spacing)
\usepackage{permute}
\renewcommand*\pmtseparator{\,}

% colors
\usepackage{xcolor}
\definecolor{p}{HTML}{FFDDDD}
\definecolor{g}{HTML}{D9FFDF}
\definecolor{y}{HTML}{FFFFCF}
\definecolor{b}{HTML}{D9FFFF}
\definecolor{o}{HTML}{FADECB}
%\definecolor{}{HTML}{}

% \highlight[<color>]{<stuff>}
\newcommand{\highlight}[2][p]{\mathchoice%
  {\colorbox{#1}{$\displaystyle#2$}}%
  {\colorbox{#1}{$\textstyle#2$}}%
  {\colorbox{#1}{$\scriptstyle#2$}}%
  {\colorbox{#1}{$\scriptscriptstyle#2$}}}%

% header/footer formatting
\pagestyle{fancy}
\fancyhead{}
\fancyfoot{}
\fancyhead[L]{MAS5312}
\fancyhead[C]{Assignment 8}
\fancyhead[R]{Sai Sivakumar}
\fancyfoot[R]{\thepage}
\renewcommand{\headrulewidth}{1pt}

% paragraph indentation/spacing
\setlength{\parindent}{0cm}
\setlength{\parskip}{10pt}
\renewcommand{\baselinestretch}{1.25}

% extra commands defined here
\newcommand{\ihat}{\boldsymbol{\hat{\textbf{\i}}}}
\newcommand{\jhat}{\boldsymbol{\hat{\textbf{\j}}}}
\newcommand{\dr}{\vec{r}~^{\prime}(t)}
\newcommand{\dx}{x^{\prime}(t)}
\newcommand{\dy}{y^{\prime}(t)}

\newcommand{\br}[1]{\left(#1\right)}
\newcommand{\sbr}[1]{\left[#1\right]}
\newcommand{\cbr}[1]{\left\{#1\right\}}

\newcommand{\dprime}{\prime\prime}
\newcommand{\lap}[2]{\mathcal{L}[#1](#2)}

\newcommand{\divides}{\mid}

% bracket notation for inner product
\usepackage{mathtools}

\DeclarePairedDelimiterX{\abr}[1]{\langle}{\rangle}{#1}

\DeclareMathOperator{\Span}{span}
\DeclareMathOperator{\nullity}{nullity}
\DeclareMathOperator\Aut{Aut}
\DeclareMathOperator\Inn{Inn}
\DeclareMathOperator{\Orb}{Orb}
\DeclareMathOperator{\lcm}{lcm}
\DeclareMathOperator{\Hol}{Hol}
\DeclareMathOperator{\Jac}{Jac}
\DeclareMathOperator{\rad}{rad}

% set page count index to begin from 1
\setcounter{page}{1}

\begin{document}
\begin{enumerate}
    \item (DF9.4.8) Prove that $K_1 = \mathbb{F}_{11}[x]/(x^2+1)$ and $K_2 = \mathbb{F}_{11}[y]/(y^2+2y+2)$ are both fields with $121$ elements. Prove that the map which sends the element $p(\overline{x})$ of $K_1$ to the element $p(\overline{y}+1)$ of $K_2$ (where $p$ is any polynomial with coefficients in $\mathbb{F}_{11}$) is well defined and gives a ring (hence field) isomorphism from $K_1$ to $K_2$. \begin{proof}
      We show that the polynomials $x^2+1$ and $y^2+2y+2$ are irreducible in $\mathbb{F}_{11}[x]$ and $\mathbb{F}_{11}[y]$ respectively. Since both polynomials are quadratic, they are reducible if and only if they have linear factors; i.e., if they have roots in $\mathbb{F}_{11}$. Brute force calculation gives \begin{table}[h]
        \centering
        \begin{tabular}{c|c|c}
        $\mathbb{F}_{11}$ & $x^2+1$ & $y^2+2y+2$\\
        \hline
         0 & 1 & 2\\
         1 & 2 & 5\\
         2 & 5 & 10\\
         3 & 10 & 6\\
         4 & 6 & 4\\
         5 & 4 & 4\\
         6 & 4 & 6\\
         7 & 6 & 10\\
         8 & 10 & 5\\
         9 & 5 & 2\\
         10 & 2 & 1\\
        \end{tabular}
        \end{table} from which it follows that both polynomials cannot have linear factors and hence are irreducible. Since $\mathbb{F}_{11}$ is a field, it follows that $\mathbb{F}_{11}[x],\mathbb{F}_{11}[y]$ are Principal Ideal Domains. Thus $x^2+1$ and $y^2+2y+2$ are also prime elements so that the nonzero ideals $(x^2+1)$ and $(y^2+2y+2)$ are prime ideals, hence maximal ideals. It follows that $K_1$ and $K_2$ are indeed fields.
        
        We can find upper bounds for the number of elements each field has. Observe that in $K_1$ we have that $\overline{x}^2 = -1$, and in $K_2$ we have that $\overline{y}^2 = 2\overline{y}+2$. It follows that for any element of $K_1$ or $K_2$, its degree (where in $K_1$ and $K_2$ we can view the elements of these fields as polynomials in $\overline{x}$ or $\overline{y}$ respectively) is strictly less than $2$, since any term with degree greater than or equal to 2 may be reduced in this manner (it may take many reductions to do so) into terms of degree $0$ and $1$. 

        It follows that any element of $K_1$ takes the form $a\overline{x} + b$ and any element of $K_2$ takes the form $c\overline{y} + d$, where $a,b,c,d\in \mathbb{F}_{11}$. Since each of $a,b,c,d$ can take on one of $11$ values, it follows by counting that there are at most $121$ elements of $K_1$ and the same for $K_2$. We can indeed exhibit all $121$ elements of each field in exactly this manner: \[K_1 = \cbr{a\overline{x}+ b\mid a,b\in \mathbb{F}_{11}}\] \[K_2 = \cbr{c\overline{y}+ d\mid c,d\in \mathbb{F}_{11}}\]
        Let $\varphi\colon K_1\to K_2$ be given by $p(\overline{x})\mapsto p(\overline{y}+1)$; i.e., sending $\overline{x}$ to $\overline{y}+1$. We show that $\varphi$ is well defined: Let $r,s$ be polynomials with $\overline{r},\overline{s} = \overline{x}$. So $r = x+ P(x)(x^2+1)^j$ and $s = x+Q(x)(x^2+1)^k$ for polynomials $P,Q$ and $k,j\geq 1$ (i.e. $r$ and $s$ project to $\overline{x}$). Then $\varphi(\overline{r}) = \overline{(y+1)+ P(y+1)((y+1)^2+1)^j} = \overline{(y+1)+ P(y+1)(y^2+2y+2)^j} = \overline{y}+1$ and $\varphi(\overline{s}) = \overline{(y+1)+Q(x)((y+1)^2+1)^k} = \overline{y}+1$ for the same reason. Hence $\overline{r},\overline{s}$ map into the same equivalence class so $\varphi$ is well defined.

        We check that $\varphi$ is a ring isomorphism: 
        
        For any elements $a(\overline{x}), b(\overline{x})\in K_1$, we have that $\varphi(a(\overline{x}) + b(\overline{x})) = a(\overline{y}+1) + b(\overline{y}+1) = \varphi(a(\overline{x})) + \varphi(b(\overline{x}))$ and $\varphi(a(\overline{x})b(\overline{x})) = a(\overline{y}+1)b(\overline{y}+1) = \varphi(a(\overline(x)))\varphi(b(\overline{x)}))$. Furthermore, the multiplicative identity in $K_1$ identity is the class $1$, which is evidently mapped into the same class $1$ in $K_2$. Thus $\varphi$ is a ring (field) homomorphism. It suffices to show that this map is injective (since injective maps between sets of the same finite size are surjective), which we do using the trivial kernel characterization: Suppose some element $a(\overline{x})$ is mapped to the zero class in $K_2$; that is, $a(y+1)$ is divisible by $y^2+2y+1$. This is equivalent to saying that $a(x)$ is divisible by $x^2+1$, since we can make the invertible change of variables $x\mapsto y+1$ to obtain $x^2+1\mapsto (y+1)^2 + 1 = y^2+2y+2$. Hence $a(\overline{x})$ had to be the zero class in $K_1$, from which it follows that the kernel of $\varphi$ is trivial and so $\varphi$ is injective. Since both $K_1,K_2$ have the same finite size ($121$), it follows that $\varphi$ is surjective also and hence bijective. Thus $\varphi$ is a field isomorphism from $K_1$ to $K_2$.
    \end{proof}
    \item (DF9.4.15) Prove that if $F$ is a field then the polynomial $X^n-x$ which has coefficients in the ring $F[[x]]$ of formal power series (cf. Exercise 3 of section 7.2) is irreducible over $F[[x]]$. [Recall that $F[[x]]$ is a Euclidean Domain --- cf. Exercise 5, Section 7.2 and Example 4, Section 8.1.] \begin{proof}
      We apply Eisenstein's criterion to the polynomial $X^n-x$ in $(F[[x]])[X]$ directly. Observe that we can take the prime ideal $P = (x)$, which is prime since $F[[x]]/(x)$ is isomorphic to $F$, which is an integral domain ($F$ is a field). Observe that $x\in P$, but $x\not \in P^2 = (x^2)$. It follows that $X^n-x$ is irreducible in $(F[[x]])[X]$.
    \end{proof}
    \item (DF9.5.1) Let $F$ be a field and let $f(x)$ be a nonconstant polynomial in $F[x]$. Describe the nilradical of $F[x]/(f(x))$ in terms of the factorization of $f(x)$ (cf. Exercise 29, Section 7.3).
    
    Since $F$ is a field, $F[x]$ is a Unique Factorization Domain, so that we can write $f(x)$ as a (unique up to associates) product of irreducible elements $p_i(x)$. In the case that $f(x)$ itself is irreducible (that is, if only one such $p_1(x)$ up to multiplication by a unit which constitutes the factorization of $f(x)$) then there are no nonzero nilpotent elements in $F[x]/(f(x))$ (every element which is nilpotent is of the form $g(x)p_1(x)+ (f(x))$ for some $g(x)\in F[x]$, and $f(x)$ will divide the representative $g(x)p_1(x)$ so that the element is itself zero in the quotient ring) since $F[x]/(f(x))$ becomes a field. 
    
    Otherwise, we can factorize $f(x)$ into a nontrivial product of irreducibles $p_1^{e_1}(x)\cdots p_n^{e_1}(x)$ for some $n$ and $e_i\geq 1$; here we adjust the choice of the irreducible factors since they are up to associates so that we can collect them into powers whenever possible. It follows that the nilradical of $F[x]/(f(x))$ is given by the set $\cbr{g(x)p_1^{\alpha_1}(x)\cdots p_n^{\alpha_1}(x)  + (f(x))\mid g(x)\in F[x],\alpha_i\geq 1}$; that is, the set of those cosets whose representative polynomial is divisible by every irreducible factor $p_i(x)$. Note that we could have used this formulation for the case when $f(x)$ itself was irreducible, but we noted before that the set consists of only the zero class in $F[x]/(f(x))$.
    
    These elements are indeed the nilpotent elements since we can always raise an element of the form\\ $g(x)p_1^{\alpha_1}(x)\cdots p_n^{\alpha_1}(x)  + (f(x))$ to a sufficiently large power $k$ to ensure that each the resulting powers $k\alpha_i$ of each $p_i(x)$ exceed $\max\cbr{e_i\mid 1\leq i \leq n}$, in which case $f(x)$ divides the representative polynomial and so the element becomes the zero class in the quotient ring. 
    
    This is similar to our determination earlier in the case of the nilpotent elements of $\mathbb{Z}/n\mathbb{Z}$ where the nilpotent elements are those which are divisible by every prime factor of $n$. Here the nilpotent elements are those which are divisible by each of the irreducible (prime) factors of $f(x)$.

    In these exercises $R$ is a ring with $1$ and $M$ is a left $R$-module.
    \item Auxiliary result for DF10.1.13: (DF10.1.7) Let $N_1\subseteq N_2\subseteq \cdots$ be an ascending chain of submodules of $M$. Prove that $M^{\prime} = \cup_{i=1}^\infty N_i$ is a submodule of $M$. \begin{proof}
      Every submodule $N_i$ is nonempty subset of $M$ so their union $M^{\prime}$ is also a nonempty subset of $M$. Let $x,y\in M^{\prime}$, so that $x\in N_j$ and $y\in N_k$; without loss of generality suppose $k\geq j$ so that $x\in N_k$ also. By the submodule structure of $N_k$ it follows that $x+ry\in N_k$ for all $r\in R$, and since $x,y$ were arbitrary it follows that $M^{\prime}$ is a submodule of $M$.
    \end{proof}
    \item Auxiliary result for DF10.1.13: (DF10.1.10) If $I$ is a right ideal of $R$, the \textit{annihilator of $I$ in $M$} is defined to be $M_1 = \cbr{m\in M\mid am = 0 \text{ for all } a\in I}$. Prove that the annihilator of $I$ in $M$ is a submodule of $M$. \begin{proof}
      Observe that $0_M$ is annihilated by $I$ due to the module structure on $M$ ($r0_M = 0_M$ for all $r\in R$), so that $0_M\in M_1$. It follows that $M_1$ is a nonempty subset of $M$, and let $x,y\in M_1$. Then for any $r\in R$, we have for any $a\in I$ that \[a(x+ry) = ax + a(ry) = 0 + (ar)y = 0+ 0 = 0,\] with $ar\in I$ since $I$ is a right ideal of $R$. Since $x,y$ were arbitrary it follows that $M_1$ is a submodule of $M$.
    \end{proof}
    \item (DF10.1.13) Let $I$ be an ideal of $R$. Let $M^{\prime}$ be the subset of elements $a$ of $M$ that are annihilated by some power $I^k$ of the ideal $I$, where the power may depend on $a$. Prove that $M^{\prime}$ is a submodule of $M$. [Use Exercise 7.] \begin{proof}
      Let $M_k$ denote the annihilator of $I^k$ in $M$, and recall that $M_k$ are submodules of $M$. We show that $M_k\subseteq M_{k+1}$ for any $k\geq 1$: For any element $m_k\in M_k$, take any element $\sum_{i=1}^n a_{1i}a_{2i}\cdots a_{(k+1)i}$ of $I^{k+1}$ and see that \begin{align*}
        \br{\sum_{i=1}^n a_{1i}a_{2i}\cdots a_{(k+1)i}}m_k &= \sum_{i=1}^n (a_{1i}a_{2i}\cdots a_{(k+1)i})m_k \\
        &= \sum_{i=1}^n a_{1i}[(a_{2i}\cdots a_{(k+1)i})m_k]\\
        &= \sum_{i=1}^n a_{1i}0_M = 0_M,
      \end{align*} since each $(a_{2i}\cdots a_{(k+1)i})$ are elements of $I^k$. It follows that $m_k\in M_{k+1}$ and we obtain a chain $M_1\subseteq M_2\subseteq \cdots$ of submodules.
      
      Observe that the union $M^{\dprime} = \cup_{i=1}^\infty M_i$ is thus a submodule of $M$, and we show that $M^\prime = M^{\dprime}$. It is clear that every element of $M^{\dprime}$ in $M^\prime$ since for any element $b\in M^{\dprime}$, we have that $b\in M_k$ for some $k$, meaning $b$ is annihilated by $I^k$. Similarly, for any $a\in M^\prime$, we have that $a$ is annihilated by $I^k$ for some $k$, meaning that $a\in M_k$. Thus $M^\prime = M^{\dprime}$, meaning $M^\prime$ is a submodule of $M$.
    \end{proof}
\end{enumerate}
\end{document}