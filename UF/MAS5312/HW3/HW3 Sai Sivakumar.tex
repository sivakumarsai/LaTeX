\documentclass[11pt]{article}

% packages
\usepackage{physics}
% margin spacing
\usepackage[top=1in, bottom=1in, left=0.5in, right=0.5in]{geometry}
\usepackage{hanging}
\usepackage{amsfonts, amsmath, amssymb, amsthm}
\usepackage{systeme}
\usepackage[none]{hyphenat}
\usepackage{fancyhdr}
\usepackage[nottoc, notlot, notlof]{tocbibind}
\usepackage{graphicx}
\graphicspath{{./images/}}
\usepackage{float}
\usepackage{siunitx}
\usepackage{esint}
\usepackage{cancel}
\usepackage{enumitem}

% permutations (second line is for spacing)
\usepackage{permute}
\renewcommand*\pmtseparator{\,}

% colors
\usepackage{xcolor}
\definecolor{p}{HTML}{FFDDDD}
\definecolor{g}{HTML}{D9FFDF}
\definecolor{y}{HTML}{FFFFCF}
\definecolor{b}{HTML}{D9FFFF}
\definecolor{o}{HTML}{FADECB}
%\definecolor{}{HTML}{}

% \highlight[<color>]{<stuff>}
\newcommand{\highlight}[2][p]{\mathchoice%
  {\colorbox{#1}{$\displaystyle#2$}}%
  {\colorbox{#1}{$\textstyle#2$}}%
  {\colorbox{#1}{$\scriptstyle#2$}}%
  {\colorbox{#1}{$\scriptscriptstyle#2$}}}%

% header/footer formatting
\pagestyle{fancy}
\fancyhead{}
\fancyfoot{}
\fancyhead[L]{MAS5312}
\fancyhead[C]{Assignment 3}
\fancyhead[R]{Sai Sivakumar}
\fancyfoot[R]{\thepage}
\renewcommand{\headrulewidth}{1pt}

% paragraph indentation/spacing
\setlength{\parindent}{0cm}
\setlength{\parskip}{10pt}
\renewcommand{\baselinestretch}{1.25}

% extra commands defined here
\newcommand{\ihat}{\boldsymbol{\hat{\textbf{\i}}}}
\newcommand{\jhat}{\boldsymbol{\hat{\textbf{\j}}}}
\newcommand{\dr}{\vec{r}~^{\prime}(t)}
\newcommand{\dx}{x^{\prime}(t)}
\newcommand{\dy}{y^{\prime}(t)}

\newcommand{\br}[1]{\left(#1\right)}
\newcommand{\sbr}[1]{\left[#1\right]}
\newcommand{\cbr}[1]{\left\{#1\right\}}

\newcommand{\dprime}{\prime\prime}
\newcommand{\lap}[2]{\mathcal{L}[#1](#2)}

\newcommand{\divides}{\mid}

% bracket notation for inner product
\usepackage{mathtools}

\DeclarePairedDelimiterX{\abr}[1]{\langle}{\rangle}{#1}

\DeclareMathOperator{\Span}{span}
\DeclareMathOperator{\nullity}{nullity}
\DeclareMathOperator\Aut{Aut}
\DeclareMathOperator\Inn{Inn}
\DeclareMathOperator{\Orb}{Orb}
\DeclareMathOperator{\lcm}{lcm}
\DeclareMathOperator{\Hol}{Hol}

% set page count index to begin from 1
\setcounter{page}{1}

\begin{document}
\begin{enumerate}
    \item (DF7.3.29) Let $R$ be a commutative ring. Recall that an element $x\in R$ is nilpotent if $x^n=0$ for some $n\in \mathbb{Z}^+$. Prove that the set of nilpotent elements form an ideal --- called the \textit{nilradical} of $R$ and denoted by $\mathfrak{R}(R)$. [Use the Binomial Theorem to show $\mathfrak{R}(R)$ is closed under addition.]
    \begin{proof}
      Let $R$ be a commutative ring.

      We check that the nilradical $\mathfrak{R}(R)$ is a subring of $R$. Observe that $0^1 = 0$, so that $0\in \mathfrak{R}(R)$ and $\mathfrak{R}(R)$ is a nonempty subset of $R$. Let $x,y\in R$ such that $x^n=0$ and $y^m=0$. By properties of the multiplication in a ring and induction it follows that $-y$ is nilpotent: \[(-y)^m = \begin{cases}
        y^m = 0 &\text{ if $m$ is even}\\
        -(y^m) = 0 &\text{ if $m$ is odd}
      \end{cases}.\] Since $R$ is commutative, it follows that \begin{align*}
        (x-y)^{n+m} &= \sum_{k=0}^{n+m} \binom{n+m}{k}x^k(-y)^{n+m-k}\\
        &= \sum_{k=0}^{n-1} \binom{n+m}{k}x^k(-y)^{n+m-k} + \sum_{k=n}^{n+m} \binom{n+m}{k}x^k(-y)^{n+m-k}.
      \end{align*} Each sum must vanish due to the exponent on either $x$ or $(-y)$ for each term being large enough to cause the term to vanish. Specifically, \begin{align*}
        (-y)^{n+m-k} = (-y)^{n-k}(-y)^m = (-y)^{n-k}0 = 0 &\quad \text{for $k < n$}\\
        x^k = x^nx^{k-n} = 0(x^{k-n}) = 0 &\quad \text{for $k\geq n$},
      \end{align*} so that the terms in each sum vanish. It follows that $(x-y)^{n+m} = 0$, so that $x-y$ is nilpotent and $\mathfrak{R}(R)$ is a subgroup of $R$. Closure under multiplication is also easy to check since $R$ is a commutative ring. The element $xy$ is nilpotent since $(xy)^{nm} = x^{nm}y^{nm} = (x^n)^m(y^m)^n = (0)(0) = 0$, so $\mathfrak{R}(R)$ is closed under multiplication and hence is a subring of $R$.

      It follows in a similar manner that $\mathfrak{R}(R)$ is an ideal. For any element $r\in R$ and $x\in \mathfrak{R}(R)$ with $x^n = 0$, we have that \begin{align*}
        (rx)^n &= r^nx^n = r^n0 = 0\\
        (xr)^n &= x^nr^n = 0r^n = 0.
      \end{align*} Thus $xr,rx$ are nilpotent and it follows that $R$ is an ideal.
    \end{proof}
    
    Let $R$ be a ring with identity $1\neq 0$.
    \item (DF7.4.1) Let $L_j$ be the left ideal of $M_n(R)$ consisting of arbitrary elements in the $j^{\text{th}}$ column and zero in all other entries and let $E_{ij}$ be the element of $M_n(R)$ whose $i,j$ entry is $1$ and whose other entries are all $0$. Prove that $L_j = M_n(R)E_{ij}$ for any $i$.
    \begin{proof}
      Let $n$ be a positive integer and let $i$ be any integer from $1$ to $n$. It is clear that $L_j$ is a left ideal since it is an additive subgroup and is closed under multiplication by matrices from $M_n(R)$ on the left: For any element $A \in L_j$ and $M\in M_n(R)$, we have \[(MA)_{rs} = \sum_{k=1}^n M_{rk}A_{ks} =  \begin{cases}
        0 &\text{if $s\neq j$}\\
        \sum_{k=1}^n M_{rk}A_{kj} &\text{if $s = j$},
      \end{cases}\] meaning that only the $j$-th column of the resulting matrix survives in the product.

      Any element $L$ of $L_j$ may be written as $ME_{ij}$, where $M\in M_n(R)$ and $E_{ij}$ is the matrix whose entries are zero except for the $i,j$-th entry being $1\in R$. Let $L_{ij} = \ell_i\in R$ for $1\leq i\leq n$ (and all other entries of $L$ are zero). Then choose $M$ to be the matrix whose entries are zero except for its $i$-th column being the $j$-th column of $L$; that is, $M_{ri} = L_{rj}$ for $1\leq r \leq n$. It follows that \[(ME_{ij})_{rs} = \sum_{k=1}^n M_{rk}(E_{ij})_{ks} = M_{ri}(E_{ij})_{is} = L_{rj}(E_{ij})_{is} = \begin{cases}
        0 &\text{if $s\neq j$}\\
        L_{rs}(1) = \ell_r &\text{if $s = j$}
      \end{cases} = L_{rs},\] and since $i$ was arbitrary, it follows that $L_j\subseteq M_n(R)E_{ij}$.

      The reverse inclusion is checked similarly. Any matrix $M$ in $M_n(R)$ multiplied by $E_{ij}$ on the right has the form we desire. Note also that because matrix multiplication distributes, we only need to check that the product of one matrix $M$ with $E_{ij}$ has the form needed to be an element of $L_j$. We have that \[(ME_{ij})_{rs} = \sum_{k=1}^n M_{rk}(E_{ij})_{ks} = M_{ri}(E_{ij})_{is} =\begin{cases}
        0 &\text{if $s\neq j$}\\
        M_{ri}(1) = M_{ri} &\text{if $s = j$},
      \end{cases}\] meaning the resulting matrix is the matrix with zeros in all entries except for the $j$-th column whose entries are taken from the $i$-th column of $M$. Since $i$ was arbitrary, it follows that $M_n(R)E_{ij}\subseteq L_j$. 

      Hence $L_j = M_n(R)E_{ij}$ for any $i$.
    \end{proof}
    \item Lemma. The preimage of a subring under a ring homomorphism is a subring, and the preimage of an ideal is an ideal.
    \begin{proof}
      Let $\varphi\colon R\to S$ be a homomorphism of rings. Let $T$ be a subring of $S$, and let $a,b\in \varphi^{-1}(T)$. Note $\varphi(0_R) = 0_S\in T$, so $\varphi^{-1}(T)$ contains $0_R$ and hence is a nonempty subset of $R$. Then $\varphi(a-b) = \varphi(a) - \varphi(b)\in T$ since $T$ is an additive group, and $\varphi(ab) = \varphi(a)\varphi(b)\in T$ since $T$ is a ring. Hence $a-b, ab\in \varphi^{-1}(T)$, so $\varphi^{-1}(T)$ is a subring of $R$.

      If $T$ is an ideal of $S$, then we check that the preimage under $\varphi$ is an ideal of $R$: By the above argument, we know that the preimage $\varphi^{-1}(T)$ is a subring of $R$. Then let $r\in R$ and $a\in \varphi^{-1}(T)$. We have $\varphi(ra) = \varphi(r)\varphi(a)\in T$ and $\varphi(ar) = \varphi(a)\varphi(r)\in T$ since $T$ is an ideal in $S$. Hence $\varphi^{-1}(T)$ is closed under multiplication on the left and right by elements of $R$, so it is an ideal.
    \end{proof}
    \item (DF7.4.13) Let $\varphi\colon R\to S$ be a homomorphism of commutative rings. \begin{enumerate}[label=\textbf{(\alph*)}]
      \item Prove that if $P$ is a prime ideal of $S$ then either $\varphi^{-1}(P) = R$ or $\varphi^{-1}(P)$ is a prime ideal of $R$. Apply this to the special case when $R$ is a subring of $S$ and $\varphi$ is the inclusion homomorphism to deduce that if $P$ is a prime ideal of $S$ then $P\cap R$ is either $R$ or a prime ideal of $R$.
      \begin{proof}
        By the previous lemma, we know that $\varphi^{-1}(P)$ is an ideal of $R$. If $ab\in \varphi^{-1}(P)$, then $\varphi(ab) = \varphi(a)\varphi(b)\in P$. Because $R,S$ are commutative rings, we can take without loss of generality that $\varphi(a)\in P$. What remains is to determine what happens if $\varphi(b)$ is in $P$ or not. We have $b\in \varphi^{-1}(P)$ if $\varphi(b)\in P$. Otherwise, if $\varphi(b)\not\in P$, then $b\not\in \varphi^{-1}(P)$. It follows that $\varphi^{-1}(P)$ is a prime ideal of $R$ if it is properly contained in $R$, since at least one of $a,b$ is in $\varphi^{-1}(P)$ whenever $ab\in \varphi^{-1}(P)$. But it is also possible for $\varphi^{-1}(P)$ to contain $1_R$ and thus be equivalent to $R$.

        When $\varphi$ is the inclusion homomorphism, it follows that $\varphi^{-1}(P) = P\cap R$ (since $P\cap R$ contains all of the elements of $R$ which map into $P$ under $\varphi$). By the previous result, it follows that $P\cap R$ is either $R$ or is a prime ideal of $R$.
      \end{proof}
      \item Prove that if $M$ is a maximal ideal of $S$ and $\varphi$ is surjective then $\varphi^{-1}(M)$ is a maximal ideal of $R$. Give an example to show that this need not be the case if $\varphi$ is not surjective.
      \begin{proof}
        Let $\pi\colon S\to S/M$ be the projection map, which is surjective. Then the composition $\pi\circ \varphi\colon R\to S/M$ is surjective since both $\pi,\varphi$ are surjective. The kernel of $\pi\circ \varphi$ is found by investigating which elements $r\in R$ are mapped to $0_S+M$ in $S/M$: If $\pi(\varphi(r)) = 0_S + M$, it follows that $\varphi(r)\in M$, so that $r\in \varphi^{-1}(M)$. Hence $\ker(\pi\circ\varphi) = \varphi^{-1}(M)$, and by the first isomorphism theorem we have that \[\frac{R}{\varphi^{-1}(M)}\cong \frac{S}{M}.\] Since $M$ is a maximal ideal in $S$, the quotient ring $S/M$ is a field, so that $R/\varphi^{-1}(M)$ is also a field. It follows from the lattice isomorphism theorem that $\varphi^{-1}(M)$ is a maximal ideal of $R$ (since there are no ideals outside of $R/\varphi^{-1}(M)$ and the trivial ideal in $R/\varphi^{-1}(M)$, it follows that there are no proper ideals of $R$ containing $\varphi^{-1}(M)$ outside of $\varphi^{-1}(M)$).
      \end{proof}
    \end{enumerate}
    \item (DF7.4.25) Assume $R$ is commutative and for each $a\in R$ there is an integer $n>1$ (depending on $a$) such that $a^n = a$. Prove that every prime ideal of $R$ is a maximal ideal.
    \begin{proof}
      Let $R$ be a commutative ring with the property that for every $a\in R$ there is an $n\in \mathbb{Z}^+$ depending on $a$ such that $a^n = a$.

      We show that for any prime ideal $P$ of $R$, that $R/P$ is a field (so that by the lattice isomorphism theorem $P$ is a maximal ideal of $R$.) Suppose by way of contradiction that there is a proper nontrivial ideal $\overline{J}$ of $R/P$. Then for some nontrivial element $j+P\in \overline{J}$ (so $j\not\in P$), we can find $n\in \mathbb{Z}^+$ depending on $j$ such that $(j+P)^n = j^n + P = j+P$, from which it follows that $j^n - j = j(j^{n-1}-1_R)\in P$. Since $P$ is a prime ideal and $j\not\in P$, it follows that $j^{n-1}-1_R\in P$. Equivalently, $j^{n-1} + P = 1_R+P$, so that by taking the product $(j^{n-2}+P)(j+P) = j^{n-1} + P = 1_R+P$, it follows from $\overline{J}$ being an ideal of $R/P$ that $\overline{J}$ contains the identity element $1_R + P$. By closure under multiplication by elements of $R/P$, it follows that $\overline{J}$ contains $R/P$ so that $\overline{J} = R/P$, which contradicts the assumption that $\overline{J}$ was a proper nontrivial ideal of $R/P$.

      Hence the ideals of $R/P$ are only $R/P$ and the trivial ideal, meaning $R/P$ is a field. Since $P$ was arbitrary, every prime ideal of $R$ is a maximal ideal of $R$.
    \end{proof}
\end{enumerate}
\end{document}