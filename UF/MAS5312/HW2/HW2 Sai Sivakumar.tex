\documentclass[11pt]{article}

% packages
\usepackage{physics}
% margin spacing
\usepackage[top=1in, bottom=1in, left=0.5in, right=0.5in]{geometry}
\usepackage{hanging}
\usepackage{amsfonts, amsmath, amssymb, amsthm}
\usepackage{systeme}
\usepackage[none]{hyphenat}
\usepackage{fancyhdr}
\usepackage[nottoc, notlot, notlof]{tocbibind}
\usepackage{graphicx}
\graphicspath{{./images/}}
\usepackage{float}
\usepackage{siunitx}
\usepackage{esint}
\usepackage{cancel}
\usepackage{enumitem}

% permutations (second line is for spacing)
\usepackage{permute}
\renewcommand*\pmtseparator{\,}

% colors
\usepackage{xcolor}
\definecolor{p}{HTML}{FFDDDD}
\definecolor{g}{HTML}{D9FFDF}
\definecolor{y}{HTML}{FFFFCF}
\definecolor{b}{HTML}{D9FFFF}
\definecolor{o}{HTML}{FADECB}
%\definecolor{}{HTML}{}

% \highlight[<color>]{<stuff>}
\newcommand{\highlight}[2][p]{\mathchoice%
  {\colorbox{#1}{$\displaystyle#2$}}%
  {\colorbox{#1}{$\textstyle#2$}}%
  {\colorbox{#1}{$\scriptstyle#2$}}%
  {\colorbox{#1}{$\scriptscriptstyle#2$}}}%

% header/footer formatting
\pagestyle{fancy}
\fancyhead{}
\fancyfoot{}
\fancyhead[L]{MAS5312}
\fancyhead[C]{Assignment 2}
\fancyhead[R]{Sai Sivakumar}
\fancyfoot[R]{\thepage}
\renewcommand{\headrulewidth}{1pt}

% paragraph indentation/spacing
\setlength{\parindent}{0cm}
\setlength{\parskip}{10pt}
\renewcommand{\baselinestretch}{1.25}

% extra commands defined here
\newcommand{\ihat}{\boldsymbol{\hat{\textbf{\i}}}}
\newcommand{\jhat}{\boldsymbol{\hat{\textbf{\j}}}}
\newcommand{\dr}{\vec{r}~^{\prime}(t)}
\newcommand{\dx}{x^{\prime}(t)}
\newcommand{\dy}{y^{\prime}(t)}

\newcommand{\br}[1]{\left(#1\right)}
\newcommand{\sbr}[1]{\left[#1\right]}
\newcommand{\cbr}[1]{\left\{#1\right\}}

\newcommand{\dprime}{\prime\prime}
\newcommand{\lap}[2]{\mathcal{L}[#1](#2)}

\newcommand{\divides}{\mid}

% bracket notation for inner product
\usepackage{mathtools}

\DeclarePairedDelimiterX{\abr}[1]{\langle}{\rangle}{#1}

\DeclareMathOperator{\Span}{span}
\DeclareMathOperator{\nullity}{nullity}
\DeclareMathOperator\Aut{Aut}
\DeclareMathOperator\Inn{Inn}
\DeclareMathOperator{\Orb}{Orb}
\DeclareMathOperator{\lcm}{lcm}
\DeclareMathOperator{\Hol}{Hol}

% set page count index to begin from 1
\setcounter{page}{1}

\begin{document}
\begin{enumerate}
    \item (DF7.2.10) Consider the following elements of the integral group ring $\mathbb{Z}S_3$ \[\alpha = 3(1\, 2) - 5(2\, 3) + 14(1\,2\,3) \quad \text{and}\quad \beta = 6(1) + 2(2\,3) -7(1\,3\,2)\] (where $(1)$ is the identity of $S_3$). Compute the following elements:
    
    \textbf{(a)} $\alpha + \beta$, \quad \textbf{(b)} $2\alpha - 3\beta$, \quad \textbf{(c)} $\alpha\beta$, \quad \textbf{(d)} $\beta\alpha$,\quad \textbf{(e)} $\alpha^2$

    We have:
    \begin{enumerate}[label=\textbf{(\alph*)}]
        \item $\alpha + \beta = (3(1\, 2) - 5(2\, 3) + 14(1\,2\,3)) + (6(1) + 2(2\,3) -7(1\,3\,2)) = \boxed{6(1) + 3(1\,2) -3(2\,3)+ 14(1\,2\,3)-7(1\,3\,2)}$
        \item  $2\alpha - 3\beta = \alpha + \alpha +(- \beta - \beta - \beta) = (6(1\, 2) - 10(2\, 3) + 28(1\,2\,3)) - (18(1) + 6(2\,3) -21(1\,3\,2))$ $= \boxed{-18(1) + 6(1\,2) - 16(2\,3)+28(1\,2\,3) +21(1\,3\,2)}$
        \item $\alpha\beta= [3(1\, 2) - 5(2\, 3) + 14(1\,2\,3)][6(1) + 2(2\,3) -7(1\,3\,2)] = 18(1\,2)(1) + 6(1\,2)(2\,3)- 21(1\,2)(1\,3\,2) - 30 (2\,3)(1) -10 (2\,3)(2\,3) + 35(2\,3)(1\,3\,2) + 84(1\,2\,3)(1) + 28(1\,2\,3)(2\,3) - 98(1\,2\,3)(1\,3\,2)$ \newline$= \boxed{-108(1) + 81(1\,2) -21 (1\,3)-30(2\,3) + 90(1\,2\,3)}$
        \item $\beta\alpha= [6(1) + 2(2\,3) -7(1\,3\,2)][3(1\, 2) - 5(2\, 3) + 14(1\,2\,3)] = 18(1)(1\,2)-30(1)(2\,3) + 84(1)(1\,2\,3) + 6(2\,3)(1\,2) - 10(2\,3)(2\,3) + 28(2\,3)(1\,2\,3) - 21(1\,3\,2)(1\,2) + 35(1\,3\,2)(2\,3)-98(1\,3\,2)(1\,2\,3)$\newline $= \boxed{-108(1) + 18(1\,2) + 63(1\,3) - 51(2\,3) + 84(1\,2\,3) + 6(1\,3\,2)}$
        \item $\alpha^2 = [3(1\, 2) - 5(2\, 3) + 14(1\,2\,3)][3(1\, 2) - 5(2\, 3) + 14(1\,2\,3)] = 9(1\,2)(1\,2) - 15(1\,2)(2\,3)+42(1\,2)(1\,2\,3) - 15(2\,3)(1\,2) + 25(2\,3)(2\,3) - 70(2\,3)(1\,2\,3) + 42(1\,2\,3)(1\,2) - 70(1\,2\,3)(2\,3)+196(1\,2\,3)(1\,2\,3)$\newline $= \boxed{34(1) - 70(1\,2) - 28(1\,3) + 42(2\,3) - 15(1\,2\,3)+181(1\,3\,2)}$
    \end{enumerate}
    
    In Section 7.3, rings are assumed to have a $1\neq 0$.
    \item (DF7.3.13) Prove that the ring $M_2(\mathbb{R})$ contains a subring isomorphic to $\mathbb{C}$.
    \begin{proof}
        Observing that \[\begin{pmatrix}
            0 & -1 \\ 1 & 0 
        \end{pmatrix}^2 = \begin{pmatrix}
            -1 & 0 \\ 0 & -1
        \end{pmatrix},\] construct the set \[S = \cbr{r_1\begin{pmatrix}
            1 & 0 \\ 0 & 1
        \end{pmatrix} + r_2\begin{pmatrix}
            0 & -1 \\ 1 & 0 
        \end{pmatrix}\colon r_1,r_2\in\mathbb{R}}.\] It is clear that this set is a nonempty subset of $M_2(\mathbb{R})$; what remains is to show is that this set under the same operations as $M_2(\mathbb{R})$ is a subring, and that this subring is isomorphic to $\mathbb{C}$.

        For arbitrary $r_1,r_2,r_3,r_4\in\mathbb{R}$, we have \[\br{r_1\begin{pmatrix}
            1 & 0 \\ 0 & 1
        \end{pmatrix} + r_2\begin{pmatrix}
            0 & -1 \\ 1 & 0 
        \end{pmatrix}}-\br{r_3\begin{pmatrix}
            1 & 0 \\ 0 & 1
        \end{pmatrix} + r_4\begin{pmatrix}
            0 & -1 \\ 1 & 0 
        \end{pmatrix}} = \begin{pmatrix}
            r_1 & -r_2 \\ r_2 & r_1
        \end{pmatrix} - \begin{pmatrix}
            r_3 & -r_4 \\ r_4 & r_3
        \end{pmatrix}\] \[ = \begin{pmatrix}
            r_1 - r_3 & - (r_2 - r_4) \\ r_2 - r_4 & r_1 - r_3
        \end{pmatrix} = (r_1-r_3)\begin{pmatrix}
            1 & 0 \\ 0 & 1
        \end{pmatrix} + (r_2-r_4)\begin{pmatrix}
            0 & -1 \\ 1 & 0 
        \end{pmatrix}\] and \[\br{r_1\begin{pmatrix}
            1 & 0 \\ 0 & 1
        \end{pmatrix} + r_2\begin{pmatrix}
            0 & -1 \\ 1 & 0 
        \end{pmatrix}}\br{r_3\begin{pmatrix}
            1 & 0 \\ 0 & 1
        \end{pmatrix} + r_4\begin{pmatrix}
            0 & -1 \\ 1 & 0 
        \end{pmatrix}}\] \[ = r_1r_3\begin{pmatrix}
            1 & 0 \\ 0 & 1
        \end{pmatrix}^2 + r_1r_4\begin{pmatrix}
            1 & 0 \\ 0 & 1
        \end{pmatrix}\begin{pmatrix}
            0 & -1 \\ 1 & 0 
        \end{pmatrix} + r_2r_3\begin{pmatrix}
            0 & -1 \\ 1 & 0 
        \end{pmatrix}\begin{pmatrix}
            1 & 0 \\ 0 & 1
        \end{pmatrix} + r_2r_4\begin{pmatrix}
            0 & -1 \\ 1 & 0 
        \end{pmatrix}^2\] \[ = (r_1r_3 - r_2r_4)\begin{pmatrix}
            1 & 0 \\ 0 & 1
        \end{pmatrix} + (r_1r_4 + r_2r_3)\begin{pmatrix}
            0 & -1 \\ 1 & 0 
        \end{pmatrix}.\] Observe that the difference of two elements of $S$ is in $S$, and the product of two elements of $S$ is also in $S$. It follows that $S$ is a subring of $M_2(\mathbb{R})$.

        To show that this subring is isomorphic to $\mathbb{C}$ we exhibit the map $\varphi \colon S \to \mathbb{C}$ given by \[\varphi\br{r_1\begin{pmatrix}
            1 & 0 \\ 0 & 1
        \end{pmatrix} + r_2\begin{pmatrix}
            0 & -1 \\ 1 & 0 
        \end{pmatrix}} = r_1 + r_2 i \quad (r_1,r_2\in\mathbb{R}),\] so that in particular we have that the identity matrix maps to $1+0i$ and the matrix $ \begin{pmatrix}
            0 & -1 \\ 1 & 0 
        \end{pmatrix} $ maps to $0+1i$. We show that this map is an isomorphism of rings. The operations of addition and multiplication are preserved: For arbitrary $r_1,r_2,r_3,r_4\in\mathbb{R}$, we have \[\varphi\br{(r_1+r_3)\begin{pmatrix}
            1 & 0 \\ 0 & 1
        \end{pmatrix} + (r_2+r_4)\begin{pmatrix}
            0 & -1 \\ 1 & 0 
        \end{pmatrix}} = (r_1+r_3) + (r_2+r_4)i\] \[ = (r_1+r_2i) + (r_3+r_4i) = \varphi\br{r_1\begin{pmatrix}
            1 & 0 \\ 0 & 1
        \end{pmatrix} + r_2\begin{pmatrix}
            0 & -1 \\ 1 & 0 
        \end{pmatrix}} + \varphi\br{r_3\begin{pmatrix}
            1 & 0 \\ 0 & 1
        \end{pmatrix} + r_4\begin{pmatrix}
            0 & -1 \\ 1 & 0 
        \end{pmatrix}}\] and \[\varphi\br{(r_1r_3 - r_2r_4)\begin{pmatrix}
            1 & 0 \\ 0 & 1
        \end{pmatrix} + (r_1r_4 + r_2r_3)\begin{pmatrix}
            0 & -1 \\ 1 & 0 
        \end{pmatrix}} = (r_1r_3 - r_2r_4) + (r_1r_4 + r_2r_3)i\] \[=(r_1+r_2i)(r_3+r_4i) = \varphi\br{r_1\begin{pmatrix}
            1 & 0 \\ 0 & 1
        \end{pmatrix} + r_2\begin{pmatrix}
            0 & -1 \\ 1 & 0 
        \end{pmatrix}}\varphi\br{r_3\begin{pmatrix}
            1 & 0 \\ 0 & 1
        \end{pmatrix} + r_4\begin{pmatrix}
            0 & -1 \\ 1 & 0 
        \end{pmatrix}}.\]

        The map $\varphi^{-1}\colon \mathbb{C}\to S$ given by \[\varphi^{-1}(r_1+r_2i) = r_1\begin{pmatrix}
            1 & 0 \\ 0 & 1
        \end{pmatrix} + r_2\begin{pmatrix}
            0 & -1 \\ 1 & 0 
        \end{pmatrix}\] is easily checked to be a two-sided inverse for $\varphi$. We have that $\varphi\varphi^{-1}$ is the identity map on $\mathbb{C}$ and that $\varphi^{-1}\varphi$ is the identity map on $S$. It follows that $\varphi$ is a bijection, and so $\varphi$ is an isomorphism of rings. Hence $S$ is isomorphic to $\mathbb{C}$ as desired.
    \end{proof}
    \item (DF7.3.22) Let $a$ be an element of the ring $R$. \begin{enumerate}[label=\textbf{(\alph*)}]
        \item Prove that $\cbr{x\in R\mid ax = 0}$ is a right ideal and $\cbr{y\in R\mid ya = 0}$ is a left ideal (called respectively the right and left \textit{annihilators} of $a$ in $R$).
        \begin{proof}
            Fix an element $a\in R$. The sets $I = \cbr{x\in R\mid ax = 0}$ and $J = \cbr{y\in R\mid ya = 0}$ are subrings:
            
            Observe $I$ is a nonempty subset of $R$ since $0\in I$; we have $a0=0$. Let $x,y\in I$. Then $a(x-y) = ax - ay = 0-0 = 0$, so that $x-y\in I$. Thus $I$ is closed under subtraction (so $I$ is a subgroup of $R$). Similarly, $a(xy) = (ax)y = 0y = 0$, so that $xy\in I$; we have that $I$ is closed under multiplication. Hence $I$ is a subring of $R$.

            Observe $J$ is a nonempty subset of $R$ since $0\in J$; we have $0a=0$. Let $x,y\in J$. Then $(x-y)a = xa - ya = 0-0 = 0$, so that $x-y\in J$. Thus $J$ is closed under subtraction (so $J$ is a subgroup of $R$). Similarly, $(xy)a = x(ya) = x0 = 0$, so that $xy\in I$; we have that $I$ is closed under multiplication. Hence $I$ is a subring of $R$.

            To show that $I$ is a right ideal of $R$, we check that $I$ is closed under right multiplication by elements of $R$. Let $r\in R$ be arbitrary, and take any element $x\in I$. We have $a(xr) = (ax)r = 0r = 0$, so that $xr\in I$. It follows that $I$ is a right ideal of $R$.
            
            We use an almost identical argument to show that $J$ is a left ideal of $R$: Let $r\in R$ be arbitrary, and take any element $y\in J$. Then $(ry)a = r(ya) = r0 = 0$, so that $ry\in J$. Thus $J$ is closed under left multiplication by elements in $R$, so that $J$ is a left ideal of $R$. 
        \end{proof}
        \item Prove that if $L$ is a left ideal of $R$ then $\cbr{x\in R\mid xa = 0 \text{ for all } a\in L}$ is a two-sided ideal (called the left \textit{annihilator} of $L$ in $R$).
        \begin{proof}
            Let $L$ be a left ideal of $R$ as given. We show that the left annihilator of $L$ in $R$, given by $I = \cbr{x\in R\mid xa = 0 \text{ for all } a\in L}$ is a subring. First, $I$ is a nonempty subset of $R$ since $0\in I$: $0a = 0$ for any $a\in L$. Let $x,y\in I$. Then for any $a\in L$, we have $(x-y)a = xa-ya = 0 - 0 = 0$, so that $x-y\in I$; similarly $(xy)a = x(ya) = 0$, so that $xy\in I$. Hence $I$ is a subring of $R$.

            We check that $I$ is closed under left and right multiplication by elements of $R$. Let $r\in R$ be arbitrary. Then for any $x\in I$ and any $a\in L$, we have $(rx)a = r(xa) = r0 = 0$ and $(xr)a = x(ra) = xa^{\prime} = 0$, where $ra = a^{\prime}\in L$ because $L$ is a left ideal of $R$. It follows that $rx$ and $xr$ are elements of $I$, so that $I$ is closed under left and right multiplication by elements of $R$. 
            
            Hence $I$, the left annihilator of $L$ in $R$, is a two-sided ideal of $R$.
        \end{proof}
    \end{enumerate}
    \item (DF7.3.34) Let $I$ and $J$ be ideals of $R$.\begin{enumerate}[label=\textbf{(\alph*)}]
        \item Prove that $I+J$ is the smallest ideal of $R$ containing both $I$ and $J$.
        \begin{proof}Let $I$ and $J$ be ideals of $R$. 
            
            We check that $I+J = \cbr{a+b\mid a\in I, b\in J}$ is an ideal of $R$. It is clear that $I+J$ is a subring of $R$: We have $0\in I+J$, since $0\in I$ and $0\in J$, and $0+0= 0$. For any $a,a^{\prime}\in I$ and any $b,b^{\prime}\in J$ we have $(a+b) - (a^{\prime} + b^{\prime}) = (a-a^{\prime})+ (b-b^{\prime})\in I+J$ since $a-a^{\prime}\in I$ and $b-b^{\prime}\in J$. We also have $(a+b)(a^{\prime}+b^{\prime}) = a(a^{\prime}+b^{\prime}) + b(a^{\prime} + b^{\prime}) \in I+J$ since $a(a^{\prime}+b^{\prime})\in I$ and $b(a^{\prime} + b^{\prime})\in J$ since $I$ and $J$ are ideals. Thus $I+J$ is a subring of $R$.
            
            For any $r\in R$ we have $r(a+b) = ra+rb \in I+J$, and $(a+b)r = ar+br\in I+J$, since $ar,ra\in I$ and $br,rb\in J$ due to $I$ and $J$ being ideals of $R$. Thus $I+J$ is an ideal of $R$.

            Let $K$ be any ideal of $R$ containing $I$ and $J$. Observe that $K$ is an additive subgroup of $R$; it follows that for any $a\in I$ and any $b\in J$, we have $a,b\in K$, so that $a+b\in K$. Hence $I+J\subseteq K$. Since $K$ was an arbitrary ideal containing $I$ and $J$, it follows that $I+J$ is the smallest ideal of $R$ containing $I$ and $J$.
        \end{proof}
        \item Prove that $IJ$ is an ideal contained in $I\cap J$.
        \begin{proof} Ler $I$ and $J$ be ideals of $R$.

            We check that $IJ = \cbr{\sum_{i=1}^n a_ib_i \mid \text{for any } a\in I, b\in J, n\in \mathbb{Z}^+}$ (set of finite sums of elements of the form $ab$ for $a\in I, b\in J$) is an ideal of $R$. Of course, $0\in I$ and $0\in J$ so that $(0)(0) = 0 \in IJ$. Let $a_1b_1 + \cdots + a_nb_n$ and $a^{\prime}_1b^{\prime}_1+ \cdots + a^{\prime}_mb^{\prime}_m$ ($n,m\in \mathbb{Z}^+$) be elements of $IJ$. Then \[(a_1b_1 + \cdots  +a_nb_n) - (a^{\prime}_1b^{\prime}_1+ \cdots + a^{\prime}_mb^{\prime}_m) = a_1b_1 + \cdots + a_nb_n + (-a^{\prime}_1)b^{\prime}_1 + \cdots + (-a^{\prime}_m)b^{\prime}_m,\] which is clearly an element of $IJ$. Without loss of generality, take $m\leq n$, so that we can write $a^{\prime}_1b^{\prime}_1+ \cdots + a^{\prime}_mb^{\prime}_m = a^{\prime}_1b^{\prime}_1+ \cdots + a^{\prime}_nb^{\prime}_n$ where if $m <n$, $a^{\prime}_i = b^{\prime}_i = 0$ for $m+1\leq i \leq n$ (i.e., add zero terms if needed). Observe that because $I$ and $J$ are ideals, that $a_ib_i\in I$ and $a^{\prime}_ib^{\prime}_i\in J$ for $1\leq i \leq n$. Then \[(a_1b_1 + \cdots + a_nb_n)(a^{\prime}_1b^{\prime}_1+ \cdots + a^{\prime}_nb^{\prime}_n) = \sum_{\substack{1\leq i \leq n \\ 1\leq j \leq n}} (a_ib_i)(a^{\prime}_jb^{\prime}_j)\] is a finite sum of elements of the form required to be in $IJ$. Hence $IJ$ is a subring of $R$.

            For any $r\in R$, we have $r(a_1b_1 + \cdots + a_nb_n) = (ra_1)b_1 + \cdots + (ra_n)b_n \in IJ$ and $(a_1b_1 + \cdots + a_nb_n)r = a_1(b_1 r) + \cdots + a_n(b_n r)\in IJ$ since $r a_i\in I$ and $b_i r\in J$ for $1\leq i \leq n$ due to $I$ and $J$ being ideals of $R$. Hence $IJ$ is an ideal of $R$.

            With $I$ and $J$ being ideals, it follows that for any $a\in I$ and $b\in J$, the element $ab$ can be viewed as an element of $I$ and also as an element of $J$; that is, $ab\in I\cap J$. Therefore, for any element $a_1b_1 + \cdots a_nb_n\in IJ$, viewing every term as an element of $I$ yields that this element is in $I$. Similarly, view every term as an element of $J$ to see that this element is in $J$. Hence $a_1b_1 + \cdots a_nb_n\in I\cap J$, so that $IJ\subseteq I\cap J$.
        \end{proof}
        \item Give an example where $IJ\neq I\cap J$.
        
        In $\mathbb{Z}$, the ideal $(2) = 2\mathbb{Z}$ may be squared to obtain \[(2)(2) = \cbr{\sum_{i=1}^n (2a_i)(2b_i) \mid \text{for any } a_i,b_i\in \mathbb{Z}, n\in \mathbb{Z}^+}\] (finite sums of products of even numbers), but because we can factor out $4$ from these finite sums, we have that $(2)(2) = 4\mathbb{Z}$. But $4\mathbb{Z}$ is properly contained in $2\mathbb{Z}\cap 2\mathbb{Z} = 2\mathbb{Z}$ (as $2\not\in 4\mathbb{Z}$, but every multiple of $4$ is divisible by $2$). 

        (It is clear that $2\mathbb{Z}$ is an ideal: We have that $2\mathbb{Z}$ is a subgroup of $\mathbb{Z}$, is a subring of $\mathbb{Z}$ since products of even integers are even, and is an ideal of $\mathbb{Z}$ since the product of an even integer with any other integer is also even.) 
        \item Prove that if $R$ is commutative and if $I+J = R$ then $IJ = I\cap J$. (Note that $R$ contains $1$ as a global assumption.)
        \begin{proof}
            Let $R$ be commutative and let $I,J$ be ideals of $R$ with $I+J = R$. The containment $IJ\subseteq I\cap J$ follows from a previous result. We show that $I\cap J\subseteq IJ$. To that end, take any element $c\in I\cap J$, so that $c\in I$ and $c\in J$.
            
            Since $R$ contains $1$, it follows that there are elements $a\in I$ and $b\in J$ with $a+b = 1$. Then \[c = c(a+b) = ca + cb = ac + cb,\] which is a finite sum, and with $a,c\in I$ and $c,b\in J$, we have that $c = ac+cb\in IJ$.
            
            Thus $I\cap J\subset IJ$, from which it follows that $IJ = I\cap J$.
        \end{proof}
    \end{enumerate}
\end{enumerate}
\end{document}