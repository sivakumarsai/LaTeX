\documentclass[11pt]{article}
\headheight = 14pt
% packages
\usepackage{physics}
% margin spacing
\usepackage[top=1in, bottom=1in, left=0.5in, right=0.5in]{geometry}
\usepackage{hanging}
\usepackage{amsfonts, amsmath, amssymb, amsthm}
\usepackage{systeme}
\usepackage[none]{hyphenat}
\usepackage{fancyhdr}
\usepackage[nottoc, notlot, notlof]{tocbibind}
\usepackage{graphicx}
\graphicspath{{./images/}}
\usepackage{float}
\usepackage{siunitx}
\usepackage{esint}
\usepackage{cancel}
\usepackage{enumitem}

% colors
\usepackage{xcolor}
\definecolor{p}{HTML}{FFDDDD}
\definecolor{g}{HTML}{D9FFDF}
\definecolor{y}{HTML}{FFFFCF}
\definecolor{b}{HTML}{D9FFFF}
\definecolor{o}{HTML}{FADECB}
%\definecolor{}{HTML}{}

% \highlight[<color>]{<stuff>}
\newcommand{\highlight}[2][p]{\mathchoice%
  {\colorbox{#1}{$\displaystyle#2$}}%
  {\colorbox{#1}{$\textstyle#2$}}%
  {\colorbox{#1}{$\scriptstyle#2$}}%
  {\colorbox{#1}{$\scriptscriptstyle#2$}}}%

% header/footer formatting
\pagestyle{fancy}
\fancyhead{}
\fancyfoot{}
\fancyhead[L]{MAP4413 Dr. Zhang}
\fancyhead[C]{HW5}
\fancyhead[R]{Sai Sivakumar}
\fancyfoot[R]{\thepage}
\renewcommand{\headrulewidth}{1pt}

% paragraph indentation/spacing
\setlength{\parindent}{0cm}
\setlength{\parskip}{5pt}
\renewcommand{\baselinestretch}{1.25}

% extra commands defined here
\newcommand{\ihat}{\boldsymbol{\hat{\textbf{\i}}}}
\newcommand{\jhat}{\boldsymbol{\hat{\textbf{\j}}}}
\newcommand{\dr}{\vec{r}~^{\prime}(t)}
\newcommand{\dx}{x^{\prime}(t)}
\newcommand{\dy}{y^{\prime}(t)}

\newcommand{\br}[1]{\left(#1\right)}
\newcommand{\sbr}[1]{\left[#1\right]}
\newcommand{\cbr}[1]{\left\{#1\right\}}

\newcommand{\dprime}{\prime\prime}
\newcommand{\lap}[2]{\mathcal{L}[#1](#2)}

% bracket notation for inner product
\usepackage{mathtools}

\DeclarePairedDelimiterX{\abr}[1]{\langle}{\rangle}{#1}

\DeclareMathOperator{\Span}{span}
\DeclareMathOperator{\nullity}{nullity}
\DeclareMathOperator\Arg{Arg}
\DeclareMathOperator\Log{Log}


% set page count index to begin from 1
\setcounter{page}{1}

\begin{document}
p.120: 1, 2; p.161 2, 3, 5, 7

p.120

1. Let $\gamma\colon [a,b]\to \mathbb{R}^2$ be a parameterization for the closed curve $\Gamma$.
\begin{enumerate}[label=(\alph*)]
    \item Prove that $\gamma$ is a parameterization by arc-length if and only if the length of the curve from $\gamma(a)$ to $\gamma(s)$ is precisely $s-a$, that is,
    \[\int_a^s \abs{\gamma^{\prime}(t)}\dd{t} = s-a.\]
    \begin{proof}
      First suppose that $\gamma$ is a parameterization by arc-length. Then $\abs{\gamma^{\prime}(s)} = 1$ for all $s$, so that \[\int_a^s \abs{\gamma^{\prime}(t)}\dd{t} = \int_a^s 1 \dd{t} = s-a.\]
      Conversely, suppose that $\int_a^s \abs{\gamma^{\prime}(t)}\dd{t} = s-a$. We know that $\gamma$ is of class $C^1$ and $\gamma^{\prime}(t)\neq 0$ for all $t$, so that $\abs{\gamma^{\prime}}$ is continuous and positive. Furthermore, $s-a =\int_a^s 1 \dd{t}$, so we must have that $0 < \abs{\gamma^{\prime}(t)} \leq 1$. If $\abs{\gamma^{\prime}}$ is less than $1$ at any point $p\in [a,s]$, because $\abs{\gamma^{\prime}}$ is continuous we may find a $\delta > 0$ small enough so that the integration outside of the $\delta$-neighborhood of $p$ \[\int_{[a,p-\delta)\cup (p+\delta, s)}\abs{\gamma^{\prime}(t)}\dd{t}\]
      is bounded above by $s-a-2\delta$. Then the integration $\int_{[p-\delta, p+\delta]}\abs{\gamma^{\prime}(t)}\dd{t}$ is strictly less than $2\delta$, so the total integration over $[a,s]$ is less than $s-a$. This is a contradiction, so we must have that $\abs{\gamma^{\prime}(t)} = 1$; that is, $\gamma$ is a parameterization by arc-length.
    \end{proof}
    \item Prove that any curve $\Gamma$ admits a parameterization by arc-length. 
    \begin{proof}
      Let $\Gamma$ be any curve and let $\eta$ be any $C^1$ parameterization of $\Gamma$ where $\eta^{\prime}(t)\neq 0$. Then let $h(s) = \int_a^s\abs{\eta^{\prime}(t)}\dd{t}$, so that the composition $\gamma = \eta \circ h^{-1}$ is differentiable. Then by directly computing, we have \[\abs{\gamma^{\prime}(t)} = \abs{(h^{-1})^{\prime}(t)\cdot \dv{\eta(h^{-1}(t))}{h^{-1}(t)}} = \abs{ \br{ \dv{h(h^{-1}(t))}{h^{-1}(t)} }^{-1} \cdot \dv{\eta(h^{-1}(t))}{h^{-1}(t)}} = \abs{\dv{\eta(h^{-1}(t))}{h^{-1}(t)}}^{-1}\cdot \abs{\dv{\eta(h^{-1}(t))}{h^{-1}(t)}},\] which means that $\abs{\gamma^{\prime}(t)} = 1$ for all $t$. This means that $\gamma$ is a parameterization by arc-length, which means $\Gamma$ admits a parameterization by arc-length.
    \end{proof}
\end{enumerate} 

2. Suppose $\gamma\colon [a,b]\to \mathbb{R}^2$ is a parameterization for a closed curve $\Gamma$, with $\gamma(t) = (x(t),y(t))$.
\begin{enumerate}[label=(\alph*)]
  \item Show that \[\frac{1}{2}\int_a^b (x(s)y^{\prime}(s) - y(s)x^{\prime}(s))\dd{s} = \int_a^b x(s)y^{\prime}(s)\dd{s} = - \int_a^b y(s)x^{\prime}(s)\dd{s}\]
  \begin{proof}
    Using integration by parts, we have \begin{align*}
      \int_a^b x(s)y^{\prime}(s)\dd{s} &= \eval{y(t)x(t)}_a^b - \int_a^b y(s)x^{\prime}(s)\dd{s}\\
      &= - \int_a^b y(s)x^{\prime}(s)\dd{s},
    \end{align*}
    where we used the fact that $\Gamma$ was a closed curve.

    Thus \[\frac{1}{2}\int_a^b (x(s)y^{\prime}(s) - y(s)x^{\prime}(s))\dd{s} = 2\cdot \frac{1}{2}\int_a^b x(s)y^{\prime}(s)\dd{s} = 2\cdot \frac{-1}{2} \int_a^b y(s)x^{\prime}(s)\dd{s}.\]
  \end{proof}
  \item Define the \textbf{reverse parameterization} of $\gamma$ by $\gamma^{-}\colon [a,b]\to \mathbb{R}^2$ with $\gamma^{-}(t) = \gamma(b+a-t)$. The image of $\gamma^{-}$ is precisely $\Gamma$, except that the points $\gamma^{-}(t)$ and $\gamma(t)$ travel in opposite directions. Thus $\gamma^{-}$ ``reverses'' the orientation of the curve. Prove that \[\int_{\gamma}(x\dd{y}-y\dd{x}) = -\int_{\gamma^{-}} (x\dd{y}-y\dd{x}).\]
  In particular, we may assume (after a possible change in orientation) that \[\mathcal{A} = \frac{1}{2}\int_a^b (x(s)y^{\prime}(s) - y(s)x^{\prime}(s))\dd{s} =\int_a^b x(s)y^{\prime}(s)\dd{s}.\]
  \begin{proof}
    Changing variables from $b+a-t\mapsto t$ and $-\dd{t}\mapsto \dd{t}$, we have
    \begin{align*}
      \int_{\gamma} (x\dd{y}-y\dd{x}) &= \int_a^b (x(t)y^{\prime}(t)\dd{t} - y(t)x^{\prime}(t)\dd{t})\\
      &= -\int_{b+a-a}^{b+a-b} (x(b+a-t)y^{\prime}(b+a-t)\dd{t} - y(b+a-t)x^{\prime}(b+a-t)\dd{t})\\
      &= -\int_{\gamma^{-}} (x\dd{y}-y\dd{x})
    \end{align*}
    where we used the fact that $\gamma^{-}(t) = \gamma(b+a-t)$.
  \end{proof}
\end{enumerate}

p.161

2. Let $f$ and $g$ be the functions defined by \[f(x) = \chi_{[-1,1]}(x) = \begin{cases}
  1 & \text{if } \abs{x}\leq 1,\\
  0 & \text{otherwise,}
\end{cases} \quad \text{and}\quad g(x) = \begin{cases}
  1-\abs{x} & \text{if } \abs{x}\leq 1,\\
  0 & \text{otherwise},
\end{cases}\]
Although $f$ is not continuous, the integral defining its Fourier series still makes sense. Show that \[\hat{f}(\xi) = \frac{\sin(2\pi\xi)}{\pi\xi}\quad \text{and}\quad \hat{g}(\xi) = \br{\frac{\sin(\pi\xi)}{\pi\xi}}^2,\] with the understanding that $\hat{f}(0) = 2$ and $\hat{g}(0) = 1$.
\begin{proof}
  Direct computation of the Fourier transforms yield \begin{align*}
    \hat{f}(\xi) = \int_{-\infty}^{\infty} \chi_{[-1,1]}(x) e^{-2\pi i x \xi}\dd{x} &= \int_{-1}^{1} e^{-2\pi i x \xi}\dd{x}\\
    &= \eval{\frac{e^{-2\pi i x \xi}}{-2\pi i \xi }}_{-1}^{1}\\
    &= \frac{\sin(2\pi\xi)}{\pi\xi}
  \end{align*}
  and \begin{align*}
    \hat{g}(\xi) = \int_{-\infty}^{\infty} g(x)e^{-2\pi i x \xi}\dd{x} &= \int_{-1}^1 (1-\abs{x})e^{-2\pi i x \xi}\dd{x}\\
    &= 2\int_0^1 (1-x)\cos(2\pi x \xi)\dd{x}\\
    &= \frac{1-\cos(2\pi\xi)}{2\pi^2\xi^2} = \br{\frac{\sin(\pi\xi)}{\pi\xi}}^2
  \end{align*}
  with $\hat{f}(0) = \lim_{\xi\to 0} \frac{\sin(2\pi\xi)}{\pi\xi} = 2$, and $\hat{g}(0) = \lim_{\xi\to 0} \br{\frac{\sin(\pi\xi)}{\pi\xi}}^2 = 1$.
\end{proof}

3. The following exercise illustrates the principle that the decay of $\hat{f}$ is related to the continuity properties of $f$.
\begin{enumerate}[label=(\alph*)]
  \item Suppose that $f$ is a function of moderate decrease on $\mathbb{R}$ whose Fourier transform $\hat{f}$ is continuous and satisfies \[\hat{f}(\xi) = O\br{\frac{1}{\abs{\xi}^{1+\alpha}}} \quad \text{as }\abs{\xi}\to \infty\] for some $0< \alpha < 1$. Prove that $f$ satisfies a H\"older condition of order $\alpha$, that is, that \[\abs{f(x+h) - f(x)}\leq M\abs{h}^{\alpha} \quad \text{for some $M > 0$ and all $x,h\in \mathbb{R}$.}\]
  \begin{proof}
    Use the Fourier inversion formula to write \[f(x+h) - f(x) = \int_{-\infty}^{\infty} \hat{f}(\xi) e^{2\pi i x \xi}\left(e^{2\pi i h \xi} - 1\right)\dd{\xi},\] so that
    \begin{align*}
      \abs{f(x+h) - f(x)}\abs{h}^{-\alpha} &\leq \abs{h}^{-\alpha}\abs{\int_{-\infty}^{\infty} \hat{f}(\xi) e^{2\pi i x \xi}\left(e^{2\pi i h \xi} - 1\right)\dd{\xi}}\\
      &\leq \int_{-\infty}^{\infty} \frac{C\abs{h}^{-\alpha}}{1 + \abs{\xi}^{1+\alpha}}\abs{2 i e^{\pi i h \xi}}\abs{\frac{e^{\pi i h \xi} - e^{-\pi i h \xi}}{2i}}\dd{\xi}\\
      &\leq \int_{-\infty}^{\infty} \frac{2C\abs{\sin(\pi h \xi)}}{\abs{h}^{\alpha} + \abs{h}^{-1} \abs{h\xi}^{1+\alpha}}\dd{\xi}\\
      &\leq \frac{2C}{\pi^{1+\alpha}}\int_{-\infty}^{\infty}\frac{\abs{\sin(t)}}{\abs{h}^{1+\alpha} + \abs{t}^{1+\alpha}}\dd{t}\\
      &\leq \frac{4C}{\pi}\int_{0}^{\infty} \frac{\abs{\sin(t)}}{t^{1+\alpha}}\dd{t}\\
      &\leq \frac{4C}{\pi}\int_{0}^{\infty} \frac{1}{t^{1+\alpha}}\dd{t}\\
      &\leq M,
    \end{align*} so that $\abs{f(x+h) - f(x)}\leq M\abs{h}^{\alpha}$.
  \end{proof}
  \item Let $f$ be a continuous function on $\mathbb{R}$ which vanishes for $\abs{x}\geq 1$, with $f(0) = 0$ and which is equal to $1/\log(1/\abs{x})$ for all $x$ in a neighborhood of the origin. Prove that $\hat{f}$ is not of moderate decrease. In fact there is no $\varepsilon >0$ so that $\hat{f}(\xi) = O(1/\abs{\xi}^{1+\varepsilon})$ as $\abs{\xi}\to \infty$.
  \begin{proof}
    Investigating $f(0) = 0$ and $f(h) = 1/\log(h^{-1})$, we have that $\abs{f(h) - f(0)}/\abs{h}^\alpha = 1/(\abs{h}^\alpha\log(h))$, but for any fixed $\alpha$ we can choose $h$ as small as we like so that this quantity becomes unbounded. So by the contrapositive to part (a), we should not have that $f$ is of moderate decrease.
  \end{proof}
\end{enumerate}

5. Suppose $f$ is continuous and of moderate decrease. 
\begin{enumerate}[label=(\alph*)]
  \item Prove that $\hat{f}$ is continuous and $\hat{f}(\xi) \to 0$ as $\abs{\xi}\to \infty$.
  \begin{proof}
    We have \begin{align*}
      \abs{\hat{f}(\xi+h) - \hat{f}(\xi)} &= \abs{\int_{-\infty}^{\infty} f(x)e^{-2\pi i x \xi}(e^{-2\pi i x h} - 1)\dd{x}}\\
      &\leq \int_{-\infty}^{\infty} \abs{f(x)}\abs{-2ie^{-\pi i x h}}\abs{\frac{e^{\pi i x h} - e^{-\pi i x h}}{2i}}\dd{x}\\
      &\leq \int_{-\infty}^{\infty} \frac{2C\abs{\sin(\pi x h)}}{1+x^2}\dd{x}.
    \end{align*}

    Then for any $\varepsilon >0$, we may choose $K$ large enough so that $\int_{\abs{x}> K}\frac{2C\abs{\sin(\pi x h)}}{1+x^2}\dd{x} < \varepsilon /2 $ and choose $\delta$ small enough so that for $\abs{h} < \delta$ we have $\int_{\abs{x} \leq K}\frac{2C\abs{\sin(\pi x h)}}{1+x^2}\dd{x} < \varepsilon /2 $ since $\sin(\pi x h)$ can be made as small as we like if we take $h$ to be small. With this choice of $\delta$ (which depended on $K$) we have $\abs{\hat{f}(\xi+h) - \hat{f}(\xi)} < \varepsilon$, so that $\hat{f}$ is continuous.

    Then observe that
    \begin{align*}
      \frac{1}{2}\int_{-\infty}^{\infty} [f(x)-f(x-1/(2\xi))]e^{-2\pi i x \xi}\dd{x} &= \frac{1}{2}\hat{f}(\xi) + \frac{1}{2} \int_{-\infty}^{\infty}-f(x-1/(2\xi))e^{-2\pi i x \xi}\dd{x}\\
      &= \frac{1}{2}\hat{f}(\xi) + \int_{-\infty}^{\infty} -f(x)e^{-2\pi i x \xi}e^{-\pi i}\dd{x} \\
      &= \frac{1}{2}\hat{f}(\xi) +\frac{1}{2}\hat{f}(\xi) = \hat{f}(\xi) .
    \end{align*}

    Then as $\abs{\xi}\to \infty$, we have $x-1/(2\xi) \to x$. So using the Lebesgue dominated convergence theorem (as $f(x-1/(2\xi))$ tends to $f(x)$ and $f(x-1/(2\xi))$ is dominated above by $f(x)+C$ for large enough $C$), \begin{align*}
      \abs{\hat{f}(\xi)} &\leq \int_{-\infty}^{\infty} \abs{f(x)-f(x-1/(2\xi))}\dd{x}\\
      &\leq 0 \quad \text{as } \abs{\xi}\to \infty.
    \end{align*}
    Hence $\hat{f}$ is continuous and $\hat{f}(\xi) \to 0$ as $\abs{\xi}\to \infty$.
  \end{proof}
  \item Show that if $\hat{f}(\xi) = 0$ for all $\xi$, then $f$ is identically zero.
  \begin{proof}
    Let $\hat{f}(\xi) = 0$ for all $\xi$. 
    
    In general, by interchanging the order of integration whenever $g\in \mathcal{S}(\mathbb{R})$, we have \[\int_{-\infty}^{\infty} f(x)\hat{g}(x)\dd{x} = \int_{-\infty}^{\infty} f(x)\int_{-\infty}^{\infty}g(y)e^{-2\pi i y x}\dd{y}\dd{x} = \int_{-\infty}^{\infty} g(y)\int_{-\infty}^{\infty}f(x)e^{-2\pi i y x}\dd{x}\dd{y} = \int_{-\infty}^{\infty} \hat{f}(y)g(y)\dd{y}.\]
    The Gauss kernel $K_\delta(t-x)$ viewed as a function of $x$ is in the Schwartz space, so it has a preimage $g(x) \in \mathcal{S}(\mathbb{R})$ under the Fourier transformation. We have for all $\delta > 0$ and any $t$ that
    \[0 = \int_{-\infty}^{\infty}\hat{f}(x)g(x)\dd{x} = \int_{-\infty}^{\infty}f(x)K_{\delta}(t-x)\dd{x},\] and because the Gauss kernel is a good kernel, as $\delta\to 0$, we have that the integral on the right converges uniformly to $f(t)$. So for any $t$, $f(t) \equiv 0$.
  \end{proof}
\end{enumerate}

7. Prove that the convolution of two functions of moderate decrease is a function of moderate decrease.
\begin{proof}
  Let $f,g$ be functions of moderate decrease. Then \begin{align*}
    \abs{f\ast g} = \abs{\int_{-\infty}^{\infty} f(x-y)g(y)\dd{y}} &\leq \int_{\abs{y}\leq \abs{x}/2} \abs{f(x-y)}\abs{g(y)}\dd{y} + \int_{\abs{y}\geq \abs{x}/2} \abs{f(x-y)}\abs{g(y)}\dd{y}\\
    &\leq \int_{\abs{y}\leq \abs{x}/2} \frac{C_1\abs{g(y)}}{1 + (x-y)^2}\dd{y} + \int_{\abs{y}\geq \abs{x}/2} \frac{C_2\abs{f(x-y)}}{1 + y^2}\dd{y}\\
    &\leq \int_{\abs{y}\leq \abs{x}/2} \frac{C_1\abs{g(y)}}{1 + (x/2)^2}\dd{y} + \int_{\abs{y}\geq \abs{x}/2} \frac{C_2\abs{f(x-y)}}{1 + (x/2)^2}\dd{y}\\
    &\leq \frac{4C_1}{4+x^2}A + \frac{4C_2}{4+x^2}B\\
    &\leq \frac{C_3}{1+ x^2}.
  \end{align*}
  The convolution is an integral so it is continuous. Thus $f\ast g$ is of moderate decrease.
\end{proof}
\newpage
5. Suppose $f$ is continuous and of moderate decrease. 
\begin{enumerate}[label=(\alph*)]
  \item Prove that $\hat{f}$ is continuous and $\hat{f}(\xi) \to 0$ as $\abs{\xi}\to \infty$.
  \begin{proof}
    We have \begin{align*}
      \abs{\hat{f}(\xi+h) - \hat{f}(\xi)} &= \abs{\int_{-\infty}^{\infty} f(x)e^{-2\pi i x \xi}(e^{-2\pi i x h} - 1)\dd{x}}\\
      &\leq \int_{-\infty}^{\infty} \abs{f(x)}\abs{-2ie^{-\pi i x h}}\abs{\frac{e^{\pi i x h} - e^{-\pi i x h}}{2i}}\dd{x}\\
      &\leq \int_{-\infty}^{\infty} \frac{2C\abs{\sin(\pi x h)}}{1+x^2}\dd{x}.
    \end{align*}

    Then for any $\varepsilon >0$, we may choose $K$ large enough so that $\int_{\abs{x}> K}\frac{2C\abs{\sin(\pi x h)}}{1+x^2}\dd{x} < \varepsilon /2 $ and choose $\delta$ small enough so that for $\abs{h} < \delta$ we have $\int_{\abs{x} \leq K}\frac{2C\abs{\sin(\pi x h)}}{1+x^2}\dd{x} < \varepsilon /2 $ since $\sin(\pi x h)$ can be made as small as we like if we take $h$ to be small. With this choice of $\delta$ (which depended on $K$) we have $\abs{\hat{f}(\xi+h) - \hat{f}(\xi)} < \varepsilon$, so that $\hat{f}$ is continuous.

    Then observe that
    \begin{align*}
      \frac{1}{2}\int_{-\infty}^{\infty} [f(x)-f(x-1/(2\xi))]e^{-2\pi i x \xi}\dd{x} &= \frac{1}{2}\hat{f}(\xi) + \frac{1}{2} \int_{-\infty}^{\infty}-f(x-1/(2\xi))e^{-2\pi i x \xi}\dd{x}\\
      &= \frac{1}{2}\hat{f}(\xi) + \int_{-\infty}^{\infty} -f(x)e^{-2\pi i x \xi}e^{-\pi i}\dd{x} \\
      &= \frac{1}{2}\hat{f}(\xi) +\frac{1}{2}\hat{f}(\xi) = \hat{f}(\xi) .
    \end{align*}

    Then as $\abs{\xi}\to \infty$, we have $x-1/(2\xi) \to x$. So using the Lebesgue dominated convergence theorem (as $f(x-1/(2\xi))$ tends to $f(x)$ and $f(x-1/(2\xi))$ is dominated above by $f(x)+C$ for large enough $C$), \begin{align*}
      \abs{\hat{f}(\xi)} &\leq \int_{-\infty}^{\infty} \abs{f(x)-f(x-1/(2\xi))}\dd{x}\\
      &\leq 0 \quad \text{as } \abs{\xi}\to \infty.
    \end{align*}
    Hence $\hat{f}$ is continuous and $\hat{f}(\xi) \to 0$ as $\abs{\xi}\to \infty$.
  \end{proof}
\end{enumerate}
\end{document}