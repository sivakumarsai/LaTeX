\documentclass[12pt]{amsart}

\textwidth = 6.2 in
\textheight = 8.5 in
\oddsidemargin = 0.0 in
\evensidemargin = 0.0 in
\topmargin = 0.0 in
\headheight = 0.0 in
\headsep = 0.3 in
\parskip = 0.05 in
\parindent = 0.3 in

\usepackage{enumerate}
\usepackage{amsmath}
\usepackage{color}
\def\cc{\color{blue}}
\usepackage[normalem]{ulem}
\usepackage{amsfonts, amsmath, amssymb, amsthm}
\usepackage{systeme}
\usepackage[none]{hyphenat}
\usepackage{graphicx}
\graphicspath{{./images/}}
\usepackage{esint}
\usepackage{cancel}
\usepackage{physics}

\title{Homework 5}
\author{Sai Sivakumar}

\newtheorem{theorem}            {Theorem}[section]
\newtheorem{proposition}        [theorem]{Proposition}

\begin{document}
\maketitle

  Define a sequence from $\mathbb R$ as follows.
  Fix $r>1$. Let $a_1=1$ and define recursively, 
\[
  a_{n+1} = \frac{1}{r}(a_n +r +1).
\]
 Show $(a_n)$ converges and find its limit.
[Suggestion: Show $(a_n)$ is bounded above by $\frac{r+1}{r-1}.$]
Note: I worked with Jude Flynn, Nicholas Kapsos, and Silas Rickards to work out the main ideas of this proof.

\begin{proof}
    Let $(a_n)$ be a recursively defined sequence as given. To show that $(a_n)$ converges, it suffices to show that the sequence is increasing and that it is bounded as well.

    To see that the sequence is increasing, we show by induction that for all $n\in\mathbb{N}$, we have $a_{n} \geq a_{n-1}$. Observe for $n=1$, we have $a_2 = \frac{1}{r}(a_1 +r+1) = \frac{1}{r}(r+2) = 1 + \frac{2}{r} \geq 1 = a_1$, since $a_1 = 1$ and we fixed $r>1$. Then suppose that $a_n \geq a_{n-1}$, and see that \begin{align*}
        a_{n+1} - a_n &= \frac{1}{r}(a_n+r+1) - \frac{1}{r}(a_{n-1}+r+1) \\
        &= \frac{1}{r}(a_n-a_{n-1}) \\
        &\geq 0,
    \end{align*} because $r>0$ and $a_n \geq a_{n-1}$. With $a_{n+1} - a_n\geq 0$, we have by induction that for $n\in \mathbb{N}$, $a_{n} \geq a_{n-1}$. Hence $(a_n)$ is an increasing sequence.

    The sequence $(a_n)$ is bounded above by $\frac{r+1}{r-1}$, which we show by induction. For $n=1$, $a_n = 1 \leq \frac{r+1}{r-1}$, because $r>1$. Then suppose that $a_n \leq \frac{r+1}{r-1}$, and we have that \begin{align*}
        a_{n+1} &= \frac{1}{r}(a_n+r+1) \\
        &\leq \frac{1}{r}\left(\frac{r+1}{r-1} + r+ 1\right) \\
        &= \frac{r+1}{r}\cdot \frac{r}{r-1} = \frac{r+1}{r-1}.
    \end{align*} Hence all terms $a_n$ are bounded above by $\frac{r+1}{r-1}$, which means the sequence $(a_n)$ is bounded above by $\frac{r+1}{r-1}$. 

    Since $(a_n)$ is an increasing and bounded sequence, $(a_n)$ converges to a unique real number. Because $(a_n)$ converges, it is a Cauchy sequence. This means that for all $\varepsilon>0$, there exists $N\in\mathbb{N}$ such that for $n>N$, $\abs{a_{n+1} - a_n} < \left(\frac{r-1}{r}\right)\varepsilon$. Since $a_{n+1} \geq a_n$ and for all $n\in\mathbb{N}$, $a_n \geq a_1 = 1$, we may omit the absolute value signs. Then 
    \[ \frac{1}{r}(a_n+r+1) = a_{n+1} \leq \left(\frac{r-1}{r}\right)\varepsilon + a_n,\]
    which with $r>1$ and some algebra, we have that 
    \[r+1 \leq (r-1)\varepsilon + (r-1)a_n,\] which implies that
    \[\frac{r+1}{r-1} - \varepsilon \leq a_n.\]
    But we showed earlier that an upper bound for the sequence $(a_n)$ was $\frac{r+1}{r-1}$, which we may increment by $\varepsilon$ since $\varepsilon >0$. Hence \[\frac{r+1}{r-1} - \varepsilon \leq a_n\leq \frac{r+1}{r-1} + \varepsilon,\] which means that for all $\varepsilon>0$, there exists $N\in\mathbb{N}$ such that for all $n>N$, \[\abs{a_n-\frac{r+1}{r-1}} \leq \varepsilon.\] It follows that $\frac{r+1}{r-1}$ is the limit of the sequence $(a_n)$.
\end{proof}
\end{document}