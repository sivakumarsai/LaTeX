\documentclass[12pt]{amsart}

\textwidth = 6.2 in
\textheight = 8.5 in
\oddsidemargin = 0.0 in
\evensidemargin = 0.0 in
\topmargin = 0.0 in
\headheight = 0.0 in
\headsep = 0.3 in
\parskip = 0.05 in
\parindent = 0.3 in

\usepackage{enumerate}
\usepackage{amsmath}
\usepackage{color}
\def\cc{\color{blue}}
\usepackage[normalem]{ulem}
\usepackage{amsfonts, amsmath, amssymb, amsthm}
\usepackage{systeme}
\usepackage[none]{hyphenat}
\usepackage{graphicx}
\graphicspath{{./images/}}
\usepackage{esint}
\usepackage{cancel}
\usepackage{physics}

\title{Homework 13}
\author{Sai Sivakumar}

\newtheorem{theorem}            {Theorem}[section]
\newtheorem{proposition}        [theorem]{Proposition}

\newcommand{\RR}{\mathbb{R}}
\newcommand{\NN}{\mathbb{N}}
\newcommand{\QQ}{\mathbb{Q}}

\begin{document}
\maketitle

Recall the function  $t:[0,1]\to \RR$ defined by
\[
  t(x) =\begin{cases} %0 & \mbox{ if } x=0 \\ 
   \frac{1}{q} & \mbox{ if } x\in \QQ \cap (0,1], \ \ 
         x=\frac{p}{q}, \ \ p,q\in \NN, \mbox{ and } \gcd(p,q)=1 \\
     0& \mbox{ otherwise}.
\end{cases}
\] 
from the course notes. Abbott refers to it as Thomae's function.

For $k\in \NN,$ define $f_k:[0,1]\to\RR$ by 
\[
  f_k(x) =\begin{cases} %0 & \mbox{ if } x=0 \\ 
   \frac{1}{q} & \mbox{ if }  x\in \QQ\cap (0,1], \ \ 
         x=\frac{p}{q}, \ \ p,q\in \NN, \ \ \gcd(p,q)=1 
  \mbox{ and } q\le k \\
     0& \mbox{ otherwise. } 
\end{cases}
\]

\bigskip

\begin{enumerate}[(a)] \itemsep=8pt
  \item Prove, if $g:[a,b]\to\RR$ is $0$ except 
  on a finite set $G\subseteq [a,b],$ then
 $g$ is Riemann integrable and 
\[
 \int_a^b g\, dx =0.
\]
%[Suggestion: Induct on the number of elements in $G$ and use Theorem 7.4.1 in Abbott.]
 \item Show $(f_k)$ converges uniformly to $t.$ 
 \item  Conclude $t$ is
 Riemann integrable and
\[
 \int_0^1 t\, dx =0.
\]
\end{enumerate}

\bigskip

\begin{proof} Let $t\colon [0,1]\to \mathbb{R}$ and $f_k\colon [0,1]\to\mathbb{R}$ for $k\in \mathbb{N}$ be as given.

\begin{enumerate}[(a)]
    \item With $g\colon[a,b]\to \mathbb{R}$ zero everywhere except on a finite set $G\subseteq [a,b]$, we show by induction on $n = \abs{G}$ that $g$ is Riemann integrable.
    
    Consider the case with one discontinuity ($\abs{G} = 1$) at the point $a < c_1 < b$ (if $g$ is not continuous at $a$ or $b$, $g$ is still integrable). Then the restrictions of $g$ given by $g|_{[a,c_1]}\colon [a,c_1]\to \mathbb{R}$ and $g|_{[c_1, b]}\colon [c_1,b]\to \mathbb{R}$ are both integrable on their domains. This follows since $g|_{[a,c_1]}$ and $g|_{[c_1, b]}$ are bounded; furthermore, for every $a< x< c_1$ the function $g|_{[a,c_1]}$ is integrable on $[a,x]$ and for every $c_1< y < b$ the function $g|_{[c_1, b]}$ is integrable on $[y,b]$. 

    It follows that $g$ is integrable. Thus suppose that $g$ is integrable when $\abs{G} = n$ and add one more point $c_{n+1}$ to $G$. Then repeat the same argument as above in the case for one discontinuity, with $c_{n+1}$ in place of $c_1$. Thus $g$ is still integrable after including $c_{n+1}$ in $G$. Hence by induction $g$ is integrable if the set of points on which $g$ is discontinuous is finite.

    We estimate the integral $\int_a^b g\dd{x}$. Let $\varepsilon > 0$ be given. With $\abs{G} = n > 0$ (if $\abs{G} = 0$ then the integral $\int_a^b g\dd{x}$ is automatically zero), write \[G = \{c_i\colon \text{$g$ is discontinuous at $c_i$ for $1\leq i \leq n$}\}.\] Then let $m$ and $M$ be the infimum and supremum of the set $\{g(x) \colon x\in [a,b]\}$, and form the partition $P = \{a, c_1-\varepsilon/n, c_1+\varepsilon/n, \dots, c_n-\varepsilon/n, c_1+\varepsilon/n, b \}$. Then \[2m\varepsilon = n\cdot 2m\varepsilon/n \leq L(g,P)\leq \int_a^b g \dd{x} \leq U(g,P) \leq n\cdot M\varepsilon/n = 2M\varepsilon.\] 
    Since $\varepsilon> 0$ was arbitrary, it follows that $\int_a^b g\dd{x} = 0$.
    \item Let $\varepsilon> 0$ be given. Let $K$ be a natural number strictly larger than $1/\varepsilon$, and for $k> K$, it follows that \begin{align*}
        \abs{t(x) - f_k(x)} &\leq \abs{t(x) - f_K(x)} + \abs{f_K(x) - f_k(x)}\\
        &\leq \frac{1}{K+1} + \frac{1}{k+1}\\
        &\leq \frac{2}{K} < 2\varepsilon.
    \end{align*} Thus $(f_k)$ converges uniformly to $t$.
    \item Every function $f_k$ for $k\in \mathbb{N}$ has finitely many discontinuities, so each $f_k$ is integrable on $[0,1]$. Then because $(f_k)$ converges uniformly to $t$, it follows that $t$ is also integrable. Furthermore, the sequence $(\int_0^1 f_k\dd{x})$ converges to $\int_0^1 t\dd{x}$. But because $f_k$ for every $k\in \mathbb{N}$ is zero except at its (finitely many) points of discontinuity, $(\int_0^1 f_k\dd{x})$ is the zero sequence, which converges to zero. Hence $\int_0^1 t\dd{x} = 0$.
\end{enumerate}
\end{proof}
\end{document}