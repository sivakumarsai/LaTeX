\documentclass[12pt]{amsart}

\textwidth = 6.2 in
\textheight = 8.5 in
\oddsidemargin = 0.0 in
\evensidemargin = 0.0 in
\topmargin = 0.0 in
\headheight = 0.0 in
\headsep = 0.3 in
\parskip = 0.05 in
\parindent = 0.3 in

\usepackage{enumerate}
\usepackage{amsmath}
\usepackage{color}
\def\cc{\color{blue}}
\usepackage[normalem]{ulem}
\usepackage{amsfonts, amsmath, amssymb, amsthm}
\usepackage{systeme}
\usepackage[none]{hyphenat}
\usepackage{graphicx}
\graphicspath{{./images/}}
\usepackage{esint}
\usepackage{cancel}
\usepackage{physics}

\title{Homework 10}
\author{Sai Sivakumar}

\newtheorem{theorem}            {Theorem}[section]
\newtheorem{proposition}        [theorem]{Proposition}

\newcommand{\RR}{\mathbb{R}}

\begin{document}
\maketitle

This homework consists of two problems.

\begin{enumerate}[(A)] \itemsep=10pt
 \item Suppose $f:\mathbb{R}\to\mathbb{R}$ is differentiable. Show, if
 there is an $M$ such that $|f^\prime(x)|\le M$ for all $x\in \mathbb{R},$
 then $f$ is uniformly continuous.

\item A point $p\in\mathbb{R}$ is a \emph{fixed point} of a function 
 $g:\mathbb{R}\to\mathbb{R}$ if $g(p)=p.$ Show, if $g$ is differentiable
 and $|g^\prime(x)|<1$ for all $x\in \mathbb{R},$ then $g$ has
 at most one fixed  point.
\end{enumerate}
 
\begin{proof}[Proof (A)]
    Let $f\colon\mathbb{R}\to \mathbb{R}$ be a differentiable function as given and suppose that there exists $M\geq 0$ such that $\abs{f^{\prime}(x)}\leq M$ for all $x\in \mathbb{R}$.

    Let $x,y\in \mathbb{R}$ with $x < y$. By the mean value theorem, there exists $c\in (x,y)$ such that $f(x)-f(y) = f^{\prime}(c)(x-y)$. By taking the absolute value, we have that \begin{align*}
        \abs{f(x)-f(y)} &= \abs{f^{\prime}(c)(x-y)}\\
        &= \abs{f^{\prime}(c)}\abs{x-y}\\
        &\leq M\abs{x-y}.
    \end{align*}
    Given $\varepsilon> 0$, choose $\delta = \varepsilon/M$. When $\abs{x-y}< \delta$, we have that $\abs{f(x)-f(y)} < M\delta = \varepsilon$. Hence $f$ is uniformly continuous.
\end{proof}
\begin{proof}[Proof (B)]
    Let $g$ be a differentiable function as given with $\abs{g^{\prime}(x)}< 1$ for all $x\in\mathbb{R}$.

    Suppose by way of contradiction that $g$ has more than one fixed point; that is, there exist $p,q\in\mathbb{R}$ with $p < q$ such that $g(p) = p$ and $g(q)= q$.

    Then by the mean value theorem, there exists $c\in (p,q)$ such that \begin{align*}
        \abs{q-p} = \abs{g(q)-g(p)} &= \abs{g^{\prime}(c)(q-p)}\\
        &= \abs{g^{\prime}(c)}\abs{q-p}\\
        &< \abs{q-p}
    \end{align*} which is a contradiction. Hence $g$ has at most one fixed point.
\end{proof}
\end{document}