\documentclass[12pt]{amsart}

\textwidth = 6.2 in
\textheight = 8.5 in
\oddsidemargin = 0.0 in
\evensidemargin = 0.0 in
\topmargin = 0.0 in
\headheight = 0.0 in
\headsep = 0.3 in
\parskip = 0.05 in
\parindent = 0.3 in

\usepackage{enumerate}
\usepackage{amsmath}
\usepackage{color}
\def\cc{\color{blue}}
\usepackage[normalem]{ulem}
\usepackage{amsfonts, amsmath, amssymb, amsthm}
\usepackage{systeme}
\usepackage[none]{hyphenat}
\usepackage{graphicx}
\graphicspath{{./images/}}
\usepackage{esint}
\usepackage{cancel}

\title{Homework 3}
\author{Sai Sivakumar}

\newtheorem{theorem}            {Theorem}[section]
\newtheorem{proposition}        [theorem]{Proposition}

\begin{document}
\maketitle

As a suggested problem, prove the following proposition.

\begin{proposition}
 Suppose $X\subseteq Y\subseteq \mathbb{R}.$ If $X$ is not empty and $Y$ is bounded
 above, then $X$ and $Y$ both have suprema and
\[
 \sup(X)\le \sup(Y).
\]
\end{proposition}


\bigskip


Let $A_1,A_2,\dots$ be a collection of subsets of $\mathbb{R}$  such that $A_n$
 is not  empty for each $n.$ Suppose $A=\cup_{n=1}^\infty A_n$ is bounded above. 
 Since, $A_n\subseteq A,$ 
 from the Proposition above,  $A,$ and each $A_n,$ has a supremum  and moreover
 $\sup(A_n)\le \sup(A)$ for all $n\in \mathbb{N}.$ 

 Let $\alpha_n=\sup(A_n)$ (thus $\alpha_n \le \sup(A)$).
  Let $B=\{\alpha_n: n\in \mathbb{N}\}$ and note $\sup(A)$ is an upper 
 bound for $B.$ Since $B$ is not empty,  it follows that $B$ has 
  a supremum and moreover $\sup(B)\le \sup(A).$

 For homework 3, 
 complete the steps below (or otherwise) to show
\[
 \sup(A)=\sup(B).
\]

\begin{enumerate}[(i)]
% \item  Show that $\sup(A)$ is an upper bound for $B.$
% \item Show  $B$ has a supremum and   $\beta=\sup(B)\le \alpha=sup(A).$ 
 \item  Show, if $y<\sup(A),$ then there is a $k\in\mathbb{N}$
 that $y<\alpha_k.$  %Thus $y<\alpha_k$.

 \vspace*{5pt}\noindent Since $y< \sup(A)$, it follows that there exists $x\in A$ such that $y<x< \sup(A)$. Then since $x\in A$, there exists $k\in\mathbb{N}$ such that $x\in A_k$. So $y < x \leq \alpha_k$.\vspace*{5pt}
 \item Show, if $y<\sup(A),$ then   $y<\sup(B).$
 
 \vspace*{5pt}\noindent If $y<\sup(A)$, we saw that there exists $k\in\mathbb{N}$ such that $y <\alpha_k$. Since $\alpha_k \leq \sup(B)$, it follows that $y < \sup(B)$.\vspace*{5pt}
 \item Show $\sup(A)\le \sup(B).$
 
 \vspace*{5pt}\noindent If $y<\sup(A)$ implies $y<\sup(B)$, then it follows that $\sup(A)\leq \sup(B)$. \vspace*{5pt}

 Hence $\sup(A) = \sup(B)$.\qed
\end{enumerate} 
\end{document}