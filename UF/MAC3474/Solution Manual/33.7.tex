\documentclass{article}
\usepackage[utf8]{inputenc}
\usepackage{amsmath}

\title{Solution Manual}
\author{N. Kapsos, M. Schrank, S. Sivakumar}
\date{}

\begin{document}

\maketitle
\setcounter{secnumdepth}{0}

\section{33.7 Exercises}

\subsection{33.7.1}

\subsection{33.7.4}

\subsection{33.7.7}

We are asked to transform from the $(u,v)$ coordinates into the $(x,y)$ coordinates. Since the region $D^{\prime}$ is a square, we may represent the boundary of the square with four line segments lying on these four lines: $u = 0$, $u = 1$, $v = 0$, and $v = 1$.

Using the equations given in the problem we may solve for the boundary of $D$ in the $(x,y)$ coordinate system. It is immediate that $x = 0$ and $x = 1$ are lines that form part of the new boundary.

Since $y$ depends on both $u$ and $v$, we may want to hold one of those variables constant and observe what happens if we vary the other (while still maintaining the bounds given by $D^{\prime}$). Trying with $v = 0$, it is apparent that $y = 0$. That is fine. Then if we try with $v = 1$, we can still vary $u$. According to the definition of $D^{\prime}$, $u$ varies from $0$ to $1$, exactly as $x$ does (as per $x=u$). Therefore we may give $y = 1-x^2$.

Hence $D$ is the area bounded by the line $x=0$, $y=0$, and $y=1-x^2$. (the area under the parabola in the first quadrant)

\subsection{33.7.11}

We first find equations of lines passing through those four coordinate points. They are given by $y=x+4$, $y=x-4$, $y=-\frac{1}{3}x+\frac{8}{3}$, and $y=-\frac{1}{3}x$. We may convert each of these lines to their corresponding lines in the $(u,v)$ coordinate system using the transformation given in the problem. Simply substitute the given equations for $y$ and $x$ into each of the lines and resolve into coordinate curves.

We find the following curves:
$u=\pm 4$, $v=0$, and $v=8$. This is a rectangular region so it is convenient in the integral. The Jacobian of transformation for this change of variables is given by the determinant: $$\bigg| \det \begin{pmatrix}
    -\frac{3}{4} && \frac{1}{4} \\
    \frac{1}{4} && \frac{1}{4}
\end{pmatrix}\bigg| = \frac{1}{4}$$

The integral becomes $$\int_{0}^{8}\int_{-4}^{4} \left(8\left(\frac{v-3u}{4}\right) + 4\left(\frac{u+v}{4}\right)\right)\left(\frac{1}{4}\right)dudv \to 6\int_0^8 v dv = 192$$

\subsection{33.7.13}

We first want to find the new bounds in the $(u,v)$ coordinate system. Observe that $$1 = x^2-y^2 = (u\cosh(v))^2-(u\sinh(v))^2 = u^2$$
$$4 = x^2-y^2 = (u\cosh(v))^2-(u\sinh(v))^2 = u^2$$ and since we are in the first quadrant we will also take $u$ to be positive, so $1\leq u \leq 2$. Then for $v$ we use the other two equations given earlier: $$x = 2y \to u\cosh(v) = 2u\sinh(v) \to \frac{1}{2} = \tanh(v) \to v = \tanh^{-1}\left(\frac{1}{2}\right)$$
$$x = 4y \to u\cosh(v) = 4u\sinh(v) \to \frac{1}{4} = \tanh(v) \to v = \tanh^{-1}\left(\frac{1}{4}\right)$$

Using the known equation for the inverse hyperbolic tangent $$\tanh^{-1}(z) = \frac{1}{2}\ln\left(\frac{1+z}{1-z}\right)$$ deduce that $\frac{1}{2}\ln\left(\frac{5}{3}\right)\leq v \leq \frac{1}{2}\ln(3)$. This is a square region in the new coordinate system, which is convenient. Compute the Jacobian matrix for the coordinate transformation given in the problem:
$$J = \bigg|\det \begin{pmatrix}
    x^{\prime}_u && y^{\prime}_u \\
    x^{\prime}_v && y^{\prime}_v
\end{pmatrix}\bigg| = \bigg|\det \begin{pmatrix}
    \cosh(v) && \sinh(v) \\
    u\sinh(v) && u\cosh(v)
\end{pmatrix}\bigg|$$

Find that $J = |u(\cosh^2(v) - \sinh^2(v))| = u$. Now we may set up the new integral and solve:
$$\int_{\frac{1}{2}\ln\left(\frac{5}{3}\right)}^{\frac{1}{2}\ln(3)}\int_{1}^{2} (u^2\cosh^2(v) - u^2\cosh^2(v))^{-\frac{1}{2}} ududv = \int_{\frac{1}{2}\ln\left(\frac{5}{3}\right)}^{\frac{1}{2}\ln(3)}\int_{1}^{2} (1) dudv$$
$$\int_{\frac{1}{2}\ln\left(\frac{5}{3}\right)}^{\frac{1}{2}\ln(3)} (1) dv = \ln(3) - \frac{1}{2}\ln(5)$$

\subsection{33.7.14}

From the original bounds we can rewrite them as $\frac{x}{y}=1$, $\frac{x}{y}=\frac{1}{2}$, $x+y=1$, and $x+y=2$. Then knowing that $u=\frac{x}{y}$ and $v=x+y$, it is apparent that $\frac{1}{2}\leq u \leq 1$ and $1\leq v \leq 2$. The Jacobian of transformation is found by observing the following:
$$dudv = J dxdy \to \frac{1}{J}dudv = dxdy$$
$$J = \bigg| \det \begin{pmatrix}
   u^{\prime}_x && v^{\prime}_x \\
   u^{\prime}_y && v^{\prime}_y
\end{pmatrix} \bigg| = \bigg| \det \begin{pmatrix}
    \frac{1}{y} && 1 \\
    \frac{-x}{y^2} && 1
 \end{pmatrix} \bigg| = \frac{x+y}{y^2}$$

 Remembering to take the reciprocal of $J$, the integral becomes $$\iint e^{\frac{x}{y}}\frac{(x+y)^3}{y^2}\left(\frac{y^2}{x+y}\right)dudv \to \int_1^2\int_{\frac{1}{2}}^1 e^u v^2 dudv \to \left(e^u|_1^2\right)\left(\frac{v^3}{3}\bigg|_1^2\right) $$
 $$= \frac{7}{3}(e-\sqrt{e})$$

\subsection{33.7.19}

The lazy way out is sometimes the easiest way out. Give $u = xy$ and $v = xy^2$. Then it is evident that $1\leq u \leq 2$ and $1\leq v \leq 2$ from the definitions of the bounding curves in the $(x,y)$ coordinate system. Then to compute the Jacobian we may want to compute it for the reverse transformation, that is to compute $J$ for $$dudv = Jdxdy$$

So compute partial derivatives of $u$ and $v$ and compute the following determinant: $$J = \bigg|\det \begin{pmatrix}
    y && 2xy \\
    x && x^2
\end{pmatrix}\bigg| = yx^2$$

So $dudv = (yx^2)dxdy$, which is already in the integral. Conveniently the integral becomes $$\int_1^2\int_1^2 dudv = 1$$

\subsection{33.7.21}

Give $u=x+y$ and $v=y-x^3$, so that $1\leq u \leq 2$ and $0 \leq v \leq 1$. Computing the Jacobian $J$ that satisfies $\frac{1}{J}dudv = dydx$ by taking the partial derivatives of $u$ and $v$, we find $$J = \bigg| \det \begin{pmatrix}
    1 && -3x^2 \\
    1 && 1
\end{pmatrix} \bigg| = 1+3x^2$$

This makes the integral convenient since it makes the new integrand $1$. The double integral becomes (by geometry) $$\int_{0}^{1}\int_{1}^{2} dudv = 1$$

\subsection{33.7.22}

With some rearranging of the parabola equations it becomes apparent that we may give $u = xy$ and $v = y-x^2$ to find that $-1\leq u \leq 1$ and $1 \leq v \leq 2$. Then we may compute the Jacobian where $$dudv = Jdxdy$$  This is given by the following determinant: $$J = \bigg|\det \begin{pmatrix}
    u^{\prime}_x && v^{\prime}_x \\
    u^{\prime}_y && v^{\prime}_y
\end{pmatrix} \bigg| = \bigg|\det \begin{pmatrix}
    y && -2x \\
    x && 1
\end{pmatrix} \bigg| = y+2x^2$$

So $dudv = (y+2x^2)dxdy$, which is conveniently already in the integral. The integral then becomes $$\int_1^2\int_{-1}^1 dudv = 2$$

\subsection{33.7.25}

\subsection{33.7.28}

\subsection{33.7.31}

\subsection{33.7.34}

\subsection{33.7.37}

\subsection{33.7.40}

\end{document}