\documentclass{article}
\usepackage[utf8]{inputenc}
\usepackage{amsmath}

\title{Solution Manual}
\author{N. Kapsos, M. Schrank, S. Sivakumar}
\date{}

\begin{document}

\maketitle
\setcounter{secnumdepth}{0}

\section{25.6 Exercises}

\subsection{25.6.1}

Recall that in order to determine if the critical point $\mathbf{r}_0$ is a local minimum, local maximum, or a saddle point we must find the roots $\lambda_1, \lambda_2$ to the characteristic polynomial (in this case a quadratic) given by
$$P_2(\lambda) = \det\begin{pmatrix}
    a-\lambda & c \\
    c & b-\lambda
\end{pmatrix} = \lambda^2 - (a+b)\lambda + (ab-c^2)$$

where
$$a = f^{\prime \prime}_{xx}(\mathbf{r}_0)~,~ b = f^{\prime \prime}_{yy}(\mathbf{r}_0)~,~ c = f^{\prime \prime}_{xy}(\mathbf{r}_0)$$

Determining the roots itself is not necessary. (See Corollary 25.1) Sufficient information comes from computing $D=ab-c^2$ and the sign of $a$ (positive or negative).

\textsf{(i)} $D = (-3)(-2)-(2)^2 = 2$, $a < 0 \implies$ The point $(0,0)$ is a local maximum.

\textsf{(ii)} $D = (3)(2)-(2)^2 = 2$, $a > 0 \implies$ The point $(0,0)$ is a local minimum.

\textsf{(iii)} $D = (1)(2)-(2)^2 = -2 \implies$ The point $(0,0)$ is a saddle point.

\textsf{(iv)} $D = (2)(2)-(2)^2 = 0 \implies$ Inconclusive. Investigate higher order differentials.

\subsection{25.6.4}

Because $f(x,y) = x^4 - 2x^2 - y^3 + 3y$ is a polynomial, the gradient $$\vec{\nabla}f(x,y) = \langle 4x^3-4x, -3y^2+3 \rangle$$ can never be undefined. We solve for points that cause the gradient to be equal to $\vec{0}$ by finding $(x,y)$ that solve the following system of equations:
$$4x^3-4x = 0$$
$$-3y^2+3 = 0$$

Evidently any combination of $x=0,1,-1$ and $y = 1,-1$ form critical points. So computing $a = f^{\prime \prime}_{xx}(x,y) = 12x^2-4$, $b = f^{\prime \prime}_{yy}(x,y) = -6y$, and $c = f^{\prime \prime}_{xy}(x,y) = 0$ we analyze 6 cases by computing $D = ab-c^2$ and noting the sign of $a$:

$(0,1)$ : $D = (-4)(-6)-(0)^2 = 24$, $a < 0 \implies$ The point is a local maximum.

$(0,-1)$ : $D = (-4)(6)-(0)^2 = -24 \implies$ The point is a saddle point.

$(1,1)$ : $D = (6)(-6)-(0)^2 = -36 \implies$ The point is a saddle point.

$(1,-1)$ : $D = (6)(6)-(0)^2 = 36$, $a > 0 \implies$ The point is a local minimum.

$(-1,1)$ : $D = (6)(-6)-(0)^2 = -36 \implies$ The point is a saddle point.

$(-1,-1)$ : $D = (6)(6)-(0)^2 = 36$, $a > 0 \implies$ The point is a local minimum.

\subsection{25.6.7}

Because $f(x,y) = \frac{1}{3}x^3+y^2-x^2-3x-y+1$ is a polynomial, the gradient $$\vec{\nabla}f(x,y) = \langle x^2-2x-3, 2y-1\rangle$$ can never be undefined. We solve for points that cause the gradient to be equal to $\vec{0}$ by solving for points $(x,y)$ that satisfy the following system of equations:
$$x^2-2x-3 = 0$$
$$2y-1 = 0$$

Evidently we have two critical points $(3,\frac{1}{2})$ and $(-1,\frac{1}{2})$. Proceed for each case by computing $a = f^{\prime \prime}_{xx}(x,y) = 2x-2$, $b = f^{\prime \prime}_{yy}(x,y) = 2$, and $c = f^{\prime \prime}_{xy}(x,y) = 0$ and then computing $D = ab-c^2$ and noting the sign of $a$:

$(3,\frac{1}{2})$ : $D = (4)(2)-(0)^2 = 8$, $a > 0 \implies$ The point is a local minimum.

$(-1,\frac{1}{2})$ : $D = (-4)(2)-(0)^2 = -8 \implies$ The point is a saddle point.

\subsection{25.6.10}

On all points outside of the origin $f(x,y) = x^2 + y^2 + \frac{1}{x^2y^2}$ is defined and the gradient $$\vec{\nabla}f(x,y) = \langle 2x-\frac{2}{x^3y^2}, 2y-\frac{2}{x^2y^3} \rangle$$ is as well. Like usual find critical points for which the gradient is equal to the zero vector, and those are any combination of $x = \pm 1$ and $y = \pm 1$ (4 points).

Proceed for each point by computing
$$a = f^{\prime \prime}_{xx}(x,y) = 2+\frac{6}{y^2x^4}$$
$$b = f^{\prime \prime}_{yy}(x,y) = 2+\frac{6}{y^4x^2}$$
$$c = f^{\prime \prime}_{xy}(x,y) = \frac{4}{x^3y^3}$$
and then computing $D = ab-c^2$ and noting the sign of $a$. Note that at all 4 points, $a,b=8$ due to the product of only even powers of $x,y$ and $x,y = \pm 1$. Similarly $c=4$ because it contains the product of odd powers of $x,y$ which forces it to be positive, and $x,y = \pm 1$ it attains that value. Hence $(1,1)$, $(1,-1)$, $(-1,1)$, $(-1,-1)$ are all local minima ($ab-c^2 = 48$ and $a>0$).

\subsection{25.6.13}

The function $f(x,y) = y^3 + 6xy + 8x^3$ is a polynomial so its gradient $$\vec{\nabla}f(x,y) = \langle 6y + 24x^2, 3y^2 + 6x \rangle$$ will not be undefined. We find critical points by finding points that satisfy $\vec{\nabla}f(x,y) = \vec{0}$, or alternatively, the following system of equations:
$$6y + 24x^2 = 0$$
$$3y^2 + 6x = 0$$

From the second equation, deduce that all critical points lie on the line $$x = -\frac{1}{2}y^2$$ and substitute this for $x$ into the first equation to find the equation (either form): $$6y + 6y^4 = 0 \leftrightarrow y(y^3 + 1) = 0$$

There are only two real solutions, $y = -1$ and $y = 0$. Substituting these back into $3y^2 + 6x = 0$ it is apparent that the critical points occur at $(0,0)$ and $(-\frac{1}{2}, -1)$. Then compute point the following: $a = f^{\prime \prime}_{xx}(x,y) = 48x$, $b = f^{\prime \prime}_{yy}(x,y) = 6y$, and $c = f^{\prime \prime}_{xy}(x,y) = 6$ and then computing $D = ab-c^2$ and noting the sign of $a$:

$(0,0)$ : $D = (0)(0)-(6)^2 = -36 \implies$ The origin is a saddle point.

$(-\frac{1}{2}, -1)$ : $D = (-24)(-6)-(6)^2 = 108$, $a < 0 \implies$ This point is a local maximum.

\subsection{25.6.16}

The function $f(x,y) = x\cos(y)$ is a product of a monomial and the cosine, which is a smooth function whose gradient $$\vec{\nabla}f(x,y) = \langle \cos(y),-x\sin(y) \rangle$$ will not be undefined. We can find all points that cause the gradient to vanish by observation. From the first component it is clear that all y values that would make the gradient vanish are $y = \frac{k\pi}{2}$ for any integer $k$. Then notice that those values of $y$ cause the second component of the gradient to become $\pm x$, which of course is only equal to $0$ when $x=0$. So all critical points come in the form $(0, \frac{k\pi}{2})$.

Then at all critical points $a = f^{\prime \prime}_{xx}(x,y) = 0$, $b = f^{\prime \prime}_{yy}(x,y) = -x\cos(y)$, and $c = f^{\prime \prime}_{xy}(x,y) = -\sin(y)$ and then computing $D = ab-c^2$ and noting the sign of $a$. However, because $f^{\prime \prime}_{xx}(x,y)$ is zero, and $c^2 = \sin^2(y)$, $D$ will always be negative at all critical points. So all critical points $(0, \frac{k\pi}{2})$ are saddle points.

\subsection{25.6.19}

The function $f(x,y) = (5x+7y-25)e^{-x^2-xy-y^2}$ is a product of a polynomial and an exponential which is nice and smooth. The gradient $$\vec{\nabla}f(x,y) = \langle e^{-x^2-xy-y^2}(5 + (-2x-y)(5x+7y-25))$$
$$, e^{-x^2-xy-y^2}(7 + (-2y-x)(5x+7y-25)) \rangle$$ will also be nice and smooth and more importantly never undefined. To find critical points it is sufficient to solve the following system of equations since the exponential is always nonzero:
$$(5 + (-2x-y)(5x+7y-25)) = -10 x^2 - 19 x y + 50 x - 7 y^2 + 25 y + 5 = 0$$
$$(7 + (-2y-x)(5x+7y-25)) = -5 x^2 - 17 x y + 25 x - 14 y^2 + 50 y + 7 = 0$$

INCOMPLETE

\subsection{25.6.22}

The function $f(x,y) = \frac{1}{3}y^3 + xy + \frac{8}{3}x^3$ is a polynomial so the gradient $$\vec{\nabla}f(x,y) = \langle 8x^2 + y,y^2 + x \rangle$$ cannot be undefined anywhere. To solve for points where the gradient will vanish use the fact that since $f^{\prime}_y = y^2 + x = 0$ then $x = -y^2$ and we can substitute this back into $f^{\prime}_x = 8x^2 + y = 0$ to find that we have critical points at $y$ values that solve: $$8y^4 + y = 0 \leftrightarrow y((2y)^3 + 1^3) = 0$$
$$\text{sum of cubes} \to y(2y+1)(4y^2-2y+1) = 0$$

The real roots are $y = 0,-\frac{1}{2}$. Plug these back into the equation $x = -y^2$ to find that the critical points are $(0,0)$ and $(-\frac{1}{4}, -\frac{1}{2})$.

Then for each point compute $a = f^{\prime \prime}_{xx}(x,y) = 16x$, $b = f^{\prime \prime}_{yy}(x,y) = 2y$, and $c = f^{\prime \prime}_{xy}(x,y) = 1$ and then computing $D = ab-c^2$ and noting the sign of $a$:

$(0,0)$ : $D = (0)(0)-(1)^2 = -1 \implies$ This point is a saddle point.

$(-\frac{1}{4}, -\frac{1}{2})$ : $D = (-4)(-1)-(1)^2 = 3$, $a < 0 \implies$ This point is a local maximum.

\subsection{25.6.25}

The function $f(x,y) = x + y + \sin(x) \sin(y)$ is a polynomial plus the product of two sinusoids so it is a smooth function where its gradient $$\vec{\nabla}f(x,y) = \langle 1 + \cos(x)\sin(y),1 + \sin(x)\cos(y) \rangle$$ will not be undefined anywhere. We want to find critical points where:
$$1 + \cos(x)\sin(y) = 0$$
$$1 + \sin(x)\cos(y) = 0$$

We may try with the first equation and find that we would like the cosine term to be equal to one and likewise the sine term as well, so a natural guess may be to give critical points as $(\pi + 2\pi j, \frac{-\pi}{2}+2\pi k)$ for integers $j,k$. However this automatically fails satisfying the second equation, and likewise any points that satisfy the second equation fail the first one. We may instead directly show this by adding both equations together to form the following, using the fact that $\sin(x+y) = \cos(x)\sin(y) + \sin(x)\cos(y)$:
$$2 + \cos(x)\sin(y) + \sin(x)\cos(y) = 0 \leftrightarrow 2 + \sin(x+y) = 0$$

Evidently since the sine function is bounded below by $-1$ there are no critical points. Hence there are no local extrema.

\subsection{25.6.28}

Give $F(x,y,z) = x^2 + y^2 + z^2 - 2x + 2y - 4z - 10 = 0$ (a spherical level set) and use the implicit differentiation equations $$z^{\prime}_x = \frac{-F^{\prime}_x}{F^{\prime}_z},~ z^{\prime}_y = \frac{-F^{\prime}_y}{F^{\prime}_z}$$ to find the gradient of such a surface $z(x,y)$, which will be $$\langle \frac{-(2x-2)}{2z-4},\frac{-(2y+2)}{2z-4} \rangle$$ and will be valid and not undefined where $2z-4 \neq 0$ as per the Implicit Function Theorem. This essentially then boils down to finding points $(x,y)$ that solve the following system:
$$2x-2 = 0$$
$$2y+2 = 0$$

Since the equations are functions of one variable each, we can simply choose the critical point to be $(1,-1)$. Then also note that we will need a $z$ value, so evaluate $F(1,-1,z) = 0$ and solve for $z$:
$$F(1,-1,z) = 1+1+z^2-2-2-4z-10 = 0$$
$$\to z^2-4z-12 = 0 \implies z = 2 \pm 4$$

So really we have two critical points, $(1,-1,6)$ and $(1,-1,-2)$. At each point we compute $a = f^{\prime \prime}_{xx}(x,y) = \frac{-2}{2z-4}$, $b = f^{\prime \prime}_{yy}(x,y) = \frac{-2}{2z-4}$, and $c = f^{\prime \prime}_{xy}(x,y) = 0$ and then compute $D = ab-c^2$ and noting the sign of $a$:

$(1,-1,6)$ : $D = (-\frac{1}{4})(-\frac{1}{4})-(0)^2 = \frac{1}{16}$, $a < 0 \implies$ The point is a local maximum.

$(1,-1,-2)$ : $D = (\frac{1}{4})(\frac{1}{4})-(0)^2 = \frac{1}{16}$, $a > 0 \implies$ The point is a local minimum.

An important observation is the geometrical significance of both critical points as it relates to the implicitly defined $z(z,y)$. Consider where the implicitly defined $z(x,y)$ \textit{cannot} be determined - that is at $z = 2$, as there is where $F^{\prime}_z$ vanishes. Then also consider that since $F(x,y,z) = 0$ was the level set of a sphere, we really have two distinct surfaces $z(x,y)$, which are the hemispheres of the sphere remove the circle that forms at the intersection of $z = 2$ and the sphere given by $F(x,y,z) = 0$. 

Geometrically each critical point refers to the peak or the trough of each respective hemisphere, which is a neat observation and makes sense geometrically. Being spherical also explains why at each critical point $a$ and $b$ have the same values.

\subsection{25.6.31}

Rearrange the equation of the plane given into the function $z(x,y) = 4-x+y$. Using this, we can make a function $d(x,y)$ that gives the square of the distance between generic points on the plane $(x,y,4-x+y)$ and $(1,2,3)$. Define it like so:
$$d(x,y) = (1-x)^2+(2-y)^2+(x-y-1)^2$$

The reasoning behind using the square of the distance is because it makes computation of the partial derivatives easier. It is justified because distance is a non-negative quantity, and so squaring such a function would not change the location of the local minumum we seek to find.

Continuing, we have the gradient as $$\langle -2(1-x) + 2(x-y-1),-2(2-y) - 2(x-y-1) \rangle$$
$$ = \langle 4x-2y-4,-2 x + 4 y - 2 \rangle$$ which will never be undefined. We seek to solve the following system of equations to find the critical point:
$$4x - 2y - 4 = 0$$
$$-2x + 4y - 2 = 0$$

Using the bottom equation, find that $x = 2y-1$, and substitute this into the top equation to get $6y-8 = 0$. Evidently $y = \frac{4}{3}$, and so $x = \frac{5}{3}$.

Then compute $a = f^{\prime \prime}_{xx}(x,y) = 4$, $b = f^{\prime \prime}_{yy}(x,y) = 4$, and $c = f^{\prime \prime}_{xy}(x,y) = -2$ and then compute $D = ab-c^2$ and noting the sign of $a$:

$(\frac{5}{3}, \frac{4}{3})$ : $D = (4)(4)-(-2)^2 = 12$, $a > 0 \implies$ This point is a local minimum.

Using the definition of the plane the point that minimizes the distance is $(\frac{5}{3}, \frac{4}{3}, \frac{11}{3})$.

An alternative solution requires no minimization at all but to simply find a real value $s$ such that the vector $\langle 1, 2 ,3 \rangle + s\langle 1, -1 ,1 \rangle$ satisfies the equation of the plane ($s = \frac{2}{3}$). This takes advantage of the fact that the line segment between $(1,2,3)$ and the point that minimizes the distance to the plane is at a right angle to the plane itself, so we can use the normal vector that determines the plane to find the minimizing point.

\end{document}