\documentclass[11pt]{article}
\headheight=13.6pt
% packages
\usepackage{physics}
% margin spacing
\usepackage[top=1in, bottom=1in, left=0.5in, right=0.5in]{geometry}
\usepackage{hanging}
\usepackage{amsfonts, amsmath, amssymb, amsthm}
\usepackage{systeme}
\usepackage[none]{hyphenat}
\usepackage{fancyhdr}
\usepackage[nottoc, notlot, notlof]{tocbibind}
\usepackage{graphicx}
\graphicspath{{./images/}}
\usepackage{float}
\usepackage{siunitx}
\usepackage{esint}
\usepackage{cancel}
\usepackage{enumitem}
\usepackage{mathrsfs}

% colors
\usepackage{xcolor}
\definecolor{p}{HTML}{FFDDDD}
\definecolor{g}{HTML}{D9FFDF}
\definecolor{y}{HTML}{FFFFCF}
\definecolor{b}{HTML}{D9FFFF}
\definecolor{o}{HTML}{FADECB}
%\definecolor{}{HTML}{}

% \highlight[<color>]{<stuff>}
\newcommand{\highlight}[2][p]{\mathchoice%
  {\colorbox{#1}{$\displaystyle#2$}}%
  {\colorbox{#1}{$\textstyle#2$}}%
  {\colorbox{#1}{$\scriptstyle#2$}}%
  {\colorbox{#1}{$\scriptscriptstyle#2$}}}%

% header/footer formatting
\pagestyle{fancy}
\fancyhead{}
\fancyfoot{}
\fancyhead[L]{MAA6616 Analysis}
\fancyhead[C]{Final}
\fancyhead[R]{Sai Sivakumar}
\fancyfoot[R]{\thepage}
\renewcommand{\headrulewidth}{1pt}

% paragraph indentation/spacing
\setlength{\parindent}{0cm}
\setlength{\parskip}{10pt}
\renewcommand{\baselinestretch}{1.25}

% extra commands defined here
\newcommand{\br}[1]{\left(#1\right)}
\newcommand{\sbr}[1]{\left[#1\right]}
\newcommand{\cbr}[1]{\left\{#1\right\}}

\newcommand{\dprime}{\prime\prime}

% bracket notation for inner product
\usepackage{mathtools}

\DeclarePairedDelimiterX{\abr}[1]{\langle}{\rangle}{#1}

\DeclareMathOperator{\Span}{span}

% set page count index to begin from 1
\setcounter{page}{1}

\begin{document}
Do any THREE problems.
\begin{enumerate}
  \item[1.] For a) implies b): for any $f\in L^1(\mu)$, we have $\int\abs{gf} = \int\abs{g}\abs{f}\leq M\int\abs{f}<+\infty$ since $f\in L^1(\mu)$.
  
  The reverse implication I feel is shady, but here was what I had so far: 

  We try a proof by contrapositive, so assume that there does not exist an $M$ such that $\abs{g}\leq M$ almost everywhere; that is, for every $M$ the set $G_M = \cbr{x\in X\colon \abs{g(x)}\geq M}$ has positive measure. Also by dominated convergence for sets we should have that as $M$ goes to infinity, the measure of the sets $G_M$ goes to zero (I am interpreting $g\colon X\to \mathbb{R}$ to mean that $g$ does not take on $+\infty$), but I did not find this to be useful.

  I formed a measurable partition of $X$ via the sets $E_n = \cbr{x\in X\colon n\leq \abs{g}(x)\leq n+1} = G_n\cap G_{n+1}^c$, so that $X = \cup_{n=0}^\infty E_n$. Some of the $E_n$ may have measure zero; more importantly, there must be an infinite number of the $E_n$ with nonzero measure, since otherwise there would be $N$ large enough with $\abs{g}\leq N$. It follows that $\mu(X) = \sum_{n=0}^\infty \mu(E_n)$ is still an infinite sum. 

  Observe that $\sum_{n=0}^{N-1} n\mathbf{1}_{E_n} + N\mathbf{1}_{G_N}$ for $N$ finite is a simple function that is less than or equal to $\abs{g}$. My idea was to define an $f\in L^1(\mu)$ such that the above simple function could be modified to become a simple function less than $\abs{gf}$ whose integral could be made arbitrarily large as $N$ tends to $+\infty$ (which means the integral of $\abs{gf}$ is arbitrarily large and so $gf$ is not in $L^1(\mu)$).
  
  A function I had in mind was $f\colon X\to\mathbb{R}$ such that $f$ takes on the value $\frac{1}{n^2\mu(E_n)}$ on $E_n$ whenever $\mu(E_n)\neq 0$ and $n\geq 1$ and zero everywhere else. Then the integral of $\abs{f} = f$ would be equal to or at least bounded above by $\sum_{\substack{n\in\mathbb{Z}_+\\ \mu(E_n)\neq 0}}\frac{1}{n^2\mu(E_n)}\mu(E_n)\leq \sum_{n=1}^\infty\frac{1}{n^2}<+\infty$, so that $f\in L^1(\mu)$. But then $\sum_{n=1}^{N-1} \frac{1}{n^2\mu(E_n)}\cdot n\mathbf{1}_{E_n} + \frac{1}{N^2\mu(G_N)}\cdot N\mathbf{1}_{G_N}$ is a simple function less than or equal to $\abs{gf}$ for all $N$, but its integral is bounded below by $\sum_{\substack{1\leq n\leq N\\ \mu(E_n)\neq 0}}\frac{1}{n}$. As $N$ is made arbitrarily large, this sum should diverge (or I hope it should), so $\int \abs{gf}$ must also diverge. This should prove the contrapositive, but somehow I feel like something is wrong.

  \item[3.] We check first that the function of two variables $g(x,t) = (1/2h)f(x-t)\mathbf{1}_{[-h,h]}(t)$ is in $L^1(\mathbb{R}^2)$. This function and its absolute value should be measurable if $f(x-t)$ is measurable, but I am not sure how to show this directly. The function $f(x-t)$ should still be measurable, and we proceed by showing that $g(x,t)$ is absolutely integrable using Tonelli's theorem: \begin{multline*}
    \int_{\mathbb{R}^2}\abs{g(x,t)}\dd{A} = \frac{1}{2h}\int_\mathbb{R}\int_\mathbb{R}\abs{f(x-t)}\mathbf{1}_{[-h,h]}(t)\dd{x}\dd{t}= \frac{1}{2h}\int_{\mathbb{R}}\mathbf{1}_{[-h,h]}(t)\int_\mathbb{R}\abs{f(x-t)}\dd{x}\dd{t}\\ =\frac{1}{2h}\int_{\mathbb{R}}\mathbf{1}_{[-h,h]}(t)\dd{t}\int_\mathbb{R}\abs{f(x)}\dd{x} = \int_\mathbb{R}\abs{f(x)}\dd{x}<+\infty.
  \end{multline*}
  Now apply Fubini's theorem to obtain the following chains of inequalities: \begin{align*}
    \abs{\int_{\mathbb{R}^2}\frac{1}{2h}f(x-t)\mathbf{1}_{[-h,h]}(t)\dd{A}} &= \abs{\int_{\mathbb{R}}\int_{\mathbb{R}}\frac{1}{2h}f(x-t)\mathbf{1}_{[-h,h]}(t)\dd{t}\dd{x}} \\
    &\leq \int_{\mathbb{R}}\abs{\int_{\mathbb{R}}\frac{1}{2h}f(x-t)\mathbf{1}_{[-h,h]}(t)\dd{t}}\dd{x}\\
    &= \int_\mathbb{R}\abs{\frac{1}{2h}\int_{x-h}^{x+h}f(t)\dd{t}}\dd{x}\\
    &= \int_\mathbb{R}\abs{f_h(x)}\dd{x}
  \end{align*} and \begin{align*}
    \int_{\mathbb{R}}\abs{\int_{\mathbb{R}}\frac{1}{2h}f(x-t)\mathbf{1}_{[-h,h]}(t)\dd{t}}\dd{x} &\leq \int_{\mathbb{R}}\int_{\mathbb{R}}\frac{1}{2h}\abs{f(x-t)}\mathbf{1}_{[-h,h]}(t)\dd{t}\dd{x}\\ 
    &=\int_{\mathbb{R}}\int_{\mathbb{R}}\frac{1}{2h}\abs{f(x-t)}\mathbf{1}_{[-h,h]}(t)\dd{x}\dd{t}\\
    &= \int_\mathbb{R}\abs{f(x)}\dd{x}
  \end{align*} by the above. It follows that $\int_\mathbb{R}\abs{f_h(x)}\dd{x}\leq\int_\mathbb{R}\abs{f(x)}\dd{x}$.
  
  \item[4.] In the background let $X = X_+\cup X_-$ be the Hahn decomposition of $X$ (so that in the Jordan decomposition of $\nu$, $\nu_+$ is a measure on $X_+$ and $\nu_-$ is a measure on $X_-$ and they are mutually singular)\begin{enumerate}
    \item We show inequalities in both ways.
    
    Let $E$ be measurable. Then $\abs{\nu(E)} = \abs{\nu_+(E) - \nu_-(E)}\leq \abs{\nu_+(E)} + \abs{\nu_-(E)} = {\nu_+(E)} +{\nu_-(E)} = \abs{\nu}(E)$. Then for any finite partition $\cbr{E_j}_{j=1}^n$ of $E$, we have $\sum_{j=1}^n\abs{\nu(E_j)}\leq \sum_{j=1}^n\abs{\nu}(E_j) = \abs{\nu}(E)$, where the last equality was due to additivity as the $E_j$ form a partition of $E$. It follows that \[\abs{\nu}(E) \geq \sup\cbr{\sum_{j=1}^n\abs{\nu(E_j)}\colon E_1,\dots,E_n \text{ form a partition for $E$}}.\]

    For the other inequality we show that $\abs{\nu}(E)$ is an element of the set we are taking the supremum over: Consider the partition of $E$ into the sets $E\cap X_+$ and $E\cap X_-$ (since $X= X_+\cup X_-$). Then since $\nu_+,\nu_-$ are mutually singular measures, we have \begin{align*}
      \abs{\nu(E\cap X_+)} + \abs{\nu(E\cap X_-)} &= \abs{\nu_+(E\cap X_+)} + \abs{\nu_-(E\cap X_-)}  \\
      &= {\nu_+(E\cap X_+)} + {\nu_-(E\cap X_-)}\\
      &={\nu_+(E\cap X_+)} + \nu_+(E\cap X_-) + {\nu_-(E\cap X_-)} + \nu_-(E\cap X_+)\\
      &= \nu_+(E) + \nu_-(E)\\
      &= \abs{\nu}(E).
    \end{align*} Thus the reverse inequality is obtained, so we have equality.
    \item Define $\abs{f}_{X_+}$ to be the function on $X$ that is $0$ on $X_-$ and is $\abs{f}$ on $X_+$; define $\abs{f}_{X_-}$ similarly to be the function which is $0$ on $X_+$ and is $\abs{f}$ on $X_-$. It follows that $\abs{f} = \abs{f}_{X_+} + \abs{f}_{X_-}$. Then we have \begin{align}
      \abs{\int f\dd{\nu}} &= \abs{\int f\dd{\nu_+} - \int f \dd{\nu_-}}\\
      &\leq \int\abs{f}\dd{\nu_+} + \int\abs{f}\dd{\nu_-}\\
      &=\int\abs{f}_{X_+}\dd{\abs{\nu}} + \int\abs{f}_{X_-}\dd{\abs{\nu}}\\
      &=\int \abs{f}_{X_+} + \abs{f}_{X_-}\dd{\abs{\nu}}\\
      &= \int \abs{f}\dd{\abs{\nu}}.
    \end{align} The equality in (3) is justified since $\abs{\nu} = \nu_+ +\nu_-$, and any simple unsigned function less than or equal to $\abs{f}_{X_+}$ (resp. $\abs{f}_{X_-}$) must be zero on $X_-$ (resp. $X_+$). Every simple function of this form may also be viewed as (by restriction) a simple function on $X_+$ (resp. $X_-$) less than or equal to $\abs{f}_{X_+}$ (resp. $\abs{f}_{X_-}$). Hence equality (3) follows.
  \end{enumerate}
\end{enumerate}
\end{document}