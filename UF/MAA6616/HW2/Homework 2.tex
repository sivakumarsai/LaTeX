\documentclass[11pt]{article}

% packages
\usepackage{physics}
% margin spacing
\usepackage[top=1in, bottom=1in, left=0.5in, right=0.5in]{geometry}
\usepackage{hanging}
\usepackage{amsfonts, amsmath, amssymb, amsthm}
\usepackage{systeme}
\usepackage[none]{hyphenat}
\usepackage{fancyhdr}
\usepackage[nottoc, notlot, notlof]{tocbibind}
\usepackage{graphicx}
\graphicspath{{./images/}}
\usepackage{float}
\usepackage{siunitx}
\usepackage{esint}
\usepackage{cancel}
\usepackage{enumitem}
\usepackage{mathrsfs}

% colors
\usepackage{xcolor}
\definecolor{p}{HTML}{FFDDDD}
\definecolor{g}{HTML}{D9FFDF}
\definecolor{y}{HTML}{FFFFCF}
\definecolor{b}{HTML}{D9FFFF}
\definecolor{o}{HTML}{FADECB}
%\definecolor{}{HTML}{}

% \highlight[<color>]{<stuff>}
\newcommand{\highlight}[2][p]{\mathchoice%
  {\colorbox{#1}{$\displaystyle#2$}}%
  {\colorbox{#1}{$\textstyle#2$}}%
  {\colorbox{#1}{$\scriptstyle#2$}}%
  {\colorbox{#1}{$\scriptscriptstyle#2$}}}%

% header/footer formatting
\pagestyle{fancy}
\fancyhead{}
\fancyfoot{}
\fancyhead[L]{MAA6616 Analysis}
\fancyhead[C]{Homework 2}
\fancyhead[R]{Sai Sivakumar}
\fancyfoot[R]{\thepage}
\renewcommand{\headrulewidth}{1pt}

% paragraph indentation/spacing
\setlength{\parindent}{0cm}
\setlength{\parskip}{10pt}
\renewcommand{\baselinestretch}{1.25}

% extra commands defined here
\newcommand{\br}[1]{\left(#1\right)}
\newcommand{\sbr}[1]{\left[#1\right]}
\newcommand{\cbr}[1]{\left\{#1\right\}}

\newcommand{\dprime}{\prime\prime}

% bracket notation for inner product
\usepackage{mathtools}

\DeclarePairedDelimiterX{\abr}[1]{\langle}{\rangle}{#1}

\DeclareMathOperator{\Span}{span}

% set page count index to begin from 1
\setcounter{page}{1}

\begin{document}
\begin{enumerate}
    \item (7.10) Let $\mathscr{A}$ be an atomic $\sigma$-algebra generated by a partition $(A_n)_{n=1}^\infty$ of a set $X$ (see Problem 7.3). \begin{enumerate}
        \item Fix $n\geq 1$. Prove that the function $\delta_n\colon\mathscr{A}\to [0,1]$ defined by \[\delta_n(A) = \begin{cases}
            1 &\text{if } A_n\subset A\\
            0 &\text{if } A_n\not\subset A
        \end{cases}\] is a measure on $\mathscr{A}$. \begin{proof}
            Every set $A_n$ is not empty, hence $\delta_n(\varnothing) = 0$ since $A_n$ could not be contained in the empty set.

            Every element of $\mathscr{A}$ is an at most countable union of members of $(A_n)_{n=1}^\infty$. If $(E_j)_{j=1}^\infty$ is a sequence of disjoint sets in $\mathscr{A}$, then for each $E_j$ we can find disjoint subsets $C_j \subset \mathbb{Z}_+$ (so $C_p\cap C_q = \varnothing$ for $p\neq q$) such that $E_j = \bigcup_{k\in C_j}A_k$, so that $E = \bigcup_{j=1}^\infty E_j = \bigcup_{k\in \bigcup_{j=1}^\infty C_j} A_k$. It follows that $\delta_n(E)$ is $1$ if $A_n$ is found in the union $\bigcup_{k\in \bigcup_{j=1}^\infty C_j} A_k$ (that is, if $n\in \bigcup_{j=1}^\infty C_j$) and is $0$ otherwise. This is the same as taking the sum $\sum_{j=1}^\infty \delta_n(E_j)$ since $A_n$ is either contained in $E_j = \bigcup_{k\in C_j}A_k$ or it is not, for each $j$; furthermore, $A_n$ would only appear at most once in $\bigcup_{j=1}^\infty E_j$ since the sets $E_j$ are disjoint.
        \end{proof}
        \item Prove that if $\mu$ is any measure on $(X,\mathscr{A})$, then there exists a unique sequence $(c_n)$ with each $c_n\in[0,+\infty]$ such that \[\mu(A) = \sum_{n=1}^\infty c_n\delta_n(A)\] for all $A\in\mathscr{A}$. \begin{proof}
            If there exists a sequence $(c_n)$ satisfying the above then it is unique: Let $(c_n)$ and $(d_n)$ be sequences satisfying the above so that for any $A\in\mathscr{A}$, we have $\mu(A) = \sum_{n=1}^\infty c_n\delta_n(A) = \sum_{n=1}^\infty d_n\delta_n(A)$. But for each $i\in\mathbb{Z}_+$ we have \[d_i = \sum_{n=1}^\infty d_n\delta_n(A_i) = \sum_{n=1}^\infty c_n\delta_n(A_i)= c_i\] from which it follows that $(c_n) = (d_n)$.

            Let $A\in\mathscr{A}$ so that $A = \bigcup_{k\in C}A_k$ for some $C\subseteq \mathbb{Z}_+$. Then \begin{align*}
                \mu(A) &= \mu\left(\bigcup_{k\in C}A_k\right)\\
                &= \sum_{k\in C}\mu(A_k) & \text{(the $A_k$ are disjoint)}\\
                &= \sum_{k\in C}\mu(A_k)\delta_k(A) & \text{(for $k\in C$, $\delta_k(A) = 1$)}\\
                &= \sum_{k\in C}\mu(A_k)\delta_k(A) + \sum_{k\in \mathbb{Z}_+\setminus C}\mu(A_k)\delta_k(A) & \text{(for $k\in \mathbb{Z}_+\setminus C$, $\delta_k(A) = 0$)}\\
                &= \sum_{n=1}^\infty \mu(A_n)\delta_n(A),
            \end{align*} and since $\mu$ maps into $[0,+\infty]$, we have our desired sequence $(c_n = \mu(A_n))$.
        \end{proof}
    \end{enumerate}
    \item (7.12) Let $X$ be a set. For a sequence of subsets $(E_n)$ of $X$, define \[\limsup E_n = \bigcap_{N=1}^\infty \bigcup_{n=N}^\infty E_n, \quad \liminf E_n = \bigcup_{N=1}^\infty\bigcap_{n=N}^\infty E_n.\] \begin{enumerate}
        \item Prove that $\limsup \mathbf{1}_{E_n} = \mathbf{1}_{\limsup E_n}$ and $\liminf \mathbf{1}_{E_n} = \mathbf{1}_{\liminf E_n}$ (thus justifying the names). Conclude that $E_n\to E$ pointwise if and only if $\limsup E_n = \liminf E_n = E$. (Hint: for the first part, observe that $x\in \limsup E_n$ if and only if $x$ lies in infinitely
        many of the $E_n$, and $x\in \liminf E_n$ if and only if $x$ lies in all but finitely many $E_n$.) \begin{proof}
            Let $x\in X$. Then $\limsup \mathbf{1}_{E_n}(x)= \lim (\sup\cbr{\mathbf{1}_{E_k}(x)\mid k\geq n})$. This limit is $1$ if and only if there exists an $N\geq 1$ such that for $n\geq N$, $x\in E_j$ for some $j\geq n$. When such an $N$ exists, since $x\in E_j$ with $j\geq n\geq N\geq 1$ it follows that $\sup\cbr{\mathbf{1}_{E_k}\mid k\geq 1} = 1$. Therefore we demand that there exists $j\geq N$ for every $N\geq 1$ such that $x\in E_j$ for $\lim (\sup\cbr{\mathbf{1}_{E_k}(x)\mid k\geq n})$ to be $1$.
            
            This condition on $x$ is the same as the condition needed for $x$ to be in $\limsup E_n = \bigcap_{N=1}^\infty \bigcup_{n=N}^\infty E_n$; that is, for every $N\geq 1$, $x$ needs to be in at least one $E_j$ for $j\geq N$. Hence $\limsup \mathbf{1}_{E_n} = \mathbf{1}_{\limsup E_n}$.

            Similarly, for $x\in X$, the quantity $\liminf \mathbf{1}_{E_n}(x)= \lim (\inf\cbr{\mathbf{1}_{E_k}(x)\mid k\geq n})$. This limit is $1$ if and only if there exists an $N\geq 1$ such that for $n\geq N$, $x\in E_j$ for every $j\geq n$. This is exactly the condition needed for $x$ to be in $\liminf E_n = \bigcup_{N=1}^\infty\bigcap_{n=N}^\infty E_n$; that is, there exists $N\geq 1$ such that for every $n\geq N$, $x\in E_n$. Hence $\liminf \mathbf{1}_{E_n} = \mathbf{1}_{\liminf E_n}$.

            The sequence $(E_n)$ converges to $E$ pointwise if and only if for every $x\in X$, the sequence $(\mathbf{1}_{E_n}(x))$ converges to $\mathbf{1}_E(x)$. This is equivalent to saying $\limsup \mathbf{1}_{E_n}(x) = \liminf \mathbf{1}_{E_n}(x) = \mathbf{1}_E(x)$. By the above two results we equivalently have that $\mathbf{1}_{\limsup E_n}(x) = \mathbf{1}_{\liminf E_n}(x) = \mathbf{1}_E(x)$, which is equivalent to $\limsup E_n = \liminf E_n = E$ as desired.
        \end{proof}
        \item Prove that if the $E_n$ are measurable, then so are $\limsup E_n$ and $\liminf E_n$. Deduce that if $(E_n)$ converges to $E$ pointwise and all the $E_n$ are measurable, then $E$ is measurable. \begin{proof}
            Since $\sigma$-algebras are closed under countable unions and intersections it is clear that $\limsup E_n = \bigcap_{N=1}^\infty \bigcup_{n=N}^\infty E_n$ and $\liminf E_n = \bigcup_{N=1}^\infty\bigcap_{n=N}^\infty E_n$ whenever every $E_n$ is measurable.

            Then if $(E_n)$ converges to $E$ pointwise we have from the above result that $\limsup E_n = \liminf E_n = E$; since $\limsup E_n$ and $\liminf E_n$ were shown to be measurable whenever every $E_n$ is measurable, $E$ is measurable.
        \end{proof}
    \end{enumerate}
    \item (7.13) [Fatou theorem for sets] Let $(X,\mathscr{M},\mu)$ me a measure space, and let $(E_n)$ be a sequence of measurable sets. \begin{enumerate}
        \item Prove that \[\mu(\liminf E_n)\leq \liminf \mu(E_n).\] \begin{proof}
            We have $\liminf E_n = \bigcup_{N=1}^\infty\bigcap_{n=N}^\infty E_n$, and observe that $\bigcap_{n=j}^\infty E_n\subseteq \bigcap_{n=j+1}^\infty E_n$ for each $j$. Then \begin{align*}
                \mu(\liminf E_n) &= \mu\left(\bigcup_{N=1}^\infty\bigcap_{n=N}^\infty E_n\right)\\
                &= \lim_{N\to\infty} \mu\left(\bigcap_{n=N}^\infty E_n\right) & \text{monotone convergence for sets}\\
                &\leq \lim_{N\to\infty}(\inf\cbr{\mu(E_n)\mid n\geq N}) & \text{for fixed $N$, $\bigcap_{n=N}^\infty E_n \subseteq E_n$ for all $n\geq N$; monotonicity}\\
                &= \liminf \mu(E_n)
            \end{align*} as desired.
        \end{proof}
        \item Assume in addition that $\mu(\bigcup_{n=1}^\infty E_n)< \infty$. Prove that \[\mu(\limsup E_n)\geq \limsup \mu(E_n).\] \begin{proof}
            We have $\limsup E_n = \bigcap_{N=1}^\infty \bigcup_{n=N}^\infty E_n$, and observe that $\bigcup_{n=j}^\infty E_n\supseteq \bigcup_{n=j+1}^\infty E_n$ for each $j$ with $\mu(\bigcup_{n=1}^\infty E_n)< \infty$. Then \begin{align*}
                \mu(\limsup E_n) &= \mu\left(\bigcap_{N=1}^\infty \bigcup_{n=N}^\infty E_n\right)\\
                &= \lim_{N\to \infty} \mu\left(\bigcup_{n=N}^\infty E_n\right) & \text{dominated convergence for sets}\\
                &\geq \lim_{N\to\infty}(\sup\cbr{\mu(E_n)\mid n\geq N}) & \text{for fixed $N$, $\bigcup_{n=N}^\infty E_n\supseteq E_n$ for all $n\geq N$; monotonicity}\\
                &= \limsup \mu(E_n)
            \end{align*} as desired.
        \end{proof}
        \item Prove the stronger form of the dominated convergence theorem for sets: suppose $(E_n)$ is a sequence of measurable sets, and there is a measurable set $F\subset X$ such that $E_n\subset F$ for all $n$ and $\mu(F)<\infty$. Prove that if $(E_n)$ converges to $E$ pointwise, then $(\mu(E_n))$ converges to $\mu(E)$. Give an example to show the finiteness hypothesis on $F$ cannot be dropped. \begin{proof}
            Since the sequence of measurable sets $(E_n)$ converges pointwise to $E$, it follows from a prior result that $E$ was measurable also. Since $E_n\subseteq F$ for all $n$ implies that $\bigcup_{n=1}^\infty E_n\subseteq F$.

            We have that $(E_n)$ converges pointwise to $E$ if and only if $\limsup E_n = E = \liminf E_n$, so that $\mu(\limsup E_n) = \mu(E) = \mu(\liminf E_n)$. Apply the previous two results (for the latter, we need $\bigcup_{n=1}^\infty E_n\subseteq F$ and $\mu(F)< \infty$) to obtain $\limsup \mu(E_n) \leq \mu(E) \leq \liminf \mu(E_n)$; in general $\limsup \mu(E_n)\geq \liminf \mu(E_n)$ so they are equal. Hence $(\mu(E_n))$ converges to $\mu(E)$.
        \end{proof}

        The $\mu(F)<\infty$ condition is required: Consider the measure space $(\mathbb{N}, 2^\mathbb{N}, \text{counting})$, and take $F = \mathbb{N}$, which has infinite measure. Then for every $n\in\mathbb{N}$, define $E_n = \cbr{m\in\mathbb{N}\mid m\geq n}$; each of these have infinite measure and are subsets of $F = \mathbb{N}$. But $(E_n)$ by inspection converges pointwise to the empty set, which has measure zero; this is not the limit of $(\mu(E_n))$, which is $\infty$ (it is a constant sequence).
        
        (For parts (a) and (b), use Theorem 2.3.)
    \end{enumerate}
\end{enumerate}
\end{document}