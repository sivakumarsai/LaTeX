\documentclass[11pt]{article}

% packages
\usepackage{physics}
% margin spacing
\usepackage[top=1in, bottom=1in, left=0.5in, right=0.5in]{geometry}
\usepackage{hanging}
\usepackage{amsfonts, amsmath, amssymb, amsthm}
\usepackage{systeme}
\usepackage[none]{hyphenat}
\usepackage{fancyhdr}
\usepackage[nottoc, notlot, notlof]{tocbibind}
\usepackage{graphicx}
\graphicspath{{./images/}}
\usepackage{float}
\usepackage{siunitx}
\usepackage{esint}
\usepackage{cancel}
\usepackage{enumitem}
%\usepackage{tikz-cd}
\usepackage{quiver}

% permutations (second line is for spacing)
\usepackage{permute}
\renewcommand*\pmtseparator{\,}

% colors
\usepackage{xcolor}
\definecolor{p}{HTML}{FFDDDD}
\definecolor{g}{HTML}{D9FFDF}
\definecolor{y}{HTML}{FFFFCF}
\definecolor{b}{HTML}{D9FFFF}
\definecolor{o}{HTML}{FADECB}
%\definecolor{}{HTML}{}

% \highlight[<color>]{<stuff>}
\newcommand{\highlight}[2][p]{\mathchoice%
  {\colorbox{#1}{$\displaystyle#2$}}%
  {\colorbox{#1}{$\textstyle#2$}}%
  {\colorbox{#1}{$\scriptstyle#2$}}%
  {\colorbox{#1}{$\scriptscriptstyle#2$}}}%

% header/footer formatting
\pagestyle{fancy}
\fancyhead{}
\fancyfoot{}
\fancyhead[L]{MTG6347 Topology}
\fancyhead[C]{Homework 4}
\fancyhead[R]{Sai Sivakumar}
\fancyfoot[R]{\thepage}
\renewcommand{\headrulewidth}{1pt}

% paragraph indentation/spacing
\setlength{\parindent}{0cm}
\setlength{\parskip}{10pt}
\renewcommand{\baselinestretch}{1.25}

% extra commands defined here
\newcommand{\br}[1]{\left(#1\right)}
\newcommand{\sbr}[1]{\left[#1\right]}
\newcommand{\cbr}[1]{\left\{#1\right\}}

\newcommand{\catname}[1]{{\textbf{#1} }}
\newcommand{\Set}{\catname{Set}}
\newcommand{\Top}{\catname{Top}}
\DeclareMathOperator{\Int}{Int}
\DeclareMathOperator{\Bd}{Bd}
\DeclareMathOperator{\id}{id}
\DeclareMathOperator{\im}{im}
\DeclareMathOperator{\coker}{coker}
\DeclareMathOperator{\Aut}{Aut}
\DeclareMathOperator{\Hom}{Hom}
\DeclareMathOperator{\Ext}{Ext}

% bracket notation for inner product
\usepackage{mathtools}

\DeclarePairedDelimiterX{\abr}[1]{\langle}{\rangle}{#1}
\DeclarePairedDelimiter{\ceil}{\lceil}{\rceil}
\DeclarePairedDelimiter{\floor}{\lfloor}{\rfloor}

% set page count index to begin from 1
\setcounter{page}{1}

\begin{document}
\begin{enumerate}
    \item Verify your computations for the cohomology of the torus, the projective plane and the Klein bottle for integer and mod $2$ coefficients from the previous assignment by using the Universal Coefficient Theorem for Cohomology.
    
    We recall the homology in $\mathbb{Z}$ coefficients we computed in Homework 2. The homology for the torus $S$ is given by $H_0(S) = \mathbb{Z}$, $H_1(S) = \mathbb{Z}^{2}$, and $H_2(S) = \mathbb{Z}$; all other groups are $0$. For the projective plane we have $H_0(\mathbb{RP}^2) = \mathbb{Z}$, $H_1(\mathbb{RP}^2)=\mathbb{Z}/2\mathbb{Z}$, and $H_2(\mathbb{RP}^2)=0$; all other groups are $0$. For the Klein bottle $K$ we have $H_0(K) = \mathbb{Z}$, $H_1(K) = \mathbb{Z}\oplus\mathbb{Z}/2\mathbb{Z}$, and $H_2(K) = 0$; all other groups are $0$. In our formulation, negative homology groups are zero.
    \begin{enumerate}
      \item Cohomology of the torus $S$:\begin{enumerate}
        \item in $G=\mathbb{Z}$ coefficients:
        
        By the UCT for Cohomology and what was recorded above, we have the following three short exact sequences:
        \[0\to \Ext_\mathbb{Z}^1(0, G)=0\to H^0(S;G)\to \Hom_\mathbb{Z}(\mathbb{Z},G)=\mathbb{Z}\to 0\]
        \[0\to \Ext_\mathbb{Z}^1(\mathbb{Z}, G)=0\to H^1(S;G)\to \Hom_\mathbb{Z}(\mathbb{Z}^2,G)=\mathbb{Z}^2\to 0\]
        \[0\to \Ext_\mathbb{Z}^1(\mathbb{Z}^2, G)=0\to H^2(S;G)\to \Hom_\mathbb{Z}(\mathbb{Z},G)=\mathbb{Z}\to 0\] 
        As $0,\mathbb{Z},\mathbb{Z}^2$ are free the $\Ext$ groups above are all zero. Since $\Hom_\mathbb{Z}(\mathbb{Z},\mathbb{Z}) = \mathbb{Z}$ and $\Hom$ distributes over direct sums in the first component, we obtain the $\Hom$ groups above. By exactness we obtain $H^0(S;G)\cong\mathbb{Z}$, $H^1(S;G)\cong\mathbb{Z}^2$, and $H^2(S;G)\cong\mathbb{Z}$.

        Since the higher homology groups of the torus are all zero (which is free, and $H_2(S)$ is free also), we have the following short exact sequence for $i\geq 3$: \[0\to \Ext_\mathbb{Z}^1(H_{i-1}(S), G)=0\to H^i(S;G)\to \Hom_\mathbb{Z}(0,G)=0\to 0\] It follows by exactness that the higher cohomology groups are all zero as well.
        \item in $G=\mathbb{Z}/2\mathbb{Z}$ coefficients:
        
        By the UCT for Cohomology and what was recorded above, we have the following three short exact sequences:
        \[0\to \Ext_\mathbb{Z}^1(0, G)=0\to H^0(S;G)\to \Hom_\mathbb{Z}(\mathbb{Z},G)=\mathbb{Z}/2\mathbb{Z}\to 0\]
        \[0\to \Ext_\mathbb{Z}^1(\mathbb{Z}, G)=0\to H^1(S;G)\to \Hom_\mathbb{Z}(\mathbb{Z}^2,G)=(\mathbb{Z}/2\mathbb{Z})^2\to 0\]
        \[0\to \Ext_\mathbb{Z}^1(\mathbb{Z}^2, G)=0\to H^2(S;G)\to \Hom_\mathbb{Z}(\mathbb{Z},G)=\mathbb{Z}/2\mathbb{Z}\to 0\] 
        As $0,\mathbb{Z},\mathbb{Z}^2$ are free the $\Ext$ groups above are all zero. Since $\Hom_\mathbb{Z}(\mathbb{Z},\mathbb{Z}/2\mathbb{Z}) = \mathbb{Z}/2\mathbb{Z}$ (the identity map and the projection map of order $2$) and $\Hom$ distributes over direct sums in the first component, we obtain the $\Hom$ groups above. By exactness we obtain $H^0(S;G)\cong\mathbb{Z}/2\mathbb{Z}$, $H^1(S;G)\cong(\mathbb{Z}/2\mathbb{Z})^2$, and $H^2(S;G)\cong\mathbb{Z}/2\mathbb{Z}$.

        Since the higher homology groups of the torus are all zero (which is free, and $H_2(S)$ is free also), we have the following short exact sequence for $i\geq 3$: \[0\to \Ext_\mathbb{Z}^1(H_{i-1}(S), G)=0\to H^i(S;G)\to \Hom_\mathbb{Z}(0,G)=0\to 0\] It follows by exactness that the higher cohomology groups are all zero as well.
      \end{enumerate}
      \item Cohomology of $\mathbb{RP}^2$:\begin{enumerate}
        \item in $G=\mathbb{Z}$ coefficients:
        
        By the UCT for Cohomology and what was recorded above, we have the following three short exact sequences:
        \[0\to \Ext_\mathbb{Z}^1(0, G)=0\to H^0(\mathbb{RP}^2;G)\to \Hom_\mathbb{Z}(\mathbb{Z},G)=\mathbb{Z}\to 0\]
        \[0\to \Ext_\mathbb{Z}^1(\mathbb{Z}, G)=0\to H^1(\mathbb{RP}^2;G)\to \Hom_\mathbb{Z}(\mathbb{Z}/2\mathbb{Z},G)=0\to 0\]
        \[0\to \Ext_\mathbb{Z}^1(\mathbb{Z}/2\mathbb{Z}, G)=\mathbb{Z}/2\mathbb{Z}\to H^2(\mathbb{RP}^2;G)\to \Hom_\mathbb{Z}(0,G)=0\to 0\] 
        From a result we proved in class we have the result giving us the third $\Ext$ group above. Since $\mathbb{Z}/2\mathbb{Z}$ has finite order there can be no nontrivial homomorphisms $\mathbb{Z}/2\mathbb{Z}\to \mathbb{Z}$. By exactness we obtain $H^0(\mathbb{RP}^2;G)\cong\mathbb{Z}$, $H^1(\mathbb{RP}^2;G)\cong 0$, and $H^2(\mathbb{RP}^2;G)\cong\mathbb{Z}/2\mathbb{Z}$.

        Since the higher homology groups of the projective plane are all zero (which is free, and $H_2(\mathbb{RP}^2)=0$ also), we have the following short exact sequence for $i\geq 3$: \[0\to \Ext_\mathbb{Z}^1(H_{i-1}(\mathbb{RP}^2), G)=0\to H^i(\mathbb{RP}^2;G)\to \Hom_\mathbb{Z}(0,G)=0\to 0\] It follows by exactness that the higher cohomology groups are all zero as well.
        \item in $G=\mathbb{Z}/2\mathbb{Z}$ coefficients:
        
        By the UCT for Cohomology and what was recorded above, we have the following three short exact sequences:
        \[0\to \Ext_\mathbb{Z}^1(0, G)=0\to H^0(\mathbb{RP}^2;G)\to \Hom_\mathbb{Z}(\mathbb{Z},G)=\mathbb{Z}/2\mathbb{Z}\to 0\]
        \[0\to \Ext_\mathbb{Z}^1(\mathbb{Z}, G)=0\to H^1(\mathbb{RP}^2;G)\to \Hom_\mathbb{Z}(\mathbb{Z}/2\mathbb{Z},G)=\mathbb{Z}/2\mathbb{Z}\to 0\]
        \[0\to \Ext_\mathbb{Z}^1(\mathbb{Z}/2\mathbb{Z}, G)=\mathbb{Z}/2\mathbb{Z}\to H^2(\mathbb{RP}^2;G)\to \Hom_\mathbb{Z}(0,G)=0\to 0\] 
        From a result we proved in class we have the result giving us the third $\Ext$ group above; in particular we have $(\mathbb{Z}/2\mathbb{Z})/2(\mathbb{Z}/2\mathbb{Z})= (\mathbb{Z}/2\mathbb{Z})/0 \cong \mathbb{Z}/2\mathbb{Z}$. The only homomorphisms $\mathbb{Z}/2\mathbb{Z}\to \mathbb{Z}/2\mathbb{Z}$ are the identity and the order $2$ map sending $1\mapsto 1$. By exactness we obtain $H^0(\mathbb{RP}^2;G)\cong\mathbb{Z}/2\mathbb{Z}$, $H^1(\mathbb{RP}^2;G)\cong \mathbb{Z}/2\mathbb{Z}$, and $H^2(\mathbb{RP}^2;G)\cong\mathbb{Z}/2\mathbb{Z}$.

        Since the higher homology groups of the projective plane are all zero (which is free, and $H_2(\mathbb{RP}^2)=0$ also), we have the following short exact sequence for $i\geq 3$: \[0\to \Ext_\mathbb{Z}^1(H_{i-1}(\mathbb{RP}^2), G)=0\to H^i(\mathbb{RP}^2;G)\to \Hom_\mathbb{Z}(0,G)=0\to 0\] It follows by exactness that the higher cohomology groups are all zero as well.
      \end{enumerate}
      \item Cohomology of the Klein bottle $K$:\begin{enumerate}
        \item in $G=\mathbb{Z}$ coefficients:
        
        By the UCT for Cohomology and what was recorded above, we have the following three short exact sequences:
        \[0\to \Ext_\mathbb{Z}^1(0, G)=0\to H^0(K;G)\to \Hom_\mathbb{Z}(\mathbb{Z},G)=\mathbb{Z}\to 0\]
        \[0\to \Ext_\mathbb{Z}^1(\mathbb{Z}, G)=0\to H^1(K;G)\to \Hom_\mathbb{Z}(\mathbb{Z}\oplus \mathbb{Z}/2\mathbb{Z},G)=\mathbb{Z}\to 0\]
        \[0\to \Ext_\mathbb{Z}^1(\mathbb{Z}\oplus \mathbb{Z}/2\mathbb{Z}, G)= \mathbb{Z}/2\mathbb{Z}\to H^2(K;G)\to \Hom_\mathbb{Z}(0,G)=0\to 0\] 
        We proved in class that the $\Ext$ functor distributes over direct sums in the first component, giving us the third $\Ext$ group above. By exactness we obtain $H^0(K;G)\cong\mathbb{Z}$, $H^1(K;G)\cong \mathbb{Z}$, and $H^2(K;G)\cong\mathbb{Z}/2\mathbb{Z}$.

        Since the higher homology groups of the Klein bottle are all zero (which is free, and $H_2(K)=0$ also), we have the following short exact sequence for $i\geq 3$: \[0\to \Ext_\mathbb{Z}^1(H_{i-1}(K), G)=0\to H^i(K;G)\to \Hom_\mathbb{Z}(0,G)=0\to 0\] It follows by exactness that the higher cohomology groups are all zero as well.
        \item in $G=\mathbb{Z}/2\mathbb{Z}$ coefficients:
        
        By the UCT for Cohomology and what was recorded above, we have the following three short exact sequences:
        \[0\to \Ext_\mathbb{Z}^1(0, G)=0\to H^0(K;G)\to \Hom_\mathbb{Z}(\mathbb{Z},G)=\mathbb{Z}/2\mathbb{Z}\to 0\]
        \[0\to \Ext_\mathbb{Z}^1(\mathbb{Z}, G)=0\to H^1(K;G)\to \Hom_\mathbb{Z}(\mathbb{Z}\oplus \mathbb{Z}/2\mathbb{Z},G)=(\mathbb{Z}/2\mathbb{Z})^2\to 0\]
        \[0\to \Ext_\mathbb{Z}^1(\mathbb{Z}\oplus \mathbb{Z}/2\mathbb{Z}, G)= \mathbb{Z}/2\mathbb{Z}\to H^2(K;G)\to \Hom_\mathbb{Z}(0,G)=0\to 0\] 
        We proved in class that the $\Ext$ functor distributes over direct sums in the first component, giving us the third $\Ext$ group above. By exactness we obtain $H^0(K;G)\cong\mathbb{Z}/2\mathbb{Z}$, $H^1(K;G)\cong (\mathbb{Z}/2\mathbb{Z})^2$, and $H^2(K;G)\cong\mathbb{Z}/2\mathbb{Z}$.

        Since the higher homology groups of the Klein bottle are all zero (which is free, and $H_2(K)=0$ also), we have the following short exact sequence for $i\geq 3$: \[0\to \Ext_\mathbb{Z}^1(H_{i-1}(K), G)=0\to H^i(K;G)\to \Hom_\mathbb{Z}(0,G)=0\to 0\] It follows by exactness that the higher cohomology groups are all zero as well.
      \end{enumerate}
    \end{enumerate}
    \item Let $N_g$ denote the non-orientable surface of genus $g$. Consider homology and cohomology with coefficients in $\mathbb{Z}_2$.\begin{enumerate}[label=(\alph*)]
        \item Sketch a $\Delta$-complex structure for $N_g$ for some small $g$.\vspace*{7cm}
        \item Identify cycles that generate homology in degrees $0,1,2$. 
        
        Some computations of Tor groups and using the universal coefficient theorem for homology give us that the $\mathbb{Z}/2\mathbb{Z}$ homology of non-orientable surfaces is given by the groups $H_0(N_g) =  \mathbb{Z}/2\mathbb{Z}$, $H_1(N_g) = (\mathbb{Z}/2\mathbb{Z})^g$, and $H_2(N_g) = \mathbb{Z}/2\mathbb{Z}$. In the following we suppress the use of brackets to denote equivalence classes. The one vertex $e_0$ that is identified in the delta complex generates the homology in degree $0$, each edge $a_i$ generates the homology in degree $1$ (their boundary is twice the vertex generating the $0$-degree homology which is $0$), and the sum of the triangles $\sum_i T_i$ generates the homology in degree $2$. 
        \item Identify cocycles that generate cohomology in degrees $0,1,2$. 
        
        There exist the usual dual basis elements $\epsilon$, $\tau$ for the single elements generating $H_1$ and $H_2$ respectively. So $H^0$ is generated by $\varepsilon$ and $H^2$ is generated by $\tau$. For $H^1$ let $\alpha_i$ count the intersections with the arcs $\alpha_i$ (so that $\delta \alpha_i = 0$ but $\alpha_i$ is $1$ on edges intersecting it and is $0$ for the other edges). Then the $\alpha_i$ generate $H^1$.
        \item Determine the Kronecker pairings of cohomology and homology.
        
        By definition we have $\abr{\alpha_i,a_j} = \alpha_i(a_j) = \delta_{ij}$. We also have in degree $0$ that $\abr{\varepsilon,e_0} =\varepsilon(e_0) = 1$ and in degree $2$ that $\abr{\tau,\sum_i T_i} = \tau(\sum_i T_i) = 1$.
        \item Determine the cup product. 
        
        As there are no cohomology groups beyond $H^2$, we are only interested in computing cup products between the elements found in both $H^0$ and $H^1$. The cup product of anything with $\varepsilon$ is itself since $\varepsilon$ always returns $1$ for the vertex $e_0$. In fact, $\varepsilon$ is the identity of the cohomology ring.

        The cup product $\alpha_i\cup\alpha_i$ is zero on each $T_k$ except for the $T_k$ with the arc $\alpha_i$ intersecting the ``first'' two legs of $T_k$ (for example if $i = 1$ then the triangle we take in the $g = 3$ picture is the bottom right one). Hence $\alpha_i\cup\alpha_i$ is $1$ on the sum of the triangles; hence it is equal to the element generating $H^2$. The cup product $\alpha_i\cup\alpha_j$ for $i\neq j$ on any $T_k$ is zero since if the arc $\alpha_j$ intersects edges of $T_k$ for which $\alpha_i$ does not intersect the adjacent edges and vice versa (in order for the product to be nonzero this must happen).
        \item Determine the cohomology ring.
        
        The cohomology ring is the graded $\mathbb{Z}/2\mathbb{Z}$-algebra given by the $\mathbb{Z}/2\mathbb{Z}$-vector space with basis $\cbr{\varepsilon,\alpha_1,\dots,\alpha_g,\tau}$ ($g+2$ elements) with multiplication given by the cup product above; that is, to quotient out the free algebra above by the relations given by the cup product above. 
    \end{enumerate}
\end{enumerate}
\end{document}