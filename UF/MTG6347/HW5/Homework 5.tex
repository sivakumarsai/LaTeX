\documentclass[11pt]{article}

% packages
\usepackage{physics}
% margin spacing
\usepackage[top=1in, bottom=1in, left=0.5in, right=0.5in]{geometry}
\usepackage{hanging}
\usepackage{amsfonts, amsmath, amssymb, amsthm}
\usepackage{systeme}
\usepackage[none]{hyphenat}
\usepackage{fancyhdr}
\usepackage[nottoc, notlot, notlof]{tocbibind}
\usepackage{graphicx}
\graphicspath{{./images/}}
\usepackage{float}
\usepackage{siunitx}
\usepackage{esint}
\usepackage{cancel}
\usepackage{enumitem}
%\usepackage{tikz-cd}
\usepackage{quiver}

% permutations (second line is for spacing)
\usepackage{permute}
\renewcommand*\pmtseparator{\,}

% colors
\usepackage{xcolor}
\definecolor{p}{HTML}{FFDDDD}
\definecolor{g}{HTML}{D9FFDF}
\definecolor{y}{HTML}{FFFFCF}
\definecolor{b}{HTML}{D9FFFF}
\definecolor{o}{HTML}{FADECB}
%\definecolor{}{HTML}{}

% \highlight[<color>]{<stuff>}
\newcommand{\highlight}[2][p]{\mathchoice%
  {\colorbox{#1}{$\displaystyle#2$}}%
  {\colorbox{#1}{$\textstyle#2$}}%
  {\colorbox{#1}{$\scriptstyle#2$}}%
  {\colorbox{#1}{$\scriptscriptstyle#2$}}}%

% header/footer formatting
\pagestyle{fancy}
\fancyhead{}
\fancyfoot{}
\fancyhead[L]{MTG6347 Topology}
\fancyhead[C]{Homework 5}
\fancyhead[R]{Sai Sivakumar}
\fancyfoot[R]{\thepage}
\renewcommand{\headrulewidth}{1pt}

% paragraph indentation/spacing
\setlength{\parindent}{0cm}
\setlength{\parskip}{10pt}
\renewcommand{\baselinestretch}{1.25}

% extra commands defined here
\newcommand{\br}[1]{\left(#1\right)}
\newcommand{\sbr}[1]{\left[#1\right]}
\newcommand{\cbr}[1]{\left\{#1\right\}}

\newcommand{\catname}[1]{{\textbf{#1} }}
\newcommand{\Set}{\catname{Set}}
\newcommand{\Top}{\catname{Top}}
\DeclareMathOperator{\Int}{Int}
\DeclareMathOperator{\Bd}{Bd}
\DeclareMathOperator{\id}{id}
\DeclareMathOperator{\im}{im}
\DeclareMathOperator{\coker}{coker}
\DeclareMathOperator{\Aut}{Aut}
\DeclareMathOperator{\Hom}{Hom}
\DeclareMathOperator{\Ext}{Ext}

% bracket notation for inner product
\usepackage{mathtools}

\DeclarePairedDelimiterX{\abr}[1]{\langle}{\rangle}{#1}
\DeclarePairedDelimiter{\ceil}{\lceil}{\rceil}
\DeclarePairedDelimiter{\floor}{\lfloor}{\rfloor}

% set page count index to begin from 1
\setcounter{page}{1}

\begin{document}
\begin{enumerate}
    \item (3.2.1) Assuming as known the cup product structure on the torus $S^1\times S^1$, compute the cup product structure in $H^\ast(M_g)$ for $M_g$ the closed orientable surface of genus $g$ by using the quotient map from $M_g$ to a wedge sum of $g$ tori. \begin{proof}
        
    \end{proof}
    \item (3.2.2) Using the cup product $H^k(X,A;R)\times H^\ell(X,B;R)\to H^{k+\ell}(X,A\cup B;R)$, show that if $X$ is the union of contractible open sets $A$ and $B$, then all cup products of positive-dimensional classes in $H^\ast(X;R)$ are zero. This applies in particular if $X$ is a suspension. Generalize to the situation that $X$ is the union of $n$ contractible open subsets, to show that all $n$-fold cup products of positive-dimensional classes are zero. \begin{proof}
        
    \end{proof}
    \item (3.2.3a) Using the cup product structure, show there is no map $\mathbb{RP}^n\to \mathbb{RP}^m$ inducing a nontrivial map $H^1(\mathbb{RP}^m;\mathbb{Z}_2)\to H^1(\mathbb{RP}^n;\mathbb{Z}_2)$ if $n>m$. What is the corresponding result for maps $\mathbb{CP}^n\to\mathbb{CP}^m$? \begin{proof}
        
    \end{proof}
    \item (3.2.7) Use cup products to show that $\mathbb{RP}^3$ is not homotopy equivalent to $\mathbb{RP}^2\vee S^3$. \begin{proof}
        
    \end{proof}
    \item (3.2.11) Using cup products, show that every map $S^{k+\ell}\to S^k\times S^\ell$ induced the trivial homomorphism $H_{k+\ell}(S^{k+\ell})\to H_{k+\ell}(S^k\times S^\ell)$, assuming $k>0$ and $\ell>0$. \begin{proof}
        
    \end{proof}
    \item (3.2.12) Show that the spaces $(S^1\times \mathbb{CP}^\infty)/(S^1\times \{x_0\})$ and $S^3\times \mathbb{CP}^\infty$ have isomorphic cohomology rings with $\mathbb{Z}$ or any other coefficients. \begin{proof}
        
    \end{proof}
\end{enumerate}
\end{document}