\documentclass[11pt]{article}

% packages
\usepackage{physics}
% margin spacing
\usepackage[top=1in, bottom=1in, left=0.5in, right=0.5in]{geometry}
\usepackage{hanging}
\usepackage{amsfonts, amsmath, amssymb, amsthm}
\usepackage{systeme}
\usepackage[none]{hyphenat}
\usepackage{fancyhdr}
\usepackage[nottoc, notlot, notlof]{tocbibind}
\usepackage{graphicx}
\graphicspath{{./images/}}
\usepackage{float}
\usepackage{siunitx}
\usepackage{esint}
\usepackage{cancel}
\usepackage{enumitem}
%\usepackage{tikz-cd}
\usepackage{quiver}

% permutations (second line is for spacing)
\usepackage{permute}
\renewcommand*\pmtseparator{\,}

% colors
\usepackage{xcolor}
\definecolor{p}{HTML}{FFDDDD}
\definecolor{g}{HTML}{D9FFDF}
\definecolor{y}{HTML}{FFFFCF}
\definecolor{b}{HTML}{D9FFFF}
\definecolor{o}{HTML}{FADECB}
%\definecolor{}{HTML}{}

% \highlight[<color>]{<stuff>}
\newcommand{\highlight}[2][p]{\mathchoice%
  {\colorbox{#1}{$\displaystyle#2$}}%
  {\colorbox{#1}{$\textstyle#2$}}%
  {\colorbox{#1}{$\scriptstyle#2$}}%
  {\colorbox{#1}{$\scriptscriptstyle#2$}}}%

% header/footer formatting
\pagestyle{fancy}
\fancyhead{}
\fancyfoot{}
\fancyhead[L]{MTG6347 Topology}
\fancyhead[C]{Homework 3}
\fancyhead[R]{Sai Sivakumar}
\fancyfoot[R]{\thepage}
\renewcommand{\headrulewidth}{1pt}

% paragraph indentation/spacing
\setlength{\parindent}{0cm}
\setlength{\parskip}{10pt}
\renewcommand{\baselinestretch}{1.25}

% extra commands defined here
\newcommand{\br}[1]{\left(#1\right)}
\newcommand{\sbr}[1]{\left[#1\right]}
\newcommand{\cbr}[1]{\left\{#1\right\}}

\newcommand{\catname}[1]{{\textbf{#1} }}
\newcommand{\Set}{\catname{Set}}
\newcommand{\Top}{\catname{Top}}
\DeclareMathOperator{\Int}{Int}
\DeclareMathOperator{\Bd}{Bd}
\DeclareMathOperator{\id}{id}
\DeclareMathOperator{\im}{im}
\DeclareMathOperator{\coker}{coker}
\DeclareMathOperator{\Aut}{Aut}
\DeclareMathOperator{\Hom}{Hom}
\DeclareMathOperator{\Ext}{Ext}

% bracket notation for inner product
\usepackage{mathtools}

\DeclarePairedDelimiterX{\abr}[1]{\langle}{\rangle}{#1}
\DeclarePairedDelimiter{\ceil}{\lceil}{\rceil}
\DeclarePairedDelimiter{\floor}{\lfloor}{\rfloor}

% set page count index to begin from 1
\setcounter{page}{1}

\begin{document}
\begin{enumerate}
    \item Compute the cohomology with $\mathbb{Z}$ and $\mathbb{Z}/2\mathbb{Z}$ coefficients of the following using a
    cellular ($\Delta$-complex) structure with two $2$-cells: \begin{enumerate}
      \item $S^1 \times S^1$,
      \begin{table}[h]
        \centering
        \begin{tabular}{cc|c|c|c}
          $X = S^1 \times S^1$\hspace*{3cm} & \hspace*{5cm} & $H^n(X;G)$ & $G = \mathbb{Z}$ & $G = \mathbb{Z}/2\mathbb{Z}$  \\ \hline
         & $\delta v^\ast = 0$, $\delta e_0^\ast = f_0^\ast - f_1^\ast$ & $H^0(X;G)$ & $\mathbb{Z}$ & $\mathbb{Z}/2\mathbb{Z}$  \\
         & $\delta e_1^\ast= f_0^\ast-f_1^\ast$, $\delta e_2^\ast= -f_0^\ast+f_1^\ast$ & $H^1(X;G)$ & $\mathbb{Z}^2$ & $(\mathbb{Z}/2\mathbb{Z})^2$  \\
         & $\delta f_0^\ast= 0$, $\delta f_1^\ast= 0$ & $H^2(X;G)$ & $\mathbb{Z}$ & $\mathbb{Z}/2\mathbb{Z}$ \\
         &
        \end{tabular}
        \end{table}

        We have the chain complex \[0\to C^0(X) = \abr{v^\ast}\xrightarrow{g}C^1(X) = \abr{e_0^\ast, e_1^\ast, e_2^\ast}\xrightarrow{h}C^2(X) =\abr{f_0^\ast, f_1^\ast}\to 0\] where $g$ and $h$ are the $\delta$ maps above labeled for convenience.
        Based on the information in the table above we can deduce that for $G =\mathbb{Z}$, \[\ker g = \abr{v^\ast}, \im g = 0, \ker h = \abr{e_2^\ast+e_0^\ast,e_2^\ast+e_1^\ast}, \im h = \abr{f_0^\ast-f_1^\ast}.\] Thus $H^0(X) = \abr{v^\ast}\cong \mathbb{Z}$, $H^1(X) = \abr{e_2^\ast+e_0^\ast,e_2^\ast+e_1^\ast} \cong \mathbb{Z}^2$, and $H^2(X) = \abr{f_0^\ast, f_1^\ast}/\abr{f_0^\ast-f_1^\ast} \cong \abr{f_0^\ast-f_1^\ast, f_1^\ast}/\abr{f_0^\ast-f_1^\ast}\cong \abr{f_1^\ast}\cong \mathbb{Z}$. Further cohomology groups are all zero.
        
        When $G = \mathbb{Z}/2\mathbb{Z}$, we have \[\ker g = \abr{v^\ast}, \im g = 0, \ker h = \abr{e_2^\ast+e_0^\ast,e_2^\ast+e_1^\ast}, \im h = \abr{f_0^\ast-f_1^\ast}.\] Thus $H^0(X) = \abr{v^\ast}\cong \mathbb{Z}/2\mathbb{Z}$, $H^1(X) = \abr{e_2^\ast+e_0^\ast,e_2^\ast+e_1^\ast} \cong (\mathbb{Z}/2\mathbb{Z})^2$, and $H^2(X) = \abr{f_0^\ast, f_1^\ast}/\abr{f_0^\ast-f_1^\ast} \cong \abr{f_0^\ast-f_1^\ast, f_1^\ast}/\abr{f_0^\ast-f_1^\ast}\cong \abr{f_1^\ast}\cong \mathbb{Z}/2\mathbb{Z}$. Further cohomology groups are all zero.
      \item $\mathbb{RP}^2$, and
      \begin{table}[h]
        \centering
        \begin{tabular}{cc|c|c|c}
          $X = \mathbb{RP}^2$\hspace*{3cm} & \hspace*{5cm} & $H^n(X;G)$ & $G = \mathbb{Z}$ & $G = \mathbb{Z}/2\mathbb{Z}$  \\ \hline
         & $\delta v^\ast= -e_0^\ast-e_1^\ast$, $\delta w^\ast= e_0^\ast + e_1^\ast$ & $H^0(X;G)$ & $\mathbb{Z}$ & $\mathbb{Z}/2\mathbb{Z}$  \\
         & $\delta e_0^\ast= f_0^\ast+f_1^\ast$, $\delta e_1^\ast= -f_0^\ast-f_1^\ast$ & $H^1(X;G)$ & $0$ & $\mathbb{Z}/2\mathbb{Z}$  \\
         & $\delta e_2^\ast= -f_0^\ast+f_1^\ast$, $\delta f_0^\ast= 0$ & $H^2(X;G)$ & $\mathbb{Z}/2\mathbb{Z}$ & $\mathbb{Z}/2\mathbb{Z}$  \\
         & $\delta f_1^\ast= 0$ &  &  &  
        \end{tabular}
        \end{table}

        We have the chain complex \[0\to C^0(X) = \abr{v^\ast,w^\ast}\xrightarrow{g}C^1(X) = \abr{e_0^\ast, e_1^\ast, e_2^\ast}\xrightarrow{h}C^2(X) =\abr{f_0^\ast, f_1^\ast}\to 0\] where $g$ and $h$ are the $\delta$ maps above labeled for convenience. Based on the information in the table above we can deduce that for $G =\mathbb{Z}$, \[\ker g = \abr{v^\ast + w^\ast}, \im g = \abr{e_0^\ast + e_1^\ast}, \ker h = \abr{e_0^\ast+e_1^\ast}, \im h = \abr{f_0^\ast+f_1^\ast,-f_0^\ast+f_1^\ast}.\] Thus $H^0(X) = \abr{v^\ast + w^\ast}\cong \mathbb{Z}$, $H^1(X) = \abr{e_0^\ast + e_1^\ast}/\abr{e_0^\ast + e_1^\ast} \cong 0$, and $H^2(X) = \abr{f_0^\ast, f_1^\ast}/\abr{f_0^\ast+f_1^\ast,-f_0^\ast+f_1^\ast} \cong \abr{f_0^\ast+f_1^\ast, f_1^\ast}/\abr{f_0^\ast+f_1^\ast,2f_1^\ast}\cong \abr{f_1^\ast}/\abr{2f_1^\ast}\cong \mathbb{Z}/2\mathbb{Z}$. Further cohomology groups are all zero.
        
        When $G = \mathbb{Z}/2\mathbb{Z}$, we have \[\ker g = \abr{v^\ast + w^\ast}, \im g = \abr{e_0^\ast + e_1^\ast}, \ker h = \abr{e_0^\ast+e_1^\ast,e_0^\ast + e_2^\ast}, \im h = \abr{f_0^\ast+f_1^\ast,-f_0^\ast+f_1^\ast}.\] Thus $H^0(X) = \abr{v^\ast + w^\ast}\cong \mathbb{Z}/2\mathbb{Z}$, $H^1(X) = \abr{e_0^\ast+e_1^\ast,e_0^\ast + e_2^\ast}/\abr{e_0^\ast + e_1^\ast} \cong \abr{e_0^\ast + e_2^\ast}\cong \mathbb{Z}/2\mathbb{Z}$, and $H^2(X) = \abr{f_0^\ast, f_1^\ast}/\abr{f_0^\ast+f_1^\ast,-f_0^\ast+f_1^\ast} \cong \abr{f_0^\ast+f_1^\ast, f_1^\ast}/\abr{f_0^\ast+f_1^\ast,2f_1^\ast}\cong \abr{f_1^\ast}\cong \mathbb{Z}/2\mathbb{Z}$. Further cohomology groups are all zero.
      \item the Klein bottle.
      \begin{table}[h]
        \centering
        \begin{tabular}{cc|c|c|c}
          $X = $ Klein bottle\hspace*{3cm} & \hspace*{5cm} & $H^n(X;G)$ & $G = \mathbb{Z}$ & $G = \mathbb{Z}/2\mathbb{Z}$  \\ \hline
         & $\delta v^\ast= 0$, $\delta e_0^\ast= f_0^\ast-f_1^\ast$ & $H^0(X;G)$ & $\mathbb{Z}$ & $\mathbb{Z}/2\mathbb{Z}$  \\
         & $\delta e_1^\ast= -f_0^\ast-f_1^\ast$, $\delta e_2^\ast= -f_0^\ast+f_1^\ast$ & $H^1(X;G)$ & $\mathbb{Z}$ & $(\mathbb{Z}/2\mathbb{Z})^2$  \\
         & $\delta f_0^\ast= 0$, $\delta f_1^\ast= 0$ & $H^2(X;G)$ & $\mathbb{Z}/2\mathbb{Z}$ & $\mathbb{Z}/2\mathbb{Z}$ \\
         &
        \end{tabular}
        \end{table}

        We have the chain complex \[0\to C^0(X) = \abr{v^\ast}\xrightarrow{g}C^1(X) = \abr{e_0^\ast, e_1^\ast, e_2^\ast}\xrightarrow{h}C^2(X) =\abr{f_0^\ast, f_1^\ast}\to 0\] where $g$ and $h$ are the $\delta$ maps above labeled for convenience.
        Based on the information in the table above we can deduce that for $G =\mathbb{Z}$, \[\ker g = \abr{v^\ast}, \im g = 0, \ker h = \abr{e_2^\ast+e_0^\ast}, \im h = \abr{f_0^\ast-f_1^\ast, -f_0^\ast-f_1^\ast}.\] Thus $H^0(X) = \abr{v^\ast}\cong \mathbb{Z}$, $H^1(X) = \abr{e_2^\ast+e_0^\ast} \cong \mathbb{Z}$, and $H^2(X) = \abr{f_0^\ast, f_1^\ast}/\abr{f_0^\ast-f_1^\ast, -f_0^\ast-f_1^\ast} \cong \abr{f_0^\ast-f_1^\ast, f_1^\ast}/\abr{f_0^\ast-f_1^\ast,2f_1^\ast}\cong \abr{f_1^\ast}/\abr{2f_1^\ast}\cong \mathbb{Z}/2\mathbb{Z}$. Further cohomology groups are all zero.
        
        When $G = \mathbb{Z}/2\mathbb{Z}$, we have \[\ker g = \abr{v^\ast}, \im g = 0, \ker h = \abr{e_2^\ast+e_0^\ast,e_0^\ast+e_1^\ast}, \im h = \abr{f_0^\ast-f_1^\ast, -f_0^\ast-f_1^\ast}.\] Thus $H^0(X) = \abr{v^\ast}\cong \mathbb{Z}/2\mathbb{Z}$, $H^1(X) = \abr{e_2^\ast+e_0^\ast,e_0^\ast+e_1^\ast} \cong (\mathbb{Z}/2\mathbb{Z})^2$, and $H^2(X) = \abr{f_0^\ast, f_1^\ast}/\abr{f_0^\ast-f_1^\ast, -f_0^\ast-f_1^\ast} \cong \abr{f_0^\ast-f_1^\ast, f_1^\ast}/\abr{f_0^\ast-f_1^\ast,2f_1^\ast}\cong \abr{f_1^\ast}\cong \mathbb{Z}/2\mathbb{Z}$. Further cohomology groups are all zero.
    \end{enumerate}
    \item For a chain complex $C$, define $\kappa : H^n(C; G) \to \Hom(H_n(C),G)$ by $\kappa([\alpha])([\beta]) = \alpha(\beta)$. Show that this is well defined. \begin{proof}
      Representatives for a given equivalence class in $H^n(C; G)$ differ by a coboundary $\gamma\in \im \delta^{n-1}\colon C^{n-1}\to C^n$, so that $\gamma = \delta^{n-1}\gamma^\prime$ for some $\gamma^\prime\in C^{n-1}$. We have for $[\alpha]\in H^n(C;G)$ and $[\beta]\in H_n(C)$ that \[(\alpha + \delta^{n-1}\gamma^\prime)(\beta) = \alpha(\beta) + \delta^{n-1}\gamma^\prime(\beta) = \alpha(\beta) + \gamma^\prime(\partial_n\beta) = \alpha(\beta) + \gamma^\prime(0) = \alpha(\beta)\] since $\beta\in \ker \partial_n$.

      Similarly, representatives for a given equivalence class in $H_n(C)$ differ by a boundary $\gamma \in \im \partial_{n+1}$, so that $\gamma = \partial_{n+1}\gamma^\prime$. Then for $[\alpha]\in H^n(C;G)$ we have \[\alpha(\beta + \partial_{n+1}\gamma^\prime) = \alpha(\beta) + \alpha(\partial_{n+1}\gamma^\prime) = \delta^n\alpha(\gamma^\prime) = 0(\gamma^\prime) = 0\] since $\alpha\in \ker \delta^n$.

      Thus $\kappa$ as defined above is well defined.
    \end{proof}
    \item Let $A,B$ be Abelian groups. Let $0 \to R \xrightarrow{\phi} F \xrightarrow{p} A \to 0$ be a free resolution. Let $\Ext(A,B) = \coker(\phi^\ast)$. Show that this cokernel does not depend on the choice of free resolution.\begin{proof}
      Let $0 \to R \xrightarrow{\phi} F \xrightarrow{p} A \to 0$ and $0 \to R^\prime \xrightarrow{\phi^\prime} F^\prime \xrightarrow{p^\prime} A \to 0$ be any two free resolutions of $A$. Then apply Lemma 11.8.1 from tom Dieck with $\id_A\colon A\to A$ (twice) to obtain the commutative diagram (each square commuting in both ways): % https://q.uiver.app/?q=WzAsMTAsWzAsMCwiMCJdLFswLDEsIjAiXSxbNCwwLCIwIl0sWzQsMSwiMCJdLFsxLDAsIlIiXSxbMiwwLCJGIl0sWzMsMCwiQSJdLFszLDEsIkEiXSxbMiwxLCJGXlxccHJpbWUiXSxbMSwxLCJSXlxccHJpbWUiXSxbMCw0XSxbMSw5XSxbOSw4LCJcXHBoaV5cXHByaW1lIl0sWzgsNywicF5cXHByaW1lIl0sWzUsNiwicCJdLFs2LDJdLFs3LDNdLFs0LDUsIlxccGhpIl0sWzQsOSwiciIsMix7Im9mZnNldCI6MX1dLFs5LDQsInJeXFxwcmltZSIsMix7Im9mZnNldCI6MX1dLFs1LDgsImYiLDIseyJvZmZzZXQiOjF9XSxbOCw1LCJmXlxccHJpbWUiLDIseyJvZmZzZXQiOjF9XSxbNiw3LCJcXGlkX0EiLDAseyJzdHlsZSI6eyJ0YWlsIjp7Im5hbWUiOiJhcnJvd2hlYWQifX19XV0=
      \[\begin{tikzcd}
        0 & R & F & A & 0 \\
        0 & {R^\prime} & {F^\prime} & A & 0
        \arrow[from=1-1, to=1-2]
        \arrow[from=2-1, to=2-2]
        \arrow["{\phi^\prime}", from=2-2, to=2-3]
        \arrow["{p^\prime}", from=2-3, to=2-4]
        \arrow["p", from=1-3, to=1-4]
        \arrow[from=1-4, to=1-5]
        \arrow[from=2-4, to=2-5]
        \arrow["\phi", from=1-2, to=1-3]
        \arrow["r"', shift right=1, from=1-2, to=2-2]
        \arrow["{r^\prime}"', shift right=1, from=2-2, to=1-2]
        \arrow["f"', shift right=1, from=1-3, to=2-3]
        \arrow["{f^\prime}"', shift right=1, from=2-3, to=1-3]
        \arrow["{\id_A}", tail reversed, from=1-4, to=2-4]
      \end{tikzcd}\] By composing we obtain the following commutative diagrams: % https://q.uiver.app/?q=WzAsMjAsWzAsMCwiMCJdLFswLDEsIjAiXSxbNCwwLCIwIl0sWzQsMSwiMCJdLFsxLDAsIlIiXSxbMiwwLCJGIl0sWzMsMCwiQSJdLFszLDEsIkEiXSxbMiwxLCJGIl0sWzEsMSwiUiJdLFs2LDAsIjAiXSxbNiwxLCIwIl0sWzcsMCwiUl5cXHByaW1lIl0sWzgsMCwiRl5cXHByaW1lIl0sWzksMCwiQSJdLFs3LDEsIlJeXFxwcmltZSJdLFs4LDEsIkZeXFxwcmltZSJdLFs5LDEsIkEiXSxbMTAsMCwiMCJdLFsxMCwxLCIwIl0sWzAsNF0sWzEsOV0sWzksOCwiXFxwaGkiXSxbOCw3LCJwIl0sWzUsNiwicCJdLFs2LDJdLFs3LDNdLFs0LDUsIlxccGhpIl0sWzYsNywiXFxpZF9BIl0sWzEwLDEyXSxbMTEsMTVdLFsxMiwxMywiXFxwaGleXFxwcmltZSJdLFsxMywxNCwicF5cXHByaW1lIl0sWzE2LDE3LCJwXlxccHJpbWUiXSxbMTUsMTYsIlxccGhpXlxccHJpbWUiXSxbMTQsMThdLFsxNywxOV0sWzE0LDE3LCJcXGlkX0EiXSxbMTMsMTYsImZmXlxccHJpbWUiXSxbMTIsMTUsInJyXlxccHJpbWUiXSxbNCw5LCJyXlxccHJpbWUgciJdLFs1LDgsImZeXFxwcmltZSBmIl1d
      \[\begin{tikzcd}
        0 & R & F & A & 0 && 0 & {R^\prime} & {F^\prime} & A & 0 \\
        0 & R & F & A & 0 && 0 & {R^\prime} & {F^\prime} & A & 0
        \arrow[from=1-1, to=1-2]
        \arrow[from=2-1, to=2-2]
        \arrow["\phi", from=2-2, to=2-3]
        \arrow["p", from=2-3, to=2-4]
        \arrow["p", from=1-3, to=1-4]
        \arrow[from=1-4, to=1-5]
        \arrow[from=2-4, to=2-5]
        \arrow["\phi", from=1-2, to=1-3]
        \arrow["{\id_A}", from=1-4, to=2-4]
        \arrow[from=1-7, to=1-8]
        \arrow[from=2-7, to=2-8]
        \arrow["{\phi^\prime}", from=1-8, to=1-9]
        \arrow["{p^\prime}", from=1-9, to=1-10]
        \arrow["{p^\prime}", from=2-9, to=2-10]
        \arrow["{\phi^\prime}", from=2-8, to=2-9]
        \arrow[from=1-10, to=1-11]
        \arrow[from=2-10, to=2-11]
        \arrow["{\id_A}", from=1-10, to=2-10]
        \arrow["{ff^\prime}", from=1-9, to=2-9]
        \arrow["{rr^\prime}", from=1-8, to=2-8]
        \arrow["{r^\prime r}", from=1-2, to=2-2]
        \arrow["{f^\prime f}", from=1-3, to=2-3]
      \end{tikzcd}\] Observe that the identity maps of $R,R^\prime,F,F^\prime$ may be used in the appropriate places in the diagrams above instead of the ones provided and the diagrams would still commute. Again use the same lemma (twice) to obtain the maps $s\colon F\to R$, $s^\prime\colon F^\prime \to R^\prime$ with $r^\prime r = s\phi+\id_R$ and $rr^\prime = s^\prime\phi^\prime + \id_{R^\prime}$. [In the lemma $s(x)$ is defined to be the preimage $y$ of $(f^\prime f - \id_F)(x)$ under $\phi$, which is a homomorphism: if $s(x) = y$ and $s(x^\prime) = y^\prime$ then since $(f^\prime f - \id_F)(x+x^\prime) = (f^\prime f - \id_F)(x)+(f^\prime f - \id_F)(x^\prime) = \phi(y) + \phi(y^\prime) =\phi(y+y^\prime)$ we have with $\phi$ injective that $y+y^\prime$ is indeed $s(x+x^\prime)$. Similarly $s^\prime$ is also a homomorphism.]

      The $\Ext$ groups are given by the cokernels of $\phi^\ast$ and ${\phi^\prime}^\ast$. The maps $r^\ast,{r^\prime}^\ast$ induce well defined maps on these cokernels, and so we show that they are inverse to each other. For $[f]\in \coker \phi^\ast$ we have $(r^\ast{r^\prime}^\ast)[f] = [fr^\prime r] = [f(s\phi+\id_R)] = [\phi^\ast(fs) + f] = [f]$ and for $[f^\prime]\in \coker {\phi^\prime}^\ast$ we have $({r^\prime}^\ast r^\ast)[f^\prime] = [f^\prime rr^\prime] = [f(s^\prime\phi^\prime+\id_{R^\prime})] = [{\phi^\prime}^\ast(f^\prime s^\prime) + f^\prime] = [f^\prime]$. It follows that the $\Ext$ groups are isomorphic as needed, so that any choice of free resolution (of the form $0 \to R \xrightarrow{\phi} F \xrightarrow{p} A \to 0$) will define the $\Ext$ group up to isomorphism.
    \end{proof}
    \item For fixed $B$, show that $\Ext(-,B)$ is a contravariant functor. For fixed $A$, show that $\Ext(A,-)$ is a functor. \begin{proof}
      The functor $\Ext(-,B)$ takes Abelian groups $A$ to the group $\Ext(A,B) = \coker \phi^\ast$ given a free resolution $0 \to R \xrightarrow{\phi} F \xrightarrow{p} A \to 0$ of $A$. Given free resolutions $0 \to R \xrightarrow{\phi} F \xrightarrow{p} A \to 0$ and $0 \to R^\prime \xrightarrow{\phi^\prime} F^\prime \xrightarrow{p^\prime} A^\prime \to 0$, the functor $\Ext(-,B)$ takes a homomorphism $f\colon A\to A^\prime$ to the induced map of cohomology groups $r^\ast\colon $ obtained via Lemma 11.8.1. Note that any choice of $r$ in the lemma works since chain homotopic maps induce the same maps of cohomology. 

      We check that the identity map is sent to the identity map of the $\Ext$ group and that composition is preserved (but reversed since $\Ext(-,B)$ is contravariant). The identity map is sent to the identity map because any such $r$ obtained via the lemma is chain homotopic to the identity map; as both $r$ and $\id_R$ make the diagram % https://q.uiver.app/?q=WzAsMTAsWzAsMCwiMCJdLFswLDEsIjAiXSxbNCwwLCIwIl0sWzQsMSwiMCJdLFsxLDAsIlIiXSxbMiwwLCJGIl0sWzMsMCwiQSJdLFszLDEsIkEiXSxbMiwxLCJGIl0sWzEsMSwiUiJdLFswLDRdLFsxLDldLFs5LDgsIlxccGhpIl0sWzgsNywicCJdLFs1LDYsInAiXSxbNiwyXSxbNywzXSxbNCw1LCJcXHBoaSJdLFs2LDcsIlxcaWRfQSJdLFs0LDksInIiLDAseyJvZmZzZXQiOi0xfV0sWzUsOCwiZiJdLFs0LDksIlxcaWRfUiIsMix7Im9mZnNldCI6MX1dXQ==
      \[\begin{tikzcd}
        0 & R & F & A & 0 \\
        0 & R & F & A & 0
        \arrow[from=1-1, to=1-2]
        \arrow[from=2-1, to=2-2]
        \arrow["\phi", from=2-2, to=2-3]
        \arrow["p", from=2-3, to=2-4]
        \arrow["p", from=1-3, to=1-4]
        \arrow[from=1-4, to=1-5]
        \arrow[from=2-4, to=2-5]
        \arrow["\phi", from=1-2, to=1-3]
        \arrow["{\id_A}", from=1-4, to=2-4]
        \arrow["r", shift left=1, from=1-2, to=2-2]
        \arrow["f", from=1-3, to=2-3]
        \arrow["{\id_R}"', shift right=1, from=1-2, to=2-2]
      \end{tikzcd}\] commute. So $r$ may be taken to be the identity map $\id_R$, and dualizing plus passing to the quotient do not change the action of the map so we still obtain the identity. 

      Given $f\colon A\to A_1$ and $g\colon A_1 \to A_2$ as well as free resolutions $0 \to R \to F \to A \to 0$, $0 \to R_1 \to F_1 \to A_1 \to 0$, and $0 \to R_2 \to F_2 \to A_2 \to 0$. Use the lemma three times to obtain maps $r\colon R\to R_1$ from $f$, $r_1\colon R_1\to R_2$ from $g$, and $r^\prime\colon R \to R_2$ from $gf$. In this case observe that $r_1r$ is chain homotopic to $r^\prime$ since both maps make the following diagram % https://q.uiver.app/?q=WzAsMTAsWzAsMCwiMCJdLFswLDEsIjAiXSxbNCwwLCIwIl0sWzQsMSwiMCJdLFsxLDAsIlIiXSxbMiwwLCJGIl0sWzMsMCwiQSJdLFszLDEsIkFfMiJdLFsyLDEsIkZfMiJdLFsxLDEsIlJfMiJdLFswLDRdLFsxLDldLFs5LDgsIlxccGhpIl0sWzgsNywicCJdLFs1LDYsInAiXSxbNiwyXSxbNywzXSxbNCw1LCJcXHBoaSJdLFs2LDcsImdmIl0sWzQsOSwicl5cXHByaW1lIiwwLHsib2Zmc2V0IjotMX1dLFs1LDhdLFs0LDksInJfMXIiLDIseyJvZmZzZXQiOjF9XV0=
      \[\begin{tikzcd}
        0 & R & F & A & 0 \\
        0 & {R_2} & {F_2} & {A_2} & 0
        \arrow[from=1-1, to=1-2]
        \arrow[from=2-1, to=2-2]
        \arrow["\phi", from=2-2, to=2-3]
        \arrow["p", from=2-3, to=2-4]
        \arrow["p", from=1-3, to=1-4]
        \arrow[from=1-4, to=1-5]
        \arrow[from=2-4, to=2-5]
        \arrow["\phi", from=1-2, to=1-3]
        \arrow["gf", from=1-4, to=2-4]
        \arrow["{r^\prime}", shift left=1, from=1-2, to=2-2]
        \arrow[from=1-3, to=2-3]
        \arrow["{r_1r}"', shift right=1, from=1-2, to=2-2]
      \end{tikzcd}\] commute. So they induce the same maps of cohomology, from which it follows that $\Ext (gf, B)$ is equal to the composition $\Ext(f,B)\circ \Ext (g, B)$ as the $\Hom$ functor $\Hom(-,B)$ is contravariant and passing to the quotient preserves composition here. Hence $\Ext (-, B)$ is a contravariant functor.

      Let $A$ be an Abelian group with free resolution $0 \to R \xrightarrow{\phi} F \xrightarrow{p} A \to 0$. Then $\Ext(A,-)$ takes $B$ to $\coker\phi^\ast_B$, where $\phi^\ast_B\colon \Hom(F,B)\to \Hom(R,B)$ is the dual map of $\phi$ (with respect to $B$, in a sense). A map $g\colon B\to B_1$ is taken to the map $g^\ast \colon \Ext(A,B)\to \Ext(A,B_1)$ by post composition, taking $[f]$ to $[gf]$ (and this is well defined since for $\phi^\ast_Bf = f\phi$ we have $gf\phi = \phi^\ast_{B_1}(gf)$). As before we can take any free resolution of $A$ since the $\Ext$ groups coming from each resolution are isomorphic, so we fix one and stick with it. It is clear that the identity map is taken to the identity map since postcomposition with the identity is the identity map of $\Ext$ groups. Given the maps $g\colon B\to B_1$ and $h\colon B_1\to B_2$, the composite map $hg$ is taken to the map $\Ext(A,B)\to \Ext(A,B_2)$ which takes $[f]$ to $[hgf] = [h(gf)]$. It follows that $\Ext(A,hg)$ is the composite $\Ext(A,h)\circ\Ext(A,g)$. Thus $\Ext(A,-)$ is a covariant functor.
    \end{proof}
\end{enumerate}
\end{document}