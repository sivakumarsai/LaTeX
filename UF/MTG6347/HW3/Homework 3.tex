\documentclass[11pt]{article}

% packages
\usepackage{physics}
% margin spacing
\usepackage[top=1in, bottom=1in, left=0.5in, right=0.5in]{geometry}
\usepackage{hanging}
\usepackage{amsfonts, amsmath, amssymb, amsthm}
\usepackage{systeme}
\usepackage[none]{hyphenat}
\usepackage{fancyhdr}
\usepackage[nottoc, notlot, notlof]{tocbibind}
\usepackage{graphicx}
\graphicspath{{./images/}}
\usepackage{float}
\usepackage{siunitx}
\usepackage{esint}
\usepackage{cancel}
\usepackage{enumitem}
%\usepackage{tikz-cd}
\usepackage{quiver}

% permutations (second line is for spacing)
\usepackage{permute}
\renewcommand*\pmtseparator{\,}

% colors
\usepackage{xcolor}
\definecolor{p}{HTML}{FFDDDD}
\definecolor{g}{HTML}{D9FFDF}
\definecolor{y}{HTML}{FFFFCF}
\definecolor{b}{HTML}{D9FFFF}
\definecolor{o}{HTML}{FADECB}
%\definecolor{}{HTML}{}

% \highlight[<color>]{<stuff>}
\newcommand{\highlight}[2][p]{\mathchoice%
  {\colorbox{#1}{$\displaystyle#2$}}%
  {\colorbox{#1}{$\textstyle#2$}}%
  {\colorbox{#1}{$\scriptstyle#2$}}%
  {\colorbox{#1}{$\scriptscriptstyle#2$}}}%

% header/footer formatting
\pagestyle{fancy}
\fancyhead{}
\fancyfoot{}
\fancyhead[L]{MTG6347 Topology}
\fancyhead[C]{Homework 3}
\fancyhead[R]{Sai Sivakumar}
\fancyfoot[R]{\thepage}
\renewcommand{\headrulewidth}{1pt}

% paragraph indentation/spacing
\setlength{\parindent}{0cm}
\setlength{\parskip}{10pt}
\renewcommand{\baselinestretch}{1.25}

% extra commands defined here
\newcommand{\br}[1]{\left(#1\right)}
\newcommand{\sbr}[1]{\left[#1\right]}
\newcommand{\cbr}[1]{\left\{#1\right\}}

\newcommand{\catname}[1]{{\textbf{#1} }}
\newcommand{\Set}{\catname{Set}}
\newcommand{\Top}{\catname{Top}}
\DeclareMathOperator{\Int}{Int}
\DeclareMathOperator{\Bd}{Bd}
\DeclareMathOperator{\id}{id}
\DeclareMathOperator{\im}{im}
\DeclareMathOperator{\coker}{coker}
\DeclareMathOperator{\Aut}{Aut}
\DeclareMathOperator{\Hom}{Hom}
\DeclareMathOperator{\Ext}{Ext}

% bracket notation for inner product
\usepackage{mathtools}

\DeclarePairedDelimiterX{\abr}[1]{\langle}{\rangle}{#1}
\DeclarePairedDelimiter{\ceil}{\lceil}{\rceil}
\DeclarePairedDelimiter{\floor}{\lfloor}{\rfloor}

% set page count index to begin from 1
\setcounter{page}{1}

\begin{document}
\begin{enumerate}
    \item Compute the cohomology with $\mathbb{Z}$ and $\mathbb{Z}/2\mathbb{Z}$ coefficients of the following using a
    cellular ($\Delta$-complex) structure with two $2$-cells: \begin{enumerate}
      \item $S^1 \times S^1$,
      \begin{table}[h]
        \centering
        \begin{tabular}{cc|c|c|c}
          $X = S^1 \times S^1$\hspace*{3cm} & \hspace*{5cm} & $H^n(X;G)$ & $G = \mathbb{Z}$ & $G = \mathbb{Z}/2\mathbb{Z}$  \\ \hline
         & $\delta v^\ast = 0$, $\delta e_0^\ast = f_0^\ast - f_1^\ast$ & $H^0(X;G)$ & $\mathbb{Z}$ & $\mathbb{Z}/2\mathbb{Z}$  \\
         & $\delta e_1^\ast= f_0^\ast-f_1^\ast$, $\delta e_2^\ast= -f_0^\ast+f_1^\ast$ & $H^1(X;G)$ & $\mathbb{Z}^2$ & $(\mathbb{Z}/2\mathbb{Z})^2$  \\
         & $\delta f_0^\ast= 0$, $\delta f_1^\ast= 0$ & $H^2(X;G)$ & $\mathbb{Z}$ & $\mathbb{Z}/2\mathbb{Z}$ \\
         &
        \end{tabular}
        \end{table}

        We have the chain complex \[0\to C^0(X) = \abr{v^\ast}\xrightarrow{g}C^1(X) = \abr{e_0^\ast, e_1^\ast, e_2^\ast}\xrightarrow{h}C^2(X) =\abr{f_0^\ast, f_1^\ast}\to 0\] where $g$ and $h$ are the $\delta$ maps above labeled for convenience.
        Based on the information in the table above we can deduce that for $G =\mathbb{Z}$, \[\ker g = \abr{v^\ast}, \im g = 0, \ker h = \abr{e_2^\ast+e_0^\ast,e_2^\ast+e_1^\ast}, \im h = \abr{f_0^\ast-f_1^\ast}.\] Thus $H^0(X) = \abr{v^\ast}\cong \mathbb{Z}$, $H^1(X) = \abr{e_2^\ast+e_0^\ast,e_2^\ast+e_1^\ast} \cong \mathbb{Z}^2$, and $H^2(X) = \abr{f_0^\ast, f_1^\ast}/\abr{f_0^\ast-f_1^\ast} \cong \abr{f_0^\ast-f_1^\ast, f_1^\ast}/\abr{f_0^\ast-f_1^\ast}\cong \abr{f_1^\ast}\cong \mathbb{Z}$. Further cohomology groups are all zero.
        
        When $G = \mathbb{Z}/2\mathbb{Z}$, we have \[\ker g = \abr{v^\ast}, \im g = 0, \ker h = \abr{e_2^\ast+e_0^\ast,e_2^\ast+e_1^\ast}, \im h = \abr{f_0^\ast-f_1^\ast}.\] Thus $H^0(X) = \abr{v^\ast}\cong \mathbb{Z}/2\mathbb{Z}$, $H^1(X) = \abr{e_2^\ast+e_0^\ast,e_2^\ast+e_1^\ast} \cong (\mathbb{Z}/2\mathbb{Z})^2$, and $H^2(X) = \abr{f_0^\ast, f_1^\ast}/\abr{f_0^\ast-f_1^\ast} \cong \abr{f_0^\ast-f_1^\ast, f_1^\ast}/\abr{f_0^\ast-f_1^\ast}\cong \abr{f_1^\ast}\cong \mathbb{Z}/2\mathbb{Z}$. Further cohomology groups are all zero.
      \item $\mathbb{RP}^2$, and
      \begin{table}[h]
        \centering
        \begin{tabular}{cc|c|c|c}
          $X = \mathbb{RP}^2$\hspace*{3cm} & \hspace*{5cm} & $H^n(X;G)$ & $G = \mathbb{Z}$ & $G = \mathbb{Z}/2\mathbb{Z}$  \\ \hline
         & $\delta v^\ast= -e_0^\ast-e_1^\ast$, $\delta w^\ast= e_0^\ast + e_1^\ast$ & $H^0(X;G)$ & $\mathbb{Z}$ & $\mathbb{Z}/2\mathbb{Z}$  \\
         & $\delta e_0^\ast= f_0^\ast+f_1^\ast$, $\delta e_1^\ast= -f_0^\ast-f_1^\ast$ & $H^1(X;G)$ & $0$ & $\mathbb{Z}/2\mathbb{Z}$  \\
         & $\delta e_2^\ast= -f_0^\ast+f_1^\ast$, $\delta f_0^\ast= 0$ & $H^2(X;G)$ & $\mathbb{Z}/2\mathbb{Z}$ & $\mathbb{Z}/2\mathbb{Z}$  \\
         & $\delta f_1^\ast= 0$ &  &  &  
        \end{tabular}
        \end{table}

        We have the chain complex \[0\to C^0(X) = \abr{v^\ast,w^\ast}\xrightarrow{g}C^1(X) = \abr{e_0^\ast, e_1^\ast, e_2^\ast}\xrightarrow{h}C^2(X) =\abr{f_0^\ast, f_1^\ast}\to 0\] where $g$ and $h$ are the $\delta$ maps above labeled for convenience. Based on the information in the table above we can deduce that for $G =\mathbb{Z}$, \[\ker g = \abr{v^\ast + w^\ast}, \im g = \abr{e_0^\ast + e_1^\ast}, \ker h = \abr{e_0^\ast+e_1^\ast}, \im h = \abr{f_0^\ast+f_1^\ast,-f_0^\ast+f_1^\ast}.\] Thus $H^0(X) = \abr{v^\ast + w^\ast}\cong \mathbb{Z}$, $H^1(X) = \abr{e_0^\ast + e_1^\ast}/\abr{e_0^\ast + e_1^\ast} \cong 0$, and $H^2(X) = \abr{f_0^\ast, f_1^\ast}/\abr{f_0^\ast+f_1^\ast,-f_0^\ast+f_1^\ast} \cong \abr{f_0^\ast+f_1^\ast, f_1^\ast}/\abr{f_0^\ast+f_1^\ast,2f_1^\ast}\cong \abr{f_1^\ast}/\abr{2f_1^\ast}\cong \mathbb{Z}/2\mathbb{Z}$. Further cohomology groups are all zero.
        
        When $G = \mathbb{Z}/2\mathbb{Z}$, we have \[\ker g = \abr{v^\ast + w^\ast}, \im g = \abr{e_0^\ast + e_1^\ast}, \ker h = \abr{e_0^\ast+e_1^\ast,e_0^\ast + e_2^\ast}, \im h = \abr{f_0^\ast+f_1^\ast,-f_0^\ast+f_1^\ast}.\] Thus $H^0(X) = \abr{v^\ast + w^\ast}\cong \mathbb{Z}/2\mathbb{Z}$, $H^1(X) = \abr{e_0^\ast+e_1^\ast,e_0^\ast + e_2^\ast}/\abr{e_0^\ast + e_1^\ast} \cong \abr{e_0^\ast + e_2^\ast}\cong \mathbb{Z}/2\mathbb{Z}$, and $H^2(X) = \abr{f_0^\ast, f_1^\ast}/\abr{f_0^\ast+f_1^\ast,-f_0^\ast+f_1^\ast} \cong \abr{f_0^\ast+f_1^\ast, f_1^\ast}/\abr{f_0^\ast+f_1^\ast,2f_1^\ast}\cong \abr{f_1^\ast}\cong \mathbb{Z}/2\mathbb{Z}$. Further cohomology groups are all zero.
      \item the Klein bottle.
      \begin{table}[h]
        \centering
        \begin{tabular}{cc|c|c|c}
          $X = $ Klein bottle\hspace*{3cm} & \hspace*{5cm} & $H^n(X;G)$ & $G = \mathbb{Z}$ & $G = \mathbb{Z}/2\mathbb{Z}$  \\ \hline
         & $\delta v^\ast= 0$, $\delta e_0^\ast= f_0^\ast-f_1^\ast$ & $H^0(X;G)$ & $\mathbb{Z}$ & $\mathbb{Z}/2\mathbb{Z}$  \\
         & $\delta e_1^\ast= -f_0^\ast-f_1^\ast$, $\delta e_2^\ast= -f_0^\ast+f_1^\ast$ & $H^1(X;G)$ & $\mathbb{Z}$ & $(\mathbb{Z}/2\mathbb{Z})^2$  \\
         & $\delta f_0^\ast= 0$, $\delta f_1^\ast= 0$ & $H^2(X;G)$ & $\mathbb{Z}/2\mathbb{Z}$ & $\mathbb{Z}/2\mathbb{Z}$ \\
         &
        \end{tabular}
        \end{table}

        We have the chain complex \[0\to C^0(X) = \abr{v^\ast}\xrightarrow{g}C^1(X) = \abr{e_0^\ast, e_1^\ast, e_2^\ast}\xrightarrow{h}C^2(X) =\abr{f_0^\ast, f_1^\ast}\to 0\] where $g$ and $h$ are the $\delta$ maps above labeled for convenience.
        Based on the information in the table above we can deduce that for $G =\mathbb{Z}$, \[\ker g = \abr{v^\ast}, \im g = 0, \ker h = \abr{e_2^\ast+e_0^\ast}, \im h = \abr{f_0^\ast-f_1^\ast, -f_0^\ast-f_1^\ast}.\] Thus $H^0(X) = \abr{v^\ast}\cong \mathbb{Z}$, $H^1(X) = \abr{e_2^\ast+e_0^\ast} \cong \mathbb{Z}$, and $H^2(X) = \abr{f_0^\ast, f_1^\ast}/\abr{f_0^\ast-f_1^\ast, -f_0^\ast-f_1^\ast} \cong \abr{f_0^\ast-f_1^\ast, f_1^\ast}/\abr{f_0^\ast-f_1^\ast,2f_1^\ast}\cong \abr{f_1^\ast}/\abr{2f_1^\ast}\cong \mathbb{Z}/2\mathbb{Z}$. Further cohomology groups are all zero.
        
        When $G = \mathbb{Z}/2\mathbb{Z}$, we have \[\ker g = \abr{v^\ast}, \im g = 0, \ker h = \abr{e_2^\ast+e_0^\ast,e_0^\ast+e_1^\ast}, \im h = \abr{f_0^\ast-f_1^\ast, -f_0^\ast-f_1^\ast}.\] Thus $H^0(X) = \abr{v^\ast}\cong \mathbb{Z}/2\mathbb{Z}$, $H^1(X) = \abr{e_2^\ast+e_0^\ast,e_0^\ast+e_1^\ast} \cong (\mathbb{Z}/2\mathbb{Z})^2$, and $H^2(X) = \abr{f_0^\ast, f_1^\ast}/\abr{f_0^\ast-f_1^\ast, -f_0^\ast-f_1^\ast} \cong \abr{f_0^\ast-f_1^\ast, f_1^\ast}/\abr{f_0^\ast-f_1^\ast,2f_1^\ast}\cong \abr{f_1^\ast}\cong \mathbb{Z}/2\mathbb{Z}$. Further cohomology groups are all zero.
    \end{enumerate}
    \item For a chain complex $C$, define $\kappa : H^n(C; G) \to \Hom(H_n(C),G)$ by $\kappa([\alpha])([\beta]) = \alpha(\beta)$. Show that this is well defined. \begin{proof}
      Representatives for a given equivalence class in $H^n(C; G)$ differ by a coboundary $\gamma\in \im \delta^{n-1}\colon C^{n-1}\to C^n$, so that $\gamma = \delta^{n-1}\gamma^\prime$ for some $\gamma^\prime\in C^{n-1}$. We have for $[\alpha]\in H^n(C;G)$ and $[\beta]\in H_n(C)$ that \[(\alpha + \delta^{n-1}\gamma^\prime)(\beta) = \alpha(\beta) + \delta^{n-1}\gamma^\prime(\beta) = \alpha(\beta) + \gamma^\prime(\partial_n\beta) = \alpha(\beta) + \gamma^\prime(0) = \alpha(\beta)\] since $\beta\in \ker \partial_n$.

      Similarly, representatives for a given equivalence class in $H_n(C)$ differ by a boundary $\gamma \in \im \partial_{n+1}$, so that $\gamma = \partial_{n+1}\gamma^\prime$. Then for $[\alpha]\in H^n(C;G)$ we have \[\alpha(\beta + \partial_{n+1}\gamma^\prime) = \alpha(\beta) + \alpha(\partial_{n+1}\gamma^\prime) = \delta^n\alpha(\gamma^\prime) = 0(\gamma^\prime) = 0\] since $\alpha\in \ker \delta^n$.

      Thus $\kappa$ as defined above is well defined.
    \end{proof}
    \item Let $A,B$ be Abelian groups. Let $0 \to R \xrightarrow{\phi} F \xrightarrow{p} A \to 0$ be a free resolution. Let $\Ext(A,B) = \coker(\phi^\ast)$. Show that this cokernel does not depend on the choice of free resolution.
    \item For fixed $B$, show that $\Ext(-,B)$ is a contravariant functor. For fixed $A$, show that $\Ext(A,-)$ is a functor.
\end{enumerate}
\end{document}