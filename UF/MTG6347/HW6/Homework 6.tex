\documentclass[11pt]{article}

% packages
\usepackage{physics}
% margin spacing
\usepackage[top=1in, bottom=1in, left=0.5in, right=0.5in]{geometry}
\usepackage{hanging}
\usepackage{amsfonts, amsmath, amssymb, amsthm}
\usepackage{systeme}
\usepackage[none]{hyphenat}
\usepackage{fancyhdr}
\usepackage[nottoc, notlot, notlof]{tocbibind}
\usepackage{graphicx}
\graphicspath{{./images/}}
\usepackage{float}
\usepackage{siunitx}
\usepackage{esint}
\usepackage{cancel}
\usepackage{enumitem}
%\usepackage{tikz-cd}
\usepackage{quiver}

% permutations (second line is for spacing)
\usepackage{permute}
\renewcommand*\pmtseparator{\,}

% colors
\usepackage{xcolor}
\definecolor{p}{HTML}{FFDDDD}
\definecolor{g}{HTML}{D9FFDF}
\definecolor{y}{HTML}{FFFFCF}
\definecolor{b}{HTML}{D9FFFF}
\definecolor{o}{HTML}{FADECB}
%\definecolor{}{HTML}{}

% \highlight[<color>]{<stuff>}
\newcommand{\highlight}[2][p]{\mathchoice%
  {\colorbox{#1}{$\displaystyle#2$}}%
  {\colorbox{#1}{$\textstyle#2$}}%
  {\colorbox{#1}{$\scriptstyle#2$}}%
  {\colorbox{#1}{$\scriptscriptstyle#2$}}}%

% header/footer formatting
\pagestyle{fancy}
\fancyhead{}
\fancyfoot{}
\fancyhead[L]{MTG6347 Topology}
\fancyhead[C]{Homework 6}
\fancyhead[R]{Sai Sivakumar}
\fancyfoot[R]{\thepage}
\renewcommand{\headrulewidth}{1pt}

% paragraph indentation/spacing
\setlength{\parindent}{0cm}
\setlength{\parskip}{10pt}
\renewcommand{\baselinestretch}{1.25}

% extra commands defined here
\newcommand{\br}[1]{\left(#1\right)}
\newcommand{\sbr}[1]{\left[#1\right]}
\newcommand{\cbr}[1]{\left\{#1\right\}}

\newcommand{\catname}[1]{{\textbf{#1} }}
\newcommand{\Set}{\catname{Set}}
\newcommand{\Top}{\catname{Top}}
\DeclareMathOperator{\Int}{Int}
\DeclareMathOperator{\Bd}{Bd}
\DeclareMathOperator{\id}{id}
\DeclareMathOperator{\im}{im}
\DeclareMathOperator{\coker}{coker}
\DeclareMathOperator{\Aut}{Aut}
\DeclareMathOperator{\Hom}{Hom}
\DeclareMathOperator{\Ext}{Ext}

% bracket notation for inner product
\usepackage{mathtools}

\DeclarePairedDelimiterX{\abr}[1]{\langle}{\rangle}{#1}
\DeclarePairedDelimiter{\ceil}{\lceil}{\rceil}
\DeclarePairedDelimiter{\floor}{\lfloor}{\rfloor}

% set page count index to begin from 1
\setcounter{page}{1}

\begin{document}
\begin{enumerate}
    \item Prove the following. [Throughout, interpret cochains as mapping into the ring $R$ to suppress having to tensor integers with elements of $R$.]\begin{enumerate}
        \item For a continuous map $f\colon X\to Y$, $f_{\#}(f^{\#}\varphi\cap c) = \varphi\cap f_{\#}c$. \begin{proof}
          Let $\varphi\in S^p(Y)$ and $c = \sum_i c_i\sigma_i\in S_{p+q}(X)$. Then we have \begin{align*}
            f_{\#}(f^{\#}\varphi\cap c) &= f_{\#}(f^{\#}\varphi\cap \sum_i c_i\sigma_i)\\
            &=  f_{\#}(\sum_i c_i(-1)^{pq}[(f^{\#}\varphi)(\sigma_i|_{[e_q,\dots,e_{p+q}]})]\sigma_i|_{[e_q,\dots,e_{p+q}]})\\
            &=  \sum_i c_i(-1)^{pq}[\varphi((f_{\#}\sigma_i)|_{[e_q,\dots,e_{p+q}]})](f_{\#}\sigma_i)|_{[e_q,\dots,e_{p+q}]}\\ 
            &= (-1)^{pq}[\varphi((f_{\#}(\sum_i c_i\sigma_i))|_{[e_q,\dots,e_{p+q}]})](f_{\#}(\sum_i c_i\sigma_i))|_{[e_q,\dots,e_{p+q}]}\\
            &= (-1)^{pq}[\varphi((f_{\#}c)|_{[e_q,\dots,e_{p+q}]})](f_{\#}c)|_{[e_q,\dots,e_{p+q}]} \\ 
            &= \varphi\cap f_{\#}c
          \end{align*} as desired.
        \end{proof}
        \item $\partial(\varphi\cap c) = \delta\varphi\cap c + (-1)^{\abs{\varphi}}\varphi\cap \partial c$ \begin{proof}
          Let $\varphi\in S^p(X)$ (so $\abs{\varphi}=p$) and $c = \sum_i c_i\sigma_i\in S_{p+q}(X)$. Then we have \begin{align}
            \partial(\varphi\cap c) &= \sum_i\sum_{j=0}^q c_i(-1)^{pq}(-1)^j\varphi(\sigma_i|_{[e_q,\dots,e_{p+q}]})\sigma_i|_{[e_0,\dots,\hat e_j,\dots,e_q]},\\
            \delta \varphi\cap c &= \sum_i \sum_{j=q-1}^{p+q}c_i(-1)^{(p+1)q}(-1)^{j-q+1}\varphi(\sigma_i|_{[e_{q-1},\dots,\hat e_j,\dots,e_{p+q}]})\sigma_i|_{[e_0,\dots,e_{q-1}]},\\
            \text{and } (-1)^{\abs{\varphi}}\varphi\cap \partial c &= \sum_i\sum_{j=0}^{q-1}c_i(-1)^p(-1)^j(-1)^{p(q-1)}\varphi(\sigma_i|_{[e_q,\dots,e_{p+q}]})\sigma_i|_{[e_0,\dots,\hat e_j,\dots,e_q]} \\
            &\hspace*{2pc}+\sum_i\sum_{j=q}^{p+q}c_i(-1)^p(-1)^j(-1)^{p(q-1)}\varphi(\sigma_i|_{[e_{q-1},\dots,\hat e_j,\dots,e_{p+q}]})\sigma_i|_{[e_0,\dots,e_{q-1}]}.
          \end{align}
          Collecting exponents and adding terms (2),(3), and (4) together we find that (2) and (4) almost cancel out, leaving (3) plus the term $\sum_ic_i(-1)^{pq}(-1)^{q}\varphi(\sigma_i|_{[e_q,\dots,e_{p+q}]})\sigma_i|_{[e_0,\dots,e_{q-1}]}$. But this sum is just the sum in (3) taken to $q$ instead of $q-1$; that is, what remains after the sum is taken is $\partial(\varphi\cap c) = \sum_i\sum_{j=0}^q c_i(-1)^{pq}(-1)^j\varphi(\sigma_i|_{[e_q,\dots,e_{p+q}]})\sigma_i|_{[e_0,\dots,\hat e_j,\dots,e_q]}$ as needed.
        \end{proof}
        \item $(\varphi\cup \psi)\cap c = \varphi\cap (\psi\cap c)$ \begin{proof}
          Let $\varphi\in S^k(X)$ and $\psi\in S^l(X)$ with $k+l = p$. Let $c = \sum_i c_i\sigma_i\in S_{p+q}(X)$. Then $\varphi\cup\psi\in S^p(X)$ and we have \begin{align*}
            (\varphi\cup\psi)\cap c &= \sum_i c_i (-1)^{pq}(\varphi\cup\psi)(\sigma_i|_{[e_q,\dots,e_{p+q}]})\sigma_i|_{[e_0,\dots,e_q]}\\
            &= \sum_i c_i (-1)^{pq}[(-1)^{kl}\varphi(\sigma_i|_{[e_q,\dots,e_{k+q}]})\psi(\sigma_i|_{[e_{k+q},\dots,e_{p+q}]})]\sigma_i|_{[e_0,\dots,e_q]}\\
            &= \sum_i c_i (-1)^{pq-ql}\varphi(\sigma_i|_{[e_q,\dots,e_{k+q}]})(-1)^{kl+ql}\psi(\sigma_i|_{[e_{k+q},\dots,e_{p+q}]})\sigma_i|_{[e_0,\dots,e_q]}\\
            &= \sum_i c_i \varphi\cap ((-1)^{kl+ql}\psi(\sigma_i|_{[e_{k+q},\dots,e_{p+q}]})\sigma_i|_{[e_0,\dots,e_{k+q}]})\\
            &= \sum_i c_i \varphi\cap (\psi\cap\sigma_i)\\
            &= \varphi\cap(\psi\cap\sigma)
          \end{align*} as needed.
        \end{proof}
        \item $1\cap c = c$.\begin{proof}
          Let $1\in S^0(X)$ be the cochain that takes every chain to $1_R$ and let $c = \sum_ic_i\in S_n(X)$. Then \[1\cap c = \sum_i c_i 1(\sigma_i|_{[e_n]})\sigma_i|_{[e_0,\dots,e_n]} = \sum_i c_i\sigma_i = c\] as needed.
        \end{proof}
    \end{enumerate}
    \item Let $N_g$ denote the non-orientable surface of genus $g$. Consider homology and cohomology with coefficients in $\mathbb{Z}_2$. Sketch a $\Delta$-complex structure for $N_g$ for small $g$, and use the cap product to determine the Poincar\'e duality for arbitrary $g$. \begin{proof}
      \vspace*{5cm}

      The following is from Homework 4. The fundamental class $[N_g]$ generating $H_2(N_g)\cong \mathbb{Z}_2$ is given by $\sum_i T_i$ and the cocycles generating $H^1(N_g)\cong \mathbb{Z}_2^g$ are the functions $\alpha_i$ which take the value $0$ for edges not intersecting the corresponding arc $\alpha_i$ in the pictures and $1$ for edges that do intersect $\alpha_i$ -- that is, $\alpha_j(a_i) = \delta_{ij}$ and $\alpha_j(f_i) = \delta_{i(2j)}$.

      Then we have for $1\leq j \leq g$ that \[\alpha_j\cap [N_g] = \sum_{i=1}^{2g} \alpha_j\cap T_i = \sum_{i=1}^g (\alpha_j([a_i])[f_{2i-1}] + \alpha_j(a_i)[f_{2i}])= \sum_{i=1}^g \delta_{ij}[f_{2i-1} + f_{2i}] = [f_{2j-1} + f_{2j}],\] which means that $\alpha_j$ is Poincar\'e dual to $[f_{2j-1} + f_{2j}]$. But $\partial T_{2j-1} = f_{2j-1} + a_j + f_{2j}$ so that $[f_{2j-1} + f_{2j}] = [-a_j] = [a_j]$. It follows that $\alpha_j$ is Poincar\'e dual to $[a_j]$.
    \end{proof}
    \item (Hatcher 3.3.20) Show that $H^0_c(X;G)=0$ if $X$ is path connected and noncompact. \begin{proof}
      Let $\varphi$ be any compactly supported $0$-cocycle, and let $K\subseteq X$ be its support. Let $x\in K$ and take any path $\gamma\colon \Delta^1\to X$ from $x$ to a point $y\not\in K$. Then as $\varphi$ is a cocycle, $\delta \varphi \gamma = \varphi(y) -\varphi(x) = 0$. Since $\varphi(y) =0$ by assumption then we must have that $\varphi(x) = 0$ also, and since $x\in K$ was arbitrary we must have $\varphi =0$.
    \end{proof}
    \item (Hatcher 3.3.21) For a space $X$, let $X^+$ be the one point compactification. If the added point, denoted $\infty$, has a neighborhood in $X^+$ that is a cone with $\infty$ the cone point, show that the evident map $H^n_c(X;G)\to H^n(X^+,\infty;G)$ is an isomorphism for all $n$. [Question: Does this result hold when $X = \mathbb{Z}\times \mathbb{R}$?] \begin{proof}
      There exists a collection of  neighborhoods of the form $Y$ in $X$ for which $(Y\times I)/(Y\times 0)$ embeds into $X^+$ with the point $Y\times 0$ mapping into $\infty$, and one can imagine that these cones could get smaller and smaller about $\infty$. Now consider the closed compact sets of $X$ which are complements of these decreasing cones. These form an increasing ``subnet'' of compact sets as every compact set is contained in some compact set of this form -- we can take the complement of a small enough cone about $\infty$. We will take a directed limit along this ``subnet'' in the conclusion.

      For any closed compact set $K$ of the form above, by excision (excising $\infty$) we have the isomorphism $H^n(X^+,X^+\setminus K;G)\cong H^n(X,X\setminus K;G)$, and in the reduced long exact sequence of a pair we have \[\cdots \to \tilde H^{n-1}(X^+\setminus K;G)\to H^n(X^+,X^+\setminus K;G)\to \tilde H^n(X^+;G)\to \tilde H^n(X^+\setminus K;G)\to\cdots.\] Since cones are contractible we have that the groups $H^i(X^+\setminus K;G)$ are trivial, and by exactness we have the isomorphisms $H^n(X^+,X^+\setminus K;G)\to \tilde H^n(X^+;G)$. Then in summary we have the isomorphisms $H^n(X^+,\infty ;G)\cong \tilde H^n(X^+;G)\cong H^n(X^+,X^+\setminus K;G)\cong H^n(X,X\setminus K;G)$. Then take the directed limit along the ``subnet'' of the closed compact $K$ which are complements of decreasingly smaller cones about $\infty$, and obtain that $H^n(X^+,\infty;G)\cong H^n_c(X;G)$ as needed.

      [As for $\mathbb{Z}\times \mathbb{R}$ we cannot have that neighborhoods around $\infty\in (\mathbb{Z}\times\mathbb{R})^+$ are cones, since a complement of a cone must be a closed compact set. By compactness such a set could not intersect every copy of $\mathbb{R}$, so we have a contradiction.]
    \end{proof}
    \item (Hatcher 3.3.22) Show that $H^n_c(X\times\mathbb{R};G)\cong H^{n-1}_c(X;G)$ for all $n$. \begin{proof}
      For any compact subset $\kappa$ of $X\times \mathbb{R}$ the projections of $\kappa$ onto the factors $X$ and $\mathbb{R}$ are each compact and in particular the projection onto $\mathbb{R}$ is contained in a sufficiently large interval. So in the following we only consider the compact sets $K\times[-r,r]$ (for $K\subset X$ compact) of $X\times \mathbb{R}$, as every compact $\kappa$ above will be contained in a sufficiently large enough subset of that form. We will use this collection of compact sets as an increasing ``subnet'' for the directed limit in the conclusion.

      Consider the decompositions $X\times \mathbb{R} = X\times (-\infty,1)\cup X\times (-1,\infty) = A\cup B$ and for $K\subseteq X$ compact $X\times \mathbb{R}\setminus K\times[-r,r] = X\times (-\infty,1)\setminus K\times[-r,r] \cup X\times (-1,\infty)\setminus K\times[-r,r]=C\cup D$. Via the picture below for the pair $(A,C)$ we see that this pair is homotopic (via a deformation retract) to the pair $(X,X)$, and similarly so is the pair $(B,D)$:\vspace{5cm} 

      Then in the relative Mayer-Vietoris sequence we have \begin{multline*}
        \cdots \to H^{n-1}(X\times(-1,1),X\times(-1,1)\setminus K\times[-r,r];G)\\\to H^n(X\times \mathbb{R},X\times \mathbb{R}\setminus K\times[-r,r];G)\to H^n(A,C;G)\oplus H^n(B,D;G)\to \cdots,
      \end{multline*} but by homotopy invariance we have that the direct sum is trivial (as before we saw that $(A,C)$ and $(B,D)$ are homotopic to $(X,X)$) and that $H^{n-1}(X\times(-1,1),X\times(-1,1)\setminus K\times[-r,r];G)\cong H^{n-1}(X,X\setminus K;G)$. So for all $n$ we have $H^n(X\times \mathbb{R},X\times \mathbb{R}\setminus K\times[-r,r];G)\cong H^{n-1}(X,X\setminus K;G)$ by exactness.
      
      Then by taking a directed limit in the subnet mentioned prior of these groups, we obtain that $H^n_c(X\times\mathbb{R};G)\cong H^{n-1}_c(X;G)$ as needed.
    \end{proof}
\end{enumerate}
\end{document}