\documentclass[12pt]{amsart}

\textwidth = 6.2 in
\textheight = 8.5 in
\oddsidemargin = 0.0 in
\evensidemargin = 0.0 in
\topmargin = 0.0 in
\headheight = 0.0 in
\headsep = 0.3 in
\parskip = 0.05 in
\parindent = 0.3 in

\usepackage{enumerate}
\usepackage{amsmath}
\usepackage{color}
\def\cc{\color{blue}}
\usepackage[normalem]{ulem}
\usepackage{amsfonts, amsmath, amssymb, amsthm}
\usepackage{systeme}
\usepackage[none]{hyphenat}
\usepackage{graphicx}
\graphicspath{{./images/}}
\usepackage{esint}
\usepackage{cancel}
\usepackage{physics}

\title{Homework 6}
\author{Sai Sivakumar}

\newtheorem{theorem}            {Theorem}[section]
\newtheorem{proposition}        [theorem]{Proposition}

\newcommand{\RR}{\mathbb{R}}
\newcommand{\NN}{\mathbb{N}}
\newcommand{\QQ}{\mathbb{Q}}
\newcommand{\CC}{\mathbb{C}}
\newcommand{\DD}{\mathbb{D}}
\newcommand{\cS}{\mathcal{S}}

\begin{document}
\maketitle

\thispagestyle{empty}

Suppose $f,g:\CC\to\CC$ are entire. 

Show, if $f^2-g^2=1,$ then there exists an entire function $\psi$ such that $f=\frac{e^\psi + e^{-\psi}}{2}$ and $g=\frac{e^{\psi} - e^{-\psi}}{2}.$ [Suggestion: What can you say about an entire function that has no zeros?]

Show, if $f^3-g^3=1,$ then $f$ and $g$ are constant. [Suggestion: Assuming $g$ is not identically $0$, apply homework 5 to  $\frac{f}{g}$. What are three points in $\CC_\infty$ that $\frac{f}{g}$ omits?]

 \bigskip

\begin{proof}
\baselineskip=24pt

\end{proof}
\end{document}