\documentclass[12pt]{amsart}

\textwidth = 6.2 in
\textheight = 8.5 in
\oddsidemargin = 0.0 in
\evensidemargin = 0.0 in
\topmargin = 0.0 in
\headheight = 0.0 in
\headsep = 0.3 in
\parskip = 0.05 in
\parindent = 0.3 in

\usepackage{enumerate}
\usepackage{amsmath}
\usepackage{color}
\def\cc{\color{blue}}
\usepackage[normalem]{ulem}
\usepackage{amsfonts, amsmath, amssymb, amsthm}
\usepackage{systeme}
\usepackage[none]{hyphenat}
\usepackage{graphicx}
\graphicspath{{./images/}}
\usepackage{esint}
\usepackage{cancel}
\usepackage{physics}

\title{Homework 3}
\author{Sai Sivakumar}

\newtheorem{theorem}            {Theorem}[section]
\newtheorem{proposition}        [theorem]{Proposition}

\newcommand{\RR}{\mathbb{R}}
\newcommand{\NN}{\mathbb{N}}
\newcommand{\QQ}{\mathbb{Q}}
\newcommand{\CC}{\mathbb{C}}
\newcommand{\DD}{\mathbb{D}}
\newcommand{\cS}{\mathcal{S}}

\begin{document}
\maketitle
\thispagestyle{empty}
Use Theorem~1.6 from Chapter X  in Conway to prove Theorem~1.1 from Chapter VI  in Conway.
[You might want to recall, if $f:G\to \CC\setminus \{0\}$ is analytic,  then $\log |f|$  is harmonic.]

 \bigskip

\begin{proof}
\baselineskip=24pt
Let $G$ be a region and suppose that $f\colon G\to \mathbb{C}$ is analytic. Suppose further that there is an $a\in G$ with $\abs{f(a)}\geq \abs{f(z)}$ for all $z\in G$.

Certainly when $f$ does not have any roots in $G$ we have a function $\log\abs{f}\colon G\to \mathbb{R}$ which is harmonic on $G$. Since the logarithm is monotonic on reals, we have that $\log\abs{f(a)}\geq \log\abs{f(z)}$ for all $z\in G$. Then by the maximum principle $\log\abs{f}$ is constant so that $\abs{f}$ is constant. Since $f$ is analytic on the region $G$ and $\abs{f}$ is constant (i.e. $\abs{f}$ attains its maximum on $G$, the one value it takes on), $f$ is constant also (by using the max modulus principle from the notes, different from the max modulus principle we are proving here). 

When $f$ has roots, suppose that $\abs{f(a)}>0$ (if $\abs{f(a)} = 0$, then $f$ must be zero) and consider the restriction of $f$ onto $G_{>0} := G\setminus\{p\in G\mid \abs{f(p)} = 0\} = \bigcup_{n\in\mathbb{Z}_+}\{p\in G\mid \abs{f(p)}> 1/n\}$ (this shows that $G_{>0}$ is open and since zeroes of analytic functions are isolated, it is also connected). We have that $f|_{G_{>0}}$ is analytic so that again by the maximum principle, $\log\abs{f|_{G_{>0}}}$, and hence $f|_{G_{>0}}$, is constant on $G_{>0}$. It is either by continuity or by noting that the set of roots of $f$ is a determining set that $f$ should be deduced to be the zero function. 
\end{proof}
\end{document}