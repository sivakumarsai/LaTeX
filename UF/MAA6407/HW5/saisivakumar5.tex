\documentclass[12pt]{amsart}

\textwidth = 6.2 in
\textheight = 8.5 in
\oddsidemargin = 0.0 in
\evensidemargin = 0.0 in
\topmargin = 0.0 in
\headheight = 0.0 in
\headsep = 0.3 in
\parskip = 0.05 in
\parindent = 0.3 in

\usepackage{enumerate}
\usepackage{amsmath}
\usepackage{color}
\def\cc{\color{blue}}
\usepackage[normalem]{ulem}
\usepackage{amsfonts, amsmath, amssymb, amsthm}
\usepackage{systeme}
\usepackage[none]{hyphenat}
\usepackage{graphicx}
\graphicspath{{./images/}}
\usepackage{esint}
\usepackage{cancel}
\usepackage{physics}

\title{Homework 5}
\author{Sai Sivakumar}

\newtheorem{theorem}            {Theorem}[section]
\newtheorem{proposition}        [theorem]{Proposition}

\newcommand{\RR}{\mathbb{R}}
\newcommand{\NN}{\mathbb{N}}
\newcommand{\QQ}{\mathbb{Q}}
\newcommand{\CC}{\mathbb{C}}
\newcommand{\DD}{\mathbb{D}}
\newcommand{\cS}{\mathcal{S}}

\begin{document}
\maketitle

\thispagestyle{empty}

Show, if $f:\CC \to \CC_\infty$ is meromorphic and omits three values, then $f$ is constant. [Suggestion: reduce to the case where one of the points omitted is $\infty;$ that is, $f$ is entire.]

 \bigskip

\begin{proof}
\baselineskip=24pt
When $f$ is meromorphic and omits three values including $\infty$, then $f$ is entire and omits two points (the two points which were not $\infty$). By Little Picard we have that $f$ is constant. 

When $f$ is meromorphic and omits three values $a,b,c$ not including $\infty$, then $f$ has poles. Then $1/(f-a)$ omits $\infty,1/(b-a),1/(c-a)$ and is entire so by the above $1/(f-a)$ is constant. It follows that $f$ is constant also.
\end{proof}
\end{document}