\documentclass[mathserif]{beamer}
\usepackage{amsfonts, amsmath, amssymb, amsthm}
\usepackage{ytableau}
\usetheme{Warsaw}

% colors
\definecolor{pastelblue}{RGB}{167, 199, 231}
\usecolortheme[named=pastelblue]{structure}
\setbeamercolor{frametitle}{fg=black!85}
\setbeamercolor{title}{bg=pastelblue,fg=black!85}
%\setbeamercolor{block title}{fg=black!85}
\setbeamercolor{palette quaternary}{bg=pastelblue!66}

% page numbers
\setbeamertemplate{footline}[frame number]

% paragraph indentation/spacing
\setlength{\parindent}{0cm}
\setlength{\parskip}{5pt}
\renewcommand{\baselinestretch}{1.25}

% operators
\DeclareMathOperator{\Span}{span}
\DeclareMathOperator{\sgn}{sgn}
\DeclareMathOperator{\col}{col}
\makeatletter
\newcommand{\extp}{\@ifnextchar^\@extp{\@extp^{\,}}}
\def\@extp^#1{\mathop{\bigwedge\nolimits^{\!#1}}}
\makeatother

% theorems
\newtheorem*{proposition}{Proposition}

\title{Schur modules and highest weight theory}
\subtitle{3/3}

\author[Sai Sivakumar]{Sai Sivakumar}
\date{27 June 2023}

\begin{document}

%
\frame{\titlepage}

%
\begin{frame}
    \frametitle{The goal}

    \begin{theorem}
        Let $k$ be an infinite field of any characteristic and $V$ a finite dimensional vector space over $k$. 
        
        There is a one-to-one correspondence between \[\left\{ \parbox[c]{1.1in}{\centering
        simple rational representations of $GL(V)$}
        \right\}\longleftrightarrow \left\{ \parbox[c]{1.1in}{\centering
        dominant weights $\lambda$ (i.e. $\lambda_1\geq\cdots\geq \lambda_q$)}
        \right\}.\]

        The simple representation $L(\lambda)$ coming from a dominant weight $\lambda$ is the submodule generated by the $\mathbf{U}$-invariant subspace of $\mathbb{S}_\lambda(V)$.
    \end{theorem}

    In characteristic $0$, $\mathbb{S}_\lambda(V)$ is already simple, but in positive characteristic we obtain a proper submodule.

\end{frame}

%
\begin{frame}
    \frametitle{Notations involving partitions}

    For $\lambda = (\lambda_1,\dots,\lambda_q)$ a partition, let $|\lambda| = \lambda_1+ \cdots +\lambda_q$.
    
    Let $\col_{\lambda}(i)$ be the number of boxes in column $i$ of the Young diagram of $\lambda$. 
    
    For $T$ a Young tableau of shape $\lambda$ we denote its entries by $T(\text{row},\text{column})$. For example:\[\lambda = (3,3,1)\quad \longleftrightarrow\quad \ytableausetup{centertableaux, boxsize=2.5em}
    \begin{ytableau}
    \scriptstyle T(1,1) & \scriptstyle T(1,2) & \scriptstyle T(1,3) \\
    \scriptstyle T(2,1) & \scriptstyle T(2,2) & \scriptstyle T(2,3) \\
    \scriptstyle T(3,1)
    \end{ytableau}\]

\end{frame}

%
\begin{frame}
    \frametitle{Example of a Schur module}

    

\end{frame}

%
\begin{frame}
    \frametitle{Construction of Schur module}

    Let $V = \Span_k\{e_1,\dots,e_n\}$ and $\lambda = (\lambda_1,\dots,\lambda_q)$ be a partition. There is a very abstract way to obtain the Schur module $\mathbb{S}_\lambda(V)$:

    There is an injection \[\extp^p V\xrightarrow{a} V^{\otimes p}\] which takes $v_1\wedge\cdots\wedge v_p$ to $\sum_\sigma \sgn(\sigma) v_{\sigma(1)}\otimes\cdots\otimes v_{\sigma(p)}$ (the anti-skew-symmetrizing map).

\end{frame}

%
\begin{frame}

    Consider the map \[\extp^{\col_\lambda(1)} V\otimes\cdots\otimes \extp^{\col_\lambda(\lambda_1)} V \xrightarrow{a_1\otimes\cdots\otimes a_{\lambda_1}} V^{\otimes |\lambda|}\] which is the tensor product of the anti-skew-symmetrizing maps defined prior. Then compose this map with the map \[V^{\otimes |\lambda|}\xrightarrow{\beta_\lambda}S^{\lambda_1}V\otimes\cdots\otimes S^{\lambda_q}V\] that takes $f_1\otimes\cdots\otimes f_{|\lambda|}$ to \begin{multline*}
        f_1f_{\col_\lambda(1)+1}f_{\col_\lambda(1)+\col_\lambda(2)+1}\cdots f_{\col_\lambda(1)+\cdots+\col_\lambda(\lambda_1-1)+1} \otimes \\ f_2f_{\col_\lambda(1)+2}f_{\col_\lambda(1)+\col_\lambda(2)+2}\cdots f_{\col_\lambda(1)+\cdots+\col_\lambda(\lambda_2-1)+2}\otimes \\
        \cdots \otimes f_{\col_\lambda(1)}f_{\col_\lambda(1)+\col_\lambda(1)}f_{\col_\lambda(1)+\col_\lambda(2)+\col_\lambda(1)}\\\cdots f_{\col_\lambda(1)+\cdots+\col_\lambda(\lambda_{\col_\lambda(1)}-1)+\col_\lambda(1)}
    \end{multline*}

\end{frame}

%
\begin{frame}

    Then the Schur module $\mathbb{S}_\lambda (V)$ is defined to be the image of the composite map $\beta_\lambda\circ a_1\otimes\cdots\otimes a_{\lambda_1}$. 

    In this way regard the Schur module $\mathbb{S}_\lambda (V)$ as a submodule of $S^{\lambda_1}V\otimes\cdots\otimes S^{\lambda_q}V$ or as a quotient of $\extp^{\col_\lambda(1)} V\otimes\cdots\otimes \extp^{\col_\lambda(\lambda_1)} V$.

    Furthermore, we associate to each tableau $T$ of shape $\lambda$ with entries in $1,\dots,n$ the coset of \begin{multline*}
        e_{T(1,1)}\wedge\cdots\wedge e_{T(\col_\lambda(1),1)}\otimes e_{T(1,2)}\wedge\cdots\wedge e_{T(\col_\lambda(2),2)}\otimes \cdots \\ \otimes e_{T(1,\lambda_1)}\wedge\cdots\wedge e_{T(\col_\lambda(\lambda_1),\lambda_1)}.
    \end{multline*}

\end{frame}

%
\begin{frame}

    \frametitle{Basis for Schur module}
    \begin{proposition}
        Let Let $V = \Span_k\{e_1,\dots,e_n\}$ and $\lambda = (\lambda_1,\dots,\lambda_q)$ be a partition. Then the set of semistandard Young tableaux of shape $\lambda$ with entries from $1,\dots,n$ form a basis of $\mathbb{S}_\lambda(V)$.
    \end{proposition}

    The semistandard Young tableaux are those that are weakly increasing in each row and strictly increasing in each column.

\end{frame}

%
\begin{frame}
    \frametitle{The $\mathbf{U}$-invariant subspace of $\mathbb{S}_\lambda(V)$}

    The ``canonical'' tableau $c_\lambda$ with $c_\lambda(i,j) = i$ is a $\mathbf{U}$-invariant element in $\mathbb{S}_\lambda(V)$. Furthermore, the \textbf{subspace} $(\mathbb{S}_\lambda(V))^{\mathbf{U}}$ of $\mathbf{U}$-invariants is one-dimensional (spanned by $c_\lambda$) of weight $\lambda$.

    The \textbf{submodule} generated by $c_\lambda$ must be a simple $GL(V)$-module. Conversely, the simple modules come in this form. The proof of both directions is omitted.
    \[c_{(4,2,2)} = \ytableausetup{centertableaux, boxsize=1.5em}
    \begin{ytableau}
    1 & 1 & 1 & 1 \\
    2 & 2 \\
    3 & 3
    \end{ytableau}\]

\end{frame}

%
\begin{frame}
    \begin{flushright}
        {\color{black!15!pastelblue}Thank you for your time.}
        \end{flushright}
        \hrule
    \begin{columns}
    \begin{column}{0.3\textwidth}
        \begin{block}{}{
        \begin{center}\Large  Questions?\end{center}}
        \end{block}\vspace{0em}
    \end{column}
    \begin{column}{0.7\textwidth}
        \begin{block}{}{
        \begin{center}\Large  References:\end{center}}
    \end{block}
(1) Jerzy Weyman: \textit{Cohomology of Vector Bundles and Syzygies}

(2) Kaan Akin, David A Buchsbaum, Jerzy Weyman: \textit{Schur Functors and Schur Complexes}

(3) Michael Perlman: \textit{Representations of the General Linear Group: A User's Guide}
    \end{column}
    \end{columns}
\end{frame}
\end{document}