\documentclass[11pt]{article}
\headheight=13.6pt

% packages
\usepackage{physics}
% margin spacing
\usepackage[top=1in, bottom=1in, left=0.5in, right=0.5in]{geometry}
\usepackage{hanging}
\usepackage{amsfonts, amsmath, amssymb, amsthm}
\usepackage{systeme}
\usepackage[none]{hyphenat}
\usepackage{fancyhdr}
\usepackage{graphicx}
\graphicspath{{./images/}}
\usepackage{float}
\usepackage{siunitx}
\usepackage{esint}
\usepackage{cancel}
\usepackage{enumitem}
\usepackage{mathrsfs}
\usepackage{hyperref}
\hypersetup{colorlinks=true,urlcolor=blue}

% header/footer formatting
\pagestyle{fancy}
\fancyhead{}
\fancyfoot{}
\fancyhead[L]{MAP6506}
\fancyhead[C]{HW1}
\fancyhead[R]{Sai Sivakumar}
\fancyfoot[R]{\thepage}
\renewcommand{\headrulewidth}{1pt}

% paragraph indentation/spacing
\setlength{\parindent}{0cm}
\setlength{\parskip}{10pt}
\renewcommand{\baselinestretch}{1.25}

% extra commands defined here
\newcommand{\br}[1]{\left(#1\right)}
\newcommand{\sbr}[1]{\left[#1\right]}
\newcommand{\cbr}[1]{\left\{#1\right\}}
\newcommand{\eq}[1]{\overset{(#1)}{=}}

% bracket notation for inner product
\usepackage{mathtools}

\DeclarePairedDelimiterX{\abr}[1]{\langle}{\rangle}{#1}

% smileys frownies
\usepackage{wasysym}
\newcommand{\happy}{\raisebox{-.28em}{\resizebox{1.5em}{!}{\smiley}}}
\newcommand{\darkhappy}{\raisebox{-.28em}{\resizebox{1.5em}{!}{\blacksmiley}}}
\newcommand{\sad}{\raisebox{-.28em}{\resizebox{1.5em}{!}{\frownie}}}
\DeclareMathOperator{\mathhappy}{\!\happy\!}
\DeclareMathOperator{\mathdarkhappy}{\!\darkhappy\!}
\DeclareMathOperator{\mathsad}{\!\sad\!}

\DeclareMathOperator{\Span}{span}
\DeclareMathOperator{\im}{im}
\DeclareMathOperator{\dist}{dist}
\DeclareMathOperator{\supp}{supp}
\newcommand{\res}[1]{\operatorname*{res}_{#1}}

% set page count index to begin from 1
\setcounter{page}{1}

\begin{document}

\begin{enumerate}
    \item Since $A$ is real, symmetric, and strictly positive, there exists an $N\times N$ orthogonal matrix $U$ such that $U^{-1}AU = U^TAU = D$ where $D$ is an $N\times N$ diagonal matrix with positive entries. Then $(\nabla, A\nabla) = (\nabla, UDU^T\nabla) = (U^T\nabla, DU^T\nabla)$. Make the change of coordinates $x = U^Ty$, so that $U^T\nabla_x = \nabla_y$ (as $U$ is linear). With $D$ diagonal and $\det U = 1$, it follows that \[(\nabla_x, A\nabla_x)G(x) = (U^T\nabla_x, DU^T\nabla_x)G(x) = (\nabla_y, D\nabla_y)G(U^Ty) = D(\Delta_yG)(U^Ty) = \delta(U^Ty) = \delta(y).\]
    \item 31.6.2 We have (since $\mathcal E^\pm\ast\delta$ exists) that \[u_\pm(x,t) = \frac{e^{-i\omega t}}{c^2}\cdot (\mathcal E^\pm \ast (\nabla, p\nabla)\delta)(x) = \frac{e^{-i\omega t}}{c^2}\cdot (\nabla, p\nabla)\mathcal E^\pm(x)\]
    \item 37.9.4

    \hrulefill
\end{enumerate}
Honor Code: \vspace*{7em}
\end{document}