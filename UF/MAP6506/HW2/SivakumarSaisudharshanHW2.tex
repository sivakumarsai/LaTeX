\documentclass[11pt]{article}
\headheight=13.6pt

% packages
\usepackage{physics}
% margin spacing
\usepackage[top=1in, bottom=1in, left=0.5in, right=0.5in]{geometry}
\usepackage{hanging}
\usepackage{amsfonts, amsmath, amssymb, amsthm}
\usepackage{systeme}
\usepackage[none]{hyphenat}
\usepackage{fancyhdr}
\usepackage{graphicx}
\graphicspath{{./images/}}
\usepackage{float}
\usepackage{siunitx}
\usepackage{esint}
\usepackage{cancel}
\usepackage{enumitem}
\usepackage{mathrsfs}
\usepackage{hyperref}
\hypersetup{colorlinks=true,urlcolor=blue}

% header/footer formatting
\pagestyle{fancy}
\fancyhead{}
\fancyfoot{}
\fancyhead[L]{MAP6506}
\fancyhead[C]{HW2}
\fancyhead[R]{Sai Sivakumar}
\fancyfoot[R]{\thepage}
\renewcommand{\headrulewidth}{1pt}

% paragraph indentation/spacing
\setlength{\parindent}{0cm}
\setlength{\parskip}{10pt}
\renewcommand{\baselinestretch}{1.25}

% extra commands defined here
\newcommand{\br}[1]{\left(#1\right)}
\newcommand{\sbr}[1]{\left[#1\right]}
\newcommand{\cbr}[1]{\left\{#1\right\}}
\newcommand{\eq}[1]{\overset{(#1)}{=}}

% bracket notation for inner product
\usepackage{mathtools}

\DeclarePairedDelimiterX{\abr}[1]{\langle}{\rangle}{#1}

% smileys frownies
\usepackage{wasysym}
\newcommand{\happy}{\raisebox{-.28em}{\resizebox{1.5em}{!}{\smiley}}}
\newcommand{\darkhappy}{\raisebox{-.28em}{\resizebox{1.5em}{!}{\blacksmiley}}}
\newcommand{\sad}{\raisebox{-.28em}{\resizebox{1.5em}{!}{\frownie}}}
\DeclareMathOperator{\mathhappy}{\!\happy\!}
\DeclareMathOperator{\mathdarkhappy}{\!\darkhappy\!}
\DeclareMathOperator{\mathsad}{\!\sad\!}

\DeclareMathOperator{\Span}{span}
\DeclareMathOperator{\im}{im}
\DeclareMathOperator{\dist}{dist}
\DeclareMathOperator{\supp}{supp}
\newcommand{\res}[1]{\operatorname*{res}_{#1}}

% set page count index to begin from 1
\setcounter{page}{1}

\begin{document}

\begin{enumerate}[label=\textbf{\arabic*.}]
    \item \textbf{Green's function for the Sturm-Liouville operator with zero eigenvalue}
    
    Consider the boundary value problem \[L_0u = -(pu^\prime)^\prime = f,\quad x\in(a,b),\quad u^\prime(a) = u^\prime(b) = 0.\]
    \begin{enumerate}[label=\textsf{(\roman*)}]
        \item The homogeneous boundary value problem $L_0u = 0$ with $-u^\prime(a) = u^\prime(b) = 0$ only has constant non-trivial solution since $q = 0$ and $\alpha_a = \alpha_b = 0$ (see Proposition 44.1).
        
        It is also necessary for the integral of $f$ to vanish over $(a,b)$: Suppose $u_p$ is a particular solution to $L_0u_p = f$ with $u_p^\prime(a) = u_p^\prime(b) = 0$. Then $\int_a^b f(x) \dd x = \int_a^b -(p(x)u^\prime_p(x))^\prime\dd x = -p(x)u^\prime_p(x)\big|_a^b = 0$.

        Suppose that $f$ satisfies $\int_a^b f(x)\dd x = 0$ and that $u_p$ is a particular solution to $L_0u_p = -(pu^\prime_p)^\prime = f$ with $u_p^\prime(a) = u_p^\prime(b) = 0$.
    \end{enumerate}
    
    \hrulefill
    \item 

    \hrulefill
    \item 

    \hrulefill
\end{enumerate}
Honor Code: \vspace*{7em}
\end{document}