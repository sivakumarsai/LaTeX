\documentclass[11pt]{article}

% packages
\usepackage{physics}
% margin spacing
\usepackage[top=1in, bottom=1in, left=0.5in, right=0.5in]{geometry}
\usepackage{hanging}
\usepackage{amsfonts, amsmath, amssymb, amsthm}
\usepackage{systeme}
\usepackage[none]{hyphenat}
\usepackage{fancyhdr}
\usepackage[nottoc, notlot, notlof]{tocbibind}
\usepackage{graphicx}
\graphicspath{{./images/}}
\usepackage{float}
\usepackage{siunitx}
\usepackage{esint}
\usepackage{cancel}
\usepackage{enumitem}

% permutations (second line is for spacing)
\usepackage{permute}
\renewcommand*\pmtseparator{\,}

% colors
\usepackage{xcolor}
\definecolor{p}{HTML}{FFDDDD}
\definecolor{g}{HTML}{D9FFDF}
\definecolor{y}{HTML}{FFFFCF}
\definecolor{b}{HTML}{D9FFFF}
\definecolor{o}{HTML}{FADECB}
%\definecolor{}{HTML}{}

% \highlight[<color>]{<stuff>}
\newcommand{\highlight}[2][p]{\mathchoice%
  {\colorbox{#1}{$\displaystyle#2$}}%
  {\colorbox{#1}{$\textstyle#2$}}%
  {\colorbox{#1}{$\scriptstyle#2$}}%
  {\colorbox{#1}{$\scriptscriptstyle#2$}}}%

% header/footer formatting
\pagestyle{fancy}
\fancyhead{}
\fancyfoot{}
\fancyhead[L]{MAS5311}
\fancyhead[C]{Assignment 6}
\fancyhead[R]{Sai Sivakumar}
\fancyfoot[R]{\thepage}
\renewcommand{\headrulewidth}{1pt}

% paragraph indentation/spacing
\setlength{\parindent}{0cm}
\setlength{\parskip}{10pt}
\renewcommand{\baselinestretch}{1.25}

% extra commands defined here
\newcommand{\ihat}{\boldsymbol{\hat{\textbf{\i}}}}
\newcommand{\jhat}{\boldsymbol{\hat{\textbf{\j}}}}
\newcommand{\dr}{\vec{r}~^{\prime}(t)}
\newcommand{\dx}{x^{\prime}(t)}
\newcommand{\dy}{y^{\prime}(t)}

\newcommand{\br}[1]{\left(#1\right)}
\newcommand{\sbr}[1]{\left[#1\right]}
\newcommand{\cbr}[1]{\left\{#1\right\}}

\newcommand{\dprime}{\prime\prime}
\newcommand{\lap}[2]{\mathcal{L}[#1](#2)}

\newcommand{\divides}{\mid}

% bracket notation for inner product
\usepackage{mathtools}

\DeclarePairedDelimiterX{\abr}[1]{\langle}{\rangle}{#1}

\DeclareMathOperator{\Span}{span}
\DeclareMathOperator{\nullity}{nullity}
\DeclareMathOperator\Aut{Aut}
\DeclareMathOperator\Inn{Inn}
\DeclareMathOperator{\lcm}{lcm}

% set page count index to begin from 1
\setcounter{page}{1}

\begin{document}

\begin{enumerate}
    \item (DF3.4.4) Use Cauchy's Theorem and induction to show that a finite abelian group has a subgroup of order $n$ for each positive divisor $n$ of its order.
    \begin{proof}
      Let $G$ be an arbitrary finite abelian group. We proceed by induction on the order of $G$; that is, we assume that for any finite abelian group $H$ such that $1\leq \abs{H} < \abs{G}$, the group $H$ has a subgroup of order $d$ for every positive divisor $d$ of $\abs{H}$. It suffices to show that $G$ has a subgroup of order $n$ for every positive divisor $n$ of $\abs{G}$. Whenever $n$ is prime, we already have by Cauchy's theorem the existence of a subgroup of prime order $n$. So without loss of generality, let $n$ be composite.
      
      Let $p$ be any prime which divides $n$, so that for some positive integer $k>1$ (as $n$ is composite), $n = pk$. By Cauchy's Theorem, it follows that $G$ has a subgroup $N$ of order $p$. Then because all subgroups of abelian groups are abelian, it follows that $N$ is normal in $G$ and so the quotient group $G/N$ is well defined.

      Because $n\mid \abs{G}$, there exists a positive integer $q$ such that $\abs{G} = nq$. So by Lagrange's theorem, the order of the quotient group $G/N$ is $\abs{G}/\abs{N} = nq/p = pkq/p = kq$, which because $k>1$, the quotient group is not trivial. Furthermore, because $k>1$ and $p$ is a prime number, it follows that $\abs{G/N} = kq < n$. Then by the inductive hypothesis, it follows that $G/N$ contains a subgroup of order $k$, which by the Lattice Isomorphism Theorem we may write as $A/N$ for some $N\unlhd A\unlhd G$ (all subgroups of $G$ are normal in larger subgroups of $G$ containing them because all subgroups of $G$ are abelian).

      Then again using Lagrange's theorem we have that $\abs{A/N} = \abs{A}/\abs{N} = \abs{A}/p = k$, which means $\abs{A} = pk = n$, so $A$ is a subgroup of $G$ with order $n$.
    \end{proof}
    \item (DF3.4.1) Prove that if $G$ is an abelian simple group then $G\cong Z_p$ for some prime $p$. (not a homework problem, but needed for DF3.4.8, the next exercise)
    \begin{proof}
      Let $G$ be an abelian simple group. Then there are no normal subgroups outside of the trivial subgroup and $G$ itself. This means that for some element $x\neq 1$ in $G$, the subgroup $\abr{x}$ is normal in $G$ because $G$ is abelian, but because $x\neq 1$, this normal subgroup is equal to $G$. Hence $G$ is cyclic. 
      
      Then suppose that $G$ is an infinite cyclic group. Then $\abr{x} = \cbr{x,x^2,\dots} = G$ but since $x^2\neq 1$ (infinite order), $\abr{x^2} = \cbr{x^2,x^4,\dots}\neq G$ but $\abr{x^2}$ is normal in $G$ because $G$ is abelian. This is in contradiction to the assumption that $G$ was simple. So $G$ has finite order.
      
      Suppose that the order of $G$ is composite. Then $\abs{G} = pk$ for some prime $p$ and some positive integer $k>1$ ($\abs{G}$ is composite). But by Cauchy's theorem, we know that there is a subgroup of $G$ of order $p$, which is strictly less than $\abs{G}$ and strictly greater than $1$, which again is in contradiction to the assumption that $G$ was simple. Hence the order of $G$ is not composite; that is, the order of $G$ is a prime number.

      Groups with prime order are known to be cyclic due to Lagrange's theorem. Therefore $G\cong Z_p$ for some prime $p$.
    \end{proof}
    \item (DF3.4.8) Let $G$ be a finite group. Prove that the following are equivalent:
    \begin{enumerate}[label=\textbf{(\roman*)}]
      \item $G$ is solvable (there is a chain of subgroups $1 = G_0 \unlhd G_1 \unlhd G_2 \unlhd \cdots \unlhd G_s = G$ such that $G_{i+1}/G_i$ is abelian for $i = 0,1,\dots,s-1$)
      \item $G$ has a chain of subgroups: $1 = H_0 \unlhd H_1 \unlhd H_2 \unlhd \cdots \unlhd H_s = G$ such that $H_{i+1}/H_i$ is cyclic, $0\leq i\leq s-1$
      \item all composition factors of $G$ are of prime order
      \item $G$ has a chain of subgroups: $1 = N_0 \unlhd N_1 \unlhd N_2 \unlhd \cdots \unlhd N_t = G$ such that each $N_i$ is a normal subgroup of $G$ and $N_{i+1}/N_i$ is abelian, $0\leq i\leq t-1$.
    \end{enumerate}
    \begin{proof}
      We show that \textbf{(i)} is equivalent to \textbf{(ii)}, \textbf{(i)} is equivalent to \textbf{(iii)}, and that \textbf{(i)} is equivalent to \textbf{(iv)}.

      From \textbf{(i)}, let $G$ be a group with a chain of subgroups $1 = G_0 \unlhd G_1 \unlhd G_2 \unlhd \cdots \unlhd G_s = G$ such that $G_{i+1}/G_i$ is abelian for $i = 0,1,\dots,s-1$). The factor groups $G_{i+1}/G_i$ may not be simple, so we must insert intermediate subgroups within the chain which make simple and abelian subgroups.

      If a factor group $G_{i+1}/G_i$ is not simple, there exists a nontrivial proper normal subgroup of $G_{i+1}/G_i$, which because $G_{i+1}/G_i$ is abelian, is also abelian. So by the Lattice Isomorphism Theorem, there exists a nontrivial proper normal subgroup $A/G_i$ such that $G_i\leq A \unlhd G_{i+1}$. Furthermore, because $G_i$ is a normal subgroup of $G_{i+1}$, it follows that $G_i$ is a normal subgroup of $A$.
      
      So $G_i\unlhd A \unlhd G_{i+1}$, and $A/G_i$ is abelian. We need to show that $G_{i+1}/A$ is abelian. Because $G_{i+1}/G_i$ is abelian, the multiplication of members $xG_i,yG_i$ (with any $x,y\in G_{i+1}$) commute:
      \[xG_iyG_i = xyG_i = yxG_i = yG_ixG_i\]
      So $(yx)^{-1}xy\in G_i$, and by inclusion, $(yx)^{-1}xy\in A$, which means that $xAyA = xyA = yxA = yAxA$, so that the factor group $G_{i+1}/A$ is abelian. It is possible for $A/G_i$ or $G_{i+1}/A$ or both to not be simple.

      Therefore, it is possible to form a new chain of subgroups from the chain \[1 = G_0 \unlhd G_1 \unlhd G_2 \unlhd \cdots \unlhd G_i \unlhd G_{i+1} \unlhd \cdots \unlhd G_s = G\] by replacing $G_i \unlhd G_{i+1}$ with $G_i\unlhd A \unlhd G_{i+1}$ when $G_{i+1}/G_i$ is not simple, to form \[1 = G_0 \unlhd G_1 \unlhd G_2 \unlhd \cdots \unlhd G_i\unlhd A \unlhd G_{i+1} \unlhd \cdots \unlhd G_s = G.\] Of course, we can keep expanding out the non-simple factors of this new series that we obtained, but because $G$ is finite, by the Jordan-H\"older theorem, we will arrive at the composition series for $G$. By repeating the above procedure until all of the abelian factor groups $N_{i+1}/N_i$ for $0\leq i\leq t-1$ are simple in the final chain $1 = N_0 \unlhd N_1 \unlhd N_2 \unlhd \cdots \unlhd N_t = G$, we arrive at the composition series for $G$.

      Then by the previous theorem which states that abelian simple groups are isomorphic to a cyclic group of prime order, all of these factor groups $N_{i+1}/N_i$ are cyclic groups, so \textbf{(i)} implies \textbf{(ii)}. Because cyclic groups are abelian by exponent rules, \textbf{(ii)} implies \textbf{(i)}. Hence \textbf{(i)} is equivalent to \textbf{(ii)}.

      Furthermore, by the same theorem, these factor groups (composition factors) $N_{i+1}/N_i$ are of prime order, so \textbf{(i)} implies \textbf{(iii)}. Then if all of the composition factors of $G$ are of prime order, then necessarily they are cyclic and hence abelian. So \textbf{(iii)} implies \textbf{(i)}. Hence \textbf{(i)} is equivalent to \textbf{(iii)}.

      What remains is to show that \textbf{(i)} is equivalent to \textbf{(iv)}. It is clear that \textbf{(iv)} implies \textbf{(i)} because \textbf{(iv)} is a stronger statement:

      \textbf{(iv)} $G$ has a chain of subgroups: $1 = N_0 \unlhd N_1 \unlhd N_2 \unlhd \cdots \unlhd N_t = G$ such that each $N_i$ is a normal subgroup of $G$ and $N_{i+1}/N_i$ is abelian, $0\leq i\leq t-1$.

      So not only are the subgroups $N_i$ normal in $N_{i+1}$ for $0\leq i\leq t-1$, but they are normal in $G$. Conversely, assume \textbf{(i)} is true. Then let $M$ be a minimal nontrivial normal subgroup of $G$, so that $M$ does not contain any nontrivial proper subgroups which are normal in $G$. Let $N\unlhd M$ be a normal subgroup of $M$ of prime index, which exists because $G$ has a composition series and so $M$ has one as well (subgroups of solvable groups are solvable), so by \textbf{(iii)}, $N$ exists. 
      
      Then the quotient $M/N$ is cyclic, so it is abelian. This means that for any $x,y\in M$, $xNyN = xyN = yxN = yNxN$, which implies that $(yx)^{-1}xy \in N$. Similarly, consider $gNg^{-1}$ for $g\in G$. This is a subgroup of $G$ that is also a normal subgroup of $M$ because for any $m\in M$, $mgNg^{-1}m^{-1}\subseteq gNg^{-1}$ is equivalent to showing that $g^{-1}mgN(g^{-1}mg)^{-1} \subseteq N$. But $M\unlhd G$, so $g^{-1}mg, (g^{-1}mg)^{-1} \in M$, and we know that $N\unlhd M$, so $g^{-1}mgN(g^{-1}mg)^{-1} \subseteq N$ is true. Then $mgNg^{-1}m^{-1}\subseteq gNg^{-1}$ is true, which means that $gNg^{-1}$ is a normal subgroup of $M$.

      Then the quotient $M/gNg^{-1}$ is well defined, and because $\abs{gNg^{-1}} = \abs{N}$, this quotient has prime order as well, which means it is cyclic and hence abelian. Similarly, for any $x,y\in M$, $xgNg^{-1}ygNg^{-1} = xygNg^{-1} = yxgNg^{-1} = ygNg^{-1}xgNg^{-1}$, so that $(yx)^{-1}xy \in gNg^{-1}$. Since $g$ was arbitrary, the following is true: $(yx)^{-1}xy\in \cap_{g\in G} gNg^{-1}$. 
      
      But observe that $\cap_{g\in G} gNg^{-1}$ is a subgroup of $N$ since it contains only those elements of $N$ which appear in each $gNg^{-1}$, and this set is a subgroup because the intersection of subgroups is a subgroup. Furthermore, $\cap_{g\in G} gNg^{-1}$ is also a normal subgroup of $G$: Let $h\in G$. We must show that \[h\br{\bigcap_{g\in G} gNg^{-1}}h^{-1} \subseteq \bigcap_{g\in G} gNg^{-1},\] but this is clear from the fact that any of the elements on the left hand side take on the form $g^{\prime}n^{\prime}(g^{\prime})^{-1}$ for every $g^{\prime}\in G$ and some $n^{\prime}\in N$. Then $hg^{\prime}n^{\prime}(g^{\prime})^{-1}h^{-1} = (g^{\prime}h)n^{\prime}(g^{\prime}h)^{-1} \in \cap_{g\in G} gNg^{-1}$ since $g^{\prime}h$ takes on any element of $G$. Hence the intersection is a normal subgroup of $G$.

      But the minimality of $M$ suggests that this subgroup of $N$ (and hence a proper subgroup of $M$) should be the trivial subgroup, and since for every $x,y\in M$, $(yx)^{-1}xy\in \cap_{g\in G} gNg^{-1} = \cbr{1}$, we must have that $(yx)^{-1}xy = 1$ so that $xy = yx$. So $M$ is abelian, and $M$ was a minimal nontrivial normal subgroup of $G$. Furthermore, this means that $M/1$ is abelian as well.

      Suppose by induction that for all groups of a strictly lower order than $G$ which are solvable, there exists a chain of the form given in \textbf{iv} for the lower order groups, where $G$ is solvable. Then consider the quotient group $G/M$, where $M$ was a minimal nontrivial normal subgroup of $G$. This group is solvable because any chain of subgroups of the form $1 = \overline{G_0} \unlhd \overline{G_1} \unlhd \overline{G_2} \unlhd \cdots \unlhd \overline{G_r} = G/M$, which by the Lattice Isomorphism Theorem we can write $\overline{G_i} = A_i/M$ for some subgroup $A_i$ containing $M$ ($M\unlhd A_i$). Then by the Third Isomorphism Theorem, we have that $\overline{G_{i+1}}/\overline{G_i} = (A_{i+1}/M)/(A_i/M) \cong A_{i+1}/A_i$, which are abelian since $G$ is solvable. Hence the quotient group is solvable, furthermore, because $M$ is nontrivial, the order of the quotient group $G/M$ is strictly lower than the order of $G$ and by the inductive hypothesis $G/M$ has a series as given in \textbf{(iv)}, say $1 = \overline{G_0} \unlhd \overline{G_1} \unlhd \overline{G_2} \unlhd \cdots \unlhd \overline{G_r} = G/M$, with each $\overline{G_i}\unlhd G/M$ where by the Lattice Isomorphism Theorem we can write $\overline{G_i} = G_i/M$ for some subgroup $G_i$ containing $M$ ($M\unlhd G_i$). Furthermore, by the Lattice Isomorphism Theorem, each $G_i\unlhd G$; and so by the Third Isomorphism Theorem it follows that $\overline{G_{i+1}}/\overline{G_i} = (G_{i+1}/M)/(G_i/M) \cong G_{i+1}/G_i$ we can form the chain of subgroups for $G$: $1 \unlhd M = G_0 \unlhd G_1 \unlhd \cdots \unlhd G_{r+1} = G$ where each $G_i$ is normal in $G$ as well. Each of the factors $G_{i+1}/G_i$ (and $M/1$) will be abelian as well since $\overline{G_{i+1}}/\overline{G_i}$ is abelian by assumption, and $M$ is abelian. Thus \textbf{(i)} implies \textbf{(iv)}, so that \textbf{(i)} is equivalent to \textbf{(iv)}.

      Therefore, all four statements are equivalent to each other.
    \end{proof}
    \item (DF3.5.12) Prove that $A_n$ contains a subgroup isomorphic to $S_{n-2}$ for each $n\geq 3$.
    \begin{proof}
      Let $n\geq 3$, since $A_1$ and $A_2$ are both the trivial group. View $S_n$ as the group of permutations of $\cbr{1,2,\dots,n}$, so that we can view $S_{n-2}$ as the group of permutations of $\cbr{1,2,\dots,n-2}$, which are all the permutations of $S_n$ which fix $n$ and $n-1$. Let $\tau$ be the transposition in $S_n$ which interchanges $n$ and $n-1$; i.e. $\tau = \pmt*{({(n-1)} n)}$. Note that $\tau$ has order $2$, and commutes with all elements of $S_{n-2}$ which appear in $S_n$.
      
      Then there is a homomorphism $\varphi\colon S_{n-2}\to A_n$ given by
      \[\varphi(\sigma) = \begin{cases}
        \sigma, & \text{if $\sigma$ is an even permutation} \\
        \sigma\tau, & \text{if $\sigma$ is an odd permutation}
      \end{cases}.\]
      In both cases we obtain an even permutation (we make odd permutations even by appending $\tau$ on the right), so the image of the homomorphism is indeed a subgroup of $A_n$. We should check that this is a homomorphism. For permutations $\sigma,\pi\in S_{n-2}$, we investigate $\varphi(\sigma\pi)$ for each of the four ways we choose the parity of $\sigma$ and $\pi$, remembering that $\tau^2 = 1$, and that $\tau$ commutes with all elements of $S_{n-2}$ which appear in $S_n$.
      \begin{center}
        \begin{tabular}{ c|c|c|c } 
         $\sigma$ & $\pi$ & $\sigma\pi$ & $\varphi(\sigma\pi)\overset{?}{=} \varphi(\sigma)\varphi(\pi)$ \\ 
         \hline
         even & even & even & $\varphi(\sigma\pi) = \sigma\pi = \varphi(\sigma)\varphi(\pi)$ \\
         even & odd & odd & $\varphi(\sigma\pi) = \sigma\pi\tau = (\sigma)(\pi\tau) = \varphi(\sigma)\varphi(\pi)$ \\
         odd & even & odd & $\varphi(\sigma\pi) = \sigma\pi\tau = (\sigma\tau)(\pi) = \varphi(\sigma)\varphi(\pi)$ \\
         odd & odd & even & $\varphi(\sigma\pi) = \sigma\pi = \sigma\pi\tau^2 = (\sigma\tau)(\pi\tau) = \varphi(\sigma)\varphi(\pi)$ \\
        \end{tabular}
        \end{center}
        So $\varphi$ is a homomorphism from $S_{n-2}$ to $A_n$. 
        
        It is also true that $\varphi$ is injective. Let $\varphi(\sigma) = \sigma\tau^a = \pi\tau^b = \varphi(\pi)$, for $a,b\in\cbr{0,1}$. We reach a contradiction when $a\neq b$, because $\sigma$ and $\tau$ were permutations in $S_{n-2}$, so they fix $n$ and $n-1$. It is not possible to write $\sigma = \pi\tau$ or $\pi = \sigma\tau$ as a result. Therefore, $a = b$, and by right cancellation $\sigma = \pi$. Hence $\varphi$ is an injective homomorphism, which means $\ker{\varphi}$ is the trivial subgroup. Then by the First Isomorphism Theorem, \[\frac{S_{n-2}}{\ker{\varphi}} = \frac{S_{n-2}}{\cbr{1}} \cong S_{n-2}\cong \varphi(S_{n-2})\leq A_n.\]
        Hence for all $n\geq 3$, $A_n$ contains a subgroup isomorphic to $S_{n-2}$.
    \end{proof}
    \item (DF3.1.36) Prove that if $G/Z(G)$ is cyclic then $G$ is abelian. 
    \begin{proof}
      Let $G$ be a group such that $G/Z(G)$ is cyclic as given. Then by definition, $G/Z(G)$ has a generator. We may choose a representative $x\in G$ so that $xZ(G)$ is a generator for $G/Z(G)$. Then because the left cosets of $Z(G)$ partition $G$, it follows that for any element $g\in G$, $g$ lies in one of the left cosets of $Z(G)$. In particular, $g$ lies in $gZ(G)$, which is an element of $G/Z(G)$. Since $xZ(G)$ generates $G/Z(G)$, there exists $n\in \mathbb{Z}$ such that $gZ(G) = (xZ(G))^n = x^nZ(G)$. Thus there exists $z_1,z_2\in Z(G)$ such that $gz_1 = x^nz_2$, which by right multiplication we have that $g = x^nz_2z_1^{-1}$. Let $z = z_2z_1^{-1}$, and because $Z(G)$ is a subgroup of $G$, $z_2z_1^{-1}=z\in Z(G)$. 

      Hence any element $g\in G$ can be written as $g = x^nz$, where $n\in \mathbb{Z}$ and $z\in Z(G)$. Then let $a,b\in G$ so that $a = x^jz_1$ and $b = x^kz_2$ for $j,k\in \mathbb{Z}$ and $z_1,z_2\in Z(G)$. Due to exponent rules and the fact that $z_1,z_2$ lie in the center of $G$, \begin{align*}
        ab &= (x^jz_1)(x^kz_2)\\
        &= (z_1x^j)(x^kz_2) \\
        &= z_1(x^jx^k)z_2 \\
        &= z_2(x^kx^j)z_1 \\
        &= (z_2x^k)(x^jz_1) \\
        &= (x^kz_2)(x^jz_1) = ba.
      \end{align*} Because $a,b$ were arbitrary elements of $G$ it follows that $G$ is abelian.
    \end{proof}
\end{enumerate}
\end{document}