\documentclass[11pt]{article}

% packages
\usepackage{physics}
% margin spacing
\usepackage[top=1in, bottom=1in, left=0.5in, right=0.5in]{geometry}
\usepackage{hanging}
\usepackage{amsfonts, amsmath, amssymb, amsthm}
\usepackage{systeme}
\usepackage[none]{hyphenat}
\usepackage{fancyhdr}
\usepackage[nottoc, notlot, notlof]{tocbibind}
\usepackage{graphicx}
\graphicspath{{./images/}}
\usepackage{float}
\usepackage{siunitx}
\usepackage{esint}
\usepackage{cancel}
\usepackage{enumitem}

% permutations (second line is for spacing)
\usepackage{permute}
\renewcommand*\pmtseparator{\,}

% colors
\usepackage{xcolor}
\definecolor{p}{HTML}{FFDDDD}
\definecolor{g}{HTML}{D9FFDF}
\definecolor{y}{HTML}{FFFFCF}
\definecolor{b}{HTML}{D9FFFF}
\definecolor{o}{HTML}{FADECB}
%\definecolor{}{HTML}{}

% \highlight[<color>]{<stuff>}
\newcommand{\highlight}[2][p]{\mathchoice%
  {\colorbox{#1}{$\displaystyle#2$}}%
  {\colorbox{#1}{$\textstyle#2$}}%
  {\colorbox{#1}{$\scriptstyle#2$}}%
  {\colorbox{#1}{$\scriptscriptstyle#2$}}}%

% header/footer formatting
\pagestyle{fancy}
\fancyhead{}
\fancyfoot{}
\fancyhead[L]{MAS5311}
\fancyhead[C]{Assignment 9}
\fancyhead[R]{Sai Sivakumar}
\fancyfoot[R]{\thepage}
\renewcommand{\headrulewidth}{1pt}

% paragraph indentation/spacing
\setlength{\parindent}{0cm}
\setlength{\parskip}{10pt}
\renewcommand{\baselinestretch}{1.25}

% extra commands defined here
\newcommand{\ihat}{\boldsymbol{\hat{\textbf{\i}}}}
\newcommand{\jhat}{\boldsymbol{\hat{\textbf{\j}}}}
\newcommand{\dr}{\vec{r}~^{\prime}(t)}
\newcommand{\dx}{x^{\prime}(t)}
\newcommand{\dy}{y^{\prime}(t)}

\newcommand{\br}[1]{\left(#1\right)}
\newcommand{\sbr}[1]{\left[#1\right]}
\newcommand{\cbr}[1]{\left\{#1\right\}}

\newcommand{\dprime}{\prime\prime}
\newcommand{\lap}[2]{\mathcal{L}[#1](#2)}

\newcommand{\divides}{\mid}

% bracket notation for inner product
\usepackage{mathtools}

\DeclarePairedDelimiterX{\abr}[1]{\langle}{\rangle}{#1}

\DeclareMathOperator{\Span}{span}
\DeclareMathOperator{\nullity}{nullity}
\DeclareMathOperator\Aut{Aut}
\DeclareMathOperator\Inn{Inn}
\DeclareMathOperator{\Orb}{Orb}
\DeclareMathOperator{\lcm}{lcm}

% set page count index to begin from 1
\setcounter{page}{1}

\begin{document}
\begin{enumerate}
    \item (DF4.5.40) Prove that the number of Sylow $p$-subgroups of $GL_2(\mathbb{F}_p)$ is $p+1$. [Exhibit two distinct Sylow $p$-subgroups.]
    \begin{proof}
      Let $P$ be any Sylow $p$-subgroup of $GL_2(\mathbb{F}_p)$ where $p$ is prime. 

      The number of Sylow $p$-subgroups of $GL_2(\mathbb{F}_p)$ is the index in $G$ of the normalizer $N_{GL_2(\mathbb{F}_p)}(P)$. The order of $GL_2(\mathbb{F}_p)$ is (by an earlier result) $(p^2-1)(p^2-p) = (p+1)(p-1)^2p$, so the order of $P$ must be $p$. Hence $P$ is cyclic.

      Consider the Sylow $p$-subgroup
      \[Q = \left\langle \begin{pmatrix}
        1 & 1 \\ 0 & 1
      \end{pmatrix} \right\rangle = \cbr{\begin{pmatrix}
        1 & 1 \\ 0 & 1
      \end{pmatrix}, \begin{pmatrix}
        1 & 2 \\ 0 & 1
      \end{pmatrix}, \dots, \begin{pmatrix}
        1 & p-1 \\ 0 & 1
      \end{pmatrix}, \begin{pmatrix}
        1 & 0 \\ 0 & 1
      \end{pmatrix}}\]
      of $GL_2(\mathbb{F}_p)$. Consider conjugating the given generator of $Q$ by an arbitary matrix \[S = \begin{pmatrix}
        a & b \\ c & d
      \end{pmatrix},\] and identify the form of a matrix belonging to the normalizer $N_{GL_2(\mathbb{F}_p)}(Q)$. We have \begin{align*}
        S\begin{pmatrix}
          1 & 1 \\ 0 & 1
        \end{pmatrix}S^{-1} &= \br{ad-bc}^{-1}\begin{pmatrix}
          a & b \\ c & d
        \end{pmatrix}\begin{pmatrix}
          1 & 1 \\ 0 & 1
        \end{pmatrix}\begin{pmatrix}
          d & -b \\ -c & a
        \end{pmatrix}\\
        &= \br{ad-bc}^{-1}\begin{pmatrix}
          ad-ca-cb & a^2 \\ -c^2 & -bc + ac + ad
        \end{pmatrix};
      \end{align*} and if we demand that $S\in N_{GL_2(\mathbb{F}_p)}(Q)$, we have that \[\br{ad-bc}^{-1}\begin{pmatrix}
        ad-ca-cb & a^2 \\ -c^2 & -bc + ac + ad
      \end{pmatrix} = \begin{pmatrix}
        1 & q \\ 0 & 1
      \end{pmatrix}\] for $1 \leq q \leq p-1$, as conjugation must send a generator of $Q$ to another generator of $Q$. This occurs if and only if $c= 0$ and $a,d\neq 0$, in which case we find that \[\begin{pmatrix}
        1 & ad^{-1} \\ 0 & 1
      \end{pmatrix} = \begin{pmatrix}
        1 & q \\ 0 & 1
      \end{pmatrix}.\]
      Hence \[N_{GL_2(\mathbb{F}_p)}(Q) = \cbr{\begin{pmatrix}
        a & b \\ 0 & d
      \end{pmatrix}\colon b\in \mathbb{F}_p,~  a,d\in \mathbb{F}_p^{\times}}.\] Since $\abs{\mathbb{F}_p} = p$ and $\abs{\mathbb{F}_p^{\times}} = p-1$, there are $p$ options for $b$ and $p-1$ options each for $a$ and $d$ so that $\abs{N_{GL_2(\mathbb{F}_p)}(Q)} = (p-1)^2p$. Hence the number of Sylow $p$-subgroups of $GL_2(\mathbb{F}_p)$ is \[\abs{GL_2(\mathbb{F}_p)\colon N_{GL_2(\mathbb{F}_p)}(Q)} = \frac{(p+1)(p-1)^2p}{(p-1)^2p} = p+1.\]
    \end{proof}
    To exhibit two distinct Sylow $p$-subgroups, conjugate one subgroup by a matrix which does not normalize that group. For instance, using the same subgroup \[Q = \left\langle \begin{pmatrix}
      1 & 1 \\ 0 & 1
    \end{pmatrix} \right\rangle,\] we conjugate its generator by the matrix \[T = \begin{pmatrix}
      1 & 0 \\ 1 & 1
    \end{pmatrix}\] so that \begin{align*}
      T\begin{pmatrix}
        1 & 1 \\ 0 & 1
      \end{pmatrix} T^{-1} &= \begin{pmatrix}
      1 & 0 \\ 1 & 1
    \end{pmatrix}\begin{pmatrix}
        1 & 1 \\ 0 & 1
      \end{pmatrix} \begin{pmatrix}
        1 & 0 \\ -1 & 1
      \end{pmatrix}\\
      &= \begin{pmatrix}
        0 & 1 \\ -1 & 2
      \end{pmatrix}.
    \end{align*} Then \[Q = \left\langle \begin{pmatrix}
      1 & 1 \\ 0 & 1
    \end{pmatrix} \right\rangle = \cbr{\begin{pmatrix}
      1 & 1 \\ 0 & 1
    \end{pmatrix}, \begin{pmatrix}
      1 & 2 \\ 0 & 1
    \end{pmatrix}, \dots, \begin{pmatrix}
      1 & p-1 \\ 0 & 1
    \end{pmatrix}, \begin{pmatrix}
      1 & 0 \\ 0 & 1
    \end{pmatrix}}\] and \[TQT^{-1} = \left\langle \begin{pmatrix}
      0 & 1 \\ -1 & 2
    \end{pmatrix}\right \rangle = \cbr{\begin{pmatrix}
      0 & 1 \\ -1 & 2
    \end{pmatrix}, \begin{pmatrix}
      -1 & 2 \\ -2 & 3
    \end{pmatrix}, \dots, \begin{pmatrix}
      -(p-2) & p-1 \\ -(p-1) & 0
    \end{pmatrix}, \begin{pmatrix}
      1 & 0 \\ 0 & 1
    \end{pmatrix}}\] are distinct Sylow $p$-subgroups of $GL_2(\mathbb{F}_p)$.
    \item (Auxiliary result for DF4.5.50) Let $H$ be a subset of $G$ and let $a\in G$. Then $N_G(aHa^{-1}) = aN_G(H)a^{-1}$.
    \begin{proof}
      Let $H$, $G$, and $a\in G$ be as given. We have that $aN_G(H)a^{-1} = \cbr{axa^{-1} \colon x\in N_G(H)}$, and we check that any element $axa^{-1}$ with $x\in N_G(H)$ sends $aHa^{-1}$ to itself by conjugation. Indeed, \[axa^{-1} aHa^{-1} (axa^{-1})^{-1} = axa^{-1} aHa^{-1} ax^{-1}a^{-1} = axH x^{-1}a^{-1}= aHa^{-1},\] so that $aN_G(H)a^{-1} \subseteq N_G(aHa^{-1})$

      Similarly, $N_G(aHa^{-1}) = \cbr{x\in G \colon xaHa^{-1}x^{-1} = H}$. With $x\in N_G(aHa^{-1})$, if $a^{-1}xa\in N_G(H)$, then $N_G(aHa^{-1}) \subseteq aN_G(H)a^{-1}$, which follows since \[a^{-1}xa H (a^{-1}xa)^{-1} = a^{-1}xa H a^{-1}x^{-1}a = a^{-1}a H a^{-1}a = H.\]

      Hence $N_G(aHa^{-1}) = aN_G(H)a^{-1}$.
    \end{proof}
    \item (DF4.5.50) Prove that if $U$ and $W$ are normal subsets of a Sylow $p$-subgroup $P$ of $G$ then $U$ is conjugate to $W$ in $G$ if and only if $U$ is conjugate to $W$ in $N_G(P)$. Deduce that two elements in the
    center of $P$ are conjugate in $G$ if and only if they are conjugate in $N_G(P)$. (A subset $U$ of
    $P$ is normal in $P$ if $N_P(U) = P$.)
    \begin{proof}
      Let $U$ and $W$ be normal subsets of a Sylow $p$-subgroup $P$ of $G$ as given. 

      Suppose that $U$ is conjugate to $W$ in $N_G(P)$, so that there exists $n\in N_G(P)$ such that $nUn^{-1} = W$. Clearly $n \in G$, so $U$ is conjugate to $W$ in $G$.

      Conversely, suppose that $U$ is conjugate to $W$ in $G$, so that there exists $g\in G$ such that $gUg^{-1} = W$. Then by using the auxiliary result, we have that \[gPg^{-1} = gN_P(U)g^{-1} \leq gN_G(U)g^{-1} = N_G(gUg^{-1}) = N_G(W).\]

      Because $gPg^{-1}\leq N_G(W) \leq G$ and both $P, gPg^{-1}$ are Sylow $p$-subgroups of $G$, it follows that $P, gPg^{-1}$ are Sylow $p$-subgroups of $N_G(W)$ and so are conjugate to each other in $N_G(W)$. There exists $n\in N_G(W)$ such that $P = ngPg^{-1}n^{-1} = ngP (ng)^{-1}$, and so $ng \in N_G(P)$.
      
      Since $gUg^{-1} = W$ and $n\in N_G(W)$, we have \[ng U (ng)^{-1} = ng U g^{-1}n^{-1} = nWn^{-1} = W,\] so that $ng$ sends $U$ to $W$ by conjugation. Since $ng \in N_G(P)$, we have that $U$ is conjugate to $W$ in $N_G(P)$.

      The center $Z(P)$ is normal in $P$, so that the singleton subsets $U = \cbr{u}$ and $W = \cbr{w}$ of $Z(P)$ are normal subsets of $P$. It follows by the above result that the two elements $u,w\in Z(P)$ are conjugate (since the singleton sets $U$ and $W$ are conjugate) in $G$ if and only if they are conjugate in $N_G(P)$.
    \end{proof}
    \item (DF5.4.7) Prove that if $p$ is a prime and $P$ is a non-abelian group of order $p^3$ then $P^{\prime} = Z(P)$.
    \begin{proof}
      Let $P$ be a non-abelian group of order $p^3$ with $p$ prime as given. Observe that $P$ is a $p$-group so it has a nontrivial center $Z(P)$; furthermore $P$ is non-abelian so $Z(P)\neq P$. This means that $\abs{Z(P)}$ is either of order $p$ or $p^2$. But if $\abs{Z(P)} = p^2$, then $\abs{P/Z(P)} = p$, which means $P/Z(P)$ is cyclic. By a previous result we know this would imply that $P$ is abelian, a contradiction. Hence $Z(P)$ is of order $p$. 

      The quotient $P/Z(P)$ has order $p^2$, which by a previous result, we have that $P/Z(P)$ is abelian. Therefore, $P^{\prime}\leq Z(P)$ (by Proposition 7). Since $P$ is non-abelian, we have that $P^{\prime}$ is nontrivial and so we must have that $\abs{P^{\prime}} \geq p$, so that $P^{\prime} = Z(P)$ as desired.
    \end{proof}
    \item (DF5.5.8) Construct a non-abelian group of order $75$. Classify all groups of order $75$ (there are three of them).
    
    Given $Z_3$ which has order $3$ and $Z_5\times Z_5$ which has order $25$, we wish to take the semidirect product $(Z_5\times Z_5)\rtimes Z_3$ with respect to the (nontrivial) homomorphism $\varphi\colon Z_3 \to \Aut(Z_5\times Z_5)$ which sends the nontrivial elements of $Z_3$ (which have order $3$) to automorphisms of $Z_5\times Z_5$ of order $3$. The choice of taking the product is motivated by the fact that in a group of order $75$, the number of Sylow $5$-subgroups must be $1$ (since this number must divide $3$), and hence the group of order $25$ must be normal in the group of order $75$. Furthermore, the group of order $25$ and the group of order $3$ intersect trivially since their orders are coprime.
    
    Note that the order of $\Aut(Z_5\times Z_5)$ is the order of $GL_2(\mathbb{F}_5) = (5^2- 1)(5^2-5) = 24\cdot 20 = 480$, which is divisible by $3$. By Cayley's theorem, $\Aut(Z_5\times Z_5)$ has a subgroup of order $3$, so there exist automorphisms of order $3$ in $\Aut(Z_5\times Z_5)$.
    
    What remains is to exhibit an automorphism of $Z_5\times Z_5$ of order $3$, and send a nontrivial element of $Z_3$ to this automorphism via $\varphi$. Because $\Aut(Z_5\times Z_5) \cong GL_2(\mathbb{F}_5)$, it suffices to find a non-identity matrix \[A = \begin{pmatrix}
      a & b \\ c & d
    \end{pmatrix}\] where $a,b,c,d\in \mathbb{F}_5$ and $ad-bc\neq 0$, such that $A^3 = I$. By expanding out $A^3$ and setting it equal to $I$, we have \[\begin{pmatrix}
      a^3 + 2 a b c + b c d & b (a^2 + a d + b c + d^2) \\
      c (a^2 + a d + b c + d^2) & a b c + 2 b c d + d^3
    \end{pmatrix} \equiv \begin{pmatrix}
      1 & 0 \\ 0 & 1
    \end{pmatrix}\pmod{5}.\] We cannot have that $b\equiv c \equiv 0$, since it forces $a\equiv d\equiv 1$ (there are no elements of order $3$ in $Z_5^{\times}$), which forms the identity matrix. So $b\not\equiv 0$ and $c\not\equiv 0$. Then simplify the problem by taking $d\equiv 0$; this is a guess/ansatz made in order to simplify the algebra. With these choices, the following system of congruences is formed: \begin{align*}
      a^3 + 2 a b c + b c d &\equiv \boxed{a^3+ 2abc \equiv 1}\\
      a^2 + a d + b c + d^2 &\equiv \boxed{a^2+bc \equiv 0}\\
      a b c + 2 b c d + d^3 &\equiv \boxed{abc \equiv 1}
    \end{align*} Substituting the third congruence into the first congruence and simplifying yields $a^3 \equiv 4$, and by inspecting the values $z^3$ for $z\in \mathbb{F}_5$, we find that only $a\equiv 4$ will satisfy the congruence. Then the second congruence with $a\equiv 4$ yields $bc\equiv 4$, and there are many choices of $b$ and $c$ which satisfy this congruence. We take $b\equiv 2$ and $c\equiv 2$ for sake of example. Thus an automorphism of $\Aut(Z_5\times Z_5)$ of order $3$ may be expressed in matrix form as \[A = \begin{pmatrix}
      4 & 2 \\ 2 & 0
    \end{pmatrix},\] and by sending one of the nontrivial elements of $Z_3$ to the preimage of $A$ (under some isomorphism from $\Aut(Z_5\times Z_5)$ to $GL_2(\mathbb{F}_5)$) via $\varphi$, the semidirect product $(Z_5\times Z_5)\rtimes_{\varphi} Z_3$ is a group of order $75$ which is not abelian because $\varphi$ is not a trivial homomorphism.

    The three groups of order $75$ up to isomorphism are: \begin{align*}
      Z_3 \times Z_{25} &\quad \text{(abelian)}\\
      Z_3 \times Z_5\times Z_5 &\quad \text{(abelian)}\\
      \text{with $\varphi$ not trivial} \quad (Z_5\times Z_5)\rtimes_{\varphi} Z_3 &\quad \text{(non-abelian)}
    \end{align*}
    \begin{proof}
      Let $G$ be a group of order $75 = 3\cdot 5^2$. Then the number of Sylow $5$-subgroups is $1$ because $1+ 5 = 6$ does not divide $3$. Hence there is a normal Sylow $5$-subgroup of order $25$ in $G$.

      The number of Sylow $3$-subgroups is either $1$ or $25$. Suppose there is only one Sylow $3$-subgroup. The Sylow $5$-subgroup $P_5$ must intersect trivially with the Sylow $3$-subgroup $P_3$ (as $3$ and $25$ are coprime, there could not be any elements in common). Furthermore, because both subgroups are normal then the elements of the Sylow $3$-subgroup commute with the elements of the Sylow $5$-subgroup (by a result proven on our midterm). 
      
      Hence $G$ is isomorphic to the direct product of these two groups, since the map from $P_3\times P_5$ to $G$ sending $(x_1,x_2)\mapsto x_1x_2$ is an injective homomorphism due to the fact that the order of an element of the form $x_1x_2$ will be the product of the orders of $x_1$ and $x_2$ (these elements commute and their orders will be relatively prime). Thus the kernel of the homomorphism must be trivial, so the homomorphism is indeed injective. The order of $P_3\times P_5$ is $75$, the same size as $G$, so the homomorphism is also bijective as a result and so is an isomorphism.
      
      There is only one group (cyclic) of order $3$ up to isomorphism, $Z_3$, and there are two groups (both abelian) of order $25$ up to isomorphism, so the only options for $P_3\times P_5$ are $Z_3 \times Z_{25}$ and $Z_3 \times (Z_5\times Z_5)\cong Z_3 \times Z_5\times Z_5$. Both of these groups are abelian. By the Fundamental Theorem of Finitely Generated Abelian Groups, we also know that these are the only abelian groups of order $75$.

      Suppose there are $25$ Sylow $3$-subgroups. We cannot say in this case that a given Sylow $3$-subgroup is normal in $G$. But because the Sylow $5$-subgroup $P_5$ is normal in $G$ and for the same reason as above that $P_5$ intersects trivially with any Sylow $3$-subgroup $P_3\cong Z_3$, we have that \[P_5\rtimes P_3 \cong G.\] Suppose that $P_5$ is isomorphic to $Z_{25}$. Then the order of $\Aut(Z_{25})$ is $\phi(25) = 20$ ($\phi$ is the Euler totient function), and so there are no automorphisms of order $3$ of $Z_{25}$. This means any semidirect product of $Z_{25}\rtimes_{\varphi} P_3 \cong Z_{25}\rtimes_{\varphi} Z_3$ will be an abelian group of order $75$ as the homomorphism $\varphi$ must be trivial.
      
      Thus for a non-abelian semidirect product we ought to take $P_5 \cong Z_5\times Z_5$; and observe that we have already constructed above a non-abelian group of order $75$ via a semidirect product $(Z_5\times Z_5)\rtimes Z_3$. It remains to show that this is the only non-abelian group of order $75$. Observe that for any nontrivial homomorphisms $\varphi$ and $\phi$ from $Z_3\to \Aut(Z_5\times Z_5)$,  $\varphi(Z_3)$ is conjugate to $\phi(Z_3)$. We can see this as the corresponding isomorphic subgroups in $GL_2(\mathbb{F}_5)$ of $\varphi(Z_3)$ and $\phi(Z_3)$ contain only the identity, a matrix, and its inverse, and so we may find a suitable change of basis matrix to conjugate one subgroup into the other.

      Thus, by a proposition proved in class, it did not matter which nontrivial homomorphism $\varphi$ from $Z_3\to \Aut(Z_5\times Z_5)$ we used in the semidirect product, and so the non-abelian group constructed earlier is the only one up to isomorphism.

      Thus there are only three groups of order $75$ as outlined above.
    \end{proof}
\end{enumerate}
\end{document}