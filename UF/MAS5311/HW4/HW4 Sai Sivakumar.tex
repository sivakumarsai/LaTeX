\documentclass[11pt]{article}

% packages
\usepackage{physics}
% margin spacing
\usepackage[top=1in, bottom=1in, left=0.5in, right=0.5in]{geometry}
\usepackage{hanging}
\usepackage{amsfonts, amsmath, amssymb, amsthm}
\usepackage{systeme}
\usepackage[none]{hyphenat}
\usepackage{fancyhdr}
\usepackage[nottoc, notlot, notlof]{tocbibind}
\usepackage{graphicx}
\graphicspath{{./images/}}
\usepackage{float}
\usepackage{siunitx}
\usepackage{esint}
\usepackage{cancel}
\usepackage{enumitem}

% permutations (second line is for spacing)
\usepackage{permute}
\renewcommand*\pmtseparator{\,}

% colors
\usepackage{xcolor}
\definecolor{p}{HTML}{FFDDDD}
\definecolor{g}{HTML}{D9FFDF}
\definecolor{y}{HTML}{FFFFCF}
\definecolor{b}{HTML}{D9FFFF}
\definecolor{o}{HTML}{FADECB}
%\definecolor{}{HTML}{}

% \highlight[<color>]{<stuff>}
\newcommand{\highlight}[2][p]{\mathchoice%
  {\colorbox{#1}{$\displaystyle#2$}}%
  {\colorbox{#1}{$\textstyle#2$}}%
  {\colorbox{#1}{$\scriptstyle#2$}}%
  {\colorbox{#1}{$\scriptscriptstyle#2$}}}%

% header/footer formatting
\pagestyle{fancy}
\fancyhead{}
\fancyfoot{}
\fancyhead[L]{MAS5311}
\fancyhead[C]{Assignment 4}
\fancyhead[R]{Sai Sivakumar}
\fancyfoot[R]{\thepage}
\renewcommand{\headrulewidth}{1pt}

% paragraph indentation/spacing
\setlength{\parindent}{0cm}
\setlength{\parskip}{10pt}
\renewcommand{\baselinestretch}{1.25}

% extra commands defined here
\newcommand{\ihat}{\boldsymbol{\hat{\textbf{\i}}}}
\newcommand{\jhat}{\boldsymbol{\hat{\textbf{\j}}}}
\newcommand{\dr}{\vec{r}~^{\prime}(t)}
\newcommand{\dx}{x^{\prime}(t)}
\newcommand{\dy}{y^{\prime}(t)}

\newcommand{\br}[1]{\left(#1\right)}
\newcommand{\sbr}[1]{\left[#1\right]}
\newcommand{\cbr}[1]{\left\{#1\right\}}

\newcommand{\dprime}{\prime\prime}
\newcommand{\lap}[2]{\mathcal{L}[#1](#2)}

\newcommand{\divides}{\mid}

% bracket notation for inner product
\usepackage{mathtools}

\DeclarePairedDelimiterX{\abr}[1]{\langle}{\rangle}{#1}

\DeclareMathOperator{\Span}{span}
\DeclareMathOperator{\nullity}{nullity}
\DeclareMathOperator\Aut{Aut}
\DeclareMathOperator\Inn{Inn}
\DeclareMathOperator{\lcm}{lcm}

% set page count index to begin from 1
\setcounter{page}{1}

\begin{document}

\begin{enumerate}
    \item (DF2.2.4) For each of $S_3$, $D_8$, and $Q_8$, compute the centralizers of each element and find the center of each group. Does Lagrange's Theorem simplify your work?
    
    Recall that the centralizer of an element in a group $G$ and the center of a group $G$ are both subgroups of $G$. This means that by Lagrange's theorem, the order of these subgroups divides the order of the group.
    
    So because $\abs{S_3} = 6 = 2\cdot 3$, $\abs{D_8} = 8 = 2^3$, and $Q_8 = 8 = 2^3$, our work is simplified because we only need to find subgroups whose order divides these orders (of course, a group may not have a subgroup whose order is a particular divisor of the group's order).

    Then for each group, \begin{align*}
      S_3 &= \cbr{1, \pmt*{(12)}, \pmt*{(13)}, \pmt*{(23)}, \pmt*{(123)}, \pmt*{(132)} }\\
      D_8 &= \cbr{1, r, r^2, r^3, s, sr, sr^2, sr^3} \\
      Q_8 &= \cbr{1,-1,i,-i,j,-j,k,-k},
    \end{align*} we compute the centralizers of each element. So for every element $g \in G$, we compute $C_G(g)$ by inspection (trying out elements in the group which commute until we reach a suitable number of elements).
    In these groups, \begin{align*}
      C_{S_3}(1) &= S_3 &\quad C_{D_8}(1) &= D_8 &\quad C_{Q_8}(1) &= Q_8 \\
      C_{S_3}(\pmt*{(12)}) &= \cbr{1, \pmt*{(12)}} &\quad C_{D_8}(r) &= \cbr{1,r,r^2,r^3} &\quad C_{Q_8}(-1) &= Q_8 \\
      C_{S_3}(\pmt*{(13)}) &= \cbr{1, \pmt*{(13)}} &\quad C_{D_8}(r^2) &= D_8 &\quad C_{Q_8}(i) &= \cbr{1,-1,i,-i} \\
      C_{S_3}(\pmt*{(23)}) &= \cbr{1, \pmt*{(23)}} &\quad C_{D_8}(r^3) &= \cbr{1,r,r^2,r^3} &\quad C_{Q_8}(-i) &= \cbr{1,-1,i,-i} \\
      C_{S_3}(\pmt*{(123)}) &= \cbr{1, \pmt*{(132)}} &\quad C_{D_8}(s) &= \cbr{1,r^2, s, sr^2} &\quad C_{Q_8}(j) &= \cbr{1,-1,j,-j} \\
      C_{S_3}(\pmt*{(132)}) &= \cbr{1, \pmt*{(123)}} &\quad C_{D_8}(sr) &= \cbr{1,r^2, sr, sr^3} &\quad C_{Q_8}(-j) &= \cbr{1,-1,j,-j} \\
      & &\quad C_{D_8}(sr^2) &= \cbr{1,r^2, sr^2, s} &\quad C_{Q_8}(k) &= \cbr{1,-1,k,-k} \\
      & &\quad C_{D_8}(sr^3) &= \cbr{1,r^2, sr^3, sr} &\quad C_{Q_8}(-k) &= \cbr{1,-1,k,-k} 
    \end{align*} It is clear from enumerating all of these centralizers that $Z(S_3) = \cbr{1}$, $Z(D_8) = \cbr{1,r^2}$, and $Z(Q_8) = \cbr{1,-1}$.
    \item (DF2.3.26) Let $Z_n$ be a cyclic group of order $n$ and for each integer $a$ let \[\sigma_a\colon Z_n \to Z_n\quad \text{ by }\quad \sigma_a(x) = x^a\text{ for all } x\in Z_n.\] \begin{enumerate}[label=(\alph*)]
      \item Prove that $\sigma_a$ is an automorphism of $Z_n$ if and only if $a$ and $n$ are relatively prime. \begin{proof}
        Let $Z_n$ be a cyclic group of order $n$ and $\sigma_a \colon Z_n\to Z_n$ be as given and suppose that $a$ is coprime to $n$. Then there exist $s,t\in\mathbb{Z}$ such that $1 = as+nt$.
        
        We show $\sigma_a$ is surjective. For any $y\in Z_n$, we have $y = y^1 = y^{as+nt} = (y^s)^a \cdot (y^n)^t = (y^s)^a \cdot 1^t = (y^s)^a$, where the fact that $\abs{y}\mid n$ was used in the fourth equality. Then for any $y\in Z_n$, we have that $y^s \in Z_n$ and so $\sigma_a(y^s) = y$.

        We show $\sigma_a$ is injective by showing its kernel is the trivial subgroup of $Z_n$. Suppose there exists $x\in Z_n$ where $x\neq 1$, such that $\sigma_a(x) = x^a = 1$. Observe that because $\gcd(a,n) = 1$, there exist integers $s,t$ such that $1 = as+nt$, so that $x = x^1 = x^{as+nt} = (x^a)^s = 1^s = 1$. But $x\neq 1$, which is a contradiction. Hence $\ker{\sigma_a} = \cbr{1}$, and $\sigma_a$ is injective.

        This map preserves the group operation as well. For $x,y\in Z_n$, because cyclic groups are abelian, $\sigma_a(xy) = (xy)^a = x^ay^a = \sigma_a(x)\sigma_a(y)$.

        Conversely, suppose $\sigma_a$ is an automorphism. When $n = 1$, the cyclic group $Z_n = \cbr{1}$, and so for every integer $a$, $\gcd(a,1) = 1$. So without loss of generality, consider cyclic groups $Z_n$ where $n > 1$. Then by contradiction, suppose that $\gcd(a,n) > 1$. Then the map $\sigma_a$ cannot be injective as assumed, because $s = \frac{n}{\gcd(a,n)} \in \mathbb{Z}$ and $t = \frac{a}{\gcd(a,n)}\in\mathbb{Z}$. Since $Z_n$ is a cyclic group, it is generated by some element $z \neq 1$ (since $n >1$), so that $Z_n = \abr{z}$, and $\abs{z} = n$. But $s < n$, since $\gcd(a,n) > 1$. Hence $z^s \neq 1$, and so \[\sigma_a(z^s) = (z^s)^a = \left(z^{\frac{n}{\gcd{a,n}}}\right)^{t\cdot \gcd(a,n)} = z^{tn} = (z^n)^t = 1^t = 1.\] So $\ker{\sigma_a}$ is not a trivial subgroup (contains a nontrivial element) of $Z_n$, and so $\sigma_a$ cannot be injective, which is in contradiction with the assumption that $\sigma_a$ is an automorphism. Hence $\gcd(a,n) = 1$.

        Therefore $\sigma_a$ is an automorphism of $Z_n$ if and only if $a$ and $n$ are relatively prime.
      \end{proof}
      \item Prove that $\sigma_a = \sigma_b$ if and only if $a\equiv b \pmod{n}$. \begin{proof}
        Let $\sigma_a$ be as given. Suppose $a\equiv b \pmod{n}$, so that there exists $k\in \mathbb{Z}$ such that $b = a+ kn$. Then for any element $x\in Z_n$, $\sigma_b(x) = x^b = x^{a+kn} = x^a(x^n)^k = x^a\cdot 1^k = x^a = \sigma_a(x)$. Since $x$ was arbitrary, the action of $\sigma_a$ and $\sigma_b$ agree on $Z_n$ and so the maps are equivalent as mappings from $Z_n$ to $Z_n$.

        Conversely, suppose $\sigma_a = \sigma_b$, so that for any $x\in Z_n$, $\sigma_a(x) = x^a = x^b = \sigma_b(x)$. Then $1 = x^bx^{-a} = x^{b-a}$, which implies that $n\mid b-a$, which by definition is equivalent to $b\equiv a \pmod{n}$, since there exists an integer $k$ such that $b - a= nk \iff b = a+nk$.
      \end{proof}
      \item Prove that \textit{every} automorphism of $Z_n$ is equal to $\sigma_a$ for some integer $a$. \begin{proof}
        Let $\sigma$ be an arbitrary automorphism of $Z_n$. Let $Z_n = \abr{z}$, so that $\abs{z} = n$ and since $z$ generates $Z_n$, we may write any element in $Z_n$ as a power of $z$. Because $\sigma$ is a bijection from $Z_n$ to $Z_n$, there exists an $a\in \mathbb{Z}$ such that $\sigma(z) = z^a$. Then for any element $x\in Z_n$, there exists an integer $k$ such that $x = z^k$. Then by properties of automorphisms, \[\sigma(x) = \sigma(z^k) = (\sigma(z))^k = (z^a)^k = (z^k)^a = x^a.\] Hence $\sigma = \sigma_a$, since $x$ was any element in $Z_n$. Since $\sigma$ was arbitrary, it follows that any automorphism of $Z_n$ is equivalent to $\sigma_a$ for some integer $a$.
      \end{proof}
      \item Prove that $\sigma_a\circ \sigma_b = \sigma_{ab}$. Deduce that the map $\overline{a}\mapsto \sigma_a$ is an isomorphism of $(\mathbb{Z}/n\mathbb{Z})^{\times}$ onto the automorphism group of $Z_n$ (so $\Aut(Z_n)$ is an abelian group of order $\varphi(n)$). \begin{proof}
        Let $\sigma_i$ be an automorphism of $Z_n$ as given. Then for any element $x\in Z_n$, \[(\sigma_a\circ \sigma_b)(x) = \sigma_a(\sigma(b)(x)) = \sigma_a(x^b) = (x^b)^a = x^{ab} = \sigma_{ab}(x).\] Note that $\sigma_{ab}$ is an automorphism if and only if $ab$ is coprime to $n$, which can be guaranteed if both $a$ and $b$ were already coprime to $n$. When $a$ and $b$ are coprime to $n$, $\sigma_a,\sigma_b$, and $\sigma_{ab}$ are automorphisms of $Z_n$. Hence $\sigma_a\circ \sigma_b = \sigma_{ab}$. This is enough to see that the map $\overline{a}\mapsto \sigma_a$ preserves the group operation; for $\overline{a}, \overline{b}\in (\mathbb{Z}/n\mathbb{Z})^{\times}$, \[\overline{a}\cdot \overline{b} = \overline{ab}\mapsto \sigma_{ab} = \sigma_a\circ \sigma_b.\] 

        Observe that the mapping is injective because in (b) we showed that $\sigma_a = \sigma_b$ if and only if $a\equiv b \pmod{n}$, which by definition of $(\mathbb{Z}/n\mathbb{Z})^{\times}$ is also equivalent to $\overline{a} = \overline{b}$. 
        
        Furthermore, the mapping is surjective because every automorphism $\sigma$ of $Z_n$ is equal to $\sigma_a$ for some integer $a$, so we can combine this with $\sigma_a = \sigma_b$ if and only if $a\equiv b \pmod{n}$, where $b$ can be chosen to be the remainder of dividing $a$ by $n$. Since $\overline{b}$ is a residue class of $(\mathbb{Z}/n\mathbb{Z})^{\times}$, we have that the preimage of $\sigma$ under this mapping is $\overline{b}$, so all automorphisms of $Z_n$ have a preimage in $(\mathbb{Z}/n\mathbb{Z})^{\times}$ under this mapping.

        Hence the mapping $\overline{a}\mapsto \sigma_a$ is an isomorphism from $(\mathbb{Z}/n\mathbb{Z})^{\times}$ onto $\Aut(Z_n)$, so the order of these groups are equal ($\abs{(\mathbb{Z}/n\mathbb{Z})^{\times}} = \abs{\Aut(Z_n)} = \varphi(n)$), and both groups are cyclic. Hence $\Aut(Z_n)$ is an abelian group of order $\varphi(n)$.
      \end{proof}
    \end{enumerate}
    \item (DF2.4.14) A group $H$ is called \textit{finitely generated} if there is a finite set $A$ such that $H=\abr{A}$. \begin{enumerate}[label=(\alph*)]
      \item Prove that every finite group is finitely generated. \begin{proof}
        Suppose $G$ is a group of finite order. Then it is clear that $G = \abr{G}$, since $G$ has finitely many elements and any finite product of elements and their inverses in $G$ is also in $G$, because $G$ is a group ($\abr{G}\subseteq G$). Furthermore, every element of $G$ can be seen as the product of one element, itself, of $G$ ($G \subseteq \abr{G}$). Hence $G = \abr{G}$ is finitely generated. 
      \end{proof}
      \item Prove that $\mathbb{Z}$ is finitely generated. \begin{proof}
        Observe that the additive group $\mathbb{Z}$ is generated by $\abr{1}$ (or $\abr{-1}$), since any integer multiple of $1$ (or $-1$) is also an integer ($\abr{\pm 1}\subseteq \mathbb{Z}$), and we may express every integer $n$ as $n\cdot 1$ (or $(-n) \cdot (-1)$) ($\mathbb{Z}\subseteq \abr{\pm 1}$). Hence $\mathbb{Z} = \abr{\pm 1}$ is finitely generated.
      \end{proof}
      \item Prove that every finitely generated subgroup of the additive group $\mathbb{Q}$ is cyclic. [If $H$ is a finitely generated subgroup of $\mathbb{Q}$, show that $H\leq \abr{1/k}$, where $k$ is the product of all the denominators which appear in a set of generators for $H$.] \begin{proof}
        Let $H$ be any finitely generated subgroup of the additive group $\mathbb{Q}$. Let $H$ be generated by the set of rational numbers $\cbr{\frac{a_1}{k_1},\frac{a_2}{k_2}, \dots, \frac{a_n}{k_n}}$, where $n$ is some positive integer ($H$ may be generated by the empty set, and in this case $H = \cbr{1}$ would already be cyclic).

        Then let $k = \prod_{i=1}^n k_i$, so that we may consider the cyclic subgroup $\abr{1/k} = \cbr{\dots, \frac{-2}{k}, \frac{-1}{k}, 1, \frac{1}{k}, \frac{2}{k}, \dots}$ of $\mathbb{Q}$. We can show that $H\leq \abr{1/k}$. 
        
        Clearly $H$ contains the same identity as $\abr{1/k}$ (by taking the empty product of the generators). Every element in $H$ is of the form $\sum$ so $H$ is a nonempty subset of $\abr{1/k}$. Then because $H$ is abelian (since addition commutes we can collect like terms), we may write every element in $H$ as $\sum_{i=1}^n c_i\frac{a_i}{k_i}$, for some integers $c_i\in \mathbb{Z}$. Then rewrite the sum where all terms have a common denominator $k$, say the sum is written in the form $\frac{C}{k}$, where $C = \sum_{i=1}^n a_ic_i\left(\prod_{j\neq i}k_j\right)$. Since $C\in \mathbb{Z}$, any element in $H$ is an element in $\abr{1/k}$, so that $H$ is a nonempty subset of $\abr{1/k}$.

        The group $H$ is closed under addition and taking inverses (subtraction). For any two elements $x = \sum_{i=1}^n c_i\frac{a_i}{k_i}, y = \sum_{i=1}^n d_i\frac{a_i}{k_i}$ of $H$, where $c_i, d_i$ are some integers, $x+y = \sum_{i=1}^n (c_i+d_i)\frac{a_i}{k_i}$ and $-x = \sum_{i=1}^n -c_i\frac{a_i}{k_i}$. Both are clearly elements of $H$. 
        
        Hence $H \leq \abr{1/k}$, and we know that subgroups of cyclic groups are cyclic, so $H$ is cyclic. Since $H$ was any finitely generated subgroup of $\mathbb{Q}$, it follows that any finitely generated subgroup of $\mathbb{Q}$ is cyclic.
      \end{proof}
      \item Prove that $\mathbb{Q}$ is not finitely generated. \begin{proof}
        Suppose via contradiction that $\mathbb{Q}$ is finitely generated, say by the set of rational numbers $\cbr{\frac{a_1}{k_1},\frac{a_2}{k_2}, \dots, \frac{a_n}{k_n}}$, where $n$ is a positive integer (clearly $n$ is not $0$). Then let $k = \prod_{i=1}^n k_i$. Observe that there is no way to form with a finite sum of these rational numbers the rational number $\frac{C}{k+1}$, since $k_i \nmid k+1$ (since $k\nmid k+1$) for $1\leq i\leq n$. This is in contradiction with the assumption that $\mathbb{Q}$ is finitely generated (we should be able to generate every rational number with a finite sum of rational numbers).

        Hence $\mathbb{Q}$ is not finitely generated.
      \end{proof}
    \end{enumerate}
    \item (DF2.4.16) A subgroup $M$ of a group $G$ is called a \textit{maximal subgroup} if $M\neq G$ and the only subgroups of $G$ which contain $M$ are $M$ and $G$. \begin{enumerate}[label=(\alph*)]
      \item Prove that if $H$ is a proper subgroup of the finite group $G$ then there is a maximal subgroup of $G$ containing $H$. \begin{proof}
        Let $H$ be a proper subgroup of $G$ as given. Then we may extend this subgroup into a larger subgroup of $G$ by generating the subgroup $H_1 = \abr{H,\cbr{m}}$, for some $m\in G$ not in $H$. This new subgroup $H_1$ is either $G$ itself or it is a subgroup of $G$ containing $H$. If $H_1 = G$, then $H$ is a maximal subgroup of $G$ containing $H$. Otherwise, we will have to extend $H_1$ into a larger subgroup of $G$ and see if this next subgroup is equal to $G$ or not.
        
        So we can form a recursive algorithm for generating even larger and larger subgroups of $G$ which contain $H$. Let $H_{i+1} = \abr{H_i,\cbr{m_i}}$, for $0 \leq i$, where $m_i$ is an element of $G$ not in $H_i$. Let $H_0 = H$. This algorithm terminates at the $j$-th step when there are no more elements $m_j$ not in $H_j$ such that the next subgroup containing $H$, $H_{j+1}$ is not equal to $G$; that is to say, if we extended $H_j$ any more we would form $G$. It follows that $H_j$ a maximal subgroup of $G$ containing $H$.

        This algorithm will terminate because $G$ is finite; furthermore $G$ is finitely generated. For instance, we could always take $m_i$ from a finite set that generates $G$, and so this algorithm is guaranteed to end in a number of steps less than or equal to the cardinality of this set.

        So in however many finite steps it takes to keep extending $H$ into larger and larger subgroups of $G$ (but not so large that the resulting subgroup is equal to $G$), we will reach a point where the resulting subgroup is indeed a maximal subgroup of $G$ containing $H$.
      \end{proof}
      \item Show that the subgroup of all rotations in a dihedral group is a maximal subgroup. \begin{proof}
        The subgroup of all rotations in a dihedral group $D_{2n}$ has order $n$. By Lagrange's theorem, the order of subgroups of $D_{2n}$ must divide $2n$.

        Observe that there are no factors of $2n$ strictly larger than $n$ aside from $2n$. Furthermore, any other group of order $n$ in $D_{2n}$ distinct from the subgroup of all rotations will not contain all $n$ rotations, so these other subgroups of order $n$ will not contain the subgroup of all rotations.

        Hence the subgroup of all rotations in a dihedral group is a maximal subgroup.
      \end{proof}
      \item Show that if $G = \abr{x}$ is a cyclic group of order $n\geq 1$ then a subgroup $H$ is maximal if and only if $H = \abr{x^p}$ for some prime $p$ dividing $n$. \begin{proof}
        Let $G = \abr{x}$ be a cyclic group of order $n\geq 1$ as given. Then suppose that $H = \abr{x^p}$ for some prime $p$ dividing $n$. Then suppose by way of contradiction that there is a proper subgroup $H^{\prime}$ of $G$ containing $H$, so that $H$ is a proper subgroup of $H^{\prime}$. Then it follows that there is an element $y = x^m$ in $H^{\prime}$ \textit{not} in $H$, where $m$ is coprime to $p$. If $m$ was not coprime to $p$, then it follows that $m$ is a multiple of $p$ and so this element would really an element of $H$.

        Because $m$ and $p$ are coprime, there exist integers $s,t$ such that $sm+pt = 1$. Since $H^{\prime}$ is a group (which contains $H = \abr{x^p}$), we may take the product $(x^m)^s(x^p)^t = x^{ms+pt} = x$. Then we may take any power of $x$ and so it follows that $H^{\prime} = G$, which is in contradiction to the assumption that $H^{\prime}$ was a proper subgroup of $G$.

        Hence $H = \abr{x^p}$ is a maximal subgroup of $G$.

        Conversely, suppose $H$ is a maximal subgroup of $G$. Because all subgroups of cyclic groups are cyclic, $H = \abr{x^m}$ for some integer $m$. Without loss of generality, let $m$ be a positive integer strictly greater than $1$ (as $m=1$ makes $H=G$). Then by way of contradiction, suppose $m$ is composite, so that there exist integers $a,b$ such that $m = ab$. Then it follows that $H = \abr{x^m} = \abr{x^{ab}} \leq \abr{x^b}$, since all elements of $H$ are in the form $x^{nab} = (x^b)^{na}$, which are elements of $\abr{x^b}$. Hence $H$ is not a maximal subgroup of $G$ as assumed, so $k$ is not composite as assumed.

        Hence $k$ is a prime number $p$. Furthermore, $p$ divides $n$ because otherwise $\gcd{p,n} = 1$ (this happens when primes are either larger than $n$ or if $p$ is not a divisor of $n$). If $\gcd(p,n) = 1$, then the order of $H = \abr{x^p}$ is $n/\gcd(p,n) = n/1 = n$, which makes $H = G$, but $H$ is a maximal subgroup of $G$, so $H$ cannot equal $G$.

        Hence $H = \abr{x^p}$ for some prime $p$ which divides $n$.

        Therefore, $H$ is a maximal subgroup of $G$ if and only if $H = \abr{x^p}$ for some prime $p$ dividing $n$.
      \end{proof}
    \end{enumerate}
\end{enumerate}
\end{document}