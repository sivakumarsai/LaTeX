\documentclass[11pt]{article}

% packages
\usepackage{physics}
% margin spacing
\usepackage[top=1in, bottom=1in, left=0.5in, right=0.5in]{geometry}
\usepackage{hanging}
\usepackage{amsfonts, amsmath, amssymb, amsthm}
\usepackage{systeme}
\usepackage[none]{hyphenat}
\usepackage{fancyhdr}
\usepackage[nottoc, notlot, notlof]{tocbibind}
\usepackage{graphicx}
\graphicspath{{./images/}}
\usepackage{float}
\usepackage{siunitx}
\usepackage{esint}
\usepackage{cancel}

% permutations (second line is for spacing)
\usepackage{permute}
\renewcommand*\pmtseparator{\,}

% colors
\usepackage{xcolor}
\definecolor{p}{HTML}{FFDDDD}
\definecolor{g}{HTML}{D9FFDF}
\definecolor{y}{HTML}{FFFFCF}
\definecolor{b}{HTML}{D9FFFF}
\definecolor{o}{HTML}{FADECB}
%\definecolor{}{HTML}{}

% \highlight[<color>]{<stuff>}
\newcommand{\highlight}[2][p]{\mathchoice%
  {\colorbox{#1}{$\displaystyle#2$}}%
  {\colorbox{#1}{$\textstyle#2$}}%
  {\colorbox{#1}{$\scriptstyle#2$}}%
  {\colorbox{#1}{$\scriptscriptstyle#2$}}}%

% header/footer formatting
\pagestyle{fancy}
\fancyhead{}
\fancyfoot{}
\fancyhead[L]{MAS5311}
\fancyhead[C]{Assignment 1}
\fancyhead[R]{Sai Sivakumar}
\fancyfoot[R]{\thepage}
\renewcommand{\headrulewidth}{1pt}

% paragraph indentation/spacing
\setlength{\parindent}{0cm}
\setlength{\parskip}{10pt}
\renewcommand{\baselinestretch}{1.25}

% extra commands defined here
\newcommand{\ihat}{\boldsymbol{\hat{\textbf{\i}}}}
\newcommand{\jhat}{\boldsymbol{\hat{\textbf{\j}}}}
\newcommand{\dr}{\vec{r}~^{\prime}(t)}
\newcommand{\dx}{x^{\prime}(t)}
\newcommand{\dy}{y^{\prime}(t)}

\newcommand{\br}[1]{\left(#1\right)}
\newcommand{\sbr}[1]{\left[#1\right]}
\newcommand{\cbr}[1]{\left\{#1\right\}}

\newcommand{\dprime}{\prime\prime}
\newcommand{\lap}[2]{\mathcal{L}[#1](#2)}

\newcommand{\divides}{\mid}

% bracket notation for inner product
\usepackage{mathtools}

\DeclarePairedDelimiterX{\abr}[1]{\langle}{\rangle}{#1}

\DeclareMathOperator{\Span}{span}
\DeclareMathOperator{\nullity}{nullity}
\DeclareMathOperator\Aut{Aut}
\DeclareMathOperator\Inn{Inn}

% set page count index to begin from 1
\setcounter{page}{1}

\begin{document}

\begin{enumerate}
    \item (DF1.1.31) Prove that any group of even order has an element of order $2$. \begin{proof}
      Let $G$ be a group of even order, say $2n$ ($n\in\mathbb{Z}^+$). We will show that there is at least one element of order two by showing that the number of elements which do not have order two is strictly less than $2n$.

      First, note that the identity element of $G$ has order $1$, so we now have to consider the remaining $2n-1$ elements. When the order of an element $g$ in $G$ is strictly greater than $2$, it means that $g^2\neq 1$, which is the same as saying $g \neq g^{-1}$. And in a similar way we have that $g^{-1} \neq (g^{-1})^{-1}$, so that the order of $g^{-1}$ is strictly greater than $2$ as well.

      We can count how many elements in $G$ have orders strictly greater than two by pulling them from $G$ in pairs and identifying two elements at a time in the set $G_{\abs{g}>2} = \cbr{g\in G\mid g\neq g^{-1}}$.

      We may scour through the remaining $2n-1$ elements and search for and pick out any element $a$ which has order strictly greater than $2$ (if such elements even exist; consider the Klein four-group). If we find an $a$, we can also pick out $a^{-1}$, so that they are paired in this way. Both elements have orders greater than $2$, so we find these elements in $G_{\abs{g}>2}$. Hence by construction $G_{\abs{g}>2}$ has even cardinality since we could only identify pairs of elements in the set.
        
      Furthermore, the number of elements in $G_{\abs{g}>2}$ must be strictly less than $2n-1$ since $2n-1$ is odd and if we are picking out pairs of elements we cannot pick all $2n-1$ elements to fit in this set, as there would be at least one element which cannot be paired up at all. Hence the size of the set $G_{\abs{g}>2}$ is less than or equal to $2n-2$. 

      Then the number of elements with order $2$ is at least $2n - 1 - (2n-2) = 1$, which means there is at least one element of order $2$ in a group with even order.\end{proof}

    \item (DF1.1.35) If $x$ is an element of finite order $n$ in a group $G$, use the Division Algorithm to show that every integral power of $x$ equals one of the elements in the set $\cbr{1,x,x^2,\dots,x^{n-1}}$. \begin{proof}
      Let $G$ be a group and let $x\in G$ have finite order $n$ as given. Then consider any integral power of $x$, say $x^k$ for any $k\in \mathbb{Z}$. From the Division Algorithm, we may write $k = nq + r$, where for every $k$ there exists unique $q,r\in\mathbb{Z}$ such that $0\leq r < n$. Hence $x^k = x^{nq+r} = x^{nq}x^r = (x^n)^qx^r = 1^qx^r$. In the case when $q$ is nonnegative, the last term is equal to $x^r$, and when $q$ is negative write $q = -p$ where $p\in\mathbb{Z}^{+}$ ($p$ is a positive integer). Then $1^q = 1^{-p} = (1^{-1})^p = 1^p = 1$, which implies that when $q$ is negative we still end with $x^r$.
    
      The integer $r$ by the Division Algorithm may only take on values from $0$ to $n-1$, so any integral power of $x$ will take on one of the elements in the set $\cbr{x^0,x^1,x^2,\dots,x^{n-1}}$, where of course $x^0 = 1$.\end{proof}
    
    \item (DF1.3.8) Show that if $\Omega = \cbr{1,2,3,\dots}$ then $S_{\Omega}$ is an infinite group (Do not say $\infty! = \infty$). \begin{proof}
      One way we can show this is to see that $\abs{S_{\Omega}} \geq \abs{\Omega} = n$ for each $n$. When $n$ equals $0$ or $1$ it is clear that $S_{\Omega}$ only contains the trivial map. For every other $n$, see that at minimum we have $n$ choices to send the first element in the set, and then $n-1$ choices for the next element, and so on. Thus the number of permutaions of these sets is bounded below by $n$.
    
      Hence in the case where $\Omega$ is a countably infinite set, we have a countably infinite number of choices for where we can send the first element to, and then again we have a countably infinite number of choices for the second element, and so on. (So for the cycle $\pmt*{1k}$ for each $k\in \cbr{1,2,3,\dots}$ there are a countably infinite number of these cycles) This means that $S_{\Omega}$ is also an infinite group whose cardinality is at least $\abs{\Omega}$.\end{proof}
    
    \item Prove that if $\tau\in S_n$, then for any $r$-cycle $(a_1a_2\cdots a_r)$, where $r\leq n$, we have \[\tau(a_1a_2\cdots a_r)\tau^{-1} = (\tau(a_1)\tau(a_2)\cdots\tau(a_r)).\] This formula is \textit{very useful}. \begin{proof} Let $\tau$ and $(a_1a_2\cdots a_r)$ be elements of $S_n$ as given, and for notational ease let $f\in S_n$ be given by $f = \tau(a_1a_2\cdots a_r)\tau^{-1}$. We can show that this permutation is equal to $(\tau(a_1)\tau(a_2)\cdots\tau(a_r))$ by showing that their actions on elements of $S_n$ agree for all $a_i\in S_n$.
    
    Without loss of generality, instead of computing the action of both permutations on each $a_i$, we may instead compute the action of each permutation on each $\tau(a_i)$ unambiguously, since $\tau$ is a bijection we would be considering the same set of elements $a_i$ but with different labelings (the same set is the image under $\tau$). The following computation is equivalent to showing that $f\tau = \tau(a_1a_2\cdots a_r) = (\tau(a_1)\tau(a_2)\cdots\tau(a_r))\tau$ agree in action for each element $a_i$.
  
    So consider the set of elements $\cbr{a_1,a_2,\dots,a_n}$ and its image under $\tau$, $\cbr{\tau(a_1), \tau(a_2), \dots, \tau(a_n)}$. We may compute the actions of $f$ and $(\tau(a_1)\tau(a_2)\cdots\tau(a_r))$ on each element $\tau(a_i)$ in this second set, and see that they are equal for all $i$.
  
    In the first case $i>r$ so that the cycle $(a_1a_2\cdots a_r)$ fixes $a_i$. Then see that \[\tau(a_1a_2\cdots a_r)\tau^{-1}(\tau(a_i)) = \tau((a_1a_2\cdots a_r)(\tau^{-1}(\tau(a_i)))) = \tau((a_1a_2\cdots a_r)(a_i)) = \tau(a_i),\] which means that overall the permutation $f$ fixes $\tau(a_i)$, which is in agreement with the action of $(\tau(a_1)\tau(a_2)\cdots\tau(a_r))$ on $\tau(a_i)$ since $i>r$.
  
    Then in the other case where $i\leq r$ so that the cycle $(a_1a_2\cdots a_r)$ does not fix $a_i$, see that \[\tau(a_1a_2\cdots a_r)\tau^{-1}(\tau(a_i)) = \tau((a_1a_2\cdots a_r)(\tau^{-1}(\tau(a_i)))) = \tau((a_1a_2\cdots a_r)(a_i)) = \tau(a_{i+1 \pmod r}).\] Again this is in agreement with how $(\tau(a_1)\tau(a_2)\cdots\tau(a_r))$ will map $\tau(a_i)$ to $\tau(a_{i+1 \pmod r})$ when $i\leq r$.
  
    So both permutations $f$ and $(\tau(a_1)\tau(a_2)\cdots\tau(a_r))$ agree in action for all elements $\tau(a_i)$ and so they must be equal.\end{proof}
\end{enumerate}

\end{document}