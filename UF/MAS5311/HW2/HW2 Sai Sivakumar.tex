\documentclass[11pt]{article}

% packages
\usepackage{physics}
% margin spacing
\usepackage[top=1in, bottom=1in, left=0.5in, right=0.5in]{geometry}
\usepackage{hanging}
\usepackage{amsfonts, amsmath, amssymb, amsthm}
\usepackage{systeme}
\usepackage[none]{hyphenat}
\usepackage{fancyhdr}
\usepackage[nottoc, notlot, notlof]{tocbibind}
\usepackage{graphicx}
\graphicspath{{./images/}}
\usepackage{float}
\usepackage{siunitx}
\usepackage{esint}
\usepackage{cancel}

% permutations (second line is for spacing)
\usepackage{permute}
\renewcommand*\pmtseparator{\,}

% colors
\usepackage{xcolor}
\definecolor{p}{HTML}{FFDDDD}
\definecolor{g}{HTML}{D9FFDF}
\definecolor{y}{HTML}{FFFFCF}
\definecolor{b}{HTML}{D9FFFF}
\definecolor{o}{HTML}{FADECB}
%\definecolor{}{HTML}{}

% \highlight[<color>]{<stuff>}
\newcommand{\highlight}[2][p]{\mathchoice%
  {\colorbox{#1}{$\displaystyle#2$}}%
  {\colorbox{#1}{$\textstyle#2$}}%
  {\colorbox{#1}{$\scriptstyle#2$}}%
  {\colorbox{#1}{$\scriptscriptstyle#2$}}}%

% header/footer formatting
\pagestyle{fancy}
\fancyhead{}
\fancyfoot{}
\fancyhead[L]{MAS5311}
\fancyhead[C]{Assignment 2}
\fancyhead[R]{Sai Sivakumar}
\fancyfoot[R]{\thepage}
\renewcommand{\headrulewidth}{1pt}

% paragraph indentation/spacing
\setlength{\parindent}{0cm}
\setlength{\parskip}{10pt}
\renewcommand{\baselinestretch}{1.25}

% extra commands defined here
\newcommand{\ihat}{\boldsymbol{\hat{\textbf{\i}}}}
\newcommand{\jhat}{\boldsymbol{\hat{\textbf{\j}}}}
\newcommand{\dr}{\vec{r}~^{\prime}(t)}
\newcommand{\dx}{x^{\prime}(t)}
\newcommand{\dy}{y^{\prime}(t)}

\newcommand{\br}[1]{\left(#1\right)}
\newcommand{\sbr}[1]{\left[#1\right]}
\newcommand{\cbr}[1]{\left\{#1\right\}}

\newcommand{\dprime}{\prime\prime}
\newcommand{\lap}[2]{\mathcal{L}[#1](#2)}

\newcommand{\divides}{\mid}

% bracket notation for inner product
\usepackage{mathtools}

\DeclarePairedDelimiterX{\abr}[1]{\langle}{\rangle}{#1}

\DeclareMathOperator{\Span}{span}
\DeclareMathOperator{\nullity}{nullity}
\DeclareMathOperator\Aut{Aut}
\DeclareMathOperator\Inn{Inn}
\DeclareMathOperator{\lcm}{lcm}

% set page count index to begin from 1
\setcounter{page}{1}

\begin{document}

\begin{enumerate}
    \item (DF1.6.14) Let $G$ and $H$ be groups and let $\varphi\colon G\to H$ be a homomorphism. Prove that the kernel of $\varphi$ is a subgroup of $G$ (1). Prove that $\varphi$ is injective if and only if the kernel of $\varphi$ is the identity subgroup of $G$ (2). \begin{proof}
      (1) To show that $\ker(\varphi)\leq G$, we can verify that $\ker(\varphi)$ is a nonempty subset of $G$ which is closed under the group operation (of $G$) and is closed under taking inverses.

      Observe that $\ker(\varphi)$ is nonempty because $1\in\ker(\varphi)$, as homomorphisms by definition map the identity of $G$ to the identity of $H$. (To see this, let $a\in G$, and see that $\varphi(a) = \varphi(1_Ga) = \varphi(1_G)\varphi(a)$, and by right cancellation, $1_H = \varphi(1_G)$.)

      Then let $a,b\in\ker(\varphi)$. Then $\varphi(ab) = \varphi(a)\varphi(b) = 1_H1_H = 1_H$. Hence $ab\in\ker(\varphi)$. Hence $\ker(\varphi)$ is closed under the group operation.

      Then also see that for $a\in\ker(\varphi)$, $1_H = \varphi(1_G) = \varphi(aa^{-1}) = \varphi(a)\varphi(a^{-1}) = 1_H\varphi(a^{-1}) = \varphi(a^{-1})$. So $a^{-1}$ is also mapped to the identity in $H$ so it must be in $\ker(\varphi)$. Hence $\ker(\varphi)$ is closed under taking inverses, and we must have that $\ker(\varphi)\leq G$.
    \end{proof} \begin{proof}
      (2) Forwards direction. Suppose that $\ker(\varphi) = \cbr{1_G}$. Then for $a,b\in G$, see that if $\varphi(a) = \varphi(b)$, then $1_H = \varphi(a)\varphi(b)^{-1} = \varphi(a)\varphi(b^{-1}) = \varphi(ab^{-1})$. Then since $\ker(\varphi) = \cbr{1_G}$, then $e = ab^{-1}$ implies $b = a$. (To see that $\varphi$ maps inverses to inverses, take $c\in G$ and see that $1_H = \varphi(1_G)) = \varphi(cc^{-1}) = \varphi(c)\varphi(c^{-1})$, and so $\varphi(c)^{-1} = \varphi(c^{-1})$.) Hence $\varphi$ is injective.

      Converse. Suppose that $\varphi$ is injective. Then suppose by contradiction that there exists $a\in G$ where $a \neq 1_G$ such that $\varphi(a) = 1_H$. Then $\varphi(a) = \varphi(1_G) = 1_H$, but $a \neq 1_G$, which is in contradiction to $\varphi$ being injective. Hence $\ker(\varphi)$ only contains the identity of $G$. (Automatically if we have injectivity, we should have that only the identity of $G$ is sent to the identity of $H$.)
    \end{proof}
    
    \item (DF1.6.19) Let $G = \cbr{z\in \mathbb{C}\mid z^n = 1 \text{ for some }n\in\mathbb{Z}^{+}}$. Prove that for any fixed integer $k>1$ the map from $G$ to itself defined by $z\mapsto z^k$ is a surjective homomorphism but is not an isomorphism. \begin{proof}
      We should first check that $G$ is a group under the standard multiplication in $\mathbb{C}$, which we know is already associative (and $G\subseteq \mathbb{C}$). It suffices to show that $G$ is closed under the operation, contains an identity element, and has inverses for every element.

      The complex number $1 = 1+0i$ is the identity of $G$, as $1$ raised to any positive integer will still be $1$, and from $\mathbb{C}$ we know that $1z = z1 = z$ for any $z\in\mathbb{C}$ and hence is also true for any $z\in G$. Hence $G$ is nonempty.

      Let $z,w\in G$. Then there exists $n,m\in\mathbb{Z}^{+}$ such that $z^n = w^m = 1$. Then observe that $\lcm(n,m)\in\mathbb{Z}^{+}$, and hence $(zw)^{\lcm(n,m)} = z^{\lcm(n,m)}w^{\lcm(n,m)} = z^{nk}w^{mj} = (z^n)^k(w^m)^j = 1^k1^j = 1$, where since $n$ and $m$ divide $\lcm(n,m)$, there exist integers $k,j\in\mathbb{Z}^{+}$ where $\lcm(n,m)=nk=mj$. Hence $zw\in G$, so $G$ is closed under the group operation.

      To find an inverse for some element $w$ in $G$, observe that there exists a $j\in\mathbb{Z}^{+}$ such that $w^j=1$. If $w = 1$, then automatically $w^{-1} = 1$. If $w$ is not the identity, see that $w\cdot w^{j-1} = w^{j-1}\cdot w = 1$, where $j-1\in\mathbb{Z}^{+}$. (Only $1$ satisfies $z^1 = 1$ in $\mathbb{C}$.) Furthermore see that $(w^{j-1})^j = w^{j\cdot(j-1)} = (w^j)^{j-1} = 1^{j-1} = 1$. Then in this case $w^{-1} = w^{j-1}$, where $j$ is a positive integer such that $w^j=1$. Hence $G$ is a group.

      For any fixed integer $k>1$, let $\varphi_k\colon G\to G$ be given by $\varphi_k(z) = z^k$. Then we must show that $\varphi_k$ is surjective but not injective (but still preserves the group operation). Clearly for $z,w \in G$, $\varphi_k(zw) = (zw)^k = z^kw^k = \varphi_k(z)\varphi_k(w)$. Then for $z\in G$, observe that we may write $z=e^{2\pi i\frac{p}{q}}$, for some $p,q\in\mathbb{Z}^{+}$ ($p,q$ are not unique for $z$), because there exists $n\in\mathbb{Z}^{+}$ such that $z^n = e^{2\pi ij} = 1$, where $j$ is any positive integer ($n$ will be any positive multiple of $q$). So then let $w = e^{2\pi i\frac{p}{qk}}$, so that there still exists a positive integer $m$ such that $w^m = 1$ (here $m$ will be a positive multiple of $qk$), and so $w\in G$. Observe that $\varphi_k(w) = w^k = \left(e^{2\pi i\frac{p}{qk}}\right)^k = e^{2\pi i\frac{p}{q}} = z$. Since $z$ was an arbitrary element of $G$ and we constructed another element $w$ in $G$ from $z$, we have that $\varphi_k$ is surjective (for every $z\in G$ we can find $w\in G$ such that $\varphi_k(w) = z$).

      We can show that $\varphi_k$ is not injective by exhibiting an element (for each $k$) which is not $1$ which maps to $1$ under $\varphi_k$. So we try $z = e^{2\pi i\frac{1}{k}}$, and see that because $k>1$, $z\neq 1$. $z\in G$ because raising $z$ to some positive multiple of $k$, a positive integer, produces $1$. Then see that $\varphi_k(z) = z^k = \left(e^{2\pi i\frac{1}{k}}\right)^k = e^{2\pi i} = 1 = 1^k = \varphi_k(1)$. Hence $\varphi_k$ is not injective.
    \end{proof}
    
    \item (DF1.6.20) Let $G$ be a group and let $\Aut(G)$ be the set of all isomorphisms from $G$ onto $G$. Prove that $\Aut(G)$ is a group under function composition (called the \textit{automorphism group} of $G$ and the elements of $\Aut(G)$ are called \textit{automorphisms} of $G$). \begin{proof}
      Let $\Aut(G)$ be the set of isomorphisms from $G$ onto $G$ as given. Then because function composition is associative, we already have associativity in $\Aut(G)$.

      Naturally the identity map will be in $\Aut(G)$ since it fixes every element and will preserve the group operation, so $\Aut(G)$ is nonempty.
      
      To show that the set is closed under function composition, recall that a composition of two bijections is a bijection also. What remains to show is that the group operation is still preserved. Let $f,g\in \Aut(G)$. Then for $a,b\in G$, $(fg)(ab) = f(g(ab)) = f(g(a)g(b)) = f(g(a))f(g(b)) = (fg)(a)(fg)(b)$, so composition also preserves the group operation and so compositions of automorphisms are also automorphisms.
      
      Then to show that the set contains inverses, see that if $f$ is an automorphism, then the inverse mapping $f^{-1}$ is bijective. We need to show that it preserves the group operation to be an automorphism. Let $a,b\in G$, and from $f^{-1}(ab)$ note that because $f$ is a bijection, there exist $c,d\in G$ such that $f(c) = a$ and $f(d) = b$. So $f^{-1}(ab) = f^{-1}(f(c)f(d)) = f^{-1}(f(cd)) = cd = f^{-1}(a)f^{-1}(b)$. Hence $f^{-1}$ is also an automorphism, and so $\Aut(G)$ is a group under function composition.
    \end{proof}
    
    \item (DF1.6.23) Let $G$ be a finite group which possesses an automorphism $\sigma$ such that $\sigma(g) = g$ if and only if $g=1$. If $\sigma^2$ is the identity map from $G$ to $G$, prove that $G$ is abelian (such an automorphism $\sigma$ is called \textit{fixed point free} of order $2$). [Show that every element of $G$ can be written in the form $x^{-1}\sigma(x)$ and apply $\sigma$ to such an expression.] \begin{proof}
      Let $G$ be a finite group as given, possessing a fixed point free automorphism $\sigma$ of order $2$.

      First we show that every element $g\in G$ can be written in the form $x^{-1}\sigma(x)$, so that $g = x^{-1}\sigma(x)$ for some unique $x\in G$. Observe that for $y,z \in G$, then if $y^{-1}\sigma(y) = z^{-1}\sigma(z)$, then we can multiply on the left and right judiciously to find that $\sigma(y)\sigma(z)^{-1} = \sigma(y)\sigma(z^{-1}) = \sigma(yz^{-1}) = yz^{-1}$. But the \textit{only} element which is fixed in $G$ under $\sigma$ is $1$, so $yz^{-1} = 1 \implies y=z$ (so $y\neq z\implies y^{-1}\sigma(y) \neq z^{-1}\sigma(z)$). With this, if we consider the set given by $\cbr{x^{-1}\sigma(x)\mid x\in G}$, then it is clear that this set has the same cardinality as $G$, and since $\sigma$ is an automorphism each element in the set is an element of $G$. Because each $x^{-1}\sigma(x)$ is a distinct element of $G$, each $x^{-1}\sigma(x)$ can be equated to a unique $g\in G$.

      Then for some $g,x \in G$ where $g = x^{-1}\sigma(x)$, see that $\sigma(g) = \sigma(x^{-1}\sigma(x)) = \sigma(x^{-1})\sigma(\sigma(x)) = \sigma(x)^{-1}\sigma^2(x) = \sigma(x)^{-1}x$. But then also see that $g^{-1} = (x^{-1}\sigma(x))^{-1} = \sigma(x)^{-1}x = \sigma(g)$, so that really the action of $\sigma$ on any element $g\in G$ is to map $g$ to $g^{-1}$.

      Then it suffices to show that if $\sigma$ is an automorphism where for any $g\in G$, $\sigma(g) = g^{-1}$, $G$ is abelian. Let $a,b\in G$ and see that $ab = \sigma(a^{-1})\sigma(b^{-1}) = \sigma(a^{-1}b^{-1}) = (a^{-1}b^{-1})^{-1} = ba$. So since $a,b$ can be any element in $G$ we have that $G$ is abelian.
    \end{proof}
    
\end{enumerate}
\end{document}