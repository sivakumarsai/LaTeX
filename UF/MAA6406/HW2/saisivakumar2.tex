\documentclass[12pt]{amsart}

\textwidth = 6.2 in
\textheight = 8.5 in
\oddsidemargin = 0.0 in
\evensidemargin = 0.0 in
\topmargin = 0.0 in
\headheight = 0.0 in
\headsep = 0.3 in
\parskip = 0.05 in
\parindent = 0.3 in

\usepackage{enumerate}
\usepackage{amsmath}
\usepackage{color}
\def\cc{\color{blue}}
\usepackage[normalem]{ulem}
\usepackage{amsfonts, amsmath, amssymb, amsthm}
\usepackage{systeme}
\usepackage[none]{hyphenat}
\usepackage{graphicx}
\graphicspath{{./images/}}
\usepackage{esint}
\usepackage{cancel}
\usepackage{physics}

\title{Homework 2}
\author{Sai Sivakumar}

\newtheorem{theorem}            {Theorem}[section]
\newtheorem{proposition}        [theorem]{Proposition}

\newcommand{\RR}{\mathbb{R}}
\newcommand{\NN}{\mathbb{N}}
\newcommand{\QQ}{\mathbb{Q}}
\newcommand{\CC}{\mathbb{C}}
\newcommand{\DD}{\mathbb{D}}

\begin{document}
\maketitle

\thispagestyle{empty}
 Let $\DD$ denote the open unit disk in the complex plane:
\[
 \DD=\{z\in\CC: |z|<1\} \subseteq \CC.
\]
 Show if  $f:\DD\to\CC$  is analytic and real-valued,
 then $f$ is constant. Is this conclusion true 
 if $\DD$ is replaced by the set $\{z\in \CC: |\Re(z)|>1\}?$
 Proof or counterexample.

\bigskip

\begin{proof}
\baselineskip=24pt
Since $f\colon\mathbb{D}\to\mathbb{C}$ is analytic, write $f = u+iv$ for $u = \Re(f) ,v = \Im(f) \colon\mathbb{C}\to \mathbb{R}$ satisfying the Cauchy-Riemann equations. With $f$ real valued, it follows $v$ is identically zero so that $u^\prime_x = v^\prime_y = 0$ and $u^\prime_y = -v^\prime_x = 0$, from which it follows $f^\prime\colon \mathbb{D}\to \mathbb{C}$ is zero. Since $\mathbb{D}$ is open and connected, by Proposition 2.10 in Conway it follows that $f$ is constant.
\end{proof}

This conclusion is not true if $\mathbb{D}$ is replaced by $P = \{z\in \CC: |\Re(z)|>1\}$, since it is open but not connected: Let $g\colon P\to C$ be specified by analytic real valued functions in the natural way, by $g_r\colon P_r = \{z\in \CC: |\Re(z)>1\}\to \mathbb{C}$ and $g_l\colon P_l = \{z\in \CC: |\Re(z)<1\}\to \mathbb{C}$. An argument similar to the above show that $g_r,g_l$ are constant on their domains. But the constants need not be the same so that $g$ need not be constant (take $g$ to be the function which is $1$ on the left half plane $P_l$ and $0$ on the right half plane $P_l$).
\end{document}