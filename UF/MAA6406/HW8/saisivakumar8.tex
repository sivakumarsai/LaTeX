\documentclass[12pt]{amsart}

\textwidth = 6.2 in
\textheight = 8.5 in
\oddsidemargin = 0.0 in
\evensidemargin = 0.0 in
\topmargin = 0.0 in
\headheight = 0.0 in
\headsep = 0.3 in
\parskip = 0.05 in
\parindent = 0.3 in

\usepackage{enumerate}
\usepackage{amsmath}
\usepackage{color}
\def\cc{\color{blue}}
\usepackage[normalem]{ulem}
\usepackage{amsfonts, amsmath, amssymb, amsthm}
\usepackage{systeme}
\usepackage[none]{hyphenat}
\usepackage{graphicx}
\graphicspath{{./images/}}
\usepackage{esint}
\usepackage{cancel}
\usepackage{physics}

\title{Homework 8}
\author{Sai Sivakumar}

\newtheorem{theorem}            {Theorem}[section]
\newtheorem{proposition}        [theorem]{Proposition}

\newcommand{\RR}{\mathbb{R}}
\newcommand{\NN}{\mathbb{N}}
\newcommand{\QQ}{\mathbb{Q}}
\newcommand{\CC}{\mathbb{C}}
\newcommand{\DD}{\mathbb{D}}
\newcommand{\cS}{\mathcal{S}}

\begin{document}
\maketitle

\thispagestyle{empty}
 Let $M_n(\CC)$ denote the $n\times n$ matrices with entries
 from $\CC.$ Let  $\|A\|$ denote the operator norm of 
 $A\in M_n(\CC).$ Since all norms on $\CC^{n^2}$ are 
 equivalent, $M_n(\CC)$ with the operator norm is 
 complete. 

 An routine geometric series  argument using $\|A^n\|\le \|A\|^n$ 
 shows, for $R>\|A\|,$  that the series
\[
 \sum_{n=0}^\infty  \frac{A^n}{R^n} e^{-ins}
\]
 converges absolutely and uniformly as a function of $s\in \RR$  to 
\[
  (I- \frac{A}{Re^{is}})^{-1}.
\]
 Use this fact to show, for $R>\|A\|$ 
 and $k\in \NN,$ that 
\[ 
A^k= \frac{1}{2\pi i} \int_{|z|=R} z^k (z-A)^{-1} \, dz,
\]
where $|z|=R$ is the curve $\gamma:[0,2\pi]\to\CC$ 
defined by $\gamma(s)=Re^{is}.$ 
The integral can be interpreted {\it in the weak sense} -- 
  for $x,y\in \CC^n,$ 
\[
  \langle A^k x,y\rangle = \int_{|z|=R} z^k \, \langle (z-A)^{-1}x,y\rangle \, dz
\]
-- if you like. 
 
 Show, given a polynomial $p=\sum_{j=0}^d p_jz^j,$
\[
 p(A) =\frac{1}{2\pi i} \int_{|z|=R} p(z)\, (z-A)^{-1} \, dz.
\]
(This formula is then a version of Cauchy's integral formula.)


Now use Cramer's rule  to prove the Cayley-Hamilton Theorem:

For  $q(z)=\det(z-A),$ 
\[
  q(A)=0.
\]


\bigskip

\begin{proof}
\baselineskip=24pt
Using the absolute and uniform convergence of the series above, we have for $R>\norm{A}$ that \begin{align*}
  \frac{1}{2\pi i}\int_{\abs{z}=R}z^k (z-A)^{-1}\dd{z} &= \frac{1}{2\pi i}\int_{\abs{z}=R}z^{k-1} (I-A/z)^{-1}\dd{z} \\
  &= \frac{1}{2\pi i}\int_{\abs{z}=R}z^{k-1} \sum_{n=0}^\infty (A/z)^n \dd{z} \\
  &= \frac{1}{2\pi i}\sum_{n=0}^\infty A^n\int_{\abs{z}=R}z^{k-1-n}\dd{z}\\
  &= \frac{A^k}{2\pi i}\int_{\abs{z}=R}z^{-1}\dd{z} = A^k,
\end{align*} where uniform convergence was used to interchange the sum and integral signs, and Cauchy's integral theorem was used to extract only the $n=k$ term.

Then for $p(z)=\sum_{j=0}^d p_jz^j$ a polynomial we have \begin{align*}
  \frac{1}{2\pi i} \int_{\abs{z}=R} p(z)(z-A)^{-1} \dd{z} &= \frac{1}{2\pi i} \int_{\abs{z}=R} \sum_{j=0}^d p_jz^j(z-A)^{-1} \dd{z}\\
  &= \sum_{j=0}^d p_j\int_{\abs{z} = R} z^j(z-A)^{-1} \dd{z}\\
  &= \sum_{j=0}^d p_jA^j = p(A)
\end{align*} by the previous result.

By Cramer's rule we have that $\det(z-A)(z-A)^{-1} = \mathrm{adj}(z-A)$, where $\mathrm{adj}(z-A)$ is the adjugate matrix of $(z-A)$. The entries of $\mathrm{adj}(z-A)$ are polynomials in $z$. So for $q(z) = \det(z-A)$, we have \[q(A) = \frac{1}{2\pi i} \int_{\abs{z}=R} q(z)(z-A)^{-1} \dd{z} = \frac{1}{2\pi i} \int_{\abs{z}=R} \mathrm{adj}(z-A) \dd{z} = 0\] since every entry of $\mathrm{adj}(z-A)$ is an analytic function. 
\end{proof}
\end{document}