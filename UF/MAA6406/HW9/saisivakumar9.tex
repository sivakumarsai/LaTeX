\documentclass[12pt]{amsart}

\textwidth = 6.2 in
\textheight = 8.5 in
\oddsidemargin = 0.0 in
\evensidemargin = 0.0 in
\topmargin = 0.0 in
\headheight = 0.0 in
\headsep = 0.3 in
\parskip = 0.05 in
\parindent = 0.3 in

\usepackage{enumerate}
\usepackage{amsmath}
\usepackage{color}
\def\cc{\color{blue}}
\usepackage[normalem]{ulem}
\usepackage{amsfonts, amsmath, amssymb, amsthm}
\usepackage{systeme}
\usepackage[none]{hyphenat}
\usepackage{graphicx}
\graphicspath{{./images/}}
\usepackage{esint}
\usepackage{cancel}
\usepackage{physics}

\title{Homework 8}
\author{Sai Sivakumar}

\newtheorem{theorem}            {Theorem}[section]
\newtheorem{proposition}        [theorem]{Proposition}

\newcommand{\RR}{\mathbb{R}}
\newcommand{\NN}{\mathbb{N}}
\newcommand{\QQ}{\mathbb{Q}}
\newcommand{\CC}{\mathbb{C}}
\newcommand{\DD}{\mathbb{D}}
\newcommand{\cS}{\mathcal{S}}

\begin{document}
\maketitle

\thispagestyle{empty}
 Do one.
\begin{enumerate}[(i)]\itemsep=15pt
\item  Show, if $p$ is a polynomial of degree at least two, then
 the sum of the residues of $\frac{1}{p}$ is zero.
\item  Suppose $a>e$ and $n\in\NN^+.$ Show $e^z-az^n=0$ has exactly 
 $n$ solutions in the unit disc $\DD=\{|z|<1\}.$
\end{enumerate}
 
\bigskip

\begin{proof}[Proof (ii)]
\baselineskip=24pt
Apply Rouch\'e's theorem.

The functions $f,g$ given by $f(z) = \exp(z)-az^n$ and $g(z) = az^n$ are meromorphic on $\mathbb{D}$. Furthermore, $f$ has no poles in $\mathbb{D}$, and $g$ has $n$ zeroes but no poles in $\mathbb{D}$.

We have for $z\in S^1$, $\abs{f(z)+g(z)} = \abs{\exp(z)} = \exp(\Re(z))\leq e$, and $a = \abs{az^n} \leq \abs{f(z)}+\abs{g(z)}$. Since $a>e$, $\abs{f(z)+g(z)}<\abs{f(z)}+\abs{g(z)}$ for all $z\in S^1$. By Rouch\'e's theorem the number of zeroes of $f(z)$ is $n$ as desired.
\end{proof}
\end{document}