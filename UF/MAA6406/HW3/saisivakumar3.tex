\documentclass[12pt]{amsart}

\textwidth = 6.2 in
\textheight = 8.5 in
\oddsidemargin = 0.0 in
\evensidemargin = 0.0 in
\topmargin = 0.0 in
\headheight = 0.0 in
\headsep = 0.3 in
\parskip = 0.05 in
\parindent = 0.3 in

\usepackage{enumerate}
\usepackage{amsmath}
\usepackage{color}
\def\cc{\color{blue}}
\usepackage[normalem]{ulem}
\usepackage{amsfonts, amsmath, amssymb, amsthm}
\usepackage{systeme}
\usepackage[none]{hyphenat}
\usepackage{graphicx}
\graphicspath{{./images/}}
\usepackage{esint}
\usepackage{cancel}
\usepackage{physics}

\title{Homework 3}
\author{Sai Sivakumar}

\newtheorem{theorem}            {Theorem}[section]
\newtheorem{proposition}        [theorem]{Proposition}

\newcommand{\RR}{\mathbb{R}}
\newcommand{\NN}{\mathbb{N}}
\newcommand{\QQ}{\mathbb{Q}}
\newcommand{\CC}{\mathbb{C}}
\newcommand{\DD}{\mathbb{D}}

\begin{document}
\maketitle

\thispagestyle{empty}
 
 Explain why  the power series $\sum_{k=1}^\infty \frac{z^k}{k}$ 
 determines an analytic function $f$ with domain $\DD=\{|z|<1\}.$



 Prove, for $z\in \DD,$ that $f(z)=-\log(1-z).$
\bigskip

\begin{proof}
\baselineskip=24pt
Via the root or ratio tests one obtains that $\abs{z}< 1$ (or $z\in N_1(0) = \mathbb{D}$) in order for the power series, viewed as a function of $z$, to converge (e.g., from the ratio test we have that the power series converges if $\lim_{k\to\infty}\abs{\frac{zk}{k+1}} = \abs{z} < 1$).

Apply Theorem 2.21 from the notes: the power series $\sum_{k=1}^\infty \frac{z^k}{k}$ is an analytic function $f\colon\mathbb{D}\to\mathbb{C}$ with $f(z) = \sum_{k=1}^\infty \frac{z^k}{k}$. Moreover, the derivative of $f$ at $z\in\mathbb{D}$ is $f^\prime(z) = \sum_{k=0}^\infty z^k = \frac{1}{1-z} = \dv{z}[-\log(1-z)]$. It follows that $f(z)$ and $-\log(1-z)$ differ by a constant $C\in \mathbb{C}$. But $0 = \sum_{k=1}^\infty \frac{0^k}{k}+\log(1-0) = C$ so that $f(z) = -\log(1-z)$ for all $z\in\mathbb{D}$.
\end{proof}
\end{document}