\documentclass[12pt]{amsart}

\textwidth = 6.2 in
\textheight = 8.5 in
\oddsidemargin = 0.0 in
\evensidemargin = 0.0 in
\topmargin = 0.0 in
\headheight = 0.0 in
\headsep = 0.3 in
\parskip = 0.05 in
\parindent = 0.3 in

\usepackage{enumerate}
\usepackage{amsmath}
\usepackage{color}
\def\cc{\color{blue}}
\usepackage[normalem]{ulem}
\usepackage{amsfonts, amsmath, amssymb, amsthm}
\usepackage{systeme}
\usepackage[none]{hyphenat}
\usepackage{graphicx}
\graphicspath{{./images/}}
\usepackage{esint}
\usepackage{cancel}
\usepackage{physics}

\title{Homework 5}
\author{Sai Sivakumar}

\newtheorem{theorem}            {Theorem}[section]
\newtheorem{proposition}        [theorem]{Proposition}

\newcommand{\RR}{\mathbb{R}}
\newcommand{\NN}{\mathbb{N}}
\newcommand{\QQ}{\mathbb{Q}}
\newcommand{\CC}{\mathbb{C}}
\newcommand{\DD}{\mathbb{D}}
\newcommand{\cS}{\mathcal{S}}

\begin{document}
\maketitle

\thispagestyle{empty}

This assignment has two problems.
\begin{enumerate}[(a)] \itemsep=10pt
\item Show, if $U:\CC\to\RR$ is harmonic and
 bounded below, then $U$ is constant.
[Suggestion: Observe there is an entire
 function $f$ whose real part is $U$
 and consider the function $\exp(-f).$]
\item Show if $f:\CC\to\CC$ is entire
 and there is an $M\ge 0$ and $N\in \NN$ such that
\[
 |f(z)| \le M |z|^N
\]
 for all $z,$ then $f$ is a polynomial.
\end{enumerate}



\bigskip

\begin{proof}
\baselineskip=24pt
\begin{enumerate}[(a)]
    \item Let $M$ be the real constant such that $U(z)\geq M$ for all $z\in\mathbb{C}$. 
    
    Since $\mathbb{C}$ is simply connected we can find a harmonic complement $V\colon \mathbb{C}\to \mathbb{R}$ such that $f\colon\mathbb{C}\to\mathbb{C}$ given by $f(z) = U(z) + iV(z)$ is analytic (so $U$ is the real part of an analytic function). Then since $-f$, $\exp$ are entire, so is $\exp(-f)$ and \begin{align*}
        \abs{\exp(-f(z))} = \abs{\exp(-U(z)-iV(z))} &= \abs{\exp(-U(z))}\abs{\exp(-iV(z))}\\
        &= \abs{\exp(-U(z))}\\
        &\leq \abs{\exp(-M)}
    \end{align*} for all $z\in\mathbb{C}$. Thus the entire function $\exp(-f)$ is bounded above so it must be a constant function by Liouville's theorem. So for all $z\in\mathbb{C}$, $\exp(-f(z)) = c$ for some $c\in\mathbb{C}$, from which it follows that $f(z)$ must be a constant; as a result $U = \Re(f)$ is constant.

    \item For any $R>0$, since $f$ is entire, $f$ is analytic on $N_R(0)$. Furthermore, $\abs{f}$ will be uniformly bounded above by $MR^N$ on $N_R(0)$. Then by a Cauchy estimate we have for each $n\geq 0$, $\abs{f^{(n)}(0)}\leq n!MR^N/R^n$. Since $R$ was arbitrary, for $n> N$ we may take $R$ arbitrarily large so that $\abs{f^{(n)}(0)}$ vanishes, and for $n<N$ we may take $R$ arbitrarily small so that $\abs{f^{(n)}(0)}$ vanishes also. Hence  for $n\neq N$, $f^{(n)}(0) = 0$, and only $\abs{f^{(N)}(0)}$ could be nonzero. 
    
    Since $f$ is entire we may expand $f$ as a power series about the origin. Then with the above we have for any $z\in\mathbb{C}$, $f(z) = \sum_{n=0}^\infty \frac{f^{(n)}(0)}{n!}z^n = \frac{f^{(N)}(0)}{N!}z^N$. So $f$ is a monomial.
\end{enumerate}
\end{proof}
\end{document}