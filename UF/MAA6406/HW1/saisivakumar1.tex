\documentclass[12pt]{amsart}

\textwidth = 6.2 in
\textheight = 8.5 in
\oddsidemargin = 0.0 in
\evensidemargin = 0.0 in
\topmargin = 0.0 in
\headheight = 0.0 in
\headsep = 0.3 in
\parskip = 0.05 in
\parindent = 0.3 in

\usepackage{enumerate}
\usepackage{amsmath}
\usepackage{color}
\def\cc{\color{blue}}
\usepackage[normalem]{ulem}
\usepackage{amsfonts, amsmath, amssymb, amsthm}
\usepackage{systeme}
\usepackage[none]{hyphenat}
\usepackage{graphicx}
\graphicspath{{./images/}}
\usepackage{esint}
\usepackage{cancel}
\usepackage{physics}

\title{Homework 1}
\author{Sai Sivakumar}

\newtheorem{theorem}            {Theorem}[section]
\newtheorem{proposition}        [theorem]{Proposition}

\newcommand{\RR}{\mathbb{R}}
\newcommand{\NN}{\mathbb{N}}
\newcommand{\QQ}{\mathbb{Q}}
\newcommand{\CC}{\mathbb{C}}

\begin{document}
\maketitle

\thispagestyle{empty}

Show, if $X$ is a metric space and $\emptyset \ne F \subseteq X$ is connected,
 then, for each $\varepsilon>0$ and each $a,b\in F$ there is an positive
 integer $n$ and $z_0,\dots,z_n\in F$ such that $a=z_0,$ and $z_n=b$ and
 $d(z_{j-1},z_j)<\varepsilon$ for $1\le j \le n.$ 

 Give an example of a closed subset $F\subseteq \CC$ that is not connected,
 but nevertheless satisfies the conclusion above.

\bigskip
\baselineskip=24pt
\begin{proof}
Let $\emptyset \neq F$ be a connected subspace of the metric space $X$ as above. Let $\varepsilon > 0$ and $a,b\in F$ be given. Define \begin{multline*}
    A = \{p\in F\mid \text{there exists $n\in \mathbb{Z}_+$, $z_0 = p, z_1,\dots,z_n = b\in F$}\\ \text{such that $d(z_{j-1},z_j<\varepsilon$) for $1\leq j\leq n$}\}\subseteq F,
\end{multline*} and observe that $A$ is not empty since $b\in A$ (take $n = 1$, $z_0 = z_1 = b$ above). Since $F$ is connected, it suffices to show that $A$ is both open and closed since the only nonempty open and closed subset of $F$ is $F$ itself.

To that end, if $q\in A$, then $N^F_\varepsilon(q) = F\cap N_\varepsilon(q)\subseteq A$: There exists $n\in \mathbb{Z}_+$ and a sequence of points $z_0 = q, z_1,\dots,z_n = b\in F$ such that $d(z_{j-1},z)<\varepsilon$ for $1\leq j \leq n$. We may prepend this sequence with any point $q^\prime\in N^F_\varepsilon(q)$ to obtain a new sequence of points $z_{-1} = q^\prime, z_0 = q, z_1,\dots,z_n = b\in F$ such that $d(z_{j-1},z)<\varepsilon$ for $0\leq j \leq n$. Hence $q^\prime\in A$; since $q^\prime$ was arbitrary, $N^F_\varepsilon(q)$ is contained in $A$ as desired. Since $q$ was arbitrary, $A$ is open.

To show that $A$ is closed we show that it contains all of its limit points. Let $q$ be a limit point of $A$ so that there exists a sequence $(q_n)_{n=1}^\infty$ of points in $A\subseteq F$ converging to $q$; note that $q$ is in $F$ since $F$ is closed in itself. There exists an $N\in \mathbb{Z}_+$ such that if $n\geq N$, then $d(q_n,q) < \varepsilon$; in particular $d(q_N,q) < \varepsilon$. Since $q_N\in A$ there exists $n\in \mathbb{Z}_+$ and a sequence of points $z_0 = q_N, z_1,\dots,z_n = b\in F$ such that $d(z_{j-1},z)<\varepsilon$ for $1\leq j \leq n$. We may prepend this sequence with $q$ to obtain a new sequence of points $z_{-1} = q, z_0 = q_N, z_1,\dots,z_n = b\in F$ such that $d(z_{j-1},z)<\varepsilon$ for $0\leq j \leq n$. Hence $q\in A$; since $q$ was an arbitrary limit point of $A$, it follows that $A$ is closed.

It follows that $A = F$. In particular, $a\in A$ so that for each $\varepsilon>0$, there exists $n\in \mathbb{Z}_+$ and a sequence of points $z_0 = a, z_1,\dots,z_n = b\in F$ such that $d(z_{j-1},z)<\varepsilon$ for $1\leq j \leq n$.
\end{proof}

Consider the set $\{t+0i\mid t\in\mathbb{R}\}\cup\{t+ie^{t}\mid t\in\mathbb{R}\}$ (the graph of the exponential function and the $x$-axis). This set is a closed subset of $\mathbb{C}$ as each of the components $\{t+0i\mid t\in\mathbb{R}\}$ and $\{t+ie^{t}\mid t\in\mathbb{R}\}$ contain their limit points. It is also disconnected since each of the above (nonempty) components are open and closed (the complement of one component is the other, and vice versa). 

Fix any $\varepsilon>0$ and two points $a,b$ which lie in different components of the above set. [If $a,b$ lie in the same component then it is easy enough to find a finite sequence of points starting from $a$ and ending with $b$ such that the distance between consecutive points is less than $\varepsilon$ since the arclength of the shortest path connecting $a$ and $b$ is finite.] Then choose $r$ real so that $d((r+i0),(r+ie^r)) = e^r < \varepsilon$ and so we can form a finite sequence of points starting from $a$, then to $r+i0$ where the distance between consecutive points in the sequence is less than $\varepsilon$ (since they lie in the same component), and prepend this to the finite sequence of points starting from $r+ie^r$ and ending at $b$ satisfying the same property above to form a desired sequence of points starting from $a$ to $b$ with the above property. 
\end{document}