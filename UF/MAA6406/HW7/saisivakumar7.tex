\documentclass[12pt]{amsart}

\textwidth = 6.2 in
\textheight = 8.5 in
\oddsidemargin = 0.0 in
\evensidemargin = 0.0 in
\topmargin = 0.0 in
\headheight = 0.0 in
\headsep = 0.3 in
\parskip = 0.05 in
\parindent = 0.3 in

\usepackage{enumerate}
\usepackage{amsmath}
\usepackage{color}
\def\cc{\color{blue}}
\usepackage[normalem]{ulem}
\usepackage{amsfonts, amsmath, amssymb, amsthm}
\usepackage{systeme}
\usepackage[none]{hyphenat}
\usepackage{graphicx}
\graphicspath{{./images/}}
\usepackage{esint}
\usepackage{cancel}
\usepackage{physics}

\title{Homework 7}
\author{Sai Sivakumar}

\newtheorem{theorem}            {Theorem}[section]
\newtheorem{proposition}        [theorem]{Proposition}

\newcommand{\RR}{\mathbb{R}}
\newcommand{\NN}{\mathbb{N}}
\newcommand{\QQ}{\mathbb{Q}}
\newcommand{\CC}{\mathbb{C}}
\newcommand{\DD}{\mathbb{D}}
\newcommand{\cS}{\mathcal{S}}

\begin{document}
\maketitle

\thispagestyle{empty}
 Let $\Omega_\pm$ denote the upper and lower half planes
 and identify $\RR$ with the real axis in $\CC$ and suppose 
  $f:\overline{\Omega_+}\to \CC.$ Show, if
\begin{enumerate}[(i)]
 \item $f$  is continuous;
 \item the restriction of $f$ to $\Omega_+$ is analytic; and
 \item $f(\RR)\subseteq\RR,$
\end{enumerate} 
  then $g:\overline{\Omega_-}\to \CC$  defined by 
  $g(z)=f(z^*)^*$ is continuous, analytic on $\Omega_-$ and
 agrees with $f$ on $\RR.$ Finally, show the function
 $F:\CC\to \CC$ defined by
\[
 F(z) =\begin{cases} f(z) & z\in \overline{\Omega_+}\\
                     g(z) & z\in \overline{\Omega_-} 
 \end{cases}
\]
 is an analytic extension of $f$ to an entire function.



\bigskip

\begin{proof}
\baselineskip=24pt
Complex conjugation is a continuous operation. Given $\varepsilon>0$, if $\abs{z-w}<\varepsilon$ then $\abs{z^\ast-w^\ast}<\varepsilon$. Since the composition of continuous functions is continuous, $g= \cdot^\ast\circ f\circ \cdot^\ast$ is continuous. 

We use Morera's theorem to show $g$ is analytic on $\Omega_-$. Given a triangle $T\subset \Omega_-$, we can find a triangle $T^\ast\subset \Omega_+$ which is just the pointwise image of the conjugation map on $T$. The orientation of the triangle will be reversed under conjugation. It follows that $\int_T g = -\int_{T^\ast} f^\ast = (\int_{T^\ast}f)^\ast = 0$ since $f$ is analytic on $\Omega_+$. Since $T$ was arbitrary it follows $g$ is analytic on $\Omega_-$. Since $f(R)\subset R$, we have for real $x$ that $g(x) = f(x^\ast)^\ast = f(x)^\ast = f(x)$, so $g$ agrees with $f$ on $\mathbb{R}$.

By the pasting lemma ($\mathbb{C} = \overline{\Omega_+}\cup \overline{\Omega_+}$ and $f,g$ are continuous on $\overline{\Omega_+},\overline{\Omega_-}$ respectively and agree on $\mathbb{R} = \overline{\Omega_+}\cap \overline{\Omega_-}$) $F$ is continuous on $\mathbb{C}$.

We use Morera's theorem again to show $F$ is analytic on $\mathbb{C}$. If $T$ is any triangle contained in $\Omega_+$ or $\Omega_-$, then by analyticity of $f$ or $g$ an integral along the boundary of $T$ will vanish. So we consider the case when $T\cap \mathbb{R}$ is not trivial. If $T\cap \mathbb{R}$ is a single point the integral along the boundary of $T$ vanishes. If $T\cap \mathbb{R}$ is an interval then the integral vanishes by continuity (literal sketch): \vspace*{7cm}

If $T\cap \mathbb{R}$ is a two point set then a similar continuity argument may be used to show the integral vanishes. Hence all integrals along triangles vanish, so $F$ is entire.
\end{proof}
\end{document}