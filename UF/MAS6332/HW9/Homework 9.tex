\documentclass[11pt]{article}

% packages
\usepackage{physics}
% margin spacing
\usepackage[top=1in, bottom=1in, left=0.5in, right=0.5in]{geometry}
\usepackage{hanging}
\usepackage{amsfonts, amsmath, amssymb, amsthm}
\usepackage{systeme}
\usepackage[none]{hyphenat}
\usepackage{fancyhdr}
\usepackage[nottoc, notlot, notlof]{tocbibind}
\usepackage{graphicx}
\graphicspath{{./images/}}
\usepackage{float}
\usepackage{siunitx}
\usepackage{esint}
\usepackage{cancel}
\usepackage{enumitem}
\usepackage{quiver}

% permutations (second line is for spacing)
\usepackage{permute}
\renewcommand*\pmtseparator{\,}

% colors
\usepackage{xcolor}
\definecolor{p}{HTML}{FFDDDD}
\definecolor{g}{HTML}{D9FFDF}
\definecolor{y}{HTML}{FFFFCF}
\definecolor{b}{HTML}{D9FFFF}
\definecolor{o}{HTML}{FADECB}
%\definecolor{}{HTML}{}

% \highlight[<color>]{<stuff>}
\newcommand{\highlight}[2][p]{\mathchoice%
  {\colorbox{#1}{$\displaystyle#2$}}%
  {\colorbox{#1}{$\textstyle#2$}}%
  {\colorbox{#1}{$\scriptstyle#2$}}%
  {\colorbox{#1}{$\scriptscriptstyle#2$}}}%

% header/footer formatting
\pagestyle{fancy}
\fancyhead{}
\fancyfoot{}
\fancyhead[L]{MAS6332 Algebra}
\fancyhead[C]{Homework 9}
\fancyhead[R]{Sai Sivakumar}
\fancyfoot[R]{\thepage}
\renewcommand{\headrulewidth}{1pt}

% paragraph indentation/spacing
\setlength{\parindent}{0cm}
\setlength{\parskip}{10pt}
\renewcommand{\baselinestretch}{1.25}

% extra commands defined here
\newcommand{\br}[1]{\left(#1\right)}
\newcommand{\sbr}[1]{\left[#1\right]}
\newcommand{\cbr}[1]{\left\{#1\right\}}

\newcommand{\dprime}{\prime\prime}

% bracket notation for inner product
\usepackage{mathtools}

\DeclarePairedDelimiterX{\abr}[1]{\langle}{\rangle}{#1}

\DeclareMathOperator{\Span}{span}
\DeclareMathOperator{\nullity}{nullity}
\DeclareMathOperator\Aut{Aut}
\DeclareMathOperator\Inn{Inn}
\DeclareMathOperator{\Orb}{Orb}
\DeclareMathOperator{\lcm}{lcm}
\DeclareMathOperator{\Hol}{Hol}
\DeclareMathOperator{\Jac}{Jac}
\DeclareMathOperator{\rad}{rad}
\DeclareMathOperator{\Tor}{Tor}
\DeclareMathOperator{\End}{End}
\DeclareMathOperator{\Gal}{Gal}
\DeclareMathOperator{\Nat}{Nat}
\DeclareMathOperator{\Frac}{Frac}
\DeclareMathOperator{\id}{id}
\DeclareMathOperator{\im}{im}
\DeclareMathOperator{\Hom}{Hom}
\DeclareMathOperator{\Ext}{Ext}
\DeclareMathOperator{\aug}{aug}

% set page count index to begin from 1
\setcounter{page}{1}

\begin{document}
\subsection*{Graded}
\begin{enumerate}
    \item (15.4.16) Prove that localization commutes with tensor products: there is a unique isomorphism of $D^{-1}R$ -modules $\varphi\colon (D^{-1}M)\otimes_{D^{-1}R} (D^{-1}N)\cong D^{-1}(M\otimes_R N)$ with $\varphi((m/d)\otimes(n/d^\prime))$ given by $(m\otimes n)/dd^\prime$ for any $R$-modules $M,N$, and multiplicatively closed set $D$ in $R$. \begin{proof}
        
    \end{proof}
    \item (15.4.17) Prove that the $R$-module $A$ is a flat $R$-module if and only if $A_P$ is a flat $R_P$-module for every prime ideal $P$ of $R$ (or just for every maximal ideal of $R$). [Use Proposition 41, Exercises 14 and 16, and the exactness properties of localization.] \begin{proof}
        
    \end{proof}
\end{enumerate}
\subsection*{Additional Problems}
\begin{enumerate}
    \item (15.4.14) Suppose $\varphi\colon M\to N$ is an $R$-module homomorphism. Prove that $\varphi$ is injective (respectively, surjective) if and only if the induced $R_P$-module homomorphism $\varphi\colon M_P\to N_P$ is injective (respectively, surjective) for every prime ideal $P$ of $R$ (or just for every maximal ideal of $R$). \begin{proof}
        
    \end{proof}
    \item (15.4.20) Suppose that $R$ is a subring of the ring $S$ with $1\in R$ and that $S$ is integral over $R$. If $D$ is any multiplicatively closed subset of $R$, prove that $D^{-1}S$ is integral over $D^{-1}R$. \begin{proof}
        
    \end{proof}
    %\item (15.4.24(a-c))
    %\item (15.4.24(d-f))
\end{enumerate}
\subsection*{Feedback}
\begin{enumerate}
    \item None.
    \item Things seem to be the same I think.
\end{enumerate}
\end{document}