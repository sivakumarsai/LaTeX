\documentclass[11pt]{article}

% packages
\usepackage{physics}
% margin spacing
\usepackage[top=1in, bottom=1in, left=0.5in, right=0.5in]{geometry}
\usepackage{hanging}
\usepackage{amsfonts, amsmath, amssymb, amsthm}
\usepackage{systeme}
\usepackage[none]{hyphenat}
\usepackage{fancyhdr}
\usepackage[nottoc, notlot, notlof]{tocbibind}
\usepackage{graphicx}
\graphicspath{{./images/}}
\usepackage{float}
\usepackage{siunitx}
\usepackage{esint}
\usepackage{cancel}
\usepackage{enumitem}
\usepackage{quiver}

% permutations (second line is for spacing)
\usepackage{permute}
\renewcommand*\pmtseparator{\,}

% colors
\usepackage{xcolor}
\definecolor{p}{HTML}{FFDDDD}
\definecolor{g}{HTML}{D9FFDF}
\definecolor{y}{HTML}{FFFFCF}
\definecolor{b}{HTML}{D9FFFF}
\definecolor{o}{HTML}{FADECB}
%\definecolor{}{HTML}{}

% \highlight[<color>]{<stuff>}
\newcommand{\highlight}[2][p]{\mathchoice%
  {\colorbox{#1}{$\displaystyle#2$}}%
  {\colorbox{#1}{$\textstyle#2$}}%
  {\colorbox{#1}{$\scriptstyle#2$}}%
  {\colorbox{#1}{$\scriptscriptstyle#2$}}}%

% header/footer formatting
\pagestyle{fancy}
\fancyhead{}
\fancyfoot{}
\fancyhead[L]{MAS6332 Algebra}
\fancyhead[C]{Homework 3}
\fancyhead[R]{Sai Sivakumar}
\fancyfoot[R]{\thepage}
\renewcommand{\headrulewidth}{1pt}

% paragraph indentation/spacing
\setlength{\parindent}{0cm}
\setlength{\parskip}{10pt}
\renewcommand{\baselinestretch}{1.25}

% extra commands defined here
\newcommand{\br}[1]{\left(#1\right)}
\newcommand{\sbr}[1]{\left[#1\right]}
\newcommand{\cbr}[1]{\left\{#1\right\}}

\newcommand{\dprime}{\prime\prime}

% bracket notation for inner product
\usepackage{mathtools}

\DeclarePairedDelimiterX{\abr}[1]{\langle}{\rangle}{#1}

\DeclareMathOperator{\Span}{span}
\DeclareMathOperator{\nullity}{nullity}
\DeclareMathOperator\Aut{Aut}
\DeclareMathOperator\Inn{Inn}
\DeclareMathOperator{\Orb}{Orb}
\DeclareMathOperator{\lcm}{lcm}
\DeclareMathOperator{\Hol}{Hol}
\DeclareMathOperator{\Jac}{Jac}
\DeclareMathOperator{\rad}{rad}
\DeclareMathOperator{\Tor}{Tor}
\DeclareMathOperator{\End}{End}
\DeclareMathOperator{\Gal}{Gal}
\DeclareMathOperator{\Nat}{Nat}
\DeclareMathOperator{\Frac}{Frac}
\DeclareMathOperator{\id}{id}
\DeclareMathOperator{\im}{im}
\DeclareMathOperator{\Hom}{Hom}
\DeclareMathOperator{\Ext}{Ext}
\DeclareMathOperator{\aug}{aug}

% set page count index to begin from 1
\setcounter{page}{1}

\begin{document}
\subsection*{Graded}
\begin{enumerate}
    \item (17.1.10) \begin{enumerate}
        \item Prove that an arbitrary direct sum $\oplus_{i\in I}P_i$ of projective modules $P_i$ is projective and that an arbitrary direct product $\prod_{j\in J}Q_j$ of injective modules $Q_j$ is injective. \begin{proof}
            Each $P_i$ is a direct summand of a free module $F_i = P_i\oplus K_i$. Then $\oplus_{i\in I}F_i$ is a free module with $\oplus_{i\in I}F_i = (\oplus_{i\in I}P_i)\oplus (\oplus_{i\in I}K_i)$. Hence $\oplus_{i\in I}P_i$ is projective also. 

            Given an injective map $L\xrightarrow{\psi}M$, we can lift maps $f_i\colon L\to Q_i$ to maps $F_i\colon M\to Q_i$ with $F_i\circ\psi = f_i$ for $i\in I$. Then a map $f\colon L\to \prod_{i\in I}Q_i$ is given by $\prod_{i\in I}f_i$ with $f_i = \pi_i\circ f$ (and $\pi_j\colon \prod_{i\in I}Q_i\to Q_j$ is the projection map). Then lifting each $f_i$ to $F_i$ we obtain a map $\prod_{i\in I}F_i\colon M\to \prod_{i\in I}Q_i$ such that $(\prod_{i\in I}F_i)\circ \psi = \prod_{i\in I}(F_i\circ \psi) = \prod_{i\in I}f_i$. It follows that $\prod_{i\in I}Q_i$ is injective.
        \end{proof}
        \item Prove that an arbitrary direct sum of projective resolutions is again projective and use this to show $\Ext_R^n(\oplus_{i\in I}A_i, B)\cong \prod_{i\in I}\Ext_R^n(A_i,B)$ for any collection of $R$-modules $A_i$ ($i\in I$). [cf. Exercise 12 in Section 10.5.] \begin{proof}
            Given projective resolutions $\cbr{P_{ik}}$ of $A_i$ for $i\in I$, the sequence \[\cdots\xrightarrow{\oplus_{i\in I}d_{i3}}\oplus_{i\in I}P_{i2}\xrightarrow{\oplus_{i\in I}d_{i2}}\oplus_{i\in I}P_{i1}\xrightarrow{\oplus_{i\in I}d_{i1}}\oplus_{i\in I}P_{i0}\xrightarrow{\oplus_{i\in I}\epsilon_i}\oplus_{i\in I} A_i\to 0\] where the maps are given by ``direct sums'' of the provided maps (as in we define the map on each component) for each resolution is exact: Kernels and images of direct sums of maps are direct sums of kernels and images, respectively. We have then that $\ker(\oplus_{i\in I}d_{ik}) = \oplus_{i\in I}\ker(d_{ik}) = \oplus_{i\in I}\im(d_{i(k+1)}) = \im(\oplus_{i\in I}d_{i(k+1)})$ for $k\geq 1$, and $\im(\oplus_{i\in I}d_{i1}) = \oplus_{i\in I}\im(d_{i1}) =\oplus_{i\in I}\ker (\epsilon_i) =\ker (\oplus_{i\in I}\epsilon_i)$, and of course $\oplus_{i\in I}\epsilon_i$ is surjective. In the previous part we proved that direct sums of projective modules are projective so the above sequence is a projective resolution of $\oplus_{i\in I}A_i$.

            By applying the $\Hom_R(-,B)$ and recalling the result of Exercise 10.5.12 we obtain the cochain complex \[0\to \prod_{i\in I}\Hom_R(A_i,B)\to \prod_{i\in I}\Hom_R(P_{i0},B)\to \prod_{i\in I}\Hom_R(P_{i1},B)\to\cdots\] and the cohomology of this sequence yields the $\Ext$ groups. The maps are given by ``direct products'' $d_k$ of the maps which precompose $d_{ik}$ (or $\epsilon_i$); they are given by $d_k(f_i) = (f_i\circ d_{ik})$. The kernel (image) of a direct product of maps is the direct product of kernels (images). Furthermore, the quotient of products is the product of quotients: If modules $B_i\subseteq A_i$ for $i\in I$ then $(\prod A_i)/(\prod B_i)\cong \prod(A_i/B_i)$ by the map taking $(a_i)+\prod B_i$ to $(a_i+B_i)$ (and this map is well defined since $(a_i+b_i)+\prod B_i$ is sent to $(a_i+B_i)$ also), which has a left and right inverse given by the map taking $(a_i + B_i)$ to $(a_i)+\prod B_i$ (which is also well defined for similar reasons).
            
            The first group is \[\Ext_R^0(\oplus_{i\in I}A_i, B) = \ker(d_0) \cong \prod_{i\in I}\ker(d_{i0}) = \prod_{i\in I}\Ext_R^0(A_i,B).\] Similarly, we have \begin{multline*}
                \Ext_R^n(\oplus_{i\in I}A_i, B) = \ker(d_{n+1})/\im(d_n) \cong [\prod_{i\in I}\ker(d_{i(n+1)})]/[\prod_{i\in I}\im(d_{in})] \cong \prod_{i\in I}[\ker(d_{i(n+1)})/\im(d_{in})]\\ = \prod_{i\in I}\Ext_R^n(A_i,B).
            \end{multline*}
        \end{proof}
        \item Prove that an arbitrary direct product of injective resolutions is an injective resolution and use this to show $\Ext_R^n(A,\prod_{j\in J}B_j)\cong \prod_{j\in J}\Ext_R^n(A,B_j)$ for any collection of $R$-modules $B_j$ ($j\in J$). [cf. Exercise 12 in Section 10.5.] \begin{proof}
            Given injective resolutions $\cbr{Q_{ik}}$ of $B_i$ for $i\in I$, the sequence \[0\to \prod_{i\in I} B_i \xrightarrow{\prod_{i\in I}\epsilon_i}\prod_{i\in I}Q_{i0}\xrightarrow{\prod_{i\in I}d_{i1}}\prod_{i\in I}Q_{i1}\xrightarrow{\prod_{i\in I}d_{i2}}\prod_{i\in I}Q_{i2}\xrightarrow{\prod_{i\in I}d_{i3}}\cdots\] is exact since kernels (images) of direct products of maps are direct products of kernels (images). Thus $\ker(\prod_{i\in I}d_{ik}) = \prod_{i\in I}\ker(d_{ik}) = \prod_{i\in I}\im(d_{i(k-1)}) = \im(\prod_{i\in I}d_{i(k-1)})$ for $k\geq 2$, and $\ker(\prod_{i\in I}d_{i1}) = \prod_{i\in I}\ker(d_{i1}) =\prod_{i\in I}\im (\epsilon_i) =\im (\prod_{i\in I}\epsilon_i)$, and of course $\prod_{i\in I}\epsilon_i$ is injective. In the first part of this problem we showed that the direct product of injective modules is injective so the above sequence is an injective resolution of $\prod_{i\in I}B_i$.

            Apply the $\Hom_R(A,-)$ functor and recall the result of Exercise 10.5.12 to obtain the cochain complex \[0\to \prod_{i\in I} \Hom_R(A,B_i) \to\prod_{i\in I}\Hom_R(A,Q_{i0})\xrightarrow{d_1}\prod_{i\in I}\Hom_R(A,Q_{i1})\xrightarrow{d_2}\cdots\] whose cohomology gives the $\Ext$ groups. The first group is given by \[\Ext_R^0(A,\prod_{i\in I}B_i) = \ker(d_1) = \ker(\prod_{i\in I}d_{i1}) \cong \prod_{i\in I}\ker(d_{i1}) = \prod_{i\in I}\Ext_R^0(A,B_i)\] and the rest are \begin{multline*}
                \Ext_R^n(A,\prod_{i\in I}B_i) = \ker(d_{n+1})/\im(d_n) \cong [\prod_{i\in I}\ker(d_{i(n+1)})]/[\prod_{i\in I}\im(d_{in})] \cong \prod_{i\in I}[\ker(d_{i(n+1)})/\im(d_{in})] \\ =\prod_{i\in I}\Ext_R^n(A,B_i).
            \end{multline*}
        \end{proof}
        \item Prove that $\Tor_n^R(A,\oplus_{j\in J}B_j)\cong \oplus_{j\in J}\Tor_n^R(A,B_j)$ for any collection of $R$-modules $B_j$ ($j\in J$). \begin{proof}
            Given projective resolutions $\cbr{P_{ik}}$ of $B_i$ for $i\in I$, form the direct sum of the projective resolutions and apply the tensor product functor to obtain the chain complex \begin{multline*}
                \cdots\to \oplus_{i\in I}(A\otimes_R P_{i2})\xrightarrow{\oplus_{i\in I}(1\otimes d_{i2})}\oplus_{i\in I}(A\otimes_R P_{i1})\\\xrightarrow{\oplus_{i\in I}(1\otimes d_{i1})}\oplus_{i\in I}(A\otimes_R P_{i0})\xrightarrow{\oplus_{i\in I}(1\otimes \epsilon_i)}\oplus_{i\in I}(A\otimes_R B_i).
            \end{multline*} The maps are indeed given by the ``direct sum'' of the tensor products of the maps provided due to the isomorphism which distributes the tensor product over direct sums.
            
            The first homology group is \begin{multline*}
                \Tor_0^R(A,\oplus_{i\in I}B_i) = (A\otimes_R (\oplus_{i\in I}P_{i0}))/\im(1\otimes(\oplus_{i\in I}d_{i1}))\cong \oplus_{i\in I}[A\otimes_R P_{i0}]/[\im(1\otimes d_{i1})]\\ =\oplus_{i\in I}\Tor_0^R(A,B_i)
            \end{multline*} and the others are \begin{multline*}
                \Tor_n^R(A,\oplus_{i\in I}B_i) = \ker(1\otimes(\oplus_{i\in I}d_{in}))/\im(1\otimes(\oplus_{i\in I}d_{i(n+1)}))\cong \oplus_{i\in I}[\ker(1\otimes d_{in})]/[\im(1\otimes d_{i(n+1)})]\\ =\oplus_{i\in I}\Tor_n^R(A,B_i).
            \end{multline*}
        \end{proof}
    \end{enumerate}
    \item (17.2.8) Suppose $G$ is cyclic of order $m$ with generator $\sigma$ and let $N = 1+\sigma + \sigma^2 + \cdots + \sigma^{m-1}\in\mathbb{Z}G$.\begin{enumerate}
        \item Prove that the \textit{augmentation map} $\aug(\sum_{i=0}^{m-1}a_i\sigma^i) = \sum_{i=0}^{m-1}a_i$ is a $G$-module homomorphism from $\mathbb{Z}G$ to $\mathbb{Z}$. \begin{proof}
            Every element of $\mathbb{Z}G$ may be written in the form $\sum_{i=0}^{m-1}a_i\sigma^i$. It is clear that this map is additive: $\aug(\sum_{i=0}^{m-1}a_i\sigma^i + \sum_{i=0}^{m-1}b_i\sigma^i) = \aug(\sum_{i=0}^{m-1}(a_i+b_i)\sigma^i) = \sum_{i=0}^{m-1}(a_i+b_i) = \sum_{i=0}^{m-1}a_i+\sum_{i=0}^{m-1}b_i = \aug(\sum_{i=0}^{m-1}a_i\sigma^i) + \aug(\sum_{i=0}^{m-1}b_i\sigma^i)$.
            
            This map respects the action of $\mathbb{Z}G$: In $\mathbb{Z}G$ we have $(\sum_{i=0}^{m-1}a_i\sigma^i)(\sum_{j=0}^{m-1}b_j\sigma^j) = \sum_{0\leq i,j\leq m-1}(a_ib_j)\sigma^{i+j}$ and in $\mathbb{Z}$ we have $(\sum_{i=0}^{m-1}a_i\sigma^i)(\sum_{j=0}^{m-1}b_j) = \sum_{0\leq i,j\leq m-1}a_i\sigma^ib_j = \sum_{0\leq i,j\leq m-1}a_ib_j$. By additivity we have \begin{align*}
                \aug\sbr{\br{\sum_{i=0}^{m-1}a_i\sigma^i}\br{\sum_{j=0}^{m-1}b_j\sigma^j}} &= \sum_{0\leq i,j\leq m-1}a_ib_j \\
                &= \br{\sum_{i=0}^{m-1}a_i\sigma^i}\br{\sum_{j=0}^{m-1}b_j} \\
                &= \br{\sum_{i=0}^{m-1}a_i\sigma^i}\aug\br{\sum_{j=0}^{m-1}b_j\sigma^j}
            \end{align*} from which it follows that $\aug\colon \mathbb{Z}G\to \mathbb{Z}$ is a $\mathbb{Z}G$-module homomorphism.
        \end{proof}
        \item Prove that multiplication by $N$ and by $\sigma-1$ in $\mathbb{Z}G$ define a free $G$-module resolution of $\mathbb{Z}$: $\cdots\xrightarrow{\sigma-1}\mathbb{Z}G\xrightarrow{N}\mathbb{Z}G\xrightarrow{\sigma-1}\cdots\xrightarrow{N}\mathbb{Z}G\xrightarrow{\sigma-1}\mathbb{Z}G\xrightarrow{\aug}\mathbb{Z}\to 0$. \begin{proof}
            The augmentation map is surjective. Given $a\in\mathbb{Z}$, $\aug(a1_G) = a$. 
            
            The image of $\sigma-1$ is equal to the kernel of the augmentation map: For any $\sum_{i=0}^{m-1}a_i\sigma^i$, we have $(\sigma-1)\sum_{i=0}^{m-1}a_i\sigma^i = (a_{m-1}-a_0)1_G + (a_0-a_1)\sigma + (a_1-a_2)\sigma^2 + \cdots + (a_{m-2}-a_{m-1})\sigma^{m-1}$, which is sent to $(a_{m-1}-a_0) + (a_0-a_1) + (a_1-a_2) + \cdots + (a_{m-2}-a_{m-1}) =0$ under the augmentation map. For the reverse inclusion, given $\sum_{i=0}^{m-1}a_i\sigma^i$ with $\sum_{i=0}^{m-1}a_i = 0$, write $a_0 = -a_1 - \cdots - a_{m-1}$ to find that $\sum_{i=0}^{m-1}a_i\sigma^i = \sum_{i=1}^{m-1}a_i(\sigma^i-1).$ Since $\sigma-1$ divides $\sigma^i-1$ for $1\leq i\leq m-1$, it follows that $\sum_{i=0}^{m-1}a_i\sigma^i = (\sigma-1)\sum_{i=1}^{m-1}a_i(\sigma^i-1)/(\sigma-1).$

            The image of $N$ is equal to the kernel of $\sigma-1$: For any $\sum_{i=0}^{m-1}a_i\sigma^i$, we have $(\sigma-1)(N(\sum_{i=0}^{m-1}a_i\sigma^i)) = [(\sigma-1)N]\sum_{i=0}^{m-1}a_i\sigma^i = (01_G)\sum_{i=0}^{m-1}a_i\sigma^i = 0$ since $(\sigma-1)N = \sigma^m-1_G = 1_G - 1_G$. For the reverse inclusion, given $\sum_{i=0}^{m-1}a_i\sigma^i$ with $(\sigma-1)\sum_{i=0}^{m-1}a_i\sigma^i = (a_{m-1}-a_0)1_G + (a_0-a_1)\sigma + (a_1-a_2)\sigma^2 + \cdots + (a_{m-2}-a_{m-1})\sigma^{m-1} = 0$, we must have for $0\leq i\leq m-1$ that $a_i = c$, for some constant $c\in\mathbb{Z}$. Then $\sum_{i=0}^{m-1}a_i\sigma^i = N(c1_G)$ as desired.

            The image of $\sigma-1$ is equal to the kernel of $N$: For any $\sum_{i=0}^{m-1}a_i\sigma^i$, we have $N((\sigma-1)(\sum_{i=0}^{m-1}a_i\sigma^i)) = [N(\sigma-1)]\sum_{i=0}^{m-1}a_i\sigma^i = (01_G)\sum_{i=0}^{m-1}a_i\sigma^i = 0$ since $N(\sigma-1) = \sigma^m-1_G = 1_G - 1_G$. For the reverse inclusion, given $\sum_{i=0}^{m-1}a_i\sigma^i$ with $N(\sum_{i=0}^{m-1}a_i\sigma^i) = \sum_{i=0}^{m-1}(\sum_{j=0}^{m-1}a_j)\sigma^i = 0$, we must have $\sum_{j=0}^{m-1}a_j = 0$ so that we may write $a_0 = -a_1-\cdots-a_{m-1}$. Then like before we have $\sum_{i=0}^{m-1}a_i\sigma^i = \sum_{i=1}^{m-1}a_i(\sigma^i-1) = (\sigma-1)\sum_{i=1}^{m-1}a_i(\sigma^i-1)/(\sigma-1)$.

            With $\mathbb{Z}G$ a free over itself, we have from the above that the sequence $\cdots\xrightarrow{\sigma-1}\mathbb{Z}G\xrightarrow{N}\mathbb{Z}G\xrightarrow{\sigma-1}\cdots\xrightarrow{N}\mathbb{Z}G\xrightarrow{\sigma-1}\mathbb{Z}G\xrightarrow{\aug}\mathbb{Z}\to 0$ is exact, hence it is a free $\mathbb{Z}G$-module resolution of $\mathbb{Z}$.
        \end{proof}
    \end{enumerate}
\end{enumerate}
\subsection*{Additional Problems}
\begin{enumerate}
    \item (17.1.9)
    %, 17.1.15, 17.1.21, 17.2.6)
    Show that \[0\to \mathbb{Z}/d\mathbb{Z}\to\mathbb{Z}/m\mathbb{Z}\xrightarrow{d}\mathbb{Z}/m\mathbb{Z}\xrightarrow{m/d}\mathbb{Z}/m\mathbb{Z}\xrightarrow{d}\mathbb{Z}/m\mathbb{Z}\xrightarrow{m/d}\cdots \] is an injective resolution of $\mathbb{Z}/d\mathbb{Z}$ as a $\mathbb{Z}/m\mathbb{Z}$-module. [Use Proposition 36 in Section 10.5.] Use this to compute the groups $\Ext_{\mathbb{Z}/m\mathbb{Z}}^n(A,\mathbb{Z}/d\mathbb{Z})$ in terms of the dual group $\Hom_{\mathbb{Z}/m\mathbb{Z}}(A,\mathbb{Z}/m\mathbb{Z})$. In particular, if $m = p^2$ and $d = p$, give another derivation of the result $\Ext_{\mathbb{Z}/p^2\mathbb{Z}}^n(\mathbb{Z}/p\mathbb{Z},\mathbb{Z}/p\mathbb{Z})\cong \mathbb{Z}/p\mathbb{Z}$.
    \item (17.1.15)\begin{enumerate}
        \item If $I$ is an ideal of $R$ and $M$ is an $R$-module, prove that $\Tor_1^R(M,R/I)$ is isomorphic to the kernel of the map $M\otimes_R I\to M$ that maps $m\otimes i$ to $mi$ for $i\in I$ and $m\in M$. [Use the $\Tor$ long exact sequence associated to $0\to I\to R \to R/I\to 0$ noting that $R$ is flat.]
        \item \textit{(A Flatness Criterion using $\Tor$)} Prove that the $R$-module $M$ is flat if and only if $\Tor_1^R(M,R/I) = 0$ for every finitely generated ideal $I$ of $R$. [Use Exercise 25 in Section 10.5.]
    \end{enumerate}
\end{enumerate}
\subsection*{Feedback}
\begin{enumerate}
    \item None.
    \item Things seem to be the same I think.
\end{enumerate}
\end{document}