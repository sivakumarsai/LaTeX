\documentclass[11pt]{article}

% packages
\usepackage{physics}
% margin spacing
\usepackage[top=1in, bottom=1in, left=0.5in, right=0.5in]{geometry}
\usepackage{hanging}
\usepackage{amsfonts, amsmath, amssymb, amsthm}
\usepackage{systeme}
\usepackage[none]{hyphenat}
\usepackage{fancyhdr}
\usepackage[nottoc, notlot, notlof]{tocbibind}
\usepackage{graphicx}
\graphicspath{{./images/}}
\usepackage{float}
\usepackage{siunitx}
\usepackage{esint}
\usepackage{cancel}
\usepackage{enumitem}
\usepackage{quiver}

% permutations (second line is for spacing)
\usepackage{permute}
\renewcommand*\pmtseparator{\,}

% colors
\usepackage{xcolor}
\definecolor{p}{HTML}{FFDDDD}
\definecolor{g}{HTML}{D9FFDF}
\definecolor{y}{HTML}{FFFFCF}
\definecolor{b}{HTML}{D9FFFF}
\definecolor{o}{HTML}{FADECB}
%\definecolor{}{HTML}{}

% \highlight[<color>]{<stuff>}
\newcommand{\highlight}[2][p]{\mathchoice%
  {\colorbox{#1}{$\displaystyle#2$}}%
  {\colorbox{#1}{$\textstyle#2$}}%
  {\colorbox{#1}{$\scriptstyle#2$}}%
  {\colorbox{#1}{$\scriptscriptstyle#2$}}}%

% header/footer formatting
\pagestyle{fancy}
\fancyhead{}
\fancyfoot{}
\fancyhead[L]{MAS6332 Algebra}
\fancyhead[C]{Homework 3}
\fancyhead[R]{Sai Sivakumar}
\fancyfoot[R]{\thepage}
\renewcommand{\headrulewidth}{1pt}

% paragraph indentation/spacing
\setlength{\parindent}{0cm}
\setlength{\parskip}{10pt}
\renewcommand{\baselinestretch}{1.25}

% extra commands defined here
\newcommand{\br}[1]{\left(#1\right)}
\newcommand{\sbr}[1]{\left[#1\right]}
\newcommand{\cbr}[1]{\left\{#1\right\}}

\newcommand{\dprime}{\prime\prime}

% bracket notation for inner product
\usepackage{mathtools}

\DeclarePairedDelimiterX{\abr}[1]{\langle}{\rangle}{#1}

\DeclareMathOperator{\Span}{span}
\DeclareMathOperator{\nullity}{nullity}
\DeclareMathOperator\Aut{Aut}
\DeclareMathOperator\Inn{Inn}
\DeclareMathOperator{\Orb}{Orb}
\DeclareMathOperator{\lcm}{lcm}
\DeclareMathOperator{\Hol}{Hol}
\DeclareMathOperator{\Jac}{Jac}
\DeclareMathOperator{\rad}{rad}
\DeclareMathOperator{\Tor}{Tor}
\DeclareMathOperator{\End}{End}
\DeclareMathOperator{\Gal}{Gal}
\DeclareMathOperator{\Nat}{Nat}
\DeclareMathOperator{\Frac}{Frac}
\DeclareMathOperator{\id}{id}
\DeclareMathOperator{\im}{im}
\DeclareMathOperator{\Hom}{Hom}
\DeclareMathOperator{\Ext}{Ext}
\DeclareMathOperator{\aug}{aug}

% set page count index to begin from 1
\setcounter{page}{1}

\begin{document}
\subsection*{Graded}
\begin{enumerate}
    \item (17.1.10) \begin{enumerate}
        \item Prove that an arbitrary direct sum $\oplus_{i\in I}P_i$ of projective modules $P_i$ is projective and that an arbitrary direct product $\prod_{j\in J}Q_j$ of injective modules $Q_j$ is injective.
        \item Prove that an arbitrary direct sum of projective resolutions is again projective and use this to show $\Ext_R^n(\oplus_{i\in I}A_i, B)\cong \prod_{i\in I}\Ext_R^n(A_i,B)$ for any collection of $R$-modules $A_i$ ($i\in I$). [cf. Exercise 12 in Section 10.5.]
        \item Prove that an arbitrary direct product of injective resolutions is an injective resolution and use this to show $\Ext_R^n(A,\prod_{j\in J}B_j)\cong \prod_{j\in J}\Ext_R^n(A,B_j)$ for any collection of $R$-modules $B_j$ ($j\in J$). [cf. Exercise 12 in Section 10.5.]
        \item Prove that $\Tor_n^R(A,\oplus_{j\in J}B_j)\cong \oplus_{j\in J}\Tor_n^R(A,B_j)$ for any collection of $R$-modules $B_j$ ($j\in J$).
    \end{enumerate}
    \item (17.2.8) Suppose $G$ is cyclic of order $m$ with generator $\sigma$ and let $N = 1+\sigma + \sigma^2 + \cdots + \sigma^{m-1}\in\mathbb{Z}G$.\begin{enumerate}
        \item Prove that the \textit{augmentation map} $\aug(\sum_{i=0}^{m-1}a_i\sigma^i) = \sum_{i=0}^{m-1}a_i$ is a $G$-module homomorphism from $\mathbb{Z}G$ to $\mathbb{Z}$. \begin{proof}
            
        \end{proof}
        \item Prove that multiplication by $N$ and by $\sigma-1$ in $\mathbb{Z}G$ define a free $G$-module resolution of $\mathbb{Z}$: $\cdots\xrightarrow{\sigma-1}\mathbb{Z}G\xrightarrow{N}\mathbb{Z}G\xrightarrow{\sigma-1}\cdots\xrightarrow{N}\mathbb{Z}G\xrightarrow{\sigma-1}\mathbb{Z}G\xrightarrow{\aug}\mathbb{Z}\to 0$.
    \end{enumerate}
\end{enumerate}
\subsection*{Additional Problems}
\begin{enumerate}
    \item (17.1.9, 17.1.15, 17.1.21, 17.2.6)
\end{enumerate}
\subsection*{Feedback}
\begin{enumerate}
    \item None.
    \item Things seem to be the same I think.
\end{enumerate}
\end{document}