\documentclass[11pt]{article}

% packages
\usepackage{physics}
% margin spacing
\usepackage[top=1in, bottom=1in, left=0.5in, right=0.5in]{geometry}
\usepackage{hanging}
\usepackage{amsfonts, amsmath, amssymb, amsthm}
\usepackage{systeme}
\usepackage[none]{hyphenat}
\usepackage{fancyhdr}
\usepackage[nottoc, notlot, notlof]{tocbibind}
\usepackage{graphicx}
\graphicspath{{./images/}}
\usepackage{float}
\usepackage{siunitx}
\usepackage{esint}
\usepackage{cancel}
\usepackage{enumitem}
\usepackage{quiver}

% permutations (second line is for spacing)
\usepackage{permute}
\renewcommand*\pmtseparator{\,}

% colors
\usepackage{xcolor}
\definecolor{p}{HTML}{FFDDDD}
\definecolor{g}{HTML}{D9FFDF}
\definecolor{y}{HTML}{FFFFCF}
\definecolor{b}{HTML}{D9FFFF}
\definecolor{o}{HTML}{FADECB}
%\definecolor{}{HTML}{}

% \highlight[<color>]{<stuff>}
\newcommand{\highlight}[2][p]{\mathchoice%
  {\colorbox{#1}{$\displaystyle#2$}}%
  {\colorbox{#1}{$\textstyle#2$}}%
  {\colorbox{#1}{$\scriptstyle#2$}}%
  {\colorbox{#1}{$\scriptscriptstyle#2$}}}%

% header/footer formatting
\pagestyle{fancy}
\fancyhead{}
\fancyfoot{}
\fancyhead[L]{MAS6332 Algebra}
\fancyhead[C]{Homework 7}
\fancyhead[R]{Sai Sivakumar}
\fancyfoot[R]{\thepage}
\renewcommand{\headrulewidth}{1pt}

% paragraph indentation/spacing
\setlength{\parindent}{0cm}
\setlength{\parskip}{10pt}
\renewcommand{\baselinestretch}{1.25}

% extra commands defined here
\newcommand{\br}[1]{\left(#1\right)}
\newcommand{\sbr}[1]{\left[#1\right]}
\newcommand{\cbr}[1]{\left\{#1\right\}}

\newcommand{\dprime}{\prime\prime}

% bracket notation for inner product
\usepackage{mathtools}

\DeclarePairedDelimiterX{\abr}[1]{\langle}{\rangle}{#1}

\DeclareMathOperator{\Span}{span}
\DeclareMathOperator{\nullity}{nullity}
\DeclareMathOperator\Aut{Aut}
\DeclareMathOperator\Inn{Inn}
\DeclareMathOperator{\Orb}{Orb}
\DeclareMathOperator{\lcm}{lcm}
\DeclareMathOperator{\Hol}{Hol}
\DeclareMathOperator{\Jac}{Jac}
\DeclareMathOperator{\rad}{rad}
\DeclareMathOperator{\Tor}{Tor}
\DeclareMathOperator{\End}{End}
\DeclareMathOperator{\Gal}{Gal}
\DeclareMathOperator{\Nat}{Nat}
\DeclareMathOperator{\Frac}{Frac}
\DeclareMathOperator{\id}{id}
\DeclareMathOperator{\im}{im}
\DeclareMathOperator{\Hom}{Hom}
\DeclareMathOperator{\Ext}{Ext}
\DeclareMathOperator{\aug}{aug}

% set page count index to begin from 1
\setcounter{page}{1}

\begin{document}
\subsection*{Graded}
\begin{enumerate}
    \item (15.2.39) Fix an element $a$ in the ring $R$. For any ideal $I$ in the ring $R$ let $I_a= \cbr{r\in R\mid ar\in I}$.\begin{enumerate}
        \item Prove that $I_a$ is an ideal and $I_a=R$ if and only if $a\in I$.
        \item Prove that $(I\cap J)_a = I_a\cap J_a$ for ideals $I$ and $J$.
        \item Suppose that $Q$ is a $P$-primary ideal and that $a\not\in Q$. Prove that $Q_a$ is a $P$-primary ideal and that $Q_a=Q$ if $a\not\in P$.
    \end{enumerate}
    \item (15.2.40) With notation as in the previous exercise, suppose $I=Q_1\cap\cdots\cap Q_m$ is a minimal primary decomposition of the ideal $I$ and let $P_i$ be the prime ideal associated to $Q_i$. \begin{enumerate}
        \item Prove that $I_a = (Q_1)_a\cap\cdots\cap (Q_m)_a$ and that $\rad (I_a) = \rad((Q_1)_a)\cap\cdots\cap \rad((Q_m)_a)$.
        \item Prove that $\rad (I_a)$ is the intersection of the prime ideals $P_i$ for which $a\not\in Q_i$. [Use the previous exercise.]
        \item Prove that if $\rad(I_a)$ is a prime ideal then $\rad(I_a) = P_j$ for some $j$. [Use the fact that prime ideals are irreducible.]
        \item For each $i=1,\dots,m$, prove that $\rad(I_a)=P_i$ for some $a\in R$. [Show there exists an $a\in R$ with $a\not\in Q_j$ for all $j\neq i$.]
        \item Show from (c) and (d) that the associated primes for a minimal primary decomposition are precisely the collection of prime ideals among the ideals $\rad(I_a)$ for $a\in R$, and conclude that they are uniquely determined by $I$ independent of the minimal primary decomposition.
    \end{enumerate}
\end{enumerate}
\subsection*{Additional Problems}
\begin{enumerate}
    \item (15.2.12)
    \item (15.2.16)
    \item (15.2.25)
    \item (15.2.51)
\end{enumerate}
\subsection*{Feedback}
\begin{enumerate}
    \item None.
    \item Things seem to be the same I think.
\end{enumerate}
\end{document}