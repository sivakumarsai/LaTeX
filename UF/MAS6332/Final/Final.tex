\documentclass[11pt]{article}
\headheight=13.6pt
% packages
\usepackage{physics}
% margin spacing
\usepackage[top=1in, bottom=1in, left=0.5in, right=0.5in]{geometry}
\usepackage{hanging}
\usepackage{amsfonts, amsmath, amssymb, amsthm}
\usepackage{systeme}
\usepackage[none]{hyphenat}
\usepackage{fancyhdr}
\usepackage[nottoc, notlot, notlof]{tocbibind}
\usepackage{graphicx}
\graphicspath{{./images/}}
\usepackage{float}
\usepackage{siunitx}
\usepackage{esint}
\usepackage{cancel}
\usepackage{enumitem}
\usepackage{quiver}

% permutations (second line is for spacing)
\usepackage{permute}
\renewcommand*\pmtseparator{\,}

% colors
\usepackage{xcolor}
\definecolor{p}{HTML}{FFDDDD}
\definecolor{g}{HTML}{D9FFDF}
\definecolor{y}{HTML}{FFFFCF}
\definecolor{b}{HTML}{D9FFFF}
\definecolor{o}{HTML}{FADECB}
%\definecolor{}{HTML}{}

% \highlight[<color>]{<stuff>}
\newcommand{\highlight}[2][p]{\mathchoice%
  {\colorbox{#1}{$\displaystyle#2$}}%
  {\colorbox{#1}{$\textstyle#2$}}%
  {\colorbox{#1}{$\scriptstyle#2$}}%
  {\colorbox{#1}{$\scriptscriptstyle#2$}}}%

% header/footer formatting
\pagestyle{fancy}
\fancyhead{}
\fancyfoot{}
\fancyhead[L]{MAS6332 Algebra}
\fancyhead[C]{Final}
\fancyhead[R]{Sai Sivakumar}
\fancyfoot[R]{\thepage}
\renewcommand{\headrulewidth}{1pt}

% paragraph indentation/spacing
\setlength{\parindent}{0cm}
\setlength{\parskip}{10pt}
\renewcommand{\baselinestretch}{1.25}

% extra commands defined here
\newcommand{\br}[1]{\left(#1\right)}
\newcommand{\sbr}[1]{\left[#1\right]}
\newcommand{\cbr}[1]{\left\{#1\right\}}

\newcommand{\dprime}{\prime\prime}

% bracket notation for inner product
\usepackage{mathtools}

\DeclarePairedDelimiterX{\abr}[1]{\langle}{\rangle}{#1}

\DeclareMathOperator{\Span}{span}
\DeclareMathOperator{\nullity}{nullity}
\DeclareMathOperator\Aut{Aut}
\DeclareMathOperator\Inn{Inn}
\DeclareMathOperator{\Orb}{Orb}
\DeclareMathOperator{\lcm}{lcm}
\DeclareMathOperator{\Hol}{Hol}
\DeclareMathOperator{\Jac}{Jac}
\DeclareMathOperator{\rad}{rad}
\DeclareMathOperator{\Tor}{Tor}
\DeclareMathOperator{\End}{End}
\DeclareMathOperator{\Gal}{Gal}
\DeclareMathOperator{\Nat}{Nat}
\DeclareMathOperator{\Frac}{Frac}
\DeclareMathOperator{\id}{id}
\DeclareMathOperator{\im}{im}
\DeclareMathOperator{\Hom}{Hom}
\DeclareMathOperator{\Ext}{Ext}
\DeclareMathOperator{\aug}{aug}

% set page count index to begin from 1
\setcounter{page}{1}

\begin{document}
\subsection*{Localizations of Projective Modules 1}
\begin{enumerate}[label=(\alph*)]
  \item We show first that the localization of a free module is a free module. In the case that $0\in D$, then $D^{-1}R$ is the trivial ring and so the localization is the zero module, since the tensor product with the zero ring is the zero module (which is free). So assume $0\not\in D$.
  
  Let $F$ be the free $R$-module with $R$-basis $A$. Then $D^{-1}F$ satisfies the universal property of free modules with $A$ as a $D^{-1}R$-basis:

  Let $M$ be any $D^{-1}R$-module and let $\varphi\colon A\to M$ be any set map. Then let $\iota\colon A\to D^{-1}F$ be the set map taking $a$ to $a/1$. This map is an injection: if $a_1/1=a_2/1$ with $a_1\neq a_2$, then there exists $y\in D$ with $y(a_1-a_2)=0$, which is impossible since $F$ is free. 

  Then define $\Phi\colon D^{-1}F\to M$ by $\Phi(a/d) = \varphi(a)/d$ for $a\in A$, and extend linearly to make $\Phi$ a $D^{-1}R$-module homomorphism. We check that $\Phi$ is well defined: if $f/d = g/e$ with $f = \sum_i f_i a_i$ and $g = \sum_j g_j a_j$, then there is $y\in D$ such that $y(e\sum_i f_ia_i - d\sum_jg_ja_j)=0$. But $M$ is also an $R$-module by the action $rm = (r/1)m$ (as $\pi\colon R\to D^{-1}R$ taking $r$ to $r/1$ is a ring homomorphism). As $F$ is a free $R$-module it follows that there is a unique $R$-module homomorphism $\tilde \Phi\colon F\to M$ such that for $a\in A$, $\tilde \Phi(a) = \varphi(a)$ (by the universal property of free modules and identifying $A\subseteq F$). It follows then that $\tilde \Phi(y(e\sum_i f_ia_i - d\sum_jg_ja_j)) = y(e\sum_i f_i\varphi(a_i) - d\sum_jg_j\varphi(a_j))=0$, from which it follows that $\Phi(f/d) = [\sum_i f_i \varphi(a_i)]/d = [\sum_j g_j \varphi(a_j)]/e = \Phi(g/e)$.

  We have that $\Phi\iota = \varphi$, and we check that $\Phi$ is unique: If there was $\Theta\colon D^{-1}F\to M$ such that $\Theta\iota = \varphi$, then we have by linearity that for any $f/d = [\sum_i f_i a_i]/d \in D^{-1}F$, that $\Theta(f/d) = [\sum_i f_i \varphi(a_i)]/d = \Phi(f/d)$. Hence $\Phi = \Theta$ as needed.

  Thus $D^{-1}F$ is the free $D^{-1}R$-module on $A$.

  If $M$ is a projective $R$-module, it is a direct summand of some free $R$-module $F$; that is, $F \cong M\oplus N$. Then by localizing (and using the fact that localization distributes over direct sums), we have that $D^{-1}F \cong D^{-1}M\oplus D^{-1}N$, and as $D^{-1}F$ is free, it follows that $D^{-1}M$ is projective.
\end{enumerate}
\subsection*{Localization and $\Ext$}
\begin{enumerate}[label=(\alph*)]
  \item A natural map $\psi\colon D^{-1}\Hom_R(M,N)\to \Hom_{D^{-1}R}(D^{-1}M,D^{-1}N)$ is given by $\psi(f/w) = (1/w)D^{-1}f$, where $D^{-1}f(m/d) = f(m)/d$. We check that this map is well defined and is an $D^{-1}R$-module homomorphism.
  
  Suppose that $f/w = g/v$ so that there was $y\in D$ such that $y(vf-wg)=0$. Then let $m/d$ be any element of $D^{-1}M$. We have that $y(vf(m)-wg(m))=0$, so that $f(m)/w = g(m)/v$. Then $(1/w)D^{-1}f(m/d) = f(m)/wd = g(m)/vd = (1/v)D^{-1}g(m/d)$. Since $m/d$ was arbitrary, it follows that $(1/w)D^{-1}f = (1/v)D^{-1}g$.

  We have \begin{multline*}
    \psi(rf/dw + g/v) = (1/dwv)D^{-1}(vrf+dwg) = (vr/dwv)D^{-1}(f) + (dw/dwv)D^{-1}(g) \\ = (r/d)(1/w)D^{-1}f + (1/v)D^{-1}g = (r/d)\psi(f/w) + \psi(g/v)
  \end{multline*} so that $\psi$ is a $D^{-1}R$-module homomorphism.

  Now let $M$ be finitely generated, by $\cbr{m_1,\dots,m_n}$. We show that $\psi$ is injective. Suppose that $\psi(f/w) = (1/w)D^{-1}f = 0$ so that $(1/w)D^{-1}f(m_i/1) = f(m_i)/w = 0$ for each $i$. Hence there exists $y_i\in D$ such that $y_if(m_i) = 0$ for each $i$. Now let $m = \sum_i r_im_i$ be any element of $M$, and note that $\prod_j y_j\in D$. Then $(\prod_j y_j)f(m) = \sum_i r_i(\prod_j y_j)f(m_i) = 0$ since in each summand $r_i(\prod_j y_j)f(m_i)$, the factor $y_if(m_i) = 0$ appears. It follows that $(\prod_j y_j)f = 0$, so that $f/w = 0$ and so $\psi$ is injective.

  Since $M$ is finitely generated and $R$ is Noetherian, there is an exact sequence $R^m\to R^n\xrightarrow{\pi} M\to 0$ (the remark). We obtain two exact sequences by applying the contravariant $\Hom$ functor and localization in different ways. 

  Applying the $\Hom_R(-,N)$ functor first we obtain the exact sequence $0\to \Hom_R(M,N)\xrightarrow{\pi^\prime}N^n\to N^m$. (The resulting sequence is exact since we are not starting out with a short exact sequence but rather just a four term exact sequence ending in $0$, see end of Theorem 33 in 10.5) Then localize the sequence (and note localization is exact) to obtain \[0\to D^{-1}\Hom_R(M,N)\xrightarrow{D^{-1}\pi^\prime}(D^{-1}N)^n\to (D^{-1}N)^m.\] Let $f/w\in D^{-1}\Hom_R(M,N)$, noting that $f\colon M\to N$. Then by tracing through several induced maps and isomorphisms we have that the map $D^{-1}\pi^\prime$ is given by the composite taking \begin{multline*}
    f/w\mapsto f\pi/w\mapsto (f\pi|_1,\dots,f\pi|_n)/w\mapsto (f\pi(1,0,\dots,0),\dots,f\pi(0,\dots,0,1))/w \\\mapsto (f\pi(1,0,\dots,0)/w,\dots,f\pi(0,\dots,0,1)/w)
  \end{multline*} [with $\pi|_i$ is the map taking $(r_1,\dots,r_n)$ to $\pi(0,\dots,r_i,\dots,0)$].

  Instead if we apply localization first we obtain the exact sequence $(D^{-1}R)^m\to (D^{-1}R)^n\xrightarrow{D^{-1}\pi} D^{-1}M\to 0$. Then applying $\Hom_{D^{-1}R}(-,D^{-1}N)$ we obtain the exact sequence \[0\to \Hom_{D^{-1}R}(D^{-1}M,D^{-1}N)\xrightarrow{(D^{-1}\pi)^\prime}(D^{-1}N)^n\to (D^{-1}N)^m.\] Again by tracing through induced maps and isomorphisms we have for $g\colon D^{-1}M\to D^{-1}N$ that $(D^{-1}\pi)^\prime$ is given by \begin{multline*}
    g\mapsto g\circ D^{-1}\pi\mapsto (g\circ D^{-1}\pi)^\dagger\mapsto ((g\circ D^{-1}\pi)^\dagger\circ \iota_1,\dots,(g\circ D^{-1}\pi)^\dagger\circ \iota_n)\\ \mapsto (g(\pi(1,0,\dots,0)/1),\dots,g(\pi(0,\dots,0,1)/1))
  \end{multline*} [with $(g\circ D^{-1}\pi)^\dagger$ the map taking $(r_1/d_1,\dots,r_n/d_n)$ to $(g\circ D^{-1}\pi)((r_1d_2\cdots d_n, \dots, d_1\cdots d_{n-1}r_n)/(d_1\cdots d_n))$ and $\iota_i$ taking $r/d$ to $(0,\dots,\underbrace{r/d}_{\text{position }i},\dots,0)$.] In the case that $g = \psi(f/w) = (1/w)D^{-1}f, $ we have that \begin{multline*}
    [(D^{-1}\pi)^\prime\circ \psi](f/w) = (D^{-1}\pi)^\prime((1/w)D^{-1}f) \\= ((1/w)D^{-1}f(\pi(1,0,\dots,0)/1),\dots,(1/w)D^{-1}f(\pi(0,\dots,0,1)/1))\\ = (f\pi(1,0,\dots,0)/w,\dots,f\pi(0,\dots,0,1)/w) = (D^{-1}\pi^\prime)(f/w).
  \end{multline*}

  It follows that the diagram below commutes (the left and right squares commute automatically): % https://q.uiver.app/?q=WzAsOCxbMCwwLCIwIl0sWzAsMSwiMCJdLFsxLDAsIkReey0xfVxcSG9tX1IoTSxOKSJdLFsyLDAsIihEXnstMX1OKV5uIl0sWzMsMCwiKEReey0xfU4pXm0iXSxbMywxLCIoRF57LTF9TilebSJdLFsyLDEsIihEXnstMX1OKV5uIl0sWzEsMSwiXFxIb21fe0Reey0xfVJ9KEReey0xfU0sRF57LTF9TikiXSxbMCwxLCJcXGlkIl0sWzIsMywiRF57LTF9XFxwaV5cXHByaW1lIl0sWzMsNF0sWzYsNV0sWzMsNiwiXFxpZCJdLFs0LDUsIlxcaWQiXSxbMCwyXSxbMSw3XSxbNyw2LCIoRF57LTF9XFxwaSleXFxwcmltZSJdLFsyLDcsIlxccHNpIl1d
  \[\begin{tikzcd}
    0 & {D^{-1}\Hom_R(M,N)} & {(D^{-1}N)^n} & {(D^{-1}N)^m} \\
    0 & {\Hom_{D^{-1}R}(D^{-1}M,D^{-1}N)} & {(D^{-1}N)^n} & {(D^{-1}N)^m}
    \arrow["\id", from=1-1, to=2-1]
    \arrow["{D^{-1}\pi^\prime}", from=1-2, to=1-3]
    \arrow[from=1-3, to=1-4]
    \arrow[from=2-3, to=2-4]
    \arrow["\id", from=1-3, to=2-3]
    \arrow["\id", from=1-4, to=2-4]
    \arrow[from=1-1, to=1-2]
    \arrow[from=2-1, to=2-2]
    \arrow["{(D^{-1}\pi)^\prime}", from=2-2, to=2-3]
    \arrow["\psi", from=1-2, to=2-2]
  \end{tikzcd}\] Since the identity maps are bijective, it follows that $\psi$ is surjective (this is from DF10.5.2, a previous homework problem from the fall).

  It follows that $\psi$ is an isomorphism.

  \item Since $M$ is finitely generated and $R$ is Noetherian, there is an exact sequence $R^m\xrightarrow{d_1} R^n\xrightarrow{\pi} M\to 0$. This exact sequence may be extended into a free resolution of $M$ via the manner described in Dummit and Foote: Find a free module $P_2$ which maps surjectively onto $\ker d_1\subseteq R^m$ and denote the composition of the surjection with the inclusion $\ker d_1\subseteq R^m$ by $d_2$, and $d_2$ is exact by construction. Continue inductively for each $n$ by finding a free module $P_{n+1}$ which maps surjectively onto $\ker d_n\subseteq P_n$ and denote the resulting map $d_{n+1}$.
  
  Thus obtain a free resolution $\cdots\to P_n\xrightarrow{d_n}P_{n-1}\to\cdots\to P_2\xrightarrow{d_2}R^m\xrightarrow{d_1} R^n\xrightarrow{\pi} M\to 0$. Localize this resolution to obtain the exact sequence $\cdots\to D^{-1}P_n\xrightarrow{D^{-1}d_n}D^{-1}P_{n-1}\to\cdots\to D^{-1}P_2\xrightarrow{D^{-1}d_2}D^{-1}R^m\xrightarrow{D^{-1}d_1} D^{-1}R^n\xrightarrow{D^{-1}\pi} D^{-1}M\to 0$. We saw earlier that the localization of a free $R$-module is a free $D^{-1}R$-module, so we obtain a free resolution of $D^{-1}M$.
\end{enumerate}
\end{document}