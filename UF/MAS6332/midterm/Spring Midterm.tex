\documentclass[11pt]{article}

% packages
\usepackage{physics}
% margin spacing
\usepackage[top=1in, bottom=1in, left=0.5in, right=0.5in]{geometry}
\usepackage{hanging}
\usepackage{amsfonts, amsmath, amssymb, amsthm}
\usepackage{systeme}
\usepackage[none]{hyphenat}
\usepackage{fancyhdr}
\usepackage[nottoc, notlot, notlof]{tocbibind}
\usepackage{graphicx}
\graphicspath{{./images/}}
\usepackage{float}
\usepackage{siunitx}
\usepackage{esint}
\usepackage{cancel}
\usepackage{enumitem}
\usepackage{quiver}

% permutations (second line is for spacing)
\usepackage{permute}
\renewcommand*\pmtseparator{\,}

% colors
\usepackage{xcolor}
\definecolor{p}{HTML}{FFDDDD}
\definecolor{g}{HTML}{D9FFDF}
\definecolor{y}{HTML}{FFFFCF}
\definecolor{b}{HTML}{D9FFFF}
\definecolor{o}{HTML}{FADECB}
%\definecolor{}{HTML}{}

% \highlight[<color>]{<stuff>}
\newcommand{\highlight}[2][p]{\mathchoice%
  {\colorbox{#1}{$\displaystyle#2$}}%
  {\colorbox{#1}{$\textstyle#2$}}%
  {\colorbox{#1}{$\scriptstyle#2$}}%
  {\colorbox{#1}{$\scriptscriptstyle#2$}}}%

% header/footer formatting
\pagestyle{fancy}
\fancyhead{}
\fancyfoot{}
\fancyhead[L]{MAS6332 Algebra}
\fancyhead[C]{Midterm}
\fancyhead[R]{Sai Sivakumar}
\fancyfoot[R]{\thepage}
\renewcommand{\headrulewidth}{1pt}

% paragraph indentation/spacing
\setlength{\parindent}{0cm}
\setlength{\parskip}{10pt}
\renewcommand{\baselinestretch}{1.25}

% extra commands defined here
\newcommand{\br}[1]{\left(#1\right)}
\newcommand{\sbr}[1]{\left[#1\right]}
\newcommand{\cbr}[1]{\left\{#1\right\}}

\newcommand{\dprime}{\prime\prime}

% bracket notation for inner product
\usepackage{mathtools}

\DeclarePairedDelimiterX{\abr}[1]{\langle}{\rangle}{#1}

\DeclareMathOperator{\Span}{span}
\DeclareMathOperator{\nullity}{nullity}
\DeclareMathOperator\Aut{Aut}
\DeclareMathOperator\Inn{Inn}
\DeclareMathOperator{\Orb}{Orb}
\DeclareMathOperator{\lcm}{lcm}
\DeclareMathOperator{\Hol}{Hol}
\DeclareMathOperator{\Jac}{Jac}
\DeclareMathOperator{\rad}{rad}
\DeclareMathOperator{\Tor}{Tor}
\DeclareMathOperator{\End}{End}
\DeclareMathOperator{\Gal}{Gal}
\DeclareMathOperator{\Nat}{Nat}
\DeclareMathOperator{\Frac}{Frac}
\DeclareMathOperator{\id}{id}
\DeclareMathOperator{\im}{im}
\DeclareMathOperator{\Hom}{Hom}
\DeclareMathOperator{\Ext}{Ext}
\DeclareMathOperator{\aug}{aug}

% set page count index to begin from 1
\setcounter{page}{1}

\begin{document}
\section{Flatness and Exact Sequences}
\begin{enumerate}[label=(\alph*)]
    \item ChatGPT claims that it is sufficient to prove that the short exact sequence \[0\to N\to M \to M/N\to 0\] splits. While this sounds reasonable since the direct sum of flat modules is also flat, we can already find a counterexample when $R= \mathbb{Z}$.
    
    Consider $\mathbb{Q}$ as a $\mathbb{Z}$-module and let $F = \mathbb{ZQ}$ be the free $\mathbb{Z}$-module with elements of $\mathbb{Q}$ as generators. Then with $p\colon F\to \mathbb{Q}$ the projection map we have the short exact sequence \[0\to \ker p\to F\xrightarrow{p}\mathbb{Q}\to 0.\] I do not actually remember proving this anywhere in any of my prior classes but in my topology class we take for granted that submodules of free $\mathbb{Z}$-modules are free (there is some topological proof of this that I do not remember; I also remember reading somewhere that submodules of modules over PIDs are free). Then $\ker p, F$ are free $\mathbb{Z}$-modules so they are flat, and $\mathbb{Q}$ we saw was flat also. But this short exact sequence does not split since $\mathbb{Q}$ is not a projective $\mathbb{Z}$-module, it cannot be a direct summand of a free module. So the approach ChatGPT has for the proof is not correct.

    Aside from this, it claims that the original short exact sequence \[0\to N\to M \to M/N\to 0\] splits anyways since $N$ is flat, which is a non sequitur since we already found a counterexample. For some reason ChatGPT does not stop here as this would have ``proved'' its claim: it attempts to produce yet again another section, and attempts to compose these two sections (which it cannot do since both sections have the same domain and codomain). Only after ``composing'' these two sections that it got from nowhere does it conclude the proof. This proof does not work, but it is impressive that it is written in the tone of a mathematical proof and uses mathematical language in a way that is ``right'' but not logically correct (e.g. it knows what it means for a short exact sequence to split but doesn't know that the above sequence splitting does not follow from either $N$ or $M/N$ being flat).
    \item \begin{proof}
        With $N$, $M/N$ flat over $R$ we have for all $n\geq 1$ and any $R$-module $B$ that $\Tor_n^R(N,B) = 0$ and $\Tor_n^R(M/N,B) = 0$ (by Proposition 16 from DF, chapter 17.1). Then if $D$ is any $R$-module, in the long exact sequence for homology  \[\cdots\to \Tor_2^R(M/N,D) =0\to \Tor_1^R(N,D) = 0\to \Tor_1^R(M,D)\to \Tor_1^R(M/N,D) = 0\to\cdots\to 0\] we find by exactness that $ \Tor_1^R(M,D) = 0$ also. Since $D$ was any $R$-module, it follows that $M$ is flat (again by Proposition 16).
    \end{proof}
    \item The familiar short exact sequence of $\mathbb{Z}$-modules \[0\to \mathbb{Z}\xrightarrow{\cdot 2}\mathbb{Z}\to \mathbb{Z}/2\mathbb{Z}\to 0\] satisfies the requirements. [Here $N = 2\mathbb{Z}\cong\mathbb{Z}, M = \mathbb{Z}$, and $M/N = \mathbb{Z}/2\mathbb{Z}$.] Both copies of $\mathbb{Z}$ are flat since $\mathbb{Z}$ is a free $\mathbb{Z}$-module, but $\mathbb{Z}/2\mathbb{Z}$ is not flat as we have seen before (an example from the text or class).
   \end{enumerate}
\section{Computing $\Ext$ and $\Tor$ over $F[t]$}
\begin{enumerate}[label=(\alph*)]
    \item 
    \item 
    \item 
   \end{enumerate}
\section{Group Cohomology is Torsion}
\begin{enumerate}[label=(\alph*)]
    \item \begin{proof}
        Let $f$ be a cocycle so that for all $g,h\in G$ we have $f(g) = f(gh)-gf(h)$. Then by fixing $g$ and summing over $h\in G$ we have \[(nf)(g) = nf(g) = \sum_{h\in G} f(g) = \sum_{h\in G} f(gh)+ g\sum_{h\in G} (-f(h)) = g\sum_{h\in G} (-f(h)) - \sum_{h\in G} (-f(h))\] since $\abs{G} = n$ and as $h$ runs over the elements of $G$, so does $gh$ for fixed $g$. The above equation holds for any $g\in G$. But $\sum_{h\in G} (-f(h))$ is an element of $A$, so it follows that $nf$ is a coboundary. Since $f$ was arbitrary, $nH^1(G,A) = 0$.
    \end{proof}
    \item 
    \item 
\end{enumerate}
\section{Feedback}
\begin{itemize}
    \item (Optional) How was this? Have you found particular topics hard/easy interesting/boring?
    Anything else you want to tell me?
\end{itemize}
\end{document}