\documentclass[11pt]{article}

% packages
\usepackage{physics}
% margin spacing
\usepackage[top=1in, bottom=1in, left=0.5in, right=0.5in]{geometry}
\usepackage{hanging}
\usepackage{amsfonts, amsmath, amssymb, amsthm}
\usepackage{systeme}
\usepackage[none]{hyphenat}
\usepackage{fancyhdr}
\usepackage[nottoc, notlot, notlof]{tocbibind}
\usepackage{graphicx}
\graphicspath{{./images/}}
\usepackage{float}
\usepackage{siunitx}
\usepackage{esint}
\usepackage{cancel}
\usepackage{enumitem}
\usepackage{quiver}

% permutations (second line is for spacing)
\usepackage{permute}
\renewcommand*\pmtseparator{\,}

% colors
\usepackage{xcolor}
\definecolor{p}{HTML}{FFDDDD}
\definecolor{g}{HTML}{D9FFDF}
\definecolor{y}{HTML}{FFFFCF}
\definecolor{b}{HTML}{D9FFFF}
\definecolor{o}{HTML}{FADECB}
%\definecolor{}{HTML}{}

% \highlight[<color>]{<stuff>}
\newcommand{\highlight}[2][p]{\mathchoice%
  {\colorbox{#1}{$\displaystyle#2$}}%
  {\colorbox{#1}{$\textstyle#2$}}%
  {\colorbox{#1}{$\scriptstyle#2$}}%
  {\colorbox{#1}{$\scriptscriptstyle#2$}}}%

% header/footer formatting
\pagestyle{fancy}
\fancyhead{}
\fancyfoot{}
\fancyhead[L]{MAS6332 Algebra}
\fancyhead[C]{Homework 5}
\fancyhead[R]{Sai Sivakumar}
\fancyfoot[R]{\thepage}
\renewcommand{\headrulewidth}{1pt}

% paragraph indentation/spacing
\setlength{\parindent}{0cm}
\setlength{\parskip}{10pt}
\renewcommand{\baselinestretch}{1.25}

% extra commands defined here
\newcommand{\br}[1]{\left(#1\right)}
\newcommand{\sbr}[1]{\left[#1\right]}
\newcommand{\cbr}[1]{\left\{#1\right\}}

\newcommand{\dprime}{\prime\prime}

% bracket notation for inner product
\usepackage{mathtools}

\DeclarePairedDelimiterX{\abr}[1]{\langle}{\rangle}{#1}

\DeclareMathOperator{\Span}{span}
\DeclareMathOperator{\nullity}{nullity}
\DeclareMathOperator\Aut{Aut}
\DeclareMathOperator\Inn{Inn}
\DeclareMathOperator{\Orb}{Orb}
\DeclareMathOperator{\lcm}{lcm}
\DeclareMathOperator{\Hol}{Hol}
\DeclareMathOperator{\Jac}{Jac}
\DeclareMathOperator{\rad}{rad}
\DeclareMathOperator{\Tor}{Tor}
\DeclareMathOperator{\End}{End}
\DeclareMathOperator{\Gal}{Gal}
\DeclareMathOperator{\Nat}{Nat}
\DeclareMathOperator{\Frac}{Frac}
\DeclareMathOperator{\id}{id}
\DeclareMathOperator{\im}{im}
\DeclareMathOperator{\Hom}{Hom}
\DeclareMathOperator{\Ext}{Ext}
\DeclareMathOperator{\aug}{aug}

% set page count index to begin from 1
\setcounter{page}{1}

\begin{document}
\subsection*{Graded}
\begin{enumerate}
    \item ($G$-modules with Trivial Cohomology) Let $G$ be a finite group and $A$ a $G$-module. Show that there exists a $G$-module $B$ with an injective map of $G$-modules $A\hookrightarrow B$ such that $H^n(G,B) = 0$ for $n>0$. (We say $B$ is cohomologically trivial for $G$.) Here is a suggested strategy, which is a streamlined version of the arguments in Section 17.2. \begin{itemize}
        \item Take $B = \Hom_{\mathbb{Z}}(\mathbb{Z}G, A)$, with $G$ acting on the codomain.
        \item Use tensor-hom adjunction to show that for any $\mathbb{Z}G$-module $P$, $\Hom_{\mathbb{Z}G}(P,B)\cong \Hom_\mathbb{Z}(P,A)$.
        \item Show $\Ext_{\mathbb{Z}G}^n(\mathbb{Z},B)\cong \Ext_\mathbb{Z}^n(\mathbb{Z},A)$ by applying $\Hom_{\mathbb{Z}G}(-,B)$ to a projective resolution of $\mathbb{Z}G$-modules for $\mathbb{Z}$.
    \end{itemize} \begin{proof}
        With $G$ a finite group and $A$ a $G$-module as above let $B = \Hom_{\mathbb{Z}}(\mathbb{Z}G,A)$, where the $G$-action is pointwise (so $gf$ is given by $(gf)(x) = gf(x)$). By the tensor-hom adjunction we have for any $\mathbb{Z}G$-module $P$ that \[\Hom_{\mathbb{Z}G}(P,B) = \Hom_{\mathbb{Z}G}(P,\Hom_{\mathbb{Z}}(\mathbb{Z}G,A))\cong \Hom_{\mathbb{Z}}(P\otimes_{\mathbb{Z}G}\mathbb{Z}G, A) \cong \Hom_\mathbb{Z}(P,A).\]

        Now let $\cbr{P_i}$ be a projective resolution of $\mathbb{Z}G$-modules for $\mathbb{Z}$: \[\cdots\to P_n\to\cdots\to P_1\to P_0\to \mathbb{Z}\to 0\] Apply the $\Hom_{\mathbb{Z}G}(-,B)$ functor to obtain the cochain complex (removing the first group) \[0\to \Hom_{\mathbb{Z}G}(P_0,B)\xrightarrow{d_1} \Hom_{\mathbb{Z}G}(P_1,B)\xrightarrow{d_2}\cdots\xrightarrow{d_n} \Hom_{\mathbb{Z}G}(P_n,B)\xrightarrow{d_{n+1}}\cdots.\] The cohomology of this complex is given by the groups $\Ext_{\mathbb{Z}G}^i(\mathbb{Z},B) = \ker{d_{i+1}}/\im{d_i}$. Using the tensor-hom adjunction we produce isomorphisms $f_i\colon \Hom_{\mathbb{Z}G}(P_i,B)\to \Hom_\mathbb{Z}(P_i,A)$; produce the cochain complex (is still a cochain complex since the $f_i$ are isomorphisms) \[0\to \Hom_\mathbb{Z}(P_0,A)\xrightarrow{f_1d_1f_0^{-1}} \Hom_\mathbb{Z}(P_1,A)\xrightarrow{f_2d_2f_1^{-1}}\cdots\xrightarrow{f_nd_nf_{n-1}^{-1}} \Hom_\mathbb{Z}(P_n,A)\xrightarrow{f_{n+1}d_{n+1}f_n^{-1}}\cdots.\] The cohomology of this complex is isomorphic to the cohomology of the previous complex given since the $f_i$ are isomorphisms: we have $\ker{d_{i+1}}\cong \ker(f_{i+1}d_{i+1}f_i^{-1})$ and $\im{d_i}\cong \im(f_id_if_{i-1}^{-1})$. So $\Ext_{\mathbb{Z}G}^n(\mathbb{Z},B)\cong \Ext_\mathbb{Z}^n(\mathbb{Z},A)$ as needed. But for $n>0$, $\Ext_\mathbb{Z}^n(\mathbb{Z},A)$ since $\mathbb{Z}$ is a projective (free) $\mathbb{Z}$-module.
    \end{proof}
    \item (17.4.10) Suppose $\mathbb{F}_q$ is a finite field with $G= \Gal(\mathbb{F}_{q^d}/\mathbb{F}_q) = \abr{\sigma_q}$ where $\sigma_q$ is the Frobenius automorphism, and let $N$ be the usual norm element for the cyclic group $G$. \begin{enumerate}
        \item Use Hilbert's Theorem 90 to prove that $|{}_N(\mathbb{F}_{q^d}^\times)| = (q^d-1)/(q-1)$, and deduce that the norm map from $\mathbb{F}_{q^d}$ to $\mathbb{F}_q$ is surjective. \begin{proof}
        
        \end{proof}
        \item Prove that $H^n(G,\mathbb{F}_{q^d}^\times) = 0$ for all $n\geq 1$.
    \end{enumerate}
\end{enumerate}
\subsection*{Additional Problems}
\begin{enumerate}
    % 17.3.11 or Problem 2
    \item (17.3.4) Let $A$ be the Klein $4$-group and let $G = \Aut(A)\cong S_3$ act on $A$ in the natural fashion. Prove that $H^1(G,A) = 0$. [Show that in the semidirect product $E = A\rtimes G$, $G$ is the normalizer of a Sylow $3$-subgroup of $E$. Apply Sylow's theorem to show that all complements to $A$ in $E$ are conjugate.]
    \item (17.4.2) Let $A=\mathbb{Z}/4\mathbb{Z}$ and let $G$ be the cyclic group of order $2$ acting trivially on $A$.\begin{enumerate}
        \item Prove that $\abs{C^2(G,A)} = 2^8$.
        \item Use the coboundary condition to show that $\abs{B^2(G,A)} = 2^3$.
        \item Use the cocycle condition to show that $\abs{Z^2(G,A)} \leq 2^4$. 
        \item Show that $\abs{H^2(G,A)} = 2$ by exhibiting two inequivalent extensions of $G$ by $A$ and their corresponding cocycles.
    \end{enumerate}
\end{enumerate}
\subsection*{Feedback}
\begin{enumerate}
    \item None.
    \item Things seem to be the same I think.
\end{enumerate}
\end{document}