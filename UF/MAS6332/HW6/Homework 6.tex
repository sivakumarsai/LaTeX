\documentclass[11pt]{article}

% packages
\usepackage{physics}
% margin spacing
\usepackage[top=1in, bottom=1in, left=0.5in, right=0.5in]{geometry}
\usepackage{hanging}
\usepackage{amsfonts, amsmath, amssymb, amsthm}
\usepackage{systeme}
\usepackage[none]{hyphenat}
\usepackage{fancyhdr}
\usepackage[nottoc, notlot, notlof]{tocbibind}
\usepackage{graphicx}
\graphicspath{{./images/}}
\usepackage{float}
\usepackage{siunitx}
\usepackage{esint}
\usepackage{cancel}
\usepackage{enumitem}
\usepackage{quiver}

% permutations (second line is for spacing)
\usepackage{permute}
\renewcommand*\pmtseparator{\,}

% colors
\usepackage{xcolor}
\definecolor{p}{HTML}{FFDDDD}
\definecolor{g}{HTML}{D9FFDF}
\definecolor{y}{HTML}{FFFFCF}
\definecolor{b}{HTML}{D9FFFF}
\definecolor{o}{HTML}{FADECB}
%\definecolor{}{HTML}{}

% \highlight[<color>]{<stuff>}
\newcommand{\highlight}[2][p]{\mathchoice%
  {\colorbox{#1}{$\displaystyle#2$}}%
  {\colorbox{#1}{$\textstyle#2$}}%
  {\colorbox{#1}{$\scriptstyle#2$}}%
  {\colorbox{#1}{$\scriptscriptstyle#2$}}}%

% header/footer formatting
\pagestyle{fancy}
\fancyhead{}
\fancyfoot{}
\fancyhead[L]{MAS6332 Algebra}
\fancyhead[C]{Homework 6}
\fancyhead[R]{Sai Sivakumar}
\fancyfoot[R]{\thepage}
\renewcommand{\headrulewidth}{1pt}

% paragraph indentation/spacing
\setlength{\parindent}{0cm}
\setlength{\parskip}{10pt}
\renewcommand{\baselinestretch}{1.25}

% extra commands defined here
\newcommand{\br}[1]{\left(#1\right)}
\newcommand{\sbr}[1]{\left[#1\right]}
\newcommand{\cbr}[1]{\left\{#1\right\}}

\newcommand{\dprime}{\prime\prime}

% bracket notation for inner product
\usepackage{mathtools}

\DeclarePairedDelimiterX{\abr}[1]{\langle}{\rangle}{#1}

\DeclareMathOperator{\Span}{span}
\DeclareMathOperator{\nullity}{nullity}
\DeclareMathOperator\Aut{Aut}
\DeclareMathOperator\Inn{Inn}
\DeclareMathOperator{\Orb}{Orb}
\DeclareMathOperator{\lcm}{lcm}
\DeclareMathOperator{\Hol}{Hol}
\DeclareMathOperator{\Jac}{Jac}
\DeclareMathOperator{\rad}{rad}
\DeclareMathOperator{\Tor}{Tor}
\DeclareMathOperator{\End}{End}
\DeclareMathOperator{\Gal}{Gal}
\DeclareMathOperator{\Nat}{Nat}
\DeclareMathOperator{\Frac}{Frac}
\DeclareMathOperator{\id}{id}
\DeclareMathOperator{\im}{im}
\DeclareMathOperator{\Hom}{Hom}
\DeclareMathOperator{\Ext}{Ext}
\DeclareMathOperator{\aug}{aug}

% set page count index to begin from 1
\setcounter{page}{1}

\begin{document}
\subsection*{Graded}
\begin{enumerate}
    \item (15.1.4) Prove that if $R$ is Noetherian, then so is the ring $R[[x]]$ of formal power series in the variable $x$ with coefficients from $R$ (cf. Exercise 3, Section 7.2). [Mimic the proof of Hilbert's Basis Theorem.] \begin{proof}
        
    \end{proof}
    \item (15.2.9) Prove that for any field $k$ the map $\mathcal{Z}$ in the Nullstellensatz is always surjective and the map $\mathcal{I}$ in the Nullstellensatz is always injective. [Use property (10) of the maps $\mathcal{Z}$ and $\mathcal{I}$ in Section 1.] Give examples (over a field $k$ that is not algebraically closed) where $\mathcal{Z}$ is not injective and $\mathcal{I}$ is not surjective. \begin{proof}
        
    \end{proof}
\end{enumerate}
\subsection*{Additional Problems}
\begin{enumerate}
    %\item (15.1.24)
    \item (15.1.28) Prove that if $V$ and $W$ are affine algebraic sets, then so is $V\times W$ and $k[V\times W]\cong k[V]\otimes_kk[W]$. \begin{proof}
        Let $V = \mathcal{Z}(S)\subseteq \mathbb{A}^n$ and $W = \mathcal{Z}(T)\subseteq \mathbb{A}^m$, where $S\subseteq k[x_1,\dots,x_n]$ and $T\subseteq k[y_1,\dots,y_m]$. Then $V\times W \subseteq \mathbb{A}^{n+m}$ is the vanishing set of $S\cup T$ where now we view $S,T\subseteq k[x_1,\dots,x_n,y_1,\dots,y_m]$
    \end{proof}
    \item (15.2.3) Prove that the intersection of two radical ideals is again a radical ideal. \begin{proof}
        
    \end{proof}
    %\item (15.2.26)
\end{enumerate}
\subsection*{Feedback}
\begin{enumerate}
    \item None.
    \item Things seem to be the same I think.
\end{enumerate}
\end{document}