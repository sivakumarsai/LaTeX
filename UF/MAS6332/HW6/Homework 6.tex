\documentclass[11pt]{article}

% packages
\usepackage{physics}
% margin spacing
\usepackage[top=1in, bottom=1in, left=0.5in, right=0.5in]{geometry}
\usepackage{hanging}
\usepackage{amsfonts, amsmath, amssymb, amsthm}
\usepackage{systeme}
\usepackage[none]{hyphenat}
\usepackage{fancyhdr}
\usepackage[nottoc, notlot, notlof]{tocbibind}
\usepackage{graphicx}
\graphicspath{{./images/}}
\usepackage{float}
\usepackage{siunitx}
\usepackage{esint}
\usepackage{cancel}
\usepackage{enumitem}
\usepackage{quiver}

% permutations (second line is for spacing)
\usepackage{permute}
\renewcommand*\pmtseparator{\,}

% colors
\usepackage{xcolor}
\definecolor{p}{HTML}{FFDDDD}
\definecolor{g}{HTML}{D9FFDF}
\definecolor{y}{HTML}{FFFFCF}
\definecolor{b}{HTML}{D9FFFF}
\definecolor{o}{HTML}{FADECB}
%\definecolor{}{HTML}{}

% \highlight[<color>]{<stuff>}
\newcommand{\highlight}[2][p]{\mathchoice%
  {\colorbox{#1}{$\displaystyle#2$}}%
  {\colorbox{#1}{$\textstyle#2$}}%
  {\colorbox{#1}{$\scriptstyle#2$}}%
  {\colorbox{#1}{$\scriptscriptstyle#2$}}}%

% header/footer formatting
\pagestyle{fancy}
\fancyhead{}
\fancyfoot{}
\fancyhead[L]{MAS6332 Algebra}
\fancyhead[C]{Homework 6}
\fancyhead[R]{Sai Sivakumar}
\fancyfoot[R]{\thepage}
\renewcommand{\headrulewidth}{1pt}

% paragraph indentation/spacing
\setlength{\parindent}{0cm}
\setlength{\parskip}{10pt}
\renewcommand{\baselinestretch}{1.25}

% extra commands defined here
\newcommand{\br}[1]{\left(#1\right)}
\newcommand{\sbr}[1]{\left[#1\right]}
\newcommand{\cbr}[1]{\left\{#1\right\}}

\newcommand{\dprime}{\prime\prime}

% bracket notation for inner product
\usepackage{mathtools}

\DeclarePairedDelimiterX{\abr}[1]{\langle}{\rangle}{#1}

\DeclareMathOperator{\Span}{span}
\DeclareMathOperator{\nullity}{nullity}
\DeclareMathOperator\Aut{Aut}
\DeclareMathOperator\Inn{Inn}
\DeclareMathOperator{\Orb}{Orb}
\DeclareMathOperator{\lcm}{lcm}
\DeclareMathOperator{\Hol}{Hol}
\DeclareMathOperator{\Jac}{Jac}
\DeclareMathOperator{\rad}{rad}
\DeclareMathOperator{\Tor}{Tor}
\DeclareMathOperator{\End}{End}
\DeclareMathOperator{\Gal}{Gal}
\DeclareMathOperator{\Nat}{Nat}
\DeclareMathOperator{\Frac}{Frac}
\DeclareMathOperator{\id}{id}
\DeclareMathOperator{\im}{im}
\DeclareMathOperator{\Hom}{Hom}
\DeclareMathOperator{\Ext}{Ext}
\DeclareMathOperator{\aug}{aug}

% set page count index to begin from 1
\setcounter{page}{1}

\begin{document}
\subsection*{Graded}
\begin{enumerate}
    \item (15.1.4) Prove that if $R$ is Noetherian, then so is the ring $R[[x]]$ of formal power series in the variable $x$ with coefficients from $R$ (cf. Exercise 3, Section 7.2). [Mimic the proof of Hilbert's Basis Theorem.] \begin{proof}
        
    \end{proof}
    \item (15.2.9) Prove that for any field $k$ the map $\mathcal{Z}$ in the Nullstellensatz is always surjective and the map $\mathcal{I}$ in the Nullstellensatz is always injective. [Use property (10) of the maps $\mathcal{Z}$ and $\mathcal{I}$ in Section 1.] Give examples (over a field $k$ that is not algebraically closed) where $\mathcal{Z}$ is not injective and $\mathcal{I}$ is not surjective. \begin{proof}
        
    \end{proof}
\end{enumerate}
\subsection*{Additional Problems}
\begin{enumerate}
    %\item (15.1.24)
    \item (15.1.28) Prove that if $V$ and $W$ are affine algebraic sets, then so is $V\times W$ and $k[V\times W]\cong k[V]\otimes_kk[W]$. \begin{proof}
        Let $V = \mathcal{Z}(S)\subseteq \mathbb{A}^n$ and $W = \mathcal{Z}(T)\subseteq \mathbb{A}^m$, where $S\subseteq k[x_1,\dots,x_n]$ and $T\subseteq k[y_1,\dots,y_m]$. Then $V\times W \subseteq \mathbb{A}^{n+m}$ is the vanishing set of $S\cup T$ where now we view $S,T\subseteq k[x_1,\dots,x_n,y_1,\dots,y_m]$: Certainly for any $v\times w\in V\times W$ we have that for $f\in S\cup T$ $f(v\times w) = 0$ since $f$ is in one of $S$ or $T$ and $v\in V, w\in W$. Conversely, if $f(v\times w) = 0$ for all $f\in S\cup T$ then $v\in V$ and $w\in W$ since $f$ is in one of $S$ or $T$ and we may restrict evaluating $f$ at $v\times w$ to one of the factors of $v\times w$. If we take any $f\in S\subset S\cup T$, see that $f(v\times w) = f(v) = 0$ so that $v\in V$; similarly by taking any $f\in T\subset S\cup T$ find that $w\in W$. It follows that $V\times W = \mathcal{Z}(S\cup T)\subseteq \mathbb{A}^{n+m}$ where $S\cup T$ is viewed as a subset of $k[x_1,\dots,x_n,y_1,\dots,y_m]$.

        We define the map $\Psi\colon k[V]\otimes_kk[W]\to k[V\times W]$ by defining it on generators. Specifically $k[V]$ and $k[W]$ are vector spaces with bases $\cbr{f_i}$ and $\cbr{g_j}$ respectively. It follows that $\cbr{f_i\otimes g_j}$ form a basis for $k[V]\otimes_kk[W]$. Let $\Psi(f_i\otimes g_j) = f_ig_j$ where the product on the right hand side means to multiply pointwise. Extend by linearity. It is also clear that this map respects the multiplication $(f\otimes g)(s\otimes t) = fs\otimes gt\mapsto fsgt$. Psi is also well defined since the product of any function with any function which is zero on $V$ or $W$ (i.e. zero on $V\times W$) is still zero.

        The mapping $\Psi$ is injective: if $\sum a_{ij}f_i\otimes g_j \mapsto \sum a_{ij}f_ig_j = 0$ then for all $v\in V$ we have $\sum a_{ij}f_i(v)g_j = 0$ on $W$. By linear independence we must have that $\sum a_{ij}f_i = 0$ on $V$ so that again by linear independence we must have $a_{ij} = 0$ for each $i,j$ as needed. The mapping $\Psi$ is surjective since any monomial in $k[V\times W]$ is the product of monomials in $k[V]$ and $k[W]$, of which we can form preimages of under $\Psi$: for any monomial $a\prod_ix_i\prod_jy_j$ a preimage is $a\prod_ix_i\otimes \prod_jy_j$ and by linearity we obtain surjectivity. Hence $\Psi\colon k[V]\otimes_kk[W]\to k[V\times W]$ is an isomorphism of algebras.
    \end{proof}
    \item (15.2.3) Prove that the intersection of two radical ideals is again a radical ideal. \begin{proof}
        For radical ideals $I,J$ of $R$ we have that the containment $I\cap J\subseteq \rad (I\cap J)$ always holds. Suppose that $r\in \rad(I\cap J)$ so that there is some power $r^k$ in both $I$ and $J$. But since $I$ and $J$ are radical it follows that $r$ is in $I$ and $J$ so that $r\in I\cap J$ also. In general we should have that $\rad I\cap \rad J = \rad (I\cap J)$ for ideals $I,J$ of $R$.
    \end{proof}
    %\item (15.2.26)
\end{enumerate}
\subsection*{Feedback}
\begin{enumerate}
    \item None.
    \item Things seem to be the same I think.
\end{enumerate}
\end{document}