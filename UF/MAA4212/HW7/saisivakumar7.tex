\documentclass[12pt]{amsart}

\textwidth = 6.2 in
\textheight = 8.5 in
\oddsidemargin = 0.0 in
\evensidemargin = 0.0 in
\topmargin = 0.0 in
\headheight = 0.0 in
\headsep = 0.3 in
\parskip = 0.05 in
\parindent = 0.3 in

\usepackage{enumerate}
\usepackage{amsmath}
\usepackage{color}
\def\cc{\color{blue}}
\usepackage[normalem]{ulem}
\usepackage{amsfonts, amsmath, amssymb, amsthm}
\usepackage{systeme}
\usepackage[none]{hyphenat}
\usepackage{graphicx}
\graphicspath{{./images/}}
\usepackage{esint}
\usepackage{cancel}
\usepackage{physics}

\title{Homework 7}
\author{Sai Sivakumar}

\newtheorem{theorem}            {Theorem}[section]
\newtheorem{proposition}        [theorem]{Proposition}

\newcommand{\RR}{\mathbb{R}}
\newcommand{\NN}{\mathbb{N}}
\newcommand{\QQ}{\mathbb{Q}}

\begin{document}
\maketitle

Let $X\ne \emptyset$ be a metric space and let $C_b(X,\RR)$ denote the
 metric space of continuous bounded real-valued functions
 on $X$ with the metric
\[
 d_\infty(f,g)=\|f-g\|_\infty = \sup\{|f(x)-g(x)|:x\in X\}
\]
for $f,g\in C_b(X,\RR).$  It is straightforward to verify
 that $C_b(X,\RR)$ contains the constant functions and
 is closed under sums and products. 

Let $p(t)=\sum_{j=0}^d p_jt^j$ be a polynomial
 and define $\Psi_p:C_b(X,\RR)\to C_b(X,\RR)$ by 
\[
 \Psi_p(f)=p\circ f.
\]
 Show $\Psi_p$ is continuous. [Suggestion: Recall
 $p$ is uniformly continuous on compact subsets
 of $\RR$ and show, for a fixed $f\in C_b(X,\RR)$
 and $\eta>0,$ there is an interval $[a,b]$ such that
 if $d_\infty(f,g)<\eta$ then $f(X),g(X)\subseteq [a,b].$]


\bigskip

\begin{proof}
Let $f\in C_b(X,\mathbb{R})$ be given. We show that $\Psi_p$ is continuous at $f$. To that end, let $\varepsilon > 0$ be given. Since $f\in C_b(X,\mathbb{R})$, we have that $f(X)$ is a bounded subset of $R$, meaning $f(X)$ is contained in some interval $[a,b]$. For any $h\in C_b(X,\mathbb{R})$ satisfying $d_\infty(f,h)<\varepsilon$, we have for any $x\in X$ that $\abs{f(x)-h(x)}\leq d_\infty(f,h) < \varepsilon$.

It follows that $f(x)-\varepsilon < h(x) < f(x) + \varepsilon$, and with $x$ arbitrary we have that $h(X)\subseteq [a-\varepsilon, b+\varepsilon]$. Since $[a,b]\subseteq [a-\varepsilon, b+\varepsilon]$, we also have that $f(X)\subseteq [a-\varepsilon, b+\varepsilon]$. 

With $p$ uniformly continuous on $[a-\varepsilon, b+\varepsilon]$, there exists $\delta '$ such that if $\abs{x-y}<\delta '$ for $x,y\in [a-\varepsilon, b+\varepsilon]$, then $\abs{p(x)-p(y)}<\varepsilon$.

Choose $\delta = \min\{\varepsilon, \delta '\}$. Suppose that $d_\infty(f,g)<\delta$, so that $f(X),g(X)\subseteq [a-\varepsilon, b+\varepsilon]$ and for any $x\in X$ we have $\abs{f(x)-g(x)}\leq d_\infty(f,g)< \delta '$, and note that $f(x),g(x)\in [a-\varepsilon, b+\varepsilon]$. Then by the uniform continuity of $p$ on $[a-\varepsilon, b+\varepsilon]$ we have that $\abs{(p\circ f)(x) -  (p\circ g)(x)}< \varepsilon$, and since $x$ was arbitrary, it follows that $d_\infty(p\circ f, p\circ g) = d_\infty(\Psi_p(f), \Psi_p(g))< \varepsilon$ ($\varepsilon$ is a uniform upper bound). 

Since $f$ was arbitrary, it follows that $\Psi_p$ is continuous on $C_b(X,\mathbb{R})$.
\end{proof}
\end{document}