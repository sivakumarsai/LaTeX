\documentclass[12pt]{amsart}

\textwidth = 6.2 in
\textheight = 8.5 in
\oddsidemargin = 0.0 in
\evensidemargin = 0.0 in
\topmargin = 0.0 in
\headheight = 0.0 in
\headsep = 0.3 in
\parskip = 0.05 in
\parindent = 0.3 in

\usepackage{enumerate}
\usepackage{amsmath}
\usepackage{color}
\def\cc{\color{blue}}
\usepackage[normalem]{ulem}
\usepackage{amsfonts, amsmath, amssymb, amsthm}
\usepackage{systeme}
\usepackage[none]{hyphenat}
\usepackage{graphicx}
\graphicspath{{./images/}}
\usepackage{esint}
\usepackage{cancel}
\usepackage{physics}

\title{Homework 3}
\author{Sai Sivakumar}

\newtheorem{theorem}            {Theorem}[section]
\newtheorem{proposition}        [theorem]{Proposition}

\newcommand{\RR}{\mathbb{R}}
\newcommand{\NN}{\mathbb{N}}
\newcommand{\QQ}{\mathbb{Q}}
\DeclareMathOperator{\Bd}{Bd}

\begin{document}
\maketitle

\begin{enumerate}\itemsep=10pt
 \item What are the boundaries of the following sets? No proof required.
  \begin{enumerate}\itemsep=7pt
   \item $\QQ$ as subset of the metric space $\RR.$
   \item $\QQ$ as a subset of the metric space $\QQ$ (as a subspace of $\RR$).
   %\item A nonempty subset $E$ of a discrete metric space $X$.
   \item $S=\{(x,\sin(\frac1x)): x\in (0,1)\}\subseteq \RR^2$ as a subset of
     the Euclidean space $\RR^2.$ Note, $S$ is the graph of the function
   $f:(0,1)\to\RR$ defined by $f(x)=\sin(\frac1x).$
 \end{enumerate}


\item Give an example of a subset $E$ of a metric space $X$ such that
 $\partial E^\circ \ne \partial E.$ No proof required, but you must
 provide some justification for your identification of both $\partial E^\circ$
 and $\partial E.$

\item Suppose $X=(X,d)$ be a metric space and $E$ a subset of $X.$ Show
 if $E$ is both open and closed, then $\partial E=\emptyset.$ Is
 the converse true? Proof or counterexample.
\end{enumerate}
\vspace*{2em}
\begin{enumerate}
    \item Boundaries: \begin{enumerate}
        \item $\partial \mathbb{Q}  = \mathbb{R}$
        \item $\partial \mathbb{Q} = \emptyset$ (viewing $\mathbb{Q}$ as a subspace of $\mathbb{R}$)
        \item $\partial S = S\cup \{(0,t)\colon t\in [-1,1]\}$
    \end{enumerate}
    \item Take $E = S$ and $X = \mathbb{R}^2$, with $S$ as defined above, so that $\partial E^\circ = \emptyset \neq S\cup \{(0,t)\colon t\in [-1,1]\} = \partial E$. The interior of $E$ is empty, because any open set containing at least one point of $E$ intersects nontrivially with $\mathbb{R}^2\setminus S$. The boundary of $E$ is the boundary of $S$, which is the set of all points for which every open set containing these points intersect nontrivially with $S$ and $\mathbb{R}^2\setminus S$.
    \item A subset of a metric space is open and closed if and only if its boundary is the empty set. \begin{proof}
        Suppose $X$ is a metric space and let $E$ be a subset of $X$.

        Suppose $E$ is open and closed so that $X\setminus E$ is open. Then $E^\circ = E$ and $(X\setminus E)^\circ = X\setminus E$, from which it follows that the boundary of $E$ is empty since $X$ is the disjoint union of the interior, boundary, and exterior of $E$.

        Conversely, suppose $\partial E$ is empty. This means that there are no open sets in $X$ which intersect nontrivially with $E$ and simultaneously intersect nontrivially with $X\setminus E$. It follows that any open set in $X$ is either contained in $E$ or is contained in $X\setminus E$. Hence every point of $E$ has a neighborhood contained in $E$; similarly, every point of $X\setminus E$ has a neighborhood contained in $X\setminus E$. We have that $E$ and its complement are open, so that $E$ is both open and closed.
    \end{proof} 
\end{enumerate}
\newpage
Redo:
\begin{enumerate}
  \item Boundaries: \begin{enumerate}
      \item $\partial \mathbb{Q}  = \mathbb{R}$
      \item $\partial \mathbb{Q} = \emptyset$ (viewing $\mathbb{Q}$ as a subspace of $\mathbb{R}$)
      \item $\partial S = S\cup \{(0,t)\colon t\in [-1,1]\}\cup \{(1,\sin(1))\}$
  \end{enumerate}
  \item Take $E = S$ and $X = \mathbb{R}^2$, with $S$ as defined above, so that $\partial E^\circ = \emptyset \neq S\cup \{(0,t)\colon t\in [-1,1]\}\cup \{(1,\sin(1))\} = \partial E$. The interior of $E$ is empty, because any open set containing at least one point of $E$ intersects nontrivially with $\mathbb{R}^2\setminus S$, and the boundary of the empty set is the empty set (since it is closed and open; follows from (3)). The boundary of $E$ is the boundary of $S$, which is the set of all points for which every open set containing these points intersect nontrivially with $S$ and $\mathbb{R}^2\setminus S$.
  \item A subset of a metric space is open and closed if and only if its boundary is the empty set. \begin{proof}
      Suppose $X$ is a metric space and let $E$ be a subset of $X$.

      Suppose $E$ is open and closed so that $X\setminus E$ is open. Then $E^\circ = E$ and $(X\setminus E)^\circ = X\setminus E$, from which it follows that the boundary of $E$ is empty since $X$ is the disjoint union of the interior, boundary, and exterior of $E$.

      Conversely, suppose $\partial E$ is empty. This means that there are no points $p$ in $X$ satisfying the condition that every neighborhood of $p$ intersects nontrivially with $E$ and simultaneously intersects nontrivially with $X\setminus E$. So for any $p\in X$, it follows that there exists a neighborhood of $p$ which either does not intersect $E$ or does not intersect $X\setminus E$. Both cannot happen simultaneously since every neighborhood of $p$ contains $p$, and $p$ is either in $E$ or $X\setminus E$. Hence every point of $E$ has a neighborhood contained in $E$; similarly, every point of $X\setminus E$ has a neighborhood contained in $X\setminus E$. We have that $E$ and its complement are open, so that $E$ is both open and closed.
  \end{proof} 
\end{enumerate}
\end{document}