\documentclass[12pt]{amsart}

\textwidth = 6.2 in
\textheight = 8.5 in
\oddsidemargin = 0.0 in
\evensidemargin = 0.0 in
\topmargin = 0.0 in
\headheight = 0.0 in
\headsep = 0.3 in
\parskip = 0.05 in
\parindent = 0.3 in

\usepackage{enumerate}
\usepackage{amsmath}
\usepackage{color}
\def\cc{\color{blue}}
\usepackage[normalem]{ulem}
\usepackage{amsfonts, amsmath, amssymb, amsthm}
\usepackage{systeme}
\usepackage[none]{hyphenat}
\usepackage{graphicx}
\graphicspath{{./images/}}
\usepackage{esint}
\usepackage{cancel}
\usepackage{physics}

\title{Homework 8}
\author{Sai Sivakumar}

\newtheorem{theorem}            {Theorem}[section]
\newtheorem{proposition}        [theorem]{Proposition}

\newcommand{\RR}{\mathbb{R}}
\newcommand{\NN}{\mathbb{N}}
\newcommand{\QQ}{\mathbb{Q}}

\begin{document}
\maketitle

Define $f:\RR^2\to \RR^2$ by 
\[
 f(x_1,x_2) = (x_1^2-x_2^2-2x_1+1, 2x_1x_2-2x_2).
\]
 For $c\in \RR^2,$ show $f$ is differentiable at $c$ and determine, with proof,
 $Df(c),$ proceeding directly from Definition 7.1.1.
 Determine those $c$ for which  $Df(c)$ fails to be invertible.


\bigskip

\begin{proof}
Fix $c = (c_1 \,\, c_2)^T$, and let \[A = \begin{pmatrix}
    2(c_1-1) & -2c_2 \\ 2c_2 & 2(c_1-1)
\end{pmatrix}.\] For any $h = (h_1\,\,h_2)^T$ we have \begin{align*}
    \norm{f(c+h) - f(c) - Ah} &= \norm{\begin{pmatrix}
        h_1^2 - h_2^2 \\ 2 h_2 h_1
    \end{pmatrix}} \\
    &= h_1^2+h_2^2 = \norm{h}^2,
\end{align*} so that \[\lim_{h\to 0} \frac{\norm{f(c+h) - f(c) - Ah}}{\norm{h}} = \lim_{h\to 0}\norm{h} = 0.\] It follows that $f$ is differentiable at $c$, and its derivative at $c$ is $Df(c) = L_A$.

Observe that $\det(A) = 4(c_1-1)^2 + 4c_2^2$, which is zero if and only if $c_1 = 1$ and $c_2 = 0$ ($\det(A)$ is a sum of squared quantities).

Since $Df(c)$ is not invertible whenever $L_A$ is not invertible, which is whenever $\det(A)$ vanishes, we have that $Df(c)$ is not invertible at only the point $c = (1 \,\, 0)^T$.
\end{proof}
\end{document}