\documentclass[10pt,leqno,twoside]{article}

% packages
\usepackage[alphabetic]{amsrefs}
\usepackage{physics}
% margin spacing
\usepackage[top=1in, bottom=1in, left=1in, right=1in]{geometry}
\usepackage{amsfonts, amsmath, amssymb, amsthm}
\usepackage{fancyhdr}
\usepackage{graphicx}
\graphicspath{{./images/}}
\usepackage{enumitem}
\usepackage{quiver}
% indexing
\usepackage{imakeidx}
\makeindex
\indexsetup{level=\section*,toclevel=section,noclearpage,firstpagestyle=frontmatter}
\usepackage{hyperref}
% \hypersetup{colorlinks=true,linkcolor=blue}
\usepackage[noabbrev]{cleveref}
\crefformat{equation}{equation~#2#1#3}
\crefformat{lemma}{\textrm{lemma}~#2#1#3}

% theorems
\theoremstyle{plain}
\newtheorem{lem}{Lemma}
\newtheorem{lemma}[lem]{Lemma}
\newtheorem{thm}[lem]{Theorem}
\newtheorem{theorem}[lem]{Theorem}
\newtheorem{prop}[lem]{Proposition}
\newtheorem{proposition}[lem]{Proposition}
\newtheorem{cor}[lem]{Corollary}
\newtheorem{corollary}[lem]{Corollary}
\newtheorem{conj}[lem]{Conjecture}
\newtheorem{fact}[lem]{Fact}
\newtheorem{form}[lem]{Formula}

\theoremstyle{definition}
\newtheorem{defn}[lem]{Definition}
\newtheorem{definition/}[lem]{Definition}
\newenvironment{definition}
  {\renewcommand{\qedsymbol}{\textdagger}%
   \pushQED{\qed}\begin{definition/}}
  {\popQED\end{definition/}}
\newtheorem{example}[lem]{Example}
\newtheorem{remark}[lem]{Remark}
\newtheorem{exercise}[lem]{Exercise}
\newtheorem{notation}[lem]{Notation}

\numberwithin{equation}{section}
\numberwithin{lem}{section}

% % colors
% \usepackage{xcolor}
% \definecolor{p}{HTML}{FFDDDD}
% \definecolor{g}{HTML}{D9FFDF}
% \definecolor{y}{HTML}{FFFFCF}
% \definecolor{b}{HTML}{D9FFFF}
% \definecolor{o}{HTML}{FADECB}
% %\definecolor{}{HTML}{}

% % \highlight[<color>]{<stuff>}
% \newcommand{\highlight}[2][p]{\mathchoice%
%   {\colorbox{#1}{$\displaystyle#2$}}%
%   {\colorbox{#1}{$\textstyle#2$}}%
%   {\colorbox{#1}{$\scriptstyle#2$}}%
%   {\colorbox{#1}{$\scriptscriptstyle#2$}}}%

% header/footer formatting
\fancypagestyle{frontmatter}{
    \fancyhf{}
    \pagestyle{fancy}
    \fancyhead[LE,RO]{\thepage}
}
\fancypagestyle{body}{
    \pagestyle{fancy}
    \fancyhead[LO,RE]{\nouppercase{\rightmark}}
    \fancyhead[LE,RO]{\thepage}
}
\renewcommand{\headrulewidth}{1pt}

% paragraph indentation/spacing
\setlength{\parindent}{0pt}
\setlength{\parskip}{6pt}
\renewcommand{\baselinestretch}{1.00}

% extra commands defined here (rarely used)
\newcommand{\br}[1]{\left(#1\right)}
\newcommand{\sbr}[1]{\left[#1\right]}
\newcommand{\cbr}[1]{\left\{#1\right\}}

% bracket notation for inner product
\usepackage{mathtools}

\DeclarePairedDelimiterX{\abr}[1]{\langle}{\rangle}{#1}

% new commands
\newcommand{\textib}[1]{\textbf{\textit{#1\index{#1}}}} % adds to index
\DeclareMathOperator{\Mat}{M}
\DeclareMathOperator{\GL}{GL}
\DeclareMathOperator{\SL}{SL}
\newcommand{\smod}[1]{\;(\bmod\; #1)}
\newcommand{\smallabcd}{\big(\!\begin{smallmatrix}
    a & b \\ c & d
\end{smallmatrix}\!\big)}
\newcommand{\abcd}{\begin{pmatrix}
    a & b \\ c & d
\end{pmatrix}}
\newcommand{\slz}{\SL_2(\mathbb{Z})}
\newcommand{\glr}{\GL_2(\mathbb{R})}
\newcommand{\glrp}{\GL_2^+(\mathbb{R})}
\newcommand{\glqp}{\GL_2^+(\mathbb{Q})}
\newcommand{\old}{\text{old}}
\newcommand{\new}{\text{new}}
\DeclareMathOperator{\sgn}{sgn}
% \DeclareMathOperator{\Span}{span}
% \DeclareMathOperator{\nullity}{nullity}
% \DeclareMathOperator\Aut{Aut}
% \DeclareMathOperator\Inn{Inn}
% \DeclareMathOperator{\Orb}{Orb}
\DeclareMathOperator{\lcm}{lcm}
% \DeclareMathOperator{\Hol}{Hol}
% \DeclareMathOperator{\Jac}{Jac}
% \DeclareMathOperator{\rad}{rad}
% \DeclareMathOperator{\Tor}{Tor}
% \DeclareMathOperator{\End}{End}
% \DeclareMathOperator{\Gal}{Gal}
% \DeclareMathOperator{\Nat}{Nat}
% \DeclareMathOperator{\Frac}{Frac}
\DeclareMathOperator{\id}{id}
% \DeclareMathOperator{\im}{im}
% \DeclareMathOperator{\Hom}{Hom}
% \DeclareMathOperator{\Ext}{Ext}
% \DeclareMathOperator{\aug}{aug}

% indices
\setcounter{page}{1}
\setcounter{section}{-1}

% array column and row separation
\arraycolsep = 2pt
\renewcommand{\arraystretch}{.8}

% temporaries for editing
\usepackage{color}
\newcommand{\tbd}{{\Huge\color{red}{\textib{TO DO}}}}
\newcommand{\sai}[1]{\textcolor{red}{#1}}

\begin{document}
\begin{titlepage}
    \begin{center}
        \vspace*{4em}
        {\Large\textbf{Strong multiplicity one for classical modular forms}}

        \vspace{12em}
        \includegraphics[scale=0.14]{uf.png}

        \vspace{14em}
        Sai Sivakumar

        Undergraduate Honors Thesis -- Spring 2024\\
        Department of Mathematics, University of Florida
    \end{center}
\end{titlepage}
\pagestyle{frontmatter}
\section*{Acknowledgements}
\newpage\tableofcontents

\newpage\section*{Introduction}
\addcontentsline{toc}{section}{Introduction}

\newpage\section{Preliminaries} \pagestyle{body} In this section, we collect the definitions and results needed to define modular forms.
\subsection{The modular group and congruence subgroups}
Let $R$ be a unital ring. The \textib{general linear group} $\GL_n(R)$ is the group of $n\times n$ invertible matrices with entries from $R$; that is, matrices with unit determinant, under matrix multiplication. The \textib{special linear group} $\SL_n(R)$ is the (normal) subgroup of $\GL_n(R)$ whose matrices have determinant $1_R$.

\begin{definition}
    The \textib{modular group} $\slz$ is the group of $2\times 2$ integer-valued matrices with determinant $1$, and is a subgroup of $\glr$.
\end{definition} 
% Equivalently, the elements of $\slz$ are the integer-valued matrices $\big(\!\begin{smallmatrix}
%     a & b \\ c & d
% \end{smallmatrix}\!\big)$ with $\gcd(c,d) = 1$, since $ad-bc = 1$ is equivalent to $c$ and $d$ being coprime. 
It is well known (e.g., \cite{serre} Chapter 7, Section 1.2, Theorem 2) that 
\begin{equation}\label{eqn: generators modular group}
    \slz = \left\langle\begin{pmatrix}
        1 & 1 \\ 0 & 1
    \end{pmatrix}, \begin{pmatrix}
        0 & -1 \\ 1 & 0
    \end{pmatrix}\right\rangle, 
\end{equation}
and that $\slz$ acts on the Riemann sphere $\widehat{\mathbb{C}} = \mathbb{C}\cup \{\infty\}$ by fractional linear transformations
\[\abcd(z) = \frac{az + b}{cz + d}.\] For $c\neq 0$, $-d/c$ is sent to $\infty$ and $\infty$ is sent to $a/c$; if $c = 0$ then $\infty$ is sent to $\infty$. We check that this action defines a group action: It is clear that the identity matrix sends $z$ to itself, and that 
\[\begin{pmatrix}
    q & r \\ s & t
\end{pmatrix}\abcd(z) = \begin{pmatrix}
    q & r \\ s & t
\end{pmatrix}\Big(\frac{az+b}{cz+d}\Big) = \frac{q\frac{az+b}{cz+d}+r}{s\frac{az+b}{cz+d}+t} = \frac{(qa+rc)z+qb+rd}{(sa+tb)z+sb+td} = \begin{pmatrix}
    qa+rc & qb+rd \\ sa+tb & sb+td
\end{pmatrix}(z)\] for $\big(\!\begin{smallmatrix}
    q & r \\ s & t
\end{smallmatrix}\!\big),\big(\!\begin{smallmatrix}
    a & b \\ c & d
\end{smallmatrix}\!\big)\in \slz$. Similarly, $\glr$ acts on $\widehat{\mathbb C}$ by fractional linear transformations.
It follows that these transformations are (bi)holomorphic and are automorphisms.
% ; indeed, \[\abcd^{-1}\abcd(z) = \begin{pmatrix}
%     d & -b \\ -c & a
% \end{pmatrix}\Big(\frac{az+b}{cz+d}\Big) = \frac{\frac{d(az+b)}{cz+d}- b}{\frac{-c(az+b)}{cz+d}+a} = \frac{(ad-bc)z}{ad-bc} = z.\]
Furthermore, both $I$ and $-I$ act as the identity on $\widehat{\mathbb{C}}$, so for any $A\in \slz$, the actions of $A$ and $-A$ agree. The generators in \cref{eqn: generators modular group} correspond to the maps
\[z\mapsto z+1\quad\text{and}\quad z\mapsto -1/z.\]
We consider particular subgroups of the modular group. 
\begin{definition}
    For $N$ a positive integer, the \textib{principal congruence subgroup of level $N$} is the subgroup
\[\varGamma(N) = \cbr{\abcd\in \slz: \abcd\equiv \begin{pmatrix}
    1 & 0 \\ 0 & 1
\end{pmatrix}\smod N}\]
with the matrix congruence interepreted as congruence modulo $N$ entrywise.
\end{definition}
Alternatively, $\varGamma(N) = (I+N\Mat_2(\mathbb Z))\cap \slz$. Furthermore, observe that if $N\mid M$, then $\varGamma(M)\subset \varGamma(N)$, since $x \equiv y\smod M$ implies $x\equiv y\smod N$ whenever $N\mid M$.
\begin{lemma}\label{lem: varGamma(N) kernel of projection}
    The principal congruence subgroup $\varGamma(N)$ is the kernel of the natural homomorphism $\slz\to \SL_2(\mathbb{Z}/N\mathbb{Z})$, so $\varGamma(N)$ is normal in $\slz$; furthermore, the natural homomorphism is surjective.
\end{lemma}
\begin{proof}
    It is clear that $\varGamma(N)$ is the kernel of the natural homomorphism $\slz\to \SL_2(\mathbb{Z}/N\mathbb{Z})$, so we prove that this map is surjective. When $N = 1$, the natural homomorphism is the zero map (so $\varGamma(1) = \slz$). Let $N>1$, and consider an element \[\begin{pmatrix}
        \overline a & \overline b \\ \overline c & \overline d
    \end{pmatrix}\in \SL_2(\mathbb{Z}/N\mathbb{Z}),\] so that $\overline{ad-bc} = \overline 1$ (here $\overline{\,\cdot\,}$ denotes reduction modulo $N$). Then for some integer $k$, $ad-bc = 1+kN$. It follows that $1+kN$ is a multiple of $g = \gcd(c,d)$, so that $\gcd(g,N)=1$ and $\gcd(c,d,N)=1$. We show that there exist integers $i,j$ such that $c+iN$ and $d+jN$ are coprime.
    
    % Suppose there is no choice of $i,j$ such that $c+iN$ and $d+jN$ are coprime, so that for fixed $i$, $\gcd(c+iN,d+jN)>1$ for every $j$. By Dirichlet's theorem on arithmetic progressions, the sequence $(d+mN)$ contains infinitely many primes, which yields a contradiction.
    
    If $c\neq 0$, consider a solution $j$ to the system of congruences 
    \[\begin{cases}
        j\equiv 1\smod{p} & p\mid g\\
        j\equiv 0\smod{p} & p\nmid g, p\mid c,
    \end{cases}\] which may be obtained via the Chinese remainder theorem. Then $\gcd(c,d+jN) = 1$ since any prime $p$ dividing $c$ will not divide $d+jN$ with $j$ chosen as above (of primes $p$ dividing $c$, when $p\mid g$, we have $p\nmid N$, and when $p\nmid g$, we have $p\nmid d$). If $c = 0$, then $d\neq 0$ and repeat this argument with $d,i$ in place of $c,j$ respectively.

    With integers $c+iN$ and $d+jN$ coprime, there exist integers $s,t$ with $s(c+iN) + t(d+jN) = 1$. Then \begin{multline*}
        \det\begin{pmatrix}
            a-(k+aj-bi)tN & b+(k+aj-bi)sN \\ c+iN & d+jN
        \end{pmatrix} = ad+ajN-(k+aj-bi)tN(d+jN)\\-bc-biN-(k+aj-bi)sN(c+iN) = 1+kN+ajN-biN-(k+aj-bi)N = 1
    \end{multline*} and 
    \[\begin{pmatrix}
        \overline{a-(k+aj-bi)tN} & \overline{b+(k+aj-bi)sN} \\ \overline{c+iN} & \overline{d+jN}
    \end{pmatrix} = \begin{pmatrix}
        \overline a & \overline b \\ \overline c & \overline d
    \end{pmatrix}\] as needed.
\end{proof}
Therefore, by the first isomorphism theorem, $\slz/\varGamma(N)\cong \SL_2(\mathbb{Z}/N\mathbb{Z})$. 
\begin{lemma}
    The index of $\varGamma(N)$ in $\slz$ is $[\slz: \varGamma(N)] = N^3\prod_{p\mid N}(1-1/p^2)$.
\end{lemma}
\begin{proof}
    We first show that $\abs{\SL_2(\mathbb{Z}/p^e\mathbb{Z})} = p^{3e}(1-1/p^2)$ for a prime $p$ by induction on $e$.

    The determinant is a surjective homomorphism from $\GL_2(\mathbb{Z}/p\mathbb{Z})$ to $(\mathbb{Z}/p\mathbb{Z})^\times\cong \mathbb{Z}/(p-1)\mathbb{Z}$ with kernel $\SL_2(\mathbb{Z}/p\mathbb{Z})$. Observe that $\abs{\GL_2(\mathbb{Z}/p\mathbb{Z})}$ is the number of ordered bases of $(\mathbb{Z}/p\mathbb{Z})^2$, given by $(p^2-1)(p^2-p)$. (The first factor counts the number of valid first basis vectors, and the second factor counts the number of valid second basis vectors after choosing a first basis vector.) Then by the first isomorphism theorem, $\abs{\SL_2(\mathbb{Z}/p\mathbb{Z})} = (p^2-1)(p^2-p)/(p-1) = p(p^2-1) = p^3(1-1/p^2)$.
    
    By \cref{lem: varGamma(N) kernel of projection}, the natural homomorphism $\slz\to \SL_2(\mathbb{Z}/p^e\mathbb{Z})$ is surjective. The surjection $\slz\to \SL_2(\mathbb{Z}/p^e\mathbb{Z})$ is equal to the composition of the natural homomorphisms $\slz\to \SL_2(\mathbb{Z}/p^{e+1}\mathbb{Z})$ and $\SL_2(\mathbb{Z}/p^{e+1}\mathbb{Z})\to \SL_2(\mathbb{Z}/p^e\mathbb{Z})$, from which it follows that $\SL_2(\mathbb{Z}/p^{e+1}\mathbb{Z})\to \SL_2(\mathbb{Z}/p^e\mathbb{Z})$ is surjective also.
    % \[\begin{tikzcd}
    %     {\slz} & {\SL_2(\mathbb{Z}/p^e\mathbb{Z})} \\
    %     {\SL_2(\mathbb{Z}/p^{e+1}\mathbb{Z})}
    %     \arrow[from=2-1, to=1-2]
    %     \arrow[two heads, from=1-1, to=2-1]
    %     \arrow[two heads, from=1-1, to=1-2]
    % \end{tikzcd}\]
    % from which we find that the map $\SL_2(\mathbb{Z}/p^{e+1}\mathbb{Z})\to \SL_2(\mathbb{Z}/p^e\mathbb{Z})$ is surjective.

    Any element $\gamma$ of $\ker(\SL_2(\mathbb{Z}/p^{e+1}\mathbb{Z})\to \SL_2(\mathbb{Z}/p^e\mathbb{Z}))$ is of the form
    \[\begin{pmatrix}
        \overline{1+ip^e} & \overline{rp^e} \\ \overline{sp^e} & \overline{1+jp^e}
    \end{pmatrix},\] for $i,j,r,s\in \{0,\dots,p-1\}$ and $i = p-j$ (so that $\det \gamma = 1$). Hence $\abs{\ker(\SL_2(\mathbb{Z}/p^{e+1}\mathbb{Z})\to \SL_2(\mathbb{Z}/p^e\mathbb{Z}))} = p^3$. Suppose that $\abs{\SL_2(\mathbb{Z}/p^e\mathbb{Z})} = p^{3e}(1-1/p^2)$. Then by the first isomorphism theorem, $\abs{\SL_2(\mathbb{Z}/p^{e+1}\mathbb{Z})} = p^3\abs{\SL_2(\mathbb{Z}/p^e\mathbb{Z})} = p^3p^{3e}(1-1/p^2) = p^{3(e+1)}(1-1/p^2)$ as needed. By induction, $\abs{\SL_2(\mathbb{Z}/p^e\mathbb{Z})} = p^{3e}(1-1/p^2)$ for any $e$.

    There is an isomorphism of matrix groups $\Mat_2(\prod_{i=1}^nR_i)\cong \prod_{i=1}^n\Mat_2(R_i)$ for rings $R_i$ which restricts to the isomorphisms $\GL_2(\prod_{i=1}^nR_i)\cong \prod_{i=1}^n\GL_2(R_i)$ and $\SL_2(\prod_{i=1}^nR_i)\cong \prod_{i=1}^n\SL_2(R_i)$ since $(\prod_{i=1}^nR_i)^\times\cong \prod_{i=1}^nR_i^\times$ and $1_{\prod_{i=1}^nR_i} = \prod_{i=1}^n1_{R_i}$.

    Let $N$ have prime factorization $N = \prod_{p\mid N} p^{e}$. By the Chinese remainder theorem, $\mathbb{Z}/N\mathbb{Z}\cong \prod_{p\mid N}\mathbb{Z}/p^{e}\mathbb{Z}$. It follows that $\SL_2(\mathbb{Z}/N\mathbb{Z})\cong \prod_{p\mid N}\SL_2(\mathbb{Z}/p^{e}\mathbb{Z})$, from which we have \[\abs{\SL_2(\mathbb{Z}/N\mathbb{Z})}= \prod_{p\mid N}\abs{\SL_2(\mathbb{Z}/p^{e}\mathbb{Z})} = \prod_{p\mid N}p^{3e}(1-1/p^2) = N^3\prod_{p\mid N}(1-1/p^2)\] as desired.
\end{proof}

\begin{definition}\label{def: congruence subgroup}
    A subgroup $\varGamma$ of $\slz$ is a \textib{congruence subgroup (of level $N$)} if for some positive integer $N$, $\varGamma(N)\subset \varGamma$.
\end{definition}
Note that congruence subgroups need not be normal in $\slz$. % terminology for congruence subgps, see keith conrad page

Let $\varGamma$ be a level $N$ congruence subgroup. We have $[\slz : \varGamma(N)] = [\slz:\varGamma][\varGamma:\varGamma(N)]$, and since $[\slz : \varGamma(N)]$ is finite, $[\slz:\varGamma]$ must also be finite; that is, congruence subgroups have finite index in $\slz$.

\begin{definition}
    Frequently used congruence subgroups are 
    \begin{align*}
        \varGamma_0(N) &= \cbr{\abcd\in\slz: \abcd\equiv\begin{pmatrix}
            \ast & \ast \\ 0 & \ast
        \end{pmatrix}\smod N}\text{ and}\\
        \varGamma_1(N) &= \cbr{\abcd\in \slz: \begin{pmatrix}
            a & b \\ c & d 
        \end{pmatrix}\equiv \begin{pmatrix}
            1 & \ast \\ 0 & 1
        \end{pmatrix}\smod N},
    \end{align*}
    where $\ast$ denotes unspecified quantities and the matrix congruences are to be taken entrywise.    
\end{definition}
For any positive integer $N$, the inclusions $\varGamma(N)\subset\varGamma_1(N)\subset\varGamma_0(N)\subset\slz$ hold.

\begin{lemma}\label{lem: principal normal in gamma 1}
    The map $\varGamma_1(N)\to \mathbb{Z}/N\mathbb{Z}$ given by $\smallabcd\mapsto \overline b$ is surjective with kernel $\varGamma(N)$. 
\end{lemma}
\begin{proof}
    For $\overline b\in\mathbb{Z}/N\mathbb{Z}$, $\big(\!\begin{smallmatrix}
        1 & b \\ 0 & 1
    \end{smallmatrix}\!\big)$ has unit determinant and maps to $\overline b$ under the above map. It is clear that the kernel of the above map is $\varGamma(N)$.
    % Let $\overline b\in\mathbb{Z}/N\mathbb{Z}$ be given. Let $a,d\equiv 1\smod N$ and $c\equiv 0\smod N$, and write $a= iN+1$, $d = jN+1$, and $c = kN$ for some integers $i,j,k$. If $1= ad-bc = ijN^2 + (i+j)N + 1 - bkN$, then $(ijN+ i + j)/k = b$. Take $k=1$ and reduce modulo $N$ to obtain the equation $i+j\equiv b \smod N$, which has many solutions, for example $i = b$ and $j =0$. These choices for $i,j,k$ produce a preimage of $b\smod N$ under the above map. Indeed, \[\begin{pmatrix}
    %     a & b \\ c & d
    % \end{pmatrix} = \begin{pmatrix}
    %     bN+1 & b \\ N & 1
    % \end{pmatrix}\mapsto b\quad\text{and}\quad \det\begin{pmatrix}
    %     bN+1 & b \\ N & 1
    % \end{pmatrix} = 1.\]
    % If $A\in \varGamma_1(N)$ and \[A = \begin{pmatrix}
    %     a & b \\ c & d
    % \end{pmatrix}\mapsto \overline 0,\] then $b \equiv 0\smod N$ for some integer $k$. Since $a,d\equiv 1\smod N$ and $c\equiv 0\smod N$, it follows that $A\in \varGamma(N)$. Certainly any element of $\varGamma(N)$ is contained in the kernel of the above map.
\end{proof} 
It follows that $\varGamma_1(N)/\varGamma(N)\cong \mathbb{Z}/N\mathbb{Z}$, so that the index of $\varGamma(N)$ in $\varGamma_1(N)$ is $N$. What follows is a similar result.

\begin{lemma}\label{lem: gamma 1 normal in gamma 0}
    The map $\varGamma_0(N)\to (\mathbb{Z}/N\mathbb{Z})^\times$ given by $\smallabcd\mapsto \overline d$ 
    is surjective with kernel $\varGamma_1(N)$.
\end{lemma}
\begin{proof}
    % Exercise 1.2.3. (d) in \cite{diamond}.
    % DO NOT DELETE SOLUTION, but recommended to omit solution
    Let $\overline d \in (\mathbb{Z}/N\mathbb{Z})^\times$ be given, and note that $\gcd(d,N)=1$. Let $c \equiv 0\smod N$ so that $c=kN$ for some integer $k$. Choose $k$ coprime to $d$ so that $\gcd(d,kN) = \gcd(d,c) = 1$. Then there exist integers $a,b$ with $ad-bc = 1$. Then \[\begin{pmatrix}
        a & b \\ c & d
    \end{pmatrix}\mapsto d\quad\text{and}\quad \det\begin{pmatrix}
        a & b \\ c & d
    \end{pmatrix} = 1.\] 
    If $A\in \varGamma_0(N)$ and \[A = \begin{pmatrix}
        a & b \\ c & d
    \end{pmatrix}\mapsto \overline 1,\] so $d = jN+1$ for some integer $j$. Since $c\equiv 0\smod N$, $c= kN$ and $\gcd(d,c) = 1$. There exist integers $a,b$ with $1 = ad-bc = a(jN+1) + bkN = a+ (aj + bk)N$. Reducing modulo $N$ gives $a\equiv 1\smod N$. It follows that $A\in \varGamma_1(N)$. Conversely, if $A\in \varGamma_1(N)$, then $A\mapsto \overline 1$.
\end{proof}
Hence $\varGamma_0(N)/\varGamma_1(N)\cong (\mathbb{Z}/N\mathbb{Z})^\times$, so that the index of $\varGamma_1(N)$ in $\varGamma_0(N)$ is $\phi(N) = \abs{(\mathbb{Z}/N\mathbb{Z})^\times} = N\prod_{p\mid N}(1-1/p)$, where $\phi$ denotes the Euler totient function. 
\begin{corollary}
    We obtain \[[\slz : \varGamma_0(N)] = \frac{[\slz : \varGamma(N)]}{[\varGamma_0(N) : \varGamma_1(N)][\varGamma_1(N) : \varGamma(N)]} = \frac{N^3\prod_{p\mid N}(1-1/p^2)}{N\cdot N\prod_{p\mid N}(1-1/p)} = N\prod_{p\mid N}(1+1/p).\]
\end{corollary} 

\subsection{Modular curves}
Recall that $\slz$, and more broadly $\glr$, acts on the Riemann sphere $\widehat{\mathbb{C}}$ by fractional linear transformations \[\abcd(z) = \frac{az+b}{cz+d}.\] 
If $\gamma\in \glr$ with $\gamma = \smallabcd$ and $z\in \mathbb C$, the computation $(az+b)(c\overline z + d) = ac\abs{z}^2+bd + adz + bc\overline z = ac\abs{z}^2+bd + adz + (ad-\det\gamma)\overline z = ac\abs{z}^2+bd + ad(z+\overline z) -\det\gamma\overline z$ shows that
\begin{equation}\label{eqn: imaginary parts}
    \Im(\gamma(z)) = \Im\Big(\frac{az+b}{cz+d}\Big) = \frac{\Im((az+b)(c\overline z + d))}{\abs{cz+d}^2} = \frac{\Im(-\det\gamma\overline z)}{\abs{cz+d}^2} = \frac{\det\gamma\Im(z)}{\abs{cz+d}^2}.
\end{equation}
From the above calculation it follows that if $\det \gamma >0$ and $z$ belongs to the \textib{upper half plane} $\mathcal{H} = \cbr{z\in\mathbb{C} : \Im(z)>0}$, then $\gamma(z)\in\mathcal H$. In particular, it follows that $\glrp = \{\gamma\in \glr : \det \gamma > 0\}$, $\slz$, and any congruence subgroup $\varGamma$ of $\slz$ act on the upper half plane $\mathcal H$ by fractional linear transformations.

\begin{definition}
    Let $\varGamma$ be a congruence subgroup of $\slz$. The \textib{modular curve} $Y(\varGamma)$ is the orbit space $\varGamma\backslash \mathcal{H} = \cbr{\varGamma z : z\in\mathcal H}$.
\end{definition}
The modular curves for the congruence subgroups $\varGamma_0(N), \varGamma_1(N)$, and $\varGamma(N)$ are denoted by $Y_0(N), Y_1(N)$, and $Y(N)$, respectively.

We topologize modular curves and briefly show they are compact Riemann surfaces. Let $\varGamma$ be a congruence subgroup. The upper half plane $\mathcal H$ is given the Euclidean topology (as a subspace of $\mathbb{C}$ or of $\mathbb{R}^2$), and the natural surjection \[\pi \colon \mathcal H \to Y(\varGamma)\quad \text{defined by}\quad  z\mapsto \varGamma z\] induces the quotient topology on $Y(\varGamma)$ (i.e., the open sets in $Y(\varGamma)$ are those with open preimages under $\pi$).

\begin{lemma}\label{lem: slz acts as open maps}
    The quotient map $\pi \colon \mathcal H \to Y(\varGamma)$ is an open map.
\end{lemma}
\begin{proof}
    It suffices to show that the projection $\pi$ takes an open disk $B$ in $\mathcal{H}$ to an open set in $Y(\varGamma)$; that is, to show that \[\bigcup_{\gamma\in \varGamma}\gamma(B)\] is open. Since each $\gamma^{-1}$ is continuous, each $\gamma(B)$ is open, so $\pi$ is an open map.
\end{proof}

\begin{proposition}\label{prop: properly discontinuous}
    The action of the modular group $\slz$ on $\mathcal H$ is properly discontinuous; that is, for $z_1,z_2\in \mathcal H$, (including the case $z_1 = z_2$) there exist neighborhoods $U_1$ of $z_1$ and $U_2$ of $z_2$ in $\mathcal H$ such that for any $\gamma\in \slz$, if $\gamma(U_1)\cap U_2\neq \emptyset$, then $\gamma(z_1) = z_2$. Equivalently, for any $\gamma\in\slz$, if $\gamma(z_1)\neq z_2$, then $\gamma(U_1)\cap U_2= \emptyset$.
\end{proposition}
\begin{proof}
    Let $V_1$ and $V_2$ be neighborhoods of $z_1$ and $z_2$, respectively, with compact closure (e.g. open disks or bounded open sets). Let $y_1 = \sup_{z\in V_1}\cbr{\Im(z)}$, $Y_1 = \sup_{z\in V_1}\cbr{\Im(z)}$, and $y_2 = \inf_{z\in V_2}\cbr{\Im(z)}$. Then from \cref{eqn: imaginary parts}, we have for $\gamma = \big(\!\begin{smallmatrix}
        a & b \\ c & d
    \end{smallmatrix}\!\big)\in \slz$ not equal to $\pm I$ (i.e., $c\neq 0$) and $z\in V_1$ that \[\Im(\gamma(z)) = \frac{\Im(z)}{\abs{cz+d}^2}\leq \frac{\Im(z)}{c^2\abs{z}^2}\leq \frac{1}{c^2\Im(z)}\leq \frac{1}{c^2y_1}\] and \[\Im(\gamma(z)) = \frac{\Im(z)}{\abs{cz+d}^2}\leq \frac{Y_1}{\Re(cz+d)^2} = \frac{Y_1}{(c\Re(z)+d)^2},\] so that $\Im(\gamma(z))\leq\min\cbr{1/c^2y_1,Y_1/(c\Re(z)+d)^2}$. All but finitely many integers $c$ may be chosen so that $1/c^2y_1< y_2$, and of the finitely many $c$ where this inequality does not hold, we may choose all but finitely many $d$ such that $Y_1/(c\Re(z)+d)^2< y_2$ uniformly in $z$ (since $V_1$ has compact closure). Thus for all but finitely many integers $c,d$, if $\gamma  = \big(\!\begin{smallmatrix}
        a & b \\ c & d
    \end{smallmatrix}\!\big)\in \slz$, then \begin{equation}\label{eqn: sup inf}
        \sup_{z\in V_1}\cbr{\Im(\gamma(z))}< \inf_{z\in V_2}\cbr{\Im(z)}.
    \end{equation} Thus for $\gamma = \big(\!\begin{smallmatrix}
        a & b \\ c & d
    \end{smallmatrix}\!\big)\in \slz$ satisfying \cref{eqn: sup inf}, $\gamma(V_1)\cap V_2=\emptyset$. We show that there are only finitely many $\gamma\in \slz$ for which $\gamma(V_1)\cap V_2\neq\emptyset$.

    We determine the matrices in $\slz$ which have fixed bottom row $(c,d)$. This is equivalent to finding all matrices $\eta = \big(\!\begin{smallmatrix}
        A & B \\ C & D
    \end{smallmatrix}\!\big)\in \slz$ such that $\eta\gamma$ has bottom row $(c,d)$, where $\gamma\in \slz$ is a fixed matrix with  bottom row $(c,d)$. Evidently $C=0$ and $D = 1$, and since $1 = AD - BC = A$, we have that $\eta$ must have the form $\big(\!\begin{smallmatrix}
        1 & B \\ 0  & 1
    \end{smallmatrix}\!\big)$, and $B$ may be taken to be any integer. Explicitly, the matrices in $\slz$ with bottom row $(c,d)$ are \[\cbr{\begin{pmatrix}
        1 & k \\ 0 & 1
    \end{pmatrix}\begin{pmatrix}
        a & b \\ c & d
    \end{pmatrix}: k\in \mathbb{Z}},\]
    where $(a,b)$ is one such pair such that $ad-bc = 1$. Thus for $\gamma\in \cbr{\big(\!\begin{smallmatrix}
        1 & k \\ 0 & 1
    \end{smallmatrix}\!\big)\big(\!\begin{smallmatrix}
        a & b \\ c & d
    \end{smallmatrix}\!\big): k\in \mathbb{Z}}$, the intersection $\gamma(V_1)\cap V_2 = \big(\big(\!\begin{smallmatrix}
        a & b \\ c & d
    \end{smallmatrix}\!\big)V_1+ k\big)\cap V_2 =\emptyset$ for all but finitely many $k$; that is, for finitely many $\gamma$. So of the finitely many pairs of integers $(c,d)$ for which \cref{eqn: sup inf} does not hold, only finitely many matrices $\gamma\in \slz$ with bottom row $(c,d)$ exist with $\gamma(V_1)\cap V_2\neq \emptyset$.

    Let $F = \cbr{\gamma\in \slz: \gamma(V_1)\cap V_2\neq \emptyset, \gamma(z_1)\neq z_2}$, which is finite by the above argument. For each $\gamma \in F$ there exist disjoint neighborhoods $V_{1,\gamma}$ of $\gamma(z_1)$ and $V_{2,\gamma}$ of $z_2$ in $\mathcal H$. Let \[U_1 = V_1\cap \Bigg(\bigcap_{\gamma\in F}\gamma^{-1}(V_{1,\gamma})\Bigg)\quad\text{and}\quad U_2 = V_2\cap \Bigg(\bigcap_{\gamma\in F}V_{2,\gamma}\Bigg),\] and note that $U_1,U_2$ are open, as elements of $\slz$ are open maps on $\mathcal H$ by $\cref{lem: slz acts as open maps}$. Then take any $\gamma\in \slz$ with $\gamma(U_1)\cap U_2\neq \emptyset$. If $\gamma\not\in F$, then we must have $\gamma(z_1) = z_2$. Suppose that $\gamma\in F$. Then $U_1\subset \gamma^{-1}(U_{1,\gamma})$ and $U_2\subset U_{2,\gamma}$ so that $ \gamma(U_1)\cap U_2\subset U_{1,\gamma}\cap U_{2,\gamma}$. But $\gamma(U_1)\cap U_2\neq \emptyset$, which is in contradiction with $U_{1,\gamma}$ and $U_{2,\gamma}$ chosen to be disjoint. Hence $\gamma\not\in F$, so that $\gamma(z_1)= z_2$.
\end{proof}

Furthermore, $\pi(U_1)\cap \pi(U_2)=\emptyset$ in $Y(\varGamma)$ is equivalent to $\big(\bigcup_{\gamma\in \varGamma}\gamma(U_1)\big)\cap U_2 = \emptyset$ in $\mathcal H$. The equality
\begin{multline}\label{eqn: disjoint nbds in modular curve}
    \emptyset = \pi^{-1}(\pi(U_1)\cap \pi(U_2)) = \Bigg(\bigcup_{\gamma\in \varGamma}\gamma(U_1)\Bigg)\cap \Bigg(\bigcup_{\gamma\in \varGamma}\gamma(U_2)\Bigg) = \bigcup_{\gamma,\eta\in \varGamma}\gamma(U_1)\cap \eta(U_2)\\ = \bigcup_{\eta\in \varGamma}\eta\Bigg[\Bigg(\bigcup_{\gamma\in \varGamma}(\eta^{-1}\gamma)(U_1)\Bigg)\cap U_2\Bigg] = \bigcup_{\eta\in \varGamma}\eta\Bigg[\Bigg(\bigcup_{\gamma^\prime\in \varGamma}\gamma^\prime(U_1)\Bigg)\cap U_2\Bigg]
\end{multline}
implies that $\big(\bigcup_{\gamma^\prime\in \varGamma}\gamma^\prime(U_1)\big)\cap U_2$ must be empty, from which the above equivalence follows.

\begin{proposition}
    The space $Y(\varGamma)$ is second-countable, connected, and Hausdorff.
\end{proposition}

\begin{proof}
    Since $\pi$ is open and $\mathcal H$ is second-countable, $Y(\varGamma)$ is second-countable. As $\mathcal H$ is connected and $\pi$ is continuous, $Y(\varGamma)$ is connected.

    Let $\pi(z_1)$ and $\pi(z_2)$ be distinct points in $Y(\varGamma)$ (so that $\gamma(z_1)\neq z_2$ for any $\gamma\in \varGamma$), and take neighborhoods $U_1$ of $z_1$ and $U_2$ of $z_2$ such that for any $\gamma\in \slz$, if $\gamma(U_1)\cap U_2\neq \emptyset$, then $\gamma(z_1) = z_2$, as per the previous result. Then $\big(\bigcap_{\gamma\in\varGamma}\gamma(U_1)\big)\cap U_2 = \emptyset$ since $\gamma(z_1)\neq z_2$ for every $\gamma\in \varGamma$. From the discussion surrounding \cref{eqn: disjoint nbds in modular curve}, we have that $\pi(U_1)\cap\pi(U_2) = \emptyset$ as needed, with $\pi(U_1),\pi(U_2)$ open since $\pi$ is an open mapping.
\end{proof}

What remains is to compactify $Y(\varGamma)$ and to put charts on the resulting compact space, which we do without verifying the details. 

The group $\glqp = \{\gamma\in \GL_2(\mathbb Q) : \det \gamma >0\}$ acts on $\mathbb Q \cup \{\infty\}$ by 
\[\abcd \Big(\frac{r}{s}\Big) = \frac{ar+bs}{cr+ds},\] taking $\infty$ to $a/c$ and $-d/c$ to $\infty$ when $c\neq 0$, and taking $\infty$ to $\infty$ when $c = 0$. Since $\smallabcd$ is invertible, the action above never produces indeterminate forms like $0/0$. Furthermore, $\slz$ acts transitively: Any rational number is of the form $a/c$, where $a$ and $c$ are coprime. Choose $b,d$ so that $ad-bc=1$, from which we have $\smallabcd(\infty) = a/c$.

Let $\mathcal H^\ast = \mathcal H\cup \mathbb Q\cup\{\infty\}$ and define the \textib{compactified modular curve} $X(\varGamma)$ as a set by \[X(\varGamma) = \varGamma\backslash\mathcal H^\ast = Y(\varGamma)\cup\varGamma\backslash(\mathbb Q\cup \{\infty\}).\] Call the points $\varGamma s\in \varGamma\backslash(\mathbb Q\cup \{\infty\})$ \textib{cusps} of $X(\varGamma)$. Denote by $X_0(N)$, $X_1(N)$, and $X(N)$ the modular curves $X(\varGamma_0(N))$, $X(\varGamma_1(N))$, and $X(\varGamma(N))$ respectively.

\begin{lemma}\label{lem: finitely many cusps}
    The modular curve $X(\varGamma)$ has finitely many cusps for any congruence subgroup $\varGamma$.
\end{lemma}

\begin{proof}
    The number of cusps of $X(\varGamma)$ is equal to the index of $\varGamma$ in $\slz$. Let $\slz$ act on $\varGamma\backslash(\mathbb Q \cup \{\infty\})$ by $\gamma(\varGamma s) = \varGamma (\gamma s)$. The action is well defined: If $\varGamma s = \varGamma t$, then $s = \gamma^\prime t$ for some $\gamma^\prime \in \varGamma$. Then $\varGamma \gamma\gamma^\prime s = \varGamma\gamma^{\prime\prime}\gamma s = \varGamma s$ for some $\gamma^{\prime\prime}\in\varGamma$. This group action is transitive since $\slz$ acts transitively on $\mathbb Q \cup \{\infty\}$, and the isotropy group of $\varGamma\backslash(\mathbb Q \cup \{\infty\})$ is $\varGamma$. Thus $[\slz:\varGamma] = \abs{\varGamma\backslash (\mathbb Q\cup \{\infty\})}$.
\end{proof}

It follows that the modular curve $X(1) = \slz\backslash \mathcal H^\ast$ has one cusp, namely, $\infty$.

A base for the topology of $\mathcal H^\ast$ consists of open disks centered at elements of $\mathcal H$ and the neighborhoods 
\[\alpha(\cbr{z\in\mathcal H : \Im(z) > M}\cup\{\infty\})\quad \text{for $M>0,\alpha\in\slz$},\] images of disks centered at $\infty$ under elements of $\slz$. Give $\mathcal H$ the topology these sets generate. As fractional linear transformations are conformal maps, for elements $\gamma\in\slz$ with $\gamma(\infty)\in\mathbb Q$, a disk centered at $\infty$ is mapped to a disk that is tangent to the real axis, containing one rational number.

Choosing this base ensures that $X(\varGamma)$ is Hausdorff (whereas taking open disks at each point in $\mathbb Q \cup \{\infty\}$ would not since $\mathbb Q$ is dense in the real line). Furthermore, every $\gamma\in \slz$ is a homeomorphism of $\mathcal H^\ast$ with itself. Give $X(\varGamma)$ the quotient topology with the quotient map $\pi\colon \mathcal H^\ast\to X(\varGamma)$ defined by extending the natural projection.

\begin{proposition}
    The modular curve $X(\varGamma)$ is Hausdorff, connected, and compact. 
\end{proposition}
\begin{proof}
    See proposition 2.4.2 in \cite{diamond}.
\end{proof}

Before defining charts on $X(\varGamma)$, we collect some results about elliptic points.

\begin{definition}
    Let $\varGamma$ be a congruence subgroup. For $z\in\mathcal H$ let $\varGamma_z = \{\gamma\in\varGamma : \gamma(z) = z\}$, the isotropy subgroup of $z$. Call $z\in\mathcal H$ (and the corresponding point $\pi(z)\in X(\varGamma)$) an \textib{elliptic point} of $\varGamma$ if $\varGamma_z$ is nontrivial; that is, if the containment $\{\pm I\}\subset \{\pm I\}\varGamma_z$ is proper.
\end{definition}

\begin{proposition}\label{prop: cyclic isotropy subgp}
    Let $\varGamma$ be a congruence subgroup. Then $X(\varGamma)$ has finitely many elliptic points, and for each elliptic point $z$ of $\varGamma$, its isotropy subgroup $\varGamma_z$ is finite cyclic.
\end{proposition}
\begin{proof}
    See proposition 2.3.3 and corollaries 2.3.4, 2.3.5 in \cite{diamond}.
\end{proof}

It follows that every point $z\in\mathcal H$ has an associated positive integer
\[h_z = \abs{\{\pm I\}\varGamma_z/\{\pm I\}} = \begin{cases}
    \abs{\varGamma_z}/2 & \text{if $-I\in\varGamma_z$}\\
    \abs{\varGamma_z} & \text{if $-I\not\in\varGamma_z$}
\end{cases}\] called the period of $z$. Note that $h_z>1$ when $z$ is an elliptic point, and that for $\gamma\in\slz$, the period of $\gamma(z)$ under $\gamma\varGamma\gamma^{-1}$ is the same as the period of $z$ under $\varGamma$. As a result, $h_z = h_{\gamma z}$ for all $\gamma\in\varGamma$, so that the period of $\varGamma z\in Y(\varGamma)\subset X(\varGamma)$ is well defined.

We now define charts for $X(\varGamma)$, starting by defining them on $Y(\varGamma)$. Since $\slz$ acts properly discontinuously on $\mathcal H$, we have the following:

\begin{corollary}[to \cref{prop: properly discontinuous}]
    Let $\varGamma$ be a congruence subgroup. Then each point $z\in\mathcal H$ is contained in a neighborhood $U\subset \mathcal H$ such that for all $\gamma\in\varGamma$, if $\gamma(U)\cap U \neq \emptyset$ then $\gamma\in\varGamma_z$. Furthermore, $U$ has no elliptic points except possibly $z$.
\end{corollary}

Given a point $\pi(z)\in Y(\varGamma)$, take a neighborhood $U$ for $z$ as in the corollary above. Let $\delta_z  = \big(\!\begin{smallmatrix}
    1 & -z \\ 1 & -\overline{z}
\end{smallmatrix}\!\big)\in \GL_2(\mathbb C)$ be the map from $\mathcal H$ (or $U$ by restriction) to $\mathbb C$ which takes $z$ to $0$ and $\overline{z}$ to $\infty$. Note that the isotropy subgroup of $0$ in the transformation group $(\delta_z\cbr{\pm I}\varGamma\delta_z^{-1})_0/\cbr{\pm I}$ is the conjugate $\delta_z(\cbr{\pm I}\varGamma/\cbr{\pm I})\delta_z^{-1}$ of the isotropy subgroup of $z$, hence is cyclic of order $h_z$ as a group of transformations by \cref{prop: cyclic isotropy subgp} (note $h_z$ may be equal to $1$). Since this group of fractional linear transformations fixes $0$ and $\infty$, these maps are given by $w\mapsto aw$, which are rotations about the origin through angular multiples of $2\pi/h_z$. We call $\delta_z$ a ``straightening map'' since it takes neighborhoods of $z$ to neighborhoods of the origin for which \sai{equivalent points (- points in an orbit of the isotropy subgroup?)} are evenly spaced apart angularly. (See figures 2.1, 2.2 in \cite{diamond}.)

Let $\rho\colon \mathbb C\to \mathbb C$ be given by $\rho(w) = w^{h_z}$. Then $\psi = \rho\circ \delta_z \colon U\to V = \psi(U)$ given by $\psi(w) = (\delta(w))^{h_z}$ straightens then ``wraps'' around the neighborhood into a disk. By the open mapping theorem $V$ is an open subset of $\mathbb C$. There exists an bijection $\varphi\colon \pi(U)\to V$ such that $\varphi\circ\pi = \psi$ (see section 2.2 in \cite{diamond}), which is a homeomorphism as well. The maps $\varphi$ and open sets $U$ for each $z$ are indeed charts for $Y(\varGamma)$.

For neighborhoods containing points in $\mathbb Q\cup\cbr{\infty}$, we specify charts as follows. Let $s\in\mathbb Q\cup\cbr{\infty}$ be a cusp. There exists some $\delta_s\in\slz$ that maps $s$ to $\infty$. We define the width of $s$ to be 
\[h_s = \abs{\slz_\infty/(\delta_s\cbr{\pm I}\varGamma\delta_s^{-1})_\infty}.\] Where the period of an elliptic point is the number of sectors of the disk containing the point that are identified under isotropy, the width of a cusp is the number of sectors (of the infinitely many that come together to $s$) that are distinct under isotropy (see figure 2.6 in \cite{diamond}). The width of $s$ is finite and independent of the choice of $\delta_s$, and similarly to elliptic points, for $\gamma\in\slz$, the width of $\gamma(s)$ under $\gamma\varGamma\gamma^{-1}$ is the same as the width of $s$ under $\varGamma$. Thus $h_s$ is the same for all the cusps in $\varGamma s$, making the width well defined on $X(\varGamma)$.

Let $U = \delta_s^{-1}(\cbr{z\in\mathcal H : \Im(z) > 2}\cup\{\infty\})$ and let $\psi = \rho\circ \delta_s$ where $\rho\colon \mathbb C\to\mathbb C$ is given by $\rho(z) = e^{2\pi i z/h_s}$. With $V = \psi(U)$ (an open subset of $\mathbb C$), obtain $\psi\colon U\to V$ with $\psi(z) = e^{2\pi i \delta_s(z)/h_s}$. The effect of $\psi$ is to straighten $U$ by making identified points differ by a constant, and the exponential map wraps the upper half plane into a disk centered at $0$ (where $\infty$ maps to). Similarly, there exists a homeomorphism $\varphi\colon \pi(U)\to V$ such that $\varphi\circ\pi = \psi$ that gives the chart as desired. Hence the maps $\phi$ and open sets $U$ for each $s$ give charts on the rest of $X(\varGamma)$.

We briefly define fundamental domains obtained from $X(\varGamma)$, without proofs of claims. We first consider the case $\varGamma = \slz$. Consider the set $\mathcal D = \cbr{z\in\mathcal H : \abs{\Re(z)}\leq 1/2,\abs{z}\geq 1}$. The map $\pi\colon\mathcal D\to Y(1) = Y(\slz)$ given by the natural projection $z\mapsto \slz z$ is a surjection, and is not injective since the boundary half lines where $\Re(z) = \pm 1/2$ are identified by the translation $\big(\!\begin{smallmatrix}
    1 & 1 \\ 0 & 1
\end{smallmatrix}\!\big)$ which takes $z$ to $z+1$ (similarly, the two halves of the boundary arc where $\abs{z} = 1$ are identified by the inversion $\big(\!\begin{smallmatrix}
    0 & -1 \\ 1 & 0
\end{smallmatrix}\!\big)$ taking $z$ to $-1/z$). (See section 2.3 of \cite{diamond}) In fact, if two distinct points in $\mathcal D$ are $\slz$-equivalent, then the points lie on the boundary of $\mathcal D$ and are either translates or inverses of each other by the aforementioned maps. So by a suitable identification of points on the boundary of $\mathcal D$ the map $\pi$ can be made a bijection. We say that $\mathcal D$ is a \textib{fundamental domain} for $\mathcal H$ under the action of $\slz$. A similar procedure may be done for $\mathcal D^\ast = \mathcal D \cup\cbr{\infty}$ and $X(1) = X(\slz)$, by which $\mathcal D^\ast$ becomes a fundamental domain for $\mathcal H^\ast$ under the action of $\slz$. 

For other congruence subgroups $\varGamma$, write $\slz = \bigcup_j \cbr{\pm I}\varGamma \gamma_j$ for some set of representatives $\gamma_j$ of the coset space $\varGamma\backslash \slz$, and consider the surjection $\pi\colon \bigcup_j \gamma_j\mathcal D\to Y(\varGamma)$ given by $z\mapsto \varGamma z$. By identifying the appropriate boundary points of $\bigcup_j \gamma_j\mathcal D$, $\pi$ may be regarded as a bijection, and so $\bigcup_j \gamma_j\mathcal D$ becomes a fundamental domain for $\mathcal H$ under the action of $\varGamma$. A similar procedure for $X(\varGamma)$ produces fundamental domains of $\mathcal H^\ast$ under the action of $\varGamma$ \sai{This last statement should be true, but maybe I should find a source}. 

\newpage\section{Modular forms}
In this section, we define modular forms, provide dimension formulas for the spaces of modular forms appearing throughout, discuss $L$-functions corresponding to modular forms, and provide definitions of Eisenstein series as important examples of modular forms.
\subsection{Definitions}
\begin{definition}
    For an integer $k$ and a congruence subgroup $\varGamma$, a meromorphic function $f\colon \mathcal H\to \mathbb{C}$ is \textib{weakly modular of weight $k$ (with respect to $\varGamma$)} if \[f(\gamma(z)) = (cz+d)^kf(z)\quad\text{for any}\quad \gamma = \abcd\in \varGamma, ~z\in \mathcal H.\qedhere\]
\end{definition}
Sometimes we will say that a meromorphic function $f\colon \mathcal H\to \mathcal C$ is just ``weakly modular'' if the weight $k$ and congruence subgroup $\varGamma$ are irrelevant or clear from context.
\begin{definition}\label{def: automorphy and weight k op}
    For any matrix $\gamma=\smallabcd\in \glr$ the \textib{factor of automorphy} $j(\gamma,z)$ for $z\in \mathbb{C}$ is defined by \[j(\gamma,z) = cz+d.\] For any integer $k$ and $\gamma\in \glrp$ define the \textib{weight-$k$ $\gamma$ operator} $|_k[\gamma]$ on functions $f\colon \mathcal H\to \mathbb{C}$ by \[(f|_k[\gamma])(z) = (\det\gamma)^{k/2} j(\gamma,z)^{-k}f(\gamma(z)),\quad\text{for } z\in \mathcal H.\qedhere\]
\end{definition}
 Note that $|_k[\gamma]$ acts on the right of $f$, and composes left to right with other weight-$k$ operators.
Since the factor of automorphy $j(\gamma,z) = cz+d$ cannot be zero or infinity (as $z\in \mathcal H$), if $f$ is meromorphic on $\mathcal H$, then so is $f|_k[\gamma]$, and the number of poles and zeroes of $f|_k[\gamma]$ and $f$ are the same.

So for an integer $k$ and a congruence subgroup $\varGamma$, a meromorphic function $f\colon \mathcal H \to \mathbb{C}$ is weakly modular of weight $k$ (with respect to $\varGamma$) if \[f|_k[\gamma] = f\quad\text{for all }\gamma\in \varGamma,\] and this is equivalent to the original definition above. Note that if $f$ is weakly modular with respect to $\varGamma$, then the zeroes of $f$ and poles of $f$ are $\varGamma$-invariant sets.

\begin{lemma}\label{lem: props of automorphy}
    For any $\gamma,\eta\in \glrp$ and $z\in \mathcal H$, we have $j(\gamma\eta,z) = j(\gamma,\eta(z))j(\eta,z)$ and $|_k[\gamma\eta] = |_k[\gamma]|_k[\eta]$.
\end{lemma}
\begin{proof}
    Elements of $\glqp$ act on column vectors in $\mathbb C^2$ by matrix multiplication, and act on points in $\mathbb C$ by fractional linear transformations. Let $\gamma = \smallabcd\in \glqp$ and observe that for $z\in\mathbb C$ we have \[\gamma\begin{pmatrix}
        z \\ 1
    \end{pmatrix} = \begin{pmatrix}
        az+b \\ cz+d
    \end{pmatrix} = \begin{pmatrix}
        \gamma(z) \\ 1
    \end{pmatrix}j(\gamma,z).\]
    Then for $\eta\in\slz$ we have
    \begin{align*}
        \gamma\eta\begin{pmatrix}
            z \\ 1
        \end{pmatrix} &= \begin{pmatrix}
            (\gamma\eta)(z) \\ 1
        \end{pmatrix}j(\gamma\eta,z)\quad\text{and}\\
        \gamma\eta\begin{pmatrix}
            z \\ 1
        \end{pmatrix} &= \gamma\begin{pmatrix}
            \eta(z) \\ 1
        \end{pmatrix}j(\eta,z) = \begin{pmatrix}
            \gamma(\eta(z)) \\ 1
        \end{pmatrix}j(\gamma,\eta(z))j(\eta,z).
    \end{align*} 
    It follows that $j(\gamma\eta,z) = j(\gamma,\eta(z))j(\eta,z)$.
    Then for $f\colon \mathcal H\to \mathbb C$,
    \begin{align*}
        (f|_k[\gamma\eta])(z) &= (\det(\gamma\eta))^{k/2}j(\gamma\eta,z)^{-k}f((\gamma\eta)(z))\\
        &= (\det \eta)^{k/2}j(\eta,z)^{-k}(\det \gamma)^{k/2}j(\gamma,\eta(z))^{-k}f(\gamma(\eta(z)))\\
        &= (\det \eta)^{k/2}j(\eta,z)^{-k}(f|_k[\gamma])(\eta(z))\\
        &= ((f|_k[\gamma])|_k[\eta])(z)
    \end{align*} so that $|_k[\gamma\eta] = |_k[\gamma]|_k[\eta]$.
\end{proof}
From the lemma it follows that if $f\colon \mathcal H\to\mathbb C$ is weakly modular with respect to some set of matrices $A$, then $f$ is weakly modular with respect to the group generated by $A$. So with $\slz = \big\langle\big(\!\begin{smallmatrix}
    1 & 1 \\ 0 & 1
\end{smallmatrix}\!\big), \big(\!\begin{smallmatrix}
    0 & -1 \\ 1 & 0
\end{smallmatrix}\!\big)\big\rangle$, a function $f\colon \mathcal H\to \mathbb C$ is weakly modular of weight $k$ with respect to $\slz$ if
\begin{equation}\label{eqn: periodicity slz}
    f(z+1) = f\big(\big(\!\begin{smallmatrix}
        1 & 1 \\ 0 & 1
    \end{smallmatrix}\!\big)z\big) = (0z+1)^kf(z) = f(z)\quad\text{and}\quad f(-1/z) = f\big(\big(\!\begin{smallmatrix}
        0 & -1 \\ 1 & 0
    \end{smallmatrix}\!\big)z\big) = z^kf(z).
\end{equation} It follows that weakly modular functions of weight $k$ with respect to $\slz$ are $\mathbb Z$-periodic. A similar phenomenon happens for weakly modular functions of weight $k$ with respect to congruence subgroups. If $\varGamma$ is a congruence subgroup of level $N$, then $\varGamma(N)\subset \varGamma$, so that $\varGamma$ contains a translation matrix of the form $\big(\!\begin{smallmatrix}
    1 & h \\ 0 & 1
\end{smallmatrix}\!\big)$ for some minimal positive integer $h$ dividing $N$. To see that $h$ necessarily divides $N$, observe that if 
\[\begin{pmatrix}
    1 & b_1 \\ 0 & 1
\end{pmatrix},\begin{pmatrix}
    1 & b_2 \\ 0 & 1
\end{pmatrix}\in\varGamma, \quad \text{then}\quad \begin{pmatrix}
    1 & \gcd(b_1,b_2) \\ 0 & 1
\end{pmatrix}\in \varGamma.\]
By a similar computation to \cref{eqn: periodicity slz}, it follows that weakly modular functions with respect to $\varGamma$ are $h\mathbb Z$-periodic. Weakly modular functions with respect to $\varGamma(N)$ are $N\mathbb{Z}$-periodic, and weakly modular functions with respect to $\varGamma_1(N)$ are $\mathbb Z$-periodic.

Let $D = \cbr{q\in\mathbb C : \abs{q} <1}$ be the open unit disk in $\mathbb C$ and let $D^\prime = D \setminus \!\cbr{0}$ denote the punctured open unit disk in $\mathbb C$. The exponential map $z\mapsto e^{2\pi i z/h} = q$ is a $h\mathbb{Z}$-periodic holomorphic map which maps $\mathcal H$ to $D^\prime$. Since $f$ is $h\mathbb{Z}$-periodic, it follows that the function $\tilde f\colon D^\prime\to\mathbb C$ corresponding to $f$ defined by $\tilde f(q) = f(h\log (q)/2\pi i)$ (so $f(z) = \tilde f(e^{2\pi i z/h})$) is well defined because a branch of the logarithm is determined up to integral multiples of $2\pi i$.

The logarithm can be defined holomorphically about each $q\in D^\prime$. It follows that $\tilde f$ is meromorphic on $D^\prime$ since $f$ is meromorphic on $\mathcal H$, and so $\tilde f$ has a Laurent expansion $\tilde f(q) = \sum_{n\in\mathbb Z}a_nq^n$ at each $q$ in a punctured neighborhood of $q=0$ (and $a_n = 0$ for all $n$ sufficiently small). Moreover, if $f$ is holomorphic on $\mathcal H$, it follows that $\tilde f$ is also holomorphic on $D^\prime$.

From $\abs{q} = e^{-2\pi \Im z/h}$, it follows that $q$ tends to zero as $\Im z$ tends to infinity. We define $f$ to be meromorphic (holomorphic) at $\infty$ if $\tilde f$ has a meromorphic (holomorphic) extension to $q = 0$ (and in the holomorphic case the Laurent series at $q=0$ sums over the nonnegative integers). The Laurent series of $\tilde f$ about $q=0$ is used to obtain a Fourier series expansion of $f$ about $\infty$, given by 
\[f(z) = \sum_{n\in\mathbb Z} a_n(f)q^n,\quad q = e^{2\pi i z/h},\] which converges absolutely and uniformly on compact subsets of the half plane $\cbr{z\in\mathbb Z: \Im z>\tau}$ for some large enough $\tau$ (so that $q$ lies in a punctured neighborhood of zero). When referring to a Fourier series of a $h\mathbb Z$-periodic meromorphic function on $\mathcal H$, we mean the expansion obtained in the above manner. If $f$ is holomorphic on $\mathcal H$ and is holomorphic at $\infty$, the Fourier series expansion becomes $f(z) = \sum_{n=0}^\infty a_n(f)q^n$, $q = e^{2\pi i z/h}$, valid for $z\in \mathcal H$. To see that a weakly modular holomorphic function $f\colon \mathcal H\to \mathbb C$ is holomorphic at $\infty$, it suffices to show that $\lim_{\Im z\to\infty} f(z)$ exists or that $f(z)$ is bounded as $\Im z$ grows unboundedly.
\begin{definition}
    Let $\varGamma$ be a congruence subgroup of $\slz$ and let $k$ be an integer. A function $f\colon \mathcal H\to\mathbb C$ is a \textib{modular form of weight $k$ \textup{(}with respect to $\varGamma$\textup{)}} if it satisfies the following: \begin{enumerate}[label=(\arabic*)]
        \item $f$ is holomorphic on $\mathcal H$,
        \item $f$ is weakly modular of weight $k$ with respect to $\varGamma$, and
        \item $f|_{k}[\alpha]$ is holomorphic at $\infty$ for all $\alpha\in\slz$.
    \end{enumerate}Furthermore, if for every $\alpha\in\slz$, the coefficient $a_0$ vanishes in the Fourier series expansion of $f|_k[\alpha]$, then we call $f$ a \textib{cusp form of weight $k$ \textup{(}with respect to $\varGamma$\textup{)}}. The set of modular forms (respectively cusp forms) of weight $k$ with respect to $\varGamma$ is denoted by \textib{$\mathcal M_k(\varGamma)$} (respectively \textib{$\mathcal S_k(\varGamma)$}).

    Define also the \textbf{\textit{weight $k$ }}\textib{Eisenstein space \textup{(}of $\varGamma$\textup{)}} by the quotient of the space of modular forms by the space of cusp forms; that is, $\mathcal E_k(\varGamma) = \mathcal M_k(\varGamma)/\mathcal S_k(\varGamma)$. We briefly discuss this space at the end of \cref{section eisenstein series}.
\end{definition}
Recall that the $\varGamma$-equivalence classes of points in $\mathbb Q\cup \infty$ are the cusps of $\varGamma$. There are finitely many cusps, at most the index of $\varGamma$ in $\slz$, which we showed was finite in the discussion following \cref{def: congruence subgroup}. Write a cusp $s$ as $\alpha(\infty)$ for some $\alpha\in\slz$. Then holomorphy at $s$ of a modular form $f$ is defined by holomorphy at $\infty$ of $f|_k[\alpha]$, where $f|_k[\alpha]$ is a modular form of weight $k$ with respect to $\alpha^{-1}\varGamma\alpha$, so interpret condition (3) of the above definition as $f$ being holomorphic at the cusps of $\varGamma$. The group $\alpha^{-1}\varGamma\alpha$ is a congruence subgroup since for some $N>0$, the principal congruence subgroup $\varGamma(N)$ is contained in $\varGamma$ and is normal in $\slz$, so that $\varGamma(N) = \alpha^{-1}\varGamma(N)\alpha\subset \alpha^{-1}\varGamma\alpha$. 

As congruence subgroups have finite index in $\slz$, only finitely many coset representatives $\alpha_j$ in a decomposition $\slz= \bigcup_{j}\varGamma\alpha_j$ are needed to verify condition (3) in the definition above, and in verifying that the term $a_0$ vanishes for all Fourier series expansions for cusp forms: we have by condition (2) of the above definition that $f|_k[\gamma\alpha_j] = f|_k[\alpha_j]$ for all $\gamma\in\varGamma$.

\begin{lemma}\label{lem: weight k operator properties}
    \begin{enumerate}[label = \textup{(\alph*)}]
        \item For any $\gamma\in\glqp$, there exists $\alpha\in\slz$, $r\in\mathbb Q^+$, and $a,b,d\in\mathbb Z$ relatively prime such that $\gamma = r\alpha \big(\!\begin{smallmatrix}
            a & b \\ 0 & d
        \end{smallmatrix}\!\big)$. 
    
        It follows that for $f\in\mathcal M_k(\varGamma)$ and $\gamma$ with $\gamma = r\alpha \big(\!\begin{smallmatrix}
            a & b \\ 0 & d
        \end{smallmatrix}\!\big)$, since $f|_k[\alpha]$ has a Fourier expansion, so does $f|_k[\gamma]$. Moreover, if the constant term in the Fourier expansion for $f|_k[\alpha]$ is $0$, the same holds for the Fourier expansion for $f|_k[\gamma]$.
        \item Let $\varGamma_1,\varGamma_2$ be congruence subgroups with $\gamma\varGamma_2\gamma^{-1}\subset \varGamma_1$ for some $\gamma\in\glqp$. If $f\in\mathcal M_k(\varGamma_1)$, then $f|_k[\gamma]\in\mathcal M_k(\varGamma_2)$, and the same result holds for cusp forms.
        \item Let $N,d$ be positive integers and let $f\in\mathcal M_k(\varGamma_i(N))$. If $g(z) = f(dz)$, then $g\in\mathcal M_k(\varGamma_i(dN))$ for $i = 0,1$, and the same result holds if $f$ is a cusp form \textup{(}i.e., $g$ is also a cusp form\textup{)}.
    \end{enumerate}  
\end{lemma}
\begin{proof}
    \tbd
\end{proof}
It is possible to determine if a weakly modular holomorphic function on $\mathcal H$ is a modular form by investigating the growth of its Fourier coefficients. First, we prove a few lemmas.
\begin{lemma}
    If $x$ is a cusp of a congruence subgroup $\varGamma$ and $\sigma\in\slz$ sends $x$ to $\infty$, then 
    \[\sigma\varGamma_x\sigma^{-1}\cdot\cbr{\pm I} = \cbr{\pm\begin{pmatrix}
        1 & h \\ 0 & 1
    \end{pmatrix}^m : m\in\mathbb Z}\] for some integer $h>0$.
\end{lemma}
\begin{proof}
    Without loss of generality take $x = \infty$ by taking $\sigma\varGamma\sigma^{-1}$ in place of $\varGamma$. From a computation, we have
    \[\varGamma_\infty \subset \cbr{\pm\begin{pmatrix}
        1 & b \\ 0 & 1
    \end{pmatrix} : b\in\mathbb Z}\]
%     There exists an element  $\gamma = \big(\!\begin{smallmatrix}
%     1 & \ell \\ 0 & 1
% \end{smallmatrix}\!\big)$ with $\ell\neq 0$ in the stabilizer $\varGamma_\infty$ of $\infty$. 
%  Suppose there exists an element $\alpha$ of $\varGamma_\infty$ of the form $\big(\!\begin{smallmatrix}
%     a & b \\ 0 & a^{-1}
% \end{smallmatrix}\!\big)$ with $a\neq \pm 1$. Assume $\abs{a}<1$ by taking $\alpha^{-1}$ in place of $\alpha$ if needed. Then for any positive integer $n$, the element 
% \[\alpha^n\gamma\alpha^{-n} = \begin{pmatrix}
%     1 & a^{2n}l \\ 0 & 1
% \end{pmatrix}\in \varGamma.\] We take as a fact that congruence subgroups are discrete topological groups, so that the previous statement yields a contradiction. It follows then that $\varGamma_\infty\subset\cbr{\pm\big(\!\begin{smallmatrix}
%     1 & b \\ 0 & 1
% \end{smallmatrix}\!\big) : b\in\mathbb R}$.
We obtain the result by observing that $\cbr{\pm\big(\!\begin{smallmatrix}
    1 & b \\ 0 & 1
\end{smallmatrix}\!\big) : b\in\mathbb Z}/\cbr{\pm I}\cong \mathbb Z$.
\end{proof}
\begin{lemma}\label{lem: bound growth fourier coeffs}
    Let $f$ be a weakly modular holomorphic function on $\mathcal H$ with respect to some congruence subgroup $\varGamma$. If there exists a real number $v$ such that $f(z) = \mathcal O(\Im(z)^{-v})$ as $\Im(z)\to\infty$ uniformly with respect to $\Re(z)$, then $f$ is a modular form of weight $k$. Moreover, if $v$ may be chosen so that $v<k$, then $f$ is a cusp form.
\end{lemma}
\begin{proof}
    We may assume that $k$ is even by squaring $f$ \sai{(why do we need $k$ even?)}. Let $x$ be a cusp of $\varGamma$ and suppose that $x$ is rational. Then there exists $\sigma\in\slz$ such that $\sigma(x) = \infty$. Let $h>0$ be an integer satisfying the result of the previous lemma. Then the Fourier series expansion of $f|_k[\sigma^{-1}]$ is 
    \[(f|_k[\sigma^{-1}])(z) = \sum_{n=-\infty}^\infty a_ne^{2\pi i n z/h},\] with \begin{equation}\label{eqn: fourier coeffs z_0}
        a_n = \frac{1}{h}\int_{z_0}^{z_0+h}(f|_k[\sigma^{-1}])(z)e^{-2\pi i n z/h}\dd z,\quad \text{for any fixed $z_0\in\mathcal H$.}
    \end{equation} Let $\smallabcd = \sigma^{-1}$, and note that $c\neq 0$, since $\sigma^{-1}$ does not fix $\infty$. We have that $\Im(\sigma^{-1}(z)) = \Im(z)/\abs{cz+d}^2 = \mathcal O(1/\Im(z))$ as $\Im(z)\to\infty$, uniformly on $\abs{\Re(z)}\leq h/2$. Then by assumption,
    \[(f|_k[\sigma^{-1}])(z) = f(\sigma^{-1}(z))j(\sigma^{-1},z)^{-k} = \mathcal O(\Im(z)^{v-k})\quad (\text{as }\Im(z)\to\infty),\] uniformly on $\abs{\Re(z)}\leq h/2$. Choosing $z_0 = iy-h/2$ in \cref{eqn: fourier coeffs z_0}, we have that $|a_n| = \mathcal O(y^{v-k}e^{2\pi n y /h})$ as $y\to\infty$.

    So if $n<0$, then $a_n = 0$; moreover, if $v<k$, then $a_0 = 0$. 

    If $x\neq\infty$, then we may send it to a rational cusp via an element of $\varGamma$ since $\varGamma\neq \varGamma_\infty$, and repeat the above argument. It follows that $f$ is holomorphic at all cusps of $\varGamma$, and has a zero at any cusp if $v<k$.
\end{proof}
\begin{lemma}\label{lem: miyake lem 4.3.3}
    Let $\{a_n\}_{n=0}^\infty$ be a sequence of complex numbers, and let $f\colon \mathcal H\to \mathbb C$ be given by $f(z) = \sum_{n=0}^\infty a_ne^{2\pi i nz}$. If $a_n = \mathcal O(n^v)$ for some $v >0$, then $\sum_{n=0}^\infty a_ne^{2\pi i nz}$ is convergent absolutely and uniformly on compact subsets of $\mathcal H$. Moreover, 
    \begin{align*}
        f(z) &= \mathcal O(\Im(z)^{-v-1}) \quad\textup{($\Im(z) \to 0$), and }\\
        f(z) - a_0 &= \mathcal O(e^{-2\pi \Im(z)}) \quad\textup{($\Im(z) \to \infty$)},
    \end{align*}
    uniformly on $\Re(z)$.
\end{lemma}
\begin{proof}
    From $\varGamma(s) = \lim_{n\to \infty}\frac{n!n^s}{s(s+1)\cdots(s+n)}$ for real $s>0$ (Euler-Gauss), we have for $v>0$ that
    \[\lim_{n\to\infty} n^v/(-1)^n\binom{-v-1}{n} = \varGamma(v).\]
    Thus there exists $L>0$ such that 
    \[\abs{a_n}\leq L(-1)^n\binom{-v-1}{n}\] for all $n\geq 0$. Let $z = x+iy$, so that 
    \begin{align}\label{eqn: miyake eqn 4.3.9}
        \sum_{n=0}^\infty\abs{a_n}|e^{2\pi i n z}|&\leq L\bigg(\sum_{n=0}^\infty (-1)^n\binom{-v-1}{n}e^{-2\pi n y}\bigg)\\
        &= L(1-e^{-2\pi y})^{-v-1}.\nonumber
    \end{align}
    It follows that $f$ is convergent absolutely and uniformly on compact subsets of $\mathcal H$. Since $1-e^{-2\pi y} = \mathcal O(y)$ as $y\to 0$, we have that $\abs{f(z)} = \mathcal O(y^{-v-1})$. Furthermore, \cref{eqn: miyake eqn 4.3.9} implies that $f$ is bounded as $y$ tends to $\infty$.

    Let $g\colon\mathcal H\to\mathbb C$ be given by $g(z) = \sum_{n=0}^\infty a_{n+1}e^{2\pi i nz}$, and observe that $g$ also satisfies the hypotheses of the lemma. It follows that $g$ is also bounded on a neighborhood of $\infty$. Therefore 
    \[f(z)-a_0 = e^{2\pi i z}g(z) = \mathcal O(e^{-2\pi y})\quad (y\to \infty).\qedhere\]
\end{proof}
Combining the previous few lemmas, we obtain the following:
\begin{proposition}\label{prop: growth of fourier coeffs}
    Let $f$ be a weakly modular holomorphic function on $\mathcal H$ with respect to some congruence subgroup $\varGamma$. If in a Fourier expansion of $f$ given by $f(z) = \sum_{n = 0}^\infty a_ne^{2\pi i n z/h}$, the coefficients $a_n$ are of order $\mathcal O(n^v)$ for some $v>0$, then $f$ is a modular form.
\end{proposition}
The following result describes the growth of cusp forms:
\begin{proposition}\label{prop: product cuspform boundedness}
    Let $f$ be a weakly modular holomorphic function on $\mathcal H$. Then $f$ is a cusp form if and only if $f(z)\Im(z)^{k/2}$ is bounded on $\mathcal H$.
\end{proposition}
\begin{proof}
    We may assume that $k$ is even, and observe that \cref{lem: bound growth fourier coeffs} provides the `if' direction of the result. So assume that $f$ is a cusp form and let $g$ be given by $g(z) = |f(z)|\Im(z)^{k/2}$. Since $g(\gamma(z)) = g(z)$ for any $\gamma\in\varGamma$, view $g$ as a continuous function on $Y(\varGamma)$. Since $\varGamma$ has only finitely many inequivalent cusps, it suffices to show that $g$ is bounded on a neighborhood of a cusp of $\varGamma$. Let $x$ be a cusp of $\varGamma$, and let $\sigma\in\slz$ take $x$ to $\infty$. Choose a positive integer $h$ so that $\sigma\varGamma_x\sigma^{-1}\cdot\cbr{\pm I} = \cbr{\pm\big(\!\begin{smallmatrix}
        1 & h \\ 0 & 1
    \end{smallmatrix}\!\big)^m : m\in\mathbb Z}$, and let $(f|_k[\sigma^{-1}])(z) = \sum_{n=1}^\infty a_ne^{2\pi i nz /h}$ be the Fourier expansion of $f$ at $x$. Then \begin{align*}
        g(\sigma^{-1}(z)) &= \abs{(f|_k[\sigma^{-1}])(z)}\Im(z)^{k/2}\\
        &= \abs{\sum_{n=1}^\infty a_ne^{2\pi i n z/h}}\Im(z)^{k/2}\quad (\text{as }\Im(z)\to\infty).
    \end{align*} Hence $g$ is bounded on a neighborhood of $x$.
\end{proof}
\begin{corollary}\label{cor: cuspform growth corollary}
    Let $f$ be an element of $\mathcal S_k(\varGamma)$, let $x$ be a cusp of $\varGamma$, and let $\sigma$ be an element of $\slz$ that maps $x$ to $\infty$. Let $(f|_k[\sigma^{-1}])(z) = \sum_{n=1}^\infty a_ne^{2\pi i nz /h}$ be the Fourier expansion of $f$ at $x$. Then $a_n = \mathcal O(n^{k/2})$.
\end{corollary}
\begin{proof}
    Let $g = f|_k[\sigma^{-1}]$, so that $g\in\mathcal S_k(\sigma\varGamma\sigma^{-1})$. By \cref{prop: product cuspform boundedness}, there exists a constant $M>0$ such that $\abs{g(z)}\leq M\Im(z)^{-k/2}$. Thus 
    \begin{align*}
        |a_n| &= \frac{1}{h}\abs{\int_0^h g(x+iy)e^{-2\pi i n(x+iy)/h}\dd x}\\
        &\leq My^{-k/2}e^{2\pi n y/h}.
    \end{align*}
    By taking $y = 2/n$, it follows that $\abs{a_n}\leq (Me^{4\pi /h}2^{-k/2})n^{k/2}$.
\end{proof}
Lastly, we record the dimensions of various spaces of modular forms below, without proof (See chapter 3 of \cite{diamond}).

\begin{theorem}
    Let $k$ be an even integer and let $\varGamma$ be a congruence subgroup. Let $g$ be the genus of $X(\varGamma)$, let $\varepsilon_2,\varepsilon_3$ be the number of elliptic points with period $2$, $3$ respectively, and let $\varepsilon_\infty$ be the number of cusps. Then

    \[\dim \mathcal M_k(\varGamma) = \begin{cases}
        (k-1)(g-1) + \lfloor\frac{k}{4}\rfloor\varepsilon_2 + \lfloor\frac{k}{3}\rfloor\varepsilon_3 + \frac{k}{2}\varepsilon_\infty & \text{if $k\geq 2$},\\
        1 & \text{if $k = 0$},\\
        0 & \text{if $k<0$},
    \end{cases} \quad \text{and}\]

    \[\dim \mathcal S_k(\varGamma) = \begin{cases}
        (k-1)(g-1) + \lfloor\frac{k}{4}\rfloor\varepsilon_2 + \lfloor\frac{k}{3}\rfloor\varepsilon_3 + (\frac{k}{2}-1)\varepsilon_\infty & \text{if $k\geq 4$},\\
        g & \text{if $k = 2$},\\
        0 & \text{if $k<0$},
    \end{cases}\]
\end{theorem}

In particular, the modular forms of weight $0$, $\mathcal M_0(\slz)$, is isomorphic to $\mathbb C$ since the genus of $X(1)$ is $0$. Furthermore, $\mathcal M_2(\slz) = 0$ and $\mathcal S_k(\slz) = 0$ for $k = 0,2$. For any even integer $k\geq 4$, $\mathcal M_k(\slz) = \mathcal S_k(\slz)\oplus \mathbb CE_k$ where $E_k$ is the normalized weight $k$ Eisenstein series, which we define in \cref{section eisenstein series}, and \[\dim S_k(\slz) = \begin{cases}
    \lfloor \frac{k}{12}\rfloor - 1 & \text{if $k\equiv 2 \smod{12}$}, \\
    \lfloor \frac{k}{12}\rfloor & \text{otherwise.}
\end{cases}\]

\begin{theorem}
    Let $k$ be an odd integer and let $\varGamma$ be a congruence subgroup. If $-I\in\varGamma$, then both $\mathcal M_k(\varGamma)$ and $\mathcal S_k(\varGamma)$ are zero. Otherwise, let $g$ be the genus of $X(\varGamma)$, $\varepsilon_3$ be the number of elliptic points with period $3$, $\varepsilon_\infty^{\mathrm{reg}}, \varepsilon_\infty^{\mathrm{irr}}$ be the number of regular and irregular cusps, respectively \textup{(}see section 3.6 in \cite{diamond}\textup{)}. Then 
    \[\dim \mathcal M_k(\varGamma) = \begin{cases}
        (k-1)(g-1) + \lfloor\frac{k}{3}\rfloor\varepsilon_3 + \frac{k}{2}\varepsilon_\infty^{\mathrm{reg}} + \frac{k-1}{2}\varepsilon_\infty^{\mathrm{irr}} & \text{if $k\geq 3$},\\
        0 & \text{if $k<0$},
    \end{cases}\quad \text{and}\]
    \[\dim \mathcal S_k(\varGamma) = \begin{cases}
        (k-1)(g-1) + \lfloor\frac{k}{3}\rfloor\varepsilon_3 + (\frac{k}{2}-1)\varepsilon_\infty^{\mathrm{reg}} + \frac{k-1}{2}\varepsilon_\infty^{\mathrm{irr}} & \text{if $k\geq 3$},\\
        0 & \text{if $k<0$}.
    \end{cases}\] If $\varepsilon_\infty^{\mathrm{reg}}>2g-2$, then $\dim \mathcal M_1(\varGamma) = \varepsilon_\infty^{\mathrm{reg}}/2$ and $\dim \mathcal S_1(\varGamma) = 0$. If $\varepsilon_\infty^{\mathrm{reg}}\leq 2g-2$, then $\dim \mathcal M_1(\varGamma)\geq \varepsilon_\infty^{\mathrm{reg}}$ and $\dim \mathcal S_1(\varGamma) = \dim \mathcal M_1(\varGamma)- \varepsilon_\infty^{\mathrm{reg}}/2$.
\end{theorem}

Dimension formulas for other spaces of modular forms (varying $k, N, \varGamma$) may be found in section 3.9 of \cite{diamond}.

Let $k$ be an integer and $\varGamma$ a congruence subgroup. Then the dimensions of the Eisenstein spaces are given by 
\[\dim \mathcal E_k(\varGamma) = \begin{cases}
    \varepsilon_\infty & \text{if $k\geq 4$ is even},\\
    \varepsilon_\infty^{\mathrm{reg}} & \text{if $k\geq 3$ is odd and $-I\not\in\varGamma$},\\
    \varepsilon_\infty-1 & \text{if $k=2$},\\
    \varepsilon_\infty^{\mathrm{reg}}/2 & \text{if $k=1$ and $-I\not\in\varGamma$},\\
    1 & \text{if $k=0$},\\
    0 & \text{if $k<0$, or if $k>0$ is odd and $-I\in\varGamma$}.
\end{cases}\]

\subsection{Eisenstein series}\label{section eisenstein series}
We define the Eisenstein series for a number of spaces of modular forms to provide some examples of modular forms, and do so without proofs of any claims.

Elementary Eisenstein series for $\slz$ are defined for even weights $k\geq 4$, by \[G_k(z) = {\sum_{(c,d)}}^\prime \frac{1}{(cz+d)^k},\quad z\in\mathcal H,\] where the notation $\sum_{(c,d)}^\prime$ denotes summation over all nonzero integer pairs $(c,d)\in\mathbb Z^2$. This sum is absolutely convergent, and converges uniformly on compact subsets of $\mathcal H$. It follows that $G_k$ is holomorphic on $\mathcal H$ and we may rearrange the terms of the sum defining $G_k$. A Fourier series for $G_k$ is given by \[G_k(z) = 2\zeta(k) + 2\frac{(2\pi i)^k}{(k-1)!}\sum_{n=1}^\infty \sigma_{k-1}(n)e^{2\pi i z},\] where $\sigma_{k-1}$ is the arithmetic function $\sigma_{k-1}(n) = \sum_{\substack{m\mid n\\ m>0}}m^{k-1}$. (See section 1.1 in \cite{diamond}) Obtain the normalized Eisenstein series $E_k(z) = G_k(z)/(2\zeta(k))$, which can be shown to be given by 
\[E_k(z) = \frac{1}{2}\sum_{\substack{(c,d)\in\mathbb Z^2\\\gcd(c,d) = 1}}\frac{1}{(cz+d)^k}\quad \text{or} \quad E_k(z) = \frac{1}{2}\sum_{\gamma\in P_+\backslash \slz}\frac{1}{j(\gamma,z)^k},\] where $P_+ = \cbr{\big(\!\begin{smallmatrix}
    1 & n \\ 0 & 1
\end{smallmatrix}\!\big) : n\in\mathbb Z}$ is the positive part of the parabolic subgroup of $\slz$. It is more elegant to verify that the normalized Eisenstein series is weakly modular of weight $k$ using the last equality above. 

Let $g_2(z) = 60 G_4(z)$ and $g_3(z) = 140G_6(z)$, and define the \textib{discriminant function} $\varDelta\colon\mathcal H\to\mathbb C$ by $\varDelta(z) = (g_2(z))^3-27(g_3(z))^2$. The discriminant function is a nonzero cusp form of weight $12$, with Fourier coefficients $a_0 = 0$, $a_1 = (2\pi)^{12}$. In fact, the only zero of $\varDelta$ is at $\infty$. It follows that the \textib{$j$-invariant} $j\colon \mathcal H\to\mathbb C$ given by $j(z) = 1728(g_2(z))^3/\varDelta(z)$ is holomorphic on $\mathcal H$. The $j$-invariant is $\slz$-invariant, is a surjection, and has a simple pole at $\infty$ with residue $1$.

We summarize what the Eisenstein series for $\varGamma(N)$ look like for weights $k\geq 3$. Let $N$ be a positive integer and let $\overline v \in (\mathbb Z/N\mathbb Z)^2$ be an element of order $2$ written as a row vector, where $v$ is some lift of $\overline v$ to $\mathbb Z^2$ (so $\overline{\,\cdot\,}$ denotes reduction modulo $N$). Let $\delta = \big(\!\begin{smallmatrix}
    a & b \\ c_v & d_v
\end{smallmatrix}\!\big)\in\slz$ with $(c_v,d_v)$ a lift of $\overline v$ to $\mathbb Z^2$, and let $k\geq 3$ be an integer. Lastly, let $\epsilon_N$ be $1/2$ if $N=1,2$ and $1$ otherwise. 

Define the Eisenstein series $E_k^{\overline v}(z)\colon\mathcal H\to\mathbb C$ by \[E_k^{\overline v}(z) =\epsilon_N\sum_{\substack{(c,d)\equiv v\smod N\\\gcd(c,d) = 1}}\frac{1}{(cz+d)^k},\] and it can be shown that \[E_k^{\overline v}(z) = \sum_{\gamma \in(P_+\cap \varGamma(N)\backslash \varGamma(N)\delta)}\frac{1}{j(\gamma,z)^k}.\] Note that when $N = 1$, we have from this definition that $E_k^{\overline v} = E_k$ since there is only one choice of $\overline v$.

A computation yields the fact that for $\gamma\in\slz$, $E_k^{\overline v}|_k[\gamma] = E_k^{\overline{v\gamma}}$, from which it follows that these series are elements of $\mathcal M_k(\varGamma(N))$. By symmetrizing, one can define Eisenstein series for any congruence subgroup $\varGamma$ by
\[E_{k,\varGamma}^{\overline v} = \sum_{\gamma_j\in\varGamma(N)\backslash \varGamma}E_k^{\overline v}|_k[\gamma_j],\] where the $\gamma_j$ constitute a set of coset representatives for $\varGamma(N)\backslash \varGamma$.

For odd weights $k$ and $N = 1,2$, the Eisenstein space $\mathcal E_k(\varGamma(N))$ has dimension $0$, since $-I\in\varGamma(N)$. For $k$ even or for $N>2$, $E_k^{\overline v}$ vanishes at $\infty$ for all $\overline v$ except for $\pm\overline{(0,1)}$. One can show that $E_k^{\overline{(c,d)}}$ is nonvanishing on points in the set $\varGamma(N)(-d/c)$ and vanishes on the other cusps of $\varGamma(N)$. A basis of $\mathcal E_k(\varGamma(N))$ may be formed by choosing a set of vectors $\cbr{\overline v} = \{\overline{(c,d)}\}$ for which the quotients $-d/c$ represent the cusps of $\varGamma(N)$; it follows from the previous sentence that $\{E_k^{\overline v}\}$ is linearly independent. The number of elements in this set is $\varepsilon_\infty$ (for $\varGamma(N)$). Thus for all $k\geq 3$, it is possible to obtain bases for the Eisenstein spaces (recall that for odd weights and $N = 1,2$ the spaces are zero). The basis elements are really cosets of the form $E_k^{\overline v} + \mathcal S_k(\varGamma(N))$, but it is possible to redefine the Eisenstein spaces $\mathcal E_k(\varGamma(N))$ as subspaces of $\mathcal M_k(\varGamma(N))$ so that the basis elements are the Eisenstein series themselves (see section 5.11 in \cite{diamond}).

In view of the definition of the normalized Eisenstein series for $N = 1$, define for any $\overline v\in (\mathbb Z/N\mathbb Z)^2$ of order $N$
\[G_k^{\overline v}(z) = \sideset{}{^\prime}\sum_{(c,d)\equiv v\smod N}\frac{1}{(cz+d)^k}.\] It can be shown that (see section 4.2 in \cite{diamond})
\[G_k^{\overline v}(z) = \frac{1}{\epsilon_N}\sum_{n\in(\mathbb Z/N\mathbb Z)^\times}\zeta_+^n(k)E_k^{n^{-1}\overline v}(z),\] where \[\zeta_+^n(k) = \sum_{\substack{m=1\\m\equiv n\smod N}}^\infty \frac{1}{m^k}\quad \text{for $n\in (\mathbb Z/N\mathbb Z)^\times$}.\] 

The Fourier series expansion of $G_k^{\overline v}(z)$ for $k\geq 3$ and $\overline v\in (\mathbb Z/N\mathbb Z)^2$ of order $N$ is given by \[G_k^{\overline v}(z) = \delta(\overline c_v)\zeta^{\overline d_v}(k) + \frac{(-2\pi i)^k}{N^k(k-1)!}\sum_{n=1}^\infty \sigma_{k-1}^{\overline v}(n)e^{2\pi i n t/N},\] where \[\delta(\overline c_v) = \begin{cases}
    1 & \text{if $\overline c_v =\overline 0$},\\
    0 & \text{otherwise},
\end{cases}\quad \zeta^{\overline d_v}(k) = \sideset{}{^\prime}\sum_{d\equiv d_v\smod N}\frac{1}{d^k},\] and 
\[\sigma_{k-1}^{\overline v}(n) = \sum_{\substack{m\mid n\\n/m \equiv c_v\smod n}}\sgn(m)m^{k-1}\mu_N^{d_vm}.\] In the sum for $\zeta^{\overline d_v}(k)$, we sum over positive and negative $d$, and similarly for $m$ in the sum for $\sigma_{k-1}^{\overline v}$. Similarly to the Eisenstein series, one may form a basis $\cbr{G_k^{\overline v}}$ of the Eisenstein space $\mathcal E_k(\varGamma(N))$ (using the same choice of $\cbr{\overline v}$ as before). Since the $n$-th Fourier coefficient is of the order $n^k$, it follows from \cref{prop: growth of fourier coeffs} that $E_k^{\overline v}$ is a modular form.

\subsection{Dirichlet characters and $L$-functions}
For any positive integer $N$, denote by $Z_N$ the group $\mathbb{Z}/N\mathbb{Z}$.
\begin{definition}
    A \textib{Dirichlet character modulo $N$} is a homomorphism 
    \[\chi\colon Z_N^\times\to \mathbb{C}^\times.\] (We will sometimes suppress ``Dirichlet'' when referring to Dirichlet characters.)
\end{definition}
The product of two Dirichlet characters $\chi,\psi$, defined by pointwise multiplication $(\chi\psi)(n) = \chi(n)\psi(n)$ for $n\in Z_N^\times$, is a Dirichlet character. The trivial map is a Dirichlet character, called the trivial character. Hence the set of Dirichlet characters of $Z_N^\times$ forms a group called the \textib{dual group} of $Z_N^\times$, denoted $\widehat{Z_N^\times}$. Since $Z_N^\times$ is a finite group, the image of any character lies in the roots of unity. Thus the inverse of a Dirichlet character is its complex conjugate character $\overline \chi$, defined by taking the complex conjugate pointwise: $\overline\chi(n) = \overline{\chi(n)}$ for $n\in Z_N^\times$, where $\overline{\,\cdot\,}$ denotes complex conjugation. Note that the only Dirichlet character of $Z_1^\times$ is the trivial character $\mathbf{1}_1$. 

\begin{proposition}
    The dual group $\widehat{Z_N^\times}$ is (noncanonically) isomorphic to $Z_N^\times$; it follows that the number of Dirichlet characters modulo $N$ is $\phi(N)$.
\end{proposition}

\begin{proof}
    This follows from Exercise 5.2.14 in \cite{df}, which states that finite Abelian groups are (noncanonically) self-dual. 

    Let $G = \abr{x_1}\times\cdots\times\abr{x_r}$ be a finite Abelian group (recall the structure theorem for finitely generated Abelian groups). Define characters $\chi_i\colon \abr{x_i}\to \mathbb C^\times$ by $x_i\mapsto e^{2\pi i/\abs{x_i}}$ for $1\leq i\leq r$, which have order $\abs{x_i}$ in $\widehat{\abr{x_i}}$. It is evident that the group $G^\prime = \abr{\chi_1}\times\cdots\times \abr{\chi_r}$ is isomorphic to $G$.

    Define the homomorphism $\rho\colon G^\prime\to \widehat G$ by $(\chi_1^{e_1},\dots,\chi_r^{e_r})\mapsto \chi_1^{e_1}\pi_1\cdots\chi_r^{e_r}\pi_r$, where $\pi_i$ denotes the projection $G\to \abr{x_i}$. It is evident that $\ker\rho$ is trivial. Let $f\colon G\to\mathbb C^\times$ be a character of $\widehat G$, and let $\iota_i\colon \abr{x_i}\to G$ denote the inclusion $x_i\mapsto (1,\dots,1,x_i,1,\dots,1)$. Then $f\iota_i\in \widehat{\abr{x_i}}$, and a preimage of $f$ under $\rho$ is $(f\iota_1,\dots,f\iota_r)$. The result follows.
\end{proof}

\begin{proposition}
    The groups $Z_N^\times$ and $\widehat{Z_N^\times}$ satisfy the following orthogonality relations:
\begin{equation}\label{eqn: ortho rels}
    \sum_{n\in Z_N^\times}\chi(n) = \begin{cases}
    \phi(N) & \text{if $\chi = \mathbf 1$}, \\
    0 & \text{if $\chi\neq \mathbf 1$},
\end{cases}\quad \sum_{\chi\in \widehat{Z_N^\times}}\chi(n) = \begin{cases}
    \phi(N) & \text{if $n = 1$}, \\
    0 & \text{if $n\neq 1$}.
\end{cases}
\end{equation}
\end{proposition}

\begin{proof}
    Let $\chi$ be a Dirichlet character. If $\chi= \mathbf 1$, then $\sum_{n\in Z_N^\times}\chi(n) = \phi(N)$. Suppose $\chi\neq \mathbf 1$, so that there exists $m\in Z_N^\times$ for which $\chi(m)\neq 1$. Then $\sum_{n\in Z_N^\times}\chi(n) = \sum_{n\in Z_N^\times}\chi(mn) = \chi(m)\sum_{n\in Z_N^\times}\chi(n)$, so that $\sum_{n\in Z_N^\times}\chi(n) = 0$.

    Similarly, let $n\in Z_N^\times$. If $n = 1$, $\sum_{\chi\in \widehat{Z_N^\times}}\chi(n) = \phi(N)$. If $n\neq 1$, there exists a character $\eta$ that is not $1$ on $n$, and similarly obtain the equality $\sum_{\chi\in \widehat{Z_N^\times}}\chi(n) =\sum_{\chi\in \widehat{Z_N^\times}}(\eta\chi)(n) =\eta(n)\sum_{\chi\in \widehat{Z_N^\times}}\chi(n)$, from which $\sum_{\chi\in \widehat{Z_N^\times}}\chi(n) = 0$ follows.
\end{proof}

Any Dirichlet character $\chi$ modulo $d$ may be lifted to a character $\chi_N$ modulo $N$ when $d\mid N$, by the rule $\chi_N(n\smod N) = \chi(n\smod d)$ for all $n\in\mathbb{Z}$ coprime to $N$. In other words, if $\pi_{N,d}\colon Z_N^\times\to Z_d^\times$ is the natural projection, then $\chi_N = \chi\circ \pi_{N,d}$.

However, given positive $N,d$ with $d\mid N$ and $\chi$ a character modulo $N$, it is not always possible to find a character $\chi_d$ modulo $d$ such that $\chi = \chi_d\circ \pi_{N,d}$. But for every character modulo $N$ there exists a divisor $d$ of $N$ and a character $\chi_d$ modulo $d$ such that $\chi = \chi_d\circ \pi_{N,d}$. 
\begin{definition}
    The \textib{conductor} of a Dirichlet character $\chi$ modulo $N$ is the smallest positive divisor $d$ of $N$ such that there exists a Dirichlet character $\chi_d$ modulo $d$ such that $\chi = \chi_d\circ \pi_{N,d}$, equivalently, such that $\chi$ is trivial on the normal subgroup 
    \[\ker(\pi_{N,d}) = \{n\in Z_N^\times : n\equiv 1\smod d\}.\]

    A Dirichlet character modulo $N$ is \textib{primitive} if its conductor is $N$.
\end{definition}

Note that the only character modulo $N$ with conductor $1$ is the trivial character $\mathbf 1_N$, so that the trivial character $\mathbf 1_N$ is primitive only for $N = 1$.

Any Dirichlet character $\chi$ modulo $N$ extends to a function (abusing notation) $\chi\colon Z_N\to\mathbb C$ by the rule $\chi(n) = 0$ for noninvertible elements $n$ of the ring $Z_N$. Further extend $\chi$ to a function $\chi\colon\mathbb Z\to\mathbb C$ by the rule $\chi(n) = \chi(n\smod N)$. The resulting map $\chi\colon \mathbb Z\to \mathbb C$ is a totally multiplicative (set) function; that is, $\chi(nm) = \chi(n)\chi(m)$ for all $n,m\in \mathbb Z$.

For example, the extension of the trivial character $\mathbf 1_N$ to a function on $\mathbb Z$ is given by
\[\mathbf 1_N(n) = \begin{cases}
    1 & \text{if $\gcd(n,N)=1$},\\
    0 & \text{if $\gcd(n,N)\neq1$}.
\end{cases}\]
Also note that the extension of any Dirichlet character $\chi$ satisfies 
\[\chi(0) = \begin{cases}
    1 & \text{if $N = 1$},\\
    0 & \text{if $N > 1$}.
\end{cases}\]
Obtain new orthogonality relations from the ones appearing in \cref{eqn: ortho rels} by summing from $0$ to $N-1$ in the first orthogonality relation and by taking $n\in\mathbb Z$ in the second:
\begin{equation}\label{eqn: new ortho rels}
    \sum_{n=0}^{N-1}\chi(n) = \begin{cases}
        \phi(N) & \text{if $\chi = \mathbf 1$},\\
        0 & \text{if $\chi\neq \mathbf 1$},
    \end{cases}\quad\sum_{\chi\in \widehat{Z_N^\times}}\chi(n) = \begin{cases}
        \phi(N) & \text{if $n\equiv 1\smod N$},\\
        0 & \text{if $n\not\equiv 1\smod N$}.
    \end{cases} 
\end{equation}
\begin{definition}
    Let $\chi$ be a Dirichlet character modulo $N$. The \textib{Gauss sum} of $\chi$ is the complex number
    \[g(\chi) = \sum_{n=0}^{N-1}\chi(n)\mu_N^n,\] where $\mu_N = e^{2\pi i/N}$.
\end{definition}
\begin{proposition}
    If $\chi$ is a primitive character modulo $N$, then for any integer $m$ we have 
    \[g(\chi) = \sum_{n=0}^{N-1}\chi(n)\mu_N^{nm} = \overline\chi(m)g(\chi).\] Furthermore, the Gauss sum of a primitive character is nonzero.
\end{proposition}
\begin{proof}
    \tbd
\end{proof}
\begin{lemma}
    Let $N$ be a positive integer. If $N = 1,2$, then every Dirichlet character $\chi$ modulo $N$ satisfies $\chi(-1)=1$. If $N>2$, then the number of Dirichlet characters modulo $N$ is even, of which half satisfy $\chi(-1) = 1$ and the other half satisfy $\chi(-1) = -1$.
\end{lemma}
\begin{proof}
    The extensions of a Dirichlet character to functions on $\mathbb Z$ are unique, so it suffices to study Dirichlet characters given by homomorphisms $Z_N^\times\to\mathbb C^\times$. For $N = 1,2$, $\phi(N) = 1$ so that the only Dirichet character is the trivial one. For $N>2$, $\phi(N)$ is even, so there are an even number of characters modulo $N$.

    For $N>2$, the map $\widehat{Z_N^\times}\to Z_2$ given by evaluation at $N-1$ (which corresponds to evaluation at $-1$) is a nontrivial homomorphism since there exists a character which is not $1$ on $N-1$. By the first isomorphism theorem, the result follows.
\end{proof}
Dirichlet characters are used to decompose the vector spaces $\mathcal M_k(\varGamma_1(N)), \mathcal S_k(\varGamma_1(N)), \mathcal E_k(\varGamma_1(N))$ into direct sums of interesting subspaces.
\begin{proposition}\label{prop: eigenspace decomp}
    For each Dirichlet character $\chi$ modulo $N$, define the \textib{$\chi$-eigenspace} of $\mathcal M_k(\varGamma_1(N))$ by 
    \[\mathcal M_k(N,\chi) = \{f\in \mathcal M_k(\varGamma_1(N)) : f|_k[\gamma] = \chi(d_\gamma)f \text{ for all } \gamma\in\varGamma_0(N)\},\] where $d_\gamma$ denotes the lower right entry of $\gamma$. Then the following decomposition holds:
    \[\mathcal M_k(\varGamma_1(N)) = \bigoplus_\chi\mathcal M_k(N,\chi).\] With similar definitions of $\chi$-eigenspaces for $\mathcal S_k(\varGamma_1(N))$ and $\mathcal E_k(\varGamma_1(N))$, we obtain the decompositions
    \[\mathcal S_k(\varGamma_1(N)) = \bigoplus_\chi S_k(N,\chi),\quad \mathcal E_k(\varGamma_1(N)) = \bigoplus_\chi \mathcal E_k(N,\chi).\]
\end{proposition}
\begin{proof}
    \tbd 
\end{proof}

\begin{definition}
    To every Dirichlet character $\chi$ modulo $N$ there is an associated Dirichlet series, called a \textib{Dirichlet $L$-function}, given by 
    \[L(s,\chi) = \sum_{n=1}^\infty \frac{\chi(n)}{n^s} = \prod_{p\in\mathbb P}\frac{1}{1-\chi(p)p^{-s}},\quad \Re(s)>1.\] Here $\mathbb P$ denotes the set of prime numbers and the second equality above is obtained by using the fundamental theorem of arithmetic combined with the fact that $(1-\chi(p)p^{-s})^{-1}$ is given by a geometric series.
\end{definition}
It can be shown that these $L$-functions extend meromorphically to the complex plane in $s$, with entire extension unless $\chi = \mathbf 1_N$, which has a simple pole at $s = 1$. (Indeed, $L(s,\mathbf 1_N) = \prod_{p\nmid N}(1-p^{-s})^{-1} = \prod_{p\in\mathbb P}(1-p^{-s})^{-1}\prod_{p\mid N}(1-p^{-s}) = \zeta(s)\prod_{p\mid N}(1-p^{-s})$.) Furthermore, when $\chi(-1) = 1$, the functional equation satisfied by $L(s,\chi)$ is 
\[\pi^{-s/2}\varGamma\Big(\frac{s}{2}\Big)N^sL(s,\chi) = \pi^{-(1-s)/2}\varGamma\Big(\frac{1-s}{2}\Big)g(\chi)L(1-s,\chi),\] and when $\chi(-1)=-1$, the functional equation is 
\[\pi^{-(s+1)/2}\varGamma\Big(\frac{s+1}{2}\Big)N^sL(s,\chi) = -i\pi^{-(2-s)/2}\varGamma\Big(\frac{2-s}{2}\Big)g(\chi)L(1-s,\chi).\] Here $\varGamma$ denotes Euler's Gamma function (extended to $\mathbb C$). 

Before defining $L$-functions for modular forms, we make a few observations. Let $\varGamma$ be a congruence subgroup, so that by definition there exists a positive integer $N$ such that $\varGamma(N)\subset \varGamma$, from which it follows that $\mathcal M_k(\varGamma)\subset \mathcal M_k(\varGamma(N))$. Furthermore, $\varGamma_1(N^2)$ is contained in 
\[\begin{pmatrix}
    N & 0 \\ 0 & 1
\end{pmatrix}^{\!-1}\varGamma(N)\begin{pmatrix}
    N & 0 \\ 0 & 1
\end{pmatrix} = \Big\{\abcd\in \slz : c\equiv 0 \smod {N^2},~ a\equiv d\equiv 1\smod N\Big\},\] from which we have
\[\begin{pmatrix}
    N & 0 \\ 0 & 1
\end{pmatrix}\varGamma_1(N^2)\subset \varGamma(N)\begin{pmatrix}
    N & 0 \\ 0 & 1
\end{pmatrix}.\] So if $f\in \mathcal M_k(\varGamma(N))$,
\[f(Nz) = f\big(\big(\!\begin{smallmatrix}
    N & 0 \\ 0 & 1
\end{smallmatrix}\!\big)z\big) = N^{-k/2}\Big(f|_k\big[\big(\!\begin{smallmatrix}
    N & 0 \\ 0 & 1
\end{smallmatrix}\!\big)\big]\Big)(z)\] and $N^{-k/2}f|_k\big[\big(\!\begin{smallmatrix}
    N & 0 \\ 0 & 1
\end{smallmatrix}\!\big)\big]\in \mathcal M_k(\varGamma_1(N^2))$. Indeed, $f(Nz)$ is holomorphic on $\mathcal H$ since $f$ is holomorphic on $\mathcal H$. For $\gamma\in\varGamma_1(N^2)$, $\big(\!\begin{smallmatrix}
    N & 0 \\ 0 & 1
\end{smallmatrix}\!\big)\gamma\in \varGamma(N)\big(\!\begin{smallmatrix}
    N & 0 \\ 0 & 1
\end{smallmatrix}\!\big)$ so that
\[N^{-k/2}f|_k\big[\big(\!\begin{smallmatrix}
    N & 0 \\ 0 & 1
\end{smallmatrix}\!\big)\big]|_k[\gamma] = N^{-k/2}f|_k\big[\gamma^\prime\big(\!\begin{smallmatrix}
    N & 0 \\ 0 & 1
\end{smallmatrix}\!\big)\big] = N^{-k/2}f|_k\big[\big(\!\begin{smallmatrix}
    N & 0 \\ 0 & 1
\end{smallmatrix}\!\big)\big],\] where $\gamma^\prime$ is an element of $\varGamma(N)$. Lastly, let $\alpha\in \slz$, and note that $\slz$ is normal in $\GL_2(\mathbb Z)$ so that $\big(\!\begin{smallmatrix}
    N & 0 \\ 0 & 1
\end{smallmatrix}\!\big)\alpha = \alpha^\prime \big(\!\begin{smallmatrix}
    N & 0 \\ 0 & 1
\end{smallmatrix}\!\big)$ for some $\alpha^\prime\in\slz$. Then 
\[N^{-k/2}\Big(f|_k\big[\big(\!\begin{smallmatrix}
    N & 0 \\ 0 & 1
\end{smallmatrix}\!\big)\big]|_k[\alpha]\Big)(z) = N^{-k/2}\Big(f|_k\big[\alpha^\prime\big(\!\begin{smallmatrix}
    N & 0 \\ 0 & 1
\end{smallmatrix}\!\big)\big]\Big)(z) = j\big(\alpha^\prime\big(\!\begin{smallmatrix}
    N & 0 \\ 0 & 1
\end{smallmatrix}\!\big),z\big)^{-k}f(\alpha^\prime(Nz)) = (f|_k[\alpha^\prime])(Nz),\] which is holomorphic at $\infty$ since $f\in\mathcal M_k(\varGamma(N))$. If the  series expansion of $f$ at $\infty$ is $f(z) = \sum_{n =0}^\infty a_ne^{2\pi i n z/N}$, we have the Fourier expansion
\[f(Nz) = \sum_{n =0}^\infty a_ne^{2\pi i n z},\] and vice versa. So for the purposes of defining $L$-functions for modular forms, it suffices to define them for modular forms in $\mathcal M_k(\varGamma_1(N))$. We require a few preliminary results about Dirichlet series:
\begin{lemma}\label{lem: miyake lem 3.2.1}
    Assume that both $\sum_{n=1}^\infty a_nn^{-s}$ and $\sum_{n=1}^\infty b_nn^{-s}$ are absolutely convergent at $s = \sigma_0$ with $\sigma_0>0$ real. If $\sum_{n=1}^\infty a_nn^{-s} = \sum_{n=1}^\infty b_nn^{-s}$ on $\Re(s)\geq \sigma_0$, then $a_n = b_n$ for all $n$.
\end{lemma}
\begin{proof}
    By taking differences, it suffices to show that if $\sum_{n=1}^\infty a_nn^{-s} = 0$, then $a_n = 0$ for all $n$. Since $\sum_{n=1}^\infty a_nn^{-s}$ is absolutely convergent at $s = \sigma_0$, it is absolutely and uniformly convergent on $\Re(s)\geq \sigma_0$. Suppose there exists a smallest integer $m$ such that $a_m \neq 0$.
    
    By hypothesis, $-a_m = \sum_{n=m+1}^\infty a_n (n/m)^{-\sigma}$. Let $\sigma = \Re(s)\geq \sigma_0$, and note that for $n>m^2$, we have $(n/m)^{-\sigma}<n^{-\sigma/2}$. It follows that
    \begin{align*}
        \abs{a_m} &\leq \sum_{n=m+1}^\infty \abs{a_n} (n/m)^{-\sigma}\\
        &\leq \sum_{n=m+1}^{m^2} \abs{a_n} (n/m)^{-\sigma} + \sum_{n=m^2+1}^\infty \abs{a_n} n^{-\sigma/2}.
    \end{align*}
    Choose $N$ large enough so that
    \[\sum_{n=N+1}^\infty \abs{a_n}n^{-\sigma_0}\leq \abs{a_m}/3,\]
    and choose $\sigma>2\sigma_0$ large enough so that 
    \[\sum_{n=m+1}^{m^2} \abs{a_n} (n/m)^{-\sigma} + \sum_{n=m^2+1}^N \abs{a_n}n^{-\sigma/2}\leq \abs{a_m}/3.\] With these choices, $\abs{a_m}\leq 2\abs{a_m}/3$, which is a contradiction.
\end{proof}
\begin{proposition}\label{prop: miyake prop pre-lemma 4.3.3}
    Let $f$ be a holomorphic function on $\mathcal H$ such that: 
    \begin{enumerate}[label = \textup{(\arabic*)}]
        \item The Fourier expansion for $f$, \[f(z) = \sum_{n=0}^\infty a_ne^{2\pi i nz},\] converges absolutely and uniformly on compact subsets of $\mathcal H$, and
        \item there exists $v>0$ such that \[f(z) = \mathcal O(\Im(z)^{-v}) \quad \textup{($\Im(z)\to 0$)}\] uniformly on $\Re(z)$. 
    \end{enumerate}
    Then $a_n = \mathcal O(n^v)$.
\end{proposition}
\begin{proof}
    By hypothesis, there exists a constant $M>0$ such that $\abs{f(z)}\leq M\Im(z)^{-v}$. Then \begin{align*}
        \abs{a_n} &= \abs{\int_0^1 f(x+iy)e^{-2\pi i n (x+iy)}\dd x}\\
        &\leq My^{-v}e^{2\pi n y}.
    \end{align*} By taking $y = 2/n$, it follows that $\abs{a_n} \leq (Me^{4\pi}2^{-v})n^v$.
\end{proof} A converse to the above proposition is \cref{lem: miyake lem 4.3.3}
\begin{lemma}[Phragmen-Lindel\"of]\label{lem: miyake lem 4.3.4}
    For two real numbers $v_1,v_2$ with $v_1<v_2$, let 
    \[F = \cbr{s\in \mathbb C : v_1\leq \Re(s)\leq v_2}.\] Let $\phi$ be a holomorphic function on a domain containing $F$ satisfying \[\abs{\phi(s)} = \mathcal O\big(e^{\abs{\tau}^\delta}\big)\quad \textup{($\abs{\tau}\to\infty$, with $s = \sigma + i\tau$),}\]
    uniformly on $F$ with $\delta >0$. For a real number $b$, if 
    \[\abs{\phi(s)} = \mathcal O(\abs{\tau}^b)\quad (\abs{\tau}\to\infty) \quad \text{on $\Re(s) = v_1$ and $\Re(s) = v_2$,}\]
    then $\abs{\phi(s)} = \mathcal O(\abs{\tau}^b)$ as $\abs{\tau}\to\infty$ uniformly on $F$.
\end{lemma}
\begin{proof}
    By hypothesis, there exists $L>0$ such that $\abs{\phi(s)}\leq Le^{\abs{\tau}^\delta}$. Suppose first that $b = 0$. Then there exists $M>0$ such that $\abs{\phi(s)}\leq M$ on $\Re(s) = v_1$ and on $\Re(s) = v_2$. Let $m$ be a positive integer with $m\equiv 2\smod 4$ and let $s = \sigma + i \tau$. Since $\Re(s^m) = \Re((\sigma  + i\tau)^m)$ is a polynomial of $\sigma$ and $\tau$ with highest term of $\tau$ given by $-\tau^m$, we have 
    \[\Re(s^m) = -\tau^m + \mathcal O(\abs{\tau}^{m-1})\quad (\abs{\tau}\to\infty),\] uniformly on $F$. With $m$ even, $\Re(s^m)$ has an upper bound on $F$. Choose $m$ and $N$ so that $m>\delta$ and $\Re(s^m)<N$, so that for any $\varepsilon>0$,
    \[\big|\phi(s)e^{\varepsilon s^m}\big|\leq Me^{\varepsilon N}\quad \text{on $\Re(s) = v_1$ and $\Re(s) = v_2$,}\]
    and
    \[\big|\phi(s)e^{\varepsilon s^m}\big| = \mathcal O\big(e^{\abs{\tau}^\delta -\varepsilon\tau^m + K\abs{\tau}^{m-1}}\big)\quad (\abs{\tau}\to\infty\text{, so this quantity tends to zero in the limit}),\] uniformly on $F$. By the maximum principle, it follows that $\big|\phi(s)e^{\varepsilon s^m}\big|\leq Me^{\varepsilon N}$ for $s\in F$. Let $\varepsilon$ tend to $0$ so that $\abs{\phi(s)}\leq M$, that is, $\phi(s) = \mathcal O(\abs{\tau}^0)$.

    Now let $b\neq 0$. Let $\psi(s) = (s-v_1+1)^b = e^{b\log(s-v_1+1)}$ (using the principal branch of the logarithm); note that $\psi$ is holomorphic. Since $\Re(\log(s-v_1+1)) = \log\abs{s-v_1+1}$, we have 
    \[\abs{\psi(s)} = \abs{s-v_1+1}^b\sim \abs{\tau}^b\quad (\abs{\tau}\to\infty),\] uniformly on $F$ (the notation $\abs{s-v_1+1}^b\sim \abs{\tau}^b$ means that $\lim_{\abs{\tau}\to\infty} \abs{s-v_1+1}^b/\abs{\tau}^b = 1$).

    Let $\phi_1(s) = \phi(s)/\psi(s)$. The function $\phi_1$ satisfies the same assumptions as $\phi$ with $b = 0$, so by repeating the above argument with $\phi_1(s)$ in place of $\phi(s)$, we find that $\phi_1(s)$ is bounded on $F$. It follows that $\abs{\phi(s)} = \mathcal O(\abs{\tau}^b)$ for $\abs{\tau}\to\infty$. \sai{I may rewrite the last part of this proof.}
\end{proof}
\begin{definition}\label{def: L function}
    To a holomorphic function $f\colon \mathcal H\to \mathbb C$ satisfying \cref{prop: miyake prop pre-lemma 4.3.3}  with Fourier series expansion $f(z) = \sum_{n=0}^\infty a_ne^{2\pi i nz}$, we associate to it an \textib{$L$-function}, given by the Dirichlet series
    \[L(s,f) = \sum_{n=1}^\infty a_nn^{-s}.\] Furthermore, for $N>0$, let 
    \[\varLambda_N(s,f) = (2\pi/\sqrt{N})^{-s}\varGamma(s)L(s,f).\qedhere\]
\end{definition}

In the above definition, since $a_n = \mathcal O(n^v)$, the Dirichlet series $L(s,f)$ converges absolutely and uniformly on compact subsets of $\cbr{s\in\mathbb C : \Re(s)>1+v}$. We are interested in $L$-functions of modular forms $f\in\mathcal M_k(\varGamma_1(N))$, which do satisfy \cref{prop: miyake prop pre-lemma 4.3.3}. 

For a positive integer $N$, define \textib{$\omega_N$}$\in\GL_2(\mathbb R)$ by 
\[\omega_n = \begin{pmatrix}
    0 & -1 \\ N & 0
\end{pmatrix}.\]
Its action on functions on $\mathcal H$ appears in several results forthcoming, so we define it here.

\begin{theorem}[Hecke]\label{thm: miyake thm 4.3.6}
    Fix positive integers $k,N$. Let $f,g$ be holomorphic functions on $\mathcal H$ satisfing the hypotheses of \cref{prop: miyake prop pre-lemma 4.3.3}, with Fourier expansions $f(z) = \sum_{n=0}^\infty a_ne^{2\pi i nz}$ and $g(z) = \sum_{n=0}^\infty b_ne^{2\pi i nz}$. Then the following conditions are equivalent: 
    \begin{enumerate}[label = \textup{(\alph*)}]
        \item $g(z) = (f|_k[\omega_N])(z) = (\sqrt{N}z)^{-k}f(-1/Nz)$
        \item Both $\varLambda_N(s,f)$ and $\varLambda(s,g)$ can be analytically continued to the whole $s$-plane, satisfy the functional equation \[\varLambda_N(s,f) = i^k\varLambda_N(k-s,g),\] and the function \[\varLambda_N(s,f) + \frac{a_0}{s} + \frac{i^kb_0}{k-s}\] is holomorphic on the whole $s$-plane and is bounded on any vertical strip.
    \end{enumerate}
\end{theorem}
\begin{proof}
    Suppose (a) holds. Since there exists $v>0$ such that $a_n = \mathcal O(n^v)$ and $b_n = \mathcal O(n^v)$, 
    \[\sum_{n=1}^\infty \abs{a_n}e^{-2\pi n t/\sqrt{N}}\quad (t>0)\] and 
    \[\sum_{n=1}^\infty \int_0^\infty \abs{a_n}t^\sigma e^{-2\pi n t/\sqrt{N}}t^{-1}\dd t \quad (\sigma > v+1)\] are convergent. Hence for $\Re(s)>v+1$, 
    \begin{align*}
        \varLambda_N(s,f) &= \sum_{n=1}^\infty a_n(2\pi n/\sqrt{N})^{-s}\int_0^\infty t^{s-1}e^{-t}\dd t\\
        &= \sum_{n=1}^\infty \int_0^\infty a_nt^s e^{-2\pi n t/\sqrt{N}}\\
        &= \int_0^\infty t^s\bigg(\sum_{n=1}^\infty a_n e^{-2\pi nt /\sqrt{N}}\bigg)t^{-1}\dd t\\
        &= \int_0^\infty t^s[f(it/\sqrt{N}) - a_0]t^{-1}\dd t\\
        &= -\frac{a_0}{s} + \int_1^\infty t^{-s}f(i/\sqrt{N}t)t^{-1}\dd t + \int_1^\infty t^s[f(it/\sqrt{N}) - a_0]t^{-1}\dd t.
    \end{align*}
    Since $g(z) = (\sqrt{N}z)^{-k}f(-1/Nz)$, the above equality becomes 
    \begin{equation}\label{eqn: new equality 4.3.6}
        \varLambda_N(s,f) = -\frac{a_0}{s} - \frac{i^kb_0}{k-s} + i^k\int_1^\infty t^{k-s}[g(it/\sqrt{N})-b_0]t^{-1}\dd t + \int_1^\infty t^s[f(it/\sqrt{N}) - a_0]t^{-1}\dd t,
    \end{equation} which holds for $\Re(s)>\max\{k,v+1\}$. By \cref{lem: miyake lem 4.3.3}, as $t$ tends to $\infty$, $f(it)-a_0 = \mathcal O(e^{-2\pi t})$ and $g(it)-a_0 = \mathcal O(e^{-2\pi t})$ so that 
    \[\int_1^\infty t^s[f(it/\sqrt{N}) - a_0]t^{-1}\dd t\quad\text{and}\quad \int_1^\infty t^{k-s}[g(it/\sqrt{N})-b_0]t^{-1}\dd t\] are convergent absolutely and uniformly on any vertical strip. Therefore the functions these integrals define are holomorphic on the whole $s$-plane. It follows by \cref{eqn: new equality 4.3.6} that $\varLambda_N(s,f)$ is a meromorphic function on the whole $s$-plane with $\varLambda_N(s,f) + a_0/s+i^kb_0/(k-s)$ an entire bounded function on any vertical strip. Similarly analytically continue $\varLambda_N(s,g)$ to the whole $s$-plane, satisfying 
    \begin{equation}\label{eqn: similar for g 4.3.6}
        i^k\varLambda_N(k-s,g) = -\frac{a_0}{s}-\frac{i^kb_0}{k-s} + i^k\int_1^\infty t^{k-s}[g(it/\sqrt{N}-b_0)]t^{-1}\dd t + \int_1^\infty t^s[f(it/\sqrt{N})-a_0]t^{-1}\dd t.
    \end{equation}
    It follows from equations \labelcref{eqn: new equality 4.3.6} and \labelcref{eqn: similar for g 4.3.6} that $\varLambda_N(s,f) = i^k\varLambda_N(k-s,g)$.

    Conversely, suppose that (b) holds. Since $e^{-e^x}e^{\sigma x}$ is a Schwartz function for $\sigma>0$, the inverse Mellin transform 
    \[e^{-t} = \frac{1}{2\pi i}\int_{\Re(s) = \sigma}\varGamma(s)t^{-s}\dd s\]
    holds for $\sigma>0$. It follows that 
    \[f(iy) = a_0 + \frac{1}{2\pi i}\sum_{n=1}^\infty a_n\int_{\Re(s) =\alpha}\varGamma(s)(2\pi n y)^{-s}\dd s\] for any $\alpha>0$. Let $\alpha>v+1$, so that $L(s,f) = \sum_{n=1}^\infty a_nn^{-s}$ is uniformly convergent and bounded on $\Re(s) = \alpha$. In this case, Stirling's estimate $\varGamma(s)\sim \sqrt{2\pi} \tau^{\sigma -1/2}e^{-\pi\abs{\tau}/2}$ (for $s = \sigma + i\tau$ and $\abs{\tau}\to\infty$) shows that $\varLambda_N(s,f) = (2\pi/\sqrt{N})^{-s}\varGamma(s)L(s,f)$ is absolutely integrable, so that the order of integration and summation above may be interchanged to obtain
    \begin{equation}\label{eqn: integral for Hecke lemma}
        f(iy) = a_0 + \frac{1}{2\pi i} \int_{\Re(s) = \alpha} (\sqrt{N}y)^{-s}\varLambda_N(s,f)\dd s.
    \end{equation}
    Since $L(s,f)$ is bounded on $\Re(s) = \alpha$, by Stirling's estimate we have for any $\mu>0$ that 
    \begin{equation}\label{eqn: varLambda big O}
        |\varLambda_N(s,f)| = \mathcal O(|\Im(s)|^{-\mu})\quad (|\Im(s)|\to\infty)
    \end{equation}
    on $\Re(s) = \alpha$. Choose $\beta$ so that $k-\beta > v+1$. Using a similar argument we deduce that for any $\mu>0$, 
    \[|\varLambda_N(s,f)| = |\varLambda_N(k-s,g)| = \mathcal O(|\Im(s)|^{-\mu})\quad (|\Im(s)|\to\infty)\] on $\Re(s) = \beta$. By assumption, $\varLambda_N(s,f) + a_0/s+i^kb_0/(k-s)$ is bounded on the vertical strip $\beta\leq \Re(s)\leq \alpha$. Thus for any $\mu>0$, \cref{lem: miyake lem 4.3.4} implies that \cref{eqn: varLambda big O} holds uniformly on the strip $\beta\leq \Re(s)\leq \alpha$.
    
    We may also assume that $\alpha>k$ and $\beta<0$. Observe that $(\sqrt{N}y)^{-s}\varLambda_N(s,f)$ has simple poles at $s = 0,k$ with corresponding residues $-a_0,(\sqrt{N}y)^{-k}i^kb_0$. Combined with the fact that \cref{eqn: varLambda big O} holds uniformly on the strip $\beta\leq \Re(s)\leq \alpha$, we may change the path of integration from $\Re(s) = \alpha$ to $\Re(s) = \beta$ in \cref{eqn: integral for Hecke lemma} to obtain
    \[f(iy) = (\sqrt{N}y)^{-k}i^kb_0 + \frac{1}{2\pi i} \int_{\Re(s) = \beta} (\sqrt{N}y)^{-s}\varLambda_N(s,f) \dd s.\]
    From the functional equation in (b), we have 
    \begin{align*}
        f(iy) &= (\sqrt{N}y)^{-k}i^kb_0 + \frac{1}{2\pi i} \int_{\Re(s) = \beta} (\sqrt{N}y)^{-s}i^k\varLambda_N(k-s,g) \dd s\\
        &= i^k\bigg[(\sqrt{N}y)^{-k}b_0 + \frac{1}{2\pi i} \int_{\Re(s) = k-\beta} (\sqrt{N}y)^{s-k}\varLambda_N(s,g) \dd s\bigg]\\
        &= i^k(\sqrt{N}y)^{-k}g(-1/iNy).
    \end{align*}
    Since $f,g$ are holomorphic on $\mathcal H$, it follows that $f(z) = i^k(\sqrt{N}z/i)^{-k}g(-1/Nz)$, or equivalently $g(z) = (\sqrt{N}z)^{-k}f(-1/Nz)$.
\end{proof}
\begin{corollary}\label{cor: miyake cor 4.3.7}
    If $f(z)\in \mathcal S_k(N,\chi)$, then because any element of $\mathcal S_k(N,\chi)$ satisfies the conditions of \cref{prop: miyake prop pre-lemma 4.3.3} \textup{(}see \cref{cor: cuspform growth corollary}\textup{)}, it follows that $\varLambda_N(s,f)$ is holomorphic in $s$ and satisfies the functional equation
    \[\varLambda_N(s,f) = i^k\varLambda_N(k-s,f|_k[\omega_N]).\]
\end{corollary}

\newpage\section{Hecke operators}
In this section we define Hecke operators on modular forms and show that they form an inner product space with the Petersson inner product, and investigate simultaneous eigenfunctions of Hecke operators \tbd
\subsection{Double coset operators}
Let $\varGamma_1,\varGamma_2$ be congruence subgroups of $\slz$, so also subgroups of $\glqp$. For $\alpha\in\glqp$, the set
\[\varGamma_1\alpha\varGamma_2 = \{\gamma_1\alpha\gamma_2 : \gamma_1\in\varGamma_1, \gamma_2\in\varGamma_2\}\] is a \textib{double coset} in $\glqp$. The group $\varGamma_1$ acts on the double coset $\varGamma_1\alpha\varGamma_2$ on the left by multiplication, partitioning it into orbits of the form $\varGamma_1\beta$, where $\beta = \gamma_1\alpha\gamma_2$ is some representative for this orbit. We show that the orbit space  $\varGamma_1\backslash\varGamma_1\alpha\varGamma_2$ is a finite disjoint union $\bigsqcup_j\varGamma_1\beta_j$ for some choice of representatives $\beta_j$.
\begin{lemma}\label{lem: conjugation congruence subgroup}
    Let $\alpha\in\glqp$ and $\varGamma$ be a congruence subgroup. Then $\alpha^{-1}\varGamma\alpha\cap\slz$ is also a congruence subgroup.
\end{lemma}
\begin{proof}
    There exists $N$ such that $\varGamma(N)$ is contained in $\varGamma$. Let $M$ be the least common multiple of $N$ and the entries of the matrices $\alpha$ and $\alpha^{-1}$ so that $\varGamma(M)\subset \varGamma$ and $M\alpha, M\alpha^{-1}$ are integer-valued matrices.
    
    Observe that $\varGamma(M^3)$ is contained in $I + M^3\Mat_2(\mathbb Z)$ so that 
    \[\alpha\varGamma(M^3)\alpha^{-1}\subset \alpha(I + M^3\Mat_2(\mathbb Z))\alpha^{-1} = I + M\cdot M\alpha \Mat_2(\mathbb Z) M\alpha^{-1}\subset I+M\Mat_2(\mathbb{Z}).\]
    
    Elements of $\alpha\varGamma(M^3)\alpha^{-1}$ have determinant $1$ so that $\alpha\varGamma(M^3)\alpha^{-1}\subset \slz$, from which it follows that $\alpha\varGamma(M^3)\alpha^{-1}\subset \varGamma(M)$. Hence $\varGamma(M^3)\subset \alpha^{-1}\varGamma(M)\alpha \subset \alpha^{-1}\varGamma\alpha$, and so $\varGamma(M^3)\subset \alpha^{-1}\varGamma\alpha\cap \slz$.
\end{proof}
\begin{lemma}\label{lem: double coset coset/orbit spaces}
    Let $\alpha\in\glqp$ and $\varGamma_1,\varGamma_2$ be congruence subgroups. Let $\varGamma_3 = \alpha^{-1}\varGamma_1\alpha\cap \varGamma_2\subset \varGamma_2$. Then the map $\varGamma_2\to \varGamma_1\alpha\varGamma_2$ given by left multiplication by $\alpha$, $\gamma_2\mapsto\alpha\gamma_2$, induces a natural bijection of the coset space $\varGamma_3\backslash\varGamma_2$ to the orbit space $\varGamma_1\backslash\varGamma_1\alpha\varGamma_2$.
\end{lemma} In other words, $\{\gamma_{2,j}\}$ is a set of coset representatives for $\varGamma_3\backslash \varGamma_2$ if and only if $\{\alpha\gamma_{2,j}\}$ is a set of orbit representatives for $\varGamma_1\backslash \varGamma_1\alpha\varGamma_2$.
\begin{proof}
    It is evident that the map $\varGamma_2\to \varGamma_1\backslash\varGamma_1\alpha\varGamma_2$ given by $\gamma_2\mapsto \varGamma_1\alpha\gamma_2$ is surjective, since elements of $\varGamma_1\backslash\varGamma_1\alpha\varGamma_2$ are of the form $\varGamma_1\gamma_1\alpha\gamma_2$ for some $\gamma_1,\gamma_2$. Two elements $\gamma_2,\gamma_2^\prime\in\varGamma_2$ map to the same orbit if $\varGamma_1\alpha\gamma_2 = \varGamma_1\alpha\gamma_2^\prime$, that is, if $\gamma_2^\prime\gamma_2^{-1}\in \alpha^{-1}\varGamma_1\alpha$. It follows that by taking $\varGamma_3 = \alpha^{-1}\varGamma_1\alpha\cap \varGamma_2$ as above, that the induced map from the coset space $\varGamma_3\backslash\varGamma_2\to \varGamma_1\backslash \varGamma_1\alpha\varGamma_2$ is a bijection.
\end{proof}
\begin{lemma}\label{lem: congruence subgroups commensurable}
    Let $\varGamma_1,\varGamma_2$ be congruence subgroups. Then they are \textib{commensurable}, that is, the indices $[\varGamma_1 : \varGamma_1\cap \varGamma_2], [\varGamma_2 : \varGamma_1\cap \varGamma_2]$ are finite. We write $\varGamma_1\approx\varGamma_2$ in this case.
\end{lemma}
\begin{proof}
    There exist positive integers $N_1,N_2$ such that $\varGamma(N_1)\subset \varGamma_1$ and $\varGamma(N_2)\subset \varGamma_2$. Then $\varGamma(\lcm(N_1,N_2))\subset \varGamma(N_1)\cap \varGamma(N_2)\subset \varGamma_1\cap \varGamma_2$ since $N_1,N_2$ divide their least common multiple. Then \begin{align*}
        [\varGamma_1 : \varGamma(\lcm(N_1,N_2))] &= [\varGamma_1 : \varGamma_1\cap \varGamma_2][\varGamma_1\cap \varGamma_2 : \varGamma(\lcm(N_1,N_2))] \text{ and}\\
        [\varGamma_2 : \varGamma(\lcm(N_1,N_2))] &= [\varGamma_2 : \varGamma_1\cap \varGamma_2][\varGamma_1\cap \varGamma_2 : \varGamma(\lcm(N_1,N_2))].
    \end{align*} The indices $[\varGamma_1 : \varGamma(\lcm(N_1,N_2))],[\varGamma_2 : \varGamma(\lcm(N_1,N_2))]$ are finite (see the discussion following \cref{def: congruence subgroup}), so that the indices $[\varGamma_1 : \varGamma_1\cap \varGamma_2], [\varGamma_2 : \varGamma_1\cap \varGamma_2]$ are also finite.
\end{proof}
In \cref{lem: conjugation congruence subgroup}, since $\alpha^{-1}\varGamma\alpha\cap\slz$ is a congruence subgroup, the coset space $\varGamma_3\backslash \varGamma_2$ from \cref{lem: double coset coset/orbit spaces} is finite, so that the orbit space $\varGamma_1\backslash \varGamma_1\alpha\varGamma_2$ is also finite as desired.

\begin{definition}\label{def: weight k double coset operator}
    Let $\alpha\in \glqp$ and $\varGamma_1,\varGamma_2$ be congruence subgroups. The \textib{weight-$k$ $\varGamma_1\alpha\varGamma_2$ operator} (``double coset operator'') $|_k[\varGamma_1\alpha\varGamma_2]$ given by 
    \[f|_k[\varGamma_1\alpha\varGamma_2] = \sum_jf|_k[\beta_j],\] where $\{\beta_j\}$ is a set of orbit representatives of $\varGamma_1\alpha\varGamma_2$ (so $\varGamma_1\alpha\varGamma_2 = \bigsqcup_j \varGamma_1\beta_j$), takes modular forms in $\mathcal M_k(\varGamma_1)$ to modular forms in $\mathcal M_k(\varGamma_2)$. It also takes cusp forms in $\mathcal S_k(\varGamma_1)$ to cusp forms in $\mathcal S_k(\varGamma_2)$.
\end{definition}
We check that this definition is well defined, that is, independent of the choice of orbit representatives for $\varGamma_1\alpha\varGamma_2$. \tbd

We also check that the double coset operator takes modular forms to modular forms and cusp forms to cusp forms. \tbd

Particular choices of $\varGamma_1,\varGamma_2$ yield notable double coset operators (with explanations \tbd):
\begin{enumerate}[label=(\arabic*)]
    \item If $\varGamma_2\subset \varGamma_1$ and $\alpha = I$, then $f|_k[\varGamma_1\alpha\varGamma_2] = f$ and the double coset operator $|_k[\varGamma_1\alpha\varGamma_2]$ is the natural inclusion of $\mathcal M_k(\varGamma_1)$ into $\mathcal M_k(\varGamma_2)$.
    \item If $\varGamma_2 = \alpha^{-1}\varGamma_1\alpha$, then $f|_k[\varGamma_1\alpha\varGamma_2] = f|_k[\alpha]$, and the double coset operator $|_k[\varGamma_1\alpha\varGamma_2]$ is the natural isomorphism of $\mathcal M_k(\varGamma_1)$ with $\mathcal M_k(\varGamma_2)$.
    \item Finally, if $\varGamma_1\subset\varGamma_2$ and $\alpha = I$, take $\{\gamma_{2,j}\}$ to be a set of coset representatives for $\varGamma_1\backslash \varGamma_2$. Then $f|_k[\varGamma_1\alpha\varGamma_2] = \sum_j f|_k[\gamma_{2,j}]$, and the double coset operator $|_k[\varGamma_1\alpha\varGamma_2]$ is the natural trace map projecting $\mathcal M_k(\varGamma_1)$ onto its subspace $\mathcal M_k(\varGamma_2)$ \sai{by symmetrizing over the quotient, a surjection (I could elaborate on this some more)}.
\end{enumerate}

Any double coset operator is a composition of these three operators. Given $\varGamma_1,\varGamma_2$, and $\alpha$, set $\varGamma_3 = \alpha^{-1}\varGamma_1\alpha \cap \varGamma_2$ and $\varGamma_3^\prime = \alpha\varGamma_3\alpha^{-1} = \varGamma_1\cap \alpha\varGamma_2\alpha^{-1}$. Then $\varGamma_3^\prime\subset \varGamma_1$, $\alpha^{-1}\varGamma_3^\prime\alpha = \varGamma_3$, and $\varGamma_3\subset \varGamma_2$. Composing their corresponding double coset operators, we have for  $f\in\mathcal M_k(\varGamma_1)$ that $f\mapsto f\mapsto f|_k[\alpha]\mapsto \sum_j f|_k[\alpha\gamma_{2,j}]$. By \cref{lem: double coset coset/orbit spaces} the composition above agrees with the double coset operator $|_k[\varGamma_1\alpha\varGamma_2]$.
\subsection{Hecke operators $\abr{n}$ and $T_n$}
Recall the congruence subgroups \begin{align*}
    \varGamma_0(N) &= \cbr{\abcd\in\slz: \abcd\equiv\begin{pmatrix}
        \ast & \ast \\ 0 & \ast
    \end{pmatrix}\smod N}\text{ and}\\
    \varGamma_1(N) &= \cbr{\abcd\in \slz: \begin{pmatrix}
        a & b \\ c & d 
    \end{pmatrix}\equiv \begin{pmatrix}
        1 & \ast \\ 0 & 1
    \end{pmatrix}\smod N}.
\end{align*} Since $\varGamma_1(N)\subset 
\varGamma_0(N)$, we have the containment $\mathcal M_k(\varGamma_0(N))\subset \mathcal M_k(\varGamma_1(N))$ of modular forms. We define the two operators $\abr{n}$ and $T_n$ on the larger space $\mathcal M_k(\varGamma_1(N))$ (note that we defined $L$-functions for elements of $\mathcal M_k(\varGamma_1(N))$ in \cref{def: L function}).

Let $\alpha\in\varGamma_0(N)$ and consider the weight-$k$ double coset operator $|_k[\varGamma_1(N)\alpha\varGamma_1(N)]$. From \cref{lem: gamma 1 normal in gamma 0} we have that $\varGamma_1(N)$ is normal in $\varGamma_0(N)$ and that $\varGamma_0(N)/\varGamma_1(N)\cong (\mathbb Z/N\mathbb Z)^\times$, so the double coset operator $|_k[\varGamma_1(N)\alpha\varGamma_1(N)]$ is of the form (2) in the list following \cref{def: weight k double coset operator} \sai{(not quite the right location, but fix later)}. Hence this double coset operator translates a function $f\in \mathcal M_k(\varGamma_1(N))$ to $f|_k[\varGamma_1(N)\alpha\varGamma_1(N)] = f|_k[\alpha]\in \mathcal M_k(\varGamma_1(N))$. In this way the group $\varGamma_0(N)$ acts on $\mathcal M_k(\varGamma_1(N))$, and its subgroup $\varGamma_1(N)$ acts trivially. Thus $\varGamma_0(N)/\varGamma_1(N)\cong (\mathbb Z/N\mathbb Z)^\times$ acts on $\mathcal M_k(\varGamma_1(N))$. In particular, if $d\in (\mathbb Z/N\mathbb Z)^\times$, there exists $\big(\!\begin{smallmatrix}
    a^\prime & b^\prime \\ c^\prime & d^\prime
\end{smallmatrix}\!\big)\in \varGamma_0(N)/\varGamma_1(N)$ with $d^\prime\equiv d\smod N$. Thus the action of $d$ on $\mathcal M_k(\varGamma_1(N))$, called the diamond operator $\abr{d}\colon \mathcal M_k(\varGamma_1(N))\to \mathcal M_k(\varGamma_1(N))$, is given by 
\[\abr{d}f = f|_k[\alpha] \quad \text{for any $\alpha = \begin{pmatrix}
    a^\prime & b^\prime \\ c^\prime & d^\prime
\end{pmatrix}\in \varGamma_0(N)$ with $d^\prime\equiv d\smod N$}.\]
For any character $\chi$, we have that the space $\mathcal M_k(N,\chi)$ from \cref{prop: eigenspace decomp} is really the ``$\chi$-eigenspace of the diamond operators''; that is,
\[\mathcal M_k(N,\chi) = \{f\in \mathcal M_k(\varGamma_1(N)) : \abr{d}f = \chi(d)f \text{ for all } d\in(\mathbb Z/N\mathbb Z)^\times\}.\] 
It follows that the diamond operator $\abr{d}$ acts on $\mathcal M_k(\varGamma_1(N)) = \bigoplus_\chi\mathcal M_k(N,\chi)$ by acting on each $\chi$-eigenspace by multiplication by $\chi(d)$.

We extend the definition of the diamond operator $\abr{d}$ for $d\in (\mathbb Z/N\mathbb Z)^\times$ to $\abr{n}$ for $n\in\mathbb{Z}^+$.
\begin{definition}
    The \textib{Hecke operator $\abr{n}$}$\colon \mathcal M_k(\varGamma_1(N))\to \mathcal M_k(\varGamma_1(N))$ for $n\in\mathbb{Z}^+$ is given by the zero operator when $\gcd(n,N)>1$ and is given by $\abr{\overline n}$ when $\gcd(n,N) = 1$ (where $\overline{\,\cdot\,}$ denotes reduction modulo $N$). Observe that the mapping $n\mapsto \abr{n}$ is totally multiplicative; that is, $\abr{nm} = \abr{n}\abr{m}$ for all positive integers $n,m$.
\end{definition}

Let $p$ be a prime number, and consider the weight-$k$ double coset operator $T_p = |_k[\varGamma_1(N)\alpha\varGamma_1(N)]\colon \mathcal M_k(\varGamma_1(N))\to \mathcal M_k(\varGamma_1(N))$ where $\alpha = \big(\!\begin{smallmatrix}
    1 & 0 \\ 0 & p
\end{smallmatrix}\!\big)$. 
\begin{lemma}
    For a prime $p$, the double coset $\varGamma_1(N)\big(\!\begin{smallmatrix}
        1 & 0 \\ 0 & p
    \end{smallmatrix}\!\big)\varGamma_1(N)$ is given by 
    \[\varGamma_1(N)\big(\!\begin{smallmatrix}
        1 & 0 \\ 0 & p
    \end{smallmatrix}\!\big)\varGamma_1(N) = \cbr{\gamma\in \Mat_2(\mathbb Z) : \gamma \equiv \begin{pmatrix}
        1 & \ast \\ 0 & p
    \end{pmatrix}, \det\gamma = p}.\]
\end{lemma}
\begin{proof}
    \tbd \sai{(p.105 in \cite{diamond})}
\end{proof}
So in the definition of $T_p$, we may replace $\alpha$ with any matrix in the double coset $\varGamma_1(N)\big(\!\begin{smallmatrix}
    1 & 0 \\ 0 & p
\end{smallmatrix}\!\big)\varGamma_1(N)$.
\begin{lemma}
    Let $d,e\in(\mathbb Z/N\mathbb Z)^\times$ and $p,q$ be prime. Then \begin{enumerate}[label = \textup{(\arabic*)}]
        \item $\abr{d}T_p = T_p\abr{d}$,
        \item $\abr{d}\abr{e} = \abr{e}\abr{d} = \abr{de}$, and 
        \item $T_pT_q = T_qT_p$.
    \end{enumerate}
\end{lemma}
\begin{proof}
    \tbd \sai{(p.173 in \cite{diamond})}
\end{proof}
\begin{definition}
    The \textib{Hecke operator $T_n$}$\colon \mathcal M_k(\varGamma_1(N))\to \mathcal M_k(\varGamma_1(N))$ for $n\in\mathbb Z^+$ is defined inductively. Let $T_1 = \id_{\mathcal M_k(\varGamma_1(N))}$ (the identity operator), and above we had defined $T_p$ for primes $p$. For prime powers, let
    \[T_{p^r} = T_pT_{p^{r-1}} - p^{k-1}\abr{p}T_{p^{r-2}}\quad \text{for $r\geq 2$}.\] Then for $n = \prod_i p_i^{r_i}$, define 
    \[T_n = \prod_i T_{p_i^{r_i}},\] where the previous lemma justifies the multiplicative notation above.
\end{definition}

\sai{Should include the explicit action of $T_p$ and $T_n$.}

\subsection{The Petersson inner product, adjoints of Hecke operators}
We study the space $\mathcal S_k(\varGamma_1(N))$ by endowing it with an inner product. Define the \textib{hyperbolic measure} $\mu$ on $\mathcal H$ by $\dd\mu(z) = \dd x \dd y / y^2$ for $z = x+iy\in\mathcal H$ (we will sometimes suppress the $(z)$). This measure is invariant under the action of $\glrp$ on $\mathcal H$: For $\alpha \in \glrp$ and $z=x+iy \in \mathcal H$, let $\alpha(z) = \sigma(x,y) + i\tau(x,y)$. Let $A$ be a measurable set in $\mathcal H$. A few computations reveal that $\big|\frac{\partial (\sigma,\tau)}{\partial(x,y)}\big| = \abs{\dv{\alpha}{z}}^2 = \big(\det\alpha/\abs{j(\alpha,z)}^2\big)^2$ (via Wirtinger derivatives) and $\tau(z) = y\det\alpha/\abs{j(\alpha,z)}^2$. Then \[\mu(\alpha(A)) = \int_{\alpha(A)} \frac{\dd\sigma\dd\tau}{\tau^2} = \int_{A}\bigg|\frac{\partial (\sigma,\tau)}{\partial(x,y)}\bigg|\frac{\dd x\dd y}{(\tau(x,y))^2} = \int_A \frac{(\det\alpha)^2}{\abs{j(\alpha,z)}^4}\frac{\abs{j(\alpha,z)}^4\dd x \dd y}{(\det\alpha)^2y^2} =\int_A \frac{\dd x \dd y}{y^2} = \mu(A)\]
as desired. In particular $\dd \mu$ is $\slz$-invariant. Since $\mathbb Q\cup\cbr{\infty}$ is countable, its measure is zero, so we may integrate over the extended upper half plane $\mathcal H^\ast$ with respect to $\mu$.

Recall that a fundamental domain for $\mathcal H^\ast$ under the action of $\slz$ is given by $\mathcal D^\ast = \cbr{z\in\mathcal H : \abs{\Re(z)}\leq 1/2,\abs{z}\geq 1}\cup\cbr{\infty}$; that is, any point of $\mathcal H$ is sent to a point in $\mathcal D$ by a suitable element of $\slz$, which is unique for most points of $\mathcal H$ (there are a few cases involving points on the boundary of $\mathcal D$). Every point $s\in\mathbb Q\cup\cbr{\infty}$ may be sent to $\infty$ by suitable elements of $\slz$. We show that the integral $\int_{\mathcal D^\ast} \varphi(\alpha(z))\dd \mu$ converges for any continuous, bounded function $\varphi\colon\mathcal H\to\mathbb C$ and any $\alpha\in\slz$: Choose $\gamma\in\slz$ such that $\gamma^{-1}\mathcal D^\ast$ is a compact set. By the invariance of the measure under the action of $\slz$, we have
\[\int_{\mathcal D^\ast} \varphi(\alpha(z))\dd \mu = \int_{\mathcal \gamma^{-1} D^\ast} \varphi(\alpha(\gamma(z)))\dd \mu, \] from which it follows that the integral converges.

Let $\varGamma$ be a congruence subgroup, and let $\cbr{\alpha_j}$ be a collection of representatives for the coset space $\cbr{\pm I}\varGamma\backslash\slz$; that is, we have the disjoint union $\slz = \bigsqcup_j \cbr{\pm I}\varGamma \alpha_j$. If $\varphi$ is $\varGamma$-invariant, then the sum $\sum_j\int_{\mathcal D^\ast}\varphi(\alpha_j(z))\dd\mu$ is independent of the choice of coset representatives $\alpha_j$. Since $\dd\mu$ is $\slz$-invariant, the sum is equal to $\int_{\bigcup_j \alpha_j(\mathcal D^\ast)}\varphi(z)\dd\mu$. Since $\bigcup_j\alpha_j(\mathcal D^\ast)$ represents the modular curve $X(\varGamma)$ up to identification of boundaries, we define 
\[\int_{X(\varGamma)}\varphi(z)\dd\mu = \int_{\bigcup_j\alpha_j(\mathcal D^\ast)}\varphi(z)\dd\mu = \sum_j\int_{\mathcal D^\ast}\varphi(\alpha_j(z))\dd\mu.\] In particular the volume of $X(\varGamma)$ is given by $V_\varGamma = \int_{X(\varGamma)}\dd\mu$. Moreover, $V_\varGamma = [\slz : \cbr{\pm I}\varGamma]V_{\slz}$: \sai{(to prove)}\tbd

Observe that for any $f,g\in\mathcal S_k(\varGamma)$ for a congruence subgroup $\varGamma$, the function $\varphi(z) = f(z)\overline{g(z)}(\Im(z))^k$ for $z\in\mathcal H$ is continuous, and more importantly, $\varGamma$-invariant. Indeed, for any $\gamma\in\varGamma$, we have
\begin{align*}
    \varphi(\gamma(z)) &= f(\gamma(z))\overline{g(\gamma(z))}(\Im(\gamma(z)))^k\\
    &= (f|_k[\gamma])(z)j(\gamma,z)^k\overline{(g|_k[\gamma])(z)}\overline{j(\gamma,z)}^k(\Im(z))^k\abs{j(\gamma,z)}^{-2k}\\
    &= (f|_k[\gamma])(z)\overline{(g|_k[\gamma])(z)}(\Im(z))^k\\
    &= \varphi(z).
\end{align*} To show that $\varphi$ is bounded on $\mathcal H$, we show that $\varphi$ is bounded on $\bigcup_j\alpha_j(\mathcal D)$, which is a finite union. Therefore it suffices to show that for any $\alpha\in\slz$, the function $\varphi\circ\alpha$ is bounded on $\mathcal D$. On any compact subset of $\mathcal D$, continuity of $\varphi\circ\alpha$ implies boundedness. For the neighborhoods $\cbr{\Im(z)> M}$ of $i\infty$ \sai{$i\infty$?}, note that the Fourier expansions
\[(f|_k[\alpha])(z) = \sum_{n =1}a_n(f|_k[\alpha])q_h^n,\quad (g|_k[\alpha])(z) = \sum_{n =1}a_n(g|_k[\alpha])q_h^n\quad\text{for $q_h = e^{2\pi i z /h}$ for some $h\in\mathbb Z^+$}\] are of order $\mathcal O(q_h)$ as $\Im(z)\to\infty$. It follows that $\varphi(\alpha(z)) = (f|_k[\gamma])(z)\overline{(g|_k[\gamma])(z)}(\Im(z))^k = \mathcal O(q_h)^2(\Im(z))^k$. Moreover, since $\abs{q_h} = e^{-2\pi \Im(z)/h}$, $\varphi(\alpha(z))\to 0$ as $\Im(z)\to\infty$, from which it follows $\varphi\circ\alpha$ is bounded on the neighborhoods $\cbr{\Im(z)> M}$ of $i\infty$ of $\mathcal D$, hence on all of $\mathcal D$.
\begin{definition}
    The \textib{Petersson inner product} $\abr{\cdot,\cdot}_\varGamma\colon \mathcal S_k(\varGamma)\times \mathcal S_k(\varGamma)\to\mathbb C$ for a congruence subgroup $\varGamma$ is given by 
    \[\abr{f,g}_\varGamma = \frac{1}{V_\varGamma}\int_{X(\varGamma)}f(z)\overline{g(z)}(\Im(z))^k\dd\mu.\] (We may sometimes omit the subscript ${}_\varGamma$ in $\abr{\cdot,\cdot}_\varGamma$.)
\end{definition}
This inner product is linear in the first component, conjugate linear in the second component, Hermitian, and positive definite. If we have the containment $\varGamma^\prime\subset \varGamma$ of congruence subgroups, then $\abr{\cdot,\cdot}_{\varGamma^\prime} = \abr{\cdot,\cdot}_\varGamma$ on $\mathcal S_k(\varGamma)$.\tbd

Recall that if $T$ is a linear operator on the inner product space $V$, then there exists a unique operator $T^\ast$ called the adjoint of $T$ that satisfies $\abr{Tv,w} = \abr{v,T^\ast w}$ for all $v,w\in V$. If $T$ commutes with $T^\ast$, we say that $T$ is a normal operator. Give $\mathcal S_k(\varGamma_1(N))$ the Petersson inner product. We show that the Hecke operators $\abr{n}$ and $T_n$ for $n$ coprime to $N$ are normal operators.

Let $\varGamma$ be a congruence subgroup, and write $\slz = \bigcup_j\cbr{\pm I}\varGamma \alpha_j$ for some representatives $\alpha_j$ of the coset space $\varGamma\backslash \slz$. If $\alpha\in\glqp$, then the map $\mathcal H\to\mathcal H$ given by $z \mapsto \alpha(z)$ induces a bijection $\alpha^{-1}\varGamma\alpha\backslash \mathcal H^\ast\to X(\varGamma)$ \tbd\sai{(to check)}: It follows that the union $\bigcup_j \alpha^{-1}\alpha_j(\mathcal D^\ast)$ up to some boundary identifications is in bijection with $\alpha^{-1}\varGamma\alpha\backslash \mathcal H^\ast$. For continuous, bounded, $\alpha^{-1}\varGamma\alpha$-invariant functions $\varphi\colon \mathcal H\to \mathbb C$ define
\[\int_{\alpha^{-1}\varGamma\alpha\backslash \mathcal H^\ast}\varphi(z)\dd\mu = \sum_j \int_{\mathcal D^\ast}\varphi(\alpha^{-1}\alpha_j(z))\dd\mu.\]

\begin{lemma}
    Let $\varGamma$ be a congruence subgroup and let $\alpha\in\glqp$.\begin{enumerate}[label = \textup{(\alph*)}]
        \item If $\varphi\colon\mathcal H\to\mathbb C$ is continuous, bounded, and $\varGamma$-invariant, then \[\int_{\alpha^{-1}\varGamma\alpha\backslash \mathcal H^\ast}\varphi(\alpha(z))\dd\mu = \int_{X(\varGamma)}\varphi(z)\dd\mu.\]
        \item If $\alpha^{-1}\varGamma\alpha\subset \slz$, then $V_{\alpha^{-1}\varGamma\alpha} = V_\varGamma$ and $[\slz : \alpha^{-1}\varGamma\alpha] = [\slz : \varGamma]$.
        \item There exist $\beta_1,\dots,\beta_n\in \glqp$ with $n=[\varGamma : \alpha^{-1}\varGamma\alpha\cap\varGamma] = [\varGamma : \alpha\varGamma\alpha^{-1}\cap \varGamma]$, such that \[\varGamma\alpha\varGamma = \bigsqcup_j\varGamma \beta_j = \bigsqcup_j\beta_j\varGamma.\]
    \end{enumerate}
    \begin{proof}
        \tbd
    \end{proof}
\end{lemma}

The following proposition will be used to compute adjoints of the Hecke operators.
\begin{proposition}
    Let $\varGamma$ be a congruence subgroup and let $\alpha\in\glqp$. \begin{enumerate}[label = \textup{(\alph*)}]
        \item If $\alpha^{-1}\varGamma\alpha\subset \slz$, then for all $f\in\mathcal S_k(\varGamma)$ and $g\in\mathcal S_k(\alpha^{-1}\varGamma\alpha)$ \[\abr{f|_k[\alpha],g}_{\alpha^{-1}\varGamma\alpha} = \abr{f,g|_k[\det(\alpha)\alpha^{-1}]}_\varGamma.\]
        \item For all $f,g\in\mathcal S_k(\varGamma)$, \[\abr{f|_k[\varGamma\alpha\varGamma],g}  = \abr{f,g|_k[\varGamma\det(\alpha)\alpha^{-1}\varGamma]}.\]
        We have $|_k[\varGamma\alpha\varGamma]^\ast = |_k[\varGamma\det(\alpha)\alpha^{-1}\varGamma]$. If $\alpha^{-1}\varGamma\alpha = \varGamma$, then $|_k[\alpha]^\ast = |_k[\det(\alpha)\alpha^{-1}]$.
    \end{enumerate}
\end{proposition}
\begin{proof}
    \tbd
\end{proof}
\begin{theorem}
    For $p\nmid N$, the Hecke operators $\abr{p},T_p\colon \mathcal S_k(\varGamma_1(N))\to \mathcal S_k(\varGamma_1(N))$ have adjoints \[\abr{p}^\ast = \abr{p}^{-1}\quad\text{and}\quad T_p^\ast = \abr{p}^{-1}T_p.\] It follows that the Hecke operators $\abr{n},T_n$ for $n$ coprime to $N$ are normal operators.
\end{theorem}
\begin{proof}
    \tbd
\end{proof}
\begin{corollary}
    By the spectral theorem, the space $\mathcal S_k(\varGamma_1(N))$ has an orthogonal basis of simultaneous eigenvectors, called \textib{eigenforms}, for the Hecke operators $\cbr{\abr{n},T_n : \gcd(n,N) = 1}$.
\end{corollary}

\subsection{Oldforms, newforms, and primitive forms}
We start with some results which are used to take forms from lower levels $M$ dividing $N$ to $N$.
\begin{lemma}
    If $M\mid N$ then $\mathcal S_k(\varGamma_1(M))\subset \mathcal S_k(\varGamma_1(N))$.
\end{lemma}
\begin{proof}
    \tbd
\end{proof}
Let $M\mid N$ and let $d$ be a factor of $N/M$. Define $\alpha_d = \big(\!\begin{smallmatrix}
    d & 0 \\ 0 & 1
\end{smallmatrix}\!\big)$ and consider the map $|_k[\alpha_d]$, which is an injective map $\mathcal S_k(\varGamma_1(M))\to\mathcal S_k(\varGamma_1(N))$ by \cref{lem: weight k operator properties}. Explicitly, for $f\colon\mathcal H\to\mathbb C$ we have $(f|_k[\alpha_d])(z) = d^{k/2}f(dz)$.

We consider the subspace of $\mathcal S_k(\varGamma_1(N))$ containing all cusp forms which may be obtained from cusp forms at lower levels via a combination of the two embeddings discussed earlier.

\begin{definition}
    For each divisor $d$ of $N$, define 
    \[i_d\colon (\mathcal S_k(\varGamma_1(Nd^{-1})))^2\to\mathcal S_k(\varGamma_1(N))\] by $(f,g)\mapsto f + g|_k[\alpha_d]$. The subpace of \textib{oldforms at level $N$} is 
    \[\mathcal S_k(\varGamma_1(N))^\old = \sum_{\substack{p\mid N \\ p\text{ prime}}}i_p((\mathcal S_k(\varGamma_1(Np^{-1})))^2),\] and the subspace of \textib{newforms at level $N$} is the orthogonal complement of the subspace of oldforms with respect to the Petersson inner product,
    \[\mathcal S_k(\varGamma_1(N))^\new = (\mathcal S_k(\varGamma_1(N))^\old)^\perp.\qedhere\]
\end{definition}

\begin{proposition}
    The subspace of oldforms and the subspace of newforms in $\mathcal S_k(\varGamma_1(N))$ are stable under the Hecke operators $\abr{n},T_n$ for all $n\in\mathbb Z^+$.
\end{proposition}
\begin{proof}
    \tbd
\end{proof}
\begin{corollary}
    The subspace of oldforms and the subspace of newforms each have orthogonal bases of eigenforms for the Hecke operators away from the level $N$; that is, for $\cbr{\abr{n},T_n : \gcd(n,N) = 1}$.
\end{corollary} The condition that $\gcd(n,N) = 1$ can be removed for the space of newforms.

\sai{The ``Main Lemma'' will appear soon.}

\newpage\section{Strong Multiplicity One}
\subsection{TEMPORARY collection of preliminary results}

\begin{lemma}\label{lem: miyake lem 4.5.2}
    For any $\alpha\in\varDelta_0(N)$ \textup{(}respectively $\varDelta_0^\ast(N)$\textup{)}, there exists a unique pair of positive integers $(l,m)$ such that $l\mid m$, $\gcd(l,N) = 1$, and 
    \[\varGamma_0(N)\alpha\varGamma_0(N) = \varGamma_0(N)\begin{pmatrix}
        l & 0 \\ 0 & m
    \end{pmatrix}\varGamma_0(N)\] \[(\text{respectively }\varGamma_0(N)\alpha\varGamma_0(N) = \varGamma_0(N)\begin{pmatrix}
        m & 0 \\ 0 & l
    \end{pmatrix}\varGamma_0(N)).\]
\end{lemma}
\begin{proof}
    \tbd
\end{proof}
\begin{theorem}\label{thm: miyake thm 4.5.5}
    The following diagram is commutative: 
    % https://q.uiver.app/#q=WzAsNCxbMCwwLCJcXG1hdGhjYWwgTV9rKE4sXFxjaGkpIl0sWzAsMSwiXFxtYXRoY2FsIE1fayhOLFxcb3ZlcmxpbmV7XFxjaGl9KSJdLFs0LDAsIlxcbWF0aGNhbCBNX2soTixcXGNoaSkiXSxbNCwxLCJcXG1hdGhjYWwgTV9rKE4sXFxvdmVybGluZXtcXGNoaX0pIl0sWzAsMSwiXFxvbWVnYV9OIl0sWzEsMywiVF5cXGFzdChuKVxcdGV4dHsgXFx0ZXh0dXB7KH1yZXNwLiAkVF5cXGFzdChtLGwpJFxcdGV4dHVweyl9fSJdLFswLDIsIlQobilcXHRleHR7IFxcdGV4dHVweyh9cmVzcC4gJFQobCxtKSRcXHRleHR1cHspfX0iXSxbMiwzLCJcXG9tZWdhX04iXV0=
\[\begin{tikzcd}
	{\mathcal M_k(N,\chi)} &&&& {\mathcal M_k(N,\chi)} \\
	{\mathcal M_k(N,\overline{\chi})} &&&& {\mathcal M_k(N,\overline{\chi}).}
	\arrow["{\omega_N}", from=1-1, to=2-1]
	\arrow["{T^\ast(n)\text{ \textup{(}resp. $T^\ast(m,l)$\textup{)}}}", from=2-1, to=2-5]
	\arrow["{T(n)\text{ \textup{(}resp. $T(l,m)$\textup{)}}}", from=1-1, to=1-5]
	\arrow["{\omega_N}", from=1-5, to=2-5]
\end{tikzcd}\]
\end{theorem}
\begin{proof}
    \tbd
\end{proof}
\begin{lemma}\label{lem: miyake lem 4.5.6}
    Let $p$ be a prime number, and $e$ a nonnegative integer.Then \begin{enumerate}[label=\textup{(\arabic*)}]
        \item A set of representatives for \[\varGamma_0(N)\backslash\varGamma_0(N)\begin{pmatrix}
            1 & 0 \\ 0 & p^e
        \end{pmatrix}\varGamma_0(N)\] is given by the set \[\begin{cases}
            \cbr{\big(\!\begin{smallmatrix}
                p^{e-f} & m \\ 0 & p^f
            \end{smallmatrix}\!\big) : 0\leq f\leq e, 0\leq m < p^f, \gcd(m,p^f,p^{e-f}) = 1} & \text{if $p\nmid N$},\\
            \cbr{\big(\!\begin{smallmatrix}
                1 & m \\ 0 & p^e
            \end{smallmatrix}\!\big) : 0\leq m < p^e} & \text{if $p\mid N$, and}
        \end{cases} \]
        \item $\deg\Big(\varGamma_0(N)\backslash\varGamma_0(N)\begin{pmatrix}
            1 & 0 \\ 0 & p^e
        \end{pmatrix}\varGamma_0(N)\Big) = \begin{cases}
            p^e + p^{e-1} & \text{if $p\nmid N$,}\\
            p^e & \text{if $p\mid N$.}
        \end{cases}$ \sai{(i have to define degree, perhaps it is the one on p.73 of miyake)}
    \end{enumerate}
\end{lemma}
\begin{proof}
    \tbd
\end{proof}
\begin{lemma}\label{lem: miyake lem 4.5.7}
    Let $p$ be a prime number and $e$ a positive integer. Then \begin{enumerate}[label=\textup{(\arabic*)}]
        \item $T(p)T(1,p^e) = T(1,p^{e+1}) + \begin{cases}
            (p+1)T(p,p) & \text{if $p\nmid N$ and $e=1$,}\\
            pT(p,p)T(1,p^{e-1}) & \text{if $p\nmid N$ and $e > 1$,}\\
            0 & \text{if $p\mid N$.}
        \end{cases}$
        \item $T(p)T(p^e) = \begin{cases}
            T(p^{e+1})+pT(p,p)T(p^{e-1}) & \text{if $p\nmid N$,}\\
            T(p^{e+1}) & \text{if $p\mid N$.}
        \end{cases}$
    \end{enumerate}
\end{lemma}
\begin{proof}
    \tbd
\end{proof}
\begin{lemma}\label{lem: miyake lem 4.5.8}
    \begin{enumerate}[label=\textup{(\arabic*)}]
        \item If $\gcd(lm,l^\prime m^\prime) = 1$, then $T(l,m)T(l^\prime,m^\prime) = T(ll^\prime,mm^\prime)$.
        \item If $\gcd(m,n) = 1$, then $T(m)T(n) = T(mn)$.
    \end{enumerate}
\end{lemma}
\begin{proof}
    \tbd
\end{proof}
\begin{lemma}\label{lem: miyake lem 4.5.11}
    Let $N$ be a positive integer and $p$ a prime. Then 
    \[\varGamma_0(pN)\begin{pmatrix}
        1 & 0 \\ 0 & p
    \end{pmatrix}\varGamma_0(N) = \begin{cases}
        \coprod_{v = 0}^{p-1}\varGamma_0(pN)\big(\!\begin{smallmatrix}
            1 & 0 \\ 0 & p
        \end{smallmatrix}\!\big)\gamma_v& \text{if $p\mid N$},\\
        \coprod_{v = 0}^p\varGamma_0(pN)\big(\!\begin{smallmatrix}
            1 & 0 \\ 0 & p
        \end{smallmatrix}\!\big)\gamma_v& \text{if $p\nmid N$};
    \end{cases}\]
    here  $\gamma_v$ for $0\leq v<p$ is an element of $\varGamma_0(N)$ such that $\gamma_v \equiv \big(\!\begin{smallmatrix}
        1 & v \\ 0 & 1
    \end{smallmatrix}\!\big)\smod{p}$, and $\gamma_p$ for $p\nmid N$ is an element of $\varGamma_0(N)$ such that 
    \[\gamma_p \equiv\begin{cases}
        \big(\!\begin{smallmatrix}
            0 & -a \\ a^{-1} & 0
        \end{smallmatrix}\!\big)\smod{p}&\text{for $a$ coprime to $p$},\\
        \big(\!\begin{smallmatrix}
            1 & 0 \\ 0 & 1
        \end{smallmatrix}\!\big)\smod{N}.
    \end{cases}\]
\end{lemma}
\begin{proof}
    \tbd
\end{proof}
\begin{lemma}\label{lem: miyake lem 4.5.12}
    Let $K$ be a commutative unital ring, and assume that two sequences $\cbr{t_n}_{n=1}^\infty,\cbr{d_n}_{n=1}^\infty$ of elements of $K$ satisfy the following conditions:
    \begin{enumerate}[label=\textup{(\roman*)}]
        \item $t_1 =d_1 = 1$, and 
        \item $d_{mn} = d_md_n$ for any positive integers $m,n$.
    \end{enumerate} Then the following are equivalent:
    \begin{enumerate}[label=\textup{(\arabic*)}]
        \item If $\gcd(m,n) = 1$, then $t_{mn} = t_mt_n$ and \[t_pt_{p^e} = t_{p^{e+1}} + pd_pt_{p^{e-1}}\] for all prime numbers $p$ and all positive integers $e$.
        \item The formal Dirichlet series $\sum_{n=1}^\infty t_nn^{-s}$ has formal Euler product \[\sum_{n=1}^\infty t_nn^{-s} = \prod_{p\in\mathbb P}(1-t_pp^{-s} + pd_pp^{-2s})^{-1}.\]
        \item For any positive integers $m,n$, \[t_mt_n = \sum_{\substack{l>0\\l\mid \gcd(m,n)}}ld_lt_{mn/l^2}.\]
    \end{enumerate}
\end{lemma}
\begin{proof}
    We show that (1) implies (2). From the first condition of (1), we have formally that 
    \[\sum_{n=1}^\infty t_nn^{-s} = \prod_{p\in\mathbb P}\bigg(\sum_{e=0}^\infty t_{p^e}p^{-es}\bigg).\]
    From the second condition of (1), we have \[(1-t_pp^{-s} + pd_pp^{-2s})\bigg(\sum_{e=0}^\infty t_{p^e}p^{-es}\bigg) = 1,\] from which (2) follows. 

    We show that (2) implies (3). Formal Dirichlet series and formal Euler products are elements of the ring of formal power series $K[[p^{-s} : p\in\mathbb P]]$. Finding inverses in $K[[p^{-s}]]$ of the elements $1-t_pp^{-s} + pd_pp^{-2s}$ recursively (cf. exercise 7.2.3)(c) in \cite{df}), and comparing coefficients of \[\sum_{n=1}^\infty t_nn^{-s} = \prod_{p\in\mathbb P}(1-t_pp^{-s} + pd_pp^{-2s})^{-1}\] reveals that the sequence $\cbr{t_n}_{n=1}^\infty$ is multiplicative. That is, if $\gcd(m,n) =1$, then $t_{mn} = t_mt_n$.
    
    Let $m = \prod_{p\in\mathbb P}p^e$ and $n = \prod_{p\in\mathbb P}p^f$ be the prime factorizations of $m$ and $n$. Then \[\sum_{\substack{l>0\\l\mid\gcd(m,n)}} ld_lt_{mn/l^2} = \prod_{p\in\mathbb P} \bigg(\sum_{0\leq g\leq\min\{e,f\}}p^gd_{p^g}t_{p^{e+f-2g}}\bigg).\] Therefore it suffices to prove that (2) implies (3) in the case that $m$ and $n$ are powers of the same prime $p$. By assumption, we have 
    \[(1-t_pp^{-s} + pd_pp^{-2s})^{-1} = \sum_{e=0}^\infty t_{p^e}p^{-es}.\] Let $\tau_p$ and $\delta_p$ be two indeterminates over $\mathbb Q$, and define a (unital) ring homomorphism $\psi\colon \mathbb Z[\tau_p,\delta_p]\to K$ by $\psi(\tau_p) = t_p$ and $\psi(\delta_p) = d_p$. Define the elements $\tau_{p^e}$ and $\delta_{p^e}$ of $\mathbb Z[\tau_p,\delta_p]$ by $\delta_{p^e} = (\delta_p)^e$ and by the formal power series equality 
    \begin{equation}\label{eqn: tau power series expansion}
        \sum_{e=0}^\infty \tau_{p^e}p^{-es} = (1-\tau_pp^{-s} + p\delta_pp^{-2s})^{-s}.
    \end{equation} It follows that $\psi(\tau_{p^e}) = t_{p^e}$ and $\psi(\delta_{p^e}) = d_{p^e}$.

    Let $u,v$ be indeterminates over $\mathbb Q$, and define a ring homomorphism $\phi\colon \mathbb Z[\tau_p,\delta_p]\to \mathbb Q[u,v]$ by $\phi(\tau_p) = u+v$ and $\phi(\delta_p) = uv/p$. Since $u+v$ and $uv/p$ are algebraically independent over $\mathbb Q$, $\phi$ is injective. Viewing $\mathbb Z[\tau_p,\delta_p]$ as a subring of $\mathbb Q[u,v]$, we may factor $1-\tau_pp^{-s}+p\delta_pp^{-2s}$ into $(1-up^{-s})(1-vp^{-s})$. Inverting this element and comparing coefficients in \cref{eqn: tau power series expansion}, we find that
    \[\tau_{p^e} = \sum_{i+j=e}u^iv^j = (u^{e+1}-v^{e+1})/(u-v).\]

    Assume that $0\leq e\leq f$. Then \begin{align*}
        \tau_{p^e}\tau_{p^f} &= \bigg(\sum_{i+j=e}u^iv^j\bigg)\cdot (u^{f+1}-v^{f+1})/(u-v)\\
        &= \bigg(u^{f+1}\sum_{j=0}^eu^{e-j}v^j - v^{f+1}\sum_{j=0}^eu^{j}v^{e-j}\bigg)\bigg/(u-v)\\
        &=\sum_{g=0}^e u^gv^g(u^{e+f-2g+1}-v^{e+f-2g+1})/(u-v)\\
        &=\sum_{g=0}^e p^g\delta_{p^g}\tau_{p^{e+f-2g}}.
    \end{align*}
    By taking $\psi$, obtain \[t_{p^e}t_{p^f} = \sum_{g=0}^ep^gd_{p^g}t_{p^{e+f-2g}}\] as desired. So in all cases, (2) implies (3).

    To see that (3) implies (1), observe that (1) is a special case of (3), with $m = p$ and $n = p^e$.
\end{proof}
\begin{theorem}\label{thm: miyake thm 4.5.13}
    We have \begin{enumerate}[label=\textup{(\arabic*)}]
        \item $T(m)T(n) = \sum_{\substack{l>0 \\ l\mid \gcd(m,n)\\ \gcd(l,N)=1}}lT(l,l)T(mn/l^2)$, and
        \item the formal Dirichlet series $\sum_{n=1}^\infty T(n)n^{-s}$ has the formal Euler product \[\sum_{n=1}^\infty T(n)n^{-s} = \prod_{p\nmid N}(1-T(p)p^{-s} + T(p,p)p^{1-2s})^{-1}\cdot \prod_{p\mid N}(1-T(p)p^{-s})^{-1}\]
    \end{enumerate}
\end{theorem}
\begin{proof}
    Apply the previous lemma with 
    \[K = \mathcal R(N),\quad t_n = T(n),\quad \text{and}\quad d_n =\begin{cases}
        T(n,n) & \text{if $\gcd(n,N) = 1$},\\
        0 & \text{if $\gcd(n,N)\neq 1$}
    \end{cases}\] in combination with \cref{lem: miyake lem 4.5.7}(2) and \cref{lem: miyake lem 4.5.8}(2) to obtain the result.
\end{proof}
\begin{lemma}\label{lem: miyake lem 4.5.14}
    Let $f$ be an element of $\mathcal M_k(N,\chi)$, and let \begin{align*}
        f(z) &= \sum_{n=0}^\infty c_ne^{2\pi i n z},\\
        (f|_k[T(m)])(z) &= \sum_{n=0}^\infty b_ne^{2\pi i nz}
    \end{align*} be Fourier series expansions. Then \[b_n = \sum_{\substack{d>0 \\ d\mid \gcd(m,n)}}\chi(d)d^{k-1}c_{mn/d^2}.\]
\end{lemma}
\begin{proof}
    Observe that $f((az+b)/d) = \sum_{n=0}^\infty c_ne^{2\pi in(az+b)/d}$. Then by the explicit action of $T(n)$ on elements of $\mathcal M_k(N,\chi)$ (recorded in \sai{insert reference -- it will occur after lemma 4.5.9}), we have
    \[(f|_k[T(m)])(z) = m^{k-1}\sum_{\substack{d>0\\d\mid n}}\sum_{b=0}^{d-1}\chi(m/d)d^{-k}\sum_{n=0}^\infty c_ne^{2\pi in((m/d)z+b)/d}.\] Then use the fact that $\sum_{b=0}^{d-1}e^{2\pi i n b/d}$ is equal to $d$ if $d\mid n$ and is $0$ otherwise to simplify the above expression into 
    \[\sum_{n=0}^\infty \sum_{\substack{d>0\\d\mid \gcd(m,n)}}\chi(m/d)(m/d)^{k-1}c_ne^{2\pi i(nm/d^2)z}.\] Change summation variables to obtain 
    \[\sum_{n=0}^\infty \sum_{a>0\\ a\mid \gcd(m,n)}\chi(a)a^{k-1}c_{dn/a}e^{2\pi inz},\] and with $ad = m$, it follows that $c_{dn/a} = c_{mn/a^2}$, and the result follows.
\end{proof}
\begin{lemma}\label{lem: miyake lem 4.5.15}
    Let $f(z) = \sum_{n=0}^\infty c_ne^{2\pi i n z}$ be an element of $\mathcal M_k(N,\chi)$, and $P$ a set of prime numbers where $f|_k[T(p)] = t_pf$ for $p\in P$, $t_p\in\mathbb C$. Then 
    \begin{enumerate}[label=\textup{(\arabic*)}]
        \item If all prime factors of a positive integer $m$ lie in $P$, then $f$ is an eigenfunction of $T(m)$. In this case, let $f|_k[T(m)] = t_mf$ for some $t_m\in\mathbb C$, from which it follows that $c_m = t_mc_1$.
        \item We have $L(s,f) = \prod_{p\in P}(1-t_pp^{-s} + \chi(p)p^{k-1-2s})^{-1}\cdot \sum_n^\prime c_nn^{-s}$, where the summation $\sum_n^\prime$ is taken over the positive integers coprime to every element of $P$.
    \end{enumerate}
\end{lemma}
\begin{proof}
    \tbd
\end{proof}
\begin{theorem}\label{thm: miyake thm 4.5.16}
    Let $f(z) = \sum_{n=0}^\infty c_ne^{2\pi i nz}$ be a nonzero element of $\mathcal M_k(N,\chi)$. Then the following are equivalent: \begin{enumerate}[label=\textup{(\arabic*)}]
        \item $f(z)$ is a common eigenfunction of every Hecke operator $T(n)$;
        \item $c_1\neq 0$ and 
        \[L(s,f) = c_1\prod_{p\in\mathbb P} (1-t_pp^{-s} + \chi(p)p^{k-1-2s})^{-1},\quad t_n = \frac{c_n}{c_1}.\]
        Moreover, if $f(z)$ satisfies the above conditions, then
        \[f|_k[T(n)] = t_nf\] for all $n\geq 1$.
    \end{enumerate}
\end{theorem}
\begin{proof}
    \tbd
\end{proof}
\begin{theorem}\label{thm: miyake thm 4.6.12}
    Let $f(z)\in \mathcal S_k^0(N,\chi)$ and $g(z)\in \mathcal S_k(N,\chi)$. If $f(z)$ and $g(z)$ are common eigenfunctions of $T(n)$ with the same eigenvalue for each $n$ coprime to some integer $L$, then $g(z)$ is a constant multiple of $f(z)$.
\end{theorem}
\begin{proof}
    \tbd
\end{proof}
\begin{theorem}\label{thm: miyake thm 4.6.15}
    \begin{enumerate}[label=\textup{(\arabic*)}]
        \item By the action of $\omega_N$, we obtain the isomorphisms $\mathcal S_k^0(N,\chi)\cong \mathcal S_k^0(N,\overline{\chi})$ and $\mathcal S_k^1(N,\chi)\cong \mathcal S_k^1(N,\overline{\chi})$, and 
        \item if $f(z)$ is a primitive form of $\mathcal S_k^0(N,\chi)$, then $f_p(z)$ is a primitive form of $\mathcal S_k^0(N,\overline{\chi})$ and 
        \[f|_k[\omega_N] = cf_p(z)\] for some $c\in\mathbb C$.
    \end{enumerate}
\end{theorem}
\begin{proof}
    \tbd
\end{proof}
\begin{theorem}\label{thm: miyake thm 4.6.16}
    With the same notation and assumptions as in the previous \sai{theorem?}, we have
    \begin{enumerate}[label=\textup{(\arabic*)}]
        \item By the action of $\eta_q$, we have the isomorphisms \[\mathcal S_k^0(N,\chi)\cong \mathcal S_k^0(N,\chi^\prime_q\overline{\chi_q}),\quad \mathcal S_k^1(N,\chi)\cong \mathcal S_k^1(N,\chi^\prime_q\overline{\chi_q}).\]
        \item By the action of $\eta^\prime_q$, we have the isomorphisms \[\mathcal S_k^0(N,\chi)\cong \mathcal S_k^0(N,\overline{\chi^\prime_q}\chi_q),\quad \mathcal S_k^1(N,\chi)\cong \mathcal S_k^1(N,\overline{\chi^\prime_q}\chi_q).\]
        \item For $f\in\mathcal S_k(N,\chi)$, we have $f|_k[\eta^2_q] = \chi_q(-1)\overline{\chi^\prime_q}(N_q)f$, $f|_k[{\eta^\prime_q}^2] = \chi^\prime_q(-1)\overline{\chi_q}(N/N_q)f$, and $f|_k[\eta_q\eta^\prime_q] = \overline{\chi^\prime_q}(N_q)f|_k[\omega_N]$.
        \item Let $f\in\mathcal S_k^0(N,\chi)$ be a primitive form with $f(z) = \sum_{n=1}^\infty a_ne^{2\pi i n z}$, and write \[(f|_k[\eta_q])(z) = c\sum_{n=1}^\infty b_ne^{2\pi i n z}\quad \text{\textup{(}with $b_1 = 1$\textup{)}}.\] Let $g_q(z) = \sum_{n=1}^\infty b_ne^{2\pi i n z}$.
        
        Then $g_q$ is a primitive form of $\mathcal S_k^0(N,\chi^\prime_q\overline{\chi_q})$, with \[b_p = \begin{cases}
            \overline{\chi_q}(p)a_p &\text{if $p\neq q$},\\
            \chi^\prime_q(p)\overline{a_p} & \text{if $p = q$}
        \end{cases}\] for any prime $p$.
    \end{enumerate}
\end{theorem}
\begin{proof}
    \tbd
\end{proof}
\begin{theorem}\label{thm: miyake thm 4.6.17}
    Let $f(z) = \sum_{n=1}^\infty a_ne^{2\pi i nz}$ be a primitive form of $\mathcal S_k^0(N,\chi)$, and $m$ the conductor of $\chi$. For a prime factor $q$ of $N$, denote by $N_q$ and $m_q$ the $q$-components of $N$ and $m$, respectively; that is, $N_q$ and $m_q$ are the largest powers of $p$ dividing $N$ and $m$, respectively. \begin{enumerate}[label=\textup{(\arabic*)}]
        \item If $N_q = m_q$, then $\abs{a_q} = q^{(k-1)/2}$.
        \item If $N_q = q$ and $m_q = 1$, then $a_q^2 = \chi^\prime_q(q)q^{k-2}$.
        \item Otherwise, that is, if $q^2\mid N_q$ and $N_q \neq m_q$, then $a_q = 0$.
    \end{enumerate}
    \begin{proof}
        \tbd
    \end{proof}
\end{theorem}
\subsection{Main result and relevant corollaries}
\begin{theorem}\label{thm: sm1 miyake thm 4.6.19}
    Let $f(z) = \sum_{n=1}^\infty a_ne^{2\pi i nz}$ be a primitive form of $\mathcal S_k^0(N,\chi)$, and $g(z) = \sum_{n=1}^\infty b_ne^{2\pi i n z}\in \mathcal S_k(M,\lambda)$. If $g(z)$ is normalized with $b_1 = 1$, is a common eigenfunction of $\mathcal R(M)\cup \mathcal R^\ast(M)$, and $a_n = b_n$ for all $n$ coprime to some integer $L$, then $N = M$ and $f(z) = g(z)$.
\end{theorem}
\begin{proof}
    Without loss of generality we may take $L$ to be a common multiple of $N$ and $M$. If $p$ is a prime number coprime to $L$, then by \cref{lem: miyake lem 4.5.7}(2) we have
    \begin{align*}
        \chi(p)p^{k-1} &= a_p^2 - a_{p^2} \\
        &= b_p^2 - b_{p^2} \\
        &= \lambda(p)p^{k-1},
    \end{align*} so that $\chi(p) = \lambda(p)$. It follows that $\chi(n) = \lambda(n)$ for all $n$ coprime to $L$.

    We show that $M\mid N$. By \cref{thm: miyake thm 4.5.16}, $L(s,f)$ and $L(s,g)$ have Euler product expansions, so that
    \begin{equation}\label{eqn: ratio varLambdas}
        \frac{\varLambda_N(s,f)}{\varLambda_M(s,g)} = \bigg(\frac{\sqrt{N}}{\sqrt{M}}\bigg)^s\prod_{p\mid L}\frac{1-b_pp^{-s} + \lambda(p)p^{k-1-2s}}{1-a_pp^{-s} + \chi(p)p^{k-1-2s}}
    \end{equation} for $\Re(s)>k/2+1$. In fact, \cref{eqn: ratio varLambdas} holds for all $s$ since the ratio in the right-hand side is a meromorphic function on the whole $s$-plane.

    On the other hand, we have
    \[\frac{\varLambda_N(s,f)}{\varLambda_M(s,g)} = \frac{\varLambda_N(k-s,f|_k[\omega_N])}{\varLambda_M(k-s,g|_k[\omega_M])}\] by \cref{cor: miyake cor 4.3.7}. Furthermore, $g|[T(n)] = b_n(g)$, and since $g(z)$ is a common eigenfunction of $T^\ast(n)$ by assumption, we have that $g|[T^\ast(n)] = \overline{b_n}g$ and 
    \[(g|_k[\omega_M])|[T^\ast(n)] = \overline{b_n}(g|_k[\omega_M])\] by \cref{thm: miyake thm 4.5.5}. Then by \cref{thm: miyake thm 4.5.16} it follows that $L(s, g|_k[\omega_M])$ also has an Euler product expansion. Combined with \cref{thm: miyake thm 4.6.15}(2), we have 
    \begin{equation}\label{eqn: ratio shifted varLambdas}
        \frac{\varLambda_N(s,f)}{\varLambda_M(s,g)} = \frac{\varLambda_N(k-s,f|_k[\omega_N])}{\varLambda_M(k-s,g|_k[\omega_M])} = c\bigg(\frac{\sqrt{N}}{\sqrt{M}}\bigg)^{k-s}\prod_{p\mid L}\frac{1-\overline{b_p}p^{s-k} + \overline{\lambda}(p)p^{2s-k-1}}{1-\overline{a_p}p^{s-k} + \overline{\chi}(p)p^{2s-k-1}}
    \end{equation} for some constant $c$. By combining equations \labelcref{eqn: ratio varLambdas} and \labelcref{eqn: ratio shifted varLambdas} together, we obtain
    \[\bigg(\frac{N}{M}\bigg)^s\prod_{p\mid L}\frac{1-b_pp^{-s} + \lambda(p)p^{k-1-2s}}{1-a_pp^{-s} + \chi(p)p^{k-1-2s}} = c\bigg(\frac{\sqrt{N}}{\sqrt{M}}\bigg)^k\prod_{p\mid L}\frac{1-\overline{b_p}p^{s-k} + \overline{\lambda}(p)p^{2s-k-1}}{1-\overline{a_p}p^{s-k} + \overline{\chi}(p)p^{2s-k-1}}.\]
    Denote by $M_p$ and $N_p$ the $p$-components of $M$ and $N$ respectively; that is, $M_p$ and $N_p$ are the largest powers of $p$ dividing $M$ and $N$, respectively. Then for any prime factor $p$ of $L$, 
    \begin{equation}\label{eqn: p-comps}
        \bigg(\frac{N_p}{M_p}\bigg)^s\frac{1-b_pp^{-s} + \lambda(p)p^{k-1-2s}}{1-a_pp^{-s} + \chi(p)p^{k-1-2s}} = c_p
        % \bigg(\frac{\sqrt{N}}{\sqrt{M}}\bigg)^k what happened to this factor? absorbed into $c_p$?
        \frac{1-\overline{b_p}p^{s-k} + \overline{\lambda}(p)p^{2s-k-1}}{1-\overline{a_p}p^{s-k} + \overline{\chi}(p)p^{2s-k-1}}
    \end{equation} holds for some constant $c_p$ by \cref{lem: miyake lem 3.2.1}. Let $x=p^{-s}$, and let $u$ and $v$ be the degrees of
    \begin{align*}
        1-a_pp^{-s} + \chi(p)p^{k-1-2s} &= 1-a_px + \chi(p)p^{k-1}x^2 \text{ and}\\
        1-b_pp^{-s} + \lambda(p)p^{k-1-2s} &= 1-b_px + \lambda(p)p^{k-1}x^2
    \end{align*} as polynomials of $x$, respectively. Thus $0\leq u,v\leq 2$. Furthermore, write $M_p/N_p = p^e$, so that \cref{eqn: p-comps} becomes \begin{equation}\label{eqn: x substitution}
        x^e\frac{1-b_px + \lambda(p)p^{k-1}x^2}{1-a_px + \chi(p)p^{k-1}x^2} = c_p\frac{1-\overline{b_p}p^{-k}x^{-1} + \overline{\lambda}(p)p^{-k-1}x^{-2}}{1-\overline{a_p}p^{-k}x^{-1} + \overline{\chi}(p)p^{-k-1}x^{-2}}.
    \end{equation} We investigate each combination of values that $u,v$ may take, and show that if a particular combination of $u,v$ implies that $e>0$, that case could not occur. Thus in the remaining cases, $e\leq 0$ so that $M_p\mid N_p$.
    \begin{enumerate}[label=(\arabic*)]
        \item If $u = v$, take $x\to \infty$ in \cref{eqn: nuts equality} (by taking $\Re(s)\to\infty$) to deduce that $e= 0$.
        \item If $u = 1$ and $v = 0$, then $a_p\neq 0$ so that \cref{eqn: x substitution} may be rewritten as 
        \[x^e(1-\overline{a_p}p^{-k}x^{-1})=c_p(1-a_px).\] Then $e = 1$, and multiplying the above equality by $a^p$ we may rearrange to obtain 
        \[-a_pc_p = \frac{a_px-\abs{a_p}^2p^{-k}}{a_px-1},\] from which it follows that $\abs{a_p}^2 = p^k$. This is in contradiction with \cref{thm: miyake thm 4.6.17}, so this case could not occur.
        \item If $u = 0$ and $v = 1$, then \cref{eqn: x substitution} becomes $x^e(1-b_px) = c_p(1-\overline{b_p}p^{-k}x^{-1})$, from which it follows that $e = -1$.
        \item If $u = 2$ and $v = 0$, rewrite \cref{eqn: x substitution} as 
        \begin{equation}\label{eqn: nuts equality}
            x^e(1-\overline{a_p}p^{-k}x^{-1} + \overline{\chi}(p)p^{-k-1}x^{-2})=c_p(1-a_px + \chi(p)p^{k-1}x^2).
        \end{equation} Then $e=2$, so that we have 
        \[\chi(p)p^{k-1}c_p = \frac{\chi(p)p^{k-1}x^2-\chi(p)\overline{a_p}p^{-1}x + \abs{\chi(p)}^2p^{-2}}{\chi(p)p^{k-1}x^2-a_px+1}.\] It follows that $\chi(p)p^{k-1}c_p = 1$ and $\abs{\chi(p)}^2 = p^{2}$, of which the latter is impossible.
        \item If $u = 0$ and $v = 2$, deduce that $e = -2$.
        \item If $u = 2$ and $v = 1$, rewrite \cref{eqn: x substitution} as 
        \[x^e\frac{1-\overline{a_p}p^{-k}x^{-1} + \overline{\chi}(p)p^{-k-1}x^{-2}}{1-\overline{b_p}p^{-k}x^{-1}} = c_p\frac{1-a_px + \chi(p)p^{k-1}x^2}{1-b_px},\] so that $e = 1$, and rewrite this equation as 
        \[\frac{x^2-\overline{a_p}p^{-k}x + \overline \chi(p)p^{-k-1}}{x-\overline{b_p}p^{-k}} = c_p\frac{x^2-\overline\chi(p)a_pp^{-k+1}x + \overline\chi(p)p^{-k+1}}{\overline\chi(p)p^{-k+1}(1-b_px)}.\] The roots of the polynomials 
        \[x^2-\overline{a_p}p^{-k}x + \overline \chi(p)p^{-k-1}\quad \text{and}\quad x^2-\overline\chi(p)a_pp^{-k+1}x + \overline\chi(p)p^{-k+1}\] agree. However, since the product of the roots of these polynomials are equal to the constant terms of each polynomial, we must have $\overline \chi(p)p^{-k-1} = \overline\chi(p)p^{-k+1}$, a contradiction. 
        
        % Let $t= x^{-1}$. The absolute values of the roots of 
        % \[t^2-a_pt+\chi(p)p^{k-1}\] are $p^{(k-1)/2}$ by \sai{eqn: miyake eqn 4.5.41, which is equivalent to the Ramanujan-Petersson conjecture} and the absolute values of the roots of 
        % \[\overline{\chi}(p)p^{-k-1}t^2 -\overline{a_p}p^{-k}t + 1 = \overline{\chi}(p)p^{-k-1}(t^2-a_ppt + \chi(p)p^{k+1}) \] are $p^{(k+1)/2}$, which makes \cref{eqn: nuts equality} and, hence this case, impossible.
        \item If $u = 1$ and $v = 2$, deduce that $e = -1$ by taking limits.
    \end{enumerate}
    In any case, it follows for any $p$ that $M_p\mid N_p$ so that $M\mid N$, and that $\chi$ is induced by $\lambda$. Then \cref{thm: miyake thm 4.6.12} implies $f(z) = g(z)$, so that $N = M$.
\end{proof}

% \newpage\section{}

\newpage\pagestyle{frontmatter}
\printindex\newpage
\begin{bibdiv}
\begin{biblist}

\bib{df}{book}{
    title = {Abstract Algebra},
    author = {Dummit, David S.},
    author = {Foote, Richard M.},
    isbn = {978-0-471-43334-7},
    year = {2004},
    publisher = {John Wiley \& Sons, Inc.}
}

\bib{diamond}{book}{
    title = {A First Course in Modular Forms},
    author = {Diamond, Fred},
    author = {Shurman, Jerry},
    isbn = {978-0-387-23229-4},
    series = {Graduate Texts in Mathematics},
    year = {2005},
    publisher = {Springer New York, NY}
}

\bib{miyake}{book}{
    title = {Modular Forms},
    author = {Miyake, Toshitsune},
    isbn = {978-3-540-29592-1},
    series = {Springer Monographs in Mathematics},
    year = {2005},
    publisher = {Springer Berlin, Heidelberg}
}

\bib{serre}{book}{
    title = {A Course in Arithmetic},
    author = {Serre, Jean-Pierre},
    isbn = {978-0-387-90040-7},
    series = {Graduate Texts in Mathematics},
    year = {1978},
    publisher = {Springer New York, NY}
}



\end{biblist}
\end{bibdiv}
\end{document}