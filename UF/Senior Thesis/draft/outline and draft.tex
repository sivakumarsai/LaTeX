\documentclass[10pt,leqno,twoside]{article}

% packages
\usepackage[alphabetic]{amsrefs}
\usepackage{physics}
% margin spacing
\usepackage[top=1in, bottom=1in, left=1in, right=1in]{geometry}
\usepackage{amsfonts, amsmath, amssymb, amsthm}
\usepackage{fancyhdr}
\usepackage{graphicx}
\graphicspath{{./images/}}
\usepackage{enumitem}
\usepackage{quiver}
% indexing
\usepackage{imakeidx}
\makeindex
\indexsetup{level=\section*,toclevel=section,noclearpage,firstpagestyle=frontmatter}
\usepackage{hyperref}
% \hypersetup{colorlinks=true,linkcolor=blue}
\usepackage[noabbrev]{cleveref}
\crefformat{equation}{equation~#2#1#3}
\crefformat{lemma}{\textrm{lemma}~#2#1#3}

% theorems
\theoremstyle{plain}
\newtheorem{lem}{Lemma}
\newtheorem{lemma}[lem]{Lemma}
\newtheorem{thm}[lem]{Theorem}
\newtheorem{theorem}[lem]{Theorem}
\newtheorem{prop}[lem]{Proposition}
\newtheorem{proposition}[lem]{Proposition}
\newtheorem{cor}[lem]{Corollary}
\newtheorem{corollary}[lem]{Corollary}
\newtheorem{conj}[lem]{Conjecture}
\newtheorem{fact}[lem]{Fact}
\newtheorem{form}[lem]{Formula}

\theoremstyle{definition}
\newtheorem{defn}[lem]{Definition}
\newtheorem{definition/}[lem]{Definition}
\newenvironment{definition}
  {\renewcommand{\qedsymbol}{\textdagger}%
   \pushQED{\qed}\begin{definition/}}
  {\popQED\end{definition/}}
\newtheorem{example}[lem]{Example}
\newtheorem{remark}[lem]{Remark}
\newtheorem{exercise}[lem]{Exercise}
\newtheorem{notation}[lem]{Notation}

\numberwithin{equation}{section}
\numberwithin{lem}{section}

% % colors
% \usepackage{xcolor}
% \definecolor{p}{HTML}{FFDDDD}
% \definecolor{g}{HTML}{D9FFDF}
% \definecolor{y}{HTML}{FFFFCF}
% \definecolor{b}{HTML}{D9FFFF}
% \definecolor{o}{HTML}{FADECB}
% %\definecolor{}{HTML}{}

% % \highlight[<color>]{<stuff>}
% \newcommand{\highlight}[2][p]{\mathchoice%
%   {\colorbox{#1}{$\displaystyle#2$}}%
%   {\colorbox{#1}{$\textstyle#2$}}%
%   {\colorbox{#1}{$\scriptstyle#2$}}%
%   {\colorbox{#1}{$\scriptscriptstyle#2$}}}%

% header/footer formatting
\fancypagestyle{frontmatter}{
    \fancyhf{}
    \pagestyle{fancy}
    \fancyhead[LE,RO]{\thepage}
}
\fancypagestyle{body}{
    \pagestyle{fancy}
    \fancyhead[LO,RE]{\nouppercase{\rightmark}}
    \fancyhead[LE,RO]{\thepage}
}
\renewcommand{\headrulewidth}{1pt}

% paragraph indentation/spacing
\setlength{\parindent}{0pt}
\setlength{\parskip}{6pt}
\renewcommand{\baselinestretch}{1.00}

% extra commands defined here (rarely used)
\newcommand{\br}[1]{\left(#1\right)}
\newcommand{\sbr}[1]{\left[#1\right]}
\newcommand{\cbr}[1]{\left\{#1\right\}}

% bracket notation for inner product
\usepackage{mathtools}

\DeclarePairedDelimiterX{\abr}[1]{\langle}{\rangle}{#1}

% new commands
\newcommand{\textib}[1]{\textbf{\textit{#1\index{#1}}}} % adds to index
\DeclareMathOperator{\Mat}{M}
\DeclareMathOperator{\GL}{GL}
\DeclareMathOperator{\SL}{SL}
\newcommand{\smod}[1]{\;(\bmod\; #1)}
\newcommand{\smallabcd}{\big(\!\begin{smallmatrix}
    a & b \\ c & d
\end{smallmatrix}\!\big)}
\newcommand{\abcd}{\begin{pmatrix}
    a & b \\ c & d
\end{pmatrix}}
\newcommand{\slz}{\SL_2(\mathbb{Z})}
% \DeclareMathOperator{\Span}{span}
% \DeclareMathOperator{\nullity}{nullity}
% \DeclareMathOperator\Aut{Aut}
% \DeclareMathOperator\Inn{Inn}
% \DeclareMathOperator{\Orb}{Orb}
% \DeclareMathOperator{\lcm}{lcm}
% \DeclareMathOperator{\Hol}{Hol}
% \DeclareMathOperator{\Jac}{Jac}
% \DeclareMathOperator{\rad}{rad}
% \DeclareMathOperator{\Tor}{Tor}
% \DeclareMathOperator{\End}{End}
% \DeclareMathOperator{\Gal}{Gal}
% \DeclareMathOperator{\Nat}{Nat}
% \DeclareMathOperator{\Frac}{Frac}
% \DeclareMathOperator{\id}{id}
% \DeclareMathOperator{\im}{im}
% \DeclareMathOperator{\Hom}{Hom}
% \DeclareMathOperator{\Ext}{Ext}
% \DeclareMathOperator{\aug}{aug}

% indices
\setcounter{page}{1}
\setcounter{section}{-1}

% array column and row separation
\arraycolsep = 3pt
\renewcommand{\arraystretch}{.8}

% temporaries for editing
\usepackage{color}
\newcommand{\tbd}{{\Huge\color{red}{\textib{TBD}}}}
\newcommand{\sai}[1]{\textcolor{red}{#1}}

\begin{document}
\begin{titlepage}
    \begin{center}
        \vspace*{4em}
        {\Large\textbf{Strong multiplicity one for classical modular forms}}

        \vspace{6em}
        \includegraphics[scale=0.14]{uf.png}

        \vspace{6em}
        Sai Sivakumar

        Undergraduate Honors Thesis -- Spring 2024\\
        Department of Mathematics, University of Florida
    \end{center}
\end{titlepage}
\pagestyle{frontmatter}
\section*{Acknowledgements}
\newpage\tableofcontents

\newpage\section*{Introduction}
\addcontentsline{toc}{section}{Introduction}

\newpage\section{Preliminaries} \pagestyle{body} In this section, we collect the definitions and results needed to define modular forms.
\subsection{The modular group and congruence subgroups}
Let $R$ be a unital ring. The \textib{general linear group} $\GL_n(R)$ is the group of $n\times n$ invertible matrices with entries from $R$; that is, matrices with unit determinant, under matrix multiplication. The \textib{special linear group} $\SL_n(R)$ is the subgroup of $\GL_n(R)$ whose matrices have determinant $1_R$.

\begin{definition}
    The \textib{modular group} $\slz$ is the group of $2\times 2$ integer-valued matrices with determinant $1$, and is a subgroup of $\GL_2(\mathbb{R})$.
\end{definition} 
% Equivalently, the elements of $\slz$ are the integer-valued matrices $\big(\!\begin{smallmatrix}
%     a & b \\ c & d
% \end{smallmatrix}\!\big)$ with $\gcd(c,d) = 1$, since $ad-bc = 1$ is equivalent to $c$ and $d$ being coprime. 
It is well known (e.g., \cite{serre} Chapter 7, Section 1.2, Theorem 2) that 
\begin{equation}\label{eq: generators modular group}
    \slz = \left\langle\begin{pmatrix}
        1 & 1 \\ 0 & 1
    \end{pmatrix}, \begin{pmatrix}
        0 & -1 \\ 1 & 0
    \end{pmatrix}\right\rangle, 
\end{equation}
and that $\slz$ acts on the Riemann sphere $\widehat{\mathbb{C}} = \mathbb{C}\cup \{\infty\}$ by fractional linear transformations
\[\abcd(z) = \frac{az + b}{cz + d}.\] For $c\neq 0$, $-d/c$ is sent to $\infty$ and $\infty$ is sent to $a/c$; if $c = 0$ then $\infty$ is sent to $\infty$. We check that this action defines a group action: It is clear that the identity matrix sends $z$ to itself, and that 
\[\begin{pmatrix}
    q & r \\ s & t
\end{pmatrix}\abcd(z) = \begin{pmatrix}
    q & r \\ s & t
\end{pmatrix}\Big(\frac{az+b}{cz+d}\Big) = \frac{q\frac{az+b}{cz+d}+r}{s\frac{az+b}{cz+d}+t} = \frac{(qa+rc)z+qb+rd}{(sa+tb)z+sb+td} = \begin{pmatrix}
    qa+rc & qb+rd \\ sa+tb & sb+td
\end{pmatrix}(z)\] for $\big(\!\begin{smallmatrix}
    q & r \\ s & t
\end{smallmatrix}\!\big),\big(\!\begin{smallmatrix}
    a & b \\ c & d
\end{smallmatrix}\!\big)\in \slz$.
It follows that these transformations are (bi)holomorphic and are automorphisms.
% ; indeed, \[\abcd^{-1}\abcd(z) = \begin{pmatrix}
%     d & -b \\ -c & a
% \end{pmatrix}\Big(\frac{az+b}{cz+d}\Big) = \frac{\frac{d(az+b)}{cz+d}- b}{\frac{-c(az+b)}{cz+d}+a} = \frac{(ad-bc)z}{ad-bc} = z.\]
Furthermore, both $I$ and $-I$ act as the identity on $\widehat{\mathbb{C}}$, so for any $A\in \slz$, the actions of $A$ and $-A$ agree. The generators in \cref{eq: generators modular group} correspond to the maps
\[z\mapsto z+1\quad\text{and}\quad z\mapsto -1/z.\]
We consider particular subgroups of the modular group. 
\begin{definition}
    For $N$ a positive integer, the \textib{principal congruence subgroup of level $N$} is the subgroup
\[\varGamma(N) = \cbr{\abcd\in \slz: \abcd\equiv \begin{pmatrix}
    1 & 0 \\ 0 & 1
\end{pmatrix}\smod N}\]
with the matrix congruence interepreted as congruence modulo $N$ entrywise.
\end{definition}
\begin{lemma}\label{lem: varGamma(N) kernel of projection}
    The principal congruence subgroup $\varGamma(N)$ is the kernel of the natural homomorphism $\slz\to \SL_2(\mathbb{Z}/N\mathbb{Z})$, so $\varGamma(N)$ is normal in $\slz$; furthermore, the natural homomorphism is surjective.
\end{lemma}
\begin{proof}
    It is clear that $\varGamma(N)$ is the kernel of the natural homomorphism $\slz\to \SL_2(\mathbb{Z}/N\mathbb{Z})$, so we prove that this map is surjective. When $N = 1$, the natural homomorphism is the zero map (so $\varGamma(1) = \slz$). Let $N>1$, and consider an element \[\begin{pmatrix}
        \overline a & \overline b \\ \overline c & \overline d
    \end{pmatrix}\in \SL_2(\mathbb{Z}/N\mathbb{Z}),\] so that $\overline{ad-bc} = \overline 1$ (here $\overline{\,\cdot\,}$ denotes reduction modulo $N$). Then for some integer $k$, $ad-bc = 1+kN$. It follows that $1+kN$ is a multiple of $g = \gcd(c,d)$, so that $\gcd(g,N)=1$ and $\gcd(c,d,N)=1$. We show that there exist integers $i,j$ such that $c+iN$ and $d+jN$ are coprime.
    
    % Suppose there is no choice of $i,j$ such that $c+iN$ and $d+jN$ are coprime, so that for fixed $i$, $\gcd(c+iN,d+jN)>1$ for every $j$. By Dirichlet's theorem on arithmetic progressions, the sequence $(d+mN)$ contains infinitely many primes, which yields a contradiction.
    
    If $c\neq 0$, consider a solution $j$ to the system of congruences 
    \[\begin{cases}
        j\equiv 1\smod{p} & p\mid g\\
        j\equiv 0\smod{p} & p\nmid g, p\mid c,
    \end{cases}\] which may be obtained via the Chinese remainder theorem. Then $\gcd(c,d+jN) = 1$ since any prime $p$ dividing $c$ will not divide $d+jN$ with $j$ chosen as above (of primes $p$ dividing $c$, when $p\mid g$, we have $p\nmid N$, and when $p\nmid g$, we have $p\nmid d$). If $c = 0$, then $d\neq 0$ and repeat this argument with $d,i$ in place of $c,j$ respectively.

    With integers $c+iN$ and $d+jN$ coprime, there exist integers $s,t$ with $s(c+iN) + t(d+jN) = 1$. Then \begin{multline*}
        \det\begin{pmatrix}
            a-(k+aj-bi)tN & b+(k+aj-bi)sN \\ c+iN & d+jN
        \end{pmatrix} = ad+ajN-(k+aj-bi)tN(d+jN)\\-bc-biN-(k+aj-bi)sN(c+iN) = 1+kN+ajN-biN-(k+aj-bi)N = 1
    \end{multline*} and 
    \[\begin{pmatrix}
        \overline{a-(k+aj-bi)tN} & \overline{b+(k+aj-bi)sN} \\ \overline{c+iN} & \overline{d+jN}
    \end{pmatrix} = \begin{pmatrix}
        \overline a & \overline b \\ \overline c & \overline d
    \end{pmatrix}\] as needed.
\end{proof}
Therefore, by the first isomorphism theorem, $\slz/\varGamma(N)\cong \SL_2(\mathbb{Z}/N\mathbb{Z})$. 
\begin{lemma}
    The index of $\varGamma(N)$ in $\slz$ is $[\slz: \varGamma(N)] = N^3\prod_{p\mid N}(1-1/p^2)$.
\end{lemma}
\begin{proof}
    We first show that $\abs{\SL_2(\mathbb{Z}/p^e\mathbb{Z})} = p^{3e}(1-1/p^2)$ for a prime $p$ by induction on $e$.

    The determinant is a surjective homomorphism from $\GL_2(\mathbb{Z}/p\mathbb{Z})$ to $(\mathbb{Z}/p\mathbb{Z})^\times\cong \mathbb{Z}/(p-1)\mathbb{Z}$ with kernel $\SL_2(\mathbb{Z}/p\mathbb{Z})$. Observe that $\abs{\GL_2(\mathbb{Z}/p\mathbb{Z})}$ is the number of ordered bases of $(\mathbb{Z}/p\mathbb{Z})^2$, given by $(p^2-1)(p^2-p)$. (The first factor counts the number of valid first basis vectors, and the second factor counts the number of valid second basis vectors after choosing a first basis vector.) Then by the first isomorphism theorem, $\abs{\SL_2(\mathbb{Z}/p\mathbb{Z})} = (p^2-1)(p^2-p)/(p-1) = p(p^2-1) = p^3(1-1/p^2)$.
    
    By \cref{lem: varGamma(N) kernel of projection}, the natural homomorphism $\slz\to \SL_2(\mathbb{Z}/p^e\mathbb{Z})$ is surjective. The surjection $\slz\to \SL_2(\mathbb{Z}/p^e\mathbb{Z})$ is equal to the composition of the natural homomorphisms $\slz\to \SL_2(\mathbb{Z}/p^{e+1}\mathbb{Z})$ and $\SL_2(\mathbb{Z}/p^{e+1}\mathbb{Z})\to \SL_2(\mathbb{Z}/p^e\mathbb{Z})$, from which it follows that $\SL_2(\mathbb{Z}/p^{e+1}\mathbb{Z})\to \SL_2(\mathbb{Z}/p^e\mathbb{Z})$ is surjective also.
    % \[\begin{tikzcd}
    %     {\slz} & {\SL_2(\mathbb{Z}/p^e\mathbb{Z})} \\
    %     {\SL_2(\mathbb{Z}/p^{e+1}\mathbb{Z})}
    %     \arrow[from=2-1, to=1-2]
    %     \arrow[two heads, from=1-1, to=2-1]
    %     \arrow[two heads, from=1-1, to=1-2]
    % \end{tikzcd}\]
    % from which we find that the map $\SL_2(\mathbb{Z}/p^{e+1}\mathbb{Z})\to \SL_2(\mathbb{Z}/p^e\mathbb{Z})$ is surjective.

    Any element $\gamma$ of $\ker(\SL_2(\mathbb{Z}/p^{e+1}\mathbb{Z})\to \SL_2(\mathbb{Z}/p^e\mathbb{Z}))$ is of the form
    \[\begin{pmatrix}
        \overline{1+ip^e} & \overline{rp^e} \\ \overline{sp^e} & \overline{1+jp^e}
    \end{pmatrix},\] for $i,j,r,s\in \{0,\dots,p-1\}$ and $i = p-j$ (so that $\det \gamma = 1$). Hence $\abs{\ker(\SL_2(\mathbb{Z}/p^{e+1}\mathbb{Z})\to \SL_2(\mathbb{Z}/p^e\mathbb{Z}))} = p^3$. Suppose that $\abs{\SL_2(\mathbb{Z}/p^e\mathbb{Z})} = p^{3e}(1-1/p^2)$. Then by the first isomorphism theorem, $\abs{\SL_2(\mathbb{Z}/p^{e+1}\mathbb{Z})} = p^3\abs{\SL_2(\mathbb{Z}/p^e\mathbb{Z})} = p^3p^{3e}(1-1/p^2) = p^{3(e+1)}(1-1/p^2)$ as needed. By induction, $\abs{\SL_2(\mathbb{Z}/p^e\mathbb{Z})} = p^{3e}(1-1/p^2)$ for any $e$.

    There is an isomorphism of matrix groups $\Mat_2(\prod_{i=1}^nR_i)\cong \prod_{i=1}^n\Mat_2(R_i)$ for rings $R_i$ which restricts to the isomorphisms $\GL_2(\prod_{i=1}^nR_i)\cong \prod_{i=1}^n\GL_2(R_i)$ and $\SL_2(\prod_{i=1}^nR_i)\cong \prod_{i=1}^n\SL_2(R_i)$ since $(\prod_{i=1}^nR_i)^\times\cong \prod_{i=1}^nR_i^\times$ and $1_{\prod_{i=1}^nR_i} = \prod_{i=1}^n1_{R_i}$.

    Let $N$ have prime factorization $N = \prod_{p\mid N} p^{e}$. By the Chinese remainder theorem, $\mathbb{Z}/N\mathbb{Z}\cong \prod_{p\mid N}\mathbb{Z}/p^{e}\mathbb{Z}$. It follows that $\SL_2(\mathbb{Z}/N\mathbb{Z})\cong \prod_{p\mid N}\SL_2(\mathbb{Z}/p^{e}\mathbb{Z})$, from which we have \[\abs{\SL_2(\mathbb{Z}/N\mathbb{Z})}= \prod_{p\mid N}\abs{\SL_2(\mathbb{Z}/p^{e}\mathbb{Z})} = \prod_{p\mid N}p^{3e}(1-1/p^2) = N^3\prod_{p\mid N}(1-1/p^2)\] as desired.
\end{proof}

\begin{definition}\label{def: congruence subgroup}
    A subgroup $\varGamma$ of $\slz$ is a \textib{congruence subgroup (of level $N$)} if for some positive integer $N$, $\varGamma(N)\subseteq \varGamma$.
\end{definition}
Any congruence subgroup $\varGamma$ has finite index in $\slz$. Let $\varGamma$ be a level $N$ congruence subgroup. By the third isomorphism theorem,
\[\frac{\slz}{\varGamma}\cong \frac{\slz/\varGamma(N)}{\varGamma/\varGamma(N)},\] and since $\slz/\varGamma(N)$ is finite the result holds. (Indeed, $\varGamma/\varGamma(N)$ is a subgroup of $\slz/\varGamma(N)$; additionally observe that $\varGamma(N)$ has finite index in $\varGamma$ due to the inclusions $\varGamma(N)\subset \varGamma \subset \slz$ or also by the above argument.)

\begin{definition}
    Frequently used congruence subgroups are 
    \[\varGamma_0(N) = \cbr{\abcd\in\slz: \abcd\equiv\begin{pmatrix}
        \ast & \ast \\ 0 & \ast
    \end{pmatrix}\smod N}\text{ and}\]
    \[\varGamma_1(N) = \cbr{\abcd\in \slz: \begin{pmatrix}
        a & b \\ c & d 
    \end{pmatrix}\equiv \begin{pmatrix}
        1 & \ast \\ 0 & 1
    \end{pmatrix}\smod N},\] where $\ast$ denotes unspecified quantities and the matrix congruences are to be taken entrywise.    
\end{definition}
For any positive integer $N$, the inclusions $\varGamma(N)\subset\varGamma_1(N)\subset\varGamma_0(N)\subset\slz$ hold.

\begin{lemma}
    The map $\varGamma_1(N)\to \mathbb{Z}/N\mathbb{Z}$ given by \[\abcd\mapsto \overline b\] is surjective with kernel $\varGamma(N)$. 
\end{lemma}
\begin{proof}
    For $\overline b\in\mathbb{Z}/N\mathbb{Z}$, \[\begin{pmatrix}
        1 & b \\ 0 & 1
    \end{pmatrix}\] has unit determinant and maps to $\overline b$ under the above map. It is clear that the kernel of the above map is $\varGamma(N)$.
    % Let $\overline b\in\mathbb{Z}/N\mathbb{Z}$ be given. Let $a,d\equiv 1\smod N$ and $c\equiv 0\smod N$, and write $a= iN+1$, $d = jN+1$, and $c = kN$ for some integers $i,j,k$. If $1= ad-bc = ijN^2 + (i+j)N + 1 - bkN$, then $(ijN+ i + j)/k = b$. Take $k=1$ and reduce modulo $N$ to obtain the equation $i+j\equiv b \smod N$, which has many solutions, for example $i = b$ and $j =0$. These choices for $i,j,k$ produce a preimage of $b\smod N$ under the above map. Indeed, \[\begin{pmatrix}
    %     a & b \\ c & d
    % \end{pmatrix} = \begin{pmatrix}
    %     bN+1 & b \\ N & 1
    % \end{pmatrix}\mapsto b\quad\text{and}\quad \det\begin{pmatrix}
    %     bN+1 & b \\ N & 1
    % \end{pmatrix} = 1.\]
    % If $A\in \varGamma_1(N)$ and \[A = \begin{pmatrix}
    %     a & b \\ c & d
    % \end{pmatrix}\mapsto \overline 0,\] then $b \equiv 0\smod N$ for some integer $k$. Since $a,d\equiv 1\smod N$ and $c\equiv 0\smod N$, it follows that $A\in \varGamma(N)$. Certainly any element of $\varGamma(N)$ is contained in the kernel of the above map.
\end{proof} 
It follows that $\varGamma_1(N)/\varGamma(N)\cong \mathbb{Z}/N\mathbb{Z}$, so that the index of $\varGamma(N)$ in $\varGamma_1(N)$ is $N$. What follows is a similar result.

\begin{lemma}
    The map $\varGamma_0(N)\to (\mathbb{Z}/N\mathbb{Z})^\times$ given by \[\abcd\mapsto \overline d\] 
    is surjective with kernel $\varGamma_1(N)$.
\end{lemma}
\begin{proof}
    Exercise 1.2.3. (d) in \cite{diamond}.% DO NOT DELETE SOLUTION Let $\overline d \in (\mathbb{Z}/N\mathbb{Z})^\times$ be given, and note that $\gcd(d,N)=1$. Let $c \equiv 0\smod N$ so that $c=kN$ for some integer $k$. Choose $k$ coprime to $d$ so that $\gcd(d,kN) = \gcd(d,c) = 1$. Then there exist integers $a,b$ with $ad-bc = 1$. Then \[\begin{pmatrix}
    %     a & b \\ c & d
    % \end{pmatrix}\mapsto d\quad\text{and}\quad \det\begin{pmatrix}
    %     a & b \\ c & d
    % \end{pmatrix} = 1.\] 
    % If $A\in \varGamma_0(N)$ and \[A = \begin{pmatrix}
    %     a & b \\ c & d
    % \end{pmatrix}\mapsto \overline 1,\] so $d = jN+1$ for some integer $j$. Since $c\equiv 0\smod N$, $c= kN$ and $\gcd(d,c) = 1$. There exist integers $a,b$ with $1 = ad-bc = a(jN+1) + bkN = a+ (aj + bk)N$. Reducing modulo $N$ gives $a\equiv 1\smod N$. It follows that $A\in \varGamma_1(N)$. Conversely, if $A\in \varGamma_1(N)$, then $A\mapsto \overline 1$.
\end{proof}
Hence $\varGamma_0(N)/\varGamma_1(N)\cong (\mathbb{Z}/N\mathbb{Z})^\times$, so that the index of $\varGamma_1(N)$ in $\varGamma_0(N)$ is $\phi(N) = \abs{(\mathbb{Z}/N\mathbb{Z})^\times} = N\prod_{p\mid N}(1-1/p)$, where $\phi$ denotes the Euler totient function. 
\begin{corollary}
    We obtain \[[\slz : \varGamma_0(N)] = \frac{[\slz : \varGamma(N)]}{[\varGamma_0(N) : \varGamma_1(N)][\varGamma_1(N) : \varGamma(N)]} = \frac{N^3\prod_{p\mid N}(1-1/p^2)}{N\cdot N\prod_{p\mid N}(1-1/p)} = N\prod_{p\mid N}(1+1/p).\]
\end{corollary} 

\subsection{Modular curves}
Recall that $\slz$ acts on the Riemann sphere $\widehat{\mathbb{C}}$ by fractional linear transformations \[\abcd(z) = \frac{az + b}{cz+d}.\] The \textib{upper half plane} $\mathcal{H} = \cbr{z\in\mathbb{C} : \Im(z)}$ inherits this action. If $\gamma\in \slz$ with $\gamma = \smallabcd$ and $z\in \mathbb C$, the computation $(az+b)(c\overline z + d) = ac\abs{z}^2+bd + adz + bc\overline z = ac\abs{z}^2+bd + adz + (ad-1)\overline z = ac\abs{z}^2+bd + ad(z+\overline z) -\overline z$ shows that
\begin{equation}\label{eq: imaginary parts}
    \Im(\gamma(z)) = \Im\Big(\frac{az+b}{cz+d}\Big) = \frac{\Im((az+b)(c\overline z + d))}{\abs{cz+d}^2} = \frac{\Im(-\overline z)}{\abs{cz+d}^2} = \frac{\Im(z)}{\abs{cz+d}^2}.
\end{equation}
It follows that if $z\in \mathcal{H}$, then $\gamma(z)\in \mathcal{H}$ also.

Since $\slz$ acts on the upper half plane, so does any congruence subgroup $\varGamma$. 
\begin{definition}
    The \textib{modular curve} $Y(\varGamma)$ is the orbit space $\varGamma\backslash \mathcal{H} = \cbr{\varGamma z : z\in\mathcal H}$.
\end{definition}
The modular curves for the congruence subgroups $\varGamma_0(N), \varGamma_1(N)$, and $\varGamma(N)$ are denoted by $Y_0(N), Y_1(N)$, and $Y(N)$, respectively.

We topologize modular curves and briefly show they are compact Riemann surfaces. Let $\varGamma$ be a congruence subgroup. The upper half plane $\mathcal H$ is given the Euclidean topology (as a subspace of $\mathbb{C}$ or of $\mathbb{R}^2$), and the natural surjection \[\pi \colon \mathcal H \to Y(\varGamma)\quad \text{defined by}\quad  z\mapsto \varGamma z\] induces the quotient topology on $Y(\varGamma)$ (i.e., the open sets in $Y(\varGamma)$ are those with open preimages under $\pi$).

\begin{lemma}\label{lem: slz acts as open maps}
    The quotient map $\pi \colon \mathcal H \to Y(\varGamma)$ is an open map.
\end{lemma}
\begin{proof}
    It suffices to show that the projection $\pi$ takes an open ball $B$ in $\mathcal{H}$ to an open set in $Y(\varGamma)$; that is, to show that \[\bigcup_{\gamma\in \varGamma}\gamma(B)\] is open. Since each $\gamma^{-1}$ is continuous, each $\gamma(B)$ is open, so $\pi$ is an open map.
\end{proof}

\begin{lemma}
    The action of the modular group $\slz$ on $\mathcal H$ is properly discontinuous; that is, for $z_1,z_2\in \mathcal H$, (including the case $z_1 = z_2$) there exist neighborhoods $U_1$ of $z_1$ and $U_2$ of $z_2$ in $\mathcal H$ such that for any $\gamma\in \slz$, if $\gamma(U_1)\cap U_2\neq \emptyset$, then $\gamma(z_1) = z_2$. Equivalently, for any $\gamma\in\slz$, if $\gamma(z_1)\neq z_2$, then $\gamma(U_1)\cap U_2= \emptyset$.
\end{lemma}
\begin{proof}
    Let $V_1$ and $V_2$ be neighborhoods of $z_1$ and $z_2$, respectively, with compact closure (e.g. open balls or bounded open sets). Let $y_1 = \sup_{z\in V_1}\cbr{\Im(z)}$, $Y_1 = \sup_{z\in V_1}\cbr{\Im(z)}$, and $y_2 = \inf_{z\in V_2}\cbr{\Im(z)}$. Then from \cref{eq: imaginary parts}, we have for $\gamma = \big(\!\begin{smallmatrix}
        a & b \\ c & d
    \end{smallmatrix}\!\big)\in \slz$ not equal to $\pm I$ (i.e., $c\neq 0$) and $z\in V_1$ that \[\Im(\gamma(z)) = \frac{\Im(z)}{\abs{cz+d}^2}\leq \frac{\Im(z)}{c^2\abs{z}^2}\leq \frac{1}{c^2\Im(z)}\leq \frac{1}{c^2y_1}\] and \[\Im(\gamma(z)) = \frac{\Im(z)}{\abs{cz+d}^2}\leq \frac{Y_1}{\Re(cz+d)^2} = \frac{Y_1}{(c\Re(z)+d)^2},\] so that $\Im(\gamma(z))\leq\min\cbr{1/c^2y_1,Y_1/(c\Re(z)+d)^2}$. All but finitely many integers $c$ may be chosen so that $1/c^2y_1< y_2$, and of the finitely many $c$ where this inequality does not hold, we may choose all but finitely many $d$ such that $Y_1/(c\Re(z)+d)^2< y_2$ uniformly in $z$ (since $V_1$ has compact closure). Thus for all but finitely many integers $c,d$, if $\gamma  = \big(\!\begin{smallmatrix}
        a & b \\ c & d
    \end{smallmatrix}\!\big)\in \slz$, then \begin{equation}\label{eq: sup inf}
        \sup_{z\in V_1}\cbr{\Im(\gamma(z))}< \inf_{z\in V_2}\cbr{\Im(z)}.
    \end{equation} Thus for $\gamma = \big(\!\begin{smallmatrix}
        a & b \\ c & d
    \end{smallmatrix}\!\big)\in \slz$ satisfying \cref{eq: sup inf}, $\gamma(V_1)\cap V_2=\emptyset$. We show that there are only finitely many $\gamma\in \slz$ for which $\gamma(V_1)\cap V_2\neq\emptyset$.

    We determine the matrices in $\slz$ which have fixed bottom row $(c,d)$. This is equivalent to finding all matrices $\eta = \big(\!\begin{smallmatrix}
        A & B \\ C & D
    \end{smallmatrix}\!\big)\in \slz$ such that $\eta\gamma$ has bottom row $(c,d)$, where $\gamma\in \slz$ is a fixed matrix with  bottom row $(c,d)$. Evidently $C=0$ and $D = 1$, and since $1 = AD - BC = A$, we have that $\eta$ must have the form $\big(\!\begin{smallmatrix}
        1 & B \\ 0  & 1
    \end{smallmatrix}\!\big)$, and $B$ may be taken to be any integer. Explicitly, the matrices in $\slz$ with bottom row $(c,d)$ are \[\cbr{\begin{pmatrix}
        1 & k \\ 0 & 1
    \end{pmatrix}\begin{pmatrix}
        a & b \\ c & d
    \end{pmatrix}: k\in \mathbb{Z}},\]
    where $(a,b)$ is one such pair such that $ad-bc = 1$. Thus for $\gamma\in \cbr{\big(\!\begin{smallmatrix}
        1 & k \\ 0 & 1
    \end{smallmatrix}\!\big)\big(\!\begin{smallmatrix}
        a & b \\ c & d
    \end{smallmatrix}\!\big): k\in \mathbb{Z}}$, the intersection $\gamma(V_1)\cap V_2 = \big(\big(\!\begin{smallmatrix}
        a & b \\ c & d
    \end{smallmatrix}\!\big)V_1+ k\big)\cap V_2 =\emptyset$ for all but finitely many $k$; that is, for finitely many $\gamma$. So of the finitely many pairs of integers $(c,d)$ for which \cref{eq: sup inf} does not hold, only finitely many matrices $\gamma\in \slz$ with bottom row $(c,d)$ exist with $\gamma(V_1)\cap V_2\neq \emptyset$.

    Let $F = \cbr{\gamma\in \slz: \gamma(V_1)\cap V_2\neq \emptyset, \gamma(z_1)\neq z_2}$, which is finite by the above argument. For each $\gamma \in F$ there exist disjoint neighborhoods $V_{1,\gamma}$ of $\gamma(z_1)$ and $V_{2,\gamma}$ of $z_2$ in $\mathcal H$. Let \[U_1 = V_1\cap \Bigg(\bigcap_{\gamma\in F}\gamma^{-1}(V_{1,\gamma})\Bigg)\quad\text{and}\quad U_2 = V_2\cap \Bigg(\bigcap_{\gamma\in F}V_{2,\gamma}\Bigg),\] and note that $U_1,U_2$ are open, as elements of $\slz$ are open maps on $\mathcal H$ by $\cref{lem: slz acts as open maps}$. Then take any $\gamma\in \slz$ with $\gamma(U_1)\cap U_2\neq \emptyset$. If $\gamma\not\in F$, then we must have $\gamma(z_1) = z_2$. Suppose that $\gamma\in F$. Then $U_1\subset \gamma^{-1}(U_{1,\gamma})$ and $U_2\subset U_{2,\gamma}$ so that $ \gamma(U_1)\cap U_2\subset U_{1,\gamma}\cap U_{2,\gamma}$. But $\gamma(U_1)\cap U_2\neq \emptyset$, which is in contradiction with $U_{1,\gamma}$ and $U_{2,\gamma}$ chosen to be disjoint. Hence $\gamma\not\in F$, so that $\gamma(z_1)= z_2$.
\end{proof}

Furthermore, $\pi(U_1)\cap \pi(U_2)=\emptyset$ in $Y(\varGamma)$ is equivalent to $\big(\bigcup_{\gamma\in \varGamma}\gamma(U_1)\big)\cap U_2 = \emptyset$ in $\mathcal H$. The equality
\begin{multline}\label{eq: disjoint nbds in modular curve}
    \emptyset = \pi^{-1}(\pi(U_1)\cap \pi(U_2)) = \Bigg(\bigcup_{\gamma\in \varGamma}\gamma(U_1)\Bigg)\cap \Bigg(\bigcup_{\gamma\in \varGamma}\gamma(U_2)\Bigg) = \bigcup_{\gamma,\eta\in \varGamma}\gamma(U_1)\cap \eta(U_2)\\ = \bigcup_{\eta\in \varGamma}\eta\Bigg[\Bigg(\bigcup_{\gamma\in \varGamma}(\eta^{-1}\gamma)(U_1)\Bigg)\cap U_2\Bigg] = \bigcup_{\eta\in \varGamma}\eta\Bigg[\Bigg(\bigcup_{\gamma^\prime\in \varGamma}\gamma^\prime(U_1)\Bigg)\cap U_2\Bigg]
\end{multline}
implies that $\big(\bigcup_{\gamma^\prime\in \varGamma}\gamma^\prime(U_1)\big)\cap U_2$ must be empty, from which the above equivalence follows.

\begin{proposition}
    The space $Y(\varGamma)$ is second-countable, connected, and Hausdorff.
\end{proposition}

\begin{proof}
    Since $\pi$ is open and $\mathcal H$ is second-countable, $Y(\varGamma)$ is second-countable. As $\mathcal H$ is connected and $\pi$ is continuous, $Y(\varGamma)$ is connected.

    Let $\pi(z_1)$ and $\pi(z_2)$ be distinct points in $Y(\varGamma)$ (so that $\gamma(z_1)\neq z_2$ for any $\gamma\in \varGamma$), and take neighborhoods $U_1$ of $z_1$ and $U_2$ of $z_2$ such that for any $\gamma\in \slz$, if $\gamma(U_1)\cap U_2\neq \emptyset$, then $\gamma(z_1) = z_2$, as per the previous result. Then $\big(\bigcap_{\gamma\in\varGamma}\gamma(U_1)\big)\cap U_2 = \emptyset$ since $\gamma(z_1)\neq z_2$ for every $\gamma\in \varGamma$. From the discussion surrounding \cref{eq: disjoint nbds in modular curve}, we have that $\pi(U_1)\cap\pi(U_2) = \emptyset$ as needed, with $\pi(U_1),\pi(U_2)$ open since $\pi$ is an open mapping.
\end{proof}

What remains is to compactify $Y(\varGamma)$ and to put charts on the resulting compact space. \tbd


\subsection{Elliptic points and cusps}

\newpage\section{Modular forms}
In this section, we define modular forms, \tbd
\subsection{Definitions}
\begin{definition}
    For an integer $k$ and a congruence subgroup $\varGamma$, a meromorphic function $f\colon \mathcal H\to \mathbb{C}$ is \textib{weakly modular of weight $k$ (with respect to $\varGamma$)} if \[f(\gamma(z)) = (cz+d)^kf(z)\quad\text{for any}\quad \gamma = \abcd\in \varGamma, ~z\in \mathcal H.\qedhere\]
\end{definition}
Sometimes we will say that a meromorphic function $f\colon \mathcal H\to \mathcal C$ is just ``weakly modular'' if the weight $k$ and congruence subgroup $\varGamma$ are clear from context.
For any matrix $\gamma=\smallabcd\in \slz$ the \textib{factor of automorphy} $j(\gamma,z)$ for $z\in \mathbb{C}$ is defined by \[j(\gamma,z) = cz+d.\] For any integer $k$ and $\gamma\in \slz$ define the \textib{weight-$k$ operator} $|_k[\gamma]$ on functions $f\colon \mathcal H\to \mathbb{C}$ by \[(f|_k[\gamma])(z) = j(\gamma,z)^{-k}f(\gamma(z)),\quad\text{for } z\in \mathcal H.\] Note that $|_k[\gamma]$ acts on the right of $f$, and composes left to right with other weight-$k$ operators.
Since the factor of automorphy $j(\gamma,z) = cz+d$ cannot be zero or infinity (since $z\in \mathcal H$), if $f$ is meromorphic on $\mathcal H$, then so is $f|_k[\gamma]$, and the number of poles and zeroes of $f|_k[\gamma]$ and $f$ are the same.

So for an integer $k$ and a congruence subgroup $\varGamma$, a meromorphic function $f\colon \mathcal H \to \mathbb{C}$ is weakly modular of weight $k$ (with respect to $\varGamma$) if \[f|_k[\gamma] = f\quad\text{for all }\gamma\in \varGamma,\] and this is equivalent to the original definition above. Note that if $f$ is weakly modular with respect to $\varGamma$, then the zeroes of $f$ and poles of $f$ are $\varGamma$-invariant sets.

\begin{lemma}\label{lem: props of automorphy}
    For any $\gamma,\eta\in \slz$ and $z\in \mathcal H$, we have $j(\gamma\eta,z) = j(\gamma,\eta(z))j(\eta,z)$ and $|_k[\gamma\eta] = |_k[\gamma]|_k[\eta]$.
\end{lemma}
\begin{proof}
    Indeed, \tbd
\end{proof}
From the lemma it follows that if $f\colon \mathcal H\to\mathbb C$ is weakly modular with respect to some set of matrices $A$, then $f$ is weakly modular with respect to the group generated by $A$. So with $\slz = \big\langle\big(\!\begin{smallmatrix}
    1 & 1 \\ 0 & 1
\end{smallmatrix}\!\big), \big(\!\begin{smallmatrix}
    0 & -1 \\ 1 & 0
\end{smallmatrix}\!\big)\big\rangle$, a function $f\colon \mathcal H\to \mathbb C$ is weakly modular of weight $k$ with respect to $\slz$ if
\[f(z+1) = f\big(\big(\!\begin{smallmatrix}
    1 & 1 \\ 0 & 1
\end{smallmatrix}\!\big)z\big) = (0z+1)^kf(z) = f(z)\quad\text{and}\quad f(-1/z) = f\big(\big(\!\begin{smallmatrix}
    0 & -1 \\ 1 & 0
\end{smallmatrix}\!\big)z\big) = z^kf(z).\] It follows that weakly modular functions of weight $k$ with respect to $\slz$ are $\mathbb Z$-periodic. A similar phenomenon happens for weakly modular functions of weight $k$ with respect to congruence subgroups. If $\varGamma$ is a congruence subgroup of level $N$, then $\varGamma(N)\subset \varGamma$, so that $\varGamma$ contains a translation matrix of the form $\big(\!\begin{smallmatrix}
    1 & h \\ 0 & 1
\end{smallmatrix}\!\big)$ for some minimal positive integer $h$ dividing $N$. To see that $h$ necessarily divides $N$, observe that if 
\[\begin{pmatrix}
    1 & b_1 \\ 0 & 1
\end{pmatrix},\begin{pmatrix}
    1 & b_2 \\ 0 & 1
\end{pmatrix}\in\varGamma, \quad \text{then}\quad \begin{pmatrix}
    1 & \gcd(b_1,b_2) \\ 0 & 1
\end{pmatrix}\in \varGamma.\]
By a similar computation, it follows that weakly modular functions of weight $k$ with respect to $\varGamma$ are $h\mathbb Z$-periodic.

Let $D = \cbr{q\in\mathbb C : \abs{q} <1}$ be the open unit disk in $\mathbb C$ and let $D^\prime = D\!\setminus \!\cbr{0}$ denote the punctured open unit disk in $\mathbb C$. The exponential map $z\mapsto e^{2\pi i z/h} = q$ is a $h\mathbb{Z}$-periodic holomorphic map which maps $\mathcal H$ to $D^\prime$. Since $f$ is $h\mathbb{Z}$-periodic, it follows that the function $\tilde f\colon D^\prime\to\mathbb C$ corresponding to $f$ defined by $\tilde f(q) = f(h\log (q)/2\pi i)$ (so $f(z) = \tilde f(e^{2\pi i z/h})$) is well defined because a branch of the logarithm is determined up to integral multiples of $2\pi i$.

The logarithm can be defined holomorphically about each $q\in D^\prime$. It follows that $\tilde f$ is meromorphic on $D^\prime$ since $f$ is meromorphic on $\mathcal H$, and so $\tilde f$ has a Laurent expansion $\tilde f(q) = \sum_{n\in\mathbb Z}a_nq^n$ at each $q$ in a punctured neighborhood of $q=0$ (and $a_n = 0$ for all $n$ sufficiently small). Moreover, if $f$ is holomorphic on $\mathcal H$, it follows that $\tilde f$ is also on $D^\prime$.

From $\abs{q} = e^{-2\pi \Im z/h}$, it follows that $q$ tends to zero as $\Im z$ tends to infinity. We \textit{define $f$ to be meromorphic (holomorphic) at $\infty$} if $\tilde f$ has a meromorphic (holomorphic) extension to $q = 0$ (and in the holomorphic case the Laurent series at $q=0$ sums over the nonnegative integers). The Laurent series of $\tilde f$ about $q=0$ is used to obtain a Fourier series expansion of $f$, given by 
\[f(z) = \sum_{n\in\mathbb Z} a_n(f)q^n,\quad q = e^{2\pi i z/h},\] valid in a half plane $\cbr{z\in\mathbb Z: \Im z>\tau}$ for $\tau$ large enough (so that $q$ lies in a punctured neighborhood of zero). If $f$ is holomorphic on $\mathcal H$ and is holomorphic at $\infty$, the Fourier series expansion becomes $f(z) = \sum_{n=0}^\infty a_n(f)q^n$, $q = e^{2\pi i z/h}$, valid for $z\in \mathcal H$. To see that a weakly modular holomorphic function $f\colon \mathcal H\to \mathbb C$ is holomorphic at $\infty$, it suffices to show that $\lim_{\Im z\to\infty} f(z)$ exists or that $f(z)$ is bounded as $\Im z$ grows unboundedly.
\begin{definition}
    Let $\varGamma$ be a congruence subgroup of $\slz$ and let $k$ be an integer. A function $f\colon \mathcal H\to\mathbb C$ is a \textib{modular form of weight $k$ (with respect to $\varGamma$)} if it satisfies the following: \begin{enumerate}[label=(\arabic*)]
        \item $f$ is holomorphic on $\mathcal H$,
        \item $f$ is weakly modular of weight $k$ with respect to $\varGamma$, and
        \item $f|_{k}[\alpha]$ is holomorphic at $\infty$ for all $\alpha\in\slz$.
    \end{enumerate}Furthermore, if for every $\alpha\in\slz$, the coefficient $a_0$ vanishes in the Fourier series expansion of $f|_k[\alpha]$, then we call $f$ a \textib{cusp form of weight $k$ (with respect to $\varGamma$)}. The set of modular forms (cusp forms) of weight $k$ with respect to $\varGamma$ is denoted by \textib{$\mathcal M_k(\varGamma)$} (\textib{$\mathcal S_k(\varGamma)$}).
\end{definition}
Recall that the $\varGamma$-equivalence classes of points in $\mathbb Q\cup \infty$ are the cusps of $\varGamma$ \sai{(of the compactified modular curve $X(\varGamma)$?)}. There are finitely many cusps, at most the index of $\varGamma$ in $\slz$ which we showed was finite in the discussion following \cref{def: congruence subgroup}. Write a cusp $s$ as $\alpha(\infty)$ for some $\alpha\in\slz$. Then holomorphy at $s$ of a modular form $f$ is defined by holomorphy at $\infty$ of $f|_k[\alpha]$, where $f|_k[\alpha]$ is a modular form of weight $k$ with respect to $\alpha^{-1}\varGamma\alpha$. The group $\alpha^{-1}\varGamma\alpha$ is a congruence subgroup since for some $N>0$, the principal congruence subgroup $\varGamma(N)$ is contained in $\varGamma$ and is normal in $\slz$, so that $\varGamma(N) = \alpha^{-1}\varGamma(N)\alpha\subset \alpha^{-1}\varGamma\alpha$\sai{(?)}. 

As congruence subgroups have finite index in $\slz$, only finitely many coset representatives $\alpha_j$ in a decomposition $\slz= \bigcup_{j}\varGamma\alpha_j$ are needed to verify condition (3) in the definition above, and in verifying that the term $a_0$ vanishes for all Fourier series expansions for cusp forms: we have by condition (2) of the above definition that $f|_k[\gamma\alpha_j] = f|_k[\alpha_j]$.
\subsection{Dimension formulas}
\subsection{Dirichlet characters and $L$-functions}
For any positive integer $N$, denote by $Z_N$ the group $\mathbb{Z}/N\mathbb{Z}$.
\begin{definition}
    A \textib{Dirichlet character modulo $N$} is a homomorphism 
    \[\chi\colon Z_N^\times\to \mathbb{C}^\times.\] (We will sometimes suppress ``Dirichlet'' when referring to Dirichlet characters.)
\end{definition}
The product of two Dirichlet characters $\chi,\psi$, defined by pointwise multiplication $(\chi\psi)(n) = \chi(n)\psi(n)$ for $n\in Z_N^\times$, is a Dirichlet character. The trivial map is a Dirichlet character, called the trivial character. Hence the set of Dirichlet characters of $Z_N^\times$ forms a group called the \textib{dual group} of $Z_N^\times$, denoted $\widehat{Z_N^\times}$. Since $Z_N^\times$ is a finite group, the image of any character lies in the roots of unity. Thus the inverse of a Dirichlet character is its complex conjugate character $\overline \chi$, defined by taking the complex conjugate pointwise: $\overline\chi(n) = \overline{\chi(n)}$ for $n\in Z_N^\times$, where $\overline{\,\cdot\,}$ denotes complex conjugation. Note that the only Dirichlet character of $Z_1^\times$ is the trivial character $\mathbf{1}_1$. 

\begin{proposition}
    The dual group $\widehat{Z_N^\times}$ is (noncanonically) isomorphic to $Z_N^\times$; it follows that the number of Dirichlet characters modulo $N$ is $\phi(N)$.
\end{proposition}

\begin{proof}
    This follows from Exercise 5.2.14 in \cite{df}, which states that finite Abelian groups are (noncanonically) self-dual. \tbd
\end{proof}

\begin{proposition}
    The groups $Z_N^\times$ and $\widehat{Z_N^\times}$ satisfy the following orthogonality relations:
\begin{equation}\label{eq: ortho rels}
    \sum_{n\in Z_N^\times}\chi(n) = \begin{cases}
    \phi(N) & \text{if $\chi = \mathbf 1$}, \\
    0 & \text{if $\chi\neq \mathbf 1$},
\end{cases}\quad \sum_{\chi\in \widehat{Z_N^\times}}\chi(n) = \begin{cases}
    \phi(N) & \text{if $n = 1$}, \\
    0 & \text{if $n\neq 1$}.
\end{cases}
\end{equation}
\end{proposition}

\begin{proof}
    Exercise 5.2.14 in \cite{df}. \tbd 
\end{proof}

Any Dirichlet character $\chi$ modulo $d$ may be lifted to a character $\chi_N$ modulo $N$ when $d\mid N$, by the rule $\chi_N(n\smod N) = \chi(n\smod d)$ for all $n\in\mathbb{Z}$ coprime to $N$. In other words, if $\pi_{N,d}\colon Z_N^\times\to Z_d^\times$ is the natural projection, then $\chi_N = \chi\circ \pi_{N,d}$.

However, given positive $N,d$ with $d\mid N$ and $\chi$ a character modulo $N$, it is not always possible to find a character $\chi_d$ modulo $d$ such that $\chi = \chi_d\circ \pi_{N,d}$. But for every character modulo $N$ there exists a divisor $d$ of $N$ and a character $\chi_d$ modulo $d$ such that $\chi = \chi_d\circ \pi_{N,d}$. 
\begin{definition}
    The \textib{conductor} of a Dirichlet character $\chi$ modulo $N$ is the smallest positive divisor $d$ of $N$ such that there exists a Dirichlet character $\chi_d$ modulo $d$ such that $\chi = \chi_d\circ \pi_{N,d}$, equivalently, such that $\chi$ is trivial on the normal subgroup 
    \[\ker(\pi_{N,d}) = \{n\in Z_N^\times : n\equiv 1\smod d\}.\]

    A Dirichlet character modulo $N$ is \textib{primitive} if its conductor is $N$.
\end{definition}

Note that the only character modulo $N$ with conductor $1$ is the trivial character $\mathbf 1_N$, so that the trivial character $\mathbf 1_N$ is primitive only for $N = 1$.

Any Dirichlet character $\chi$ modulo $N$ extends to a function (abusing notation) $\chi\colon Z_N\to\mathbb C$ by the rule $\chi(n) = 0$ for noninvertible elements $n$ of the ring $Z_N$. Further extend $\chi$ to a function $\chi\colon\mathbb Z\to\mathbb C$ by the rule $\chi(n) = \chi(n\smod N)$. The resulting map $\chi\colon \mathbb Z\to \mathbb C$ is a totally multiplicative (set) function; that is, $\chi(nm) = \chi(n)\chi(m)$ for all $n,m\in \mathbb Z$.

For example, the extension of the trivial character $\mathbf 1_N$ to a function on $\mathbb Z$ is given by
\[\mathbf 1_N(n) = \begin{cases}
    1 & \text{if $\gcd(n,N)=1$},\\
    0 & \text{if $\gcd(n,N)\neq1$}.
\end{cases}\]
Also note that the extension of any Dirichlet character $\chi$ satisfies 
\[\chi(0) = \begin{cases}
    1 & \text{if $N = 1$},\\
    0 & \text{if $N > 1$}.
\end{cases}\]
Obtain new orthogonality relations from the ones appearing in \cref{eq: ortho rels} by summing from $0$ to $N-1$ in the first orthogonality relation and by taking $n\in\mathbb Z$ in the second:
\begin{equation}\label{eq: new ortho rels}
    \sum_{n=0}^{N-1}\chi(n) = \begin{cases}
        \phi(N) & \text{if $\chi = \mathbf 1$},\\
        0 & \text{if $\chi\neq \mathbf 1$},
    \end{cases}\quad\sum_{\chi\in \widehat{Z_N^\times}}\chi(n) = \begin{cases}
        \phi(N) & \text{if $n\equiv 1\smod N$},\\
        0 & \text{if $n\not\equiv 1\smod N$}.
    \end{cases} 
\end{equation}
\begin{definition}
    Let $\chi$ be a Dirichlet character modulo $N$. The \textib{Gauss sum} of $\chi$ is the complex number
    \[g(\chi) = \sum_{n=0}^{N-1}\chi(n)\mu_N^n,\] where $\mu_N = e^{2\pi i/N}$.
\end{definition}
\begin{proposition}
    If $\chi$ is a primitive character modulo $N$, then for any integer $m$ we have 
    \[g(\chi) = \sum_{n=0}^{N-1}\chi(n)\mu_N^{nm} = \overline\chi(m)g(\chi).\] Furthermore, the Gauss sum of a primitive character is nonzero.
\end{proposition}
\begin{proof}
    \tbd
\end{proof}
\begin{lemma}
    Let $N$ be a positive integer. If $N = 1,2$, then every Dirichlet character $\chi$ modulo $N$ satisfies $\chi(-1)=1$. If $N>2$, then the number of Dirichlet characters modulo $N$ is even, of which half satisfy $\chi(-1) = 1$ and the other half satisfy $\chi(-1) = -1$.
\end{lemma}
\begin{proof}
    \tbd
\end{proof}
Dirichlet characters are used to decompose the vector spaces $\mathcal M_k(\varGamma_1(N)), \mathcal S_k(\varGamma_1(N)), \mathcal E_k(\varGamma_1(N))$ into direct sums of interesting subspaces.
\begin{proposition}
    For each Dirichlet character $\chi$ modulo $N$, define the \textib{$\chi$-eigenspace} of $\mathcal M_k(\varGamma_1(N))$ by 
    \[\mathcal M_k(N,\chi) = \{f\in \mathcal M_k(\varGamma_1(N)) : f|_k[\gamma] = \chi(d_\gamma)f \text{ for all } \gamma\in\varGamma_0(N)\},\] where $d_\gamma$ denotes the lower right entry of $\gamma$. Then the following decomposition holds:
    \[\mathcal M_k(\varGamma_1(N)) = \bigoplus_\chi\mathcal M_k(N,\chi).\] With similar definitions of $\chi$-eigenspaces for $\mathcal S_k(\varGamma_1(N))$ and $\mathcal E_k(\varGamma_1(N))$, we obtain the decompositions
    \[\mathcal S_k(\varGamma_1(N)) = \bigoplus_\chi S_k(N,\chi),\quad \mathcal E_k(\varGamma_1(N)) = \bigoplus_\chi \mathcal E_k(N,\chi).\]
\end{proposition}
\begin{proof}
    \tbd also factor of automorphy for elements of quotient $\mathcal E_k(\varGamma_1(N))$??
\end{proof}

\begin{definition}
    To every Dirichlet character $\chi$ modulo $N$ there is an associated Dirichlet series, called a \textib{Dirichlet $L$-function}, given by 
    \[L(x,\chi) = \sum_{n=1}^\infty \frac{\chi(n)}{n^s} = \prod_{p\in\mathbb P}\frac{1}{1-\chi(p)p^{-s}},\quad \Re(s)>1.\] Here $\mathbb P$ denotes the set of prime numbers and the second equality above is obtained by using the fundamental theorem of arithmetic combined with the fact that $(1-\chi(p)p^{-s})^{-1}$ is given by a geometric series.
\end{definition}
It can be shown that these $L$-functions extend meromorphically to the complex plane in $s$, with entire extension unless $\chi = \mathbf 1_N$ (\tbd ex 4.4.4), which has a simple pole at $s = 1$. Furthermore, when $\chi(-1) = 1$, the functional equation satisfied by $L(s,\chi)$ is 
\[\pi^{-s/2}\varGamma\Big(\frac{s}{2}\Big)N^sL(s,\chi) = \pi^{-(1-s)/2}\varGamma\Big(\frac{1-s}{2}\Big)g(\chi)L(1-s,\chi),\] and when $\chi(-1)=-1$, the functional equation is 
\[\pi^{-(s+1)/2}\varGamma\Big(\frac{s+1}{2}\Big)N^sL(s,\chi) = -i\pi^{-(2-s)/2}\varGamma\Big(\frac{2-s}{2}\Big)g(\chi)L(1-s,\chi).\] Here $\varGamma$ denotes Euler's Gamma function (extended to $\mathbb C$).

To define $L$-functions for modular forms, 

\subsection{Eisenstein series}

\newpage\section{Hecke operators}
In this section we define Hecke operators on modular forms and show that they form an inner product space with the Petersson inner product, and investigate simultaneous eigenfunctions of Hecke operators \tbd
\subsection{Double coset operators}

\subsection{Hecke operators $\abr{n}$ and $T_n$}
\subsection{The Petersson inner product, adjoints of Hecke operators}
\subsection{Oldforms, newforms, and primitive forms}
We start with some results which are used to take forms from lower levels $M$ dividing $N$ to $N$.
\begin{lemma}
    If $M\mid N$ then $\mathcal S_k(\varGamma_1(M))\subset \mathcal S_k(\varGamma_1(N))$.
\end{lemma}
\begin{proof}
    \tbd
\end{proof}


\newpage\section{Strong Multiplicity One}
\subsection{Main result and relevant corollaries}
\subsection{applications!}

\newpage\section{unknown!}

\newpage\pagestyle{frontmatter}
\printindex\newpage
\begin{bibdiv}
\begin{biblist}

\bib{df}{book}{
    title = {Abstract Algebra},
    author = {Dummit, David S.},
    author = {Foote, Richard M.},
    isbn = {978-0-471-43334-7},
    year = {2004},
    publisher = {John Wiley \& Sons, Inc.}
}

\bib{diamond}{book}{
    title = {A First Course in Modular Forms},
    author = {Diamond, Fred},
    author = {Shurman, Jerry},
    isbn = {978-0-387-23229-4},
    series = {Graduate Texts in Mathematics},
    year = {2005},
    publisher = {Springer New York, NY}
}

\bib{miyake}{book}{
    title = {Modular Forms},
    author = {Miyake, Toshitsune},
    isbn = {978-3-540-29592-1},
    series = {Springer Monographs in Mathematics},
    year = {2005},
    publisher = {Springer Berlin, Heidelberg}
}

\bib{serre}{book}{
    title = {A Course in Arithmetic},
    author = {Serre, Jean-Pierre},
    isbn = {978-0-387-90040-7},
    series = {Graduate Texts in Mathematics},
    year = {1978},
    publisher = {Springer New York, NY}
}



\end{biblist}
\end{bibdiv}
\end{document}