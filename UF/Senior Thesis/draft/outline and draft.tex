\documentclass[10pt,leqno]{article}

% packages
\usepackage{physics}
% margin spacing
\usepackage[top=1in, bottom=1in, left=1in, right=1in]{geometry}
\usepackage{hanging}
\usepackage{amsfonts, amsmath, amssymb, amsthm}
\usepackage{fancyhdr}
\usepackage[nottoc, notlot, notlof]{tocbibind}
\usepackage{graphicx}
\graphicspath{{./images/}}
\usepackage{float}
\usepackage{enumitem}
\usepackage{quiver}
\usepackage{hyperref}
%\hypersetup{colorlinks=true,linkcolor=blue}
\usepackage[capitalize,noabbrev]{cleveref}

% % permutations (second line is for spacing)
% \usepackage{permute}
% \renewcommand*\pmtseparator{\,}

% % colors
% \usepackage{xcolor}
% \definecolor{p}{HTML}{FFDDDD}
% \definecolor{g}{HTML}{D9FFDF}
% \definecolor{y}{HTML}{FFFFCF}
% \definecolor{b}{HTML}{D9FFFF}
% \definecolor{o}{HTML}{FADECB}
% %\definecolor{}{HTML}{}

% % \highlight[<color>]{<stuff>}
% \newcommand{\highlight}[2][p]{\mathchoice%
%   {\colorbox{#1}{$\displaystyle#2$}}%
%   {\colorbox{#1}{$\textstyle#2$}}%
%   {\colorbox{#1}{$\scriptstyle#2$}}%
%   {\colorbox{#1}{$\scriptscriptstyle#2$}}}%

% header/footer formatting
\pagestyle{fancy}
\fancyhead{}
\fancyfoot{}
\fancyhead[L]{Strong Multiplicity One}
\fancyhead[C]{}
\fancyhead[R]{Sai Sivakumar}
\fancyfoot[R]{\thepage}
\renewcommand{\headrulewidth}{1pt}

% paragraph indentation/spacing
\setlength{\parindent}{0cm}
\setlength{\parskip}{6pt}
\renewcommand{\baselinestretch}{1.00}

% extra commands defined here
\newcommand{\br}[1]{\left(#1\right)}
\newcommand{\sbr}[1]{\left[#1\right]}
\newcommand{\cbr}[1]{\left\{#1\right\}}

% bracket notation for inner product
\usepackage{mathtools}

\DeclarePairedDelimiterX{\abr}[1]{\langle}{\rangle}{#1}

% new commands
\newcommand{\textib}[1]{\textbf{\textit{#1}}}
\DeclareMathOperator{\Mat}{M}
\DeclareMathOperator{\GL}{GL}
\DeclareMathOperator{\SL}{SL}
\newcommand{\smod}[1]{\;(\bmod\; #1)}
% \DeclareMathOperator{\Span}{span}
% \DeclareMathOperator{\nullity}{nullity}
% \DeclareMathOperator\Aut{Aut}
% \DeclareMathOperator\Inn{Inn}
% \DeclareMathOperator{\Orb}{Orb}
% \DeclareMathOperator{\lcm}{lcm}
% \DeclareMathOperator{\Hol}{Hol}
% \DeclareMathOperator{\Jac}{Jac}
% \DeclareMathOperator{\rad}{rad}
% \DeclareMathOperator{\Tor}{Tor}
% \DeclareMathOperator{\End}{End}
% \DeclareMathOperator{\Gal}{Gal}
% \DeclareMathOperator{\Nat}{Nat}
% \DeclareMathOperator{\Frac}{Frac}
% \DeclareMathOperator{\id}{id}
% \DeclareMathOperator{\im}{im}
% \DeclareMathOperator{\Hom}{Hom}
% \DeclareMathOperator{\Ext}{Ext}
% \DeclareMathOperator{\aug}{aug}

% set page count index to begin from 1
\setcounter{page}{1}

% toc header rename
\renewcommand{\contentsname}{Outline:}

% array column and row separation
\arraycolsep = 3pt
\renewcommand{\arraystretch}{.8}


\begin{document}
Prove Strong Multiplicity One as it appears in Miyake 4.6.19.
\tableofcontents
\newpage

\section{Introduction}

\newpage\section{Preliminaries} In this section we collect basic definitions and results used to define modular forms. %There would be minimal coverage of automorphy.
\subsection{The modular group and congruence subgroups}
Let $R$ be a unital ring. The \textib{general linear group} $\GL_n(R)$ is the group of $n\times n$ invertible matrices with entries from $R$; that is, matrices with unit determinant, under matrix multiplication. The \textib{special linear group} $\SL_n(R)$ is the subgroup of $\GL_n(R)$ whose matrices have determinant $1_R$.

The \textib{modular group} $\SL_2(\mathbb{Z})$ is the group of $2\times 2$ integer-valued matrices with determinant $1$, and is a subgroup of $\GL_2(\mathbb{R})$. Equivalently, the elements of $\SL_2(\mathbb{Z})$ are the integer-valued matrices $\begin{psmallmatrix}
    a & b \\ c & d
\end{psmallmatrix}$ with $\gcd(c,d) = 1$, since $ad-bc = 1$ is equivalent to $c$ and $d$ being coprime. It is well known (see \cref{text: serre} Chapter 7, Section 1, Theorems 1 and 2) that 
\begin{equation}\label{eq: generators modular group}
    \SL_2(\mathbb{Z}) = \left\langle\begin{pmatrix}
        1 & 1 \\ 0 & 1
    \end{pmatrix}, \begin{pmatrix}
        0 & -1 \\ 1 & 0
    \end{pmatrix}\right\rangle, 
\end{equation}
and that $\SL_2(\mathbb{Z})$ acts on the Riemann sphere $\widehat{\mathbb{C}} = \mathbb{C}\cup \{\infty\}$ by fractional linear transformations
\[\begin{pmatrix}
    a & b \\ c & d
\end{pmatrix}(z) = \frac{az + b}{cz + d}.\] These transformations are (bi)holomorphic and are automorphisms; for example, \[\begin{pmatrix}
    a & b \\ c & d
\end{pmatrix}^{-1}\begin{pmatrix}
    a & b \\ c & d
\end{pmatrix}(z) = \begin{pmatrix}
    d & -b \\ -c & a
\end{pmatrix}\bigg(\frac{az+b}{cz+d}\bigg) = \frac{\frac{d(az+b)}{cz+d}- b}{\frac{-c(az+b)}{cz+d}+a} = \frac{(ad-bc)z}{ad-bc} = z.\] For $c\neq 0$, $-d/c$ is sent to $\infty$ and $\infty$ is sent to $a/c$; if $c = 0$ then $\infty$ is sent to $\infty$. Both $I$ and $-I$ act as the identity on $\widehat{\mathbb{C}}$, so for any $A\in \SL_2(\mathbb{Z})$, the action of $A$ and $-A$ are the same. The generators in \cref{eq: generators modular group} correspond to the maps
\[z\mapsto z+1\quad\text{and}\quad z\mapsto -1/z.\]
We consider particular subgroups of the modular group. For $N$ a positive integer, the \textib{principal congruence subgroup of level $N$} is the subgroup
\[\varGamma(N) = \cbr{\begin{pmatrix}
    a & b \\ c & d
\end{pmatrix}\in \SL_2(\mathbb{Z}): \begin{pmatrix}
    a & b \\ c & d
\end{pmatrix}\equiv \begin{pmatrix}
    1 & 0 \\ 0 & 1
\end{pmatrix}\smod N}\]
with the matrix congruence interepreted as congruence modulo $N$ entrywise. Note $\varGamma(1) = \SL_2(\mathbb{Z})$, and that $\varGamma(N)$ is the kernel of the natural homomorphism $\SL_2(\mathbb{Z})\to \SL_2(\mathbb{Z}/N\mathbb{Z})$, so $\varGamma(N)$ is normal in $\SL_2(\mathbb{Z})$. The natural homomorphism is surjective.
\begin{proof}
    When $N = 1$, the natural homomorphism is the zero map. Let $N>1$, and consider an element \[\begin{pmatrix}
        \overline a & \overline b \\ \overline c & \overline d
    \end{pmatrix}\in \SL_2(\mathbb{Z}/N\mathbb{Z}),\] so that $\overline{ad-bc} = \overline 1$ (here $\overline{\,\cdot\,}$ denotes passage from $\mathbb{Z}$ to $\mathbb{Z}/N\mathbb{Z}$). Then for some integer $k$, $ad-bc = 1+kN$. It follows that $1+kN$ is a multiple of $g = \gcd(c,d)$, so that $\gcd(g,N)=1$ and $\gcd(c,d,N)=1$. We show that there exist integers $i,j$ such that $c+iN$ and $d+jN$ are coprime.
    
    Suppose there is no choice of $i,j$ such that $c+iN$ and $d+jN$ are coprime, so that for fixed $i$, $\gcd(c+iN,d+jN)>1$ for every $j$. By Dirichlet's theorem on arithmetic progressions, the sequence $(d+mN)$ contains infinitely many primes, which yields a contradiction.
    
    Alternatively, if $c\neq 0$, consider a solution $j$ to the system of congruences 
    \[\begin{cases}
        j\equiv 1\smod{p} & p\mid g\\
        j\equiv 0\smod{p} & p\nmid g, p\mid c,
    \end{cases}\] which may be obtained via the Chinese remainder theorem. Then $\gcd(c,d+jN) = 1$ since any prime $p$ dividing $c$ will not divide $d+jN$ with $j$ chosen as above [of primes $p$ dividing $c$, when $p\mid g$, still $p\nmid N$, and when $p\nmid g$, still $p\nmid d$]. If $c = 0$, then $d\neq 0$ and repeat this argument with $d,i$ in place of $c,j$ respectively.

    With integers $c+iN$ and $d+jN$ coprime, there exist integers $s,t$ with $s(c+iN) + t(d+jN) = 1$. Then \begin{multline*}
        \det\begin{pmatrix}
            a-(k+aj-bi)tN & b+(k+aj-bi)sN \\ c+iN & d+jN
        \end{pmatrix} = ad+ajN-(k+aj-bi)tN(d+jN)\\-bc-biN-(k+aj-bi)sN(c+iN) = 1+kN+ajN-biN-(k+aj-bi)N = 1
    \end{multline*} and 
    \[\begin{pmatrix}
        \overline{a-(k+aj-bi)tN} & \overline{b+(k+aj-bi)sN} \\ \overline{c+iN} & \overline{d+jN}
    \end{pmatrix} = \begin{pmatrix}
        \overline a & \overline b \\ \overline c & \overline d
    \end{pmatrix}\] as needed.
\end{proof}
Therefore, $\SL_2(\mathbb{Z})/\varGamma(N)\cong \SL_2(\mathbb{Z}/N\mathbb{Z})$. The index of $\varGamma(N)$ in $\SL_2(\mathbb{Z})$ is $[\SL_2(\mathbb{Z}): \varGamma(N)] = N^3\prod_{p\mid N}(1-1/p^2)$.
\begin{proof}
    We first show that $\abs{\SL_2(\mathbb{Z}/p^e\mathbb{Z})} = p^{3e}(1-1/p^2)$ for a prime $p$ by induction on $e$.

    The determinant is a surjective homomorphism from $\GL_2(\mathbb{Z}/p\mathbb{Z})$ to $(\mathbb{Z}/p\mathbb{Z})^\times\cong \mathbb{Z}/(p-1)\mathbb{Z}$ with kernel $\SL_2(\mathbb{Z}/p\mathbb{Z})$. Observe that $\abs{\GL_2(\mathbb{Z}/p\mathbb{Z})}$ is the number of ordered bases of $(\mathbb{Z}/p\mathbb{Z})^2$, given by $(p^2-1)(p^2-p)$. [The first factor counts the number of valid first basis vectors, and the second factor counts the number of valid second basis vectors after choosing a first basis vector.] Then by the first isomorphism theorem, $\abs{\SL_2(\mathbb{Z}/p\mathbb{Z})} = (p^2-1)(p^2-p)/(p-1) = p(p^2-1) = p^3(1-1/p^2)$.
    
    Suppose that $\abs{\SL_2(\mathbb{Z}/p^e\mathbb{Z})} = p^{3e}(1-1/p^2)$ for some $e$. From the previous result, we have that the natural homomorphisms $\SL_2(\mathbb{Z})\to \SL_2(\mathbb{Z}/p^{e+1}\mathbb{Z})$ and $\SL_2(\mathbb{Z})\to \SL_2(\mathbb{Z}/p^e\mathbb{Z})$ are surjective. Composing the map $\SL_2(\mathbb{Z})\to \SL_2(\mathbb{Z}/p^{e+1}\mathbb{Z})$ with the natural homomorphism $\SL_2(\mathbb{Z}/p^{e+1}\mathbb{Z})\to \SL_2(\mathbb{Z}/p^e\mathbb{Z})$ yields the commutative diagram \[\begin{tikzcd}
        {\SL_2(\mathbb{Z})} & {\SL_2(\mathbb{Z}/p^e\mathbb{Z})} \\
        {\SL_2(\mathbb{Z}/p^{e+1}\mathbb{Z})}
        \arrow[from=2-1, to=1-2]
        \arrow[two heads, from=1-1, to=2-1]
        \arrow[two heads, from=1-1, to=1-2]
    \end{tikzcd}\] from which we find that the map $\SL_2(\mathbb{Z}/p^{e+1}\mathbb{Z})\to \SL_2(\mathbb{Z}/p^e\mathbb{Z})$ is surjective.

    We deduce $\ker(\SL_2(\mathbb{Z}/p^{e+1}\mathbb{Z})\to \SL_2(\mathbb{Z}/p^e\mathbb{Z}))$ directly. If \[\begin{pmatrix}
        \overline a & \overline b \\ \overline c & \overline d
    \end{pmatrix}\mapsto \begin{pmatrix}
        \overline 1 & \overline 0 \\ \overline 0 & \overline 1
    \end{pmatrix},\] then all possible preimages of \[\begin{pmatrix}
        \overline 1 & \overline 0 \\ \overline 0 & \overline 1
    \end{pmatrix}\] are of the form \[\begin{pmatrix}
        \overline{1+ip^e} & \overline{rp^e} \\ \overline{sp^e} & \overline{1+jp^e}
    \end{pmatrix},\] for $i,j,r,s\in \{0,\dots,p-1\}$. Of these preimages, only $p^3$ many are elements of $\SL_2(\mathbb{Z}/p^{e+1}\mathbb{Z})$. [We may take any $r,s$ in the above range, but we require $i=-j$.] Then by the first isomorphism theorem, $\abs{\SL_2(\mathbb{Z}/p^{e+1}\mathbb{Z})} = p^3\abs{\SL_2(\mathbb{Z}/p^e\mathbb{Z})} = p^3p^{3e}(1-1/p^2) = p^{3(e+1)}(1-1/p^2)$ as needed. By induction, $\abs{\SL_2(\mathbb{Z}/p^e\mathbb{Z})} = p^{3e}(1-1/p^2)$ for any $e$.

    There is an isomorphism of matrix groups $\Mat_2(\prod_{i=1}^nR_i)\cong \prod_{i=1}^n\Mat_2(R_i)$ for rings $R_i$ which restricts to the isomorphisms $\GL_2(\prod_{i=1}^nR_i)\cong \prod_{i=1}^n\GL_2(R_i)$ and $\SL_2(\prod_{i=1}^nR_i)\cong \prod_{i=1}^n\SL_2(R_i)$ since $(\prod_{i=1}^nR_i)^\times\cong \prod_{i=1}^nR_i^\times$ and $1_{\prod_{i=1}^nR_i} = \prod_{i=1}^n1_{R_i}$.

    Let $N$ have prime factorization $N = \prod_{p\mid N} p^{e}$ so that by the Chinese remainder theorem, $\mathbb{Z}/N\mathbb{Z}\cong \prod_{p\mid N}\mathbb{Z}/p^{e}\mathbb{Z}$. It follows that $\SL_2(\mathbb{Z}/N\mathbb{Z})\cong \prod_{p\mid N}\SL_2(\mathbb{Z}/p^{e}\mathbb{Z})$, from which we have \[\abs{\SL_2(\mathbb{Z}/N\mathbb{Z})}= \prod_{p\mid N}\abs{\SL_2(\mathbb{Z}/p^{e}\mathbb{Z})} = \prod_{p\mid N}p^{3e}(1-1/p^2) = N^3\prod_{p\mid N}(1-1/p^2)\] as desired.
\end{proof}

A subgroup $\varGamma$ of $\SL_2(\mathbb{Z})$ is a \textib{congruence subgroup (of level $N$)} if for some positive integer $N$, $\varGamma(N)\subseteq \varGamma$. Any congruence subgroup $\varGamma$ has finite index in $\SL_2(\mathbb{Z})$. Let $\varGamma$ be a level $N$ congruence subgroup. Then 
\[\frac{\SL_2(\mathbb{Z})}{\varGamma}\cong \frac{\SL_2(\mathbb{Z})/\varGamma(N)}{\varGamma/\varGamma(N)},\] and since $\SL_2(\mathbb{Z})/\varGamma(N)$ is finite the result holds. [Indeed, $\varGamma/\varGamma(N)$ is a subgroup of $\SL_2(\mathbb{Z})/\varGamma(N)$; additionally observe that $\varGamma(N)$ has finite index in $\varGamma$ due to the inclusions $\varGamma(N)\subset \varGamma \subset \SL_2(\mathbb{Z})$ or also by the above argument.]

Frequently used congruence subgroups are 
\[\varGamma_0(N) = \cbr{\begin{pmatrix}
    a & b \\ c & d
\end{pmatrix}\in\SL_2(\mathbb{Z}): \begin{pmatrix}
    a & b \\ c & d
\end{pmatrix}\equiv\begin{pmatrix}
    \ast & \ast \\ 0 & \ast
\end{pmatrix}\smod N}\text{ and}\]
\[\varGamma_1(N) = \cbr{\begin{pmatrix}
    a & b \\ c & d
\end{pmatrix}\in \SL_2(\mathbb{Z}): \begin{pmatrix}
    a & b \\ c & d 
\end{pmatrix}\equiv \begin{pmatrix}
    1 & \ast \\ 0 & 1
\end{pmatrix}\smod N},\] where $\ast$ denotes unspecified quantities and the matrix congruences are to be taken entrywise. Furthermore, we have the inclusions $\varGamma(N)\subset\varGamma_1(N)\subset\varGamma_0(N)\subset\SL_2(\mathbb{Z})$.

The map $\varGamma_1(N)\to \mathbb{Z}/N\mathbb{Z}$ given by \[\begin{pmatrix}
    a & b \\ c & d
\end{pmatrix}\mapsto \overline b\] is surjective with kernel $\varGamma(N)$. 
\begin{proof}
    Let $\overline b\in\mathbb{Z}/N\mathbb{Z}$ be given. Let $a,d\equiv 1\smod N$ and $c\equiv 0\smod N$, and write $a= iN+1$, $d = jN+1$, and $c = kN$ for some integers $i,j,k$. If $1= ad-bc = ijN^2 + (i+j)N + 1 - bkN$, then $(ijN+ i + j)/k = b$. Take $k=1$ and reduce modulo $N$ to obtain the equation $i+j\equiv b \smod N$, which has many solutions, for example $i = b$ and $j =0$. These choices for $i,j,k$ produce a preimage of $b\smod N$ under the above map. Indeed, \[\begin{pmatrix}
        a & b \\ c & d
    \end{pmatrix} = \begin{pmatrix}
        bN+1 & b \\ N & 1
    \end{pmatrix}\mapsto b\quad\text{and}\quad \det\begin{pmatrix}
        bN+1 & b \\ N & 1
    \end{pmatrix} = 1.\]
    If $A\in \varGamma_1(N)$ and \[A = \begin{pmatrix}
        a & b \\ c & d
    \end{pmatrix}\mapsto \overline 0,\] then $b \equiv 0\smod N$ for some integer $k$. Since $a,d\equiv 1\smod N$ and $c\equiv 0\smod N$, it follows that $A\in \varGamma(N)$. Certainly any element of $\varGamma(N)$ is contained in the kernel of the above map.
\end{proof} 
It follows that $\varGamma_1(N)/\varGamma(N)\cong \mathbb{Z}/N\mathbb{Z}$, so that the index of $\varGamma(N)$ in $\varGamma_1(N)$ is $N$.

Similarly, the map $\varGamma_0(N)\to (\mathbb{Z}/N\mathbb{Z})^\times$ defined by \[\begin{pmatrix}
    a & b \\ c & d
\end{pmatrix}\mapsto \overline d\] 
is surjective with kernel $\varGamma_1(N)$.
\begin{proof}
    Let $\overline d \in (\mathbb{Z}/N\mathbb{Z})^\times$ be given, and note that $\gcd(d,N)=1$. Let $c \equiv 0\smod N$ so that $c=kN$ for some integer $k$. Choose $k$ coprime to $d$ so that $\gcd(d,kN) = \gcd(d,c) = 1$. Then there exist integers $a,b$ with $ad-bc = 1$. Then \[\begin{pmatrix}
        a & b \\ c & d
    \end{pmatrix}\mapsto d\quad\text{and}\quad \det\begin{pmatrix}
        a & b \\ c & d
    \end{pmatrix} = 1.\] 
    If $A\in \varGamma_0(N)$ and \[A = \begin{pmatrix}
        a & b \\ c & d
    \end{pmatrix}\mapsto \overline 1,\] so $d = jN+1$ for some integer $j$. Since $c\equiv 0\smod N$, $c= kN$ and $\gcd(d,c) = 1$. There exist integers $a,b$ with $1 = ad-bc = a(jN+1) + bkN = a+ (aj + bk)N$. Reducing modulo $N$ gives $a\equiv 1\smod N$. It follows that $A\in \varGamma_1(N)$. Conversely, if $A\in \varGamma_1(N)$, then $A\mapsto \overline 1$.
\end{proof}
Hence $\varGamma_0(N)/\varGamma_1(N)\cong (\mathbb{Z}/N\mathbb{Z})^\times$, so that the index of $\varGamma_1(N)$ in $\varGamma_0(N)$ is $\phi(N) = \abs{(\mathbb{Z}/N\mathbb{Z})^\times} = N\prod_{p\mid N}(1-1/p)$, where $\phi$ denotes the Euler totient function. Finally, we obtain \[[\SL_2(\mathbb{Z}) : \varGamma_0(N)] = \frac{[\SL_2(\mathbb{Z}) : \varGamma(N)]}{[\varGamma_0(N) : \varGamma_1(N)][\varGamma_1(N) : \varGamma(N)]} = \frac{N^3\prod_{p\mid N}(1-1/p^2)}{N\cdot N\prod_{p\mid N}(1-1/p)} = N\prod_{p\mid N}(1+1/p).\]

\subsection{Modular curves}
Recall that $\SL_2(\mathbb{Z})$ acts on the Riemann sphere $\widehat{\mathbb{C}}$ by fractional linear transformations \[\begin{pmatrix}
    a & b \\ c & d
\end{pmatrix}(z) = \frac{az + b}{cz+d}.\] The \textib{upper half plane} $\mathcal{H} = \cbr{z\in\mathbb{C} : \Im(z)}$ inherits this action. If $\gamma\in \SL_2(\mathbb{Z})$ with $\gamma = \begin{psmallmatrix}
    a & b \\ c & d
\end{psmallmatrix}$ and $z\in \mathbb C$, the computation $(az+b)(c\overline z + d) = ac\abs{z}^2+bd + adz + bc\overline z = ac\abs{z}^2+bd + adz + (ad-1)\overline z = ac\abs{z}^2+bd + ad(z+\overline z) -\overline z$ shows that
\begin{equation}\label{eq: imaginary parts}
    \Im(\gamma(z)) = \Im\bigg(\frac{az+b}{cz+d}\bigg) = \frac{\Im((az+b)(c\overline z + d))}{\abs{cz+d}^2} = \frac{\Im(-\overline z)}{\abs{cz+d}^2} = \frac{\Im(z)}{\abs{cz+d}^2}.
\end{equation}
It follows that if $z\in \mathcal{H}$, then $\gamma(z)\in \mathcal{H}$ also.

Since $\SL_2(\mathbb{Z})$ acts on the upper half plane, so does any congruence subgroup $\varGamma$. The \textib{modular curve} $Y(\varGamma)$ is the orbit space $\varGamma\backslash \mathcal{H} = \cbr{\varGamma z : z\in\mathcal H}$. The modular curves for the congruence subgroups $\varGamma_0(N), \varGamma_1(N)$, and $\varGamma(N)$ are denoted by $Y_0(N), Y_1(N)$, and $Y(N)$, respectively.

We topologize modular curves and briefly show they are compact Riemann surfaces. Let $\varGamma$ be a congruence subgroup. The upper half plane $\mathcal H$ is given the Euclidean topology (as a subspace of $\mathbb{C}$ or of $\mathbb{R}^2$), and the natural surjection \[\pi \colon \mathcal H \to Y(\varGamma)\quad \text{defined by}\quad  z\mapsto \varGamma z\] induces the quotient topology on $Y(\varGamma)$ (i.e., the open sets in $Y(\varGamma)$ are those with open preimages under $\pi$). In fact, $\pi$ is an open mapping. It suffices to show that the projection $\pi$ takes an open ball $B$ in $\mathcal{H}$ to an open set in $Y(\varGamma)$; that is, to show that \[\bigcup_{\gamma\in \varGamma}\gamma(B)\] is open. By the Open Mapping Theorem, each $\gamma(B)$ is open (or since $\gamma^{-1}$ is continuous), so the result holds. Note that since $\pi$ is open, $Y(\varGamma)$ is second-countable.

Furthermore, $\pi(U_1)\cap \pi(U_2)=\emptyset$ in $Y(\varGamma)$ is equivalent to $\big(\bigcup_{\gamma\in \varGamma}\gamma(U_1)\big)\cap U_2 = \emptyset$ in $\mathcal H$. The equality
\begin{multline}\label{eq: disjoint nbds in modular curve}
    \emptyset = \pi^{-1}(\pi(U_1)\cap \pi(U_2)) = \Bigg(\bigcup_{\gamma\in \varGamma}\gamma(U_1)\Bigg)\cap \Bigg(\bigcup_{\gamma\in \varGamma}\gamma(U_2)\Bigg) = \bigcup_{\gamma,\eta\in \varGamma}\gamma(U_1)\cap \eta(U_2)\\ = \bigcup_{\eta\in \varGamma}\eta\Bigg[\Bigg(\bigcup_{\gamma\in \varGamma}(\eta^{-1}\gamma)(U_1)\Bigg)\cap U_2\Bigg] = \bigcup_{\eta\in \varGamma}\eta\Bigg[\Bigg(\bigcup_{\gamma^\prime\in \varGamma}\gamma^\prime(U_1)\Bigg)\cap U_2\Bigg]
\end{multline}
implies that $\big(\bigcup_{\gamma^\prime\in \varGamma}\gamma^\prime(U_1)\big)\cap U_2$ must be empty, from which the above equivalence follows.
As a remark, $\mathcal H$ is connected, so $Y(\varGamma)$ is connected as well.

We show that $Y(\varGamma)$ is Hausdorff. Observe that the action of $\SL_2(\mathbb{Z})$ (hence also of $\varGamma$) on $\mathcal H$ is properly discontinuous. Given $z_1,z_2\in \mathcal H$, (including the case $z_1 = z_2$) there exist neighborhoods $U_1$ of $z_1$ and $U_2$ of $z_2$ in $\mathcal H$ such that for any $\gamma\in \SL_2(\mathbb{Z})$, if $\gamma(U_1)\cap U_2\neq \emptyset$, then $\gamma(z_1) = z_2$. Equivalently, for any $\gamma\in\SL_2(\mathbb{Z})$, if $\gamma(z_1)\neq z_2$, then $\gamma(U_1)\cap U_2= \emptyset$.
\begin{proof}
    Take $V_1$ and $V_2$ to be neighborhoods of $z_1$ and $z_2$, respectively, with compact closure (e.g. open balls or bounded open sets). Let $y_1 = \sup_{z\in V_1}\cbr{\Im(z)}$, $y_2 = \inf_{z\in V_2}\cbr{\Im(z)}$, and $Y_1 = \sup_{z\in V_2}\cbr{\Im(z)}$. Then from \cref{eq: imaginary parts}, we have for $\gamma = \begin{psmallmatrix}
        a & b \\ c & d
    \end{psmallmatrix}\in \SL_2(\mathbb{Z})$ (i.e., $\gcd(c,d) = 1$) and $z\in V_1$ that \[\Im(\gamma(z)) = \frac{\Im(z)}{\abs{cz+d}^2}\leq \frac{\Im(z)}{c^2\abs{z}^2}\leq \frac{1}{c^2\Im(z)}\leq \frac{1}{c^2y_1}\] and \[\Im(\gamma(z)) = \frac{\Im(z)}{\abs{cz+d}^2}\leq \frac{Y_1}{\Re(cz+d)^2} = \frac{Y_1}{(c\Re(z)+d)^2},\] so that $\Im(\gamma(z))\leq\min\cbr{1/c^2y_1,Y_1/(c\Re(z)+d)^2}$. All but finitely many integers $c$ may be chosen so that $1/c^2y_1< y_2$, and of the finitely many $c$ where the inequality does not hold, we may choose all but finitely many $d$ such that $Y_1/(c\Re(z)+d)^2< y_1$ uniformly in $z$ (since $V_1$ has compact closure). Thus for all but finitely many coprime integers $c,d$ we have \begin{equation}\label{eq: sup inf}
        \sup_{\substack{\gamma  = \begin{psmallmatrix}
            a & b \\ c & d
        \end{psmallmatrix}\in \SL_2(\mathbb{Z})\\z\in V_1}}\cbr{\Im(\gamma(z))}< \inf_{z\in V_2}\cbr{\Im(z)}.
    \end{equation} Thus for $\gamma = \begin{psmallmatrix}
        a & b \\ c & d
    \end{psmallmatrix}\in \SL_2(\mathbb{Z})$ satisfying \cref{eq: sup inf}, $\gamma(V_1)\cap V_2=\emptyset$. We show that there are only finitely many $\gamma\in \SL_2(\mathbb{Z})$ for which $\gamma(V_1)\cap V_2\neq\emptyset$.

    We determine the matrices in $\SL_2(\mathbb{Z})$ which have fixed bottom row $(c,d)$ for fixed coprime $c,d$. This is equivalent to finding all matrices $\eta = \begin{psmallmatrix}
        A & B \\ C & D
    \end{psmallmatrix}\in \SL_2(\mathbb{Z})$ such that $\eta\gamma$ has bottom row $(c,d)$, where $\gamma\in \SL_2(\mathbb{Z})$ is a fixed matrix with  bottom row $(c,d)$. Evidently $C=0$ and $D = 1$, and since $1 = AD - BC = A$, we have that $\eta$ must have the form $\begin{psmallmatrix}
        1 & B \\ 0  & 1
    \end{psmallmatrix}$, and $B$ may be taken to be any integer. [As expected, $\det(\eta\gamma) = \det(\eta)\det(\gamma) = 1$.] Explicitly, the matrices in $\SL_2(\mathbb{Z})$ with bottom row $(c,d)$ are \[\cbr{\begin{pmatrix}
        1 & k \\ 0 & 1
    \end{pmatrix}\begin{pmatrix}
        a & b \\ c & d
    \end{pmatrix}: k\in \mathbb{Z}},\]
    where $(a,b)$ is one such pair such that $ad-bc = 1$. Thus for $\gamma\in \cbr{\begin{psmallmatrix}
        1 & k \\ 0 & 1
    \end{psmallmatrix}\begin{psmallmatrix}
        a & b \\ c & d
    \end{psmallmatrix}: k\in \mathbb{Z}}$, the intersection $\gamma(V_1)\cap V_2 = \big(\begin{psmallmatrix}
        a & b \\ c & d
    \end{psmallmatrix}V_1+ k\big)\cap V_2 =\emptyset$ for all but finitely many $k$; that is, for finitely many $\gamma$. So of the finitely many pairs of coprime integers $(c,d)$ for which \cref{eq: sup inf} does not hold, only finitely many matrices $\gamma\in \SL_2(\mathbb{Z})$ with bottom row $(c,d)$ exist with $\gamma(V_1)\cap V_2\neq \emptyset$.

    Let $F = \cbr{\gamma\in \SL_2(\mathbb{Z}): \gamma(V_1)\cap V_2\neq \emptyset, \gamma(z_1)\neq z_2}$, which is finite by the above argument. For each $\gamma \in F$ there exist disjoint neighborhoods $V_{1,\gamma}$ of $\gamma(z_1)$ and $V_{2,\gamma}$ of $z_2$ in $\mathcal H$. Let \[U_1 = V_1\cap \Bigg(\bigcap_{\gamma\in F}\gamma^{-1}(V_{1,\gamma})\Bigg)\quad\text{and}\quad U_2 = V_2\cap \Bigg(\bigcap_{\gamma\in F}V_{2,\gamma}\Bigg),\] and note that $U_1,U_2$ are open, as elements of $\SL_2(\mathbb{Z})$ are open maps on $\mathcal H$. Then take any $\gamma\in \SL_2(\mathbb{Z})$ with $\gamma(U_1)\cap U_2\neq \emptyset$. If $\gamma\not\in F$, then we must have $\gamma(z_1) = z_2$. Suppose that $\gamma\in F$. Then $U_1\subset \gamma^{-1}(U_{1,\gamma})$ and $U_2\subset U_{2,\gamma}$ so that $ \gamma(U_1)\cap U_2\subset U_{1,\gamma}\cap U_{2,\gamma}$. But $\gamma(U_1)\cap U_2\neq \emptyset$, which is in contradiction with $U_{1,\gamma}$ and $U_{2,\gamma}$ chosen to be disjoint. Hence $\gamma\not\in F$, so that $\gamma(z_1)= z_2$.
\end{proof}

A corollary of this result is that for any congruence subgroup $\varGamma$ of $\SL_2(\mathbb{Z})$, the modular curve $Y(\varGamma)$ is Hausdorff. 
\begin{proof}
    Let $\pi(z_1)$ and $\pi(z_2)$ be distinct points in $Y(\varGamma)$ (so that $\gamma(z_1)\neq z_2$ for any $\gamma\in \varGamma$), and take neighborhoods $U_1$ of $z_1$ and $U_2$ of $z_2$ such that for any $\gamma\in \SL_2(\mathbb{Z})$, if $\gamma(U_1)\cap U_2\neq \emptyset$, then $\gamma(z_1) = z_2$, as per the previous result. Then $\big(\bigcap_{\gamma\in\varGamma}\gamma(U_1)\big)\cap U_2 = \emptyset$ since $\gamma(z_1)\neq z_2$ for every $\gamma\in \varGamma$. From the discussion surrounding \cref{eq: disjoint nbds in modular curve}, we have that $\pi(U_1)\cap\pi(U_2) = \emptyset$ as needed, with $\pi(U_1),\pi(U_2)$ open since $\pi$ is an open mapping.
\end{proof} 

What remains is to compactify $Y(\varGamma)$ and to put charts on the resulting compact space.


\subsection{Elliptic points and cusps}

\newpage\section{Modular forms}
\subsection{Basic definitions}
For an integer $k$ and a congruence subgroup $\varGamma$, a meromorphic function $f\colon \mathcal H\to \mathbb{C}$ is \textib{weakly modular of weight $k$ (with respect to $\varGamma$)} if \[f(\gamma(z)) = (cz+d)^kf(z)\quad\text{for any}\quad \gamma = \begin{pmatrix}
    a & b \\ c & d
\end{pmatrix}\in \varGamma, z\in \mathcal H.\]
For any matrix $\gamma=\begin{psmallmatrix}
    a & b \\ c & d
\end{psmallmatrix}\in \SL_2(\mathbb{Z})$ the \textib{factor of automorphy} $j(\gamma,z)$ for $z\in \mathbb{C}$ is defined by \[j(\gamma,z) = cz+d.\] For any integer $k$ and $\gamma\in \SL_2(\mathbb{Z})$ define the \textib{weight-k operator} $[\gamma]_k$ on functions $f\colon \mathcal H\to \mathbb{C}$ by \[(f[\gamma]_k)(z) = j(\gamma,z)^{-k}f(\gamma(z)),\quad\text{for } z\in \mathcal H.\]
Since the factor of automorphy $j(\gamma,z) = cz+d$ cannot be zero or infinity (since $z\in \mathcal H$), if $f$ is meromorphic on $\mathcal H$, then so is $f[\gamma]_k$. Furthermore, the poles and zeroes of $f[\gamma]_k$ are the same as those of $f$ since if $z-z_0 = 0$, then
\[\gamma(z)-z_0 = \frac{az+b}{cz+d}-z_0 = \frac{ac}{}\]
So for an integer $k$ and a congruence subgroup $\varGamma$, a meromorphic function $f\colon \mathcal H \to \mathbb{C}$ is weakly modular of weight $k$ (with respect to $\varGamma$) if \[f[\gamma]_k = f\quad\text{for all }\gamma\in \varGamma,\] and this is equivalent to the original definition above.
\subsection{Dimension formulas}
\subsection{Eisenstein series}
\subsection{Dirichlet characters and $L$-functions}

\newpage\section{Hecke operators}
\subsection{Double coset operators}
\subsection{Hecke operators $\abr{n}$ and $T_n$}
\subsection{The Petersson inner product, adjoints of Hecke operators}
\subsection{Oldforms, newforms, and primitive forms}

\newpage\section{Strong Multiplicity One}
\subsection{Main result and relevant corollaries}
\subsection{applications!}

\newpage\section{unknown!}

\end{document}