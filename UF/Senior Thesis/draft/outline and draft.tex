\documentclass[11pt,leqno]{article}

% packages
\usepackage{physics}
% margin spacing
\usepackage[top=1in, bottom=1in, left=1in, right=1in]{geometry}
\usepackage{hanging}
\usepackage{amsfonts, amsmath, amssymb, amsthm}
\usepackage{fancyhdr}
\usepackage[nottoc, notlot, notlof]{tocbibind}
\usepackage{graphicx}
\graphicspath{{./images/}}
\usepackage{float}
\usepackage{enumitem}
\usepackage{quiver}
\usepackage{hyperref}
%\hypersetup{colorlinks=true,linkcolor=blue}
\usepackage[capitalize,noabbrev]{cleveref}

% % permutations (second line is for spacing)
% \usepackage{permute}
% \renewcommand*\pmtseparator{\,}

% % colors
% \usepackage{xcolor}
% \definecolor{p}{HTML}{FFDDDD}
% \definecolor{g}{HTML}{D9FFDF}
% \definecolor{y}{HTML}{FFFFCF}
% \definecolor{b}{HTML}{D9FFFF}
% \definecolor{o}{HTML}{FADECB}
% %\definecolor{}{HTML}{}

% % \highlight[<color>]{<stuff>}
% \newcommand{\highlight}[2][p]{\mathchoice%
%   {\colorbox{#1}{$\displaystyle#2$}}%
%   {\colorbox{#1}{$\textstyle#2$}}%
%   {\colorbox{#1}{$\scriptstyle#2$}}%
%   {\colorbox{#1}{$\scriptscriptstyle#2$}}}%

% header/footer formatting
\pagestyle{fancy}
\fancyhead{}
\fancyfoot{}
\fancyhead[L]{Strong Multiplicity One}
\fancyhead[C]{}
\fancyhead[R]{Sai Sivakumar}
\fancyfoot[R]{\thepage}
\renewcommand{\headrulewidth}{1pt}

% paragraph indentation/spacing
\setlength{\parindent}{0cm}
\setlength{\parskip}{10pt}
\renewcommand{\baselinestretch}{1.25}

% extra commands defined here
\newcommand{\br}[1]{\left(#1\right)}
\newcommand{\sbr}[1]{\left[#1\right]}
\newcommand{\cbr}[1]{\left\{#1\right\}}

% bracket notation for inner product
\usepackage{mathtools}

\DeclarePairedDelimiterX{\abr}[1]{\langle}{\rangle}{#1}

% new commands
\newcommand{\textib}[1]{\textbf{\textit{#1}}}
\DeclareMathOperator{\GL}{GL}
\DeclareMathOperator{\SL}{SL}
\newcommand{\smod}[1]{\;(\bmod\; #1)}
% \DeclareMathOperator{\Span}{span}
% \DeclareMathOperator{\nullity}{nullity}
% \DeclareMathOperator\Aut{Aut}
% \DeclareMathOperator\Inn{Inn}
% \DeclareMathOperator{\Orb}{Orb}
% \DeclareMathOperator{\lcm}{lcm}
% \DeclareMathOperator{\Hol}{Hol}
% \DeclareMathOperator{\Jac}{Jac}
% \DeclareMathOperator{\rad}{rad}
% \DeclareMathOperator{\Tor}{Tor}
% \DeclareMathOperator{\End}{End}
% \DeclareMathOperator{\Gal}{Gal}
% \DeclareMathOperator{\Nat}{Nat}
% \DeclareMathOperator{\Frac}{Frac}
% \DeclareMathOperator{\id}{id}
% \DeclareMathOperator{\im}{im}
% \DeclareMathOperator{\Hom}{Hom}
% \DeclareMathOperator{\Ext}{Ext}
% \DeclareMathOperator{\aug}{aug}

% set page count index to begin from 1
\setcounter{page}{1}

% toc header rename
\renewcommand{\contentsname}{Outline:}

\begin{document}
Prove Strong Multiplicity One as it appears in Miyake 4.6.19.
\tableofcontents
\newpage

\section{Introduction}

\newpage\section{Preliminaries} In this section we collect basic definitions and results used to define modular forms, without many proofs? There would be minimal coverage of automorphy.
\subsection{The modular group and congruence subgroups}
Let $R$ be a unital ring. The \textib{general linear group} $\GL_n(R)$ is the group of $n\times n$ invertible matrices with entries from $R$; that is, matrices with unit determinant, under matrix multiplication. The \textib{special linear group} $\SL_n(R)$ is the subgroup of $\GL_n(R)$ whose matrices have determinant $1_R$.

The \textib{modular group} $\SL_2(\mathbb{Z})$ is the group of $2\times 2$ integer-valued matrices with determinant $1$, and is a subgroup of $\GL_2(\mathbb{R})$. It is well known that 
\begin{equation}\label{eq: generators modular group}
    \SL_2(\mathbb{Z}) = \left\langle\begin{pmatrix}
        1 & 1 \\ 0 & 1
    \end{pmatrix}, \begin{pmatrix}
        0 & -1 \\ 1 & 0
    \end{pmatrix}\right\rangle, 
\end{equation}
and that $\SL_2(\mathbb{Z})$ acts on the Riemann sphere $\widehat{\mathbb{C}} = \mathbb{C}\cup \{\infty\}$ by fractional linear transformations
\[\begin{pmatrix}
    a & b \\ c & d
\end{pmatrix}(z) = \frac{az + b}{cz + d}.\] In this way it is also clear that these fractional linear transformations are automorphisms, for example \[\begin{pmatrix}
    a & b \\ c & d
\end{pmatrix}^{-1}\begin{pmatrix}
    a & b \\ c & d
\end{pmatrix}(z) = \begin{pmatrix}
    d & -b \\ -c & a
\end{pmatrix}\bigg(\frac{az+b}{cz+d}\bigg) = \frac{\frac{d(az+b)}{cz+d}- b}{\frac{-c(az+b)}{cz+d}+a} = \frac{(ad-bc)z}{ad-bc} = z.\] For $c\neq 0$, $-d/c$ is sent to $\infty$ and $\infty$ is sent to $a/c$; if $c = 0$ then $\infty$ is sent to $\infty$. Both $I$ and $-I$ act as the identity on $\widehat{\mathbb{C}}$, so for any $A\in \SL_2(\mathbb{Z})$, the action of $A$ and $-A$ are the same. The generators in \cref{eq: generators modular group} correspond to the maps
\[z\mapsto z+1\quad\text{and}\quad z\mapsto -1/z.\]
We consider particular subgroups of the modular group. For $N$ a positive integer, the \textib{principal congruence subgroup of level $N$} is the subgroup
\[\Gamma(N) = \cbr{\begin{pmatrix}
    a & b \\ c & d
\end{pmatrix}\in \SL_2(\mathbb{Z})\colon \begin{pmatrix}
    a & b \\ c & d
\end{pmatrix}\equiv \begin{pmatrix}
    1 & 0 \\ 0 & 1
\end{pmatrix}\smod N}\]
with the matrix congruence interepreted as congruence modulo $N$ entrywise. Note $\Gamma(1) = \SL_2(\mathbb{Z})$, and that $\Gamma(N)$ is the kernel of the natural homomorphism $\SL_2(\mathbb{Z})\to \SL_2(\mathbb{Z}/N\mathbb{Z})$, so $\Gamma(N)$ is normal in $\SL_2(\mathbb{Z})$. This natural homomorphism is also surjective since a matrix \[\overline \gamma = \begin{pmatrix}
    \overline a & \overline b \\ \overline c & \overline d
\end{pmatrix}\] with determinant $\overline 1$ has preimage \[\gamma = \begin{pmatrix}
    a & b \\ c & d
\end{pmatrix}\] under the natural homomorphism, and it is a routine computation to see that $\det\gamma =1$. 
\subsection{Modular curves}
\subsection{Elliptic points and cusps}

\newpage\section{Modular forms}
\subsection{Basic definitions}
\subsection{Dimension formulas}
\subsection{Eisenstein series}
\subsection{Dirichlet characters and $L$-functions}

\newpage\section{Hecke operators}
\subsection{Double coset operators}
\subsection{Hecke operators $\abr{n}$ and $T_n$}
\subsection{The Petersson inner product, adjoints of Hecke operators}
\subsection{Oldforms, newforms, and primitive forms}

\newpage\section{Strong Multiplicity One}
\subsection{Main result and relevant corollaries}
\subsection{applications!}

\newpage\section{unknown!}

\end{document}