\documentclass[11pt]{article}
\headheight = 14pt
% packages
\usepackage{physics}
% margin spacing
\usepackage[top=1in, bottom=1in, left=0.5in, right=0.5in]{geometry}
\usepackage{hanging}
\usepackage{amsfonts, amsmath, amssymb, amsthm}
\usepackage{systeme}
\usepackage[none]{hyphenat}
\usepackage{fancyhdr}
\usepackage[nottoc, notlot, notlof]{tocbibind}
\usepackage{graphicx}
\graphicspath{{./images/}}
\usepackage{float}
\usepackage{siunitx}
\usepackage{esint}
\usepackage{cancel}
\usepackage{enumitem}
\usepackage{tikz-cd}
\usepackage{soul}

% colors
\usepackage{xcolor}
\definecolor{p}{HTML}{FFDDDD}
\definecolor{g}{HTML}{D9FFDF}
\definecolor{y}{HTML}{FFFFCF}
\definecolor{b}{HTML}{D9FFFF}
\definecolor{o}{HTML}{FADECB}
%\definecolor{}{HTML}{}

% \highlight[<color>]{<stuff>}
\newcommand{\highlight}[2][p]{\mathchoice%
  {\colorbox{#1}{$\displaystyle#2$}}%
  {\colorbox{#1}{$\textstyle#2$}}%
  {\colorbox{#1}{$\scriptstyle#2$}}%
  {\colorbox{#1}{$\scriptscriptstyle#2$}}}%

% header/footer formatting
\pagestyle{fancy}
\fancyhead{}
\fancyfoot{}
\fancyhead[L]{MTG4303}
\fancyhead[C]{HW 5}
\fancyhead[R]{Sai Sivakumar}
\fancyfoot[R]{\thepage}
\renewcommand{\headrulewidth}{1pt}

% paragraph indentation/spacing
\setlength{\parindent}{0cm}
\setlength{\parskip}{5pt}
\renewcommand{\baselinestretch}{1.25}

% extra commands defined here
\newcommand{\br}[1]{\left(#1\right)}
\newcommand{\sbr}[1]{\left[#1\right]}
\newcommand{\cbr}[1]{\left\{#1\right\}}

% bracket notation for inner product
\usepackage{mathtools}

\DeclarePairedDelimiterX{\abr}[1]{\langle}{\rangle}{#1}

\DeclareMathOperator{\Span}{span}
\DeclareMathOperator{\card}{card}
\DeclareMathOperator{\Int}{Int}
\DeclareMathOperator{\Bd}{Bd}
\DeclareMathOperator{\id}{id}

\newtheorem*{definition}{Definition}

% set page count index to begin from 1
\setcounter{page}{1}

\begin{document}
\begin{enumerate}
    \item (36.1) Prove that every manifold is regular and hence metrizable. Where do you use the Hausdorff condition? \begin{proof}
        Let $M$ be any $m$-manifold, with $m$ arbitrary. Recall that any topological space satisfying the $T_1$ axiom is regular if and only if for every point $x$ in the space and a neighborhood $U$ containing the point, we can find a neighborhood $V$ containing $x$ such that $\overline{V}\subset U$.

        Let $x\in M$ and $U$ a neighborhood containing $x$ be given. Since $M$ is a manifold there is a neighborhood $E$ containing $x$ such that $E$ is homeomorphic to an open subset of $\mathbb{R}^m$; let $\varphi$ be one such homeomorphism.

        Observe that $E\cap U$ is an open subset of $E$ containing $x$ which is mapped homeomorphically into an open neighborhood of $\varphi(x)$ in $\mathbb{R}^m$. Since $\mathbb{R}^m$ is regular (it is metrizable), we can find a neighborhood $V^\prime$ of $\varphi(x)$ such that $\overline{V^\prime}$ is contained in $\varphi(E\cap U)$. We show that by choosing $V = \varphi^{-1}(V^\prime)$ which contains $x$, then $\overline{V}$ is contained in $U$.

        We have that $\varphi^{-1}(\overline{V^\prime})\subset \overline{\varphi^{-1}(V^\prime)}$ by continuity of $\varphi$. We show the reverse containment by way of contradiction. Suppose that there is a point $p \in \overline{\varphi^{-1}(V^\prime)}$ not contained in $\varphi^{-1}(\overline{V^\prime})$. Since manifolds have countable basis we can find a sequence of points $(\varphi^{-1}(x_n))$ in $M$ which converges to $p$. But since $\varphi^{-1}$ is continuous, the sequence $(x_n)$ converges to a limit point $q\in \overline{V^\prime}$, by continuity we have that $(\varphi^{-1}(x_n))$ converges to $\varphi^{-1}(q)$. But $p$ could not equal $\varphi^{-1}(q)$ by assumption, which is in contradiction to the fact that in Hausdorff spaces limits are unique (this is why $M$ needs to be Hausdorff specifically).

        Hence $\varphi^{-1}(\overline{V^\prime}) = \overline{\varphi^{-1}(V^\prime)}$, so that by choosing $V = \varphi^{-1}(V^\prime)$, we have the desired containment. Observe that since $\overline{V^\prime}$ is contained in $\varphi(E\cap U)$, by continuity we have that $\varphi^{-1}(\overline{V^\prime}) = \overline{\varphi^{-1}(V^\prime)} = \overline{V}\subset E\cap U$. Hence $M$ is regular, and since $M$ was an arbitrary manifold, every manifold is regular.

        Since manifolds are regular and have countable basis, by the Urysohn metrization theorem, manifolds are metrizable.
    \end{proof}
    \item (36.3) Let $X$ be a Hausdorff space such that each point of $X$ has a neighborhood that is homeomorphic with an open subset of $\mathbb{R}^m$. Show that if $X$ is compact, then $X$ is an $m$-manifold. \begin{proof}
        We show that $X$ has a countable basis. Let $\mathcal{A} = \cbr{A_x\mid x\in X}$ where $A_x$ is a neighborhood of $x$ homeomorphic to an open neighborhood of $\mathbb{R}^m$. We have that $\mathcal{A}$ is an open cover of $X$, so let $\mathcal{A}^\prime = \cbr{A_1,\dots,A_n}$ be the resulting finite subcover of $X$ coming from compactness. Then view $\mathcal{A}^\prime$ as a subbase for $X$ since it covers $X$. Viewing $X$ under this topology generated by the subbase, it has a countable basis since every basis element is formed by taking finite intersections of elements of $\mathcal{A}^\prime$; since $\mathcal{A}^\prime$ has finitely many sets, it follows that there are countably many basis elements. It follows that $X$ is an $m$-manifold since all of the othe requirements are met.
    \end{proof}
    \item (36.5) The Hausdorff condition is an essential part of the definition of a manifold; it is not implied by the other parts of the definition. Consider the following space: Let $X$ be the union of the set $\mathbb{R}-\{0\}$ and the two point set $\cbr{p,q}$. Topologize $X$ by taking as the basis the collection of all open intervals in $\mathbb{R}$ that do not contain $0$, along with all sets of the form $(-a,0)\cup\cbr{p}\cup(0,a)$ and all sets of the form $(-a,0)\cup\cbr{q}\cup(0,a)$, for $a>0$. The space $X$ is called the \textit{\textbf{line with two origins}}. \begin{enumerate}
        \item Check that this is a basis for a topology. \begin{proof}
            Every nonzero real number $x$ is found in some interval $(x-\delta,x+\delta)$ for $\delta>0$ chosen sufficiently small. Similarly, $p$ and $q$ are found in $(-a,0)\cup\cbr{p}\cup(0,a)$ and $(-a,0)\cup\cbr{q}\cup(0,a)$ for some $a>0$.
            
            The intersection of any two basis elements not containing either of the origins $p$ or $q$ will clearly contain a smaller open interval.
            
            The intersection of a basis element not containing the origins with any basis element containing either of the origins may intersect only with the intervals $(-a,0)$ or $(0,a)$ so that the resulting intersection would be a union of at most two intervals which we could find a union of at most two smaller open intervals.

            The intersection of any two basis elements containing either of the origins will also appear as the union of two intervals with or without \textit{one} of the origins.
        \end{proof}
        \item Show that each of the spaces $X-\cbr{p}$ and $X - \cbr{q}$ is homeomorphic to $\mathbb{R}$.\begin{proof}
            We show the proof for $X-\cbr{p}$; the other follows by symmetry. We can map the basis elements appearing in $X-\cbr{p}$ to $\mathbb{R}$ by sending intervals not containing zero to themselves, and sending the basis elements which contain $p$ given by $(-a,0)\cup\cbr{p}\cup(0,a)$ for $a>0$ to $(-a,a)$. Extend this map into a continuous map from $X-\cbr{p}$ to $\mathbb{R}$.
            
            Furthermore, this map is invertible in the obvious manner, and the inverse is also continuous for similar reasons. It follows that the map given is a homeomorphism of $X-\cbr{p}$ with $\mathbb{R}$.
        \end{proof}
        \item Show that $X$ satisfies the $T_1$ axiom, but is not Hausdorff. \begin{proof}
            We show that singleton sets are closed; so that finite point sets are closed as a result of being a finite union of closed sets. The form for which the complement of the singleton $\cbr{x}$ takes on depends on whether or not $x$ is an origin or not. If $x$ is not an origin, then the form of its complement is either $\br{-\infty,x}\cup(x,0)\cup\cbr{p}\cup\cbr{q}\cup(0,\infty)$ or $\br{-\infty,0}\cup\cbr{p}\cup\cbr{q}\cup(0,x)\cup(x,\infty)$ depending on the sign of $x$. If $x = p$ then its complement is $\br{-\infty,0}\cup\cbr{q}\cup(0,\infty)$, and similarly for $q$ by reversing the roles of $p$ and $q$. 

            In all cases the complements obtained are open (union of basis elements) so singleton sets and hence finite point sets are closed. It follows that $X$ satisfies the $T_1$ axiom.
        \end{proof}
        \item Show that $X$ satisfies all the conditions for a $1$-manifold except for the Hausdorff condition. \begin{proof}
            We could take the collection of all intervals not containing zero which have rational endpoints, and sets of the form $(-a,0)\cup\cbr{p}\cup(0,b)$ or $(-a,0)\cup\cbr{q}\cup(0,b)$ with $a,b >0$ rational (these would be the intervals containing the origin with rational endpoints). This forms a countable basis for $X$.

            Every nonzero point is found in a small enough open interval, which is homeomorphic trivially to an open interval of $\mathbb{R}$. The origins are contained in open sets of the form $(-a,0)\cup\cbr{p}\cup(0,a)$ or $(-a,0)\cup\cbr{q}\cup(0,a)$ for $a>0$, and we can homeomorphically map these into the interval $(-a,a)$ by sending all the nonzero points to themselves and $p$ or $q$ to $0$ (one map for $p$, another for $q$). Hence each point in $X$ is contained in a neighborhood homeomorphic to an open subset of $\mathbb{R}$.

            But $X$ is not Hausdorff, as the origin $p$ and $q$ are not contained in disjoint open sets. Any open sets containing $p$ and $q$ must intersect since it would contain the basis elements $(-a,0)\cup\cbr{p}\cup(0,a)$ and $(-b,0)\cup\cbr{q}\cup(0,b)$ for suitable $a,b>0$, which intersect nontrivially. 

            Hence $X$ satisfies all the conditions for a $1$-manifold except for the Hausdorff condition.
        \end{proof}
    \end{enumerate}
    \item (37.1) Let $X$ be a space. Let $\mathcal{D}$ be a collection that is maximal with respect to the finite intersection property. \begin{enumerate}
        \item Show that $x\in \overline{D}$ for every $D\in \mathcal{D}$ if and only if every neighborhood of $x$ belongs to $\mathcal{D}$. Which implication uses maximality of $\mathcal{D}$? \begin{proof}
            Suppose that every neighborhood of $x$ is contained in $\mathcal{D}$. Then by the finite intersection property, for any given $D\in \mathcal{D}$ and any neighborhood $U$ of $x$ we have that $D\cap U$ is nonempty. Since $U$ was any neighborhood of $x$, we have that $x\in \overline{D}$. Since $D$ was arbitrary, we have that $x\in \overline{D}$ for every $D\in\mathcal{D}$.

            Conversely, suppose that $x\in \overline{D}$ for every $D\in\mathcal{D}$ so that every neighborhood of $x$ intersects nontrivially with every $D\in \mathcal{D}$. Suppose by way of contradiction that there is a neighborhood $U$ of $x$ which is not found in $\mathcal{D}$. Since $U\cap D\neq \emptyset$ for every $D\in \mathcal{D}$, it follows that $\cbr{U}\cup\mathcal{D}$ forms a larger collection of sets with the finite intersection property. But this is in contradiction to the fact that $\mathcal{D}$ was maximal, so instead $U$ must have been in $\mathcal{D}$. Hence every neighborhood of $x$ is in $\mathcal{D}$. [This direction uses maximality of $\mathcal{D}$.]
        \end{proof}
        \item Let $D\in\mathcal{D}$. Show that if $A\supset D$, then $A\in D$. \begin{proof}
            Observe that $A$ intersects nontrivially with every element in $\mathcal{D}$ since $D\subset A$, and $\mathcal{D}$ has the finite intersection property. But since $\mathcal{D}$ was chosen maximally with respect to the finite intersection property, we have that $\cbr{A}\cup \mathcal{D} = \mathcal{D}$ so that $A\in \mathcal{D}$ as desired.
        \end{proof}
        \item Show that if $X$ \st{satisfies the $T_1$ axiom} is Hausdorff, there is at most one point belonging to $\bigcap_{D\in\mathcal{D}} \overline{D}$.\begin{proof}
            Suppose that there are at least two distinct points $x$ and $y$ in $\bigcap_{D\in\mathcal{D}} \overline{D}$. By the Hausdorff condition we can find neighborhoods $U$ and $V$ containing $x$ and $y$ respectively which are disjoint from each other. But $x,y\in \overline{D}$ for every $D\in \mathcal{D}$, so that $U$ and $V$ intersects with every $D$ nontrivially. From the above results we have that $U,V\in \mathcal{D}$, from which it follows that $U\cap V$ is nonempty by the finite intersection property. This is in contradiction with our choice of $U$ and $V$. Hence there is at most one point belonging to $\bigcap_{D\in\mathcal{D}} \overline{D}$.
        \end{proof}
    \end{enumerate}
    \item (38.4) Let $Y$ be an arbitrary compactification of $X$; let $\beta(X)$ be the (concrete) Stone-\v{C}ech compactification. Show there is a continuous surjective closed map $g\colon\beta(X)\to Y$ that equals the identity on $X$. [This exercise makes precise what we mean by saying that $\beta(X)$ is the ``maximal'' compactification of $X$. It shows that every compactification of $X$ is equivalent to a quotient space of $\beta(X)$.] \begin{proof}
        Since $Y$ is a compact Hausdorff space, we can extend the inclusion mapping $\iota\colon X\hookrightarrow Y$ (which is continuous viewing $X$ as a subspace of $Y$) to a continuous map $g$ from $\beta(X)$ to $Y$. We show that $g$ is a closed surjective map which equals the identity on $X$.

        Since $\beta(X)$ is compact, any closed subset of $\beta(X)$ is also compact, and since $g$ is continuous, the image of a compact set will be compact. But a compact subset of a Hausdorff space is closed (as its complement can be shown to be open). Hence $g$ is a closed map.

        Similarly, since $\overline{X} = Y$, we have that $g$ maps $\overline{X}$ (or rather the closure of $X$ found in $\beta(X)$ ?) surjectively onto $Y$. Furthermore, the map $g$ will agree with the inclusion map on $X$, which is just the identity map.
    \end{proof}
    \item (38.6) Let $X$ be completely regular. Show that $X$ is connected if and only if $\beta(X)$ is connected. [\textit{Hint}. If $X = A\cup B$ is a separation of $X$, let $f(x) = 0$ for $x\in A$ and $f(x) = 1$ for $x\in B$.] \begin{proof}
        We show that $X$ is disconnected if and only if $\beta(X)$ is disconnected. Let $A,B$ be open sets of $X$ which form a separation of $X$. Define a continuous (surjective) function $f$ from $X = A\cup B$ to the compact Hausdorff space $\cbr{0,1}$ with the discrete topology such that $f(A) = \cbr{0}$ and $f(B) = \cbr{1}$ (continuous since preimages of open sets in $\cbr{0,1}$ are open).

        Since $f$ is continuous, we extend $f$ uniquely into a continuous map $g$ from $\beta(X)$ to $\cbr{0,1}$ which agrees with $f$ on $X$. We have that $g$ is surjective as a result, so that by taking preimages of $\cbr{0}$ and $\cbr{1}$ under $g$ we have open sets $A^\prime$ and $B^\prime$ respectively which are disjoint (and nonempty since $A$, $B$ are nonempty). It follows that $\beta(X)$ is separated since $A^\prime,B^\prime$ form a separation of $\beta(X)$.
    \end{proof}
    \item (38.7) Let $X$ be a discrete space; consider the space $\beta(X)$. \begin{enumerate}
        \item Show that if $A\subset X$, then $\overline{A}$ and $\overline{X-A}$ are disjoint, where the closures are taken in $\beta(X)$. \begin{proof}
            We employ the same trick where $A$ and $X-A$ form a separation of $X$ since both are open (clopen) sets of $X$, and define a continuous surjective function $f$ from $X$ to $\cbr{0,1}$ sending $A$ to $\cbr{0}$ and $X-A$ to $\cbr{1}$. Then extend $f$ into a surjective continuous function $g$ from $\beta(X)$ to $\cbr{0,1}$, and by taking the preimage of $\cbr{0}$ and $\cbr{1}$ under $g$ we obtain a separation of $\beta(X)$ into disjoint clopen sets $g^{-1}(\cbr{0}),g^{-1}(\cbr{1})$.
            
            Since $g$ agrees with $f$ on $A$ and $X-A$ it follows that $\overline{A}\subseteq g^{-1}(\cbr{0})$ and $\overline{X-A}\subseteq g^{-1}(\cbr{1})$ where the closures are taken in $\beta(X)$. It follows that $\overline{A}$ and $\overline{X-A}$ are disjoint.

            Furthermore, see that $\beta(X)-\overline{A} = \overline{X}-\overline{A}\subseteq \overline{X-A}$, but $\overline{X-A}$ is disjoint from $\overline{A}$ so that $\overline{X-A}\subseteq \beta(X) - \overline{A}$. Hence $\overline{X-A}$ is the complement of $\overline{A}$, meaning that $\overline{X-A}$ and also $\overline{A}$ are clopen.
        \end{proof}
        \item Show that if $U$ is open in $\beta(X)$, then $\overline{U}$ is open in $\beta(X)$. \begin{proof}
            In the previous result, take $A = U\cap X$ (open in the discrete space $X$ viewed as a subspace of $\beta(X)$) to find that $\overline{U\cap X}$ is a clopen set in $\beta(X)$. We wish to show that $\overline{U} = \overline{U\cap X}$ to obtain the result.

            Since $U\cap X\subseteq U$ we have that $\overline{U\cap X}\subseteq \overline{U}$. For any $x\in \overline{U}$, every neighborhood $V$ of $x$ intersects $U$ nontrivially. Because $U\cap V$ is a nonempty subset of $\overline{X}$, every neighborhood of a point found in $U\cap V$ will intersect $X$ nontrivially; since $U\cap V$ is the intersection of two open sets it is also open and is one such neighborhood. It follows that $V\cap (U\cap X)$ is nonempty; with $V$ chosen arbitrarily it follows that $x\in \overline{U\cap X}$.
            
            Hence $\overline{U}\subseteq \overline{U\cap X}$ so that $\overline{U} = \overline{U\cap X}$ is a clopen set in $\beta(X)$.
        \end{proof}
        \item Show that $\beta(X)$ is totally disconnected. \begin{proof}
            Let $Z$ be a set consisting of at least two distinct points $x,y$ in $\beta(X)$. We show that $Z$ is disconnected, so that only the singleton subsets of $\beta(X)$ could be connected (they are trivially connected).

            Using the Hausdorff condition we find open disjoint neighborhoods $U_x$ and $U_y$ of $x$ and $y$ respectively. Observe by part (b) that $\overline{U_x}$ is also open, so that $Z-\overline{U_x}$ is closed. We want $y$ to be in this set so that it is possible to find a separation at all. Suppose by way of contradiction that $y$ was not in the closed set $Z-\overline{U_x}$; that is, $y$ was in $\overline{U_x}$. But then every neighborhood of $y$ would intersect nontrivially with $U_x$; in particular $U_y$ would, which is in contradiction to $U_x$ and $U_y$ being disjoint. Hence $y\in Z-\overline{U_x}$, an open set since $\overline{U_x}$ is closed. Thus we have formed a separation of $Z$.

            Since $Z$ was arbitrary, it follows that the only connected sets are the singleton sets.
        \end{proof}
    \end{enumerate}
    \item (38.9a) If $X$ is normal and $y$ is a point of $\beta(X) - X$, show that $y$ is not the limit of a sequence of points of $X$. \begin{proof}
        Suppose by way of contradiction that there is a sequence $(x_n)$ of points in $X$ converging to $y\in \beta(X)-X$; without loss of generality assume that every $x_n$ is distinct (no repeating entries). Observe that the sequences $A = (x_{2k})$ and $B = (x_{2k+1})$ for $k\geq 0$ also converge to $y$.
        
        We show that the range of the sequences as subsets of $X$ are closed: Since they do not converge in $X$, there are no limit points in $X$ outside of the points appearing in the sequences themselves. Hence $A,B$ are closed disjoint subsets of $X$. By The Urysohn lemma we can form a continuous map $f$ from $X$ into $[0,1]$ sending $A$ to $\cbr{0}$ and $B$ to $\cbr{1}$. But we can extend this map to a continuous map $g$ from $\beta(X)$ to $\sbr{0,1}$ which agrees with $f$ on $X$. However, by sending the sequences $(x_{2k})$ and $(x_{2k+1})$ through $g$, by continuity the sequencex $(g(x_{2k})) = (0)$ and $(g(x_{2k+1})) = (1)$ must converge to $g(y)$, which is one point (but the sequences converge to different points in $\sbr{0,1}$). This is a contradiction. Hence $y$ is not the limit of a sequence of points of $X$.
    \end{proof}
    \item (38.10) We have constructed a correspondence $X\to \beta(X)$ that assigns, to each completely regular space, its Stone-\v{C}ech compactification. Now let us assign, to each continuous map $f\colon X\to Y$ of completely regular spaces, the unique continuous map $\beta(f)\colon \beta(X)\to \beta(Y)$ that extends the map $i\circ f$, where $i\colon Y\to \beta(Y)$ is the inclusion map. Verify the following: \begin{enumerate}
        \item If $1_X\colon X\to X$ is the identity map of $X$, then $\beta(1_X)$ is the identity map of $\beta(X)$. \begin{proof}
            From the diagram % https://q.uiver.app/?q=WzAsNCxbMCwwLCJYIl0sWzEsMCwiXFxiZXRhKFgpIl0sWzEsMSwiWCJdLFsxLDIsIlxcYmV0YShYKSJdLFswLDIsIjFfWCIsMl0sWzIsMywiXFxpb3RhIiwyXSxbMSwyLCIiLDIseyJzdHlsZSI6eyJib2R5Ijp7Im5hbWUiOiJkYXNoZWQifX19XSxbMSwzLCJcXGlkX3tcXGJldGEoWCl9IiwwLHsiY3VydmUiOi0yfV0sWzAsMSwiXFxzdWJzZXRlcSIsMSx7InN0eWxlIjp7ImJvZHkiOnsibmFtZSI6Im5vbmUifSwiaGVhZCI6eyJuYW1lIjoibm9uZSJ9fX1dXQ==
            \[\begin{tikzcd}
                X & {\beta(X)} \\
                & X \\
                & {\beta(X)}
                \arrow["{1_X}"', from=1-1, to=2-2]
                \arrow[hook', from=2-2, to=3-2]
                \arrow[dashed, from=1-2, to=2-2]
                \arrow["{\id_{\beta(X)}}", bend left=45, from=1-2, to=3-2]
                \arrow["\subseteq"{description}, draw=none, from=1-1, to=1-2]
            \end{tikzcd}\] observe that the identity map of $\beta(X)$ commutes with the extension of $i\circ 1_X$ so that by uniqueness of the extension we must have that $\beta(1_X)$ is the identity map of $\beta(X)$.
        \end{proof}
        \item If $f\colon X\to Y$ and $g\colon Y\to Z$, then $\beta(g\circ f) = \beta(g) \circ \beta(f)$. \begin{proof}
            From the diagram % https://q.uiver.app/?q=WzAsNixbMCwwLCJYIl0sWzAsMSwiWSJdLFswLDIsIloiXSxbMSwzLCJcXGJldGEoWikiXSxbMSwyLCJcXGJldGEoWSkiXSxbMSwwLCJcXGJldGEoWCkiXSxbMiwzLCIiLDAseyJzdHlsZSI6eyJ0YWlsIjp7Im5hbWUiOiJob29rIiwic2lkZSI6InRvcCJ9fX1dLFsxLDQsIiIsMCx7InN0eWxlIjp7InRhaWwiOnsibmFtZSI6Imhvb2siLCJzaWRlIjoidG9wIn19fV0sWzAsNSwiXFxzdWJzZXRlcSIsMSx7InN0eWxlIjp7ImJvZHkiOnsibmFtZSI6Im5vbmUifSwiaGVhZCI6eyJuYW1lIjoibm9uZSJ9fX1dLFswLDEsImYiLDJdLFsxLDIsImciLDJdLFs1LDQsIlxcYmV0YShmKSIsMCx7InN0eWxlIjp7ImJvZHkiOnsibmFtZSI6ImRhc2hlZCJ9fX1dLFs0LDMsIlxcYmV0YShnKSIsMCx7InN0eWxlIjp7ImJvZHkiOnsibmFtZSI6ImRhc2hlZCJ9fX1dLFs1LDMsIlxcYmV0YShnXFxjaXJjIGYpIiwwLHsiY3VydmUiOi0yfV1d
            \[\begin{tikzcd}
                X & {\beta(X)} \\
                Y \\
                Z & {\beta(Y)} \\
                & {\beta(Z)}
                \arrow[hook, from=3-1, to=4-2]
                \arrow[hook, from=2-1, to=3-2]
                \arrow["\subseteq"{description}, draw=none, from=1-1, to=1-2]
                \arrow["f"', from=1-1, to=2-1]
                \arrow["g"', from=2-1, to=3-1]
                \arrow["{\beta(f)}"', dashed, from=1-2, to=3-2]
	            \arrow["{\beta(g)}"', dashed, from=3-2, to=4-2]
                \arrow["{\beta(g\circ f)}", bend left=45, from=1-2, to=4-2]
            \end{tikzcd}\] we obtain that $\beta(g\circ f)$ commutes with $\beta(g)\circ\beta(f)$; by uniqueness of the extensions involved we have that $\beta(g\circ f) = \beta(g) \circ \beta(f)$ as desired.
        \end{proof}
    \end{enumerate} These properties tell us that the correspondence we have constructed is what is called a \textit{\textbf{functor}}; it is a functor from the ``category'' of completely regular spaces and continuous maps of such spaces, to the ``category'' of compact Hausdorff spaces and continuous maps of such spaces.
    \item Let $Y$ be the concrete Stone-\v{C}ech compactification of the compact completely regular space $X$. Show that $X = Y$. \begin{proof}
        Since $X$ is a compact Hausdorff space we can form the following diagram, with $\iota\colon X\to Y$ being the unique inclusion map of $X$ into $Y$: % https://q.uiver.app/?q=WzAsNCxbMCwwLCJYIl0sWzEsMCwiWSJdLFsxLDEsIlgiXSxbMSwyLCJZIl0sWzAsMiwiXFxpZF9YIiwyLHsiY3VydmUiOjF9XSxbMCwxLCJcXGlvdGEiLDAseyJzdHlsZSI6eyJ0YWlsIjp7Im5hbWUiOiJob29rIiwic2lkZSI6InRvcCJ9fX1dLFsxLDIsIlxcb3ZlcmxpbmV7XFxpZF9YfSIsMix7InN0eWxlIjp7ImJvZHkiOnsibmFtZSI6ImRhc2hlZCJ9fX1dLFsyLDMsIlxcaW90YSIsMix7InN0eWxlIjp7InRhaWwiOnsibmFtZSI6Imhvb2siLCJzaWRlIjoidG9wIn19fV0sWzEsMywiXFxpZF9ZIiwwLHsiY3VydmUiOi0yfV1d
        \[\begin{tikzcd}
            X & Y \\
            & X \\
            & Y
            \arrow["{\id_X}"', bend left=-30, from=1-1, to=2-2]
            \arrow["\iota", hook, from=1-1, to=1-2]
            \arrow["{\overline{\id_X}}"', dashed, from=1-2, to=2-2]
            \arrow["\iota"', hook, from=2-2, to=3-2]
            \arrow["{\id_Y}", bend left=45, from=1-2, to=3-2]
        \end{tikzcd}\] Since the diagram commutes and the maps involved are all unique and continuous, we have that $\overline{\id_X}\circ \iota = \id_X$ and $\iota\circ\overline{\id_X}$. It follows that $\iota$ and $\overline{\id_X}$ are homeomorphisms (of $X$ with $Y$), and since $X$ is contained in $Y$ it follows that $X = Y$.

        Alternatively, since $\overline{X} = Y$ in $Y$, the inclusion map from $X$ to $\overline{X}$ can be extended to a continuous map from $Y$ to $\overline{X} = Y$ which by uniqueness must be the identity map on $Y$. It follows similarly that $X = Y$.
    \end{proof}
    \item If $Y$ and $Y^\prime$ are diagrammatic Stone-\v{C}ech compactifications of a completely regular space $X$ under $\varphi$ and $\varphi^\prime$ respectively, then there exists a unique homeomorphism $h\colon Y\to Y^\prime$ with % https://q.uiver.app/?q=WzAsMyxbMSwwLCJZIl0sWzEsMSwiWV5cXHByaW1lIl0sWzAsMCwiWCJdLFswLDEsImgiLDAseyJzdHlsZSI6eyJib2R5Ijp7Im5hbWUiOiJkYXNoZWQifX19XSxbMiwxLCJcXHZhcnBoaV5cXHByaW1lIiwyXSxbMiwwLCJcXHZhcnBoaSJdXQ==
    \[\begin{tikzcd}
        X & Y \\
        & {Y^\prime}
        \arrow["h", dashed, from=1-2, to=2-2]
        \arrow["{\varphi^\prime}"', from=1-1, to=2-2]
        \arrow["\varphi", from=1-1, to=1-2]
    \end{tikzcd}.\] \begin{proof}
        With $Y$ and $Y^\prime$ diagrammatic Stone-\v{C}ech compactifications, we obtain the diagrams % https://q.uiver.app/?q=WzAsNixbMCwwLCJYIl0sWzEsMCwiWSJdLFsxLDEsIkMiXSxbMiwwLCJYIl0sWzMsMCwiWSJdLFszLDEsIkNeXFxwcmltZSJdLFswLDIsImYiLDJdLFswLDEsIlxcdmFycGhpIl0sWzEsMiwiXFxvdmVybGluZXtmfSIsMCx7InN0eWxlIjp7ImJvZHkiOnsibmFtZSI6ImRhc2hlZCJ9fX1dLFszLDUsImZeXFxwcmltZSIsMl0sWzMsNCwiXFx2YXJwaGleXFxwcmltZSJdLFs0LDUsIlxcb3ZlcmxpbmV7Zl5cXHByaW1lfSIsMCx7InN0eWxlIjp7ImJvZHkiOnsibmFtZSI6ImRhc2hlZCJ9fX1dXQ==
        \[\begin{tikzcd}
            X & Y & X & Y^\prime \\
            & C && {C^\prime}
            \arrow["f"', from=1-1, to=2-2]
            \arrow["\varphi", from=1-1, to=1-2]
            \arrow["{\overline{f}}", dashed, from=1-2, to=2-2]
            \arrow["{f^\prime}"', from=1-3, to=2-4]
            \arrow["{\varphi^\prime}", from=1-3, to=1-4]
            \arrow["{\overline{f^\prime}}", dashed, from=1-4, to=2-4]
        \end{tikzcd}\] for any $C,C^\prime$. By taking $C$ to be $Y^\prime$ and $C^\prime$ to be $Y$ (and changing the appropriate maps), and by stacking diagrams we obtain % https://q.uiver.app/?q=WzAsOCxbMCwwLCJYIl0sWzEsMCwiWSJdLFsxLDEsIlleXFxwcmltZSJdLFsyLDAsIlgiXSxbMywwLCJZXlxccHJpbWUiXSxbMywxLCJZIl0sWzEsMiwiWSJdLFszLDIsIlleXFxwcmltZSJdLFswLDIsIlxcdmFycGhpXntcXHByaW1lfSIsMl0sWzAsMSwiXFx2YXJwaGkiXSxbMSwyLCJcXG92ZXJsaW5le2Z9IiwwLHsic3R5bGUiOnsiYm9keSI6eyJuYW1lIjoiZGFzaGVkIn19fV0sWzMsNSwiXFx2YXJwaGkiLDJdLFszLDQsIlxcdmFycGhpXlxccHJpbWUiXSxbNCw1LCJcXG92ZXJsaW5le2ZeXFxwcmltZX0iLDAseyJzdHlsZSI6eyJib2R5Ijp7Im5hbWUiOiJkYXNoZWQifX19XSxbMyw3LCJcXHZhcnBoaV5cXHByaW1lIiwyLHsiY3VydmUiOjJ9XSxbMCw2LCJcXHZhcnBoaSIsMix7ImN1cnZlIjoyfV0sWzIsNiwiXFxvdmVybGluZXtmXlxccHJpbWV9IiwwLHsic3R5bGUiOnsiYm9keSI6eyJuYW1lIjoiZGFzaGVkIn19fV0sWzUsNywiXFxvdmVybGluZXtmfSJdLFsxLDYsIlxcaWRfWSIsMSx7ImN1cnZlIjotMn1dLFs0LDcsIlxcaWRfe1leXFxwcmltZX0iLDEseyJjdXJ2ZSI6LTJ9XV0=
        \[\begin{tikzcd}
            X & Y & X & {Y^\prime} \\
            & {Y^\prime} && Y \\
            & Y && {Y^\prime}
            \arrow["{\varphi^{\prime}}"', from=1-1, to=2-2]
            \arrow["\varphi", from=1-1, to=1-2]
            \arrow["{\overline{\varphi}}", dashed, from=1-2, to=2-2]
            \arrow["\varphi"', from=1-3, to=2-4]
            \arrow["{\varphi^\prime}", from=1-3, to=1-4]
            \arrow["{\overline{\varphi^\prime}}", dashed, from=1-4, to=2-4]
            \arrow["{\varphi^\prime}"', bend left=-45, from=1-3, to=3-4]
            \arrow["\varphi"', bend left=-45, from=1-1, to=3-2]
            \arrow["{\overline{\varphi^\prime}}", dashed, from=2-2, to=3-2]
            \arrow["{\overline{\varphi}}", from=2-4, to=3-4]
            \arrow["{\id_Y}"{description}, bend left=45, from=1-2, to=3-2]
            \arrow["{\id_{Y^\prime}}"{description}, bend left=45, from=1-4, to=3-4]
        \end{tikzcd}\] Because the diagram commutes and $\overline{\varphi},\overline{\varphi^\prime}$ were unique, it follows that $\overline{\varphi}\circ\overline{\varphi^\prime} = \id_{Y^\prime}$ and $\overline{\varphi^\prime}\circ\overline{\varphi} = \id_{Y}$, so that $\overline{\varphi}$ is the unique homeomorphism (continuous bijection with continuous inverse $\overline{\varphi^\prime}$) of $Y$ with $Y^\prime$.
    \end{proof}
\end{enumerate}
\end{document}