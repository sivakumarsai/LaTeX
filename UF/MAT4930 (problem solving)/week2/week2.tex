\documentclass[11pt]{article}

% packages
\usepackage{physics}
% margin spacing
\usepackage[top=1in, bottom=1in, left=0.5in, right=0.5in]{geometry}
\usepackage{hanging}
\usepackage{amsfonts, amsmath, amssymb, amsthm}
\usepackage{systeme}
\usepackage[none]{hyphenat}
\usepackage{fancyhdr}
\usepackage[nottoc, notlot, notlof]{tocbibind}
\usepackage{graphicx}
\graphicspath{{./images/}}
\usepackage{float}
\usepackage{siunitx}
\usepackage{esint}
\usepackage{cancel}

% permutations (second line is for spacing)
\usepackage{permute}
\renewcommand*\pmtseparator{\,}

% colors
\usepackage{xcolor}
\definecolor{p}{HTML}{FFDDDD}
\definecolor{g}{HTML}{D9FFDF}
\definecolor{y}{HTML}{FFFFCF}
\definecolor{b}{HTML}{D9FFFF}
\definecolor{o}{HTML}{FADECB}
%\definecolor{}{HTML}{}

% \highlight[<color>]{<stuff>}
\newcommand{\highlight}[2][p]{\mathchoice%
  {\colorbox{#1}{$\displaystyle#2$}}%
  {\colorbox{#1}{$\textstyle#2$}}%
  {\colorbox{#1}{$\scriptstyle#2$}}%
  {\colorbox{#1}{$\scriptscriptstyle#2$}}}%

% header/footer formatting
\pagestyle{fancy}
\fancyhead{}
\fancyfoot{}
\fancyhead[L]{\textbf{Pigeonhole Principle}}
\fancyhead[C]{}
\fancyhead[R]{Sai Sivakumar}
\fancyfoot[R]{\thepage}
\renewcommand{\headrulewidth}{0pt}

% paragraph indentation/spacing
\setlength{\parindent}{0cm}
\setlength{\parskip}{10pt}
\renewcommand{\baselinestretch}{1.25}

% extra commands defined here
\newcommand{\ihat}{\boldsymbol{\hat{\textbf{\i}}}}
\newcommand{\jhat}{\boldsymbol{\hat{\textbf{\j}}}}
\newcommand{\dr}{\vec{r}~^{\prime}(t)}
\newcommand{\dx}{x^{\prime}(t)}
\newcommand{\dy}{y^{\prime}(t)}

\newcommand{\br}[1]{\left(#1\right)}
\newcommand{\sbr}[1]{\left[#1\right]}
\newcommand{\cbr}[1]{\left\{#1\right\}}

\newcommand{\dprime}{\prime\prime}
\newcommand{\lap}[2]{\mathcal{L}[#1](#2)}

\newcommand{\divides}{\mid}

% bracket notation for inner product
\usepackage{mathtools}

\DeclarePairedDelimiterX{\abr}[1]{\langle}{\rangle}{#1}

\DeclareMathOperator{\Span}{span}
\DeclareMathOperator{\nullity}{nullity}
\DeclareMathOperator\Aut{Aut}
\DeclareMathOperator\Inn{Inn}

% set page count index to begin from 1
\setcounter{page}{1}

\begin{document}
\textbf{5.} Given a set of $n+1$ positive integers, none of which exceeds $2n$, show that at least one member of the set must divide another member of the set. \begin{proof}
    The set of $n+1$ positive integers is a subset of $S = \cbr{1,2,\dots,2n}$. Consider the set of \textit{odd} numbers in $S$, $\cbr{1,3,\dots,2n-1}$, and note that there are $n$ of these odd numbers. Then consider writing the prime factorization of all of the numbers $x\in S$ as $x = 2^k\cdot j$, where $k\geq 0$ and $j$ is an odd positive integer (which by division is less than or equal to $x$).
    
    So what we can do is treat the odd numbers in $S$ as boxes, and place each $x = 2^k\cdot j$ in each box where the odd number $j$ matches with the odd number of the box. This means that the odd number of the box divides the number we placed in the box. Furthermore, for any numbers can be placed in this box (including the odd numbers themselves), the smaller numbers will divide the larger numbers in the box. (See that for $0 \leq l \leq k$, $2^lj\mid 2^kj$.) In other words, the map from $S$ to the subset of the odd numbers in $S$ given by $x\mapsto j$ (same $j$ as before in the prime factorization for $x$) is what we are interested in. This map is definitely not injective if we take the domain to be a subset of $S$ of size strictly greater than $n$.

    Observe then for an $n+1$ sized subset of $S$ we can apply this map where we first factor out all of the $2$'s from each element in the subset and identify the remainder with an odd number in $S$, and so two numbers in the subset will map to the same odd number in $S$. This means that one of these two numbers will divide the other one.
\end{proof}

\textbf{19.} Forty-one rooks are placed on a $10\times 10$ chessboard. Prove that we can choose $5$ of these rooks which don't attack each other. \begin{proof}
    To show this, all we need to do is partition the chessboard into ten groupings of ten tiles each, with the requirement that for any tiles in the same grouping, they cannot attack each other.

    Pictorially, one example (this is not the only valid grouping) is the following: \begin{table}[h]\centering
        \begin{tabular}{|l|l|l|l|l|l|l|l|l|l|}
        \hline
        1& 2 & 3 & 4 & 5 & 6 & 7 & 8 & 9 & 0 \\ \hline
        0& 1 & 2 & 3 & 4 & 5 & 6 & 7 & 8 & 9 \\ \hline
        9& 0 & 1 & 2 & 3 & 4 & 5 & 6 & 7 & 8 \\ \hline
        8& 9 & 0 & 1 & 2 & 3 & 4 & 5 & 6 & 7 \\ \hline
        7& 8 & 9 & 0 & 1 & 2 & 3 & 4 & 5 & 6 \\ \hline
        6& 7 & 8 & 9 & 0 & 1 & 2 & 3 & 4 & 5 \\ \hline
        5& 6 & 7 & 8 & 9 & 0 & 1 & 2 & 3 & 4 \\ \hline
        4& 5 & 6 & 7 & 8 & 9 & 0 & 1 & 2 & 3 \\ \hline
        3& 4 & 5 & 6 & 7 & 8 & 9 & 0 & 1 & 2 \\ \hline
        2& 3 & 4 & 5 & 6 & 7 & 8 & 9 & 0 & 1 \\ \hline
        \end{tabular}
        \end{table}


So all rooks in group $1$ cannot attack each other, similarly for all rooks in each grouping $n$ for $0\leq n \leq 9$. So if $41$ rooks have to be distributed across these ten groupings, at least one of these groupings will have at least $5$ rooks. These rooks will be incapable of attacking each other by construction of the groupings/boxes. Hence we can choose $5$ rooks which cannot attack each other on this chessboard.
\end{proof}
\end{document}