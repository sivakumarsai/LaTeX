\documentclass[11pt]{article}

% packages
\usepackage{physics}
% margin spacing
\usepackage[top=1in, bottom=1in, left=0.5in, right=0.5in]{geometry}
\usepackage{hanging}
\usepackage{amsfonts, amsmath, amssymb, amsthm}
\usepackage{systeme}
\usepackage[none]{hyphenat}
\usepackage{fancyhdr}
\usepackage[nottoc, notlot, notlof]{tocbibind}
\usepackage{graphicx}
\graphicspath{{./images/}}
\usepackage{float}
\usepackage{siunitx}
\usepackage{esint}
\usepackage{cancel}

% permutations (second line is for spacing)
\usepackage{permute}
\renewcommand*\pmtseparator{\,}

% colors
\usepackage{xcolor}
\definecolor{p}{HTML}{FFDDDD}
\definecolor{g}{HTML}{D9FFDF}
\definecolor{y}{HTML}{FFFFCF}
\definecolor{b}{HTML}{D9FFFF}
\definecolor{o}{HTML}{FADECB}
%\definecolor{}{HTML}{}

% \highlight[<color>]{<stuff>}
\newcommand{\highlight}[2][p]{\mathchoice%
  {\colorbox{#1}{$\displaystyle#2$}}%
  {\colorbox{#1}{$\textstyle#2$}}%
  {\colorbox{#1}{$\scriptstyle#2$}}%
  {\colorbox{#1}{$\scriptscriptstyle#2$}}}%

% header/footer formatting
\pagestyle{fancy}
\fancyhead{}
\fancyfoot{}
\fancyhead[L]{\textbf{Elementary Number Theory}}
\fancyhead[C]{}
\fancyhead[R]{Sai Sivakumar}
\fancyfoot[R]{\thepage}
\renewcommand{\headrulewidth}{0pt}

% paragraph indentation/spacing
\setlength{\parindent}{0cm}
\setlength{\parskip}{10pt}
\renewcommand{\baselinestretch}{1.25}

% extra commands defined here
\newcommand{\ihat}{\boldsymbol{\hat{\textbf{\i}}}}
\newcommand{\jhat}{\boldsymbol{\hat{\textbf{\j}}}}
\newcommand{\dr}{\vec{r}~^{\prime}(t)}
\newcommand{\dx}{x^{\prime}(t)}
\newcommand{\dy}{y^{\prime}(t)}

\newcommand{\br}[1]{\left(#1\right)}
\newcommand{\sbr}[1]{\left[#1\right]}
\newcommand{\cbr}[1]{\left\{#1\right\}}

\newcommand{\dprime}{\prime\prime}
\newcommand{\lap}[2]{\mathcal{L}[#1](#2)}

\newcommand{\divides}{\mid}

% bracket notation for inner product
\usepackage{mathtools}

\DeclarePairedDelimiterX{\abr}[1]{\langle}{\rangle}{#1}

\DeclareMathOperator{\Span}{span}
\DeclareMathOperator{\nullity}{nullity}
\DeclareMathOperator\Aut{Aut}
\DeclareMathOperator\Inn{Inn}

% set page count index to begin from 1
\setcounter{page}{1}

\begin{document}
\textbf{1.} Let $a,b$ be integers which are relatively prime. Prove that $\gcd(a^2-ab+b^2, a+b)\leq 3$.
\begin{proof}
    Let $a,b$ be coprime integers as given. Write $a^2-ab+b^2$ equivalently as $(a+b)^2 - 3ab$. Then $\gcd(a^2-ab+b^2, a+b) = \gcd((a+b)^2 - 3ab, a+b) = \gcd(-(a+b)^2 + 3ab, a+b)$, and apply one step of the Euclidean algorithm to find
    \begin{align*}
        -(a+b)^2 + 3ab &= -(a+b)\cdot (a+b) + 3ab,
    \end{align*}
    so that $\gcd(-(a+b)^2 + 3ab, a+b) = \gcd(a+b,3ab)$. But because $a,b$ were coprime, any divisor of $a+b$ will not divide $a$ or $b$, and similarly, any divisor of $a$ or $b$ will not divide $a+b$. This is the same as saying that $\gcd(a+b, a) = \gcd(a+b,b) = \gcd(a+b,ab) = 1$, which makes sense from considering prime factorizations for $a$ and $b$. Thus any common divisor of $a+b$ and $3ab$ is actually a common divisor of $a+b$ and $3$.

    So we may simplify $\gcd(a+b,3ab)$ into $\gcd(a+b,3)$, and because $3$ is prime, the greatest common factor can either be $1$ or $3$. Hence $\gcd(a^2-ab+b^2, a+b)\leq 3$.
\end{proof}
\textbf{2.} Prove that two successive Fibonacci numbers are relatively prime.
\begin{proof}
    Let the Fibonacci numbers start with $f_0 = 1, f_1 = 1$, and we have that $f_{n+2} = f_{n+1} + f_n$ for all $n\geq 0$. We have that $\gcd(1,1) = \gcd(2,1) = 1$, so the first couple pairs of successive Fibonacci numbers are relatively prime. Then suppose that $\gcd(f_j, f_{j-1}) = 1$ for all $0\leq j \leq n$. We show that $\gcd(f_{n+1}, f_n) = 1$.

    Writing $f_{n+1} = f_n + f_{n-1}$, we have that \[\gcd(f_{n+1}, f_n) = \gcd(f_n + f_{n-1}, f_n),\] which by the Euclidean algorithm we have that \[\gcd(f_n + f_{n-1}, f_n) = \gcd(f_n, f_{n-1}).\] But by the inductive hypothesis, the greatest common factor is $1$, so $\gcd(f_{n+1}, f_n) = 1$.

    Hence, by induction, any two successive Fibonacci numbers are relatively prime.
\end{proof}
\end{document}