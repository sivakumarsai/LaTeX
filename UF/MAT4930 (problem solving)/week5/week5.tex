\documentclass[11pt]{article}

% packages
\usepackage{physics}
% margin spacing
\usepackage[top=1in, bottom=1in, left=0.5in, right=0.5in]{geometry}
\usepackage{hanging}
\usepackage{amsfonts, amsmath, amssymb, amsthm}
\usepackage{systeme}
\usepackage[none]{hyphenat}
\usepackage{fancyhdr}
\usepackage[nottoc, notlot, notlof]{tocbibind}
\usepackage{graphicx}
\graphicspath{{./images/}}
\usepackage{float}
\usepackage{siunitx}
\usepackage{esint}
\usepackage{cancel}

% permutations (second line is for spacing)
\usepackage{permute}
\renewcommand*\pmtseparator{\,}

% colors
\usepackage{xcolor}
\definecolor{p}{HTML}{FFDDDD}
\definecolor{g}{HTML}{D9FFDF}
\definecolor{y}{HTML}{FFFFCF}
\definecolor{b}{HTML}{D9FFFF}
\definecolor{o}{HTML}{FADECB}
%\definecolor{}{HTML}{}

% \highlight[<color>]{<stuff>}
\newcommand{\highlight}[2][p]{\mathchoice%
  {\colorbox{#1}{$\displaystyle#2$}}%
  {\colorbox{#1}{$\textstyle#2$}}%
  {\colorbox{#1}{$\scriptstyle#2$}}%
  {\colorbox{#1}{$\scriptscriptstyle#2$}}}%

% header/footer formatting
\pagestyle{fancy}
\fancyhead{}
\fancyfoot{}
\fancyhead[L]{\textbf{Congruence}}
\fancyhead[C]{}
\fancyhead[R]{Sai Sivakumar}
\fancyfoot[R]{\thepage}
\renewcommand{\headrulewidth}{0pt}

% paragraph indentation/spacing
\setlength{\parindent}{0cm}
\setlength{\parskip}{10pt}
\renewcommand{\baselinestretch}{1.25}

% extra commands defined here
\newcommand{\ihat}{\boldsymbol{\hat{\textbf{\i}}}}
\newcommand{\jhat}{\boldsymbol{\hat{\textbf{\j}}}}
\newcommand{\dr}{\vec{r}~^{\prime}(t)}
\newcommand{\dx}{x^{\prime}(t)}
\newcommand{\dy}{y^{\prime}(t)}

\newcommand{\br}[1]{\left(#1\right)}
\newcommand{\sbr}[1]{\left[#1\right]}
\newcommand{\cbr}[1]{\left\{#1\right\}}

\newcommand{\dprime}{\prime\prime}
\newcommand{\lap}[2]{\mathcal{L}[#1](#2)}

\newcommand{\divides}{\mid}

% bracket notation for inner product
\usepackage{mathtools}

\DeclarePairedDelimiterX{\abr}[1]{\langle}{\rangle}{#1}

\DeclareMathOperator{\Span}{span}
\DeclareMathOperator{\nullity}{nullity}
\DeclareMathOperator\Aut{Aut}
\DeclareMathOperator\Inn{Inn}

% set page count index to begin from 1
\setcounter{page}{1}

\begin{document}
\textbf{4.} Prove that any set of $n$ integers contains a nonempty subset the sum of whose elements is divisible by $n$.

Every subset of the empty set is empty, so I assume we take $n>0$.
\begin{proof}
    Let $\cbr{s_1,s_2,\dots,s_n}$ be a set of $n>0$ integers. Then there are $2^n-1$ nonempty subsets of this set (the powerset without the empty set). We do not need to investigate all of these subsets. Consider the $n$ nested nonempty subsets $\cbr{s_1}\subset \cbr{s_1,s_2}\subset \cdots \subset \cbr{s_1,s_2,\dots,s_n}$. Then each of the subsets have the corresponding sums:
    \begin{align*}
        S_1 &= s_1 \\
        S_2 &= s_1 + s_2 \\
        &~\,\vdots \\
        S_n &= s_1 + \cdots + s_n 
    \end{align*}
    Divide each sum $S_i$ by $n$ using the division algorithm to obtain nonnegative remainders $r_i$ strictly less than $n$.

    If some $r_i$ is equal to zero, then the subset corresponding to the sum $S_i$ is a nonempty subset the sum of whose elements is divisible by $n$.

    If none of the $r_i$ are equal to zero, we invoke the pigeonhole principle to place each $r_i$ in $n-1$ boxes; these boxes being the $n-1$ nonzero residue classes in $\mathbb{Z}/n\mathbb{Z}$. There are $n$ remainders, and $n-1$ nonzero residue classes, so it follows that at least two remainders $r_k$ and $r_j$ are equal to each other. Without loss of generality, take $k>j$ so that the subset corresponding to $S_j$ is contained in the subset corresponding to $S_k$.

    If $r_k = r_j$, then $n\mid S_k-S_j$. But $S_k - S_j = s_{j+1} + \cdots + s_k$. But $\cbr{s_{j+1},\dots, s_k}\subseteq\cbr{s_1,s_2,\dots,s_n}$, and we have exhibited a nonempty subset the sum of whose elements is divisible by $n$.
\end{proof}
\textbf{7.} Prove that the set $\cbr{11,111,1111,\dots}$ contains no squares.
\begin{proof}
    Rewrite the set $\cbr{11,111,1111,\dots}$ as $\cbr{8+3,108+3,1108+3,\dots}$. Each of these elements are of the form $4k+3$ for some $k\in \mathbb{Z}$, since both $10^n$ for $n\geq 2$ and $8$ are divisible by $4$.

    But there are no square integers of the form $4k+3$. Consider the even integers which take on either the form $4n+0$ or $4n+2$ for integers $n$. Then $(4n)^2 = 16n^2 \equiv 0 \pmod{4}$ and $(4n+2)^2 = 4(4n^2 + 4n + 1) \equiv 0 \pmod{4}$. 

    The odd integers take on either the form $4n+1$ or $4n+3$ for integers $n$. Then again $(4n+1)^2 = 4(4n^2 + 2n) + 1 \equiv 1 \pmod{4}$ and $(4n+3)^2 = 4(4n^2 + 6n + 2) + 1 \equiv 1 \pmod{4}$.

    So for all integers after squaring them and reducing modulo $4$, none take on the form $4k+3$ for some integer $k$. But every element in the set takes on this form, so it is impossible for there to be a squared integer within the set $\cbr{11,111,1111,\dots}$.
\end{proof}
\end{document}