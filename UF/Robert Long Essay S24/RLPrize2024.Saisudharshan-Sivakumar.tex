\documentclass[10pt,leqno]{article}

% packages
\usepackage[alphabetic]{amsrefs}
\usepackage{physics}
% margin spacing
\usepackage[top=1in, bottom=1in, left=1in, right=1in]{geometry}
\usepackage{amsfonts, amsmath, amssymb, amsthm}
\usepackage{fancyhdr}
\usepackage{graphicx}
\graphicspath{{./images/}}
\usepackage{enumitem}
\usepackage{quiver}
\usepackage{hyperref}
% \hypersetup{colorlinks=true,linkcolor=blue}
\usepackage[capitalize,noabbrev]{cleveref}

% % colors
% \usepackage{xcolor}
% \definecolor{p}{HTML}{FFDDDD}
% \definecolor{g}{HTML}{D9FFDF}
% \definecolor{y}{HTML}{FFFFCF}
% \definecolor{b}{HTML}{D9FFFF}
% \definecolor{o}{HTML}{FADECB}
% %\definecolor{}{HTML}{}

% % \highlight[<color>]{<stuff>}
% \newcommand{\highlight}[2][p]{\mathchoice%
%   {\colorbox{#1}{$\displaystyle#2$}}%
%   {\colorbox{#1}{$\textstyle#2$}}%
%   {\colorbox{#1}{$\scriptstyle#2$}}%
%   {\colorbox{#1}{$\scriptscriptstyle#2$}}}%

% header/footer formatting
\fancypagestyle{frontmatter}{
    \fancyhf{}
    \pagestyle{fancy}
    \fancyhead[R]{\thepage}
}
\fancypagestyle{body}{
    \pagestyle{fancy}
    \fancyhead[L]{\rightmark}
    \fancyhead[R]{\thepage}
}
\renewcommand{\headrulewidth}{1pt}

% paragraph indentation/spacing
\setlength{\parindent}{0pt}
\setlength{\parskip}{6pt}
\renewcommand{\baselinestretch}{1.00}

% extra commands defined here
\newcommand{\br}[1]{\left(#1\right)}
\newcommand{\sbr}[1]{\left[#1\right]}
\newcommand{\cbr}[1]{\left\{#1\right\}}

% bracket notation for inner product
\usepackage{mathtools}

\DeclarePairedDelimiterX{\abr}[1]{\langle}{\rangle}{#1}

% new commands
\newcommand{\textib}[1]{\textbf{\textit{#1}}}
\DeclareMathOperator{\Mat}{M}
\DeclareMathOperator{\GL}{GL}
\DeclareMathOperator{\SL}{SL}
\newcommand{\smod}[1]{\;(\bmod\; #1)}
% \DeclareMathOperator{\Span}{span}
% \DeclareMathOperator{\nullity}{nullity}
% \DeclareMathOperator\Aut{Aut}
% \DeclareMathOperator\Inn{Inn}
% \DeclareMathOperator{\Orb}{Orb}
% \DeclareMathOperator{\lcm}{lcm}
% \DeclareMathOperator{\Hol}{Hol}
% \DeclareMathOperator{\Jac}{Jac}
% \DeclareMathOperator{\rad}{rad}
% \DeclareMathOperator{\Tor}{Tor}
% \DeclareMathOperator{\End}{End}
% \DeclareMathOperator{\Gal}{Gal}
% \DeclareMathOperator{\Nat}{Nat}
% \DeclareMathOperator{\Frac}{Frac}
% \DeclareMathOperator{\id}{id}
% \DeclareMathOperator{\im}{im}
% \DeclareMathOperator{\Hom}{Hom}
% \DeclareMathOperator{\Ext}{Ext}
% \DeclareMathOperator{\aug}{aug}

% indices
\setcounter{page}{1}
\setcounter{section}{-1}

% array column and row separation
\arraycolsep = 3pt
\renewcommand{\arraystretch}{.8}

\begin{document}
\begin{titlepage}
    \begin{center}
        \vspace*{4em}
        {\Large\textbf{The Fourier transform and its avatars}}

        \vspace{6em}
        \includegraphics[scale=0.14]{uf.png}

        \vspace{6em}
        Sai Sivakumar

        Robert Long Essay Competition -- Spring 2024\\
        Department of Mathematics, University of Florida
    \end{center}
\end{titlepage}
\pagestyle{frontmatter}
\tableofcontents\newpage


\pagestyle{body}
% \newpage\section{Acknowledgements}

% \newpage\section{Introduction}

\newpage\section{Preliminaries} In this section, we collect the definitions and results needed

To cover: \begin{itemize}
    \item genesis of Fourier series/transform
    \item orthonormal basis in inner product space
    \item Fourier series and fourier transform
    \item modular forms
    \item fast Fourier transform, and naive alg over $\mathbb{Z}/n\mathbb{Z}$
    \item Fourier analysis over finite abelian and lca groups
    \item noncomm harmonic analysis
    \item ODE, PDE, and sturm liouville operators
    \item Riemannian manifolds, homogeneous spaces
    \item distributions
    \item $C^\ast$-algebras
    \item the Laplace transform
    \item historical content throughout
\end{itemize}\newpage % <- delete
\subsection{The modular group and congruence subgroups}
Let $R$ be a unital ring

\newpage\pagestyle{frontmatter}
\begin{bibdiv}
\begin{biblist}
\bib{serre}{book}{
    title = {A Course in Arithmetic},
    author = {Serre, Jean-Pierre},
    isbn = {978-0-387-90040-7},
    series = {Graduate Texts in Mathematics},
    year = {1978},
    publisher = {Springer New York, NY}
}
\end{biblist}
\end{bibdiv}
\end{document}