\documentclass[11pt]{article}

% packages
\usepackage{physics}
% margin spacing
\usepackage[top=1in, bottom=1in, left=0.5in, right=0.5in]{geometry}
\usepackage{hanging}
\usepackage{amsfonts, amsmath, amssymb, amsthm}
\usepackage{systeme}
\usepackage[none]{hyphenat}
\usepackage{fancyhdr}
\usepackage[nottoc, notlot, notlof]{tocbibind}
\usepackage{graphicx}
\graphicspath{{./images/}}
\usepackage{float}
\usepackage{siunitx}
\usepackage{esint}
\usepackage{cancel}
\usepackage{enumitem}
\usepackage{mathrsfs}

% colors
\usepackage{xcolor}
\definecolor{p}{HTML}{FFDDDD}
\definecolor{g}{HTML}{D9FFDF}
\definecolor{y}{HTML}{FFFFCF}
\definecolor{b}{HTML}{D9FFFF}
\definecolor{o}{HTML}{FADECB}
%\definecolor{}{HTML}{}

% \highlight[<color>]{<stuff>}
\newcommand{\highlight}[2][p]{\mathchoice%
  {\colorbox{#1}{$\displaystyle#2$}}%
  {\colorbox{#1}{$\textstyle#2$}}%
  {\colorbox{#1}{$\scriptstyle#2$}}%
  {\colorbox{#1}{$\scriptscriptstyle#2$}}}%

% header/footer formatting
\pagestyle{fancy}
\fancyhead{}
\fancyfoot{}
\fancyhead[L]{MAA6617 Analysis}
\fancyhead[C]{Homework 4}
\fancyhead[R]{Sai Sivakumar}
\fancyfoot[R]{\thepage}
\renewcommand{\headrulewidth}{1pt}

% paragraph indentation/spacing
\setlength{\parindent}{0cm}
\setlength{\parskip}{10pt}
\renewcommand{\baselinestretch}{1.25}

% extra commands defined here
\newcommand{\br}[1]{\left(#1\right)}
\newcommand{\sbr}[1]{\left[#1\right]}
\newcommand{\cbr}[1]{\left\{#1\right\}}

\newcommand{\dprime}{\prime\prime}

% bracket notation for inner product
\usepackage{mathtools}

\DeclarePairedDelimiterX{\abr}[1]{\langle}{\rangle}{#1}

\DeclareMathOperator{\Span}{span}
\DeclareMathOperator{\im}{im}
\DeclareMathOperator{\dist}{dist}

% set page count index to begin from 1
\setcounter{page}{1}

\begin{document}
\begin{enumerate}
    \item (24.4) Suppose $f\in L^{p_0}\cap L^\infty$ for some $p_0<\infty$. Prove $f\in L^p$ for all $p_0\leq p\leq \infty$, and $\lim_{p\to\infty}\norm{f}_p = \norm{f}_\infty$. \begin{proof}
      Let $f\in L^{p_0}\cap L^\infty$ for some $p_0<\infty$. Note $(\int\abs{f}^{p_0})^{1/p_0}<\infty$ implies $\int\abs{f}^{p_0}<\infty$; we have $\abs{f(x)}\leq \norm{f}_\infty< \infty$ for a.e. $x$. Then \[\int \abs{f}^p = \int\abs{f}^{p_0}\abs{f}^{p-p_0}\leq \norm{f}_{\infty}^{p-p_0}\int \abs{f}^{p_0} <\infty\] so that $(\int \abs{f}^p)^{1/p}<\infty$ as needed.

      Let $\varepsilon>0$ be given (and taken smaller than $\norm{f}_\infty$). Let $E = \cbr{\abs{f}\geq \norm{f}_\infty - \varepsilon}$. By definition of $\norm{\cdot}_\infty$ we must have that $\mu(E)>0$, but $\mu(E)<\infty$ also since $f$ is also in $L^p$ for $p\geq p_0$. Then for any $p>p_0$ we have \[\br{\int\abs{f}^p}^{\frac{1}{p}}\geq \br{\int_E\abs{f}^p}^{\frac{1}{p}}\geq \br{\int_E(\norm{f}_\infty-\varepsilon)^p}^{\frac{1}{p}} = (\norm{f}_\infty-\varepsilon)\mu(E)^{\frac{1}{p}}.\] By taking $p\to\infty$ in the above inequality, we obtain $\liminf_{p\to\infty} \norm{f}_p \geq \norm{f}_\infty-\varepsilon$. As this is true for all $\varepsilon$, we have $\liminf_{p\to\infty} \norm{f}_p \geq \norm{f}_\infty$.

      We also have for $p$ large enough (by the first estimate in this proof) that \[\br{\int \abs{f}^p}^{\frac{1}{p}} \leq \norm{f}_{\infty}^{1-\frac{p_0}{p}}\br{\int \abs{f}^{p_0}}^{\frac{1}{p}};\] by taking $p\to\infty$ we obtain that $\limsup_{p\to\infty}\norm{f}_p \leq \norm{f}_{\infty}$. It follows that $\lim_{p\to\infty}\norm{f}_p = \norm{f}_\infty$.
    \end{proof}
    \item (24.6) Suppose $p_0<p<p_1$ and $f\in L^{p_0}\cap L^{p_1}$. Prove $f\in L^p$ and $\norm{f}_p\leq \norm{f}_{p_0}^{1-\theta}\norm{f}_{p_1}^\theta$, where $0<\theta<1$ is chosen so that $\frac{1}{p} = \frac{1-\theta}{p_0}+\frac{\theta}{p_1}$. When does equality hold? \begin{proof}
      In the case that $p_1=\infty$, by the previous problem we have $f\in L^p$. We have by a previous estimate that \[\norm{f}_p\leq \norm{f}_{\infty}^{1-\frac{p_0}{p}}\br{\int \abs{f}^{p_0}}^{\frac{p_0}{p_0p}} = \norm{f}_\infty^\theta\norm{f}_{p_0}^{1-\theta},\] where $0<\theta<1$ is chosen so that $\frac{1}{p} = \frac{1-\theta}{p_0}$.

      For $p_1$ finite and $0<\theta<1$ chosen so that $\frac{1}{p} = \frac{1-\theta}{p_0}+\frac{\theta}{p_1}$ observe that $\frac{1-\theta}{p_0}<\frac{1}{p}$ implies $\frac{p_0}{(1-\theta)p}>1$ and similarly we have that $\frac{p_1}{\theta p}>1$. Then $\abs{f}^{(1-\theta)p}\in L^{p_0/(1-\theta)p}$ since $f\in L^{p_0}$ and $\abs{f}^{\theta p}\in L^{p_1/\theta p}$ since $f\in L^{p_1}$: \[\int(\abs{f}^{(1-\theta)p})^{p_0/(1-\theta)p} = \int\abs{f}^{p_0}<\infty \quad\text{and}\quad \int(\abs{f}^{\theta p})^{p_1/\theta p} = \int\abs{f}^{p_1}<\infty.\] Then we use H\"older's inequality ($1 = \frac{1}{p_0/(1-\theta)p} + \frac{1}{p_1/\theta p}$) to obtain \[\norm{f}_p^p = \int\abs{f}^p = \int\abs{f}^{(1-\theta)p}\abs{f}^{\theta p}\leq \norm{\abs{f}^{(1-\theta)p}}_{p_0/(1-\theta)p} \norm{\abs{f}^{\theta p}}_{p_1/\theta p}<\infty,\] which implies $f\in L^p$. In the above estimate take $p$-th roots to obtain \[\norm{f}_p \leq \br{\int\abs{f}^{p_0}}^{\frac{1-\theta}{p_0}}\br{\int\abs{f}^{p_1}}^{\frac{\theta}{p_1}} =\norm{f}_{p_0}^{1-\theta}\norm{f}_{p_1}^\theta\] as needed.
    \end{proof}
    \item (24.12) Prove, if $f\in L^p$ then \[\lim_{t\to 0}t^p\lambda_f(t) = \lim_{t\to\infty}t^p\lambda_f(t) = 0.\] (One way to proceed is to first suppose $f$ is a simple function. Another is to consider the integrals $\int_{\frac{s}{2}}^s t^{p-1}\lambda_f(t)\dd{t}$.) \begin{proof}
      The result is true when $f = \sum_{j=1}^n c_j\mathbf{1}_{E_j}$. We have for $0\leq t< \min\{c_j\}$ that $\lambda_f(t)$ is equal to the sum of the measures of the $E_j$. So as $t$ tends to zero the quantity $t^p\lambda_f(t)$ tends to zero also. When $\max\{c_j\}<t$ we have that $\lambda_f(t) = 0$, so $\lim_{t\to\infty}t^p\lambda_f(t) = 0$ also.
      
      When $f\in L^p$ there exists a sequence of simple functions $s_k$ increasing pointwise a.e. to $f$ converging in the $p$-norm to $f$. Let $\varepsilon>0$ (and strictly smaller than $1$) be given. Then choose $k$ large enough so that $\norm{f-s_k}_p\leq \varepsilon$. Then by subadditivity we have $t^p\lambda_f(t) = t^p\lambda_{f-s_k+s_k}(t)\leq t^p\lambda_{f-s_k}(t/2)+ t^p\lambda_{s_k}(t/2)$, from which using Chebyshev's inequality we have \[t^p\lambda_f(t) - t^p\lambda_{s_k}(t/2)\leq t^p\lambda_{f-s_k}(t/2)\leq 2^p\int\abs{f-s_k}^p\leq 2^p\varepsilon^p.\] Then we may take limits as $t$ goes to $0$ or $\infty$ and the inequality is preserved. It follows that $\lim_{t\to 0}t^p\lambda_f(t) = \lim_{t\to\infty}t^p\lambda_f(t) = 0$ since $\lim_{t\to 0}2^p(t/2)^p\lambda_{s_k}(t/2) = \lim_{t\to\infty}2^p(t/2)^p\lambda_{s_k}(t/2) = 0$ (since $s_k$ is simple).
    \end{proof}
\end{enumerate}
\end{document}