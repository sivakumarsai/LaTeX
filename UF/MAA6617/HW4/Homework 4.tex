\documentclass[11pt]{article}

% packages
\usepackage{physics}
% margin spacing
\usepackage[top=1in, bottom=1in, left=0.5in, right=0.5in]{geometry}
\usepackage{hanging}
\usepackage{amsfonts, amsmath, amssymb, amsthm}
\usepackage{systeme}
\usepackage[none]{hyphenat}
\usepackage{fancyhdr}
\usepackage[nottoc, notlot, notlof]{tocbibind}
\usepackage{graphicx}
\graphicspath{{./images/}}
\usepackage{float}
\usepackage{siunitx}
\usepackage{esint}
\usepackage{cancel}
\usepackage{enumitem}
\usepackage{mathrsfs}

% colors
\usepackage{xcolor}
\definecolor{p}{HTML}{FFDDDD}
\definecolor{g}{HTML}{D9FFDF}
\definecolor{y}{HTML}{FFFFCF}
\definecolor{b}{HTML}{D9FFFF}
\definecolor{o}{HTML}{FADECB}
%\definecolor{}{HTML}{}

% \highlight[<color>]{<stuff>}
\newcommand{\highlight}[2][p]{\mathchoice%
  {\colorbox{#1}{$\displaystyle#2$}}%
  {\colorbox{#1}{$\textstyle#2$}}%
  {\colorbox{#1}{$\scriptstyle#2$}}%
  {\colorbox{#1}{$\scriptscriptstyle#2$}}}%

% header/footer formatting
\pagestyle{fancy}
\fancyhead{}
\fancyfoot{}
\fancyhead[L]{MAA6617 Analysis}
\fancyhead[C]{Homework 4}
\fancyhead[R]{Sai Sivakumar}
\fancyfoot[R]{\thepage}
\renewcommand{\headrulewidth}{1pt}

% paragraph indentation/spacing
\setlength{\parindent}{0cm}
\setlength{\parskip}{10pt}
\renewcommand{\baselinestretch}{1.25}

% extra commands defined here
\newcommand{\br}[1]{\left(#1\right)}
\newcommand{\sbr}[1]{\left[#1\right]}
\newcommand{\cbr}[1]{\left\{#1\right\}}

\newcommand{\dprime}{\prime\prime}

% bracket notation for inner product
\usepackage{mathtools}

\DeclarePairedDelimiterX{\abr}[1]{\langle}{\rangle}{#1}

\DeclareMathOperator{\Span}{span}
\DeclareMathOperator{\im}{im}
\DeclareMathOperator{\dist}{dist}

% set page count index to begin from 1
\setcounter{page}{1}

\begin{document}
\begin{enumerate}
    \item (24.4) Suppose $f\in L^{p_0}\cap L^\infty$ for some $p_0<\infty$. Prove $f\in L^p$ for all $p_0\leq p\leq \infty$, and $\lim_{p\to\infty}\norm{f}_p = \norm{f}_\infty$.
    \item (24.6) Suppose $p_0<p<p_1$ and $f\in L^{p_0}\cap L^{p_1}$. Prove $f\in L^p$ and $\norm{f}_p\leq \norm{f}_{p_0}^{1-\theta}\norm{f}_{p_1}^\theta$, where $0<\theta<1$ is chosen so that $\frac{1}{p} = \frac{1-\theta}{p_0}+\frac{\theta}{p_1}$. When does equality hold?
    \item (24.12) Prove, if $f\in L^p$ then \[\lim_{t\to 0}t^p\lambda_f(t) = \lim_{t\to\infty}t^p\lambda_f(t) = 0.\] (One way to proceed is to first suppose $f$ is a simple function. Another is to consider the integrals $\int_{\frac{s}{2}}^s t^{p-1}\lambda_f(t)\dd{t}$.)
\end{enumerate}
\end{document}