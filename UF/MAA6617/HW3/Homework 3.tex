\documentclass[11pt]{article}

% packages
\usepackage{physics}
% margin spacing
\usepackage[top=1in, bottom=1in, left=0.5in, right=0.5in]{geometry}
\usepackage{hanging}
\usepackage{amsfonts, amsmath, amssymb, amsthm}
\usepackage{systeme}
\usepackage[none]{hyphenat}
\usepackage{fancyhdr}
\usepackage[nottoc, notlot, notlof]{tocbibind}
\usepackage{graphicx}
\graphicspath{{./images/}}
\usepackage{float}
\usepackage{siunitx}
\usepackage{esint}
\usepackage{cancel}
\usepackage{enumitem}
\usepackage{mathrsfs}

% colors
\usepackage{xcolor}
\definecolor{p}{HTML}{FFDDDD}
\definecolor{g}{HTML}{D9FFDF}
\definecolor{y}{HTML}{FFFFCF}
\definecolor{b}{HTML}{D9FFFF}
\definecolor{o}{HTML}{FADECB}
%\definecolor{}{HTML}{}

% \highlight[<color>]{<stuff>}
\newcommand{\highlight}[2][p]{\mathchoice%
  {\colorbox{#1}{$\displaystyle#2$}}%
  {\colorbox{#1}{$\textstyle#2$}}%
  {\colorbox{#1}{$\scriptstyle#2$}}%
  {\colorbox{#1}{$\scriptscriptstyle#2$}}}%

% header/footer formatting
\pagestyle{fancy}
\fancyhead{}
\fancyfoot{}
\fancyhead[L]{MAA6617 Analysis}
\fancyhead[C]{Homework 3}
\fancyhead[R]{Sai Sivakumar}
\fancyfoot[R]{\thepage}
\renewcommand{\headrulewidth}{1pt}

% paragraph indentation/spacing
\setlength{\parindent}{0cm}
\setlength{\parskip}{10pt}
\renewcommand{\baselinestretch}{1.25}

% extra commands defined here
\newcommand{\br}[1]{\left(#1\right)}
\newcommand{\sbr}[1]{\left[#1\right]}
\newcommand{\cbr}[1]{\left\{#1\right\}}

\newcommand{\dprime}{\prime\prime}

% bracket notation for inner product
\usepackage{mathtools}

\DeclarePairedDelimiterX{\abr}[1]{\langle}{\rangle}{#1}

\DeclareMathOperator{\Span}{span}
\DeclareMathOperator{\im}{im}
\DeclareMathOperator{\dist}{dist}

% set page count index to begin from 1
\setcounter{page}{1}

\begin{document}
\begin{enumerate}
    \item (23.2) (Adjoint operators) Let $H$ be a Hilbert space and $T\colon H\to H$ a bounded linear operator. \begin{enumerate}
      \item Prove there is a unique bounded operator $T^\ast\colon H\to H$ satisfying $\abr{Tg,h}= \abr{g,T^\ast h}$ for all $g,h\in H$, and $\norm{T^\ast} = \norm{T}$. \begin{proof}
        We check that if such an adjoint map exists it is unique. If $S\colon H\to H$ is linear with $\abr{Tf,h} = \abr{f,Sh}$ for any $f,h\in H$ then since $\abr{f,Sh} = \abr{f,T^\ast h}$ for all $f,h\in H$ we have $\abr{f,(S-T^\ast)h} = 0$ for all $f,h\in H$. It follows that $(S-T^\ast)h = 0$ for all $h\in H$ (by taking $f = (S-T^\ast)h$ above). Hence $S = T^\ast$ as needed.
        
        For $h\in H$, let $\lambda_h\colon H\to \mathbb{C}$ be defined by $\lambda_h(f) = \abr{Tf,h}$. The map $\lambda_h$ is linear since inner products are bilinear (we have fixed the second component) and $T$ is linear, and is bounded since $T$ is bounded (use Cauchy-Schwarz inequality). By the Riesz representation theorem there is a unique vector which we will denote by $T^\ast h$ such that $\lambda_h(f) = \abr{Tf,h} = \abr{f,T^\ast h}$ for all $f\in H$. Define $T^\ast\colon H\to H$ in this manner.

        The vector $T^\ast(cg+h)$ is the unique vector satisfying $\lambda_{cg+h}(f) = \abr{Tf,cg+h} = \abr{f,T^\ast(cg+h)}$ for all $f\in H$. But by bilinearity of the inner product $\lambda_{cg+h} = \overline{c}\lambda_g+\lambda_h$ so that the equality $\abr{f,T^\ast(cg+h)} = \abr{f,cT^\ast g+T^\ast h}$ is also satisfied for all $f\in H$. By uniqueness (in the Riesz representation theorem) we must have $T^\ast(cg+h)=cT^\ast g+T^\ast h$ as needed.
        
        For nonzero $x\in H$ we have $\norm{T^\ast x}^2 = \abr{T^\ast x, T^\ast x} = \abr{TT^\ast x,x}\leq \norm{T}\norm{T^\ast x}\norm{x}$; i.e., $\norm{T^\ast x}\leq \norm{T}\norm{x}$ so that $\norm{T^\ast}\leq \norm{T}$. Similarly for nonzero $x\in H$ we have $\norm{Tx}^2 = \abr{Tx,Tx} = \abr{x,T^\ast Tx}\leq \norm{x}\norm{T^\ast}\norm{Tx}$; i.e., $\norm{T x}\leq \norm{T^\ast}\norm{x}$ so that $\norm{T}\leq \norm{T^\ast}$. Hence $\norm{T^\ast} = \norm{T}$.
      \end{proof}
      \item Prove, if $S,T\in B(H)$, then $(aS+T)^\ast = \overline{a}S^\ast + T^\ast$ for all $a\in K$, and that $T^{\ast\ast} = T$. \begin{proof}
        As before define for any $P\in B(H)$ and $h\in H$ the functional $\lambda_{P,h}\colon H\to \mathbb{C}$ by $\lambda_{P,h}(f) = \abr{Pf, h}$. Let $S,T\in B(H)$ and $a\in \mathbb{C}$ as above.

        The vector $S^\ast h$ is the unique vector satisfying $\lambda_{S,h}(f) = \abr{Sf,h} = \abr{f,S^\ast h}$ for all $f\in H$. Similarly, the vector $T^\ast h$ is the unique vector satisfying $\lambda_{T,h}(f) = \abr{Tf,h} = \abr{f,T^\ast h}$ for all $f\in H$. Then for all $f\in H$ and $a\in\mathbb{C}$, we have $\abr{(aS+T)f,h} = a\abr{f,S^\ast h} + \abr{f,T^\ast h} = \abr{f,\overline{a}S^\ast h + T^\ast h}$ for all $f\in H$. It follows by uniqueness of the adjoint that $(aS+T)^\ast = \overline{a}S^\ast + T^\ast$ for all $a\in \mathbb{C}$ and $S,T\in B(H)$.

        For all $f,h\in H$ we have $\abr{f,Th} = \overline{\abr{Th,f}} = \overline{\abr{h,T^\ast f}} = \abr{T^\ast f, h} = \abr{f,T^{\ast\ast}h}$. By uniqueness it follows that $T^{\ast\ast} = T$.
      \end{proof}
      \item Prove $\norm{T^\ast T} = \norm{T}^2$. \begin{proof}
        For nonzero $h\in H$ we have $\norm{Th}^2 = \abr{Th,Th} = \abr{h,T^\ast T h}\leq \norm{h}^2\norm{T^\ast T}$. It follows that $\norm{T}^2\leq \norm{T^\ast T}$. In general we have $\norm{T^\ast T}\leq \norm{T^\ast}\norm{T} = \norm{T}^2$. Thus $\norm{T^\ast T} = \norm{T}^2$.
      \end{proof}
      \item Prove $\ker T$ is a closed subspace of $H$, $\overline{\im T} = (\ker T^\ast)^\perp$ and $\ker T^\ast = (\im T)^\perp$. \begin{proof}
        The preimage of $\cbr{0}\subseteq H$ under $T$ is the kernel of $T$, and since $T$ is continuous the preimage of a closed set is closed. The kernel of $T$ is already a subspace of $H$. So $\ker T$ is a closed subspace of $H$. (Similarly $\ker T^\ast$ is a closed subspace of $H$.)

        First we show that $\ker T^\ast = (\im T)^\perp$. Note also that the orthogonal complement of a subset is a closed subspace so $(\im T)^\perp$ is a closed subspace. Let $x\in \ker T^\ast$ and let $(x_n)$ be a subsequence from $\ker T^\ast$ converging to $x$. Then by continuity of the inner product and $T^\ast$ for any $h\in H$ we have that the sequence $(\abr{h,T^\ast x_n} = 0 = \abr{Th,x_n})$ converges to $\abr{Th,x}$ which must be zero since the sequence was a zero sequence. Thus $x$ is orthogonal to any vector in the image of $T$, that is, $x\in (\im T)^\perp$. Similarly, let $y\in (\im T)^\perp$ and let $(y_n)$ be a sequence from $(\im T)^\perp$ converging to $y$. Then for any $h\in H$, the sequence $(\abr{Th, y_n} = 0 = \abr{h,T^\ast y_n})$ converges to $\abr{h,T^\ast y} = 0$ by continuity of the inner product and $T^\ast$. Since $\abr{h,T^\ast y} = 0$ holds for all $h\in H$, we must have that $T^\ast y = 0$ (e.g. take $h = T^\ast y$). Hence $y\in \ker T^\ast$, and so $\ker T^\ast = (\im T)^\perp$. 

        We have that $(\ker T^\ast)^\perp = ((\im T)^\perp)^\perp$, which is the smallest closed subspace of $H$ containing $\im T$; that is, $\overline{\im T}$. 
      \end{proof} 
    \end{enumerate}
    \item (23.9) (Weak convergence) \begin{enumerate}
      \item Prove, if $(h_n)$ converges to $h$ in norm, then also $(h_n)$ converges to $h$ weakly. (Hint: Cauchy-Schwarz.) \begin{proof}
        It suffices to see that for any $g\in H$, $(\abr{h_n-h,g})$ converges to $0$ in $\mathbb{C}$ (translation in $\mathbb{C}$ is a homeomorphism). Indeed, with $g$ fixed we have $\abs{\abr{h_n-h,g}}\leq \norm{h_n-h}\norm{g}$ so that as $n$ tends to infinity, $\abr{h_n-h,g}$ must tend to $0$. Thus norm convergence implies weak convergence.
      \end{proof}
      \item Prove, if $H$ is infinite-dimensional, and $(e_n)$ is an orthonormal sequence in $H$, then $e_n\to 0$ weakly, but $e_n\not\to 0$ in norm. (Thus weak convergence does not imply norm convergence.) \begin{proof}
        Let $g\in H$ be any vector. Then Bessel's inequality yields $\sum_{n=1}^\infty\abs{\abr{g,e_n}}^2\leq \norm{g}^2$. For the series $\sum_{n=1}^\infty\abs{\abr{g,e_n}}^2$ to converge we must have the terms tend to zero; that is $(\abs{\abr{g,e_n}}^2)$ converges to $0$, from which it follows that $(\abr{g,e_n})$ converges to $0$ for any $g\in H$.

        However, $(e_n)$ does not converge to $0$ in norm as it is not even Cauchy: for any $i,j$ we have by the Pythagorean theorem that $\norm{e_i-e_j}^2 = \norm{e_i}^2 + \norm{e_j}^2 = 2$ so that $(e_n)$ could not be Cauchy and hence could not converge. Thus weak convergence does not imply norm convergence. 
      \end{proof}
      \item Prove $(h_n)$ converges to $h$ in norm if and only if $(h_n)$ converges to $h$ weakly and $\norm{h_n}\to \norm{h}$. \begin{proof}
        Certainly if $(h_n)$ converges to $h$ in norm then $(h_n)$ converges to $h$ weakly. By the reverse triangle inequality we have also that $\abs{\norm{h_n}-\norm{h}}\leq \norm{h_n-h}$, which tends to $0$ by assumption so that $(\norm{h_n})$ converges to $\norm{h}$.

        Conversely, let $\varepsilon$ be given. Then choose $N$ large enough so that $\abs{\abr{h_n-h,h}}<\varepsilon$ (so $-\Re\abr{h_n,h}<\varepsilon -\Re\abr{h,h}$) and $\abs{\norm{h_n}-\norm{h}}\leq \varepsilon$. Then for $n\geq N$, we have $\norm{h_n-h}^2 = \abr{h_n-h,h_n-h} = \norm{h_n}^2 + \norm{h}^2 - 2\Re\abr{h_n,h}\leq \norm{h_n}^2 + \norm{h}^2 - 2\Re\abr{h,h} + \varepsilon = \norm{h_n}^2 - \norm{h}^2 + \varepsilon\leq 2\varepsilon$. Hence $\norm{h_n-h}$ tends to zero as needed.
      \end{proof}
      \item Prove if $(h_n)$ converges to $h$ weakly, then $\norm{h}\leq \liminf\norm{h_n}$. \begin{proof}
        By Cauchy-Schwarz we have $\abs{\abr{h_n,h/\norm{h}}}\leq \norm{h_n}$. By taking $\liminf$ on both sides (as the inequality is preserved) and using the fact that $\lim \abs{\abr{h_n,h/\norm{h}}}$ exists and is equal to $\abr{h,h/\norm{h}} = \norm{h}$ (since norms are continuous and $h_n$ converges weakly to $h$) we have $\norm{h} = \lim \abs{\abr{h_n,h/\norm{h}}} = \liminf \abs{\abr{h_n,h/\norm{h}}}\leq \liminf\norm{h_n}$.
      \end{proof}
    \end{enumerate}
    \item (23.10) Suppose $H$ is countably infinite-dimensional (separable) Hilbert space. Prove, if $h\in H$ and $\norm{h}\leq 1$, then there is a sequence $(h_n)$ in $H$ with $\norm{h_n}=1$ for all $n$, and $(h_n)$ converges to $h$ weakly, but $(h_n)$ does not converge to $h$ strongly. \begin{proof}
      Fix a countable orthonormal basis $\cbr{e_i}_{i=1}^\infty$ for $H$ and let $h\in H$ with $\norm{h}\leq 1$. Note that $h = \sum_{i=1}^\infty \abr{h,e_i}e_i$.
      
      Let $h_n = \sum_{i\neq n}\abr{h,e_i}e_i + \sqrt{1-\sum_{i\neq n}\abs{\abr{h,e_i}}^2}\,e_n$; indeed, the quantity under the square root is nonnegative since $\norm{h}^2 = \sum_{i=1}^\infty \abs{\abr{h,e_i}}^2\leq 1$ so that by omitting the $n$-th term in the sum we preserve the inequality. We have by the Pythagorean theorem that $\norm{h_n}^2 = \sum_{i\neq n}\abs{\abr{h,e_i}} + 1-\sum_{i\neq n}\abs{\abr{h,e_i}}^2 = 1$ so that $\norm{h_n} = 1$ for all $n$ (this also tells us that the defining sum for $h_n$ converges to begin with).

      For any $g\in H$, $\abr{h_n,g} = \sum_{i=1}^\infty\abr{h_n,e_i}\abr{e_i,g}$. This sum agrees with $\abr{h,g}$ for all $i$ except for $i=n$. Specifically we have $\abs{\abr{h_n-h,g}}\leq \abs{\sqrt{1-\sum_{i\neq n}\abs{\abr{h,e_i}}^2}-\abr{h,e_n}}\abs{\abr{e_n,g}}\leq C\abs{\abr{e_n,g}}$, since $\sqrt{1-\sum_{i\neq n}\abs{\abr{h,e_i}}^2} = \sqrt{1-\norm{h}^2+\abs{\abr{h,e_n}}^2} = \abs{\abr{h,e_n}} + O(1-\norm{h}^2)\leq \abs{\abr{h,e_n}} + C$ (from a Taylor series; in fact $C$ can be taken to be $1$). By Bessel's inequality we must have $\abs{\abr{e_n,g}}$ tend to $0$ so that we have weak convergence as needed.

      If $\norm{h}<1$ then it is clear that we do not have norm convergence since $\norm{h_n}=1$ and we proved in the previous problem that it is necessary in order for weak convergence to be promoted to norm convergence. In the case that $\norm{h} = 1$ we should be able to repeat a similar construction to obtain $h_n$ with $\norm{h_n}=1+\delta$ for some fixed $\delta>0$ which similarly will converge to $h$ weakly but not in norm [we could try taking $h_n = \sum_{i\neq n}\abr{h,e_i}e_i + \sqrt{1+\delta-\sum_{i\neq n}\abs{\abr{h,e_i}}^2}\,e_n$], but I am not super sure. 
    \end{proof}
\end{enumerate}
\end{document}