\documentclass[11pt]{article}

% packages
\usepackage{physics}
% margin spacing
\usepackage[top=1in, bottom=1in, left=0.5in, right=0.5in]{geometry}
\usepackage{hanging}
\usepackage{amsfonts, amsmath, amssymb, amsthm}
\usepackage{systeme}
\usepackage[none]{hyphenat}
\usepackage{fancyhdr}
\usepackage[nottoc, notlot, notlof]{tocbibind}
\usepackage{graphicx}
\graphicspath{{./images/}}
\usepackage{float}
\usepackage{siunitx}
\usepackage{esint}
\usepackage{cancel}
\usepackage{enumitem}
\usepackage{mathrsfs}

% colors
\usepackage{xcolor}
\definecolor{p}{HTML}{FFDDDD}
\definecolor{g}{HTML}{D9FFDF}
\definecolor{y}{HTML}{FFFFCF}
\definecolor{b}{HTML}{D9FFFF}
\definecolor{o}{HTML}{FADECB}
%\definecolor{}{HTML}{}

% \highlight[<color>]{<stuff>}
\newcommand{\highlight}[2][p]{\mathchoice%
  {\colorbox{#1}{$\displaystyle#2$}}%
  {\colorbox{#1}{$\textstyle#2$}}%
  {\colorbox{#1}{$\scriptstyle#2$}}%
  {\colorbox{#1}{$\scriptscriptstyle#2$}}}%

% header/footer formatting
\pagestyle{fancy}
\fancyhead{}
\fancyfoot{}
\fancyhead[L]{MAA6617 Analysis}
\fancyhead[C]{Homework 3}
\fancyhead[R]{Sai Sivakumar}
\fancyfoot[R]{\thepage}
\renewcommand{\headrulewidth}{1pt}

% paragraph indentation/spacing
\setlength{\parindent}{0cm}
\setlength{\parskip}{10pt}
\renewcommand{\baselinestretch}{1.25}

% extra commands defined here
\newcommand{\br}[1]{\left(#1\right)}
\newcommand{\sbr}[1]{\left[#1\right]}
\newcommand{\cbr}[1]{\left\{#1\right\}}

\newcommand{\dprime}{\prime\prime}

% bracket notation for inner product
\usepackage{mathtools}

\DeclarePairedDelimiterX{\abr}[1]{\langle}{\rangle}{#1}

\DeclareMathOperator{\Span}{span}
\DeclareMathOperator{\im}{im}
\DeclareMathOperator{\dist}{dist}

% set page count index to begin from 1
\setcounter{page}{1}

\begin{document}
\begin{enumerate}
    \item (23.2) (Adjoint operators) Let $H$ be a Hilbert space and $T\colon H\to H$ a bounded linear operator. \begin{enumerate}
      \item Prove there is a unique bounded operator $T^\ast\colon H\to H$ satisfying $\abr{Tg,h}= \abr{g,T^\ast h}$ for all $g,h\in H$, and $\norm{T^\ast} = \norm{T}$.
      \item Prove, if $S,T\in B(H)$, then $(aS+T)^\ast = \overline{a}S^\ast + T^\ast$ for all $a\in K$, and that $T^{\ast\ast} = T$.
      \item Prove $\norm{T^\ast T} = \norm{T}^2$.
      \item Prove $\ker T$ is a closed subspace of $H$, $\overline{\im T} = (\ker T)^\perp$ and $\ker T^\ast = (\im T)^\perp$.
    \end{enumerate}
    \item (23.9) (Weak convergence) \begin{enumerate}
      \item Prove, if $(h_n)$ converges to $h$ in norm, then also $(h_n)$ converges to $h$ weakly. (Hint: Cauchy-Schwarz.)
      \item Prove, if $H$ is infinite-dimensional, and $(e_n)$ is an orthonormal sequence in $H$, then $e_n\to 0$ weakly, but $e_n\not\to 0$ in norm. (Thus weak convergence does not imply norm convergence.)
      \item Prove $(h_n)$ converges to $h$ in norm if and only if $(h_n)$ converges to $h$ weakly and $\norm{h_n}\to \norm{h}$.
      \item Prove if $(h_n)$ converges to $h$ weakly, then $\norm{h}\leq \liminf\norm{h_n}$.
    \end{enumerate}
    \item (23.10) Suppose $H$ is countably infinite-dimensional (separable Hilbert space). Prove, if $h\in H$ and $\norm{h}\leq 1$, then there is a sequence $(h_n)$ in $H$ with $\norm{h_n}=1$ for all $n$, and $(h_n)$ converges to $h$ weakly, but $(h_n)$ does not converge to $h$ strongly.
\end{enumerate}
\end{document}