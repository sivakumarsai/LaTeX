\documentclass[mathserif
, handout
]{beamer}

\usepackage{amsfonts, amsmath, amssymb, amsthm}
\usepackage{physics}
\usepackage{hyperref}
\usepackage{stmaryrd}
\usepackage{tikz-cd}
\usepackage{mathtools}

% commands
\DeclareMathOperator{\Circ}{circ}
\DeclareMathOperator{\diag}{diag}
\DeclareMathOperator{\rev}{rev}
\DeclareMathOperator{\GL}{GL}
\DeclareMathOperator{\SL}{SL}
\DeclareMathOperator{\Char}{char}
\DeclareMathOperator{\Span}{span}
\newcommand{\PP}{\mathbb P}
\renewcommand{\AA}{\mathbb A}
\newcommand{\ZZ}{\mathbb Z}
\newcommand{\FF}{\mathbb F}
\newcommand{\QQ}{\mathbb Q}
\newcommand{\NN}{\mathbb N}
\newcommand{\CC}{\mathbb C}
\newcommand{\RR}{\mathbb R}
\DeclareMathOperator{\Ker}{Ker}
\DeclareMathOperator{\image}{Im}

\usetheme{Warsaw}

% colors
\definecolor{gold60}{RGB}{166, 217, 247}
\definecolor{palettemed}{RGB}{178, 222, 247}
\definecolor{palettelight}{RGB}{183,201,255}
\definecolor{darkgold60}{RGB}{255,250,160}
\usecolortheme[named=gold60]{structure}
\setbeamercolor{frametitle}{fg=black!85}
\setbeamercolor{title}{bg=gold60,fg=black!85}
\setbeamercolor{block title}{fg=black!85}
\setbeamercolor{palette quaternary}{bg=palettemed}

% page numbers
\setbeamertemplate{footline}[frame number]

% paragraph indentation/spacing
\setlength{\parindent}{0cm}
\setlength{\parskip}{5pt}
\renewcommand{\baselinestretch}{1.25}

\title
{\color{black!85}{Free and Virtual Resolutions}}

\subtitle{Undergraduate Mathematics Research Symposium 2024}

\author[Sai Sivakumar]
{Sai Sivakumar}

\institute[Georgia Tech]{\color{blue!40!palettelight}{University of Florida}}
\date{26 April 2024}

\begin{document}

% title page.
\frame{\titlepage}

\begin{frame}{}{}
{\color{blue!40!palettelight}Thank you for inviting me to speak!}
\vspace{0.75pc}
\hrule
\vspace{0.75pc}
This are joint works with Dr. Michael Perlman, Bjorn Cattell-Ravdal, Erin Delargy, Akash Ganguly, and Sean Guan,
then separately with Dr. Christine Berkesch, Isidora Bailly-Hall, Karina Dovgodko, Sean Guan, and Jishi Sun.

\vspace{0.75pc}
This project arose from the University of Minnesota, Twin Cities Combinatorics and Algebra REU 2023. 
\end{frame}

\begin{frame}{Invariant theory}
    Classifying invariants of $G\curvearrowright \mathbb C[t_1,\dots,t_n]$.

    Example: $G = C_3 = \{1,g,g^2\}$ acting on $R = \mathbb C[u,v]$ by \[g\cdot u = e^{2\pi i /3}u,~ g\cdot v = e^{2\pi i /3}v,~ \text{and} ~ g\cdot f(u,v) = f(g\cdot u,g\cdot v).\]
    
    Invariants are elements of $\mathbb C[u^3,u^2v,uv^2,v^3]$.
\end{frame}

\begin{frame}
    Probe $R$ with $S = \mathbb C[x,y,z,w]$ by the surjection $R\leftarrow S$, given by
    \[\begin{cases}
        u^3 \mapsfrom x \\
        u^2v \mapsfrom y \\
        uv^2 \mapsfrom z \\
        v^3 \mapsfrom w .
    \end{cases}\]
    Study $R$ using $S$.
\end{frame}

\begin{frame}
    First syzygies:
% https://q.uiver.app/#q=WzAsMyxbMiwwLCJTXjEiXSxbMCwwLCJSIl0sWzUsMCwiU14zIl0sWzAsMSwiXFxzdWJzdGFja3t1XjMgXFxtYXBzZnJvbSB4LCB+dV4ydiBcXG1hcHNmcm9tIHlcXFxcIHV2XjIgXFxtYXBzZnJvbSB6IH52XjMgXFxtYXBzZnJvbSB3fSJdLFsyLDAsIihcXCFcXGJlZ2lue3NtYWxsbWF0cml4fXh6LXleMiAmIHl6LXh3ICYgeXctel4yXFxlbmR7c21hbGxtYXRyaXh9XFwhKSIsMl1d
\[\begin{tikzcd}[ampersand replacement=\&]
	R \&\& {S^1} \&\&\& {S^3}
	\arrow["{\substack{u^3 \mapsfrom x, ~u^2v \mapsfrom y\\ uv^2 \mapsfrom z ~v^3 \mapsfrom w}}", from=1-3, to=1-1]
	\arrow["{(\!\begin{smallmatrix}xz-y^2 & yz-xw & yw-z^2\end{smallmatrix}\!)}"', from=1-6, to=1-3]
\end{tikzcd}\]

Find relations between $x,y,z,w$ with respect to the projection $R\leftarrow S$.
\end{frame}

\begin{frame}
    Second syzygies:
    % https://q.uiver.app/#q=WzAsNSxbNiwwLCJTXjIiXSxbNSwwLCJTXjMiXSxbMiwwLCJTXjEiXSxbMCwwLCJSIl0sWzcsMCwiMCJdLFswLDEsIlxcQmlnKFxcIVxcYmVnaW57c21hbGxtYXRyaXh9dyAmIHogXFxcXCB6ICYgeSBcXFxcIHkgJiB4IFxcZW5ke3NtYWxsbWF0cml4fVxcIVxcQmlnKSIsMl0sWzEsMiwiKFxcIVxcYmVnaW57c21hbGxtYXRyaXh9eHoteV4yICYgeXoteHcgJiB5dy16XjJcXGVuZHtzbWFsbG1hdHJpeH1cXCEpIiwyXSxbMiwzLCJcXHN1YnN0YWNre3VeMyBcXG1hcHNmcm9tIHgsIH51XjJ2IFxcbWFwc2Zyb20geVxcXFwgdXZeMiBcXG1hcHNmcm9tIHogfnZeMyBcXG1hcHNmcm9tIHd9Il0sWzQsMF1d
\[\begin{tikzcd}[ampersand replacement=\&]
	R \&\& {S^1} \&\&\& {S^3} \& {S^2} \& 0
	\arrow["{\Big(\!\begin{smallmatrix}w & z \\ z & y \\ y & x \end{smallmatrix}\!\Big)}"', from=1-7, to=1-6]
	\arrow["{(\!\begin{smallmatrix}xz-y^2 & yz-xw & yw-z^2\end{smallmatrix}\!)}"', from=1-6, to=1-3]
	\arrow["{\substack{u^3 \mapsfrom x, ~u^2v \mapsfrom y\\ uv^2 \mapsfrom z ~v^3 \mapsfrom w}}", from=1-3, to=1-1]
	\arrow[from=1-8, to=1-7]
\end{tikzcd}\]

Find relations between relations. There are no further relations.

We can deduce properties of $R$ from the matrices.
\end{frame}

\begin{frame}
    Second syzygies:
    % https://q.uiver.app/#q=WzAsNSxbNCwwLCJTXjMiXSxbMSwwLCJTXjEiXSxbMCwwLCJcXHVuZGVyc2V0e1xcdGV4dHsoY29tcGxpY2F0ZWQpfX17Un0iXSxbNywwLCIwIl0sWzYsMCwiU14yIl0sWzAsMSwiKFxcIVxcYmVnaW57c21hbGxtYXRyaXh9XFx0ZXh0e2RlZ3JlZSB0d28gcG9seW5vbWlhbHN9XFxlbmR7c21hbGxtYXRyaXh9XFwhKSIsMl0sWzEsMl0sWzMsNF0sWzQsMCwiXFxCaWcoXFwhXFxiZWdpbntzbWFsbG1hdHJpeH0gXFx0ZXh0e2RlZ3JlZSBvbmV9XFxcXFxcdGV4dHtwb2x5bm9taWFsc30gXFxlbmR7c21hbGxtYXRyaXh9XFwhXFxCaWcpIiwyXV0=
\[\begin{tikzcd}[ampersand replacement=\&]
	{\underset{\text{(complicated)}}{R}} \& {S^1} \&\&\& {S^3} \&\& {S^2} \& 0
	\arrow["{(\!\begin{smallmatrix}\text{degree two polynomials}\end{smallmatrix}\!)}"', from=1-5, to=1-2]
	\arrow[from=1-2, to=1-1]
	\arrow[from=1-8, to=1-7]
	\arrow["{\Big(\!\begin{smallmatrix} \text{degree one}\\\text{polynomials} \end{smallmatrix}\!\Big)}"', from=1-7, to=1-5]
\end{tikzcd}\]

Find relations between relations. There are no further relations.

We can deduce properties of $R$ from the matrices.

The stuff to the right of $R$ is easier to investigate.
\end{frame}

\begin{frame}{Free resolutions}
    Let $S$ be a ring and $M$ an $S$-module.
    \begin{definition}
    A \textbf{free resolution} of the $S$-module $M$ is an exact sequence of free $S$-modules
    \[0 \leftarrow M \leftarrow \bigoplus_{d\in \mathbb{N}} S^{\beta_{0,d}} \leftarrow \bigoplus_{d\in \mathbb{N}} S^{\beta_{1,d}} \leftarrow \cdots \]

    A free resolution is \textbf{minimal} (MFR) if each free module has the smallest possible number of generators.
\end{definition}

\setbeamercolor{block title}{fg=black!85,bg=black!20!palettelight}\begin{block}{Fact (Hilbert's syzygy theorem)}
    If $S$ is a polynomial ring, then the minimal free resolution of $M$ has length at most the number of indeterminates of $S$.
\end{block}
\end{frame}

\begin{frame}{One project}
    Let $k$ be an algebraically closed field with $\Char{k}\neq 0$, and let $\GL_n(k)$ act on $\Span\{x_1,\dots,x_n\}$ in the ``natural way''.
    
    Then let $\GL_n(k)$ act on $S = k[x_1,\dots,x_n]$ by 
    \[A\cdot f(x_1,\dots,x_n) \coloneqq f(Ax_1,\dots,Ax_n).\]
    \setbeamercolor{block title}{fg=black!85,bg=black!20!palettelight}\begin{block}{Example}
        Let $S = k[x,y]$. Then \begin{multline*}
            \big(\!\begin{smallmatrix}
            1 & a \\ 0 & 1
        \end{smallmatrix}\!\big) \cdot xy^3 = Ax(Ay)^3 = x(ax+y)^3 \\ = a^3x^4 +  3 a^2 x^3 y + 3 a x^2 y^2 + xy^3 .
        \end{multline*}
    \end{block}
\end{frame}

\begin{frame}
    An ideal $I$ in $S = k[x_1,\dots,x_n]$ is called $\GL_n(k)$-stable (``stable'', also ``invariant'') if $Af\in I$ for all $A\in \GL_n(k)$, $f\in I$.

    We want to find free resolutions of $S/I$ for stable ideals $I$.
\end{frame}

\begin{frame}
    \begin{theorem}
        [C-D-G-G-P-S, 2023+]
        Let $S = k[x,y]$, where $\Char(k) = p$.
        There is a large class of ``simple'' stable ideals for which we can find the following:
        \begin{enumerate}
        \item The number of generators of $I$.
           \item The number of distinct degrees of syzygies of the minimal generators of $I$.
            \item The distinct degrees of syzygies of the minimal generators of $I$.
         
            \item The multiplicity of each degree of syzygy.
            
        \end{enumerate}
    \end{theorem}
    This information is enough to construct minimal free resolutions of these kinds of stable ideals.
\end{frame}

\begin{frame}

    Tools we used: 
    \begin{itemize}
    \item Arithmetic in base $p$
    \item Combinatorial arguments (?)
    \item some results from representation theory, also the Doty paper on submodules of $S$.
    \end{itemize}
\end{frame}

\begin{frame}{The other project}
    Let $X$ be a subset of $\PP^n\times \PP^m$. 
    The \textbf{Cox ring} of $\PP^n\times \PP^m$ is 
    \[
    S = k[x_0,\dots,x_n,y_0,\dots,y_m]
    \]
    and is $\ZZ^2$-graded, where \(\deg(x_i) = (1,0)\) and \(\deg(y_j) = (0,1)\).
    
    The set
\[I(X) = \{f\in S\mid f(x) = 0\text{ for all } x\in X\}\] is the bihomogeneous \textbf{defining ideal} of $X$.

We also have the \textbf{irrelevant ideal} 
\[B=\langle x_0,\dots,x_n\rangle \cap \langle y_0,\dots,y_m \rangle.\]
\end{frame}

\begin{frame}
    We want to find resolutions for the $S$-modules $S/I(X)$ for finite sets of points $X\subset \PP^n\times\PP^m$.

    Free resolutions are not quite good enough for this situation due to the multigrading. Need new technology.
\end{frame}

\begin{frame}{Virtual resolutions}
    \small
    \begin{definition}
        \textbf{Virtual resolutions} (VRs) are complexes of free $S$-modules which are not necessarily exact: 
        \begin{multline*}
        0 \longleftarrow S/I(X) \xleftarrow{\phi_0} \bigoplus_{\underline{d}\in \mathbb{N}^2} 
        S(-\underline{d})^{\beta_{0,\underline{d}}}   \xleftarrow{\phi_1} \bigoplus_{\underline{d}\in \mathbb{N}^2} S(-\underline{d})^{\beta_{1,\underline{d}}} \xleftarrow{\phi_2} \cdots 
        \end{multline*}
        The modules $\Ker(\phi_{i-1})/\image(\phi_i)$ are allowed to have support in the irrelevant ideal 
        \[B = \langle x_0,  \ldots, x_n \rangle \cap  \langle y_0, \ldots, y_m\rangle.\] 
    \end{definition}
        (1) Every MFR is a VR;

        (2) In $\PP^n\times \PP^m$, while MFRs have length bounded by $n+m+2$, VRs can have length bounded by $n+m$.

\end{frame}

\begin{frame}{Second Approach: Intersection with \(\langle \underline{x} \rangle^a\)}
    \begin{theorem} [Harada--Nowroozi--Van Tuyl 2022]
    Let $X$ be a finite set of points in $\mathbb{P}^1 \times \mathbb{P}^1$. Let t denote the number of unique first coordinates. Then for all $a \geq t- 1$, the MFR of $S/(I(X) \cap \langle x_0, x_1 \rangle ^a)$ is a VR of $S/I(X)$ of length two.
    
    \end{theorem}
    
    Our result: 
    \begin{theorem}[B-B-D-G-S-S 2023+]
    Let $X$ be a set of points in $\alert{\PP^n \times  \PP^m}$. Let $t$ denote the number of first coordinates. For all $a \ge t -1$, the MFR of $S / ( I(X) \cap \alert{\langle x_0, \ldots, x_n\rangle}^a)$ is a VR of $S/I(X)$ of length \alert{$n+m$}.
    \end{theorem}
    
 \end{frame}

 \begin{frame}
    Tools we used: 
\begin{itemize}
\item Auslander--Buchsbaum 
\item Primary decomposition 
\item Short exact sequences and additivity of the Hilbert Function 
\end{itemize}
 \end{frame}
\end{document}