\documentclass[mathserif
, handout
]{beamer}

\usepackage{amsfonts, amsmath, amssymb, amsthm}
\usepackage{mathtools}
\usepackage{physics}
\usepackage{hyperref}

% double hat macros
\makeatletter
\newcommand{\dwidehat}[1]{% 
\begingroup%
  \let\macc@kerna\z@%
  \let\macc@kernb\z@%
  \let\macc@nucleus\@empty%
  \widehat{\raisebox{.3ex}{\vphantom{\ensuremath{#1}}}\smash{\widehat{#1}}}%
\endgroup%
}
\makeatother

\makeatletter
\newcommand{\dhat}[1]{% 
\begingroup%
  \let\macc@kerna\z@%
  \let\macc@kernb\z@%
  \let\macc@nucleus\@empty%
  \hat{\raisebox{.3ex}{\vphantom{\ensuremath{#1}}}\smash{\hat{#1}}}%
\endgroup%
}
\makeatother

\usetheme{Warsaw}

% colors
\definecolor{gold60}{RGB}{245,213,128}
\definecolor{palettemed}{RGB}{142,139,118}
\definecolor{palettelight}{RGB}{214,219,212}
\definecolor{darkgold60}{RGB}{184,160,96}
\usecolortheme[named=gold60]{structure}
\setbeamercolor{frametitle}{fg=black!85}
\setbeamercolor{title}{bg=gold60,fg=black!85}
\setbeamercolor{block title}{fg=black!85}
\setbeamercolor{palette quaternary}{bg=palettemed}

% page numbers
\setbeamertemplate{footline}[frame number]

% paragraph indentation/spacing
\setlength{\parindent}{0cm}
\setlength{\parskip}{5pt}
\renewcommand{\baselinestretch}{1.25}

\title
{\textcolor{black!85}{Fourier analysis on LCA groups and Pontryagin Duality}}

\author[Sai Sivakumar]{Sai Sivakumar}
\date{17 April 2023}

\begin{document}

% title slide
\frame{\titlepage}

% motivation and history
\begin{frame}{Some motivation for this talk}
    Fourier transforms and series have been used in physics to solve partial differential equations like the heat and wave equation. % Sturm-Liouville operators, eigenfunction expansions, also note Poisson summation formula, Heisenberg uncertainty principle
    \pause

    The fast Fourier transform is an improvement ($O(N\log N)$) to the usual discrete Fourier transform of a sequence ($O(N^2)$), and is used in signal processing. % digital recording/encoding, large-integer multiplication, solving recurrence relations
    \pause

    A proof of Dirichlet's theorem that there are infinitely many primes of the form $a+kb$ for $a,b$ coprime involves finite Fourier analysis.
    \pause

    I find the theory to be quite elegant and I wish to learn more (e.g. noncommutative harmonic analysis).
\end{frame}

% section 1
\begin{frame}{}
    \begin{block}{}{
        \begin{center}\Large I. Topological groups and their representations\end{center}}
    \end{block}
\end{frame}

% topological groups
\begin{frame}{Topological groups}
    \begin{block}{Definition}
        A \textbf{topological group} is a group $G$ and a topology on $G$ such that the map $G\times G\to G$ \[(x,y)\mapsto x^{-1}y\] is continuous.
    \end{block} 
    \pause

    A locally compact group (LCA) group is a topological group that is locally compact, Hausdorff, and Abelian.
    \pause

    Of interest are the groups $\mathbb{R}, \mathbb{R}/\mathbb{Z}\cong S^1, \mathbb{Z}$.
    \pause

    There is a similar definition for topological vector spaces.
\end{frame}

% representations of topological groups
\begin{frame}{Representations of topological groups}
    \begin{block}{Definition}
        Let $G$ be a topological group and $V$ a topological vector space.
        
        Then a \textbf{representation} of $G$ on $V$ is a %(continuous)
        homomorphism $\rho\colon G\to GL(V)$ such that for any $v\in V$, the map \[x\mapsto \rho(x)v\] is continuous from $G$ to $V$. % i.e. $\rho$ is continuous in the strong operator topology
    \end{block} 
    \pause

    We say a representation $\rho\colon G\to GL(V)$ of a topological group $G$ on $V$ is a \textbf{unitary representation} if $V$ is a (complex) Hilbert space and $T$ maps into the unitary operators on $V$. % i.e. bounded operators with inverse their adjoint
\end{frame}

% irreducible representations
\begin{frame}
    \frametitle{Irreducible representations}
    A representation $\rho\colon G\to GL(V)$ is irreducible if there are no closed subspaces of $V$ which are invariant under the action of $G$ (i.e. there are no subrepresentations).
    \pause

    \setbeamercolor{block title}{fg=black!85,bg=black!20!palettelight}
    \begin{Theorem}[Unitary Schur]
        A unitary representation $\rho$ is irreducible if and only if \begin{center}
            $\mathcal{C}(\rho)\coloneqq \{T\in GL(V)\mid T\rho(g) = \rho(g) T\forall g\in G\}$
        \end{center} % set of intertwining operators
        contains only scalar multiples of the identity operator.
    \end{Theorem} Proof in (2) p.71.
\end{frame}

% one dimensional representations
\begin{frame}
    \frametitle{Irreducible representations of Abelian groups}
    \begin{Corollary}
        Irreducible unitary representations of Abelian groups are one-dimensional.
    \end{Corollary}
    \pause

    \begin{proof}[Proof (from (2) p.71)]
        Let $\rho\colon G\to GL(V)$ be an irreducible unitary representation of the Abelian group $G$. \pause 
        
        For all $g\in G$, the operators $\rho(g)$ commute with each other. \pause 
        
        By Unitary Schur we have for each $g\in G$ that $\rho(g) = c_gI$. \pause

        It follows that every one-dimensional subspace of $V$ is invariant, so by irreducibility $\dim V = 1$.
    \end{proof} % Gelfand-Raikov theorem gives existence of irreducible unitary representations of a locally compact group separating points in the group
\end{frame}

% characters of representations
\begin{frame}{Characters}
    \begin{definition}
        For $\tau$ a finite dimensional representation of a group $G$, the \textbf{character} of $\tau$ is the class function $\chi_\tau\colon G\to \mathbb{C}$ with \[\chi_\tau(g) = \Tr \tau(g).\]
    \end{definition}
    \pause

    Hence if $\rho$ is an irreducible unitary representation of an Abelian group $G$ we have for $g\in G$ that \[\chi_\rho(g) = \Tr \rho(g) = \rho(g).\]
    \pause
    It follows that characters of irreducible unitary representations of Abelian groups are (continuous) group homomorphisms $\chi\colon G\to S^1$.
\end{frame}

% section 2
\begin{frame}{}
    \begin{block}{}{
        \begin{center}\Large II. Fourier analysis on LCA groups\end{center}}
    \end{block}
\end{frame}

% dual group
\begin{frame}{The dual group}
    \begin{definition}
        For $G$ an LCA group, define the \textbf{dual group} $\widehat{\hspace{0pt}G}$ to be the group of characters of irreducible unitary representations of $G$ under pointwise multiplication.\pause 
        
        That is, $\widehat{\hspace{0pt}G}$ is the set of (continuous) group homomorphisms $\chi\colon G\to S^1$.
    \end{definition} % in general $\widehat{\hspace{0pt}G}$ is the set of equivalence classes of irreducible unitary representations of $G$
    \pause

    We topologize dual groups by giving them the compact-open topology.
\end{frame}

% compact-open topology
\begin{frame}
    \frametitle{The compact-open topology}
    \begin{definition}
        For $X$ a topological space, let $C(X)$ be the $\mathbb{C}$-vector space of all continuous maps $X\to \mathbb C$.
        \pause 

        For compact sets $K\subseteq X$ and open sets $U\subseteq \mathbb C$ the sets \begin{center}
            $L(K,U)\coloneqq\{f\in C(X)\mid f(K)\subseteq U\}$
        \end{center} generate the \textbf{compact-open} topology on $C(X)$.
    \end{definition} 
    \pause
    
    So for LCA groups $G$, the dual group $\widehat{\hspace{0pt}G}\subseteq C(G)$ is given the (subspace) compact-open topology.\pause 

    In general, the dual of an LCA group will be an LCA group, the dual of a compact group is discrete, and vice versa.
\end{frame}

% relevant examples 1. $\mathbb{Z}$
\begin{frame}
    \frametitle{Example 1}
The characters of $\mathbb{Z}$ are the maps $k\mapsto \exp(2\pi i k x)$ for $x\in \mathbb{R}/\mathbb{Z}\cong S^1$, so $\widehat{\mathbb{Z}}\cong S^1$. \pause 
\begin{proof}
    For a character $\chi\colon \mathbb{Z}\to S^1$ we have $\chi(1) = \exp(2\pi i x)$ for some $x\in \mathbb{R}/\mathbb{Z}$. Then $\chi(k) = \chi(1)^k = \exp(2\pi i k x)$ as desired.
\end{proof}
\pause 

The topology that $\widehat{\mathbb{Z}}\cong S^1$ inherits agrees with the usual topology on $S^1$.
\end{frame}

% relevant examples 2. $\mathbb{R}$
\begin{frame}
    \frametitle{Example 2}
The charaters of $\mathbb{R}$ are of the form $x\mapsto \exp(2\pi i x\xi)$ for real $\xi$ so that $\widehat{\mathbb{R}}\cong\mathbb{R}$.
\pause 
\begin{block}{Proof (Keith Conrad to the rescue again)} % actually he gets this from Conway's "A Course in Functional Analysis"
    Let $f \colon \mathbb{R}\to S^1$ be a character, and note $f(0) = 1$.\pause 

    By continuity of $f$, for $0<\varepsilon<1$ small we can fix $a>0$ sufficiently small so that \begin{center}
        $0<a(1-\varepsilon)\leq \int_0^a f(t)\leq a(1+\varepsilon)$.
    \end{center} Let $c = \int_0^a f(t)$.
\end{block}
\end{frame}

% continued
\begin{frame}
    \begin{block}{}
        Then for all real $x$ we have \begin{center}
            $\int_x^{x+a}f(t) = \int_0^a f(x+t) = f(x)\int_0^a f(t) = cf(x)$.
        \end{center} \pause 
        Dividing, we obtain $f(x) = \frac{1}{c}\int_x^{x+a}f(t)$. \pause

        Now $f$ is differentiable by the fundamental theorem of calculus. Differentiating, we obtain the differential equation \begin{center}
            $f^\prime(x) = \frac{f(a)-1}{c} f(x),\quad f(0) = 1$.
        \end{center} Let $s = \frac{f(a)-1}{c}$. \pause 

        It follows from ordinary differential equations that $f(x) = \exp(sx)$. \pause 

        For all $x\in \mathbb{R}$, $|f(x)|= \exp(\Re(s) x)=1$, so $\Re( s) = 0$. \pause 

        Writing $s = 2\pi i \xi$ for some $\xi\in \mathbb{R}$, the result follows. \pushQED{\qed}\qedhere\popQED
    \end{block} \pause The topology on $\widehat{\mathbb{R}}$ is the usual metric topology.
\end{frame}

% relevant examples 3. $\mathbb{R}/\mathbb{Z}\cong S^1$
\begin{frame}
    \frametitle{Example 3}
The characters of $\mathbb{R}/\mathbb{Z}\cong S^1$ are of the form $z\mapsto z^n$ for integer $n$, so that $\widehat{S^1}\cong \mathbb{Z}$. \pause \begin{proof}
    Precompose any character $\chi$ of $\mathbb{R}/\mathbb{Z}$ with the quotient map to obtain a character $\chi^\prime$ of $\mathbb{R}$.\pause 

    That is, $\chi^\prime(x) = \exp(2\pi i x\xi)$ for some real $\xi$. Hence $\chi(x+\mathbb{Z}) = \exp(2\pi i x\xi)$.\pause 

    But then $1 = \chi(1) = \exp(2\pi i \xi)$, so that $\xi\in \mathbb{Z}$ as needed. \pause 

    Identifying $x+\mathbb{Z}$ with $\exp(2\pi i x)$ gives the result.
\end{proof} \pause
The topology on $\widehat{S^1}\cong\mathbb{Z}$ is the discrete topology as expected.
\end{frame}

% haar measure
\begin{frame}{Haar measure for locally compact groups}
    \setbeamercolor{block title}{fg=black!85,bg=black!20!palettelight}
    \begin{Theorem}[Haar]
        Let $G$ be a locally compact group. \pause 

        There exists a nonzero left translation-invariant outer Radon (outer and inner regular) measure on $G$, and it is uniquely determined up to positive multiples.

        Call any such measure on $G$ a Haar measure.
    \end{Theorem}
    \pause

    For example, counting measure on any discrete group works, in particular for $\mathbb{Z}$. 
    \pause 

    Lebesgue measure on $\mathbb{R}$ is a Haar measure, and the arclength measure on $S^1$ is also.
\end{frame}

% fourier transform
\begin{frame}{The Fourier transform}
    \begin{definition}
        Let $G$ be an LCA group and let $L^1(G)$ be the Banach algebra of integrable functions $G\to \mathbb{C}$. \pause 

        Then the Fourier transform $\mathcal{F} \colon L^1(G)\to C_0(\widehat{\hspace{0pt}G})$ [Or $C(\widehat{\hspace{0pt}G})$ when $\widehat{\hspace{0pt}G}$ is compact] is given by $\mathcal{F}f =\hat{f}$, where \begin{center}
            $\hat{f}(\chi) = \int_G f(g)\overline{\chi(g)}$.
        \end{center}
    \end{definition}
    \pause
    So the Fourier transform for $G= \mathbb{R}$ is the familiar \begin{center}
        $f\mapsto \hat{f},\quad \hat{f}(\xi) = \int_{\mathbb{R}}f(x)\exp(-2\pi i x\xi)\dd{x}.$
    \end{center}\pause 
    Similarly the Fourier transform for $G = \mathbb{R}/\mathbb{Z}\cong S^1$ is \begin{center}
        $f\mapsto \hat{f},\quad \hat{f}(n) = \frac{1}{2\pi}\int_{0}^{2\pi}f(x)\exp(- in x)\dd{x}.$
    \end{center}
\end{frame}

% section 3
\begin{frame}{}
    \begin{block}{}{
        \begin{center}\Large III. Pontryagin duality\end{center}}
    \end{block}
\end{frame}

% pontryagin duality
\begin{frame}{Preliminary considerations}
    If $G$ is an LCA group then $\widehat{\hspace{0pt}G}$ is also. We consider the dual group of $\widehat{\hspace{0pt}G}$, denoted $\dwidehat{\hspace{0pt}G}$. \pause

    There is a natural map $\delta\colon G\to \dwidehat{\hspace{0pt}G}$ given by $x\mapsto \delta_x$, where $\delta_x$ is the map given by evaluation at $x$: $\delta_x(\chi) = \chi(x)$.\pause 

    We check briefly that $\delta_x$ is really a character of $\widehat{\hspace{0pt}G}$: We have \[\delta_x(\chi\eta) = (\chi\eta)(x) = \chi(x)\eta(x) = \delta_x(\chi)\delta_x(\eta),\] \pause and we obtain continuity since convergence in the compact-open topology implies pointwise convergence: if we have a convergent net $\chi_j\to \chi$ in $\widehat{\hspace{0pt}G}$, the net $\delta_x(\chi_j) = \chi_j(x)$ converges to $\chi(x)= \delta_x(\chi)$ as needed.
\end{frame}

% more pontryagin duality
\begin{frame}
    \frametitle{Statement of the theorem}
    \setbeamercolor{block title}{fg=black!85,bg=black!20!palettelight}
\begin{Theorem}[Pontryagin duality]
    Let $G$ be an LCA group. The map $\delta\colon G\to \dwidehat{\hspace{0pt}G}$ given by $x\mapsto\delta_x$ defined prior is an isomorphism of LCA groups.
\end{Theorem} Proof in (3) pp.73-76. \pause

There is a fun and much easier version for finite Abelian groups -- see exercise 5.2.14 in Dummit and Foote.\pause 

Indeed, following the previous examples we do have that $\mathbb{R}\cong \dwidehat{\mathbb{R}}$ and $S^1\cong \dwidehat{S^1}$.
\end{frame}

% inversion
\begin{frame}
    \frametitle{Fourier inversion}
    Let $G$ be an LCA group. We may embed $C_0(G)\cap L^1(G)$ into the Banach space $C_0(G)\times C_0(\widehat{\hspace*{0pt}G})$ (with norm $\norm{\cdot} = \max\{\norm{\cdot}_G, \norm{\cdot}_{\widehat{\hspace*{0pt}G}}\}$) by mapping $f$ to $(f,\hat{f})$. Denote by $C_0^\ast(G)$ the closure of the image of this embedding. \pause

    \begin{block}{Proposition}
        For $G$ an LCA group, the Fourier transform is an isometric isomorphism of Banach spaces $\mathcal{F}\colon C^\ast_0(G)\to C^\ast_0(\widehat{\hspace{0pt}G})$ with inverse map given by the dual Fourier transform $\widehat{\mathcal{F}}\colon C^\ast_0(\widehat{\hspace{0pt}G})\to C^\ast_0(G)$, where $\widehat{\mathcal{F}}(g)(x)\coloneqq \hat{g}(\delta_{x^{-1}})$.
    \end{block} Proof in (3) p.76.
\end{frame}

% more Fourier inversion
\begin{frame}
    \setbeamercolor{block title}{fg=black!85,bg=black!20!palettelight}
\begin{theorem}[Inversion formula]
    Let $G$ be an LCA group, and let $f\in L^1(G)$ be such that $\hat{f}\in L^1(\widehat{\hspace*{0pt}G})$. Then $f$ is continuous, and for every $x\in G$ we have \[f(x) = \dhat{f}(\delta_{x^{-1}}).\]
\end{theorem} \pause \setbeamercolor{block title}{fg=black!85,bg=darkgold60}\begin{proof}
    With $f$ as above, we have that $\hat{f}\in C_0(\widehat{\hspace*{0pt}G})\cap L^1(\widehat{\hspace*{0pt}G})\subseteq C^\ast_0(\widehat{\hspace*{0pt}G})$.\pause 

    By the previous proposition, there is $g\in C^\ast_0(G)$ with $\hat{g} = \hat{f}$ and $g(x) = \hat{f}(\delta_{x^{-1}})$ for every $x\in G$. It is true that the Fourier transform is injective on $C^\ast_0(G)$, so that $f = g$ as needed.
\end{proof}
\end{frame}

% questions/end
\begin{frame}
    \begin{flushright}
        {\color{darkgold60}Thank you for your time.}
        \end{flushright}
        \hrule\pause
    \begin{columns}
    \begin{column}{0.3\textwidth}
        \begin{block}{}{
        \begin{center}\Large  Questions?\end{center}}
        \end{block}\vspace{0em}
    \end{column}
    \begin{column}{0.7\textwidth}
        \begin{block}{}{
        \begin{center}\Large  References:\end{center}}
    \end{block}
(1) Aleksandr A. Kirillov: \textit{Elements of the Theory of Representations}

(2) Gerald B. Folland: \textit{A Course in Abstract Harmonic Analysis}

(3) Anton Deitmar, Siegfried Echterhoff: \textit{Principles of Harmonic Analysis}

(4) Anton Deitmar: \textit{A First Course in Harmonic Analysis}

(5) \textcolor{blue}{\href{https://mathoverflow.net/a/89520}{Keith Conrad's mathoverflow post}}%
    \end{column}
    \end{columns}
\end{frame}

% test frame
%\begin{frame}{test frame with some stuff}
%\begin{enumerate}[]
%    \item
%\end{enumerate}
%\end{frame}
\end{document}