\documentclass[mathserif
%, handout
]{beamer}

\usepackage{amsfonts, amsmath, amssymb, amsthm}

\usetheme{Warsaw}

% colors
\definecolor{GTbuzzgold60}{RGB}{245,213,128}
\definecolor{GTpalettemed}{RGB}{142,139,118}
\usecolortheme[named=GTbuzzgold60]{structure}
\setbeamercolor{frametitle}{fg=black!85}
\setbeamercolor{title}{bg=GTbuzzgold60,fg=black!85}
\setbeamercolor{palette quaternary}{bg=GTpalettemed}

% page numbers
\setbeamertemplate{footline}[frame number]

% paragraph indentation/spacing
\setlength{\parindent}{0cm}
\setlength{\parskip}{5pt}
\renewcommand{\baselinestretch}{1.25}

\title
{\textcolor{black!85}{Fourier analysis on LCA groups and Pontryagin Duality}}

\author[Sai Sivakumar]{Sai Sivakumar}
\date{17 April 2023}

\begin{document}

% title slide
\frame{\titlepage}

% motivation and history
\begin{frame}{History and motivation for this talk}
    genesis of Fourier analysis 

    why generalize at all

    where else in math this shows up

    my own personal interest

    not many proofs but I would like to give sketches/the idea
\end{frame}

% section 1
\begin{frame}{}
    \begin{block}{}{
        \begin{center}\Large I. Topological groups and their representations\end{center}}
    \end{block}
\end{frame}

% topological groups
\begin{frame}{Topological groups}
    definition of topological group

    definition of locally compact (and Abelian) group

    examples
\end{frame}

% representations of topological groups
\begin{frame}{Representations of topological groups}
    definition of linear representation of topological group

    definition of unital representation

    examples (relating to Fourier analysis and others)
\end{frame}

% characters of representations
\begin{frame}{Characters}
    recall definitions

    characters of unitary representations

    examples (LCA group representations too)
\end{frame}

% haar measure
\begin{frame}{Haar measure}
    definition of Haar measure on a group

    examples
\end{frame}

% section 2
\begin{frame}{}
    \begin{block}{}{
        \begin{center}\Large II. Fourier analysis on LCA groups\end{center}}
    \end{block}
\end{frame}

% dual group
\begin{frame}{The dual group}
    definition of dual group

    topology on dual group

    examples with their topologies
\end{frame}

% fourier transform
\begin{frame}{The Fourier transform}
    definition of Fourier transform for LCA group

    properties

    examples (recover Fourier series, Fourier transform on $\mathbb{R}$, etc.)
\end{frame}

% section 3
\begin{frame}{}
    \begin{block}{}{
        \begin{center}\Large III. Pontryagin duality\end{center}}
    \end{block}
\end{frame}

% pontryagin duality
\begin{frame}{Statement of theorem}
    statement of Pontryagin duality for LCA groups

    proof

    examples

    Fourier inversion formula?
\end{frame}

% questions/end
\begin{frame}{Thank you for coming}
    \begin{columns}
    \begin{column}{0.5\textwidth}
        \begin{block}{}{
        \begin{center}\Large  Questions?\end{center}}
        \end{block}\vspace{15.5em}
    \end{column}
    \begin{column}{0.5\textwidth}
        \begin{block}{}{
        \begin{center}\Large  Texts:\end{center}}
    \end{block}

    Deitmar - first course

    Deitmar - principles in

    Kirillov - elements of representation

    Kirillov - intro to representations and noncommutative harmonic analysis

    etc (use empty enumerate environment below or itemize idc).
    \end{column}
    \end{columns}
\end{frame}

% test frame
%\begin{frame}{test frame with some stuff}
%\begin{enumerate}[]
%    \item
%\end{enumerate}
%\end{frame}
\end{document}