\documentclass[11pt]{article}
\headheight=13.6pt

% packages
\usepackage{physics}
% margin spacing
\usepackage[top=1in, bottom=1in, left=0.5in, right=0.5in]{geometry}
\usepackage{hanging}
\usepackage{amsfonts, amsmath, amssymb, amsthm}
\usepackage{systeme}
\usepackage[none]{hyphenat}
\usepackage{fancyhdr}
\usepackage{graphicx}
\graphicspath{{./images/}}
\usepackage{float}
\usepackage{siunitx}
\usepackage{esint}
\usepackage{cancel}
\usepackage{enumitem}
\usepackage{mathrsfs}

% header/footer formatting
\pagestyle{fancy}
\fancyhead{}
\fancyfoot{}
\fancyhead[L]{MAP6505}
\fancyhead[C]{HW2}
\fancyhead[R]{Sai Sivakumar}
\fancyfoot[R]{\thepage/??}
\renewcommand{\headrulewidth}{1pt}

% paragraph indentation/spacing
\setlength{\parindent}{0cm}
\setlength{\parskip}{10pt}
\renewcommand{\baselinestretch}{1.25}

% extra commands defined here
\newcommand{\br}[1]{\left(#1\right)}
\newcommand{\sbr}[1]{\left[#1\right]}
\newcommand{\cbr}[1]{\left\{#1\right\}}
\newcommand{\eq}[1]{\overset{(#1)}{=}}

% bracket notation for inner product
\usepackage{mathtools}

\DeclarePairedDelimiterX{\abr}[1]{\langle}{\rangle}{#1}

% smileys frownies
\usepackage{wasysym}
\newcommand{\happy}{\raisebox{-.28em}{\resizebox{1.5em}{!}{\smiley}}}
\newcommand{\darkhappy}{\raisebox{-.28em}{\resizebox{1.5em}{!}{\blacksmiley}}}
\newcommand{\sad}{\raisebox{-.28em}{\resizebox{1.5em}{!}{\frownie}}}
\DeclareMathOperator{\mathhappy}{\!\happy\!}
\DeclareMathOperator{\mathdarkhappy}{\!\darkhappy\!}
\DeclareMathOperator{\mathsad}{\!\sad\!}

\DeclareMathOperator{\Span}{span}
\DeclareMathOperator{\im}{im}
\DeclareMathOperator{\dist}{dist}
\newcommand{\res}[1]{\operatorname*{res}_{#1}}

% set page count index to begin from 1
\setcounter{page}{1}

\begin{document}

\begin{enumerate}
    \item[13.2] Consider the sequence of functions $f_n\colon\mathbb{R}^3\to\mathbb{R}$ defined by \[f_n(x) = \begin{cases}
        c_{1/n}\exp(-\frac{1/n^2}{1/n^2- (\abs{x}-a)^2}) & \abs{\abs{x}-a}<1/n\\
        0 & \abs{\abs{x}-a}\geq 1/n,
    \end{cases}\] where $c_{1/n}$ is a normalization constant given by $c_{1/n} = n/2c_1$, with $c_1 = \int_0^1\exp(-1/(1-u^2))\dd{u}$. In other words, $f_n(x)$ is equal to the hat function $\omega_{1/n}(\abs{x}-a)$, which is evidently locally integrable for all $n$ (the $f_n$ are continuous). Then for any test function $\varphi\in \mathcal{D}(\mathbb{R}^3)$ and $n$ such that $1/n<a$,
    \begin{align*}
        \lim_{n\to\infty} (f_n,\varphi) &\eq{1} \lim_{n\to\infty}\int_{\mathbb{R}^3}\omega_{1/n}(\abs{x}-a)\varphi(x)\dd[3]x\\
        &\eq{2} \lim_{n\to\infty}\int_{\abs{r-a}<1/n}\omega_{1/n}(r-a)\int_0^{2\pi}\int_0^\pi\varphi(x(r,\phi,\theta))\sin(\phi)r^2 \dd\phi \dd\theta \dd r\\
        &\eq{3} \lim_{n\to\infty}\int_{\abs{u}<1/n}n\omega_{1}(nu)\int_0^{2\pi}\int_0^\pi\varphi(x(u+a,\phi,\theta))\sin(\phi)(u+a)^2 \dd\phi \dd\theta \dd u\\
        &\eq{4} \lim_{n\to\infty}\int_{\abs{z}<1}\omega_{1}(z)\int_0^{2\pi}\int_0^\pi\varphi(x(z/n+a,\phi,\theta))\sin(\phi)(z/n+a)^2 \dd\phi \dd\theta \dd z\\
        &\eq{5} a^2\int_0^{2\pi}\int_0^\pi\varphi(x(a,\phi,\theta))\sin(\phi) \dd\phi \dd\theta\\
        &\eq{6} \int_{\abs{x}=a}\varphi(x)\dd{S},
    \end{align*}where \begin{enumerate}
        \item[(1)] is by definition,
        \item[(2)] is converting to spherical coordinates, applying the support of the hat function, using Fubini's theorem,
        \item[(3)] is the change of coordinates $u = r-a$ followed by applying the property of $\omega_\alpha(x) = \omega_1(x/\alpha)/\alpha$ of our hat function,
        \item[(4)] is the change of coordinates $u = z/n$,
        \item[(5)] is by the Lebesgue dominated convergence theorem (on the bounded set we are integrating over, $\varphi$ is continuous hence bounded, and $\abs{z/n+a}\leq 1+a$), and
        \item[(6)] is by definition.
    \end{enumerate} Hence $f_n(x)$ converges to $\delta_{S_a}$.

    \hrulefill

    \item[13.4](v) \textbf{WHAT} Indeed, for each $t>0$, the functions $t^ne^{itx}\theta(x)$ are locally integrable and hence define regular distributions. For any test function $\varphi\in \mathcal{D}(\mathbb{R})$ and $t$ such that $t>1$, \begin{align*}
        \lim_{t\to\infty}(t^ne^{itx}\theta(x),\varphi(x)) &\eq{1} \lim_{t\to\infty}\int_0^\infty (-i)^n\biggl(\dv{x}\biggr)^ne^{itx} \varphi(x)\dd{x}\\
        &\eq{2} \lim_{t\to\infty}(-i)^n\Bigg[\sum_{i=0}^{n-1}(-1)^i\biggl(\dv{x}\biggr)^{n-1-i}e^{itx}\varphi^{(i)}(x)\bigg|_0^\infty + (-1)^n\frac{e^{itx}}{it}\varphi^{(n)}(x)\bigg|_0^\infty \\&\hspace{7em} + (-1)^{n+1}\int_0^\infty \frac{e^{itx}}{it}\varphi^{(n+1)}(x)\dd{x}\Bigg]\\
        &\eq{3} \lim_{t\to\infty}(-i)^n\Bigg[(-1)^{n}\varphi^{(n-1)}(0) + (-1)^{n+1}\frac{1}{it}\varphi^{(n)}(0) \\&\hspace{7em} + (-1)^{n+1}\int_0^R \frac{e^{itx}}{it}\varphi^{(n+1)}(x)\dd{x}\Bigg]\\
        &\eq{4} i^n\varphi^{(n-1)}(0)
    \end{align*} where \begin{enumerate}
        \item[(1)] is by definition, $\theta(x)$ gives the bounds of integration, and since \[\biggl(\dv{x}\biggr)^ne^{itx} = (it)^ne^{itx},\]
        \item[(2)] is integration by parts, 
        \item[(3)] is by evaluating the boundary terms: use the fact that $\varphi$ and its derivatives all have bounded support to find that all but the $n-1$ term in the sum survive, and  
    \end{enumerate}

    \hrulefill

    \item[14.7]

    \hrulefill

    \item[14.15](iii)

    \hrulefill

    \item[16.2]

    \hrulefill

    \item[16.9]

    \hrulefill

    \item[17.3]

    \hrulefill

    \item[17.7]

    \hrulefill

    \item[18.6]

    \hrulefill

    \item[19.1](ii)

    \hrulefill

    \item[20.1](v)

    \hrulefill

    \item[21.2](ii)

    \hrulefill

\end{enumerate}
Honor Code: \vspace*{7em}
\end{document}