\documentclass[11pt]{article}
\headheight=13.6pt

% packages
\usepackage{physics}
% margin spacing
\usepackage[top=1in, bottom=1in, left=0.5in, right=0.5in]{geometry}
\usepackage{hanging}
\usepackage{amsfonts, amsmath, amssymb, amsthm}
\usepackage{systeme}
\usepackage[none]{hyphenat}
\usepackage{fancyhdr}
\usepackage{graphicx}
\graphicspath{{./images/}}
\usepackage{float}
\usepackage{siunitx}
\usepackage{esint}
\usepackage{cancel}
\usepackage{enumitem}
\usepackage{mathrsfs}
\usepackage{hyperref}
\hypersetup{colorlinks=true,urlcolor=blue}

% header/footer formatting
\pagestyle{fancy}
\fancyhead{}
\fancyfoot{}
\fancyhead[L]{MAP6505}
\fancyhead[C]{HW3}
\fancyhead[R]{Sai Sivakumar}
\fancyfoot[R]{\thepage/N}
\renewcommand{\headrulewidth}{1pt}

% paragraph indentation/spacing
\setlength{\parindent}{0cm}
\setlength{\parskip}{10pt}
\renewcommand{\baselinestretch}{1.25}

% extra commands defined here
\newcommand{\br}[1]{\left(#1\right)}
\newcommand{\sbr}[1]{\left[#1\right]}
\newcommand{\cbr}[1]{\left\{#1\right\}}
\newcommand{\eq}[1]{\overset{(#1)}{=}}

% bracket notation for inner product
\usepackage{mathtools}

\DeclarePairedDelimiterX{\abr}[1]{\langle}{\rangle}{#1}

% smileys frownies
\usepackage{wasysym}
\newcommand{\happy}{\raisebox{-.28em}{\resizebox{1.5em}{!}{\smiley}}}
\newcommand{\darkhappy}{\raisebox{-.28em}{\resizebox{1.5em}{!}{\blacksmiley}}}
\newcommand{\sad}{\raisebox{-.28em}{\resizebox{1.5em}{!}{\frownie}}}
\DeclareMathOperator{\mathhappy}{\!\happy\!}
\DeclareMathOperator{\mathdarkhappy}{\!\darkhappy\!}
\DeclareMathOperator{\mathsad}{\!\sad\!}

\DeclareMathOperator{\Span}{span}
\DeclareMathOperator{\im}{im}
\DeclareMathOperator{\dist}{dist}
\DeclareMathOperator{\supp}{supp}
\newcommand{\res}[1]{\operatorname*{res}_{#1}}

% set page count index to begin from 1
\setcounter{page}{1}

\begin{document}

\begin{enumerate}
    \item\begin{enumerate}[label=(\roman*)]
        Let $L = D^2+2\gamma D+ \omega^2I$ for $\gamma\geq 0$ denote the differential operator for an oscillator.
        \item 
    \end{enumerate}

    \hrulefill

    \item\begin{enumerate}[label=(\roman*)]
        \item We have
        \[\mathcal F\Big[\mathcal P\frac{1}{x^2}\Big](k) \eq{1} - \mathcal F\Big[\dv{x}\mathcal P\frac{1}{x}\Big](k) \eq{2} ik\mathcal F\Big[\mathcal P \frac{1}{x}\Big](k) \eq{3} -\pi k\varepsilon(k) \eq{4} -\pi\abs{k},\] where \begin{enumerate}
            \item[(1)] is by differentiation of $\mathcal P\frac{1}{x}$, 
            \item[(2)] is by the property $\mathcal F[D^\alpha f] = (-ik)^\alpha\mathcal F[f]$ of the Fourier transform,
            \item[(3)] is by a computation (Example 27.4.6 in the notes), and
            \item[(4)] holds since $k\varepsilon(k) = \abs{k}$, where $\varepsilon(x) = \theta(x)-\theta(-x)$ is the sign function.
        \end{enumerate}
        \item Take the inverse Fourier transform in (i) above to obtain
        \[-2\mathcal P \frac{1}{k^2} \eq{1} -2\mathcal F^{-1}\Big[\mathcal F\Big[\mathcal P\frac{1}{k^2}\Big](x)\Big](k) \eq{2} -2\mathcal F^{-1}[-\pi\abs{x}](k) \eq{3} -2(2\pi)^{-1}\mathcal F[-\pi\abs{-x}](k) \eq{4} \mathcal{F}[\abs{x}](k),\] where \begin{enumerate}
            \item[(1)] is by the property $\mathcal F^{-1}[\mathcal F[f]] = f$ of the Fourier transform,
            \item[(2)] is by (i) above,
            \item[(3)] is by the property $\mathcal F[f(k)] = (2\pi)^{-N}\mathcal F[f(-k)]$ of the Fourier transform, and
            \item[(4)] is algebra.
        \end{enumerate}
        \item Let $f(x_0,x) = \theta(x_0-\abs{x})$. We verify that $f_a(x_0,x) = e^{-ax_0}f(x_0,x)$ converges to $f(x_0,x)$ in $\mathcal S^\prime$ as $a\to 0^+$. Let $\varphi\in \mathcal S$ be any test function. Then 
        \begin{align*}
            \lim_{a\to0^+}(f_a,\varphi) &\eq{1} \lim_{a\to0^+}\int_{\mathbb R^2}e^{-ax_0}\theta(x_0-\abs{x})\varphi(x_0,x)\dd x_0\dd x\\
            &\eq{2} \lim_{a\to0^+}\int_0^\infty e^{-ax_0}\int_{-x_0}^{x_0}\varphi(x_0,x)\dd x\dd x_0\\
            &\eq{3} \int_0^\infty \int_{-x_0}^{x_0}\varphi(x_0,x)\dd x\dd x_0\\
            &\eq{4} (f,\varphi),
        \end{align*} where \begin{enumerate}
            \item[(1)] holds since for any $a>0$, $f_a(x_0,x)$ is locally integrable and $\abs{f_a(x_0,x)} = e^{-ax_0}\theta(x_0-\abs{x}) \leq 1$ (note $\supp f\subset \{x_0\geq 0\}$); hence $f_a$ defines a regular temperate distribution for any $a>0$,
            \item[(2)] is by Fubini's theorem,
            \item[(3)] is by the Lebesgue dominated convergence theorem (or even MCT) since $|e^{-ax_0}\int_{-x_0}^{x_0}\varphi(x_0,x)\dd x |\leq \int_{-x_0}^{x_0}\abs{\varphi(x_0,x)}\dd x$, which is integrable in $x_0$ since $\varphi\in \mathcal L(\mathbb R^2)$, and
            \item[(4)] is since $f(x_0,x)$ is also locally integrable with $\abs{f(x_0,x)}\leq 1$; hence $f$ defines a regular distribution with $(f,\varphi) = \int_0^\infty \int_{-x_0}^{x_0}\varphi(x_0,x)\dd x\dd x_0$ by Fubini's theorem.
        \end{enumerate} It follows that $f_a\to f$ in $\mathcal S^\prime$ as needed. By continuity of the Fourier transform, $\lim_{a\to 0^+}\mathcal F[f_a](k_0,k) = \mathcal F[f](k_0,k)$ and for $a>0$, $f_a(x_0,x)$ are integrable ($\int_0^\infty \int_{-x_0}^{x_0}e^{-ax_0}\dd x\dd x_0 = 2/a^2$ exists), we have
        \begin{align*}
            \lim_{a\to 0^+}\mathcal F[f_a](k_0,k) &\eq{1} \lim_{a\to 0^+} \int_0^\infty e^{(-a+ik_0)x_0}\int_{-x_0}^{x_0}e^{ikx}\dd x \dd x_0\\
        &\eq{2} \lim_{a\to 0^+} \lim_{R\to+\infty} \frac{2}{k^2}\int_0^{kR} e^{(-a+ik_0)u/k}\sin(u) \dd u\\
        &\eq{3} \lim_{a\to 0^+}  \frac{2}{k^2}\bigg[\lim_{R\to+\infty} \frac{-e^{-aR+ik_0R}\cos(kR)+1+\frac{-a+ik_0}{k}e^{-aR+ik_0R}\sin(kR)}{[(-a+ik_0)/k]^2+1}\bigg]\\
        &\eq{4} \lim_{a\to 0^+}  \frac{2}{k^2-(k_0+ia)^2}\\
        &\eq{5} \textbf{???},
        \end{align*} where \begin{enumerate}
            \item[(1)] is by definition and Fubini's theorem,
            \item[(2)] is by evaluating the inner integral and then making the change of variables $u = kx_0$,
            \item[(3)] is by integration,
            \item[(4)] is by taking the inner limit, followed by algebra,
            \item[(5)] \textbf{???}
        \end{enumerate}
        \item Let $\varphi\in\mathcal S$ be any test function. Then \begin{align*}
            (\mathcal F[\delta_{C_a}(x)](k),\varphi(k)) &\eq{1} (\delta_{C_a}(x),\mathcal F[\varphi(k)](x)) \\
            &\eq{2} \int_{\abs{x} = a}\int_{\mathbb{R}^2}e^{i(x,k)}\varphi(k)\dd k \dd x\\
            &\eq{3} \int_{\mathbb{R}^2}\int_{\abs{x} = a}e^{i(x,k)}\varphi(k) \dd x \dd k\\
            &\eq{4} \int_{\mathbb R^2}\varphi(k)a\int_0^{2\pi} e^{ia \abs{k}\cos\phi}\dd \phi\dd k\\
            &\eq{5} \int_{\mathbb R^2}\varphi(k)\frac{2\pi a}{\pi}\int_0^{\pi} \frac{1}{2}\big(e^{ia \abs{k}\cos\phi}+e^{-ia \abs{k}\cos\phi}\big)\dd \phi\dd k\\
            &\eq{6} \int_{\mathbb R^2}2\pi a J_0(a\abs{k}) \varphi(k)\dd k\\
            &\eq{7} (2\pi a J_0(a\abs{k}),\varphi(k)),
        \end{align*} where \begin{enumerate}
            \item[(1)] is by definition of the Fourier transform,
            \item[(2)] is by definition of the Fourier transform and the spherical delta function,
            \item[(3)] is by Fubini's theorem since $\int_{\mathbb R^2}\int_{\abs{x} = a}|e^{i(x,k)}\varphi(k)|\dd x \dd k\leq \int_{\mathbb R^2}\abs{\varphi(k)}\int_{\abs{x} = a}\dd x \dd k = 2\pi a\int\abs{\varphi(k)}\dd k < \infty$ since $\varphi\in \mathcal L$,
            \item[(4)] is by changing to polar coordinates about the vector $k$, so $\phi$ is the angle between $x$ and $k$ and $(x,k) = \abs{x}\abs{k}\cos\phi$,
            \item[(5)] is by additivity of the integral ($\int_0^{2\pi} = \int_0^\pi + \int_\pi^{2\pi}$) and making a change of coordinates $\phi\mapsto\phi - \pi$ in the second integral plus algebra,
            \item[(6)] holds since the inner integrand is equal to $\cos(a\abs{k}\cos\phi)$ and by \hyperref{https://dlmf.nist.gov/10.9}{website}{https://dlmf.nist.gov/10.9}{https://dlmf.nist.gov/10.9} equation 10.9.1,
            \item[(7)] holds since $J_0(a\abs{k})$ should define a regular distribution (it is locally integrable since $J_0$ is continuous, and has moderate growth since $|\frac{1}{\pi}\int_0^{\pi} \frac{1}{2}\big(e^{ia \abs{k}\cos\phi}+e^{-ia \abs{k}\cos\phi}\big)\dd \phi|\leq \frac{1}{\pi}\int_0^\pi \dd\phi = 1$).
        \end{enumerate} Hence $\mathcal F[\delta_{C_a}(x)](k) = 2\pi a J_0(a\abs{k})$.
    \end{enumerate}

    \hrulefill

    \item\begin{enumerate}[label=(\roman*)]
        \item Certainly $\theta(x), x\theta(x)\in \mathcal D_+^\prime$, so the convolutions $\theta(x)\ast x\theta(x)$ and $\dv x[x\theta(x)\ast x\theta(x)]$ exist (Theorem 24.2 and section 23.5.2 in the notes). Furthermore, 
        \[\dv x[x\theta(x)\ast x\theta(x)] = \dv x(x\theta(x))\ast x\theta(x) = (\theta(x) + x\delta(x))\ast x\theta(x) = \theta(x)\ast x\theta (x).\] (Note $x\delta(x) = 0$, so we don't need to check distributivity, even though distributivity is satisfied anyways.) The distributions $\theta(x),x\theta(x)$ are regular distributions (they are locally integrable), so their convolution may be computed directly as \[[\theta(t)\ast t\theta(t)](x) = \int_{\mathbb R}\theta(x)(x-\tau)\theta(x-\tau)\dd \tau = \theta(x)\int_0^x x-\tau \dd \tau = \frac{x^2}{2}\theta(x).\]
        % Using the property $\mathcal F[D^\alpha f] = (-ik)^\alpha\mathcal F[f]$ of the Fourier transform, we have \begin{align*}
        %     (-ik)^{-1} = (-ik)^{-1}\mathcal F[\delta(x)](k) &= (-ik)^{-1}\mathcal F\bigg[\dv x \theta(x)\bigg](k) = \mathcal F[\theta(x)](k)\\
        %     (-ik)^{-2} = (-ik)^{-2}\mathcal F[\delta(x)](k) &= (-ik)^{-2}\mathcal F\bigg[\dv[2]x x\theta(x)\bigg](k) = \mathcal F[x\theta(x)](k)\\
        %     (-ik)^{-3} = (-ik)^{-3}\mathcal F[\delta(x)](k) &= (-ik)^{-3}\mathcal F\bigg[\dv[2]x \frac{x^2}{2}\theta(x)\bigg](k) = \mathcal F\bigg[\frac{x^2}{2}\theta(x)\bigg](k),
        % \end{align*} where $\dv[2]x \frac{x^2}{2}\theta(x) = \delta(x)$ by a similar computation as above. Then with the properties $\mathcal F^{-1}[\mathcal F[f]] = f$ and $\mathcal F[f\ast g] = \mathcal F[f]\mathcal F[g]$ of the Fourier transform, we have
        % \begin{align*}
        %     \theta(x)\ast x\theta(x) &= \mathcal F^{-1}[\mathcal F[\theta(x)\ast x\theta(x)](k)](x)\\
        %     &= \mathcal F^{-1}[\mathcal F[\theta(x)](k)\mathcal F[x\theta(x)](k)](x)\\
        %     &= \mathcal F^{-1}[(-ik)^{-3}](x)\\
        %     &= \frac{x^2}{2}\theta(x).
        % \end{align*}
        \item The support of $\delta_{S_a}(x)$ (a sphere) is compact, so the convolution $\frac{1}{\abs{x}}\ast\delta_{S_a}(x)$ exists and 
        \[\Delta\bigg[\frac{1}{\abs{x}}\ast\delta_{S_a}(x)\bigg] = \Delta\frac{1}{\abs{x}}\ast\delta_{S_a}(x) = \frac{1}{\abs{x}}\ast \Delta \delta_{S_a}(x).\] (Theorem 24.1 and section 23.5.2) Since $\Delta \frac{1}{\abs{x}} = -4\pi\delta(x),$ it follows that $\frac{1}{\abs{x}}\ast \Delta \delta_{S_a} = -4\pi\delta(x)\ast \delta_{S_a}(x) = -4\pi \delta_{S_a}$.
    \end{enumerate}

    \hrulefill

    \item\begin{enumerate}[label=(\roman*)]
        \item 
    \end{enumerate}

    \hrulefill
\end{enumerate}
Honor Code: \vspace*{7em}
\end{document}