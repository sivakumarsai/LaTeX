\documentclass[11pt]{article}
\headheight=13.6pt

% packages
\usepackage{physics}
% margin spacing
\usepackage[top=1in, bottom=1in, left=0.5in, right=0.5in]{geometry}
\usepackage{hanging}
\usepackage{amsfonts, amsmath, amssymb, amsthm}
\usepackage{systeme}
\usepackage[none]{hyphenat}
\usepackage{fancyhdr}
\usepackage{graphicx}
\graphicspath{{./images/}}
\usepackage{float}
\usepackage{siunitx}
\usepackage{esint}
\usepackage{cancel}
\usepackage{enumitem}
\usepackage{mathrsfs}

% header/footer formatting
\pagestyle{fancy}
\fancyhead{}
\fancyfoot{}
\fancyhead[L]{MAP6505}
\fancyhead[C]{HW1}
\fancyhead[R]{Sai Sivakumar}
\fancyfoot[R]{\thepage/n}
\renewcommand{\headrulewidth}{1pt}

% paragraph indentation/spacing
\setlength{\parindent}{0cm}
\setlength{\parskip}{10pt}
\renewcommand{\baselinestretch}{1.25}

% extra commands defined here
\newcommand{\br}[1]{\left(#1\right)}
\newcommand{\sbr}[1]{\left[#1\right]}
\newcommand{\cbr}[1]{\left\{#1\right\}}

% bracket notation for inner product
\usepackage{mathtools}

\DeclarePairedDelimiterX{\abr}[1]{\langle}{\rangle}{#1}

\DeclareMathOperator{\Span}{span}
\DeclareMathOperator{\im}{im}
\DeclareMathOperator{\dist}{dist}

% set page count index to begin from 1
\setcounter{page}{1}

\begin{document}

\begin{enumerate}
    \item[3.3] \begin{enumerate}
        \item Not Lebesgue integrable. Lebesgue integrability implies absolute integrability, but \[\int \abs{\frac{\sin x}{x}}\dd x = 2\int_0^\infty \frac{\abs{\sin x}}{x}\dd x = \infty\] (via estimates from p.37).
        \item Not Lebesgue integrable. Again \[\int \abs{\frac{e^{ikx}}{x}}\dd x = 2\int_0^\infty \frac{1}{x}\dd x=\infty ,\] so $e^{ikx}/x$ failed to be absolutely integrable.
        \item Not Lebesgue integrable. In any neighborhood around the origin, 
        \item Not Lebesgue integrable. The function $e^{-x}$ fails absolute integrability (note it is already nonnegative), since $\int e^{-x}\dd x = \int e^x \dd x = \infty$.
        \item Lebesgue integrable. In any interval centered at the origin, $x^{100}e^{-x^2}$ is bounded. 
    \end{enumerate}
    \item[4.8]
    \item[5.6]
    \item[8.6]
    \item[9.2]
\end{enumerate}
\vspace*{7em}

\end{document}