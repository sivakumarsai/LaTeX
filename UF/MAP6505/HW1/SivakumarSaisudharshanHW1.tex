\documentclass[11pt]{article}
\headheight=13.6pt

% packages
\usepackage{physics}
% margin spacing
\usepackage[top=1in, bottom=1in, left=0.5in, right=0.5in]{geometry}
\usepackage{hanging}
\usepackage{amsfonts, amsmath, amssymb, amsthm}
\usepackage{systeme}
\usepackage[none]{hyphenat}
\usepackage{fancyhdr}
\usepackage{graphicx}
\graphicspath{{./images/}}
\usepackage{float}
\usepackage{siunitx}
\usepackage{esint}
\usepackage{cancel}
\usepackage{enumitem}
\usepackage{mathrsfs}

% header/footer formatting
\pagestyle{fancy}
\fancyhead{}
\fancyfoot{}
\fancyhead[L]{MAP6505}
\fancyhead[C]{HW1}
\fancyhead[R]{Sai Sivakumar}
\fancyfoot[R]{\thepage/n}
\renewcommand{\headrulewidth}{1pt}

% paragraph indentation/spacing
\setlength{\parindent}{0cm}
\setlength{\parskip}{10pt}
\renewcommand{\baselinestretch}{1.25}

% extra commands defined here
\newcommand{\br}[1]{\left(#1\right)}
\newcommand{\sbr}[1]{\left[#1\right]}
\newcommand{\cbr}[1]{\left\{#1\right\}}

% bracket notation for inner product
\usepackage{mathtools}

\DeclarePairedDelimiterX{\abr}[1]{\langle}{\rangle}{#1}

% smileys frownies
\usepackage{wasysym}
\newcommand{\happy}{\raisebox{-.28em}{\resizebox{1.5em}{!}{\smiley}}}
\newcommand{\darkhappy}{\raisebox{-.28em}{\resizebox{1.5em}{!}{\blacksmiley}}}
\newcommand{\sad}{\raisebox{-.28em}{\resizebox{1.5em}{!}{\frownie}}}
\DeclareMathOperator{\mathhappy}{\!\happy\!}
\DeclareMathOperator{\mathdarkhappy}{\!\darkhappy\!}
\DeclareMathOperator{\mathsad}{\!\sad\!}

\DeclareMathOperator{\Span}{span}
\DeclareMathOperator{\im}{im}
\DeclareMathOperator{\dist}{dist}

% set page count index to begin from 1
\setcounter{page}{1}

\begin{document}

\begin{enumerate}
    \item[3.3] \begin{enumerate}
        \item No. Lebesgue integrability implies absolute integrability, but \[\int \abs{\frac{\sin x}{x}}\dd x = 2\int_0^\infty \frac{\abs{\sin x}}{x}\dd x \to \infty\] (via estimates from p.37).
        \item No. Again \[\int \abs{\frac{e^{ikx}}{x}}\dd x = 2\int_0^\infty \frac{1}{x}\dd x\to \infty ,\] so $e^{ikx}/x$ failed to be absolutely integrable.
        \item No. In any neighborhood around the origin, $\abs{ \cos x}/\sqrt{\abs{x}}\leq 1/\sqrt{\abs{x}}$, which is integrable around the origin. Note $\cos x/\sqrt{\abs{x}}$ is an even function. Then integrating the estimate \[\frac{\sqrt 2\abs{\cos x}}{\sqrt{\pi(2k+1)}}\leq \frac{\abs{\cos x}}{\sqrt{\abs{x}}}\leq \frac{\sqrt 2\abs{\cos x}}{\sqrt{\pi(2k-1)}}\] valid for for positive integers $k$ and $x\in [(2k-1)\pi/2,(2k+1)\pi/2]$ from $(2k-1)\pi/2$ to $(2k+1)\pi/2$ yields \[\frac{2\sqrt 2}{\sqrt{\pi(2k+1)}}\leq \int_{\frac{(2k-1)\pi}{2}}^{\frac{(2k+1)\pi}{2}}\frac{\abs{\cos x}}{\sqrt{\abs{x}}}\dd x\leq \frac{2\sqrt 2}{\sqrt{\pi(2k-1)}}.\] Then \[\int_{\mathbb{R}\setminus B_{\pi/2}} \frac{\abs{\cos x}}{\sqrt{\abs{x}}}\dd x = 2\int_{\pi/2}^\infty \frac{\abs{\cos x}}{\sqrt{x}}\dd x\to \infty,\] and by the above estimate and the first comment, it follows that $\cos x /\sqrt{\abs{x}}$ is not absolutely integrable.
        \item No. The function $e^{-x}$ fails absolute integrability (note it is already nonnegative), since $\int e^{-x}\dd x = \int e^x \dd x = \infty$.
        \item Yes. First, $x^{100}e^{-x^2}$ is bounded, and nonnegative. Then we have the estimate \[\frac{1}{x^{-100}e^{x^2}} = \frac{1}{\sum_{k=0}^\infty x^{2k-100}/k!}\leq \frac{C}{D+x^2}\] for some positive constants $C,D$, for any $x$. It follows that $x^{100}e^{-x^2}$ is absolutely integrable and hence Lebesgue integrable.
    \end{enumerate}
    \hrulefill
    \item[4.8] For each $n$, change variables $nx\mapsto x$ to obtain \[\int f_n(x)\varphi(x)\dd x = \int nf(nx)\varphi(x)\dd x = \int f(x)\varphi(x/n)\dd x.\] The sequence of functions $g_n(x) = f(x)\varphi(x/n)$ converges to $f(x)\varphi(0)$ almost everywhere and is uniformly bounded above by $Mf(x)\in \mathcal L$, where $\abs{\varphi(x)}\leq M$ for all $x$. (Note $\varphi(x)$ has bounded support so $\varphi(x/n)$ does also, hence for all $n$, $\abs{\varphi(x/n)}$ is uniformly bounded by $M$ using the fact that continuous functions with bounded support are bounded.) Then by the Lebesgue dominated convergence theorem, \[\lim_{n\to\infty} \int f(x)\varphi(x/n)\dd x = \varphi(0)\int f(x)\dd x= \varphi(0)\] as desired.
    
    \hrulefill
    \item[5.6] \begin{enumerate}
        \item[(i)] We have from the change of variables $x=y\tan \theta $ that 
        \[\int_1^\infty\bigg(\int_1^\infty \frac{\abs{x^2-y^2}}{(x^2+y^2)^2} \dd x\bigg)\dd y = \int_{1}^\infty \frac{1}{y} \bigg(\int_{\arctan(y^{-1})}^{\pi/2}\abs{\cos(2\theta)}\dd \theta\bigg) \dd y,\] but since $0<y^{-1}< 1$, $0 < \arctan(y^{-1}) < \pi/4$ ($\arctan$ is monotonic). Thus 
        \[\int_{1}^\infty \frac{1}{y} \bigg(\int_{\arctan(y^{-1})}^{\pi/2}\abs{\cos(2\theta)}\dd \theta\bigg) \dd y\geq \bigg(\int_{1}^\infty \frac{1}{y} \dd y\bigg)\bigg(\int_{\pi/4}^{\pi/2}-\cos(2\theta)\dd \theta\bigg) = \frac{1}{2}\int_1^\infty \frac{1}{y}\dd y \to \infty.\] If $f$ were Lebesgue integrable on $\Omega$, computing the first integral using Fubini's theorem would have yielded a finite value, but $\int_1^\infty (\int_1^\infty |f(x,y)|\dd x) \dd y \to \infty$. So $f$ is not Lebesgue integrable on $\Omega$.
        \item[(ii)] The iterated integrals (note $x,y\neq 0$): 
        \begin{multline*}
            \int_1^\infty\bigg(\int_1^\infty \frac{x^2-y^2}{(x^2+y^2)^2} \dd x\bigg)\dd y = \int_1^\infty\bigg(\int_1^\infty \pdv{x}\frac{-x}{x^2+y^2} \dd x\bigg)\dd y = \int_1^\infty\bigg(\lim_{a\to\infty} \frac{-x}{x^2+y^2}\bigg|_{1}^a \bigg)\dd y \\
            = \int_1^\infty \frac{1}{1+y^2}\dd y = \lim_{b\to\infty} \arctan(y)\big|_{1}^b = \pi/4,
        \end{multline*}\begin{multline*}
            \int_1^\infty\bigg(\int_1^\infty \frac{x^2-y^2}{(x^2+y^2)^2} \dd y\bigg)\dd x = \int_1^\infty\bigg(\int_1^\infty \pdv{y}\frac{y}{x^2+y^2} \dd y\bigg)\dd x = \int_1^\infty\bigg(\lim_{a\to\infty} \frac{y}{x^2+y^2}\bigg|_{1}^a\bigg)\dd x \\
            = \int_1^\infty \frac{-1}{1+x^2}\dd x = -\lim_{b\to\infty} \arctan(x)\big|_{1}^b = -\pi/4.
        \end{multline*}
        The iterated integrals do not match and are off by a sign.
    \end{enumerate}
    \hrulefill
    \item[8.6]
    
    \hrulefill
    \item[9.2]
    
    \hrulefill
\end{enumerate}
\vspace*{7em}

\end{document}