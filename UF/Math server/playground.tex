\documentclass[11pt]{article}

% packages
\usepackage{physics}
% margin spacing
\usepackage[top=1in, bottom=1in, left=1in, right=1in]{geometry}
\usepackage{hanging}
\usepackage{amsfonts, amsmath, amssymb, amsthm}
\usepackage{systeme}
\usepackage[none]{hyphenat}
\usepackage{fancyhdr}
\usepackage[nottoc, notlot, notlof]{tocbibind}
\usepackage{graphicx}
\graphicspath{{./images/}}
\usepackage{float}
\usepackage{siunitx}
%\usepackage{esint}
\usepackage{cancel}
\usepackage{musixtex}

% smileys frownies
\usepackage{wasysym}
\newcommand{\happy}{\raisebox{-.28em}{\resizebox{1.5em}{!}{\smiley}}}
\newcommand{\darkhappy}{\raisebox{-.28em}{\resizebox{1.5em}{!}{\blacksmiley}}}
\newcommand{\sad}{\raisebox{-.28em}{\resizebox{1.5em}{!}{\frownie}}}
\DeclareMathOperator{\mathhappy}{\!\happy\!}
\DeclareMathOperator{\mathdarkhappy}{\!\darkhappy\!}
\DeclareMathOperator{\mathsad}{\!\sad\!}


% colors\
\usepackage{xcolor}
\definecolor{Go}{HTML}{00AAFF}
\definecolor{Gators}{HTML}{FF5500}

% header/footer formatting
\pagestyle{fancy}
\fancyhead{}
\fancyfoot{}
\fancyhead[L]{sai sivakumar}
\fancyhead[C]{}
\fancyhead[R]{}
\fancyfoot[R]{}
\renewcommand{\headrulewidth}{0pt}

% paragraph indentation/spacing
\setlength{\parindent}{0cm}
\setlength{\parskip}{5pt}
\renewcommand{\baselinestretch}{1.25}

% extra commands defined here
\newcommand{\ihat}{\boldsymbol{\hat{\textbf{\i}}}}
\newcommand{\jhat}{\boldsymbol{\hat{\textbf{\j}}}}
\newcommand{\dr}{\vec{r}~^{\prime}(t)}
\newcommand{\dx}{x^{\prime}(t)}
\newcommand{\dy}{y^{\prime}(t)}

\newcommand{\br}[1]{\left(#1\right)}
\newcommand{\sbr}[1]{\left[#1\right]}
\newcommand{\cbr}[1]{\{#1\}}

\newcommand{\dprime}{\prime\prime}
\newcommand{\lap}[2]{\mathcal{L}[#1](#2)}

% bracket notation for inner product
\usepackage{mathtools}

\DeclarePairedDelimiterX{\abr}[1]{\langle}{\rangle}{#1}

\DeclareMathOperator{\Span}{span}

% set page count index to begin from 1
\setcounter{page}{1}

\begin{document}

Let $f(t)$ be piecewise continuous on $[0,\infty)$ and of exponential order. Find $$\lim_{s\to \infty}\mathcal{L}\{f\}$$ where $\mathcal{L}\{f\}$ is the Laplace transform of $f(t)$. \textit{Let $s$ be real.}\vspace{1cm}

$\mathcal{L}\cbr{f} = 0$.

\begin{proof}
Since $f(t)$ is of exponential order, there exist $\alpha, M>0, T>0 \in\mathbb{R}$ such that $\abs{f(t)}\leq Me^{\alpha{t}}$ for all $t>T$.

With $$\mathcal{L}\cbr{f} = \int_0^{\infty}e^{-st}f(t)\dd{t},$$ observe that since $f(t)$ is of exponential order, $\mathcal{L}\cbr{f}$ exists for all $s>\alpha$ (the integral converges). We have that $$\abs{\mathcal{L}\cbr{f}} = \abs{\int_0^{\infty}e^{-st}f(t)\dd{t}} \leq \int_0^{\infty}\abs{e^{-st}f(t)}\dd{t} = \int_0^{\infty}\abs{e^{-st}}\abs{f(t)}\dd{t}$$ $$\leq \int_0^{T}e^{-st}\abs{f(t)}\dd{t} + \int_T^{\infty}e^{-st}Me^{\alpha t}\dd{t}.$$

Then since $f(t)$ is piecewise continuous on $[0,\infty)$, we have that there exists a real value $C\geq 0$ such that for $t\in\sbr{0,T}$, $\abs{f(t)}\leq C$. Then $$\int_0^{T}e^{-st}\abs{f(t)}\dd{t} + \int_T^{\infty}e^{-st}Me^{\alpha t}\dd{t} \leq \int_0^{T}e^{-st} C\dd{t} + \int_T^{\infty}e^{-st}Me^{\alpha t}\dd{t},$$ and so what remains is to take the limit as $s$ tends to infinity.

Since $e^{-st}$ and $e^{(\alpha -s)t}$ converge uniformly to the identically zero function as $s\to \infty$, we can compute the limit $$\lim_{s\to\infty} \br{\int_0^{T}e^{-st} C\dd{t} + \int_T^{\infty}e^{-st}Me^{\alpha t}\dd{t}} = \lim_{s\to\infty}\int_0^{T}e^{-st} C\dd{t} + \lim_{s\to\infty}\int_T^{\infty}e^{-st}Me^{\alpha t}\dd{t}$$ by passing the limit into the integral. Then $$\lim_{s\to\infty}\int_0^{T}e^{-st} C\dd{t} + \lim_{s\to\infty}\int_T^{\infty}e^{-st}Me^{\alpha t}\dd{t} = C\int_0^{T}\lim_{s\to\infty}\br{e^{-st}} \dd{t} + M\int_T^{\infty}\lim_{s\to\infty}\br{e^{(\alpha-s)t}}\dd{t}$$ $$= C\int_0^T \br{0}\dd{t} + M\int_T^{\infty}\br{0}\dd{t} = 0.$$

Hence $$\lim_{s\to \infty} \abs{\mathcal{L}\cbr{f}}\leq 0,$$ and so by the squeeze theorem (and continuity of absolute value), $\mathcal{L}\cbr{f} = 0$.
\end{proof}

ahdsgasdfj \smiley
tghtrjtu \frownie

asldgkf Hiw \happy HOW HWO fda HoW $\sad(x)$ \sad.12

A\sad A

How \sad.
Starting with \sad but ending with \happy is great \darkhappy.

$\smiley$

\[\smiley\sin(x)\mathhappy(x)\]
\[\mathhappy = \mathhappy(y)\]
\[ \int_\Omega\mathhappy(x)\dd^n{x}\]
\smiley

\darkhappy \happy

$\oint$ \[\oint\]

\end{document}