\documentclass[11pt]{article}

\usepackage[top=1in, bottom=1in, left=1in, right=1in]{geometry}
\usepackage{hanging}
\usepackage{amsfonts, amsmath, amssymb}
\usepackage[none]{hyphenat}
\usepackage{fancyhdr}
\usepackage[nottoc, notlot, notlof]{tocbibind}
\usepackage{graphicx}
\usepackage{float}

\pagestyle{fancy}
\fancyhead{}
\fancyfoot{}
\fancyhead[L]{Sai Sivakumar}
\fancyhead[R]{\thepage}

\parindent 0ex
\renewcommand{\baselinestretch}{1.5}

\begin{document}

\title{\textbf{A Property of Lines Passing Through the Foci of Nondegenerate Conic Sections}}
\author{Sai Sivakumar}
\maketitle
\thispagestyle{empty}
\pagebreak
	
\tableofcontents
\thispagestyle{empty}
\clearpage
\pagebreak	

\setcounter{page}{1}

\section{Abstract}
\vfill
\begin{abstract}

Any two lines coplanar with the curve of a conic section that:\\
-pass through one unique foci of a conic section\\
-intersect each other on the path of the conic section\\
will form congruent angles with the normal line constructed on the conic section at the point of intersection.

For the Parabola the path of the conic section must be redefined as an ellipse whose second focus is at an infinite distance from the first one \cite{MSE}, in which case the line that passes through the second focus will be a vertical line (from any point of intersection on the path of the conic section). There are 
cases where the redefinition of the parabola is disregarded and it is assumed one of the two lines is a vertical line without any consideration of there being a second focus. Then it is shown that the vertical line forms the same angle with the normal drawn on the point of intersection of the line that passes through the focus.

For the Ellipse and Hyperbola 

\end{abstract}
\vfill
\pagebreak

\section{The Parabola}
$\frac{d}{dx}\left( \sin{(x)}^{(e^x\tan^-1{(x)}+7)^2} \right) = \sin^(e^x\tan^-1{(x)}+7)^2{(x)} \left[ {(e^x\tan^-1{x}+7)}^2\cot(x)+2\ln(\sin(x))((e^x\tan^-1{(x)}+7)(\frac{e^x}{x^2+1}+e^x\tan^-1({x})) \right] $

\pagebreak
	
	\subsection{Case I}
\pagebreak	

	\subsection{Case II}
\pagebreak

\section{The Ellipse}
\pagebreak	

	\subsection{Case I}
\pagebreak

	\subsection{The Circle}
\pagebreak

\section{The Hyperbola}
\pagebreak

\begin{thebibliography}{1}
	\bibitem[1]{MSE}
	\begin{flushleft}
	Kumar (https://math.stackexchange.com/users/56813/kumar), Parabola is an ellipse, but with one focal point at infinity, URL (version: 2014-05-03): \texttt{https://math.stackexchange.com/q/778285}\end{flushleft}
\end{thebibliography}
\end{document}