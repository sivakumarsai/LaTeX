\documentclass[11pt]{article}

% packages
\usepackage{physics}
% margin spacing
\usepackage[top=1in, bottom=1in, left=0.5in, right=0.5in]{geometry}
\usepackage{hanging}
\usepackage{amsfonts, amsmath, amssymb, amsthm}
\usepackage{systeme}
\usepackage[none]{hyphenat}
\usepackage{fancyhdr}
\usepackage[nottoc, notlot, notlof]{tocbibind}
\usepackage{graphicx}
\graphicspath{{./images/}}
\usepackage{hyperref}

% colors
\usepackage{xcolor}
\definecolor{p}{HTML}{FFDDDD}
\definecolor{g}{HTML}{D9FFDF}
\definecolor{y}{HTML}{FFFFCF}
\definecolor{b}{HTML}{D9FFFF}
\definecolor{o}{HTML}{FADECB}
%\definecolor{}{HTML}{}

% \highlight[<color>]{<stuff>}
\newcommand{\highlight}[2][p]{\mathchoice%
  {\colorbox{#1}{$\displaystyle#2$}}%
  {\colorbox{#1}{$\textstyle#2$}}%
  {\colorbox{#1}{$\scriptstyle#2$}}%
  {\colorbox{#1}{$\scriptscriptstyle#2$}}}%

% header/footer formatting
\pagestyle{fancy}
\fancyhead{}
\fancyfoot{}
\fancyfoot[R]{\thepage}
\renewcommand{\headrulewidth}{0pt}

% paragraph indentation/spacing
\setlength{\parindent}{0cm}
\setlength{\parskip}{5pt}
\renewcommand{\baselinestretch}{1.25}

\begin{document}
{\huge Sai Sivakumar}

Email: \href{mailto:sivakumars@ufl.edu}{\texttt{sivakumars@ufl.edu}}\hspace{5cm} DoB: July 27\textsuperscript{th}, 2002

Cell: (904)--708--0721\hspace{6.3cm} \href{https://github.com/metalninja27}{github.com/metalninja27} \\

\noindent\makebox[1.3cm][l]{\rule{1cm}{5pt}}{\textbf{Goals}}

Currently seeking to obtain a B.Sc. in mathematics, then to earn a Ph.D in mathematics with a currently unknown specialization. I am also seeking minors in computer science and in physics.

\noindent\makebox[1.3cm][l]{\rule{1cm}{5pt}}{\textbf{Education}}

\textsl{August 2020 - present:} Working on B.Sc. in Mathematics with a 3.97 GPA currently, University of Florida (UF).

\textsl{August 2016 - June 2020:} Graduated with IB diploma with a 4.00 GPA, Stanton College Preparatory High School.

\noindent\makebox[1.3cm][l]{\rule{1cm}{5pt}}{\textbf{Research}}

[none]

\noindent\makebox[1.3cm][l]{\rule{1cm}{5pt}}{\textbf{Publications, Talks/Presentations}}

\textsl{June 2021:} Discussed the integral definition of the inverse Laplace transform, as well as how to compute the integral using the residue theorem, at an elementary level. ({\color{blue}\href{https://youtu.be/20Xbrit2chw}{YouTube}})

\textsl{March 2021:} Gave a talk on proving the fundamental theorem of calculus at a highschool/pre-real analysis level. ({\color{blue}\href{https://youtu.be/l4GO-n-2ETQ}{YouTube}})

\noindent\makebox[1.3cm][l]{\rule{1cm}{5pt}}{\textbf{Experience}}

\textsl{Expected August 2021 - December 2021:} Exam proctoring and grading position for \texttt{MAP2302} Elementary Differential Equations.

\noindent\makebox[1.3cm][l]{\rule{1cm}{5pt}}{\textbf{Skills}}

2+ years of \LaTeX\hspace{1pt} experience (high proficiency).

Basic proficiency in Java, C++, expecting to have experience in Python and MATLAB. 

Presently working on some neural network basics using Python with guidance from Daniel Wilczak\\ (\href{https://github.com/danielwilczak101}{github.com/danielwilczak101}).

\noindent\makebox[1.3cm][l]{\rule{1cm}{5pt}}{\textbf{Outreach/Service}}

\textsl{August 2021 - May 2022:} To-be Academic Director of the University Math Society at UF.

\textsl{May 2021 - present:} Transcribing via \LaTeX\hspace{1pt} notes from my introductory courses in abstract algebra and complex analysis to be shared with early undergraduates.

\textsl{March 2021 - present:} Moderator for a large online community (exceeding 40,000 members globally) which seeks to stimulate mathematical discussion and interest, as well as to provide assistance with math problems/concepts.

\textsl{August 2020 - present:} Administrator for an online community for mathematics students at UF to discuss mathematics in (created in response to the Covid--19 pandemic).

\textsl{August 2020 - December 2020:} Helped write and type up several solutions for \textsl{Concepts in Calculus
III} by Miklos Bona and Sergei Shabanov (around 47 pages or so, working with two others to form in total 141 pages of solutions compiled in a solution manual).

\textsl{August 2019 - February 2020:} Started a small unofficial mathematics club (in highschool) where students presented on topics of mathematical interest; there I gave two or three informal talks.

\noindent\makebox[1.3cm][l]{\rule{1cm}{5pt}}{\textbf{Honors/Awards}}

\textsl{2020} National Merit Scholarship Commended

\textsl{2020} National AP Scholar

\noindent\makebox[1.3cm][l]{\rule{1cm}{5pt}}{\textbf{Relevant Coursework}}

\textsl{From most recent to earliest, where items marked with a * are expected to take place in the next semester, and items marked with a \textsuperscript{\textdagger} are at the graduate or mixed graduate/undergraduate level:}

\textsl{MAA4211*:} Advanced Calculus 1 --

\textsl{MAS4115*:} Linear Algebra for Data Science --

\textsl{MAT4930*\textsuperscript{\textdagger}:} Algebra 1 -- (as a special topics course)

\textsl{MAP4305:} Ordinary Differential Equations --

\textsl{MAA4402:} Introductory Complex Analysis --

\textsl{MAS4105:} Introductory Linear Algebra --

\textsl{MAS4203:} Introductory Number Theory --

\textsl{MAS4301:} Introductory Abstract Algebra --

\textsl{MAC3474:} Honors Calculus III --

\textsl{MAP2302:} Honors Elementary Differential Equations --

\textsl{MHF3202:} Sets and Logic --

\end{document}