\documentclass[11pt,a4paper,sans]{moderncv}
\moderncvstyle{casual}
\moderncvcolor{purple}
\usepackage[scale=0.75]{geometry}
\usepackage{amsfonts, amsmath, amssymb, amsthm}
\DeclareMathOperator{\GL}{GL}
\name{Sai}{Sivakumar}
\begin{document}
{\huge Sai Sivakumar}\vspace{1em}

Email: \href{mailto:sivakumars@ufl.edu}{\texttt{sivakumars@ufl.edu}}

University of Florida
\section{Goals}
\cvitem{}{Pursuing a B.Sc. in Mathematics, minors in Computer Science and Physics. Intending to complete a Ph.D. in Mathematics.}
\section{Education}
\cventry{2020--}{B.Sc. in Mathematics}{University of Florida}{}{\textit{3.97 GPA}}{}
\cventry{2020--}{Minor in Computer Science}{University of Florida}{}{\textit{4.0 GPA}}{}
\cventry{2020--}{Minor in Physics}{University of Florida}{}{\textit{4.0 GPA}}{}
\cventry{2016--2020}{IB Diploma}{Stanton College Preparatory High School}{}{\textit{4.0 GPA}}{}
\section{Research}
\cventry{Summer 2023}{REU at the University of Minnesota, Twin Cities}{}{Two projects:}{}{\begin{enumerate}
    \item[1)] Worked with Dr. Christine Berkesch and four undergraduates on a project in algebraic geometry and homological algebra. We extended results in {\color{blue}\href{https://arxiv.org/abs/2106.02759}{previous work}} by Harada et al. to provide two methods for finding {\color{blue}\href{https://arxiv.org/abs/1703.07631}{virtual resolutions}} (Berkesch et al.) for ideals of sets of points in $\mathbb{P}^n\times \mathbb P^m$, and created {\color{blue}\href{https://www-users.cse.umn.edu/~reiner/REU/REU2023notes/2_VResPointsReport.pdf}{this report}} (preprint in progress).
    \item[2)] Worked with Dr. Michael Perlman and four undergraduates on a project in commutative algebra, homological algebra, and representation theory. We studied $\GL_n(k)$-stable ideals of polynomial rings in positive characteristics and their free resolutions. We provide a method for finding minimal free resolutions for a wide class of $\GL_2(k)$-stable ideals generated in a single degree in any positive characteristic. We created {\color{blue}\href{https://www-users.cse.umn.edu/~reiner/REU/REU2023notes/4_GL_stable_Ideals_in_Positive_Characteristic.pdf}{this report}} (preprint in progress).
\end{enumerate}}
\cventry{Summer 2022}{REU at the Georgia Institute of Technology}{}{}{}{Worked with Dr. Ashley Wheeler and two other undergraduates on a project in algebraic geometry. We studied the toric varieties given by the vanishing of principal $2$-minor ideals in affine and projective space. We proved a number of properties about these varieties, and created {\color{blue}\href{https://math.gatech.edu/sites/default/files/images/boniface-rodriguez-sivakumar-wheeler.pdf}{this poster}}.}
\section{Talks and Presentations}
\cventry{Oct. 2023}{\textnormal{Virtual resolutions of points in $\mathbb{P}^n\times \mathbb P^m$}}{}{}{}{To the UF Algebra seminar; outlining the first project I worked on at the UMN Twin Cities mathematics REU.}
\cventry{Sep. 2023}{\textnormal{Stable ideals and their syzygies}}{}{}{}{Talk for undergraduates about the second project I worked on at the UMN Twin Cities mathematics REU.}
\cventry{Apr., Jul. 2023}{\textnormal{Fourier analysis on LCA groups and Pontryagin duality.}}{}{}{}{For Dr. Sin's MAS7397 Introduction to representation theory class and for the Student Summer Representation Theory Seminar at the University of Minnesota, Twin Cities ({\color{blue}\href{https://sites.google.com/umn.edu/robertangarone/ssrts-23}{abstract and recording}}).}
\cventry{Oct. 2022}{\textnormal{What is representation theory?}}{}{}{}{Intro to the topic for undergraduates, applications in Fourier analysis.}
\cventry{Feb.--Mar. 2022}{\textnormal{Fourier analysis on finite Abelian groups.}}{}{}{}{Five lectures briefly outlining the theory and applications as they appear in Stein and Shakarchi I.}
\cventry{Feb. 2022}{\textnormal{Combinatorial proof of existence of Sylow subgroups.}}{}{}{}{To the UF Algebra seminar; proof due to Wielandt in 1959.}
\cventry{Oct. 2021}{\textnormal{The Hamilton quaternions}}{}{}{}{With classmate; discussed history, properties, functions of quaternionic variables, applications.}
\cventry{Falls 2021--2023}{\textnormal{Annual \LaTeX\hspace{1pt} seminar.}}{}{}{}{Joint with the UF Graduate Mathematics Association (GMA) and the UF Association for Women in Mathematics (AWM) chapter.}
\cventry{Jun. 2021}{\textnormal{The inverse Laplace transform.}}{}{}{}{Integral definition of the inverse Laplace transform, computations using the integral using the residue theorem. ({\color{blue}\href{https://youtu.be/20Xbrit2chw}{YouTube}}; gave an abridged version of this talk Jan. 2023.)}
\cventry{Mar. 2021}{\textnormal{The fundamental theorem of calculus.}}{}{}{}{Proved the theorem at a high school/pre-real analysis level.}
\section{Skills}
\cvitem{\LaTeX\hspace{1pt}}{4\texttt{+} years}
\cvitem{Macaulay2}{ used in both REUs I attended}
\cvitem{Java, C\texttt{++}}{with understanding of data structures and algorithms}
\section{Outreach and Service}
\cvitem{Aug. 2023--}{President of the University Math Society at UF. I manage the club (delegating responsibilities, reserving venues, responding to queries, etc.), ensure all events run smoothly, form collaborations with the AWM chapter and the GMA, and encourage a friendly, inclusive environment for all members.}
\cvitem{Aug. 2021--}{Teaching assistant and grader for \textsl{MAP2302} Elementary Differential Equations.}
\cvitem{Mar. 2021--}{Moderator for a large online community exceeding 160,000 members globally, which seeks to stimulate mathematical discussion and interest, as well as to provide assistance with math.}
\cvitem{Aug. 2021--May 2023}{Academic Director of the University Math Society at UF. I scheduled talks from professors, gave talks myself, and encouraged undergraduate students to give talks.}
\cvitem{Aug.--Dec. 2020}{Contributed around 47 pages to the solution manual for \textsl{Concepts in Calculus
III} by Miklos Bona and Sergei Shabanov, working with two other students to form 141 pages of solutions which are posted on the course page.}
\section{Honors and Awards}
\cventry{2023}{\textnormal{The Kermit Sigmon Scholarship.}}{}{}{}{Awarded in the spring each year for a promising undergraduate mathematics major ({\color{blue}\href{https://math.ufl.edu/mathematics-major/opportunities-for-undergraduates/the-kermit-sigmon-scholarship/}{info}}).} 
\cventry{2020--2023}{\textnormal{Dean's list.}}{}{}{}{Fall 2020, Spring 2021, Summer 2021, Fall 2021, Spring 2022, Fall 2022, Spring 2023.} 
\section{Mathematics Coursework} \textsl{Items marked by a \textsuperscript{\textdagger} are graduate or mixed graduate/undergraduate level courses:}\vspace{1em}

\cventry{No code}{\textnormal{Modular Forms}}{Fall 2023}{(reading course)}{}{With Dr. Jeremy Booher, reading from Ch. 7 of A Course In Arithmetic by Serre and from A First Course in Modular Forms by Diamond and Shurman. In preparation for senior thesis.}
\cventry{MAT6932\textsuperscript{\textdagger}}{\textnormal{Analytic Number Theory}}{Fall 2023}{(audited)}{}{Topics in analytic number theory including the prime number theorem, sieve methods, probablistic number theory. Professor's notes.}
\cventry{MTG6256\textsuperscript{\textdagger}}{\textnormal{ Differential Geometry 1}}{Fall 2023}{(audited)}{}{Basic concepts of differential and Riemannian geometry. Using do Carmo.}
\cventry{MAP6505\textsuperscript{\textdagger}}{\textnormal{Mathematical Methods for Physics and Engineering I}}{Fall 2023}{}{}{Topics in real analysis, complex analysis, geometry which are used in physics and engineering. Theory of distributions, distributional solutions to linear differential equations, and Green’s functions. Professor's notes.}
\cventry{MAS7397\textsuperscript{\textdagger}}{\textnormal{Introduction to Representation Theory}}{Spring 2023}{}{}{Module-theoretic representation and character theory of finite groups and of finite-dimensional semisimple complex Lie algebras, theorems of Burnside and Frobenius. Professor's lectures following content from Fulton-Harris.}
\cventry{MTG6347\textsuperscript{\textdagger}}{\textnormal{Topology II}}{Spring 2023}{}{}{Singular, axiomatic, and cellular homology/cohomology, and their algebraic structure and dualities. Professor's notes and chapters 10-12, parts of chapters 16-18 in tom Dieck.}
\cventry{MAA6407\textsuperscript{\textdagger}}{\textnormal{Complex Analysis II}}{Spring 2023}{}{}{Weierstrass factorization, analytic continuation and Riemann zeta function, harmonic functions, Picard theorems, and other topics. Professor's notes and chapters 7, 9, 10, parts of 12 in Conway.}
\cventry{MAA6617\textsuperscript{\textdagger}}{\textnormal{Analysis II}}{Spring 2023}{}{}{Introductory functional analysis. Theory of Banach and Hilbert spaces, linear operators, $L^p$ spaces and their duality, $L^1$ and $L^2$ Fourier transform, theory of distributions. Touched on Banach algebras. Professor's notes.}
\cventry{MAS6332\textsuperscript{\textdagger}}{\textnormal{Algebra II}}{Spring 2023}{}{}{Projective, injective, and flat modules. Introduction to homological algebra, group cohomology, category theory, commutative algebra, and algebraic geometry. Chapters 15-17 in Dummit and Foote and professor's notes.}
\cventry{MAT6932\textsuperscript{\textdagger}}{\textnormal{Calculus of Variations and Optimal Control}}{Fall 2022}{}{}{Covered basic theory of calculus of variations and optimal control following several examples. Professor's lectures.}
\cventry{MTG6346\textsuperscript{\textdagger}}{\textnormal{Topology I}}{Fall 2022}{}{}{Topology I -- Covered the fundamental group, covering spaces, intro homotopy theory, cofibrations and fibrations, homotopy groups, CW complexes, singular homology. Chapters 1-6, 8, 9 of tom Dieck.}
\cventry{MAA6406\textsuperscript{\textdagger}}{\textnormal{Complex Analysis I}}{Fall 2022}{}{}{Analytic functions, integral formulas, zeroes and singularities of functions, Morera's and Goursat's theorems, Cauchy's theorem and integral formula, Laurent series, spaces of holomorphic functions. Professor's notes and chapters 1-7 in Conway.}
\cventry{MAA6616\textsuperscript{\textdagger}}{\textnormal{Analysis I}}{Fall 2022}{}{}{Sigma-algebras, measures, the Lebesgue measure, signed measures, integration of measurable functions, modes of convergence, and differentiation theorems. Professor's notes.}
\cventry{MAS6331\textsuperscript{\textdagger}}{\textnormal{Algebra I}}{Fall 2022}{}{}{Field and Galois theory, as well as tensor products and some coverage of projective modules. Chapters 10.4, 10.5, 13, 14 of Dummit and Foote.}
\cventry{MTG4303\textsuperscript{\textdagger}}{\textnormal{Introductory Topology II}}{Spring 2022}{}{}{Basic algebraic topology and topics from point-set topology. Chapters 5-6, 9-12 from Munkres.}
\cventry{MAA4212}{\textnormal{Advanced Calculus II}}{Spring 2022}{}{}{Analysis in metric spaces and theory of functions of several real variables. Professor's notes.}
\cventry{MAP4341\textsuperscript{\textdagger}}{\textnormal{Introduction to Partial Differential Equations}}{Spring 2022}{}{}{Elementary theory of solving partial differential equations. Professor's notes and lectures.}
\cventry{MAS5312\textsuperscript{\textdagger}}{\textnormal{Introduction to Algebra II}}{Spring 2022}{}{}{Rings, fields, modules. Chapters 7-13 of Dummit and Foote.}
\cventry{MAA5228\textsuperscript{\textdagger}}{\textnormal{Modern Analysis I}}{Fall 2021}{(audited)}{}{Metric spaces and topology, convergence of sequences and series, continuity, differentiation. Chapters 1-5 of Rudin PMA.}
\cventry{MAA4211}{\textnormal{Advanced Calculus I}}{Fall 2021}{}{}{Introductory real analysis. Chapters 1-7 of Abbott.}
\cventry{MAS4413}{\textnormal{Fourier Analysis}}{Fall 2021}{}{}{Elementary theory of Fourier analysis. Chapters 1-7 of Stein and Shakarchi I.}
\cventry{MAS5311\textsuperscript{\textdagger}}{\textnormal{Introduction to Algebra I}}{Fall 2021}{}{}{Group theory. Chapters 1-6 from Dummit and Foote.}
\cventry{MAP4305}{\textnormal{Ordinary Differential Equations}}{Summer 2021}{}{}{Second course in ordinary differential equations. Covered methods of using matrices for systems of linear ODEs, the method of Frobenius for second order ODEs, solving regular Sturm-Liouville boundary value problems, and using Green's functions. Professor's lectures.}
\cventry{MAA4402}{\textnormal{Introductory Complex Analysis}}{Spring 2021}{}{}{Elementary theory of functions of a complex variable. Chapters 1-7 of Brown and Churchill.}
\cventry{MAS4105}{\textnormal{Introductory Linear Algebra}}{Spring 2021}{}{}{Proof-based linear algebra. Chapters 1-6 of Friedberg, Insel, Spence.}
\cventry{MAS4203}{\textnormal{Introductory Number Theory}}{Spring 2021}{}{}{Elementary concepts in number theory. Chapters 1-3 of Niven and Zuckerberg.}
\cventry{MAS4301}{\textnormal{Introductory Abstract Algebra}}{Spring 2021}{}{}{Elementary group theory. Chapters 1-11 of Gallian.}
\cventry{MAC3474}{\textnormal{Honors Calculus III}}{Fall 2020}{}{}{Basic multivariable calculus. Chapters 1-5 of \textsl{Concepts in Calculus III} by Miklos Bona and Sergei Shabanov.}
\cventry{MAP2302}{\textnormal{Honors Elementary Differential Equations}}{Fall 2020}{}{}{Introduction to solving ordinary differential equations, existence and uniqueness, and applications to physics. Chapters 1-8 in Nagle Saff Snider 7th edition.}
\cventry{MHF3202}{\textnormal{Sets and Logic}}{Fall 2020}{}{}{Elementary set theory and how to write basic proofs. Chapters 1,2,3, 5-10, 12, 14 in Hammack.}
\end{document}