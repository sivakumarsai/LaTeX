\documentclass[11pt]{article}
\renewcommand{\familydefault}{\sfdefault}

% packages
\usepackage{physics}
% margin spacing
\usepackage[top=1in, bottom=1in, left=1in, right=1in]{geometry}
\usepackage{hanging}
\usepackage{amsfonts, amsmath, amssymb, amsthm}
\usepackage{systeme}
\usepackage[none]{hyphenat}
\usepackage{fancyhdr}
\usepackage[nottoc, notlot, notlof]{tocbibind}
\usepackage{graphicx}
\graphicspath{{./images/}}
\usepackage{hyperref}

% colors
\usepackage{xcolor}
\definecolor{p}{HTML}{FFDDDD}
\definecolor{g}{HTML}{D9FFDF}
\definecolor{y}{HTML}{FFFFCF}
\definecolor{b}{HTML}{D9FFFF}
\definecolor{o}{HTML}{FADECB}
%\definecolor{}{HTML}{}

% \highlight[<color>]{<stuff>}
\newcommand{\highlight}[2][p]{\mathchoice%
  {\colorbox{#1}{$\displaystyle#2$}}%
  {\colorbox{#1}{$\textstyle#2$}}%
  {\colorbox{#1}{$\scriptstyle#2$}}%
  {\colorbox{#1}{$\scriptscriptstyle#2$}}}%

% header/footer formatting
\pagestyle{fancy}
\fancyhead{}
\fancyfoot{}
\fancyfoot[R]{\thepage}
\renewcommand{\headrulewidth}{0pt}

% paragraph indentation/spacing
\setlength{\parindent}{0cm}
\setlength{\parskip}{5pt}
\leftskip 1cm
\parindent -1cm
\renewcommand{\baselinestretch}{1.25}

\DeclareMathOperator{\GL}{GL}

\begin{document}
{\huge SaSivakumar}

Email: \href{mailto:sivakumars@ufl.edu}{\texttt{sivakumars@ufl.edu}}\hspace{8cm}Cell: (904)--708--0721\\%\hspace{5cm} DoB: July 27\textsuperscript{th}, 2002\\
\hspace*{-1cm}University of Florida\\
%\hspace*{-1cm}U.S. Citizen
%\hspace*{-1cm}Cell: (904)--708--0721%\hspace{6.38cm} \href{https://github.com/metalninja27}{\texttt{github.com/metalninja27}} \\

\makebox[1.3cm][l]{\rule{1cm}{5pt}}{\textbf{Goals}}

Pursuing a B.Sc. in mathematics with minors in computer science and physics. Intending to complete a Ph.D. in mathematics.

\makebox[1.3cm][l]{\rule{1cm}{5pt}}{\textbf{Education}}

\textsl{July 2020 - present:} B.Sc. in Mathematics with minors in computer science and physics, at the University of Florida, Gainesville, Florida. Currently with a 3.97 GPA, to graduate May 2024.

\textsl{August 2016 - June 2020:} Graduated with IB diploma with a 4.0 GPA, Stanton College Preparatory High School, Jacksonville, Florida.

\makebox[1.3cm][l]{\rule{1cm}{5pt}}{\textbf{Research}}

\textsl{Summer 2023:} REU at the University of Minnesota, Twin Cities. Two projects: \\
1) Worked with Dr. Christine Berkesch and four undergraduates on a project in algebraic geometry and homological algebra. We extended results in {\color{blue}\href{https://arxiv.org/abs/2106.02759}{previous work}} by Harada et al. to provide two methods for finding {\color{blue}\href{https://arxiv.org/abs/1703.07631}{virtual resolutions}} (Berkesch et al.) for ideals of sets of points in $\mathbb{P}^n\times \mathbb P^m$, and created {\color{blue}\href{https://www-users.cse.umn.edu/~reiner/REU/REU2023notes/2_VResPointsReport.pdf}{this report}} (preprint in progress).\\
2) Worked with Dr. Michael Perlman and four undergraduates on a project in commutative algebra, homological algebra, and representation theory. We studied $\GL_n(k)$-stable ideals of polynomial rings in positive characteristics and their free resolutions. We provide a method for finding minimal free resolutions for a wide class of $\GL_2(k)$-stable ideals generated in a single degree in any positive characteristic. %A Macaulay2 package containing functionality surrounding $\GL_n(k)$-stable ideals of polynomial rings is in preparation, and w
We created {\color{blue}\href{https://www-users.cse.umn.edu/~reiner/REU/REU2023notes/4_GL_stable_Ideals_in_Positive_Characteristic.pdf}{this report}} (preprint in progress).

\textsl{Summer 2022:} REU at the Georgia Institute of Technology. Worked with Dr. Ashley Wheeler and two other undergraduates on a project in algebraic geometry. We studied the toric varieties given by the vanishing of principal $2$-minor ideals in affine and projective space. We proved a number of properties about these varieties, and created {\color{blue}\href{https://math.gatech.edu/sites/default/files/images/boniface-rodriguez-sivakumar-wheeler.pdf}{this poster}}.

\makebox[1.3cm][l]{\rule{1cm}{5pt}}{\textbf{Talks and Presentations}}

\textsl{July 2023:} Gave talk on Fourier analysis on locally compact Abelian groups and Pontryagin duality to the Summer Representation Theory Seminar at the University of Minnesota, Twin Cities. ({\color{blue}\href{https://sites.google.com/umn.edu/robertangarone/ssrts-23}{abstract and recording}})

\textsl{October 2022:} Small talk for math undergraduates motivating representations of groups as compared with group actions and briefly mentioned their applications in Fourier analysis. 

\textsl{February - March 2022:} Five lectures/talks briefly outlining Fourier analysis on finitely generated Abelian groups with some neat results, as they appear in Stein and Shakarchi I. 

\textsl{February 2022:} Gave a talk to the UF Algebra seminar on a combinatorial proof of the existence of Sylow subgroups due to Wielandt in 1959.

\textsl{October 2021:} A classmate and I discussed the real Hamilton quaternions, their history as well as their algebraic properties, and mentioned functions of quaternionic variables. Also spoke about applications in computer science and physics.

\textsl{Falls 2021-2023:} Gave the annual \LaTeX\hspace{1pt} joint seminar with the UF Graduate Mathematics Association (GMA) and the Association for Women in Mathematics (AWM) chapter. This seminar is designed to demonstrate how \LaTeX\hspace{1pt} works and what it can do, and to encourage mathematics students to learn \LaTeX\hspace{1pt}.

\textsl{June 2021:} Discussed the integral definition of the inverse Laplace transform, as well as how to compute the integral using the residue theorem, at an elementary level. ({\color{blue}\href{https://youtu.be/20Xbrit2chw}{YouTube}}) (Gave an abridged version of this talk January 2023.)

\textsl{March 2021:} Gave a talk on proving the fundamental theorem of calculus at a highschool/pre-real analysis level. ({\color{blue}\href{https://youtu.be/l4GO-n-2ETQ}{YouTube}})

%\noindent\makebox[1.3cm][l]{\rule{1cm}{5pt}}{\textbf{Experience}}

\makebox[1.3cm][l]{\rule{1cm}{5pt}}{\textbf{Skills}}

\LaTeX\hspace{1pt} (4+ years).

Macaulay2 (used in both REUs above)

Java, C\verb!++!, and understanding of data structures and algorithms. 

\makebox[1.3cm][l]{\rule{1cm}{5pt}}{\textbf{Outreach/Service}}

\textsl{August 2023 - present:} President of the University Math Society at UF. I manage the club (delegating responsibilities, reserving venues, responding to queries, etc.), ensure all events run smoothly, form collaborations with the AWM chapter and the GMA, and encourage a friendly, inclusive environment for all members.

\textsl{August 2021 - present:} Teaching assistant for \textsl{MAP2302} Elementary Differential Equations.

\textsl{March 2021 - present:} Moderator for a large online community (exceeding 160,000 members globally) which seeks to stimulate mathematical discussion and interest, as well as to provide assistance with math problems/concepts.

\textsl{August 2021 - May 2023:} Academic Director of the University Math Society at UF. I scheduled talks from professors, gave talks myself, and encouraged undergraduate students to give talks.

\textsl{August 2020 - December 2020:} Typed up many solutions for \textsl{Concepts in Calculus
III} by Miklos Bona and Sergei Shabanov: around 47 pages or so, working with two other students to form 141 pages of solutions which are posted on the course page.

\textsl{August 2019 - February 2020:} (high school) Started a small unofficial mathematics club  where students presented on topics of mathematical interest; there I gave three informal talks.

\makebox[1.3cm][l]{\rule{1cm}{5pt}}{\textbf{Honors/Awards}}

The Kermit Sigmon Scholarship \textsl{2023} ({\color{blue}\href{https://math.ufl.edu/mathematics-major/opportunities-for-undergraduates/the-kermit-sigmon-scholarship/}{info}}).

Dean's list, \textsl{Fall 2020, Spring 2021, Summer 2021, Fall 2021, Spring 2022, Fall 2022, Spring 2023}.

National Merit Scholarship Commended \textsl{2020}.

National AP Scholar \textsl{2020}.

\makebox[1.3cm][l]{\rule{1cm}{5pt}}{\textbf{Math Coursework}}

\textsl{Items marked by a \textsuperscript{\textdagger} are graduate or mixed graduate/undergraduate level courses:}

\textsl{MAS7397(MAT4930)\textsuperscript{\textdagger}:} Topics in Algebra II -- Introduction to representation theory of groups and Lie algebras, from a module-theoretic view. Representation and character theory of finite groups and of finite-dimensional semisimple complex Lie algebras, theorems of Burnside and Frobenius. Professor's lectures following content from Fulton-Harris. Spring 2023

\textsl{MTG6347(MAT4930)\textsuperscript{\textdagger}:} Topology II -- Singular, axiomatic, and cellular homology/cohomology, and their algebraic structure and dualities. Professor's notes and chapters 10-12, parts of chapters 16-18 in tom Dieck. Spring 2023

\textsl{MAA6407(MAT4930)\textsuperscript{\textdagger}:} Complex Analysis II -- Continuation of previous semester of complex analysis. Covered Weierstrass factorization, analytic continuation and Riemann zeta function, harmonic functions, Picard theorems, and other topics. Professor's notes and chapters 7, 9, 10, parts of 12 in Conway. Spring 2023

\textsl{MAA6617(MAT4930)\textsuperscript{\textdagger}:} Analysis II -- Introductory functional analysis. Theory of Banach and Hilbert spaces, linear operators, $L^p$ spaces and their duality, $L^1$ and $L^2$ Fourier transform, theory of distributions. Touched on Banach algebras. Professor's notes. Spring 2023

\textsl{MAS6332(MAT4930)\textsuperscript{\textdagger}:} Algebra II -- Projective, injective, and flat modules. Introduction to homological algebra, group cohomology, category theory, commutative algebra, and algebraic geometry. Chapters 15-17 in Dummit and Foote and professor's notes. Spring 2023

\textsl{MAT6932(MAT4930)\textsuperscript{\textdagger}:} Calculus of Variations and Optimal Control -- Covered basic theory of calculus of variations and optimal control following several examples. Professor's lectures. Fall 2022

\textsl{MTG6346(MAT4930)\textsuperscript{\textdagger}:} Topology I -- Covered the fundamental group, covering spaces, intro homotopy theory, cofibrations and fibrations, homotopy groups, CW complexes, singular homology. Chapters 1-6, 8, 9 of tom Dieck. Fall 2022

\textsl{MAA6406(MAT4930)\textsuperscript{\textdagger}:} Complex Analysis I -- Standard coverage of analytic functions, integral formulas, zeroes and singularities of functions, Morera's and Goursat's theorems, Cauchy's theorem and integral formula, Laurent series, spaces of holomorphic functions. Professor's notes and chapters 1-7 in Conway. Fall 2022

\textsl{MAA6616(MAT4930)\textsuperscript{\textdagger}:} Analysis I -- Sigma-algebras, measures, the Lebesgue measure, signed measures, integration of measurable functions, modes of convergence, and differentiation theorems. Professor's notes. Fall 2022

\textsl{MAS6331(MAT4930)\textsuperscript{\textdagger}:} Algebra I -- Field and Galois theory, as well as tensor products and some coverage of projective modules. Chapters 10.4, 10.5, 13, 14 of Dummit and Foote. Fall 2022

\textsl{MTG4303\textsuperscript{\textdagger}:} Introductory Topology II -- Second semester of topology, covering basic algebraic topology and topics from point-set topology. Chapters 5-6, 9-12 from Munkres. Spring 2022

\textsl{MAA4212:} Advanced Calculus II -- Second semester of introductory real analysis, covering analysis in metric spaces and theory of functions of several real variables. Professor's notes. Spring 2022

\textsl{MAP4341\textsuperscript{\textdagger}:} Introduction to Partial Differential Equations -- Elementary theory of solving partial differential equations. Professor's notes and lectures. Spring 2022

\textsl{MAS5312(MAT4930)\textsuperscript{\textdagger}:} Introduction to Algebra II -- Second semester graduate level algebra; covering rings, fields, modules. Chapters 7-13 of Dummit and Foote. Spring 2022

\textsl{MAA4211:} Advanced Calculus I -- First semester of introductory real analysis. Chapters 1-7 of Abbott. Fall 2021

\textsl{MAS4413:} Fourier Analysis -- Elementary theory of Fourier analysis. Chapters 1-7 of Stein and Shakarchi I. Fall 2021

\textsl{MAS5311(MAT4930)\textsuperscript{\textdagger}:} Introduction to Algebra I -- First semester graduate level algebra; covering group theory. Chapters 1-6 from Dummit and Foote. Fall 2021

\textsl{MAP4305:} Ordinary Differential Equations -- Second course in ordinary differential equations. Covered methods of using matrices for systems of linear ODEs, the method of Frobenius for second order ODEs, solving regular Sturm-Liouville boundary value problems, and using Green's functions. Professor's lectures. Summer 2021

\textsl{MAA4402:} Introductory Complex Analysis -- Elementary theory of functions of a complex variable. Chapters 1-7 of Brown and Churchill. Spring 2021

\textsl{MAS4105:} Introductory Linear Algebra -- Proof-based linear algebra. Chapters 1-6 of Friedberg, Insel, Spence. Spring 2021

\textsl{MAS4203:} Introductory Number Theory -- Elementary concepts in number theory. Chapters 1-3 of Niven and Zuckerberg. Spring 2021

\textsl{MAS4301:} Introductory Abstract Algebra -- Elementary group theory. Chapters 1-11 of Gallian. Spring 2021

\textsl{MAC3474:} Honors Calculus III -- Basic multivariable calculus. Chapters 1-5 of \textsl{Concepts in Calculus III} by Miklos Bona and Sergei Shabanov. Fall 2020

\textsl{MAP2302:} Honors Elementary Differential Equations -- Covered how to solve various basic ODEs, basic notions of existence and uniqueness, and applications to physics. Chapters 1-8 in Nagle Saff Snider 7th edition. Fall 2020

\textsl{MHF3202:} Sets and Logic -- Elementary set theory and how to write basic proofs. Chapters 1,2,3, 5-10, 12, 14 in Hammack. Fall 2020
\end{document}